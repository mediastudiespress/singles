\documentclass[nohyper,openany,nobib]{tufte-book}

% \hypersetup{colorlinks=true,allcolors=[RGB]{97,15,11}}

% For print version
    % 1. change document class to add \documentclass[nohyper]
    % 2. comment out \hypersetup above
    % 3. get rid of the doi marginnotes
    % 4. export PDF, then use Print Production -> preflight -> tool icon -> scale 229 x 177


%%
% Book metadata
\title{Our Master's Voice}
\date{Advertising}
\author[James Rorty]{James Rorty}
\publisher{a mediastudies.press public domain edition}


%%
% If they're installed, use Bergamo and Chantilly from www.fontsite.com.
% They're clones of Bembo and Gill Sans, respectively.
%\IfFileExists{bergamo.sty}{\usepackage[osf]{bergamo}}{}% Bembo
%\IfFileExists{chantill.sty}{\usepackage{chantill}}{}% Gill Sans

%\usepackage{microtype}


% restart footnotes each chapter
\let\oldchapter\chapter
\def\chapter{%
  \setcounter{footnote}{0}%
  \oldchapter
}

%my addtion for toc
\newcounter{oldtocdepth}

\newcommand{\hidefromtoc}{%
  \setcounter{oldtocdepth}{\value{tocdepth}}%
  \addtocontents{toc}{\protect\setcounter{tocdepth}{-10}}%
}

\newcommand{\unhidefromtoc}{%
  \addtocontents{toc}{\protect\setcounter{tocdepth}{\value{oldtocdepth}}}%
}
\usepackage{hyperref}
\usepackage{bookmark}

%%
% Just some sample text
\usepackage{lipsum}

%%
% For nicely typeset tabular material
\usepackage{booktabs}

%%
% Another table package
\usepackage{tabu}

\usepackage{longtable}

%%
% For table spacing
\usepackage{verbatimbox}

%%
% For graphics / images
\usepackage{graphicx}
\setkeys{Gin}{width=\linewidth,totalheight=\textheight,keepaspectratio}
\graphicspath{{graphics/}}

% The fancyvrb package lets us customize the formatting of verbatim
% environments.  We use a slightly smaller font.
\usepackage{fancyvrb}
\fvset{fontsize=\normalsize}

%%
% Prints argument within hanging parentheses (i.e., parentheses that take
% up no horizontal space).  Useful in tabular environments.
\newcommand{\hangp}[1]{\makebox[0pt][r]{(}#1\makebox[0pt][l]{)}}

%%
% Prints an asterisk that takes up no horizontal space.
% Useful in tabular environments.
\newcommand{\hangstar}{\makebox[0pt][l]{*}}

%%
% Prints a trailing space in a smart way.
\usepackage{xspace}

%%
% Some shortcuts for Tufte's book titles.  The lowercase commands will
% produce the initials of the book title in italics.  The all-caps commands
% will print out the full title of the book in italics.
\newcommand{\vdqi}{\textit{VDQI}\xspace}
\newcommand{\ei}{\textit{EI}\xspace}
\newcommand{\ve}{\textit{VE}\xspace}
\newcommand{\be}{\textit{BE}\xspace}
\newcommand{\VDQI}{\textit{The Visual Display of Quantitative Information}\xspace}
\newcommand{\EI}{\textit{Envisioning Information}\xspace}
\newcommand{\VE}{\textit{Visual Explanations}\xspace}
\newcommand{\BE}{\textit{Beautiful Evidence}\xspace}

\newcommand*{\justlastragged}{%
\leftskip=0pt plus 1fil
\rightskip=-\leftskip
\parfillskip=\leftskip
\parindent=0pt}

\newcommand{\TL}{Tufte-\LaTeX\xspace}

% Prints the month name (e.g., January) and the year (e.g., 2008)
\newcommand{\monthyear}{%
  \ifcase\month\or January\or February\or March\or April\or May\or June\or
  July\or August\or September\or October\or November\or
  December\fi\space\number\year
}


% Prints an epigraph and speaker in sans serif, all-caps type.
\newcommand{\openepigraph}[2]{%
  %\sffamily\fontsize{14}{16}\selectfont
  \sffamily\large
  \begin{doublespace}
  \noindent\allcaps{#1}\\% epigraph
  \noindent\allcaps{#2}% author
  \end{doublespace}
}

\usepackage{enumitem}
\setlist[enumerate]{itemsep=0mm}

% Inserts a blank page
\newcommand{\blankpage}{\newpage\hbox{}\thispagestyle{empty}\newpage}

\usepackage{units}

% Typesets the font size, leading, and measure in the form of 10/12x26 pc.
\newcommand{\measure}[3]{#1/#2$\times$\unit[#3]{pc}}

% Macros for typesetting the documentation
\newcommand{\hlred}[1]{\textcolor{Maroon}{#1}}% prints in red
\newcommand{\hangleft}[1]{\makebox[0pt][r]{#1}}
\newcommand{\hairsp}{\hspace{1pt}}% hair space
\newcommand{\hquad}{\hskip0.5em\relax}% half quad space
\newcommand{\TODO}{\textcolor{red}{\bf TODO!}\xspace}
\newcommand{\ie}{\textit{i.\hairsp{}e.}\xspace}
\newcommand{\eg}{\textit{e.\hairsp{}g.}\xspace}
\newcommand{\na}{\quad--}% used in tables for N/A cells
\providecommand{\XeLaTeX}{X\lower.5ex\hbox{\kern-0.15em\reflectbox{E}}\kern-0.1em\LaTeX}
\newcommand{\tXeLaTeX}{\XeLaTeX\index{XeLaTeX@\protect\XeLaTeX}}
% \index{\texttt{\textbackslash xyz}@\hangleft{\texttt{\textbackslash}}\texttt{xyz}}
\newcommand{\tuftebs}{\symbol{'134}}% a backslash in tt type in OT1/T1
\newcommand{\doccmdnoindex}[2][]{\texttt{\tuftebs#2}}% command name -- adds backslash automatically (and doesn't add cmd to the index)
\newcommand{\doccmddef}[2][]{%
  \hlred{\texttt{\tuftebs#2}}\label{cmd:#2}%
  \ifthenelse{\isempty{#1}}%
    {% add the command to the index
      \index{#2 command@\protect\hangleft{\texttt{\tuftebs}}\texttt{#2}}% command name
    }%
    {% add the command and package to the index
      \index{#2 command@\protect\hangleft{\texttt{\tuftebs}}\texttt{#2} (\texttt{#1} package)}% command name
      \index{#1 package@\texttt{#1} package}\index{packages!#1@\texttt{#1}}% package name
    }%
}% command name -- adds backslash automatically
\newcommand{\doccmd}[2][]{%
  \texttt{\tuftebs#2}%
  \ifthenelse{\isempty{#1}}%
    {% add the command to the index
      \index{#2 command@\protect\hangleft{\texttt{\tuftebs}}\texttt{#2}}% command name
    }%
    {% add the command and package to the index
      \index{#2 command@\protect\hangleft{\texttt{\tuftebs}}\texttt{#2} (\texttt{#1} package)}% command name
      \index{#1 package@\texttt{#1} package}\index{packages!#1@\texttt{#1}}% package name
    }%
}% command name -- adds backslash automatically
\newcommand{\docopt}[1]{\ensuremath{\langle}\textrm{\textit{#1}}\ensuremath{\rangle}}% optional command argument
\newcommand{\docarg}[1]{\textrm{\textit{#1}}}% (required) command argument
\newenvironment{docspec}{\begin{quotation}\ttfamily\parskip0pt\parindent0pt\ignorespaces}{\end{quotation}}% command specification environment
\newcommand{\docenv}[1]{\texttt{#1}\index{#1 environment@\texttt{#1} environment}\index{environments!#1@\texttt{#1}}}% environment name
\newcommand{\docenvdef}[1]{\hlred{\texttt{#1}}\label{env:#1}\index{#1 environment@\texttt{#1} environment}\index{environments!#1@\texttt{#1}}}% environment name
\newcommand{\docpkg}[1]{\texttt{#1}\index{#1 package@\texttt{#1} package}\index{packages!#1@\texttt{#1}}}% package name
\newcommand{\doccls}[1]{\texttt{#1}}% document class name
\newcommand{\docclsopt}[1]{\texttt{#1}\index{#1 class option@\texttt{#1} class option}\index{class options!#1@\texttt{#1}}}% document class option name
\newcommand{\docclsoptdef}[1]{\hlred{\texttt{#1}}\label{clsopt:#1}\index{#1 class option@\texttt{#1} class option}\index{class options!#1@\texttt{#1}}}% document class option name defined
\newcommand{\docmsg}[2]{\bigskip\begin{fullwidth}\noindent\ttfamily#1\end{fullwidth}\medskip\par\noindent#2}
\newcommand{\docfilehook}[2]{\texttt{#1}\index{file hooks!#2}\index{#1@\texttt{#1}}}
\newcommand{\doccounter}[1]{\texttt{#1}\index{#1 counter@\texttt{#1} counter}}

% Generates the index
\usepackage{makeidx}
\makeindex

\titleformat{\section}%
  {\normalfont\LARGE\scshape}% format applied to label+text
  {\llap{\colorbox{orange}{\parbox{1.5cm}{\hfill\color{white}\thesection}}}}% label
  {1em}% horizontal separation between label and title body
  {}% before the title body
  []% after the title body

\titleformat{\subsection}%
  {\normalfont\large\scshape}% format applied to label+text
  {\llap{\colorbox{orange}{\parbox{1.5cm}{\hfill\color{white}\thesection}}}}% label
  {1em}% horizontal separation between label and title body
  {}% before the title body
  []% after the title body

%%%% Kevin Goody's code for title page and contents from https://groups.google.com/forum/#!topic/tufte-latex/ujdzrktC1BQ
\makeatletter
\renewcommand{\maketitlepage}{%
\begingroup%
\setlength{\parindent}{10pt}

{\fontsize{24}{24}\selectfont\textit{\@author}\par}

\vspace{1.75in}{\fontsize{27}{54}{\allcaps{\@title}}\par}

\vspace{0.5in}{\fontsize{22}{24}{\allcaps{\textit{\@date}}}\par}

\vfill{\fontsize{14}{14}\selectfont\smallcaps{\@publisher}\par}

\thispagestyle{empty}
\endgroup
}
\makeatother

\titlecontents{part}%
    [0pt]% distance from left margin
    {\addvspace{0.25\baselineskip}}% above (global formatting of entry)
    {\allcaps{Part~\thecontentslabel}\allcaps}% before w/ label (label = ``Part I'')
    {\allcaps{Part~\thecontentslabel}\allcaps}% before w/o label
    {}% filler and page (leaders and page num)
    [\vspace*{0.5\baselineskip}]% after

\titlecontents{chapter}%
    [4em]% distance from left margin
    {}% above (global formatting of entry)
    {\contentslabel{2em}\textit}% before w/ label (label = ``Chapter 1'')
    {\hspace{0em}\textit}% before w/o label
    {\qquad\thecontentspage}% filler and page (leaders and page num)
    [\vspace*{0.5\baselineskip}]% after

%%%% End additional code by Kevin Godby

\begin{document}

% Front matter
\frontmatter\pagenumbering{roman}\setcounter{page}{3}

% ONE full title page
\begin{fullwidth}


\maketitle


% TWO copyright page
\newpage

~\vfill
\thispagestyle{empty}
\setlength{\parindent}{0pt}
\setlength{\parskip}{\baselineskip}
\emph{Our Master's Voice: Advertising}, originally published in 1934 by the \smallcaps{John Day Company}, is in the public domain. 

\par Published by \smallcaps{mediastudies.press} in the \smallcaps{Public Domain} series

\href{http://mediastudies.press}{mediastudies.press} | 414 W. Broad St., Bethlehem, PA 18018, USA

\par New materials are licensed under a Creative Commons Attribution-Noncommercial 4.0 (\href{https://creativecommons.org/licenses/by-nc/4.0/legalcode}{\smallcaps{CC BY-NC 4.0}})

\par \smallcaps{Cover design}: Mark McGillivray

\par \smallcaps{Credit for scan}: Internet Archive, contributor Prelinger Library, \href{https://archive.org/details/ourmastersvoicea00rortrich}{2006 upload}

\par \smallcaps{Credit for LaTeX template}: \href{https://www.overleaf.com/latex/templates/book-design-inspired-by-edward-tufte/gcfbtdjfqdjh}{Book design inspired by Edward Tufte}, by \href{https://ctan.org/pkg/tufte-latex}{The Tufte-LaTeX Developers}

\par \smallcaps{ISBN} 978-1-951399-00-9 (print) | \smallcaps{ISBN} 978-1-951399-01-6 (ebook)

\par \smallcaps{DOI} \href{https://doi.org/10.21428/3f8575cb.dbba9917}{10.21428/3f8575cb.dbba9917}

\par \smallcaps{Library of Congress Control Number} 202094177

\par\textit{Edition 1 published in October 2020}


% THREE dedication
\newpage
\thispagestyle{empty}

\begin{adjustwidth}{0pt}{0pt}

\par\centering{\vspace*{.75in}{\fontsize{21}{20}{\emph{Dedicated to the memory of Thorstein Veblen, and to those technicians of the word whose ``conscientious withdrawal of efficiency'' may yet accomplish that burial of the ad-man's pseudoculture which this book contemplates with equanimity.}}}}

\end{adjustwidth}

\end{fullwidth}

% FIVE about the author
\thispagestyle{empty}
\begingroup
\setlength{\parindent}{0cm}\setlength{\parskip}{2ex plus 0.3ex minus 0.1ex}
\hidefromtoc
\chapter{About the Author}
\unhidefromtoc


\newthought{James Rorty} was born March 30, 1890 in Middletown, New York. He was
educated in the public schools, served an early journalistic
apprenticeship on a daily newspaper in Middletown, and was graduated
from Tufts College. Mr. Rorty was a copy-writer for an advertising
agency from 1913 to 1917, at which time he enlisted as a stretcher
bearer in the United States Army Ambulance Service. He was awarded the
Distinguished Service Cross for service in the Argonne offensive.

Since the war Mr. Rorty has worked variously as an advertising
copy-writer, publicity man, newspaper and magazine free lance. He is the
author of two books of verse, ``What Michael Said to the Census Taker''
and "Children of the Sun'', and has contributed to the \emph{Nation},
\emph{New Republic}, \emph{New Masses}, \emph{Freeman}, \emph{New
Freeman}, and \emph{Harpers.}

\endgroup

% TOC

\begin{fullwidth}

\tableofcontents

\end{fullwidth}


% FOREWORD
\chapter{FOREWORD}
\label{ch:foreword}

\newthought{Two basic} definitions will perhaps assist the reader to 
understand the scope and intent of this book.\footnote{[Clarifying footnotes from the reprint editor, Jefferson Pooley, will appear in brackets]} 

The \emph{advertising business} is taken to mean the total apparatus of
newspaper and magazine publishing in America, plus radio broadcasting,
and with important qualifications the movies; plus the advertising
agency structure, car card, poster, and direct-by-mail companies, plus
the services of supply: printing, lithography, engraving, etc. which are
largely dependent upon the advertising business for their existence.

The \emph{advertising technique} is taken to mean the technique of
manufacturing customers by producing systematized illusions of value or
desirability in the minds of the particular public at which the
technique is directed.

The book is an attempt, by an advertising man and journalist, to tell
how and why the traditional conception and function of journalism has
lapsed in this country. It describes the progressive seizure and use, by
business, of the apparatus of social communication in America.
Naturally, this story has not been ``covered'', has not been considered
fit to print, in any newspaper or magazine dependent for its existence
upon advertising.

In attempting to examine the phenomenon of American advertising in the
context of the culture it became necessary to examine the culture itself
and even to trace its economic and ideological origins. This enlargement
of scope necessitated a somewhat cursory and inadequate treatment of
many detailed aspects of the subject. The writer accepted this
limitation, feeling that what was chiefly important was to establish, if
possible, the essential structure and functioning of the phenomena.

Since the book is presented not as sociology, but as journalism, the
writer felt free to use satirical and even fictional literary techniques
for whatever they might yield in the way of understanding and emphasis.
The writer wishes to acknowledge gratefully the help and encouragement
he has received from many friends in and out of the advertising
business. The section on ``The Magazines'' is almost wholly the work of
Winifred Raushenbush and Hal Swanson. Thanks are due to Professor Robert
Lynd for reading portions of the manuscript and for many stimulating
suggestions; to Professor Sidney Hook for permission to quote from
unpublished manuscripts; to F. J. Schlink and his associates on the
staff of Consumers' Research for permission to use certain data; to
Stuart Chase for much useful counsel and encouragement; to Dr. Meyer
Schapiro for valuable criticisms of the manuscript and to Elliot E.
Cohen for help in revising the proofs; to the officials of the Food and
Drug Administrations for courteously and conscientiously answering
questions.



% PREFACE to mediastudies.press edition 
\chapter{PREFACE to the mediastudies.press edition}
\label{ch:prefacemsp}

  
\newthought{James Rorty's} \emph{Our Master's Voice} is buried treasure. The book set
off tremors when published in 1934, perhaps because its author so
decisively repudiated his former profession. But after the Second World
War, Rorty and his spirited takedown of advertising fell into near
obscurity. The scholarly literature that coalesced around ``mass
communication'' in the early postwar decades makes almost no mention of
the book. Popular treatments of advertising---like Vance Packard's 1957
best seller \emph{The Hidden Persuaders}---neglect the book too.\footnote{Vance O. Packard, \emph{\href{http://www.worldcat.org/oclc/245181}{The
  Hidden Persuaders}} (New York: McKay, 1957).} And
when \emph{Our Master's Voice} does surface today, there's usually a
filial explanation: The book tends to appear in biographical sketches of
Rorty's far more famous son, Richard.\footnote{See, for example, Neil Gross,
  \emph{\href{http://www.worldcat.org/oclc/474963500}{Richard Rorty: The
  Making of an American Philosopher}} (Chicago: University of Chicago
  Press, 2008), chap. 1.}

So no one reads James Rorty anymore. This is too bad, since the book
remains remarkably spry eighty-five years after its first printing. In
fact, Rorty's dissection of the ad business has fresh things to say to
scholars of Google-style ``surveillance capitalism.'' The good-natured
urgency of Rorty's prose resonates too---maybe especially because his
aim to bury the ``ad-man's pseudoculture'' proved a spectacular failure.
We can, in 2020, pick up where Rorty left off.

Thus \emph{Our Master's Voice} is the right book to inaugurate our
Public Domain series. It is, of course, in the public domain, having
lapsed out of copyright in 1962. But that copy-freedom is just the
book's baseline qualification: We are, at mediastudies.press, looking to
re-publish works that cling to relevance, even if they've long since
fallen out of print. An even narrower wedge of books stands out, like
\emph{Our Master's Voice}, for their unmerited banishment from the
field's memory. Such books---unheralded for no good reason---are what we
have in mind for the new series.

The Public Domain project has a pair of inspirations. The first is the
University of Chicago Press's long-running Heritage of Sociology series,
established by Morris Janowitz in the early 1960s on his return to
Chicago. The first handful of volumes were devoted to prominent figures
in what was, by then, known as the ``Chicago School.''\footnote{In his history of the Chicago department, Andrew Abbott called
  Janowitz} But the series
grew more catholic over time, with volumes devoted to scholars\marginnote{``the most industrious retrospective creator of the first
  Chicago school'' and a ``self-appointed prophet of the past''---all on
  the strength of the Heritage series. Andrew Delano Abbott,
  \emph{\href{http://www.worldcat.org/oclc/924890866}{Department \&
  Discipline: Chicago Sociology at One Hundred}} (Chicago: University of
  Chicago Press, 1999), 18--19.}---Kenneth
Burke and Martin Buber---far beyond the orbit of Chicago or even
sociology itself.

That ecumenical spirit also animates the second inspiration for the
Public Domain series, a 2004 reader titled \emph{Mass Communication and
American Social Thought: Key Texts, 1919--1968}, edited by John Durham
Peters and Peter Simonson.\footnote{John Durham Peters and Peter Simonson, eds.,
  \emph{\href{http://www.worldcat.org/oclc/54374652}{Mass Communication
  and American Social Thought: Key Texts, 1919--1968}} (Lanham, Md.:
  Rowman \& Littlefield, 2004).} The tome (and it really is one) collects
almost seventy excerpts and reprints of media-related reflection. What
unites a 1919 Sherwood Anderson short story and, say, the obscure 1959
study ``The Social-Anatomy of the Romance-Confession Cover Girl''? These
texts---and the other entries in the anthology---all offer sedimented
reflections on what was a then new panoply of mass mediums. ``These
observers,'' Peters and Simonson write,

\begin{quote}
hold unique historical positions as part of the first generations to
live with commercially supported, national-scope broadcast technologies.
They are at once informants, ancestors, and teachers. As informants,
they tell us about experiencing and studying `mass communication' as a
generation new to it. As ancestors, they speak languages we recognize
but in dialects different than our own. As teachers, their role is more
complex. Often they speak with more clarity and conceptual insight than
do the journals and books of our own day, and thus they teach by precept
and example. At other times, they display their blind spots, weaknesses,
or arrogance in such a way that we either swear never to follow their
lead or perhaps see something better because of their failure.\footnote{Peters and Simonson, \emph{Mass Communication and American Social
  Thought}, 2.}
\end{quote}

\noindent The editors sifted through their candidate texts---``blowing dust off
bound volumes''---with an eye for works that have something to say to
the present.\footnote{Peters and Simonson, \emph{Mass Communication and American Social
  Thought}, 495. Perhaps unsurprisingly, the editors included an excerpt
  from \emph{Our Master's Voice}: ``The Business Nobody Knows,'' 106--9.} This is our aim too. We endorse, moreover, the view that
a work's warrant for attention may take a variety of forms. A jarring
anachronism may merit a reader as much as, or more than, a still
apposite line of reasoning.

Peters and Simonson fault media and communication research for its
``rather pinched view of the past,'' and position their anthology as a
recovery project for the field's forgotten pluralism.\footnote{Peters and Simonson, \emph{Mass Communication and American Social
  Thought}, 8.} In the same
spirit, this Public Domain series seeks to ventilate the field's memory
of itself.

On the model of \emph{Our Master's Voice}, then, we plan to re-publish
works that:

\begin{enumerate}[leftmargin=3\parindent]
\item
  are in the public domain;
\item
  promise contemporary relevance; and yet,
\item
  have settled into obscurity.
\end{enumerate}

\noindent The first criterion constitutes an undeniable limitation, but an
important one. We are committed to open access (OA) on principle, so
charging readers to cover copyright fees isn't an option for us.
Fortunately, all works published in the United States before 1924 are
already in the public domain. What's less well known is that many books
published between 1924 and 1963 are also owned by the public. Before the
Copyright Renewal Act of 1992 made renewal automatic, copyright holders
were required to file for an extension before their twenty-eight-year
initial term ran out. Books published in 1964 were up for renewal when
the 1992 law passed, so they (and all subsequent published works) remain
intellectual property---and will stay locked for a long time.\footnote{The best book on the corporate enclosure of public knowledge remains
  James Boyle, \emph{\href{http://www.worldcat.org/oclc/317471891}{The
  Public Domain: Enclosing the Commons of the Mind}} (New Haven, Conn.:
  Yale University Press, 2008), which is, fittingly,
  \href{http://www.thepublicdomain.org/download/}{free to download}.} The
good news is that up to 80 percent of the copyright holders that
published between 1924 and 1963 failed to renew---so those works are now
owned by the public.\footnote{Sean Redmond, ``U.S. Copyright History,~1923--1964,'' \emph{New York
  Public Library Blog}, May 31, 2019,
  \url{https://www.nypl.org/blog/2019/05/31/us-copyright-history-1923-1964}.} \emph{Our Master's Voice} falls into that
category: Rorty and/or the John Day Company, the volume's publisher, did
not file for renewal, thus the copyright lapsed.

So our Public Domain books are on the open web and---crucially---they're
discoverable. We assign a new ISBN for each reprint, DOIs for each
chapter, and otherwise work to ensure that the volumes show up in
library, OA directory, and web searches. Because they're digital,
\emph{Our Master's Voice} and other volumes in the series are easy to
search and excerpt. Our underlying PubPub platform---nonprofit and open
source---adds public annotation, citation formatting, and a robust array
of auto-generated download options. We include a high-quality scan of
the corresponding originals, in all their sepia-and-Baskerville glory.
Corrections and updates are simple to make, since there's no fixed
version of record.

Major advantages thus adhere to our web-based model of open publishing.
Like the Heritage of Sociology series, we commission freshly written
introductions to contextualize the republished work. But we sidestep the
copyright muck, and the costs passed on to readers. The Peters and
Simonson volume includes four dense pages of small-print
permissions---and it's priced accordingly, out of reach for most
readers.\footnote{Peters and Simonson, \emph{Mass Communication and American Social
  Thought}, 519--23.}

Rorty, back in 1934, summarized \emph{Our Master's Voice} as ``an
attempt, by an advertising man and journalist, to tell how and why the
traditional conception and function of journalism has lapsed in this
country.'' The book describes ``the progressive seizure and use, by
business, of the apparatus of social communication in America.''\footnote{Rorty, \emph{Our Master's Voice}, ix.}
Eighty-five years later, and we are still domiciled.

\begin{flushright}\emph{Jefferson Pooley}\\ \emph{Bethlehem, PA}\end{flushright}


\newpage
\thispagestyle{plain} % empty
\mbox{}

% INTRODUCTION TO MSP EDITION
\chapter[JAMES RORTY'S VOICE: Introduction to the mediastudies.press edition]{JAMES RORTY'S VOICE: Introduction to the\\ mediastudies.press edition}
\label{ch:introduction-msp}
\chaptermark{James Rorty's Voice}

\emph{\smallcap{\LARGE{Jefferson Pooley}}}

\vspace{0.5in}

\newthought{James Rorty} announced his working knowledge of the trade in the opening
paragraph of \emph{Our Master's Voice}. Thirty years before, he reports,
he had taken a job as a copywriter at an advertising agency in New York
City. Though he preferred poetry and journalism, Rorty would continue to
work intermittently in the ad business through the 1920s. \emph{Our
Master's Voice}, among the most penetrating critiques of advertising
ever published, offers an insider's account: ``I was an ad-man once,''
Rorty confesses.\footnote{Rorty, \emph{Our Master's Voice}, ix. Page references are to the
  mediastudies.press edition; subsequent citations to the book are
  rendered as \emph{OMV}.}

The book is Rorty's coming-to-terms with an institution he knew. But it
neither chronicles his career nor gives an accounting of his
impressions. Rather, it has a different, and surprising, character:
Steeped in Rorty's leftist politics, \emph{Our Master's Voice} presents
advertising as the linchpin of a capitalist economy that it also helps
justify.

Who dared take on the publication of \emph{Our Master's Voice} in 1934?
The John Day Company, a New York firm that had---amid a steep,
Depression-era drop-off in books sales---published a series of
forty-five pamphlets notable for left-wing topics and authors.\footnote{Rorty published his own thirty-two-page pamphlet, \emph{Order on the
  Air!}, in The John Day Pamphlets series the same year. Rorty,
  \emph{\href{https://books.google.com/books/about/Order_on_the_Air.html?id=2vBCAAAAIAAJ}{Order
  on the Air!}} (New York: John Day Company, 1934). For an overview of
  Rorty's critique of commercial radio in particular, see Bruce
  Lenthall,~\emph{\href{http://www.worldcat.org/oclc/84838887}{Radio's
  America: The Great Depression and the Rise of Modern Mass Culture}}
  (Chicago: University of Chicago Press, 2007), 30--39; and Kathleen M.
  Newman, \emph{\href{http://www.worldcat.org/oclc/473847893}{Radio
  Active: Advertising and Consumer Activism, 1935--1947}} (Berkeley:
  University of California Press, 2004), 60--63.}
\emph{Our Master's Voice} appeared in this spirit, though dense and
promiscuous across twenty-six chapters and nearly four hundred pages in
its original printing. It contains fictional interludes, detours through
New Deal regulatory skirmishes, and a chapter devoted to Gillette's
campaign against the beard.

Rorty made no apologies for the book's undisciplined format. Indeed, he
disclaimed any academic purpose on the first page. \emph{Our Master's
Voice} was presented, he wrote, as journalism, ``not as sociology.''\textsuperscript{3}
Thus he granted himself license to code-switch, with what amounts to a
short story slotted in as the fourth chapter, and another devoted \marginnote{\textsuperscript{3} \emph{OMV}, ix.}\setcounter{footnote}{3}to
composite portraits (``names, places and incidents have been
disguised'') of ad workers he had known. Nevertheless, the book abounds
with dense and sophisticated analysis that is, by any measure, academic.
One especially lengthy, chart-filled chapter, co-authored with his wife
and another colleague, reports on a major empirical study of magazines.
Throughout the book Rorty spars with the country's leading social
scientists, quoting and then lacerating their work in what should
undeniably be counted as academic debate.

More important, and despite its pastiche quality, the book presents a
coherent and original theory of advertising. Its main tenet holds that
the ad business can only be understood within the totality of the
country's economy and culture. The alternative---to treat the business
of publicity as a ``carbuncular excrescence''---misses its centrality,
its foundational place in American life.\footnote{\emph{OMV}, 9.} Rorty thus insisted on a
holistic approach---in conscious contrast to the bounded inquiries of
his analytic rivals in the university system.

Rorty believed that the ad-man and his persuasive copy propped up
American society---its capitalist economy, its culture of competitive
emulation.\footnote{I have adopted Rorty's gender-exclusive language to remain faithful to
  the book's historical context, but do not otherwise condone the
  phrasing.
} In effect, he makes his argument at two levels. The first
is economic: All the billboards and radio spots, according to Rorty,
provide the fuel that keeps people buying---the coal powering the
country's merchandising juggernaut. American business would collapse
without the ad-man's ventilation.

The book's second, complementary point is that the system---an
exploitative one, in Rorty's view---relies on advertising for its
ideological warrant. This claim emerges with greater subtlety, or at
least erected around a series of sub-arguments, in the book's first few
chapters. But the key takeaway suggests that advertising serves to
ratify the prevailing American regime of class-stratified consumption.
Rorty's former coworkers are, as it were, the master's voice.

Published into the Great Depression in 1934, the book agitated an
already wounded publicity industry. It generated spirited reviews in the
popular press, too. But social scientists---the sociologists and
psychologists taking up the study of media and their audiences in small
but growing numbers---ignored \emph{Our Master's Voice}. They paid the
book no heed when it was published, and media scholars have scarcely
noticed it since.

\section{He Was an Ad-Man Once}\label{he-was-an-ad-man-once}}

One reason for the neglect, then and since, lies with Rorty himself. He
was no academic, and he didn't write like one. He was an
intellectual---a poet, an essayist, a political journalist---in the
orbit of the New York literary world. Like many of his peers, he
embraced a radical worldview that, over the course of the 1920s, became
more explicitly Marxist.

Rorty was born in 1890 in Middletown, New York, to an Irish immigrant,
himself an aspiring poet, and his schoolteacher wife. The family ran a
struggling dry goods business.\footnote{Daniel Pope, ``His Master's Voice: James Rorty and the Critique of
  Advertising,'' \emph{Maryland Historian} 19 (1988): 6. In addition to
  Pope's excellent account, the two other biographical sources on Rorty
  are Neil Gross,
  \emph{\href{http://www.worldcat.org/oclc/901849681}{Richard Rorty: The
  Making of an American Philosopher}} (Chicago: University of Chicago
  Press, 2008), chap.~1; and John Michael Boles,
  ``\href{https://doi.org/10.1177\%2F0921810698112002}{James Rorty's
  Social Ecology: Technology, Culture, and the Economic Base of an
  Environmentally Sustainable Society},'' \emph{Organization \&
  Environment} 11, no. 2 (1998): 155--79.} We know nothing much of the young
Rorty's life, but in high school he apprenticed at a local newspaper
before attending Tufts College. After graduating in 1913, he took a
copywriting post at the New York advertising agency H. K. McCann, his
first of three stints in the business. When the U.S. joined the war,
Rorty enlisted in the Army ambulance corps, served in France, and earned
a Distinguished Service Cross.\footnote{Gross, \emph{Richard Rorty}, 36; and Pope, ``His Master's Voice,'' 6.} He briefly returned to New York after
the war, then moved to California, where he wrote poetry and covered the
San Francisco literary and artistic scene for the \emph{Nation}. In need
of funds, he soon resumed work for advertising agencies, including a
stint at McCann's San Francisco office.\footnote{Pope, ``His Master's Voice,'' 7; and Gross, \emph{Richard Rorty}, 36.} A first marriage collapsed,
but Rorty soon afterward met Winifred Raushenbush, then a research
assistant to the Chicago sociologist Robert E. Park.\footnote{Raushenbush trained in Chicago's famed Sociology Department and,
  along with other research support, assisted Robert Park in his 1922
  \emph{\href{http://www.worldcat.org/oclc/493080567}{The Immigrant
  Press and Its Control}} (New York: Harper \& Bros.). Raushenbush, a
  writer in her own right, worked closely with Rorty on his prose
  projects, including \emph{Our Master's Voice}. Late in life she
  published a biography of Park,
  \emph{\href{http://www.worldcat.org/oclc/963123300}{Robert E. Park:
  Biography of a Sociologist}} (Durham, N.C.: Duke University Press,
  1979).} Rorty and
Raushenbush, the daughter of a prominent social gospel minister, fueled
each other's radical politics on their return to New York in the
		mid-1920s.\footnote{In 1927 Raushenbush and Rorty, for example, were arrested in Boston
  for protesting the imminent executions of Nicolo Sacco and Bartolomeo
  Vanzetti. Boles, ``James Rorty's Social Ecology,'' 159.} Both were steeped in the city's intellectual culture of
so-called little magazines, including Marxist organs like the \emph{New
Masses}.\footnote{Rorty was a founding co-editor of the \emph{New Masses} in 1926,
  though he was ousted the next year after political and editorial
  disputes. Pope, ``His Master's Voice,'' 8; Gross, \emph{Richard
  Rorty}, 30n4; and Alan M. Wald,
  \emph{\href{http://www.worldcat.org/oclc/14273419}{The New York
  Intellectuals: The Rise and Decline of the Anti-Stalinist Left from
  the 1930s to the 1980s}} (Chapel Hill: University of North Carolina
  Press, 1987), 54--55.
}

During this period, working from a rural Connecticut cabin, Rorty
reluctantly picked up advertising work a third time. Daniel Pope quotes
Rorty's unpublished memoir: ``I returned to my advertising vomit,
prodding my fair white soul up and down Madison Avenue and offering it
for sale to the highest bidder.''\footnote{Pope, ``His Master's Voice,'' 8.} Yet with the economy's collapse,
Rorty was laid off in 1930.\footnote{Newman, \emph{Radio Active}, 59--60.} Like many other intellectuals in the
wake of the Depression, Rorty turned to Marxist politics with new
avidity. For a short stint, he even worked on behalf of the Communist
Party's 1932 presidential slate, though he soon fell out with the party,
which he never joined. In the cause of the recently exiled Leon Trotsky,
Rorty's politics took on a decidedly anti-Stalinist cast.\footnote{Gross, \emph{Richard Rorty}, 51--52; and Boles, ``James Rorty's Social
  Ecology,'' 160. For a detailed account of Rorty's early 1930s
  entanglements with the Communist} As Richard
Rorty, Raushenbush and Rorty's only child and a future
post-philosophical luminary, recounted in a memoir, ``my parents had
been classified by the \emph{Daily Worker} as `Trotskyites,' and they
more or less accepted the description.''\textsuperscript{15}

The Hitler-Stalin Pact of 1939 stiffened Rorty's anti-Soviet posture. By
then his radical ardor had also cooled, and he began to endorse, for the
first time, New Deal interventions like the Tennessee Valley Authority.
In the war years his freelance writing, which he assiduously continued
to produce for a variety of popular and literary magazines, shifted to
health, nutrition, and consumer topics.\textsuperscript{16} By the 1950s \marginnote{Party, fast disillusion, and
  Trotskyite sympathies, see Wald, \emph{The New York Intellectuals},
  56--62, 102--5, 271. A trio of prominent anti-Stalinist Marxist
  intellectuals---Sidney Hook, Elliot Cohen, and Meyer Schapiro---are
  thanked in \emph{Our Master's Voice} for their help with the
  manuscript. \emph{OMV}, x.}he had become \marginnote{\textsuperscript{15} Richard Rorty, ``Trotsky and the Wild Orchids,'' in
  \emph{\href{http://www.worldcat.org/oclc/41311603}{Philosophy and
  Social Hope}} (New York: Penguin, 1999), 6. James Rorty had nearly
  accompanied the philosopher John Dewey to Mexico for Dewey's
  investigation into the Moscow Trotsky show trials. As Richard Rorty
  remembers, the two-volume Dewey Commission report ``were books that
  radiated redemptive truth and moral splendor'' (``Trotsky and the Wild
  Orchids,'' 5).}
an aggressive \marginnote{\textsuperscript{16} Boles, ``James Rorty's Social Ecology,`` 162--63.}Cold Warrior, penning anti-Soviet scripts for the
\emph{Voice of America} and clamoring for the American Communist Party's
legal shuttering.\setcounter{footnote}{16}\footnote{Pope, ``His Master's Voice,'' 14.} His 1954 \emph{McCarthy and the Communists},
co-authored with Moshe Decter, faulted the Wisconsin senator for
botching the anticommunist cause---for discrediting the otherwise urgent
campaign to purge Reds.\footnote{James Rorty and Moshe Decter,
  \emph{\href{http://www.worldcat.org/oclc/290550}{McCarthy and the
  Communists}} (Boston: Beacon, 1954). Rorty's anti-communism soon took
  a paranoid turn, as Pope notes: ``Rorty was convinced that the
  Communist Party had planted its agents as handymen on his Connecticut
  farm, had joined forces against him with Morris Fishbein of the
  American Medical Association, and had induced fellow-traveling
  bookstore clerks to hide his writings from public display.'' Pope,
  ``His Master's Voice,'' 14n41. See also Wald, \emph{The New York
  Intellectuals,} 272--73.}

Rorty wrote on a range of other topics through the early 1960s,
including technology, race relations, food culture, and, notably,
ecological issues---the last an area he had addressed, precociously, all
the way back in the early 1930s.\footnote{Boles, ``James Rorty's Social Ecology,`` 161.} Even as Rorty drifted right, he
remained a critic of the country's acquisitive culture. In an
unpublished reflection---written a decade before his 1972 death---he
looked back on his Depression-era critique of advertising:

\begin{quote}
I wrote \emph{Our Master's Voice} with the object of curing surgically
what I considered a malignant degeneration of culture: Advertising. Not
only did I not cure it; the disease like a cancer increased not only
relatively to the total culture but absolutely so that one might well
say that the American culture is dying from this malignancy.\footnote{Quoted in Pope, ``His Master's Voice,'' 14.}
\end{quote}

\section{Systematized Illusions}\label{systematized-illusions}}

It was Thorstein Veblen, not Marx, who supplied for Rorty the book's
argumentative anchor. Rorty acknowledged his debts to the splenetic
economist-cum-social critic with such regularity, and with such
reverence, that the book can be read---at one register---as an extension
of Veblen's scattered remarks on advertising. Though Veblen treated
``salesmanship'' as an important constituent of the pecuniary culture,
he never devoted a treatise to the business of selling. One of just two
sustained meditations on advertising appeared in a late work, the 1923
\emph{Absentee Ownership}, and it was this chapter (on ``Manufactures
and Salesmanship'') that animated Rorty's analysis.\footnote{Thorstein Veblen,
  \emph{\href{http://www.worldcat.org/oclc/752183}{Absentee Ownership
  and Business Enterprise in Recent Times: The Case of America}} (New
  York: B. W. Huebsch, 1923), chap.~11. The other treatment, which Rorty
  rarely cited, appears in Veblen,
  \emph{\href{http://www.worldcat.org/oclc/220847741}{The Theory of
  Business Enterprise}} (New York: Charles Scribner's Sons, 1904),
  55--60. For a superb treatment of both works in the wider context of
  Veblen's project, see Sidney Plotkin,
  ``\href{https://dx.doi.org/10.1108/JHRM-01-2014-0003}{Misdirected
  Effort: Thorstein Veblen's Critique of Advertising},'' \emph{Journal
  of Historical Research in Marketing} 6, no. 4 (2014): 501--22.} Yet Veblen's
imprint sinks deeper than that. Rorty's scabrous ironizing, for example,
pays explicit homage to his onetime teacher. And the concept of
emulation---the dynamic of prestige and consumption that Veblen outlined
in \emph{The Theory of the Leisure Class} (1899)---is the real engine of
\emph{Our Master's Voice}.\footnote{Veblen, \emph{\href{http://www.worldcat.org/oclc/222222126}{The Theory
  of the Leisure Class: An Economic Study in the Evolution of
  Institutions}} (New York: Macmillan, 1899).} Rorty notably refused to isolate selling
from the wider ``pseudoculture,'' opting instead for a fisheye-lens
approach. In that respect \emph{Our Master's Voice} constitutes an
enlargement, even a gentle overhaul, of Veblen's critique of
advertising.

Rorty was already familiar with Veblen's work when he attended the elder
scholar's classes at the New School for Social Research in the early
1920s.\footnote{Veblen was among the New School's} According to Rorty's unpublished memoirs, he and Veblen
struck up a brief friendship while living in the same New York City
boarding house. Rorty and the building's owner detailed \marginnote{founding faculty. See Peter M.
  Rutkoff and William B. Scott,
  \emph{\href{http://www.worldcat.org/oclc/12555131}{New School: A
  History of the New School for Social Research}} (New York: Free Press,
  1986), 14--16.}to Veblen their
experiences in the ad business---testimony that, Rorty later claimed,
informed Veblen's analysis in \emph{Absentee Ownership}. Wrote Rorty:
``What he got out of us was transmuted into the refined gold of the long
footnote'' on religion in the book's advertising chapter.\footnote{This account of Rorty's brief personal exposure to Veblen is drawn
  from Boles, ``James Rorty's Social Ecology,'' 157. Boles cites, and
  quotes from, Rorty, ``Unpublished Memoirs: Version 1,'' n.d., box 2,
  James Rorty Papers, Special Collections, University of Oregon.
  Veblen's ``Note'' appears in \emph{Absentee Ownership}, 319--25. The
  owner of the boarding house, Alice Boughton, was research director at
  the J. Walter Thompson Company. Pope, ``His Master's Voice,'' 6--7.} If Rorty
was right---that Veblen's excursus on the ``propagation of faith''
reflected their conversations from the early 1920s---then the compliment
was returned in \emph{Our Master's Voice}. He singled out Veblen's
``footnote''---really a six-page addendum to the chapter---as the key to
grasping the resonance of Christianity and the ``modern Church of
Advertising.''\footnote{Rorty, before quoting Veblen's first paragraph, wrote: ``The close
  analogy between the sales publicity methods of the Christian Church
  and those of the modern Church of Advertising was noted in 1923 by
  Thorstein Veblen, who missed little, if any, of the comedy of the
  American scene. Veblen's long foot-note (p.~319, \emph{Absentee
  Ownership}) should be read in its entirety in this connection.''
  \emph{OMV}, 208.
}

Rorty dedicated \emph{Our Master's Voice} to the ``memory of Thorstein
Veblen,'' and he quoted him in one of the book's three epigraphs.\footnote{\emph{OMV}, v, 2.}
Veblenian lacerations---phrases like \emph{doctrinal memoranda} and
\emph{creative psychiatry}---pockmark Rorty's pages.\footnote{\emph{OMV}, 13, 176, 182, 185, 201, 274, 278, 285.} And sentences
like ``Again, Veblen furnishes us with the essential clue,'' are
typical.\footnote{\emph{OMV}, 152.} Veblen's name appears more than three dozen times in
Rorty's treatise---or once every seven pages. Thus it seems fair to
conclude, at first pass, that \emph{Our Master's Voice} is the book
Veblen would have written had he devoted himself to the task.

Rorty certainly encouraged that inference. He lavished particular praise
on \emph{Absentee Ownership}. Veblen's ``brief treatment of
advertising'' in the book, Rorty wrote, ``remains today the most exact
description of the nature of the advertising phenomenon which has yet
appeared.''\footnote{\emph{OMV}, 173.} Late in \emph{Our Master's Voice}, Rorty admitted that
Veblen's volume, ``in general, has supplied the framework of theory for
this analysis.''\footnote{\emph{OMV}, 223.} Readers might thus easily get the impression that
\emph{Our Master's Voice} offers but a book-length elaboration of
Veblen's penetrating, if brief, reflections on advertising.

This isn't quite right. Rorty, for all his borrowings, departed from his
teacher in a handful of significant ways. He placed advertising at the
center of things where Veblen, if anything, deflated its importance. For
Veblen, advertising didn't change much; its main effect was to shuffle
the allotment of sales among firms all vying for a fixed, zero-sum
buying capacity. Yet Rorty, writing in the wake of the Gatsby-esque
1920s, realized that advertising had helped change the economy itself,
expanding (together with popular credit instruments) the role of
everyday consumption. Without using the phrase, \emph{Our Master's
Voice} articulated the idea of \emph{demand stimulation}---the ad-fueled
fanning of consumer desire that helped remake the country's economy and
culture. Rorty's reflections on the interlaced economics of publicity
and consumption were, to be sure, tempered by the brute fact of the
Depression. But the blueprint of an advertising-stimulated consumption
economy---an answer to overproduction and slack demand---exists in
\emph{Our Master's Voice.} The book anticipates, more than Veblen's
work, the fuller postwar articulation of advertising's
Keynesianism-through-desire.\footnote{The classic statements of publicity-driven demand stimulation
  vis-a-vis the wider U.S. economy are John Kenneth Galbraith,
  \emph{\href{http://www.worldcat.org/oclc/296586}{American Capitalism}}
  (New York: Houghton Mifflin, 1952), 98--102; and Galbraith,
  \emph{\href{http://www.worldcat.org/oclc/167255}{The Affluent
  Society}} (New York: Houghton Mifflin Harcourt, 1958), chap. 10. My
  interpretation of Veblen's economics of advertising differs from those
  of Sidney Plotkin, Georgios Patsiaouras, and James Fitchett, who draw
  a more direct line from Veblen to analyses like Galbraith's. See
  Plotkin, ``Misdirected Effort,'' 502; and Georgios Patsiaouras and
  James A. Fitchett,
  ``\href{https://doi.org/10.1108/17557501211195109}{The Evolution of
  Conspicuous Consumption},'' \emph{Journal of Historical Research in
  Marketing} 4, no. 1 (2012): 164--65.}

Crucially, Veblen embeds his treatment of the ``business of publicity''
in his broader analysis of the U.S. economy.\footnote{Veblen, \emph{Absentee Ownership}, 300.} The core idea, from
\emph{The Theory of Business Enterprise} (1904) onward, is that
businesses deliberately scale back production to protect their
profits---to prevent prices from falling below costs. Veblen called this
``sabotage,'' with profit-hoarding ``business'' hallowing out
``industry.'' Since the ``market is not to be overstocked to an
unprofitable extent,'' the captains of business turn to the ``strategic
withholding of productive efficiency.''\footnote{Veblen, \emph{Absentee Ownership}, 285.} Veblen regarded the
slackening as deeply offensive---an affront to the country's productive
capacity and a deplorable and selfish waste, one that underwrote a
parasitic leisure class.

Veblen applied this sabotage framework, including its Norwegian
asceticism and producerist ethic, to advertising itself---resulting in a
strikingly autarkic analysis. Spending on ``salesmanship,'' Veblen's
preferred term, was growing rapidly, leading to higher prices for
consumers. Yet all those advertising outlays merely reshuffled a deck
of, ultimately, capped size: ``The total volume of sales at any given
time is fixed within a narrow margin.'' Salesmanship is all about
winning customers from competitors---``the art of taking over a
disproportionate share of this run of sales.''\footnote{Veblen, \emph{Absentee Ownership}, 287. Only in a footnote did Veblen
  make a qualified concession to the stimulative potential, or at least
  diversion from savings, of advertising---and even then there's only a
  ``little something'' at stake: ``There is the qualification . . . that
  the current, very urgent, sales-publicity may be presumed to divert a
  little something from savings to consumptive expenditures, and so may
  add that much of a margin for funds to the volume or purchasing-power
  currently available for expenditure on advertised goods'' (309n14).}

Yes, Veblen concluded, advertising matters; after all, it's taking a
growing share of the economy and running up production costs (and
therefore prices). Yet he ultimately considered it waste,
professionalized waste, since what's at stake is market share among big
profit-protecting firms. To Veblen, the proportion of the economy given
over to consumption was a zero-sum game.\footnote{Veblen made the point repeatedly, without ambiguity: ``The total
  volume of purchasing funds available at any given time {[}is{]} fixed
  within a relatively narrow margin of fluctuation. So that each of
  these competitive sellers can gain only at a corresponding loss to the
  rest.'' Veblen, \emph{Absentee Ownership}, 299. Advertising operates
  in a ``closed market,'' one in which ``one seller's gain is another's
  loss'' (299--300).} Salesmanship resembled
trench warfare, with small, meaningless gains made at great expense. The
whole sector, then, was irrational, if also explainable: Firms ramp up
publicity spending as a competitive necessity, since otherwise their
competitors will drive them out of business with their own
campaigns.\footnote{The competitive inter-firm emulation---the advertising arms
  race---leads to ``a continued increase of sell-costs and a continually
  more diligent application to salesmanship.''} This arms race generates a sprawling, even routinized
advertising industry---staffed by ``publicity engineers'' trained (to
Veblen's disgust) at the country's most august universities.\textsuperscript{37}

Thus salesmanship, to Veblen, constituted a wasteful cog in a system
characterized, even defined, by business sabotage. Modern capitalism was
the story of business deliberately holding back the country's productive
capacity. This claim served as the bedrock of Veblen's economics, and he
erected his analysis of advertising on its foundation. Advertising, in
fact, was just another layer of business sabotage in Veblen's
terms---indeed a symptom rather than a cause. He called it
``salesmanlike sabotage.''\textsuperscript{38}

The closest Veblen got to conceding advertising's broader stirring of
desire\marginnote{Veblen, \emph{Absentee
  Ownership}, 288. Advertising, once one company starts spending,
  imposes a ``necessity to all the rest, on pain of extinction.'' The
  result is a ``competitive multiplication'' of the ``ways and means of
  salesmanship''; firms have no choice but to ramp up their expenditures
  as a defensive maneuver, on ``penalty of failure'' (303--4).}---its stimulus \marginnote{\textsuperscript{37} Veblen, \emph{Absentee Ownership}, 296. Veblen devoted an acidic,
  footnoted paragraph to the emergence of business, marketing, and
  advertising degree programs. Universities, he wrote, are ``turning out
  a rapidly swelling volume of graduates in this art of `putting it
  over.'\,'' This ``scholastic propagation of salesmen'' is both a
  contributor to, and a reflection of, the ad profession's
  formalization---its ``standardised'' processes and output (306n12).}to an emerging \marginnote{\textsuperscript{38} Veblen, \emph{Absentee Ownership}, 296.}\setcounter{footnote}{38}consumer culture---is in passing
reference to the production of \emph{customers}. If salesmen make
anything, he claimed, it's the buyers for their clients' products.
Advertisers may write copy, design billboards, and the rest, but they're
really all about the ``fabrication of customers,'' the manufacture of
consumers.\footnote{Veblen, \emph{Absentee Ownership}, 306. Veblen: ``Judicious and
  continued expenditures on publicity and the like expedients of
  salesmanship will result in what may fairly be called a
  quantity-production of customers for the purchase of goods or services
  in question'' (305).
} This is, indeed, in the territory of demand
stimulation---and it's a claim, however fleeting, that Rorty ran with in
\emph{Our Master's Voice}. Veblen himself pulled back from the full
implications of the production of desire, on the same autarkic grounds
that animate his wider analysis. ``There is, of course, no actual
fabrications of persons endowed with purchasing-power \emph{ad
hoc}''---even if ad agencies liked to claim otherwise. The reason? The
economy is a closed system, with a fixed customer base. ``Viewed in the
large, what actually is effected is only a diversion of customers from
one to an other of the competing sellers, of course.''\footnote{Veblen, \emph{Absentee Ownership}, 305n11.} So salesmen
manufacture customers, but only within the economy's existing enclosure.

Rorty's claims notwithstanding, the debts that \emph{Our Master's Voice}
owes to Veblen are more protean. There is the cutting moralism itself.
Salesmanship, to both men, was tragic \emph{and} farcical---the practice
(in Veblen's words) of getting ``a margin of something for nothing, and
the wider the margin the more perfect the salesman's work.''\footnote{Veblen, \emph{Absentee Ownership}, 291. Veblen contrasted salesmanship
  with ``workmanship'' as ``two habits of thought''---the latter defined
  as the ``old order of industry, under the regime of husbandry,
  handicraft and neighborhood workmanship.'' Publicity and the art of
  the sale are gaining on workmanship, which however survives as a
  ``slow-dying prejudice'' in pockets of the culture (291--92).} Rorty
adopted Veblen's caustic comedy as his own prose style too. Phrases like
the ``blandishments of the huckstering salesman'' could appear in the
paragraphs of either writer.\footnote{Veblen, \emph{Absentee Ownership}, 290.} A handful of the Veblenian witticisms
indeed appear repeatedly in \emph{Our Master's Voice}, and these mark
the real register of the senior scholar's influence. Such
arguments-in-a-phrase, moreover, are often rescued from Veblen's
footnotes---mined and polished by Rorty, then expanded into
chapter-length meditations.

Consider a single, high-density footnote in \emph{Absentee Ownership}:

\begin{quote}
The production of customers by sales-publicity is evidently the same
thing as a production of systematised illusions organized into
serviceable `action patterns'---serviceable, that is, for the use of the
seller on whose account and for whose profit the customer is being
produced. It follows therefore that the technicians in charge of this
work, as also the skilled personnel of the working-force, are by way of
being experts and experimenters in applied psychology, with a
workmanlike bent in the direction of what may be called creative
psychiatry. Their day's work will necessarily run on the creative
guidance of habits and bias, by recourse to shock effects, tropismatic
reactions, animal orientation, forced movements, fixation of ideas,
verbal intoxication. It is a trading on that range of human infirmities
which blossom in devout observances and bear fruit in the psychopathic
wards.\footnote{Veblen, \emph{Absentee Ownership}, 306--7n12. The footnote's first
  paragraph, on the uptake of advertising in higher education, is not
  quoted here.
}
\end{quote}

\noindent \emph{Our Master's Voice}, to a remarkable extent, offers a four
hundred--page meditation on this single passage from the small-type
depths of Veblen's tome. The paired-word phrases---\emph{systematized
illusions}, \emph{action patterns}, and \emph{creative psychiatry}---for
Rorty supplied the key insight. He invoked the terms, quoted them with
reverence, and then unspooled them with a sustained concentration that
exceeded (or delivered on) Veblen's fleeting mentions. Even the
footnote's last sentence, with its ``human infirmities'' and
``psychopathic wards,'' registers in an outsized manner in \emph{Our
Master's Voice}, featured as one of the book's three epigraphs.\footnote{Rorty's quoted version, ``A trading on that range of human infirmities
  that blossoms in devout observances and bears fruit in the
  psychopathic wards,'' is slightly different. \emph{OMV}, 2.}

Veblen's footnote, and the other bits of \emph{Absentee Ownership} that
drew Rorty's attention, center on the psychology of advertising's
appeal. The business of publicity, in Veblen's phrase, is ``applied
psychology,'' the calculated exploitation of human irrationality.
Veblen's treatment of the theme remained, again, brief: This footnote
and two additional, probing pages.\footnote{Veblen, \emph{Absentee Ownership}, 310--11.} The advertiser's ``raw
material,'' to Veblen, was ``human credulity,'' his product,
``profitable fixed ideas.'' The main strategy preyed on fear in general,
and on fear of losing prestige in particular.\footnote{Veblen, \emph{Absentee Ownership}, 310.} The prospect of
embarrassment, the shame at falling behind one's peers, marks the target
of the ad-man's ``intoxicating verbiage.''\footnote{Veblen, \emph{Absentee Ownership}, 311n17.}

Here Veblen had re-entered the territory of his earlier and most famous
work on competitive emulation, \emph{The Theory of the Leisure Class}
(1899). It's \emph{this} Veblen that animates Rorty's book, more than
the later works' economics of business sabotage. To Rorty, advertising's
fundamental mechanism exploited the emulative yearnings of consumers.
Publicity, indeed, serves as the main prop to a wholesale \emph{culture}
of acquisitive emulation---in the thick, pervasive sense of ``culture.''
For Rorty, more than for his teacher, advertising cut deep.

He was quick, for example, to grant some autonomy to advertisers
themselves---to their aesthetic pretensions and professional
self-regard. As ``advertising craftsman,'' we (Rorty included himself)
are motivated not just by money but also by ``an obsessed delight in the
materials of our craft.'' Thus business may indeed sabotage industry in
the broad sense. ``True,'' Rorty wrote. But as creative workers, ``we
were and are parasites and unconscious saboteurs too.'' The ad-man's
artistic self-image comes in for relentless mockery, but at the same
time Rorty carved out a certain space---and considerable sympathy---for
his peers in the ranks of copywriters and graphic artists. He even went
so far as to suggest that capitalism's ``exploitative functionaries,''
in their craft-driven sabotage, may yet bring the system down from
within.\footnote{\emph{OMV}, 153. See also \emph{OMV}, 242--43.} This, at least, is the implication of the book's first-page
encomium to Veblen:
\pagebreak
\begin{quote}
Dedicated to the memory of Thorstein Veblen, and to those technicians of
the word whose `conscientious withdrawal of efficiency' may yet
accomplish that burial of the ad-man's pseudoculture which this book
contemplates with equanimity.
\end{quote}

\noindent The quoted phrase, the ``conscientious withdrawal of efficiency,'' had
been invoked by the Industrial Workers of the World (IWW), a radical
union, as a tactic of sabotage. Beginning in 1922, Veblen had repurposed
the expression as an arch shorthand for his theory of business
sabotage.\footnote{See Veblen,
  \emph{\href{http://www.worldcat.org/oclc/1048055745}{Engineers and the
  Price System}} (New York: B. W. Huebsch, 1921), 1, 8--23, 166; and
  Veblen, \emph{Absentee Ownership}, 217--21, 285--86, 394--403.} And so it appears fitting that Rorty restored the phrase's
IWW meaning in the book's dedication, calling on his fellow ad workers
(``technicians of the word'') to sabotage their own cultural machinery.

The broader point: Rorty took advertising far more seriously than his
teacher. He conceded to Veblen that salesmanship constituted a form of
``economic parasitism.''\footnote{See, for example, Rorty: ``In the \emph{Theory of Business Enterprise}
  and elsewhere in the whole body of his work, Veblen notes that
  advertising is one element of the `conscientious sabotage' by which
  business keeps the endlessly procreative force of science-in-industry
  from breaking the chains of the profit system.'' \emph{OMV}, 152--53.
  See also 54--55.} But for Rorty, the institution of publicity
extended far beyond the economy, to the ``culture considered as a system
of values and motivations by which people live.''\footnote{\emph{OMV}, 79.} Thus when he
brushed up against Veblen's portrait of advertising---as a closed system
of allocative waste---Rorty gently pushed back. He noted that in the
early 1920s, when Veblen was writing, the salesman was still an
``upstart and a parvenu''---a mere cog in the businessman's
self-sabotaging gear-works. ``But times have changed,'' Rorty insisted.
Advertising had since become an industry ``no less essential than coal
or steel.'' It was now no longer merely an appendage to business: in the
decade since \emph{Absentee Ownership}, the ad-man had become the
``first lieutenant of the new Caesars of America's commercial imperium
not merely on the economic front but also on the cultural front.''\footnote{\emph{OMV}, 233--34. Even Veblen's meditation on the twinned
  propagandas of religion and advertising---a point that Rorty,
  apparently, had helped inform in the course of the two men's brief
  friendship in the early 1920s---struck Rorty, by the 1930s, as
  obsolete. Veblen's ``ironic patronage of the emerging priesthood of
  advertising,'' Rorty wrote, ``sounds astonishingly inept and dated.''
  Religion proper had lost its hold since Veblen's book, while the
  ``religion of the ad-man is everywhere dominant both as to prestige
  and in the matter of administrative control'' (209).}
By culture Rorty meant the whole American belief system, one
increasingly fixed on status competition---on emulation and
one-upmanship, fueled by advertising's appeal to human infirmity.


\section{The Theory of the Leisure
Economy}\label{the-theory-of-the-leisure-economy}}

\emph{Our Master's Voice} was published at the Depression's nadir, so
it's surprising that Rorty focused his attention elsewhere. The book
does occasionally nod to the economy's free fall, often in service to
the claim that capitalism would soon collapse. There are other moments
of note, including five phantasmagoric pages on advertising as a giant
machine---a ``coldly whirring turbine'' that emits life-draining
``jabberwocky,'' even as its human fuel runs down in the Depression's
fourth punishing year.\footnote{Wrote Rorty: ``After four depression years the jabberwocky is hungry.
  It has devoured large sections of the lower and lower middle classes
  and expelled} But to a remarkable extent, the book remains
focused on the fulsome 1920s and the decade's ``endless chain of
selling.''\textsuperscript{54} The Depression itself comes off as a late-arriving
character, \marginnote{their dry bones, burned clean of their buying power, into
  the out darkness. There the electric breath of the jabberwocky still
  plays on them, but they are ash and slag. They cannot burn, they
  cannot feed the machine'' (\emph{OMV}, 54).}granted a few short lines. The spotlight, instead, shines on
advertising's success---via emulation and ``style-terror''---at
manufacturing new \marginnote{\textsuperscript{54} \emph{OMV}, 31.}\setcounter{footnote}{54}desire.\footnote{\emph{OMV}, 157.}

Rorty claimed that the economy, weighed down by surplus production,
required an artificial stimulus of demand. The problem, in the
``\thinspace`surplus economy' phase of industrial capitalism,'' is
overproduction.\footnote{\emph{OMV}, 211.} The solution is advertising. On this point Rorty was
blunt and repetitive: The engine of the economy needs the ``ad-man's
foot on the throttle, speeding up consumption, preaching emulative
expenditure, `styling' clothes, kitchens, automobiles---everything in
the interest of more rapid obsolescence and replacement.''\footnote{\emph{OMV}, 8--9.} The
economist's account of supply and demand in natural harmony, in
self-regulating equilibrium, was itself obsolete. The crucial function
of publicity, then, was to rescue capitalism---to animate, or even to
create whole cloth, customers to consume the system's excess capacity.
Any lingering ``puritanism in consumption'' in the populace proved
``intolerable,'' and had to be snuffed out.\footnote{\emph{OMV}, 176.} Here is Rorty's key
departure from Veblen: Where the teacher saw deliberate slackening of
supply---sabotage---the pupil saw ventilation of demand.

This was advertising's indispensable role, and it served as the basis
for Rorty's otherwise startling claim that newspapers, magazines, radio,
and the cinema were, at their core, ``advertising media.''\footnote{\emph{OMV}, 115.} All the
column-inches of newsprint, the radio dramas, the latest Hollywood
releases amounted to ``filler,'' intended merely to entice readers or
moviegoers to consume the ads.\footnote{\emph{OMV}, 66. ``To the magazine editor and the ad-man the magazine
  consists of two parts: advertisement and filler,'' wrote Rory. ``The
  filler is designed to carry the advertisements. With rare exceptions,
  no way has so far been discovered of getting the public to pay for
  advertisements presented without filler. Hence the filler.''} If the commercial media had an
overriding objective, it was to ``nourish and stimulate the buying
motive.''\footnote{\emph{OMV}, 56.} The point of the media's editorial or narrative trappings,
in other words, was to package and deliver audiences to advertisers.\footnote{\emph{OMV}, 115.}
It makes for a striking argument, partly because it anticipates, by a
half century, the claims of scholars like Sut Jhally and Dallas Smythe
that the ``audience commodity'' constitutes the real product of
commercial mass media.\footnote{See Dallas W. Smythe,
  \emph{\href{http://www.worldcat.org/oclc/8176628}{Dependency Road:
  Communications, Capitalism, Consciousness, and Canada}} (Norwood, NJ:
  Ablex, 1981); and Sut Jhally,
  ``\href{https://journals.uvic.ca/index.php/ctheory/article/download/13928/4701}{Probing
  the Blindspot: The Audience Commodity},'' \emph{Canadian Journal of
  Political and Social Theory} 6, no. 1--2 (1982): 204--10.
}

For Rorty, the mechanism for making buyers out of citizens came in the
form of induced emulation. In the spirit of Veblen's \emph{Theory of the
Leisure Class}, advertising preyed on the anxieties of
comparative social worth to spur consumption. If advertising at core was
the ``competitive manufacture of consumption habits,'' its technique (in
Rorty's favorite Veblenism) was ``creative psychiatry.''\footnote{\emph{OMV}, 274.} The
populace is driven to buy so as to forestall social slippage: this is
the governing logic of a consuming culture fanned by the agencies and
the media businesses they underwrite. ``Advertising,'' Rorty wrote, in
one of many equally vigorous summations, ``is a doctrine of material
emulation, keeping up with the Joneses, conspicuous waste.''\footnote{\emph{OMV}, 14. See also \emph{OMV}, 24, 56--57, 157--58, 179, 196,
  211, and 224.} Rorty's
shorthand for all this, the ad-man's ``pseudoculture,'' is also the
book's key term, its indictment by neologism.

Ironically to Rorty, advertisers turned to an older, ``organic'' culture
for their source material, one they were at the same time busily
dislodging.\footnote{Wrote Rorty, for example: ``The advertising-manufactured substitute
  for these organic cultures is a national, standardized, more or less
  automatic mechanism, galvanized chiefly by pecuniary motivations and
  applying emulative pressures to all classes of the population.''
  \emph{OMV}, 50. In a prescient aside on the rise of the
  advertising-mocking magazine \emph{Balyhoo} in the early 1930s, Rorty
  observed how easily advertisers adapted to its fang-less satire.
  \emph{Balyhoo} is an enterprise in ``tertiary parasitism'':
  advertising is a parasite on business, and the magazine, in turn,
  ``parasites on the grotesque, bloated body of advertising'' (278).} In other words, the acquisitive social psychology
demanded by the economy fed off the country's past---its pastoral
humanism and small-town craftsmanship, creating a parasitic relationship
between advertising and the country's organic culture. Yet
problematically for advertisers, the pseudoculture held only a shallow
appeal, since the population ``wistfully desire{[}d{]}'' the ``older
more human culture.''\footnote{\emph{OMV}, 57. Interestingly, Rorty identified sexual frankness with
  the older organic culture (60, 79--80). In the ``field of sex,'' the
  ``mature artist exhibits neither timidity nor shame,'' he wrote,
  citing D. H. Lawrence and Walt Whitman (the latter of whom Rorty's
  poetry was often compared). The ``commercial sex fictioneer,'' by
  contrast, must make his prudish surrender to ``Puritan conviction''
  (87--88). There is an unmistakeable Freudian undercurrent to the
  book's treatment of sex, which seems unsurprising given Rorty's
  intellectual milieu. As he wrote in the chapter devoted to the theme,
  ``The enterprise of turning people, with their normal sexual desires
  and human affections, into gold, is greatly helped by the fact that
  our Puritan cultural heritage is peculiarly rich in the
  psychopathology of sex'' (162, chap.~12). Rorty's aversion to the
  ``residual Puritanism'' informed his extended, and vituperative,
  dismissal of the early 1930s Payne Fund studies of movies and children
  (188--94). ``Although the investigators made much pother about the
  `objective' `scientific' nature of this fact-finding study, they could
  scarcely escape value judgments, and Mr.~Forman {[}in the summary
  volume{]} frankly applies such judgments in his popularization. They
  are middle-class value judgments, derived from the conventional mores
  of the middle-class community, and applied to an industry which is
  organized to serve not the classes, but the masses'' (192--93). The
  Payne Fund actually asked Rorty to review the studies' popular summary} As a result, the editorial recipe for the
advertising-dependent media needed---if it wanted Americans to watch,
read, and listen---to include ingredients from the country's
half-displaced organic past.

Rorty develops the argument in the book's remarkable sixth chapter, a
sprawling, chart-filled report on a study of thirteen mass-circulation
magazines that, by the author's own account, was ``almost wholly'' the
work of his wife, Winifred Raushenbush, and a colleague.\textsuperscript{68} The
chapter offers a self-contained, empirically rich treatment of the
country's stratified magazine market, one tailored to specific ``class
cultures.''\textsuperscript{69} Only those titles targeting the wealthy, like
\emph{Harper's Bazaar}, bathed readers in undiluted snobbism. In the
rest---those outlets aimed at the poor and the middle class---the
acquisitive culture ``battles'' with an ``older tradition and culture.''
Many titles leaned emulative, in other words, while the remainder
favored the ``organic.'' Either way, they presented a ``considerable
admixture'' of the new and old---and by necessity. ``One may say, in
summary, that the acquisitive culture cannot stand on its own feet,''
the authors wrote; ``it does not satisfy.''\textsuperscript{70} Hence the need for
parasitism.

Rorty and his coauthors found a measure of hope in the population's
implicit rejection of raw emulation: ``The American people do not like
this pseudoculture, cannot live by it, and, indeed, never have lived by
it.'' Here and elsewhere in the book, a residue of romantic nostalgia
emerges, a plaintive register of displacement---despite the work's many
professions of forward-facing radicalism.\textsuperscript{71} For example, the authors
claim that the Depression-ravaged country yearns to ``discover by what
virtues, by what pattern of life, the Americans of earlier days
succeeded in being admirable people, and in sustaining a life, which, if
it did not have ease and luxury, did seem to have dignity and charm.''
If that sounds like an endorsement, the Rorty and his colleagues quickly
pivoted to more radical prospects. Yes, the organic past was the
population's ``main drift of desire,'' but ``other drifts'' existed too:
``Some editors and readers even envision revolution''---a substitution
of a ``new culture'' for the organic and acquisitive alternatives.\textsuperscript{72}
This last point, however, is delivered in haste. It appears \marginnote{volume before its publication; Rorty excoriated the book so savagely
  that the Fund considered stopping publication. See Garth Jowett, Ian
  C. Jarvie, and Kathryn H. Fuller,
  \emph{\href{http://www.worldcat.org/oclc/32390197}{Children and the
  Movies: Media Influence and the Payne Fund Controversy}} (Cambridge:
  Cambridge University Press, 1996), 107--8.}limp \marginnote{\textsuperscript{68} \emph{OMV}, x. The chapter's third author is named as Hal Swanson,
  without further identification.}and
\marginnote{\textsuperscript{69} ``The United States,'' the authors wrote, ``does not have one
  homogenous culture; it has class cultures'' (\emph{OMV}, 60).}convictionless---\marginnote{\textsuperscript{70} \emph{OMV}, 61.}a forced \marginnote{\textsuperscript{71} In his memoir the philosopher Sidney Hook captured some of his
  friend's yearning for an authentic past, including its ecological
  dimension: ``James Rorty was at heart a poet, sickened by the
  commercialism of capitalist life and culture and up in arms at the
  cruelties and injustices of the depression. He made his political
  choices on the basis of his moral empathy and his sense for the
  integrity and authenticity of the persons with whom he associated. He
  had a love for the soil and the natural life, and long before the
  environmental movement was born, he held forth against the evils of
  pollution and the dangers of the use of chemicals and preservatives in
  the nation's food supply.'' Hook,
  \emph{\href{http://www.worldcat.org/oclc/901420772}{Out of Step: An
  Unquiet Life in the 20th Century}} (New York: Harper \& Row, 1987),
  182. As Daniel Pope observes, Rorty's putative Marxism coexisted with
  a longing---shared by many other Depression-era intellectuals---for
  ``community and authenticity in a fragmented and baffling society.''
  Pope, ``His Master's Voice,'' 10.}incantation of \marginnote{\textsuperscript{72} \emph{OMV}, 79.}\setcounter{footnote}{72}radical faith that the book's
authors, in the end, seem to doubt.


\section{It Could Happen Here}\label{it-could-happen-here}}

The ad-man's pseudoculture resembles a living thing, but it is, to
Rorty, devoid of all life---inorganic and artificial. His prose turns
purple on this point. The pseudoculture

\begin{quote}
is a robot contraption, strung together with the tinsel of material
emulation, galvanized with fear, and perfumed with fake sex. It exhibits
a definite glandular imbalance, being hyperthyroid as to snobbism, but
with a deficiency of sex, economics, politics, religion, science, art
and sentiment. It is ugly, nobody loves it, and nobody really wants it
except the business men who make money out of it. It has a low brow, a
long emulative nose, thin, bloodless, asexual lips, and the receding
chin of the will-less, day-dreaming fantast. The stomach is distended
either by the abnormal things-obsessed appetite of the middle-class and
the rich, or by the starved flatulence of the poor. Finally it is
visibly dying for lack of blood and brains.\footnote{\emph{OMV}, 85.}
\end{quote}

\noindent It's the last line's claim---that the publicity regime would soon
collapse under its own diseased weight---that Rorty had trouble
sustaining in the balance of the book. In Rorty's holistic terms, the
demise of advertising amounted to the end of capitalism, as the two
share a fate. The publicity industry may be an effect of, an emanation
from, the market economy, but it remained indispensable all the same.
Behind the ad-man lay the ``whole pressure of the capitalist organism,''
Rorty proclaimed, ``which must sell or perish.''\footnote{\emph{OMV}, 34.}

So the question of when, or whether, advertising and its enfolding
economy would, in fact, perish haunts \emph{Our Master's Voice}. One
thread in the book seems hopeful: The system is edging, inevitably and
soon, over the cliffs of history. American capitalism cannot maintain
itself for long, because its ``underlying economic and social premises
are obsolete in the modern world.''\footnote{\emph{OMV}, 9. For an interesting discussion of Rorty's perch between
  hope and disillusion, see Dan Schiller,
  \emph{\href{http://www.worldcat.org/oclc/133167542}{Theorizing
  Communication: A History}} (New York: Oxford University Press, 1996),
  69--71.} So too with advertising: ``One
needs but little knowledge of history, or of the movement of
contemporary economic and social forces, to know that it can't last.''
Its tower, Rorty added, is tottering.\footnote{\emph{OMV}, 25.}

Is it possible to rehabilitate the ad-man's pseudoculture? The answer,
to Rorty, is the ``same answer which must be given to the question: `Is
it possible to rehabilitate the capitalist economy?'\thinspace'' No.~Both the
economy and its acquisitive culture are caught in late-stage
decadence---``very frail and ephemeral,'' primed for revolutionary
toppling. And so, in this thread of the book, Rorty dismissed efforts at
reform, relentlessly pummeling liberal social critics, some of them
social scientists. Their carefully targeted interventions---their calls
for ethics and standards in the profession, for example---appear like
the snake oil ads they aim to eradicate. The competitive pressures of
advertising \emph{required} mendacity; codes and reforms, ``under our
existing institutional setup,'' would either deprive stockholders or
inflate consumer costs. The alternative to bad advertising wasn't good
advertising; it was ``no advertising.''\footnote{\emph{OMV}, 12.}

To Rorty, the effort to isolate the trade from its economic enclosure,
then to rub away its most appalling stains, constitutes an act of
self-congratulatory futility. He deemed criticism of advertising's
corruption of journalism, for example, ``beside the point,'' since its
roots sank so deep: ``the objective forces of the competitive capitalist
economy.''\footnote{\emph{OMV}, 115.} Likewise, draft New Deal legislation to stymie the
publicity industry's most egregious charlatans would leave the machinery
of advertising whirring: ``Congress can and probably will legislate
itself blue in the face, without changing an iota of the basic economic
and cultural determinants.''\footnote{\emph{OMV}, 139. The book's last three chapters (23--25), before the
  brief conclusion, offer a meticulously detailed chronicle of
  then-pending advertising legislation. For a history of these debates,
  and their complicated denouement, see Inger L. Stole,
  \emph{\href{http://www.worldcat.org/oclc/728157413}{Advertising on
  Trial: Consumer Activism and Corporate Public Relations in the 1930s}}
  (Champaign: University of Illinois Press, 2006), chap.~3. Though Rorty
  repeated his arguments against reform---downplaying the legislation's
  merits and even the nascent consumer movement behind the push---he
  also, half-grudgingly, admitted that the laws were worth passing
  anyway. Even the most robust bill would ``still leave untouched the
  major contradictions of capitalism.'' The fight is ``none the less
  important and fruitful,'' in part because the agitation itself has
  ``brought to light serious cleavages between the vested interests
  affected'' (\emph{OMV}, 283).} The industry's mendacity could not be
burned off; it proved elemental, impervious to the starchy meliorism of
liberal do-gooders.

Rorty took the anti-reformist position to its logical conclusion,
refusing the commonplace distinction between propaganda and education.
For post--World War I critics of propaganda, education stood as the
salutary other---an antidote to manipulation and inoculation in the
classroom.\footnote{On propaganda critics and education in the interwar years, see J.
  Michael Sproule,
  \emph{\href{http://www.worldcat.org/oclc/875916432}{Propaganda and
  Democracy: The American Experience of Media and Mass Persuasion}} (New
  York: Cambridge University Press, 1997).
} Rorty would have none of that. He lumped in schools and
universities with the most shameless propaganda factories: the ``purpose
and effect of these combined institutions'' was ``rule''---by which he
meant ``their shaping and control of the economic, social and
psychological patterns of the population in the interests of a
profit-motivated dominant class, the business class.''\footnote{\emph{OMV}, 114.} To Rorty, the
defenders of education themselves engaged in acts of propaganda, in
contrast to the advertising man, who was at least unblinkered about his
art's pervasive reach.\footnote{Rorty took sustained aim at the sociologist Frederick Lumley and his
  recently published
  \emph{\href{http://www.worldcat.org/oclc/741008656}{The Propaganda
  Menace}} (New York: The Century Co., 1933). Lumley hinged his critique
  on the education/propaganda contrast, Rorty noted. ``And it is
  precisely there that his definition falls down'' (\emph{OMV}, 116).
  Rorty was especially irritated by Lumley's apparent indifference to
  the} In their way, schools and colleges proved
more insidious than the overt persuasion industries, since educators
cloaked their fealty to the ``interest and prejudices of the ruling
class.''\textsuperscript{83} At any rate, Rorty saw nothing redemptive about schooling
in a capitalist order:

\begin{quote}
Advertising is propaganda, advertising is education, propaganda is
advertising, education is propaganda, educational institutions use and
are used by advertising and propaganda. Shuffle the terms any way you
like . . . all three, each in itself, or in combination, are
\emph{instruments of rule}.\textsuperscript{84}
\end{quote}

\noindent The reformist road to social change, for Rorty, equaled an
accommodationist dead end. Reporters' codes of ethics and
truth-in-advertising regulations made things worse by applying a patina
of legitimacy to a corrupt order. That position, of course, presumed
that \marginnote{underlying economics, which ``is itself a kind of propaganda''
  (117).}revolution \marginnote{\textsuperscript{83} \emph{OMV}, 118.}was \marginnote{\textsuperscript{84} \emph{OMV}, 125.}\setcounter{footnote}{84}possible, even likely. And in his dismissal of
evolutionary, stepwise change, Rorty---no doubt knowingly---joined a
debate among Marxists as old as the movement itself. If the system's
collapse is imminent, and guaranteed by its own contradictions, aren't
reformist palliatives just delaying the desired inevitable? It's a view
fueled by the confidence that the revolution is coming---with good
results.

A second strand of \emph{Our Master's Voice}, sometimes awkwardly
juxtaposed to the first, questions both postulates. Much of the book's
thrust suggests the resilience of American capitalism---and that the
system's staying power is grounded, to a large extent, in advertising
itself. That's the premise of the volume's title. The ad-man's
systematized illusions and creative psychiatry are what saves an
exploitative system from those it exploits. This ideological role proves
no less important than its economic priming: advertising is the
``shaping of the economic, social, moral and ethical patterns of the
community into serviceable conformity with the profit-making
interests.''\footnote{\emph{OMV}, 13.} This is nothing less than ``American
rule-by-advertising.''\footnote{\emph{OMV}, 16.} By promoting a culture of acquisitive
emulation, the New York firms proffer a service to the ``real rulers''
in business and finance.\footnote{\emph{OMV}, 19.} They, and the media they underwrite, are
the master's voice:

\begin{quote}
The point of view adhered to in this book is that of regarding the
instruments of social communication as \emph{instruments of rule, of
government}. In this view the people who control and manage our daily
and periodical press, radio, etc., become a sort of administrative
bureaucracy acting in behalf of the vested interests of business.\footnote{\emph{OMV}, 107.}
\end{quote}

\noindent Here Rorty tapped into another, more pessimistic current in Marxist
thought. Forced to confront the thwarted European revolutions after
World War I---and anomalous success in Russia---a number of Marxist
intellectuals sought to explain capitalism's durability. Theirs was the
problem of consent: Why do the working classes accept, even tighten,
their own chains? Figures like Antonio Gramsci and Georg Luk\'acs, in a
tradition often labeled ``Western'' Marxism, tended to respond that the
masses took on the system's values and internalized its principles as
common sense.\footnote{The best overview of Western Marxism remains Martin Jay's magisterial
  \emph{\href{http://www.worldcat.org/oclc/9970019}{Marxism and
  Totality: The Adventures of a Concept from Luk\'acs to Habermas}}
  (Berkeley: University of California Press, 1984).} Many such accounts view organs of mass communication
as the principal means of cultural reinforcement. \emph{Our Master's
Voice} is an installment in that Western Marxist project, seeking to
explain---like the others---why the revolution is always deferred.

There's a third, and final, thread in the book, an unholy mix of the
first two: fear that radical social change, all too imminent, will bring
fascism rather than socialism. The Weimar collapse, and the sudden
visibility of homegrown fascists, weighs on the manuscript, tempering
its optimism. Casual references to the average American's susceptibility
to demagoguery appear with surprising, and discordant, frequency. In the
magazine chapter, the authors observed that ``it is clear that the
typical \emph{American Magazine} reader would go fascist.'' Whether
another magazine's readers would ``go fascist or communist'' remained,
they added, an open question.\footnote{\emph{OMV}, 95, 96.} The chapter's conclusion announces
that the ``democratic dogma is dying if not already dead.'' The poor are
``oriented toward crime, and potentially at least toward revolution,''
while the middle classes are ``oriented toward fascism.''\footnote{\emph{OMV}, 98--99.} The book
elsewhere deploys ``Italy'' and ``Germany'' (and ``Russia'' too) as
shorthand for the possible American future.\footnote{\emph{OMV}, 98--99.}

Even before the publication of \emph{Our Master's Voice}, Rorty had set
out on a seven-month road trip around the United States, writing
magazine dispatches and a chronicle of his trip. Appearing in 1936 as
\emph{Where Life Is Better}, this second volume registered Rorty's
dissipating confidence in the country's workers---their failure to
recognize capitalism's fundamental flaws.\footnote{James Rorty,
  \emph{\href{http://www.worldcat.org/oclc/1131621563}{Where Life Is
  Better: An Unsentimental American Journey}} (New York: Reynal \&
  Hitchcock, 1936). In keeping with \emph{Our Master's Voice}, Rorty
  blamed the media industry: ``Hollywood specializes in the manufacture
  of the soothing, narcotic dreams of love,'' while ``in New York, NBC
  and Columbia {[}CBS{]} specialize in the manufacture of cheerio radio
  optimism, pre-barbaric dance rhythms, and commodity fetishism intoned
  by unctuous announcers'' (107, quoted in Gross, \emph{Richard Rorty,}
  48; see also 47--50).} He fretted about
Americans' likely embrace of fascism instead---a theme foreshadowed in
the haunting conclusion to \emph{Our Master's Voice}, whose last page
recounts a conversation with a ``very eminent advertising man.'' He was,
as Rorty realized with a ``sudden chill,'' praising the new Nazi regime.
``I venture to predict,'' Rorty wrote in the book's closing sentence,
``that when a formidable Fascist movement develops in America, the
ad-man will be right up front; that the American version of Minister of
Propaganda and Enlightenment Goebels {[}\emph{sic}{]} . . . will be both
numerous and powerful.''\footnote{\emph{OMV}, 286.}


\section{A Master's Voice}\label{a-masters-voice}}

Despite its unrepentant leftism and fretting over fascism, \emph{Our
Master's Voice} received good press. Newspaper and magazine reviews were
generally positive, and occasionally rhapsodic. By telling contrast,
academics ignored the book. Not a single review appeared in any social
science journal, and the first citation to the book, in the journal
literature at least, came \emph{fourteen years later}, in a law review
article published in the late 1940s.\footnote{Ralph S. Brown,
  ``\href{https://www.jstor.org/stable/793310}{Advertising and the
  Public Interest: Legal Protection of Trade Symbols},'' \emph{Yale Law
  Journal} 57, no. 7 (1948): 1167. The reference appears in a footnote
  listing advertising's ``detractors'': ``Noteworthy among general
  attacks on the institution was Rorty, \emph{Our Master's Voice}
  (1934).''
} A thorough but non-exhaustive
search of 1930s scholarly books on media uncovered a smattering of
references. Rorty's tome did warrant a listing in a 1935 bibliographic
project by the political scientist Harold Lasswell, \emph{Propaganda and
Promotional Activities}. Yet \emph{Our Master's Voice}, one among
hundreds of references, was annotated with a single line: ``Criticism of
advertising as a handmaiden of American `pseudo-culture.'\thinspace''\footnote{Harold D. Lasswell, Ralph D. Casey, and Bruce Lannes Smith,
  \emph{\href{http://www.worldcat.org/oclc/1013666}{Propaganda and
  Promotional Activities: An Annotated Bibliography}} (Minneapolis:
  University of} Only a
handful of additional mentions occurred in the book literature, some
dismissive and none of them substantial.\textsuperscript{97}

The \marginnote{Minnesota Press, 1935), 136. The book was dropped in
  Lasswell's 1946 sequel, \emph{Propaganda, Communication, and Public
  Opinion}, a reflection, perhaps, of the twin volumes' unblushing
  fixation on successful propaganda. Bruce Lannes Smith, Harold D.
  Lasswell, and Ralph D. Casey,
  \emph{\href{http://www.worldcat.org/oclc/441760134}{Propaganda,
  Communication, and Public Opinion: A Comprehensive Reference Guide}}
  (Princeton, N.J.: Princeton Univeristy Press, 1946).}popular \marginnote{\textsuperscript{97} The sociologist Alfred McClung Lee, for example, included the book in
  a listing of ``recent attacks on advertised products and
  advertising.'' Lee,
  \emph{\href{http://www.worldcat.org/oclc/926758421}{The Daily
  Newspaper in America}} (New York: Macmillan, 1937), 332. Likewise, the
  journalism scholar Robert Desmond footnoted \emph{Our Master's Voice}
  among four other critical books in support of the statement that
  newspaper ``owners take every step to protect their investments and,
  while this is natural, the public often suffers.'' Desmond,
  \emph{\href{http://www.worldcat.org/oclc/1672803}{The Press and World
  Affairs}} (New York: D. Appleton-Century Co., 1937), 374. Hadley
  Cantril and Gordon Allport's
  \emph{\href{http://www.worldcat.org/oclc/895657266}{The Psychology of
  Radio}} (New York: Harper \& Bros., 1935), perhaps the decade's most
  celebrated book-length academic treatment of media, buries Rorty's
  book in a footnote on censorship. They do briefly summarize Rorty's
  radio-oriented pamphlet \emph{Order on the Air} (1934), but only to
  solicit and republish in full a two-page rebuttal from the National
  Association of Broadcasters---``since the impartial observer must
  learn the other side of the story'' (46, 57--59). The final mention
  appears in the sociologist William Albig's
  \emph{\href{http://www.worldcat.org/oclc/848906812}{Public Opinion}}
  (New York: McGraw-Hill, 1939). Citing the journalist Stuart Chase
  alongside \emph{Our Master's Voice}, Albig wrote that a ``number of
  intellectuals, evidencing that they felt the appeals and wiles of the
  advertising man to be a personal insult, have indicated their
  revulsion in no uncertain terms''---before mounting a qualified
  defense of the industry (306--9).}\setcounter{footnote}{97}and literary press proved far more attentive. The \emph{New
York Times} granted the book a full-page review, including a respectful
summary registering Veblen's influence alongside Rorty's indictment of
the media industry at large.\footnote{The review, by the \emph{Times} star reviewer R. L. Duffus, recommends
  the ``suggestive'' book, but faults Rorty for exaggerating the press's
  fealty to advertisers. The review carries the subhead: ``There Is
  Truth in His Picture, but What He Shows Is by No Means the Whole
  Picture.'' Duffus, ``Mr Rorty's Biased View of Modern Advertising,''
  \emph{New York Times}, May 20, 1934, 4, 14.} \emph{North American Review}, a
literary magazine, called \emph{Our Master's Voice} a ``fiery discussion
of the advertising racket''---``superb'' on the ``debunking,'' but
hobbled by the author's revolutionary politics.\textsuperscript{99} \emph{The New
Yorker} described the book as a ``vigorous, athletic, witty, and in
parts profound analysis of and attack upon the advertising game in its
broadest aspect. . . . Highly recommended.''\textsuperscript{100} Syndicated treatments
in the country's newspapers were at least grudgingly favorable. Rorty
``takes advertising for a good humored but rather rough ride,'' read
one. Another noted Rorty's ``Socialist tendencies,'' but admitted that
the ``author has worked hard with his material'' and praised the book's
``mass of facts.'' A third syndicated review, after a taut summary,
concluded:

\begin{quote}
If all that sounds like quite a mouthful, you will find it worth your
while to read Mr.~Rorty's book . . . all in all, this is a serious and
instructive book. Some advertising men will denounce it; others, I
suspect, will welcome it. And the general reader will find it
exceedingly informative.\textsuperscript{101}
\end{quote}

\noindent The popular reviews---some of them flattering, none dismissive---proffer
ironic testimony, perhaps, to the limits of Rorty's monovocal theory of
the press. Regardless, they stand in striking counterpose to the silence
from academics, then and since.

In retrospect, it's not hard to explain media scholars' neglect of
\emph{Our Master's Voice}---and its subsequent disappearance from the
field's collective memory. After all, they ignored Rorty's fusillade in
its own time. The book's peculiar form weighed it down from the
beginning---its manic eclecticism and rhetorical overspillage, page by
relentless page. There was, too, its author's radicalism, out of step
(unblushingly so) with the performance of detachment demanded by the
reigning academic norms.\textsuperscript{102} Reformist commitments, when tempered by
professions of value freedom, were permissible---but not the
Marx-quoting pyrotechnics of the book's prose. Rorty's status as a
journalist created its own reception liability, made worse by the
itinerant, topically promiscuous, fiction-tainted character of his other
work. Since the 1920s, American social scientists had been avidly
professionalizing, and the campaign had only gathered momentum. So mere
journalism, or, worse still, social criticism, was primed for spurning
by scholars who had only just won a fragile legitimacy.

And of course we shouldn't neglect the book's venom-tipped attack on
social science itself. In the spirit of Veblen's 1918 polemic \emph{The
Higher Learning}, Rorty castigated social scientists for abdicating
their assigned role as free-thinking analysts.\textsuperscript{103} He lit into the
``dozens of Greek-porticoed'' \marginnote{\textsuperscript{99} ``{[}I{]}f communism came,'' continued the reviewer, ``we should have
  all our advertising and publicity agencies turning out propaganda for
  the Reds, and it wouldn't be a bit more fundamentally honest than the
  tripe for which they are responsible today.'' Herschel Brickell,
  ``\href{https://www.jstor.org/stable/25114480}{The Literary
  Landscape},'' \emph{The North American Review} 238, no. 1 (1934):
  89--90.}business \marginnote{\textsuperscript{100} The quote is from the capsule in the regular ``Reader's Reminder
  List,'' \emph{New Yorker}, June 2, 1934, 92. The original review is
  equally fawning: ``The neatest, the most amusing, and at the same time
  the most thoughtful piece of fundamental muckraking of the last season
  or so is to be found in James Rorty's new book.'' Cliffton Fadiman,
  ``Books: Three Reports on the State of the Nation,'' \emph{New
  Yorker}, May 19, 1934, 102.} schools, \marginnote{\textsuperscript{101} John Shelby, ``Scanning New Books,'' \emph{Sarasota Herald}, May 14,
  1934, 8; Allen Smith, ``Bound to Be Read,'' \emph{Piqua Daily Call},
  May 16, 1934, 4; and Bruce Catton, ``A Book a Day,'' \emph{Sandusky
  Star Journal}, May 28, 1934, 6.}staffed \marginnote{\textsuperscript{102} Rorty soon had an inauspicious brush with the team of researchers who,
  in the subsequent decade, would help establish communication research
  as an interdisciplinary field. It is a remarkable fact that in 1937
  Rorty was recruited to work for Paul Lazarsfeld's Princeton Radio
  Research Project, the Rockefeller Foundation--funded institute that
  would become, in the 1940s, the Bureau of Applied Social Research at
  Columbia. Rorty was commissioned to conduct a study for the Project on
  ``Radio Commentators,'' including listener reactions. Lazarsfeld and
  one of his associate directors, the psychologist Hadley Cantril,
  soured on Rorty for a variety of reasons, including work style and the
  draft itself. Cantril was assigned to resurrect the manuscript, but
  the ``Radio Commentators'' monograph was never published and Rorty
  was, in effect, jettisoned from the Project. The secondary accounts of
  Rorty's stint do not cite politics as the main site of conflict,
  though Lazarsfeld---in a memo mounting a qualified defense of the
  Frankfurt School refugee Theodor Adorno's quixotic (and now-notorious)
  contribution to the Project---used Rorty as a point of contrast: ``It
  is true that I still have some difficulty in getting W {[}Adorno{]}
  down to earth but there can be no doubt of his originality and the
  fruitfulness of his approach. With R}by a ``new
priesthood of `business economists'\,'' who translated the ``techniques
of mass prevarication into suitable academic euphemisms.''\textsuperscript{104} The
whole discipline of economics, meanwhile, ``stood aside'' while
advertising proceeded to ``play jackstraws'' with ``orthodox economic
doctrine.''\textsuperscript{105} Rorty in fact devoted an entire chapter to psychology's
prostitution to advertising, citing the for-profit Psychological
Corporation and behaviorist John B. Watson's move to a big-time ad
agency.\textsuperscript{106} The ``prestige of business dominates the American
psychology,'' Rorty wrote, ``not excepting the psychology of American
psychologists.''\textsuperscript{107} And all the disciplines come in for repeated
reprimand for claiming objectivity while propping up the status
quo.\textsuperscript{108}

Given the upbraiding, social scientists had plenty of reason to look
away.\textsuperscript{109} The result, though, was the premature burial of a trenchant
volume. In re-publication, \emph{Our Master's Voice} joins a
well-established literature on consumer culture, some of it
critical---though nothing as vigorous, athletic, and witty as Rorty's
forgotten study. A book about advertising, he reminded us, is inevitably
a critique of the surrounding society. His example is worth
emulating.

\newpage

\marginnote{{[}Rorty{]}, I do not even know
  whether he has produced a new aspect although Had {[}Cantril{]} might
  correct me on this point.'' Christian Fleck,
  \emph{\href{http://www.worldcat.org/oclc/751453444}{A Transatlantic
  History of the Social Sciences: Robber Barons, the Third Reich, and
  the Invention of Empirical Social Research}} (New York: Bloomsbury
  Publishing, 2011), 183--84. The whole episode deserves more study,
  especially since Rorty could be the third major leftist figure---after
  the well-documented cases of Adorno and the sociologist C. Wright
  Mills in the mid-1940s---sidelined by the Princeton Radio Research
  Project/Bureau of Applied Social Research, with all its many radio
  industry links in this period.}

\marginnote{\vspace{1mm}}

\marginnote{\textsuperscript{103} Thorstein Veblen,
  \emph{\href{http://www.worldcat.org/oclc/249848091}{The Higher
  Learning in America: A Memorandum on the Conduct of Universities by
  Business Men}} (New York: B. W. Huebsch, 1918). Advertising, Rorty
  wrote in the opening chapter, is the ``business nobody knows.'' He
  continued: ``As evidence of this general ignorance, one has only to
  cite a few of the misapprehensions which have confused the very few
  contemporary economists, sociologists and publicists who have
  attempted to treat the subject'' (\emph{OMV}, 11).}

\marginnote{\textsuperscript{104} \emph{OMV}, 233.}

\marginnote{\textsuperscript{105} \emph{OMV}, 173--74. Some of Rorty's special disdain for orthodox
  economics can be explained by the Veblen counterexample. There is,
  too, a familial touchstone: Rorty's brother Malcolm, fifteen years his
  senior, was a prominent economist, AT\&T executive, and defender of
  laissez-faire. See Gross, \emph{Richard Rorty}, 44.}

\marginnote{\textsuperscript{106} \emph{OMV}, chap. 15.}

\marginnote{\textsuperscript{107} \emph{OMV}, 179.}

\marginnote{\textsuperscript{108} The objectivity critique is an extension of Rorty's broader assault on
  the professed neutrality of education: ``Many teachers, even of the
  social sciences, are quite unconscious of these {[}economic{]}
  determinants and preserve the confident illusion of `scientific
  objectivity' in the very act of asserting creedal absolutes which are
  obviously a product of social and economic class conditioning''
  (\emph{OMV}, 118). See also 174--75, including a long quote from an
  unpublished Sidney Hook manuscript attacking objectivity.}

\marginnote{\textsuperscript{109} There was, importantly, no such thing as communication research in the
  United States when Rorty published \emph{Our Master's Voice}---no
  organized discipline, not even a label. Yes, the} 

\newpage  
  
\marginnote{\emph{study} of
  communication was already well underway. Indeed, American social
  science, since its emergence in the late nineteenth century, had fixed
  the organs of mass communication as objects of study---as tokens of
  modern social upheaval. In the wake of the Great War---fought, in
  part, through publicity---scholars and journalists alike took up the
  question of propaganda and its implications for politics and social
  life. By the early 1930s a large social-scientific literature had
  formed, with important studies published in and around Rorty's own
  1934 contribution. Still, there was no recognized category called
  ``communication researcher.'' There were, instead, political
  scientists, psychologists, sociologists, a handful of economists and
  others drawn from what were unevenly differentiated disciplines. Only
  in the subsequent decade---roughly by the end of World War II---did
  ``communication'' cohere as an academic formation. Thus the field
  arrived late, and in a moment of generational turnover. The fight
  against the Axis powers, and then the new Soviet enemy, yoked the
  proto-discipline's intellectual agenda to questions of successful
  persuasion---that is, how to get it working well. By the 1950s and
  '60s, communication scholars had finally established a home in
  journalism schools. In their struggle for legitimacy, they drafted
  histories of media study that, in the cocksure spirit of the times,
  cast \emph{all} pre-war scholarship as naive and impressionistic. One
  result was amnesia on a field-wide scale: Almost nothing was read or
  remembered from the 1920s or '30s. Even today students of
  communication learn that there's nothing worth reading before 1945. So
  \emph{Our Master's Voice} was destined to be forgotten, even had its
  social science contemporaries paid it real heed. The book was ignored
  twice over, in other words---once in the 1930s and then ever since.}





% REPEAT TITLE PAGE
\newpage
\begin{fullwidth}


\thispagestyle{empty}

\begingroup
\setlength{\parindent}{10pt}

\vspace*{2.25in}{\fontsize{27}{54}{\allcaps{Our Master's Voice}}\par}

\vspace*{0.5in}{\fontsize{22}{24}{\allcaps{\textit{Advertising}}}\par}

\endgroup

% MAINMATTER
\mainmatter\pagenumbering{arabic}\setcounter{page}{1}

% EPIGRAPHS
\newpage
\thispagestyle{empty}

\begingroup

\setlength{\parindent}{0cm}\setlength{\parskip}{2ex}

\begin{adjustwidth}{125pt}{25pt}

\begin{flushright}
\par\vspace*{.75in} ''I come to bury Caesar, not to praise him.''\\
---WILLIAM SHAKESPEARE 

\par ''A trading on that range of human infirmities 
that blossoms in devout observances and bears 
fruit in the psychopathic wards.''\\
---THORSTEIN VEBLEN 

\par ''Business succeeds rather better than the state 
in imposing its restraints upon individuals, because its imperatives are disguised as choices.''\\
---WALTON HAMILTON

\end{flushright}

\end{adjustwidth}

\endgroup

\end{fullwidth}

\newpage
\thispagestyle{plain} % empty
\mbox{}

% PREFACE
\chapter{PREFACE: I Was an Ad-man Once}
\label{ch:preface}
\chaptermark{PREFACE}


\newthought{Imagine}, if you can, the New York of 1913. In that year a young man just
out of college was laying siege to the city desks of the metropolitan
papers. He had good legs, but his past record included nothing more
substantial than having been fired out of college, and having worked
before college, and during vacations, on a small-city paper upstate;
also on a Munsey-owned Boston paper. It was the last count that did for
him. He couldn't laugh that off anywhere, and funds were getting low.

Finally, a relative got the young man a job as a copy writer in an
advertising agency, housed near the Battery in an ancient loft building
which has since been torn down. Perhaps it is time to drop the third
person. The young man was myself. I remember him well, although at this
distance both the person and his actions seem a little unreal.

The young man didn't know anybody, or anything much. At that time he
hadn't even read H. G. Wells' \emph{Tono-Bungay}. But he was full of
fervor. His father was an Irish Fenian who believed to the end of his
days that the world was just on the point of becoming decent and
sensible, and the young man, to tell the truth, has had trouble in
overcoming that paternal misapprehension.

In those days business had pretty well beaten the muckraking magazines
by the painless process of seizing them through the business office. But
the old \emph{Masses} was going full blast, and the blond beasts of the
\emph{New Republic} were about to launch their forays upon the
sheepfolds of the Faithful.

The young man was a Socialist already, in sympathy at least, although in
the matter of fundamental economics and sociology he was as illiterate
as most of his contemporaries. He was literary; that is to say, he knew
Ibsen, and Hauptman [\emph{sic}], and Shaw, and Jack London, and Samuel Butler---even
a little Nietzsche. Not until some years later did he come to know Karl
Marx and Thorstein Veblen.

But life was real and landladies were earnest. The young man was hungry.
He had a job now and he was taking no chances. He was assured that at
the end of the month he would be paid sixty dollars for his services, in
negotiable currency. It was up to him to earn that sixty dollars. He was
young and energetic. During the economy wave under which Mr. Munsey
extinguished the \emph{Boston Journal}, he, a cub reporter, had covered
as many as three supposedly important assignments in one day, being
obliged, of course, to steal or fake most of his facts.

The young man was given his first advertising-copy assignment: to write
some forty advertisements commending a certain brand of agricultural
machinery about which he knew nothing whatever. The young man took off
his coat.

I wrote those forty advertisements in three days, with my eye on the
clock. Three days is ten per cent of thirty days. Ten per cent of sixty
dollars is six dollars. Were those forty advertisements a big enough
stint to earn those six dollars? Trembling, I turned in my copy \ldots{}
it was enough for a year.

The copy was fully up to current standards, too, as advertising copy,
although of course it went through endless meaningless revisions. As
news and information it didn't, at the time, seem to me to be worth the
price. I still don`t think so. But in those three days I learned all
that any bright young man needed to know about the mysteries of
advertising copy-writing in order to earn, in 1929, not sixty dollars a
month, but a hundred and sixty dollars a week. I say this in the teeth
of the Harvard School of Business Administration, the apprentice courses
of all the agencies, Dr. John B. Watson, and the old sea lion in the
Aquarium to whom, in my dazed and shaken condition, I turned for comfort
and understanding.

The Aquarium was close at hand. During the noon hour I would sit on a
bench in Battery Park, eating my necessarily frugal lunch of peanuts and
chocolate, and then spend the remaining half-hour wandering among the
glass cases and peering at the fishes, who peered back at me with their
flat eyes and said nothing. Sometimes one of them would turn on his
side, his gills waving faintly. Nothing to do, nowhere to go. We cried
our eyes out over each other, I and the other poor fishes.

Then I discovered the sea lion, who occupied a big pool in the center of
the main floor. The sea lion, I soon became convinced, had some kind of
an idea. There was a slanting float at one end of the pool. He would
start at the other end, dive, emerge halfway up the float with a
tremendous rush, and whoosh! he would blow water on the mob of children
and adults who crowded around the tank. Always they would shriek,
giggle, and retreat. Then, gradually, they would come back; the sea lion
would then repeat the performance with precisely the same effect.

It has taken me years to understand that sea lion. I know now that he
was an advertising man. Recently, I became acquainted with his human
reincarnation, one of the ablest, most philosophical, and best paid
advertising men in New York. If there is a ``science'' of advertising,
he has mastered it. Yet his formula is very simple. It is this: ``Figure
out what they want, promise 'em everything, and blow hard.''

This philosophical ancient is greatly valued as an instructor of the
young. His students are very promising, although some of them are not
wholly literate. He is, however, indulgent of their cultural
limitations, remarking kindly: ``What are a few split infinitives
between morons?''

In the annex to the Aquarium where I served my advertising
apprenticeship there were many mansions, housing as varied a collection
of the human species as I have ever encountered together in one place.
Through a stroke of luck, the agency had started with a nucleus of
important accounts and expanded rapidly. Its owner, a quiet Swede who
never, to my knowledge while I was in his employ, wrote a single piece
of advertising copy himself, became a millionaire in a few years. He
was, then, an economist, a commercial engineer, an executive of
tremendous driving power? Not so that anybody could notice it. His
success is quite unexplainable in terms of logic or common sense. I
think he was just a ``natural.'' Also, he played golf well, but not too
well. Puzzling over this phenomenon, I remembered hearing the Socialists
tell me there is no sense in trying to make sense out of the people and
institutions of our chaotic capitalist civilization.

Nevertheless, the boss was a natural. Either by shrewdness or by
accident, he gathered into his organization a considerable number of
able and interesting people. They didn't know much about advertising.
Nobody did in those days. Six months after my initiation, the company
moved to a neighboring skyscraper, and the expanded copy staff soon
numbered eight people. We all sat in one large room. By right of
priority, I had a desk next the window where I could look out and watch
the ferry boats swimming about like water beetles, and the tugs pushing
liners out to sea, as ants push big crumbs. They seemed so earnest, so
determined\ldots. Every now and then an office boy would stroll by and
deposit in one of the desk baskets a yellow printed form with here and
there a little typing on it. The form called for one, two, six or twelve
advertisements about a certain product, to fit specified spaces in
certain scheduled publications. Usually the form was destitute of other
information or instruction.

I think, although I am not sure, that those forms were the bequest of an
efficiency expert who functioned briefly during the early months of my
employment. He was a tall, gangling man, with a high white brow, a
drooping forelock and a rapt and questing eye. He dictated inspirational
talks to his stenographer. While so engaged, he would pace up and down
his office and quite literally beat his breast. In fact, he had all the
equipment of a medicine man except the buffalo horns and the rattlesnake
belt. It was he, I think, who started the idea of timing and
systematizing the copy production of the office. Years after he had
left, unfortunate copy writers were still digging the splinters of that
system out of their pants.

You got a yellow form, then, which required that you write so many
pieces of copy and turn them in by a certain date. What kind of copy?
The form was silent. The headline goes at the top, the slug at the
bottom and what goes in between you rewrite from a booklet or make up
out of your head. Sometimes an illustration was called for. In such
cases you conferred with the art director, who was of the opinion that
you, your words, and especially your ideas about pictures were a damned
nuisance and so informed you.

I felt it necessary to resent such acerbities, but I could never do so
with any great conviction. Privately, I suspected that he was right.
Sometimes I was tempted to put my hands on my hips and retort stoutly,
``You're another.'' But I never did so. That would have been to widen
the field of discussion intolerably. And there were always closing dates
to meet.

Feeling as I did about it, it frequently seemed to me that one
advertisement would do exactly as well as six. But I always wrote six.
Anything to keep busy. There were never enough yellow forms.

Sometimes, unable to control my restlessness, I would wander upstairs,
knock on the door of the account executive's office, and ask mildly if
anybody knew anything about that product and what it was supposed to be
used for. I knew that many heavy conferences had preceded the planning
of that campaign. But the decisions reached in those conferences never
seemed to get typed on that yellow form. Usually I got nothing out of
such interviews except the suggestion that I do some more like last
year's, or that an ad was an ad, wasn't it, and I was to have six done
by Friday. Such admonitions were heartbreaking. The ads were already
done. Nothing to do now except to stew miserably in the juice of my
frustrated energies.

In time, merciful nature came to my aid. I, who was normally facile, as
even a cub reporter has to be, found that writing even a six-line
tradepaper advertisement cost me intolerable effort. My brain wouldn't
function. My fingers were paralyzed. I was fighting the cold wind of
absurdity blowing off the waste lands of our American commercial chaos.
The workman in me had been insulted. Very well, then, he would strike. I
dawdled. I covered reams of paper with idiotic pencilings. I missed
closing dates and didn't care. My fellow copy writers, suffering the
same tortures, would go out and get drunk. One of them, in fact, who had
genuine literary talent, ultimately drank himself to death.

Since I was still a virtuous youth, I had no such escapes. Even my
health, which had been excellent, was shaken. I began mumbling to myself
on the street. Once, for three weeks, an office associate converted me
to Christian Science.

The Truth and the Light, he said, were in Mrs. Eddy's \emph{Science and
Health}, which I accordingly undertook to read for several evenings. I
do not think I ever got beyond page 38, although I tried very hard. The
difficulty was that it didn't make sense at first reading, so that on
resuming the book I was always obliged to start over again from the
beginning. It was like driving a model T Ford uphill through sand. At
the end of three weeks I was utterly exhausted, and sleeping soundly,
but unable to bear another word of Mary Baker Eddy.

I cite the episode merely to indicate how acute was my condition. If my
friend had been a Holy Roller, I think I would have rolled for him
cheerfully.

The workman in me was paralyzed. Even when, outside the office, I tried
to write poetry and plays the words and ideas stared coldly at me from
the page.

But the reformer in me still lived and was shortly to have his inning.
The house acquired as a client a company manufacturing a proprietary
remedy. As it happened, it was an excellent product, which, minus its
proprietary name, was much used and recommended by the medical
profession. There was my chance. I would make the advertising of that
product honest. I did make it honest, for a while. I had every word of
my copy censored by representative medical men. I fought everybody in
the office, singly and in groups. I was obsessed, invincible and absurd.

But the client became impatient---sales weren't growing as fast as he
thought they should. He hired as advertising manager an experienced and
entirely unscrupulous patent-medicine salesman---a leather-hided saurian
who scrapped all my carefully censored copy and furnished as a model for
future advertising an illiterate screed recommending the product,
directly or by implication, as a cure for everything from tuberculosis
to athlete's foot.

I threw him out of my office. I rushed over to the client and talked
very crudely to a very eminent gentleman. Even that wasn't enough. I
considered blowing the works to the organized medical profession,
although I never actually did so. Instead, I wrote a furious and
entirely unactable play about a patent medicine wage-slave who went
straight and took a correspondence course in burglary.

I wasn`t fired, although logically I should have been. The President of
the United States had just declared war, and in the confusion I escaped
into the army as a buck private. Even the war, I thought, was more
rational than the advertising business. I was wrong, but that is another
story.

I was an ad-man once. Indeed, I am, in a small way, an ad-man still,
although I no longer carry a spear in the monotonously hilarious
spectacles which the orthodox priests continue sweatingly to produce in
the Byzantine, Chino-Spanish and Dada-Gothic temples of advertising
which crowd the Grand Central district of New York.

I still practice, however, after my fashion. My motto, ``The Less
Advertising the Better,'' appeals poignantly to certain eminent
industrialists to whom I have talked. My sales argument goes something
like this:

``Mr. Hoffschnagel, you and I are practical men. I don't need to tell
you that advertising is not an end in itself. Neither is selling. The
end, Mr. Hoffschnagel, the true objective of the manufacturer and
dispenser of products and services, should be the efficient and
economical delivery to the consumer of precisely what the consumer wants
and needs: what the consumer needs to buy, I repeat, not what the
manufacturer needs to sell him. In any functional relationship between
producer and consumer, advertising and sales expenditures are just so
much frictional loss; in the ideal setup, which of course we can`t even
approximate under present conditions, released buying energy would be
substituted entirely for the selling energy which you now spend in
breaking down `sales resistance.' My task, therefore, is to redefine and
reinterpret your relationship with your customers; not to pile up sales
and advertising expenses''---Mr. Hoffschnagel nods energetically---``but
to cut them. What do your customers want from you? Service! What do you
want to give them? Service! Not advertising---the less advertising the
better---that`s just so much friction and loss. But service! The end, Mr.
Hoffschnagel, the end is service!''

Mr. Hoffschnagel meditates, while as if unconsciously his hand strays to
the right-hand drawer of his desk.

``Have a drink,'' says Mr. Hoffschnagel.

It is possible to get a good deal of hospitality in this way, and even
some business. Sometimes, as I listen to myself talk, I sound like one
of these newly spawned capitalist economic planners. I am not. I know,
or think I know, that the advertising business, with all of its wastes
and chicaneries intact, is woven into the very fabric of our competitive
economic system; that the only equilibrium possible for such a system is
the unstable equilibrium of accelerating change, with the ad-man's foot
on the throttle, speeding up consumption, preaching emulative
expenditure, ``styling'' clothes, kitchens, automobiles---everything, in
the interest of more rapid obsolescence and replacement. Up to a certain
point it is possible to build, and after the inevitable crash, to
rebuild such a system always with a progressive and cumulative
intensification of wastes and conflicts. It is not possible to operate
such a system sanely and permanently, because its underlying economic
and social premises are obsolete in the modern world.

If this is so---even some advertising men apprehend that it may be
so---then it would be, perhaps, not a bad idea, if ad-men removed their
tongues from their long-swollen cheeks and tried talking approximate
sense for a change. It wouldn`t do much if any immediate good, of
course, but it might provide a desirable mental discipline, a kind of
intellectual preparation for the severer disciplines which the future
may hold in store for the profession.

As a matter of fact, the abler people in advertising are becoming
increasingly mature, realistic, and cynical. They don't believe in the
racket themselves. But they insist that the guinea pigs, not merely the
consumers outside the office, but the minor employees inside the office,
\emph{must} believe in it. The r\^ole of the advertising agency guinea
pig---the minor copy writer, layout man, forwarding clerk or other
carrier of messages to Garcia---is hard indeed. The outside guinea pig,
the consumer, can't be fired. But the inside guinea pig can be and is
fired unless he is utterly and sincerely credulous and faithful. A good,
loyal guinea pig is a pearl without price in any agency. I am even told
that in some of the larger agencies, eugenic experiments are being
conducted with the idea ultimately of breeding advertising guinea pigs,
or pearls---I admit the metaphor is hopelessly mixed---who will come into
the world crying ``It Pays To Advertise''.



\newthought{To such heights} of fantasy are we lifted by an attempt to examine the
phenomenon of contemporary advertising in America. It is not, as
contemporary liberal historians and social critics have tended to regard
it, a superficial phenomenon: a carbuncular excrescence of our
acquisitive society, curable by appropriate reformist treatment, or
perhaps by a minor operation.

A book about advertising therefore becomes inevitably a critique of the
society.

Much of the data presented in this book I have gathered in my personal
experience as an employee of various advertising agencies. If some of
this material seems absurd, even incredible to the lay reader, I can
only reply, helplessly, that I did not make the advertising business;
nobody made it; that is why it is so absurd. Whether one regards the
advertising business as farce or as tragedy, one is convinced that the
play is badly made; there are no heroes and the villains have a way of
turning into victims under one's eyes; none of them is consistently bad,
consistently sad or even consistently funny.

As I shall try to show in a later section entitled ``The Natural History
of Advertising,'' the advertising business just grew. It is the economic
and cultural causes, the economic and cultural consequences of this
growth that I shall try to describe in this book.



% CHAPTER ONE
\chapter[1 \hspace*{1mm} THE BUSINESS NOBODY KNOWS]{1 THE BUSINESS NOBODY KNOWS}
\label{ch:business}

\newthought{The} title of this chapter was chosen, not so much to parody the title of
Mr. Bruce Barton's widely-read volume of New Testament exegesis,\footnote{{[}Bruce Barton,
  \emph{\href{http://www.worldcat.org/oclc/70421692}{The Man Nobody
  Knows}} (New York: Grosset \& Dunlap, 1924).{]}} as
to suggest that, in the lack of serious critical study, we really know
very little about advertising: how the phenomenon happened to achieve
its uniquely huge and grotesque dimensions in America; how it has
affected our individual and social psychology as a people; what its r\^ole
is likely to be in the present rapidly changing pattern of social and
economic forces.

The advertising business is quite literally the business nobody knows;
nobody, including, or perhaps more especially, advertising men. As
evidence of this general ignorance, one has only to cite a few of the
misapprehensions which have confused the very few contemporary
economists, sociologists and publicists who have attempted to treat the
subject.

Perhaps the chief of these misapprehensions is that of regarding
advertising as merely the business of preparing and placing
advertisements in the various advertising media: the daily and
periodical press, the mails, the radio, motion picture, car cards,
posters, etc. The error here is that of mistaking a function of the
thing for the thing itself. It would be much more accurate to say that
our daily and periodical press, plus the radio and other lesser media,
\emph{are} the advertising business. The commercial press is supported
primarily by advertising---roughly the ratio as between advertising
income and subscription and news-stand sales income averages about two
to one. It is quite natural, therefore, that the publishers of
newspapers and magazines should regard their enterprises as
\emph{advertising businesses}. As a matter of fact, every advertising
man knows that they do so regard them and so conduct them. These
publishers are business men, responsible to their stockholders, and
their proper and necessary concern is to make a maximum of profit out of
these business properties. They do this by using our major instruments
of social communication, whose free and disinterested functioning is
embodied in the concept of a democracy, to serve the profit interests of
the advertisers who employ and pay them. Within certain limits they give
their readers and listeners the sort of editorial content which
experience proves to be effective in building circulations and
audiences, these to be sold in turn at so much a head to advertisers.
The limits are that regardless of the readers' or listeners' true
interests, nothing can be given them which seriously conflicts with the
profit-interests of the advertisers, or of the vested industrial and
financial powers back of these; also nothing can be given them which
seriously conflicts with the use and wont, embodied in law and custom,
of the competitive capitalist economy and culture.

In defining the advertising business it must be remembered also that
newspapers and magazines use paper and ink: a huge bulk of materials, a
ramified complex of services by printers, lithographers, photographers,
etc. Radio uses other categories of materials and services---the whole
art of radio was originally conceived of as a sales device to market
radio transmitters and receiving sets. All these services are necessary
to advertising and advertising is necessary to them. These are also the
advertising business. Surely it is only by examining this business as a
whole that we can expect to understand anything about it.

The second misapprehension is that invidious moral value judgments are
useful in appraising the phenomena. Advertising is merely an instrument
of sales promotion. Good advertising is efficient
advertising---advertising which promotes a maximum of sales for a
minimum of expenditure. Bad advertising is inefficient advertising,
advertising which accomplishes its purpose wastefully or not at all. All
advertising is obviously special pleading. Why should it be considered
pertinent or useful to express surprise and indignation because special
pleading, whether in a court of law, or in the public prints, is
habitually disingenuous, and frequently unscrupulous and deceptive? Yet
liberal social critics, economists and sociologists, have wasted much
time complaining that advertising has ``elevated mendacity to the status
of a profession.'' The pressure of competition forces advertisers and
the advertising agencies who serve them to become more efficient; to
advertise more efficiently frequently means to advertise more
mendaciously. Do these liberal critics want advertising to be less
efficient? Do they want advertisers to observe standards of ethics,
morals and taste which would, under our existing institutional setup,
result either in depriving stockholders of dividends, or in loading
still heavier costs on the consumer?

There is, of course, a third alternative, which is neither good
advertising nor bad advertising, but no advertising. But that is outside
the present institutional setup. It should be obvious that in the
present (surplus economy) phase of American capitalism, advertising is
an industry no less essential than steel, coal, or electric power. If
one defines advertising as the total apparatus of American publishing
and broadcasting, it is in fact among the twelve greatest industries in
the country. It is, moreover, one of the most strategically placed
industries. Realization of this fact should restrain us from loose talk
about ``deflating the advertising business.'' How would one go about
organizing ``public opinion'' for such an enterprise when the
instruments of social communication by which public opinion must be
shaped and organized are themselves the advertising business?

As should be apparent from the foregoing, the writer has only a
qualified interest in ``reforming'' advertising. Obviously it cannot be
reformed without transforming the whole institutional context of our
civilization. The bias of the writer is frankly in favor of such a
transformation. But the immediate task in this book is one of
description and analysis. Although advertising is forever in the
public's eye---and in its ear too, now that we have radio---the average
layman confines himself either to applauding the tricks of the ad-man,
or to railing at what he considers to be more or less of a public
nuisance. In neither case does he bother to understand what is being
done to him, who is doing it, and why.

The typical view of an advertisement is that it is a selling
presentation of a product or service, to be judged as ``good'' or
``bad'' depending upon whether the presentation is accurate or
inaccurate, fair or deceptive. But to an advertising man, this seems a
very shallow view of the matter.

Advertising has to do with the shaping of the economic, social, moral
and ethical patterns of the community into serviceable conformity with
the profit-making interests of advertisers and of the advertising
business. Advertising thus becomes a body of doctrine. Veblen defined
advertisements as ``doctrinal memoranda,'' and the phrase is none the
less precise because of its content of irony. It is particularly
applicable to that steadily increasing proportion of advertising
classified as ``inter-industrial advertising'': that is to say,
advertising competition between industries for the consumer's dollar.
What such advertising boils down to is special pleading, directed at the
consumer by vested property interests, concerning the material, moral
and spiritual content of the Good Life. In this special pleading the
editorial contents of the daily and periodical press, and the sustaining
programs of the broadcasters, are called upon to do their bit, no less
manfully, though less directly than the advertising columns or the
sponsor's sales talk. Such advertising, as Veblen pointed out, is a
lineal descendant of the ``Propaganda of the Faith.'' It is a less
unified effort, and less efficient because of the conflicting pressure
groups involved; also because of the disruptive stresses of the
underlying economic forces of our time. Yet it is very similar in
purpose and method.\footnote{{[}Thorstein Veblen,
  \emph{\href{http://www.worldcat.org/oclc/752183}{Absentee Ownership
  and Business Enterprise in Recent Times: The Case of America}} (New
  York: B. W. Huebsch, 1923), 300, 319--25.{]}}

An important point which the writer develops in detail in later chapters
is that advertising is an effect resulting from the unfolding of the
economic processes of modern capitalism, but becomes in turn a cause of
sequential economic and social phenomena. The earlier causal chain is of
course apparent. Mass production necessitated mass distribution which
necessitated mass literacy, mass communication and mass advertising. But
the achieved result, mass advertising, becomes in turn a generating
cause of another sequence. Mass advertising perverts the integrity of
the editor-reader relationship essential to the concept of a democracy.
Advertising doctrine---always remembering that the separation of the
editorial and advertising contents of a modern publication is for the
most part formal rather than actual---is a doctrine of material
emulation, keeping up with the Joneses, conspicuous waste. Mass
advertising plus, of course, the government mail subsidy, makes possible
the five-cent price for national weeklies, the ten- to thirty-five-cent
price for national monthlies. Because of this low price and because of
the large appropriations for circulation-promotion made possible by
advertising income, the number of mass publications and the volume of
their circulation has hugely increased. These huge circulations are
maintained by editorial policies dictated by the requirements of the
advertisers. Such policies vary widely but have certain elements in
common. Articles, fiction, verse, etc., are conceived of as
``entertainment.'' This means that controversial subjects are avoided.
The contemporary social fact is not adequately reported, interpreted, or
criticized; in fact the run of commercial magazines and newspapers are
extraordinarily empty of social content. On the positive side, their
content, whether fiction, articles or criticism, is definitely shaped
toward the promotion and fixation of mental and emotional patterns which
predispose the reader to an acceptance of the advertiser's doctrinal
message.

This secondary causal chain therefore runs as follows: Mass advertising
entails the perversion of the editor-reader relationship; it entails
reader-exploitation, cultural malnutrition and stultification.

This situation came to fruition during the period just before, during
and after the war; a period of rapid technical, economic and social
change culminating in the depression of 1929. At precisely the moment in
our history when we needed a maximum of open-minded mobility in public
opinion, we found a maximum of inertia embodied in our instruments of
social communication. Since these have become advertising businesses,
and competition is the life of advertising, they have a vested interest
in maintaining and promoting the competitive acquisitive economy and the
competitive acquisitive social psychology. Both are essential to
advertising, but both are becoming obsolete in the modern world. In
contemporary sociological writing we find only vague and passing
reference to this crucial fact, which is of incalculable influence in
determining the present and future movement of social forces in America.

In later chapters the writer will be found dealing coincidentally with
advertising, propaganda and education. Contemporary liberal criticism
tends to regard these as separate categories, to be separately studied
and evaluated. But in the realm of contemporary fact, no such separation
exists. All three are \emph{instruments of rule}. Our ruling class,
representing the vested interests of business and finance, has primary
access to and control over all these instruments. One supplements the
other and they are frequently used coordinately. Liberal sociologists
would attempt to set up the concept of education, defined as a
disinterested objective effort to release capacity, as a contrasting
opposite to propaganda and advertising. In practice no such clear
apposition obtains, or can obtain, as is in fact acknowledged by some of
our most distinguished contemporary educators.

There is nothing unique, isolate or adventitious about the contemporary
phenomena of advertising. Your ad-man is merely the particular kind of
eccentric cog which the machinery of a competitive acquisitive society
required at a particular moment of its evolution. He is, on the average,
much more intelligent than the average business man, much more
sophisticated, even much more socially minded. But in moving day after
day the little cams and gears that he has to move, he inevitably empties
himself of human qualities. His daily traffic in half-truths and
outright deceptions is subtly and cumulatively degrading. No man can
give his days to barbarous frivolity and live. And ad-men don't live.
They become dull, resigned, hopeless. Or they become d{\ae}monic fantasts
and sadists. They are, in a sense, the intellectuals, the male het{\ae}r{\ae} of our American commercial culture. Merciful nature makes some of them into
hale, pink-fleshed, speech-making morons. Others become gray-faced
cynics and are burned out at forty. Some ``unlearn hope'' and jump out
of high windows. Others become extreme political and social radicals,
either secretly while they are in the business, or openly, after they
have left it.

This, then, is the advertising business. The present volume is merely a
reconnaissance study. In addition to what is indicated by the foregoing,
some technical material is included on the organization and practices of
the various branches of the business. Some attempt is made to answer the
questions: how did it happen that America offered a uniquely favorable
culture-bed for the development of the phenomena described? What are the
foreign equivalents of our American rule-by-advertising? How will
advertising be affected by the present trend toward state capitalism,
organized in the corporative forms of fascism, and how will the social
inertias nourished and defended by advertising condition that trend?

The writer also attempts tentative measurements of the mental levels of
various sections of the American population, using the criteria provided
by our mass and class publications. Advertising men are obliged to make
such measurements as a part of their business; they are frequently
wrong, but since their conclusions are the basis of more or less
successful business practice they are worthy of consideration.

The one conclusion which the writer offers in all seriousness is that
the advertising business is in fact the Business Nobody Knows. The
trails marked out in this volume are brief and crude. It is hoped that
some of our contemporary sociologists may be tempted to clear them a
little further. Although, of course, there is always the chance that the
swift movement of events may eliminate or rather transform that
particular social dilemma, making all such studies academic, even
archaic. In that case it might happen that ad-men would be preserved
chiefly as museum specimens, to an appreciation of which this book might
then serve as a moderately useful guide.

Advertising has, of course, a very ancient history. But since the modern
American phenomenon represents not merely a change in degree but a
change in kind, the chronological tracing of its evolution would be only
confusing. It has seemed better first to survey the contemporary
phenomena in their totality and then present in a later chapter the
limited amount of historical data that seemed necessary and pertinent.\\



% CHAPTER TWO
\chapter[2 \hspace*{1mm} THE APPARATUS OF ADVERTISING]{2 THE APPARATUS OF ADVERTISING}

\newthought{When} we come to describe and measure the apparatus of advertising, some
more or less arbitrary breakdown is necessary. Let us therefore start
with the advertising agency, which is the hub of the advertising
business proper, where all the lines converge. We shall then draw
concentric circles, representing increasingly remote but genuinely
related institutions, people and activities.

In \emph{Advertising Agency Compensation} Professor James A. Young, of
the University of Chicago, estimates that in 1932 there were 2,000
recognized national and local advertising agencies engaged in the
preparation and placing of newspaper, magazine, direct-by-mail, carcard,
poster, radio and all miscellaneous advertising.\footnote{{[}James A. Young,
  \emph{\href{http://www.worldcat.org/oclc/3291593}{Advertising Agency
  Compensation}}\href{http://www.worldcat.org/oclc/3291593}{~}\emph{\href{http://www.worldcat.org/oclc/3291593}{in
  Relation to the Total Cost of Advertising}} (Chicago: University of
  Chicago Press, 1933).{]}} These 2,000 agencies
served 16,573 advertisers. Advertisers served by agencies having
recognition by individual publishers only are excluded from this
estimate.

Prof. Young estimates the 1930 volume of advertising placed through 440
recognized agencies at \$600,000,000. An additional 370 agencies placed
\$37,000,000 in that year. The trend during the post-war decade was
steadily toward the concentration of the business in the larger agencies
with a further concentration brought about by mergers of some of these
already large units.

In 1930 there were six agencies doing an annual business of \$20,000,000
or over, and fourteen with an annual volume of from \$5,000,000 to
\$20,000,000. A further indication of the trend is contained in the
figures showing the advertising income of \emph{American Magazine},
\emph{Colliers}, \emph{Saturday Evening Post}, \emph{Delineator},
\emph{Good Housekeeping}, \emph{Ladies' Home Journal}, \emph{McCalls}
and \emph{Woman's Home Companion}. In 1922, 57.8 per cent of the
combined advertising income of these publications came from the ten
leading agencies. In 1931 this proportion had risen to 68.3 per cent.

A similar trend toward concentration in the sources of advertising
revenue is apparent. Advertisers spending between \$10,000 and \$100,000
annually dropped from 43.8 per cent of the total volume in 1921 to 21.1
per cent of the total volume in 1930. Advertisers spending between
\$100,000 and \$1,000,000 annually increased from 51.3 per cent of the
total volume in 1921 to 55.9 per cent in 1930. Finally, advertisers
spending over a million a year increased their percentage of the total
volume from 4.9 per cent in 1921 to 23 per cent in 1930.

The agency employee, whether he writes advertising copy, draws
advertising pictures or is concerned with one of many routine,
mechanical and clerical processes of the agency traffic, must be listed
as an advertising person; he makes his living directly out of the
advertising business.

The manufacturer's or merchant's advertising staff is also clearly to be
listed as a part of the personnel of the advertising business.

A publisher's representative, or ``space salesman'', is also clearly an
advertising man; so is the circulation promotion manager and his
staff---his budget is an advertising budget. But how about the editorial
department of the newspaper or magazine? Here we are on debatable
ground. If the newspaper or magazine is primarily an advertising
business, since most of its income is derived from advertisers, and all
of its activities, editorial and otherwise, are finally evaluated
according to the degree of their utility in making the publication an
effective and profitable advertising medium, then the total staff of the
publication is an advertising staff; they too make their livings out of
the advertising business.

Without attempting to settle the question, let us first consider certain
statistical trends which show clearly enough the progressive
transformation of our daily and periodical press into advertising
businesses.

In 1909, 63 per cent of newspaper income and 51.6 per cent of magazine
income was from advertising. By 1929 the proportion of advertising
income had moved sharply upward to 74.1 per cent for newspapers and 63.4
per cent for periodicals. Approximately three-quarters of the
newspaper's dollar and two-thirds of the periodical's dollar came from
advertisers.

To correspond with this trend we should expect to find a certain
re-orientation of the function of the newspaper and periodical press,
and that is precisely what we do find. The reader is asked to follow a
digression at this point, since it is important to the general argument.

Increasingly over the past thirty years we find the newspaper asserting
its freedom---\emph{in political terms}. Coincidentally, of course, it
has come more and more under the hegemony of business exercised through
advertising contracts to be either given or withheld. In 1900, 732
dailies acknowledged themselves to be ``democratic'' and 801,
``republican.'' By 1930, papers labeled ``independent democrat'' and
``independent republican'' had increased fivefold, while papers
pretending to be ``independent'' politically jumped from 377 in 1900 to
792 in 1930, when such papers constituted the largest single category.
In commenting on this trend Messrs. Willey and Rice remark, in
\emph{Recent Social Trends}:

\begin{quote}
This increase in claimed political independence may indicate that the
newspaper is becoming less important as an adjunct of the political
party, that it seeks greater editorial freedom, or that \emph{it desires
to include various political adherents within its circulation and
advertising clientele.}\footnote{{[}Malcolm M. Willey and Stuart A. Rice, ``The Agencies of
  Communication,'' in
  \emph{\href{http://www.worldcat.org/oclc/544930}{Recent Social
  Trends}}, vol. 1 (New York: McGraw-Hill, 1933), 205.{]}}
\end{quote}

The italics are the writer's. What this statistical trend would appear
to show, especially when coupled with the coordinate increase of the
newspaper's dependence upon advertising income, is that the newspapers
have realistically adapted themselves to the exigencies of a changing
social and economic situation. This holds almost equally true of the
periodicals. Politics as a means of government was definitely recessive
during this period, and public interest in politics correspondingly
declined. The powers of government were shifting to business. Hence the
press became more and more ``free.'' It freed itself from involvement
with the nominal rulers, the political parties, in order that it might
be free to court the patronage of the real rulers, the vested interests
of business, industry, finance; in return for this patronage, the press
became increasingly an instrument of rule operated in behalf of
business. The press, being itself a profit-motivated business was in
fact obliged to achieve this transition; to orient itself to the
emerging focus of power, and to become in fact though not in name, an
advertising business. In essence, what happened was that both major
political parties had become, in respect to the class interests which
they represented, one party, the party of business; the press, as an
advertising medium, tended to represent that party.

Taking 1909 to 1929 as representing the crucial period of this
transition we find that in 1909 the volume of newspaper advertising was
\$149,000,000 and of periodical advertising \$54,000,000. By 1929 the
figures were \$792,000,000 for newspaper advertising and \$320,000,000
for periodical advertising. Except for the movies, the automobile, and
the radio, no other major American industry has rivaled the swift
expansion of the advertising business.

We have then a combined total of \$1,112,000,000 as the contribution of
newspaper and magazine advertisers to the advertising ``pot.'' In
computing the total contents of this pot we must duly add at least
\$75,000,000 for time on the air bought by advertisers from commercial
broadcasters. The radio, since all its income is derived from
advertisers, must be rated as essentially an advertising business. We
must add \$400,000,000 for direct-by-mail advertising, \$75,000,000 for
outdoor advertising, \$20,000,000 for street-car advertising,
\$75,000,000 for business papers, and \$25,000,000 for premiums,
programs, directories, etc. The foregoing are 1927 figures cited by
Copeland in \emph{Recent Economic Changes}.\footnote{{[}Morris A. Copeland, ``The National Income and its Distribution,'' in
  \emph{\href{http://www.worldcat.org/oclc/637998047}{Recent Economic
  Changes in the United States}}, vol. 2 (New York: National Bureau of
  Economic Research, 1929).{]}} Advertising volume in all
categories went up in 1928 and 1929 and radio volume continued to go up
during the first three years of the depression. Also in these figures no
allowance is made for radio talent bought and paid for by the
advertiser, and none for art and mechanical costs of printed
advertising, billed by the agency to the advertiser with a 15-per-cent
commission added. Hence Copeland's grand total of \$1,782,000,000 for
all advertising must be taken as a very conservative estimate of the
peak volume of the business. Two billion would probably be closer. As to
the number of workers engaged in the various branches of the business,
detailed estimates are difficult to get, chiefly because of the
confusion of categories.

The General Report on Occupations of the 15th Census gives figures of
5,453 men and 400 women as the personnel of advertising agencies, but
under \emph{Advertising Agents and Other Pursuits in the Trade} the
figures are 43,364 men and 5,656 women. Printing, publishing and
engraving must be considered as in large part services of supply for the
advertising business as above defined, and the personnel of these
trades, including printers, compositors, linotypers, typesetters,
electrotypers, stereotypers, lithographers and engravers totals 269,030
men and 33,333 women. In 1927 printing, publishing and allied industries
ranked as the fifth industry in the United States with a total volume of
\$2,094,000,000.

The question, who is or is not connected with the advertising business
is indeed baffling. Is the printer, who makes all or most of his living
out of the advertising business, an advertising man? How about the
engraver, the lithographer, the matmaker, the makers and sellers of
paper and ink---all the hordes of people who as producers, service
technicians, salesmen, clerks operate back of the lines as advertising's
Service of Supplies? Many of these people, especially the salesmen,
certainly think of themselves as advertising people. They are members in
good standing of Advertising Clubs. Toss a chocolate eclair into the air
at any Thursday noon luncheon of the Advertising Club of Kenosha,
Wisconsin, or Muncie, Indiana, and the chances are three to one it will
land on a printer or on an engraver. They are there strictly on
business, of course, and their dues are carried as part of the firm's
overhead. But how they believe in advertising!

Spread the net a little more widely and all kinds of strange fish flop
and writhe in the meshes of advertising. The Alumni Secretary of dear
Old Siwash---is he an advertising man? No? Then why is he a member of
the local advertising club? And how about the football squad, their
trainer, coach, waterboy, cheer-leaders, etc. are they advertising men?
Well, the team advertises the college, and, by general \newpage
\noindent agreement, is
maintained chiefly for that purpose. Why, then, isn't the personnel
involved an advertising personnel?

Then there are the advertising departments of our numerous
university-sanctioned Schools of Business Administration. Are these
fellows advertising men or educators? Dr. Abraham Flexner maintains that
they are not educators, while practical agency heads insist with equal
energy that they are not advertising men. But they can't belong to
nobody and the writer's guess is that they must, however reluctantly, be
categoried as part of the personnel of the advertising business.

Hastening back to firm ground, we can agree that advertising
copy-writers employed by agencies or advertisers are unmistakably
advertising men. So are the fellows who sell space in publications. But
how about the staffs of the various institutes, bureaus, etc., such as
Good Housekeeping Institute, whose job is to test and pass on the
products and appliances advertised in the publication? The raison d'etre
of such departments is that they nourish the confidence of the reader
and thus increase the value of the publication to the advertiser. Are
these fellows scientists, engineers or advertising men?

Without attempting to answer this embarrassing question, let us go
across the hall or upstairs to the editorial department of a modern
publication. The ``travel editor'' is busy computing the current and
prospective lineage bought by various steamship and railroad lines. On
the result of this computation will depend whether next month she will
praise the joys of California's sun-kist climate or the more de luxe
attractions of the Riviera. Is the young woman an editor, a literary
person or an advertising woman?

The fiction editor has on his desk a very suitable manuscript. It has
neither literary nor other distinction, but the subject matter and
treatment are excellent from a pragmatic point of view. The story tells
how a young man was nobody and got nowhere until he bought some
well-tailored clothes; with the aid of these clothes and other items of
conspicuous waste, he established his social status and shrewdly used
his newly-won acquaintances to promote his business career. He ends up
as partner in the firm where he was formerly a despised bookkeeper.
Moral: it pays to wear smart clothes, even if you have to go in debt to
buy them. The story is in effect an excellent institutional
advertisement for the men's clothing industry, and will be so regarded
by present and prospective clothing advertisers. Is its author a
literary man or an advertising man? Is the editor who chose this story,
for the reasons indicated above, an editor and critic or an advertising
man? The story will be illustrated by an artist who specializes in his
knowledge of styles in men's clothing. When he makes his illustrations
he will have before him as ``scrap'' the latest catalogues of the
clothing houses. Is he an artist, an illustrator or an advertising man?

It may seem unkind to press the point, but we have barely begun to list
the peripheral personnel of the advertising business. The electrician
who repairs the neon signs on Broadway---is he an electrician or an
advertising man? The truck driver who delivers huge rolls of paper to
the press rooms of the newspapers---where would he be, but for the
advertising business that keeps those presses busy dirtying that paper?
And the bargemen who floated that newsprint across the Hudson? And the
train crew that freighted it down from Maine? And the loggers in the
Maine woods that supply the pulp mills? And the writers for the
``pulps'' who go to Maine for their vacations?

It is not necessary to project this unbroken continuity into the realm
of fantasy. Both in respect to the number of persons employed and the
total value of manufactured products, advertising is, or was in 1929,
one of the twelve major industries of the country. We are living in a
fantastic ad-man's civilization, quite as truly as we are living in what
historians are pleased to call a machine age, and a very cursory
examination of the underlying economic trends will be sufficient to show
how we got there.

The essential dynamic of course is the emergence of our ``surplus
economy'' predicament, generated by the application of our highly
developed technology to production for profit. Advertising played a more
or less functional though barbaric and wasteful role during the whole
expansionist era of American capitalism. The obsolescence, the reductio
ad absurdum of advertising is betrayed by the exaggerations, the
grotesqueries, which accompanied its period of greatest expansion during
the postwar decade. Like many another social institution, it flowered
most impressively at the very moment when its roots had been cut by the
shift of the underlying economic forces.

Between 1870 and 1930 several millions of people were squeezed out of
production. Where did they go? The statistical evidence is plain. In
1870 about 75 per cent of the gainfully employed people of the United
States were engaged in the production of physical goods in agriculture,
mining, manufacture and construction. In 1930 only about 50 per cent of
the labor supply was so required. In 1870, ten per cent of the employed
population was engaged in transportation and distribution. In 1930, 20
per cent was engaged in transportation and distribution. What caused
this shift was chiefly the increase in man-hour productivity made
possible by improvements in machine technology and in the technique of
management. The chapter on ``Trends in Economic Organization'' by Edwin
F. Gay and Leo Wolman in \emph{Recent Social Trends} documents this
increase as follows:
\pagebreak
\begin{quote}
The combined physical production of agriculture and of the
manufacturing, mining and construction industries increased 34 per cent
from 1922 to 1929.... The advance in output was steady throughout the
period and even in the recession years, 1924 and 1927, the decline was
surprisingly small. Much more important, however, is the comparison
between the rate of increase in physical output in the prewar and
postwar periods. Per capita output, reflecting retardation in the rate
of population growth, as well as the rise in production, advanced twice
as fast in the later years as in the earlier, as is indicated by the
average annual rate of increase.\footnote{{[}Edwin F. Gay and Leo Wolfman, ``The Agencies of Communication,'' in
  \emph{\href{http://www.worldcat.org/oclc/544930}{Recent Social
  Trends}}, vol. 1 (York, PA: Maple Press Company, 1933), 232.{]}}
\end{quote}

\addvbuffer[8pt 15pt]{\begin{tabu} to 0.9\textwidth { X[l] X[c] X[c] X[c] } 
 \emph{Period} & \emph{Volume of production per cent} & \emph{Population per cent} & \emph{Per capita production per cent} \\
 1901--1913 & +3.1 & +2.1 & +1.1 \\ 
 1922--1929 & +3.8 & +1.4 & +2.4 \\ 
\end{tabu}}

Although real wage levels rose slightly during this period they did not
rise proportionately to the increase in man-hour productivity, the
increase in profits, the increase in plant investment, and the increase
in capital claims upon the product of industry. The result of these
conflicting trends was to place an increasing burden upon the machinery
of selling. This is reflected in the rising curve of sales overhead, the
increase in small loan credit and installment selling and the meteoric
rise of advertising expenditure during the post-war period. According to
the estimate of Robert Lynd in \emph{Recent Social Trends} the total
volume of retail installment sales in 1910 was probably under a billion
dollars. By 1929 it had increased to seven billion dollars.\footnote{{[}Robert S. Lynd, ``The People as Consumers,'' in
  \emph{\href{http://www.worldcat.org/oclc/544930}{Recent Social
  Trends}}, vol. 2 (York, PA: Maple Press Company, 1933).{]}}

Undoubtedly this six-billion-dollar shot in the arm postponed the
crisis, intensified its severity and contributed importantly to the
Happy Days of advertising during the New Era. After the crash it was of
course the ad-men who were urged to put Humpty-Dumpty back on the wall.
They tried manfully, but since it is impossible to advertise a defunct
buying power back into existence, they didn't succeed. And now, after
four years of depression it would appear that the ad-man has learned
nothing and forgotten nothing.

That two-billion-dollar advertising budget is a lot of money. In 1929 it
represented about two per cent of the national income for that year, or
\$15 per capita. It might well be alleged that the bill was high, would
have been high even for a competently administered service of
information. And, as already indicated, advertising is scarcely that.
What that two billion represented, what the present billion and a half
advertising volume represents, is in considerable part the tax which
business levies on the consumer to support the machinery of its
super-government---the daily and periodical press, the radio, the
apparatus of advertising as we have described it. By this
super-government the economic, social, ethical and cultural patterns of
the population are shaped and controlled into serviceable conformity to
the profit motivated interests of business.

Our notoriously extravagant official government is really much more
modest, considering that it gives us in return such tangible values as
roads, sewers, water, schools, police and fire departments, and such
grandiose luxuries as the army and navy. The combined tax bill of the
nation, Federal, State, and local, amounted to only \$10,077,000,000 in
1930 or roughly about \$75 per capita.

It will be argued, of course, that even if advertising is thrown out of
court as a service of information, since that is neither its intent nor
its effect, nevertheless this two-billion dollar industry does net us
something. But for advertising, we should not be able to enjoy the radio
free, or read the \emph{Saturday Evening Post} at five cents a copy, or
Mr. Hearst's \emph{American Weekly}, which is thrown in free with his
Sunday newspapers. In other words, it will be argued that advertising is
justifiable as an indirect subsidy of our daily and periodical press and
the radio; that for this two billion dollars, which has to be charged
ultimately to the consumer, we get a tremendous quantity of news,
information, criticism, culture, pretty pictures, education and
entertainment. We do, indeed, and as taxpayers we value this
contribution to our welfare so highly that our Post Office Department
also heavily subsidizes our daily and periodical press. Also we pay the
Federal Radio Commission's annual million-dollar budget, consumed
chiefly in adjusting commercial dog-fights over wave lengths.

But the actual quality and usefulness of what we get is another matter.
In exchange for these official and unofficial subsidies we get a daily
and periodical press which has practically ceased to function as a
creative instrument of democratic government: which does, however,
function effectively as an instrument of obscuration, suppression and
cultural stultification, used by business in behalf of business; which
levels all cultural values to the common denominator of emulative
acquisition and social snobbism, which draws its daily and weekly
millions to feast on the still-born work of hamstrung reporters,
escape-formula fictioneers, and slick-empty artists; which, having
stupefied its readers with this sour-sweet stew of nothingness, can be
counted on to be faithful to them in all issues which don't particularly
matter and to betray them systematically and thoroughly whenever their
interests run counter to the vested interests of business.

In this indictment it is not denied that we have in America many honest
newspapers and honest magazines, honest editors, honest reporters and
honest advertising men. They are honest and blameless within the limits
of the pattern prescribed for them by the economic determinants of the
institutions which they serve. Some of them even struggle at great peril
and sacrifice to break through and transcend these limits. It is
inevitable that they should do so, since not only their readers but
themselves are violated by the compulsions of the system in which both
are caught.

But the system itself is substantially as described. The American
apparatus of advertising is something unique in history and unique in
the modern world; unique, fantastic and fragile. One needs but little
knowledge of history, or of the movement of contemporary economic and
social forces, to know that it can't last. It is like a grotesque,
smirking gargoyle set at the very top of America's sky-scraping
adventure in acquisition \emph{ad infinitum}. The tower is tottering,
but it probably will be some time before it falls. And so long as the
tower stands the gargoyle will remain there to mock us.

The gargoyle's mouth is a loud speaker, powered by the vested interest
of a two-billion-dollar industry, and back of that the vested interests
of business as a whole, of industry, of finance. It is never silent, it
drowns out all other voices, and it suffers no rebuke, for is it not the
Voice of America? That is its claim and to a degree it is a just claim.
For at least two generations of Americans---the generations that grew up
during the war and after---have listened to that voice as to an oracle.
It has taught them how to live, what things to be afraid of, what to be
proud of, how to be beautiful, how to be loved, how to be envied, how to
be successful. In the most tactful manner, and without offending either
the law or the moralities, it has discussed the most intimate facts of
life. It has counselled with equal gravity the virtue of thrift and the
virtue of spending. It has uttered the most beautiful sentiments
concerning the American Home, the Glory of Motherhood, the little
rosebud fingers that clutch at our heartstrings, the many things that
must be done, and the many, many things that must be bought, so that the
little ones may have their chance. It has spoken, too, of the mystery of
death, and the conspicuous reverence to be duly bought and paid for when
Father passes away.

So that today, when one hears a good American speak, it is almost like
listening to the Oracle herself. One hears the same rasping,
over-amplified, whisky-contralto voice, expressing the same ideas,
declaring allegiance to the same values.

So that when somebody like the writer rises to say that the Oracle is a
cheat and a lie: that he himself was the oracle, for it was he who cooed
and cajoled and bellowed into the microphone off stage; that he did it
for money and that all the other priests of the Advertising Oracle were
and are similarly motivated: that the Gargoyle-oracle never under any
circumstances tells the truth, the whole truth and nothing but the
truth, for the truth is not in her: that she corrupts everything she
touches---art, letters, science, workmanship, love, honor, manhood....

Why, then, your American is not in the least abashed. He knows the
answer. It was pretty smart, wasn't it? It certainly does pay to
advertise! You know, I've always thought I'd like to write
advertisements! How does one get into the Advertising Business?



% CHAPTER THREE
\chapter[3 \hspace*{1mm} HOW IT WORKS]{3 HOW IT WORKS: The Endless Chain of \newline Salesmanship}
\chaptermark{3 HOW IT WORKS}

\newthought{The} apparatus of advertising, conceived of as the total apparatus of
daily and periodical publishing, the radio, and, in somewhat different
quality and degree, the movie and formal education, is ramified
interlocking and collusive, but \emph{not unified}. This distinction
must be kept carefully in mind. Most of the residual and fortuitous
mercies and benefits that the public at large derives from the system
are traceable to the fact that the apparatus of advertising is not
unified; it exhibits all the typical conflicts of competitive business
under capitalism plus certain strains and stresses peculiar to itself.

With the system operating at the theoretical maximum of its efficiency,
the sucker, that is to say the consumer, would never get a break. In
practice, of course, he gets a good many breaks: a percentage of
excellent and reasonably priced products, a somewhat higher percentage
of unbiased news, a still higher percentage of good entertainment both
on the air and in the daily and periodical press. He even gets a modicum
of genuine and salutary education---more, or less, depending on his
ability to separate the wheat from the chaff.

No system is perfect and the apparatus of advertising suffers not merely
from human frailty and fallibility but from the lag, leak, and friction
inherent in its design.

The apparatus of advertising is designed to sell products for the
advertiser, and to condition the reflexes of the individual and group
mind favorably with respect to the interests of the advertiser. The
desired end result of the operation of the apparatus is a maximum of
profitable sales in the mass or class market at which the advertising
effort is directed.

But the apparatus itself is made up of a series of selling operations as
between the constituent parts of the system. Each of these parts \pagebreak is
manned by rugged individuals, all bargaining sharply, not merely for
their respective organizations but for themselves. In attempting to
trace this endless chain of selling one wonders where to begin. Perhaps
the advertising agency is as good a starting point as any.


\section{THE ADVERTISING AGENCY.}\label{the-advertising-agency}}

The advertising agent was originally a space broker dealing in the white
space that newspapers and periodicals had for sale. He bought space
wholesale from the publishers as cheaply as possible and retailed it for
as much as he could get from advertisers. In the early days he
frequently made a handsome profit---so handsome that the more powerful
publishers attempted to stabilize the system by appointing recognized
agents and granting them a commission on such space as they sold to
advertisers. The amount of the commission varied. For the compensation
they delivered a service consisting of selling, credit and collection.
The advertiser planned and wrote his own advertisements and had them set
up and plated; he did his own research, merchandising, and so forth.

But more and more the agent tended to take over these functions. He
dealt with many advertisers and hence was in an excellent position to
become a clearing house of experience. From a seller of white space he
became a producer of advertising. In a comparatively short period of
years the larger national advertisers were placing their advertising
through agents whose functions were the following: planning and
preparing the advertisement in consultation with the sales or
advertising manager of the advertiser; attending to all details of art
purchase, mechanical production, etc.; selection of publication media in
which the advertising campaign would appear; checking the insertions in
these media. ``Research,'' ``Merchandizing,'' etc., were later functions
of the agency, which in the larger agencies today are handled by
well-established departments.

The advertising agency is thus in the somewhat ambiguous position of
being responsible to the advertiser whom he is serving but being paid by
the advertising, publication or other advertising medium, his commission
being based on the volume of the advertiser's expenditure. Objection to
this commission method of agency compensation has been chronic for
years. There are today a few relatively small agencies that operate on a
service fee basis. But the commission method of compensation has
persisted and is a factor in the endless chain of selling that links the
whole advertising apparatus.

Before the agent is entitled to receive commissions from the various
advertising media---magazines, newspapers, radio broadcasters, carcard
and outdoor advertising companies---he must first be ``recognized.'' To
secure recognition he therefore presents to each of these media groups,
which maintain appropriate trade committees for this purpose, evidence
that he is financially responsible and controls the placing of a certain
minimum of advertising business. The first selling job is therefore that
of the agent in ``selling'' his competence and responsibility to the
organized media.

When recognition is once granted, however, the agent steps into the
buyer's position in respect to the media. His duty is then to his
clients, the advertisers. In return for the commission paid by the media
which has been more or less stabilized at 15 per cent less a two per
cent discount for cash, which is passed on to the client, the agent is
expected to prepare effective advertising, properly co-ordinated with
manufacturing and sales tactics, and place it in the media most
effective for the purpose.

Walk into the lobby of any large advertising agency and you will see
about a dozen bright young men with brief cases waiting to see agency
account executives or media department heads. They are space salesmen.
The brief cases contain lavishly printed and illustrated promotion
booklets which serve as reference texts for the salesmen. Many thousands
of dollars go into the compilation of the data printed in one of these
booklets. In it the publication's advertising manager proves that his
``book'' has so many subscribers and is bought at newsstands by so many
people, as attested by the impartial Audit Bureau of Circulations. These
readers are concentrated in such and such areas. They represent an
average annual unit buying power of so much as evidenced by the property
ownership of houses, automobiles, etc., etc. Their devotion to the
publication is evidenced by such and such a turnover of subscribers and
such and such a curve of circulation increase. Their confidence and
response to advertising placed in the publication is evidenced by the
success of advertisers A, B and C, whose campaigns last year proved that
advertising in the \emph{Universal Weekly} brings inquiries for only so
much per inquiry; furthermore such and such a percentage of these
inquiries were materialized into sales. The \emph{Universal Weekly} also
exercises an important influence upon dealers. The broadside reproducing
his campaign with which advertiser A circularized the trade, resulted in
stocking so and so many new dealers. The advertising department of the
\emph{Universal Weekly} also co-operates earnestly with advertisers; in
fact staff representatives of the publication delivered so and so many
of these broadsides, and are even responsible for the addition to the
advertiser's list of so and so many new outlets.

The editorial department of the \emph{Universal Weekly} is also warmly
co-operative. During the year 1932 the \emph{Universal Weekly} applied
the editorial pulmotor to its readers' flagging will-to-buy with
measurable success. Note also the ``constructive'' quality of the
articles printed in the \emph{Universal Weekly}, that it gives also abundant
quality in its fiction did it not pay Pete Muldoon the highest price
ever paid a fictioneer for a serial?

These promotion booklets constitute an important and greatly neglected
source of economic and sociological data. Moreover, some of them are
honest from start to finish. They had better be, on the whole. The
agency's space buyer is hardboiled. He sees \emph{all} the promotion
booklets. Moreover, he has access to the advertising and sales records
of a variety of clients. He can and does construct his own private pie
charts; he can and occasionally does send his own crew of college-bred
doorbell ringers into the field to find out what sort of people read
what. On the basis of this calculus he says yes or no to the publisher's
representative.... Well, not quite that. The publisher's representative
has also seen the advertiser's advertising manager. And the publisher
himself played golf last week with the Chairman of the Advertiser's
Board. And the wife of the publisher's advertising manager gave a tea
yesterday to the wife of the agency's vice-president who would like to
get into the Colony Club. Also, the space salesman and the agency's
space buyer are both enthusiastic members of the Zeta chapter of Epsilon
Sigma Rho---remember that time we smuggled Prexy's prize pig into the
choir loft?

There are certain other considerations. Agencies select media subject to
the approval of the client. But publishers' representatives are also in
a position to recommend agencies to manufacturers who are about to make
their debut as advertisers or to regular advertisers who are thinking of
changing agencies. Also agency space buyers sometimes change jobs. They
may go to other agencies or become space salesmen themselves. And space
salesmen frequently graduate into agency account executives.

What with one thing and another the agency space buyer is likely to say
yes \emph{and} no---until \emph{all} the data of his calculus is in
hand.

It is necessary to sketch this background of intrigue because it is
unquestionably a factor in the traffic of advertising where the stakes
are large and a decision one way or another can readily be justified on
entirely ethical grounds. It is a minor factor. Curiously enough there
is probably less of it in the advertising business than in most other
businesses; much less, for instance than in the movie industry, or in
the field of investment banking. It is indeed puzzling that the ad-man,
whose stock-in-trade in his relations with the public, is pretty much
bunk, should exhibit, in the internal traffic of the business, a
relatively high standard of personal integrity. Yet the writer is
convinced that this is so, and in later chapters will offer tentative
explanations why this should be so.

The agency-publication-advertiser relation is of course only one loop of
the endless chain of selling. To complete the circuit in detail would
scarcely be useful at this point. The major sequences may be summarized
briefly as follows:

\section{SERVICES OF SUPPLY.}\label{services-of-supply}}

The raw material of advertising consists of ink, paper, paint,
photographic materials and talk. The techniques involved are too
numerous to list, especially since new techniques are constantly
emerging. In the lobby of the agency swapping cigarettes and gossip with
the space salesmen are regularly to be seen the salesmen representing
advertising's services of supply. They are all there in person or
represented by their salesmen. The printer, the lithographer, the
photographer, the carcard and outdoor advertising companies, the
direct-by-mail house, which is a printing house with much of the
production personnel and equipment of the agency; the advertising
``novelty'' house, a ``public relations'' expert, a couple of
broadcasting companies and three specimens of radio talent. Also the de
luxe young woman who serves as go-between in the testimonial racket;
also half a dozen people of both sexes who are looking for jobs. They
have heard that the agency has just captured the Primrose Cheese
account.

All told it makes quite a mob. The reception clerk is either gray-haired
and dignified, or young, pretty and amiable. She is busy continuously on
the telephone, glibly translating the account executive's ``Nothing
doing'' into ``Mr. Blotz is \emph{so} sorry. Couldn't you come tomorrow
at about this time?'' Eventually most of these salesmen are seen by
somebody. The agency is in the selling business too and can't afford to
upstage anybody. While they are waiting they improve their time by
selling each other. The printer sells the direct-by-mail house
executive; the engraver sells the printer; the lithographer sells the
outdoor advertising representative; the radio talent sells the
broadcaster. Only the testimonial racketeer remains uninterested.
Deciding that there isn't a profitable date in a carload of these
people, she gives it up and goes home.

\section{INTRA-MURAL SELLING.}\label{intra-mural-selling}}

It must be understood that an advertising agency is a loose aggregation
of rugged individuals each of whom is very busy carving out his or her
professional career. This occasions more or less continuous conflict and
confusion. The technique of combat is salesmanship. The movement is the
circular movement of the dance, with alternating tempos of dreamy waltz
and frantic fox-trot. There is much cutting-in and swapping of partners.
Everybody is busy selling everybody else; this entails much weaving from
desk to desk; many prolonged luncheon conferences; many convivial
midnight parties in Bronxville, Great Neck and Montclair. The mulberry
bush around which this dance revolves is known in the trade jargon as
the Billing, that is to say, the total volume of advertising on which
the agency gets commissions. Everybody knows the amount of the
commission and everybody knows or can guess approximately the amount of
the Billing. Hence everybody is constantly doing mental calculations in
which the opposing factors are ``How much do I do?'' and ``How much am I
paid?'' The answer never comes out right for anybody. The copy-writer
notes that he writes all the copy on three accounts the total annual
billing on which averages say a million dollars. Fifteen per cent of a
million dollars is \$150,000. The copy-writer's salary is \$5,000 and
this year no bonus was paid at Christmas time. The discrepancy is
obvious. The copy-writer considers that all the other processes of the
agency, such as art production for which a separate added commission is
charged, media selection, client contact, new business getting,
forwarding, billing and other routine tasks, are just as much overhead
and that there is too damned much of it; also too damned much profit
going into the salaries and dividends received by the heads of the
agency. All the other members of the ``creative'' staff entertain
similar views differing only in the focus of the particular grievance;
whereas the lowly clerical and mechanical workers are convinced that the
agency wouldn't get paid unless the advertisements got into the
newspapers and magazines. They too have their grievances. The way out
for all these people is salesmanship. Hence everybody sells everybody
else; the copy writer and the art director sell the account executives
on the relative importance of copy \emph{versus} art or art
\emph{versus} copy; the research director sends memoranda up to the top
pointing out that it is impossible to sell shoes without an adequate
economic and anatomical study of feet; the new-business-getter inquires
with some acerbity, who brought this account into the house?

Observing this disorder in the ranks, the heads of the agency are
puzzled and heartsick. They work hard---yes, many of them do work
preposterously hard. Few of them make large fortunes out of the agency
business directly. They give more or less secure employment to hundreds
of people. And in return they get an amount of grouching, chiseling and
intrigue that is positively appalling.

The dance around the mulberry bush grows dreamier and dreamier, or
wilder and wilder. Since the generated energy is centrifugal in nature,
it happens at more or less regular intervals that one of the dancers
furtively leaves the floor and runs across the street with a sprig of
the mulberry bush in his teeth. Panic ensues. A chosen few of the
apostate's intimates follow their leader across the street. If the
mulberry sprig roots and flowers, a new agency is established, the music
strikes up, and a new dance begins around the new mulberry bush.

Meanwhile, in the parent agency a period of stricter discipline is
inaugurated. Disaffected staff members are scared or flattered back into
line. New management devices are introduced, which have as their
objective an improved agency morale. They are selling devices primarily.
The staff is sold on the integrity and fairness of the directing heads;
they are sold on the honor and dignity of the advertising profession;
they are assured that the way to the top is always open; that copy
writers, junior executives, etc., who work hard and keep their eyes off
the clock will be given higher responsibilities, with commensurate
increases in salary. The virtues of the ad-man are industry, alertness
and loyalty, and the greatest of these is loyalty. On the anniversary of
his employment with the agency each employee finds on his desk a white
rose. All are urged to take a greater interest in the business. Monday
morning staff conferences are instituted. A frequent subject of
discussion at such conferences is the obligation, falling on every
ad-man, to believe in what he is selling. How can he sell the public
until he has first sold himself? This would seem a somewhat harsh
requirement, but the reader is asked to believe that a percentage of
ad-men fulfill it quite literally. By a process of self-hypnosis they
become deliriously enthusiastic about whatever they are obliged to sell
at the moment.

Their homes are museums of advertised toothpastes, soaps, antiseptics
and gadgets. From themselves, their wives and their children, they exact
the last full measure of devotion. They are alternately constipated with
new condiments and purged with new laxatives, while their lives are
forever being complicated with new gadgets.

Since accounts change hands frequently, a certain openmindedness of
judgment, and a certain emotional flexibility are parts of the necessary
equipment of the ad-man. He must be prepared at a moment's notice to
forswear toothpaste A and announce undying devotion to toothpaste B; to
rip out a whole line of bathroom equipment and install a new line; to
turn in his McKinley Six for a Hoover Eight, whether he can afford it or
not. His ability to do all these things without any outward evidence of
insincerity is little short of miraculous.

The ad-man is indeed a kind of Candide. His world is the best of all
possible worlds, as the Russians say, every change is good, even for the
worse. For instance, he may work for a small agency and passionately
proclaim the efficiency of the smaller service organization as against
that of the half-dozen mammoths of the business. But let his agency be
merged with one of these mammoths and he will make speech at the ensuing
convention of the joined staffs, in which he declares with tears in his
eyes that this marriage was made in heaven. If, as sometimes happens,
the merger was in fact a shotgun marriage consummated more or less at
the behest of the sheriff, his fervor will be heightened only by this
circumstance, which he will stoutly deny to all and sundry. He is not
consciously lying. He literally believes what he is saying. His is
indeed the faith that passes understanding.

In puzzling over such phenomena, it has occurred to the writer that
there is something feminine about the makeup of your died-in-the-wool
ad-man. This is probably an acquired characteristic, a sort of
industrial hazard, or occupational disease peculiar to the business. The
point will become more clear when it is remembered that the advertising
agency is the scene of frequent accouchements---this is indeed the
business-as-usual of the agency. Your ad-man is continuously either
\emph{enceinte} with big ideas, or nursing their infant helplessness. In
this delicate condition he can scarcely be held intellectually or
morally responsible for his opinions and acts. Behind him is the whole
pressure of the capitalist organism, which must sell or perish.

Hence the ad-man's morning sickness, his tell-tale fits of dizziness
after lunch, his periods of lachrymose sentimentality, his sleepless
vigils after hours, his indifference to considerations of elementary
logic---the charming hysteria, in general, of his high-strung
temperament. Hence his trepidation as he approaches the ultimate ordeal
to be described in the next chapter---the Presentation to the Client.





% CHAPTER FOUR
\chapter[4 \hspace*{1mm} PRIMROSE CHEESE: An Advertising Accouchement]{4 PRIMROSE CHEESE: An Advertising Accouchement}
\chaptermark{4 PRIMROSE CHEESE}

\section{1. Prelude}

\newthought{From} his window close to the top of one of the minor skyscrapers of the
Grand Central district, Eddie Butts, for two months now, has been
watching the spectral towers of Radio City climb into the western sky.



Eddie Butts sighs. It is after hours, and Eddie is tired. The sigh flies
out the window, wreathes itself jocosely around the topmost tower, and
returns as an ironic, incomprehensible whisper in Eddie's ear.

Eddie Butts shakes his head like a blind horse troubled by flies. He
must get down to business. He must get out his work-sheet for the next
day. Eddie turns to the dictaphone.

``Follow Schmalz on XYZ schedule stop Have Chapin phone Universal on LHJ
extension stop Call up Hank Prentice stop Ask him how the hell he is
stop Follow Chris on revises BDB layouts stop Call Gene at the Club
{[}Gene is getting drunk with a client tonight strictly in line of duty,
and it is standard practice to wake him up at noon of the next day{]}
Revise plan for Primrose Cheese stop Lather Lulu a little stop {[}Lulu
is the radio prima donna who got miffed at the last Cheery Oats
broadcast{]} Organize Vita-pep research stop Follow Mac on Spermentine
publicity stop Tell him to damn well watch his step stop Follow stop
Follow stop---err Stop.''

A telephone is ringing persistently at the other end of the floor.
Probably nothing important---some girl friend calling one of the boys in
the checking room. But you can never tell. Eddie's sense of duty is
strong. He decides not to take a chance.

``Hello ... Hello ... Who? Oh, hello, Bob. This is Eddie. What's the
matter? Are you in trouble? ... Oh, so I'm in trouble am I? ... Go on,
you're drunk ... What's that? Sure, that's right. We're all ready to
shoot. Old Himmelschlussel himself will be on here from Racine, day
after tomorrow, and we give him the works, see? What? Oh, swell. Swell
slant. Swell art. Thought I told you about it. Cheese and beer, cheese
and cigarettes. Cheese for dessert. The continental idea, you know. Put
cheese on the map. Himmelschlussel? No, I've never met him. What? Who
says so? Who's Oscar? Yes? Well, is he sure about that? What? Say, how
soon can you get over here? Sure, bring Oscar. Step on it. I'll wait for
you."

Eddie Butts' shoulders sag slightly as he stumbles along the half-lit
corridor back to his office. This might be just a space salesman's wise
crack. On the other hand, it might be a real one---another fire alarm.
In which case---

Eddie went to the bookcase and took down the three elaborately bound
volumes that represented the agency's submission on the Primrose Cheese
account.

Vol. I. Section 1. Market analysis, plan, and consumer, copy, (the
layouts are already tacked up on the wall in the conference room)
Section 2. Report of the domestic science Bureau. Section 3.
Merchandizing plan, trade copy, dealer helps.

Vol. II. Report of the Research department.

Volume III. Media analysis and estimates. (This is an oversize volume
composed of charts and hand-lettered captions.)

For the layman, a word of explanation is perhaps required at this point.
The submission as listed above involves an investment by the agency of
approximately \$10,000. It is a gambling investment, even though in this
instance the client has signed a contract appointing the XYZ company as
his advertising agent, and certain frail safeguards to the agent are
embodied in this contract. It is a gambling investment because all this
work has been done subject to the client's approval, and most of it be
paid for only when and if the client o.k.'s the campaign and the
advertising begins to appear.

In some cases such presentations are sheerly speculative, since they are
made \emph{before} the agent is appointed, as a means of selling the
client and securing the account. Such speculative selling by the agency
is frowned upon by the organized profession and is prohibited in the NRA
agency code of fair competition. There are, of course, many ways of
evading this prohibition, and since the agency field is highly
competitive, such evasions will probably continue, much as in the past.

It may be asked: why this extraordinary and costly elaborateness of
selling? The explanation resides chiefly in the commission method of
compensation. To the client that 15\% commission looks like a lot of
money---\emph{is} a lot of money when applied to a total annual
expenditure by the client of, say, \$12,000,000 for advertising a single
brand of cigarette.

The economic logic of the situation induces two opposing points of view.
From the agency's point of view, the client is the squirming,
recalcitrant fly in the otherwise pure ointment of that 15\% commission.
All clients are unreasonable in theory and frequently so in fact. In
justice to the agency it should be said that the majority of reputable
agencies strive earnestly to earn their commissions. They work hard and
even in the best of all possible worlds they make big money only by a
lucky break, to be discounted by a succession of bad breaks next year.
But the client either doesn't know this or doesn't care. On the
principle of caveat emptor, the client has to be shown.

To put it crudely, the agent, from the advertiser's point of view, is a
bunk-shooter, a hi-jacker, with whom he is obliged to deal merely
because he has to pay that 15\% commission anyway. In its relations to
clients, the agency may be neither a bunk-shooter nor a hi-jacker, but
it is guilty as charged until it proves itself innocent. When possible
the client forces the agency to split the commission; or the advertiser
may finance his own ``house agency.'' There are arguments against both
these devices. When they seem plausible, recourse is had to other forms
of chiseling. The agency is perhaps asked to pay the salary of the
client's advertising or sales manager. In any event the client insists
on ``service'' and lots of it. He demands free research and
merchandizing service, for which the agency would like to charge, and
sometimes does charge an additional fee. He insists on dealing with the
principals of the agency, whether his account is large or small, and
irrespective of the competence of the staff workers assigned to the
account. The advertising manager expects the agency's art department to
design his Christmas cards and forget to bill him. The advertisers'
statistician expects the agency's copy department to find a publisher
for the verse of the Wunkerkind spawned by his sister-in-law. When the
advertiser's advertising manager, or sales manager, or vice president of
the Company, their wives, cousins, etc., come to New York, they are duly
entertained in more or less Babylonian fashion, depending upon their
estimated importance, and their previously ascertained habits and
tastes. The bill for this entertainment is duly applied to the agency's
overhead on that particular account.

But the necessitated elements of conspicuous waste are most apparent in
the Presentation to the Client which our friend Eddie Butts, in the
nocturnal solitude of his skyscraper eyrie, is now somewhat morosely
examining.

The service embodied in that presentation must look as if it were worth
at least twice what the client is asked to pay for it, as determined by
15\% of the net recommended expenditure for publication, radio,
car-card, poster, direct, and other miscellaneous advertising. In this
respect it is like the presentation of any advertised product to the
consumer. The jar of cold cream worth 8 cents must look as if it were
worth the \$2.00 that is charged for it. The cheap car must look like an
expensive car. The \$1.98 dress must look like a million dollars. All
this is what is known as ``psychological'' selling, and the principle
operates in unbroken continuity through the whole fabric of the
advertising business.

Eddie Butts conducts his examination of the agency's highly styled and
psychologized product from back to front. The client, when the
presentation is made to him, will proceed similarly, since the nub of
the argument lies in the recommended net expenditure, a figure which
appears inconspicuously at the end of Volume III.

In this case, the figure is only moderate---about \$500,000---and as
Eddie Butts, reading from right to left, weaves through the maze of
charts, tables, graphs, copy and merchandizing these, etc., etc., he
reflects ruefully that this presentation not only looks like a million
dollars, but as a matter of fact, it has already cost the agency a good
deal more than it should have cost.

There has been a lot of grief on this account. In the beginning it
dropped into the house more or less out of the blue. Old Hanson came
back from a trip through the Middle West with the contract in his
pocket. Everybody was considerably surprised, since Hanson's function in
the agency had come to be regarded as almost wholly ornamental. A rather
handsome, gray-haired, middle-aged person, his appearance and manner
suggested extreme probity, conservatism, and a certain wise and
sophisticated benignity. Copy writers, art directors and other
``creative'' workers occasionally testified to each other that Hanson
was stupid, and produced more or less convincing evidence to this
effect. But the heads of the agency, being a shade more sophisticated
than either Hanson or his critics, were aware that certain varieties of
handsomely packaged stupidity are not without their uses in the
advertising business. So that Hanson's position was secure.

But he certainly had pulled a boner on this account. Eddie recalled the
preliminary conference called to consider the problem of Primrose Cheese
and to devise appropriate solutions.
\enlargethispage{\baselineskip}
The stenographer's record listed as among those present Hanson, Butts,
(Eddie was the group director having supervisory responsibility for the
account) McNear, the art director and Appleton, his young assistant;
Blashfield, the brilliant copy-art-plan man, the outstanding advertising
genius of the Kidd, Kirby \& Dougherty Agency; Shean, the copy man,
whose strictly disinterested facility made him a useful understudy for
Blashfield and others; Mrs. Betts, the head of the Domestic Science
Bureau, a rather grandiose, gray-haired personality, full of sex
antagonism and quite without a sense of humor; 
Harmsworth and Billings,
the last-named being merely a couple of obscure copy hacks.
\clearpage
The day previous to the conference, all these people had received, along
with notice of their mobilization, a sample of Primrose Cheese, with
strict injunction to eat it that evening. It was a large sample, and
Eddie recalled that some of the conferees looked a little the worse for
wear that morning.

In opening the meeting, Eddie made the usual preliminary pep talk, duly
deposited the problem on the long mahogany table, and called for
solutions.

\emph{Mr. Hanson:} Since I am more or less responsible for bringing this
account into the house, perhaps I should tell you some of the
circumstances. Mr. Outerbridge, the advertising manager of the Primrose
Cheese Company, is a college classmate of mine, and it is through him
that the account was secured. The Primrose Cheese Company is one of the
four largest manufacturers of cheese in America. Yet hitherto it has
never advertised its products, except in the grocery trade press. The
reputation of Primrose Cheese with the trade is unexcelled. It is sold
from Coast to Coast and from Maine to Florida. Recently sales have been
declining. The competition of advertised packaged brands has been
steadily eating into their business. They've got to advertise. Mr.
Outerbridge is convinced of this. His principal, Mr.---Mr.
Himmelschlussel, President of the Primrose Cheese Company, whom I did
not have the privilege of meeting, is I understand still reluctant. But
he realizes that something has to be done, and he has consented to the
appointment of this agency subject to his approval of our
recommendations. We've got a tough selling job on all fronts, gentlemen.
We've got the whole job to do: packaging, merchandizing, branding,
pricing, merchandizing the whole works. It's an old conservative firm
and their credit is A1. Mrs. Betts is experimenting with Primrose Cheese
and the Research department has already started its work. What we want
today, I take it, is some first class advertising ideas. I have an idea
myself, but I shan't spring it until I've heard from some of the rest of
you.

\emph{Mr. Shean:} What kind of cheese is it?

\emph{Mr. Hanson:} Just good, one hundred per cent American cheese. You
ought to know. You ate some of it, didn't you?

\emph{Mr. Shean:} Yeah, I did. Will you excuse me a moment. I'll be
right back.

(Silence)

\emph{Mr. Butts:} Charley, why don't you start the ball rolling
yourself. You said you had an idea.

\emph{Mr. Hanson:} Very well. I have here, gentlemen, an option signed
by the originator of Mickey and Minnie Mouse. By the terms of this
option, it is understood that in consideration of a payment of one
thousand dollars, which I took the liberty of making on my own
responsibility, both Mickey and Minnie Mouse will positively refrain
from writing testimonials for any other cheese for the next three
months. My recommendation, gentlemen, is that our campaign be based on
the testimonials of Mickey and Minnie Mouse. When anybody says cheese,
what's the first thing you think of? Mice. Who's the world's most famous
mouse? Mickey Mouse. Gentlemen, it's never been done before, and it's a
natural. What do you think?

(Silence)

\emph{Mrs. Betts:} What do we need Mickey for? It's Minnie that runs the
kitchen, isn't it? Excuse me for a moment, please. I'll be right back.

(Silence)

\emph{Mr. Billings:} (Who has recently escaped from the copy desk of a
tabloid) Ha!

\emph{Mr. Butts:} Billings, will you stop that obscene cackle?

The stenographer's record became defective at this point. Eddie's memory
supplied the details. Harmsworth, Princeton, 1928, who had recently
graduated from the apprentice course of the agency, had also elected
that moment to be brought to bed with a big idea of some sort.
Harmsworth was typical of the class of Unhappy Rich Boys for whom
advertising agencies have been required increasingly to serve as dumping
grounds. He was the nephew of the chairman of the board of Planetary
Founders Corporation. It was rumored that on attaining his majority, he
had inherited three million dollars from his mother. He didn't have to
work. He played polo rather well, but not well enough to rate any great
distinction in his set. And being a serious minded youth with no vices
and no talents, it was necessary for him to have some occupation, some
r\^{o}le in life, to which he could refer in his conversations with Junior
League debutantes. Advertising, a romantic, more or less literary
profession, filled the bill admirably.

Harmsworth got in at nine o'clock every morning and frequently stayed
until six. With the other apprentices, he did his bit on research, which
meant days of hot and heavy footwork in the wilds of Queens and the
Oranges, ringing doorbells, and asking impertinent questions of stolidly
uncooperative housewives.

This was Harmsworth's first agency conference and his first Big Idea.
Its delivery was complicated by the fact that in moments of great
excitement, Harmsworth stuttered painfully.

\emph{Mr. Harmsworth:} C-c-can't we t-t-tie this c-campaign up to the
n-n-to the n-n-news? How about hooking it up with relativity? There's so
much f-f- so much food value in ch-ch-cheese. Relatively, you know. More
f-f food value than meat. More than eggs. Maybe we could g-g-g-g-maybe
we could get Einstein!

\emph{Mr. Billings} (who is frantically waving two fingers): Excuse me,
please.

\emph{Mr. Butts:} All right, Billings.

\emph{Mr. Harmsworth:} Of course, it may be a b-b- a bum hunch. I just
thought---

(Silence)

By this time the conference was pretty well mired. Something had to be
done, and as usual, Blashfield did it. Blashfield's salary was thirty
thousand dollars a year, plus his participation as a stockholder in the
agency's profits. Blashfield didn't think that was enough. Every day, in
every possible way, he proved it wasn't enough. Cruelly, sadistically,
he exposed the incompetence, the muddleheadedness of his associates. He
had a string of copy writers and layout men working under him, all of
whom hated him cordially. Their work was rarely used, except as a foil
to exhibit the superior brilliance of the agency's star copy-art-plan
performer. At the last moment, in a day or two days, he would knock out
the copy, rough layouts, plan and marketing strategy for a whole
campaign. Artists, printers, engravers, the mechanical production staff
of the agency, would be called upon to work nights and Sundays to
complete the job. Blashfield's overtime bills were notorious.

Then, with the plan memorandum snatched from the stenographer and
flanked by two or three subordinates carrying unwieldy art and other
exhibits, he would lope out of his office, pile into a taxi, and catch
the train for Baltimore just as it was moving out of the station. The
next morning he would lope back into the office, like a half-back
completing an end run, and deposit the okayed plan, copy, layout and
appropriation on Eddie Butts' desk.

Blashfield had done it again: \emph{his} plan, \emph{his} copy,
\emph{his} layouts, \emph{his} sale. Alone in Baltimore he had dazzled
the client with the coruscations of his wit, the machine gun rattle of
his logic, the facile improvisations of genius answering every objection
with pungent phrase or graphic line. O.K. Now Eddie, it's up to you to
follow it.

From sad experience, Eddie had learned what to do on such occasions. The
first thing to do was to take the train to Baltimore himself and pick up
the pieces. Eddie knew what he would find. He would find a group of
business men experiencing a perfectly dreadful morning after hangover,
and indulging in the usual orgy of remorse and mutual recrimination.

Blashfield had been, shone and conquered. Blashfield was a brilliant
fellow---an advertising genius. Sure, and they hoped to God they never
saw him again. Now about this damned contract they had signed....

Eddie was no genius. As an advertising man he was only mediocre. But as
a fixer he was an expert. Even so, he would be lucky if, after two weeks
of hard work, he emerged with a modified appropriation and a revised
campaign, in which some remnants of Blashfield's initial performance
might or might not be discernible. The campaign as carried out might be
better or worse than Blashfield's original. Usually it was worse, for
Blashfield's competence was genuine enough. But for better or worse it
was duly billed and commissioned, which was the sort of thing the
agency's treasurer was forever grousing about. So that Eddie Butts'
salary was thirty-five thousand dollars a year, a fact that forever
festered like a thorn in the Achilles' heel of the agency genius.

Because of the repetition of such experience, the heads of the agency
had increasingly restricted Blashfield's pyrotechnics to the home
grounds, where he could be carefully watched and protected against
himself. No let-up of the Blashfield drive had resulted, but his hobbled
ego required more and more bloody human sacrifices. His performance at
the Primrose Cheese conference had been sanguinary in the extreme.

Beginning suavely, he had made some incisive remarks about the standards
of agency practice, the nature and purpose of agency conferences.
Abruptly he swung into a disquisition on the natural history and
personal habits of mice; mice that live in old houses but are never
housebroken; old mice, young mice; the love life of the mouse; mother
mice and their pink and squirming progeny; mice and elephants, and the
tactlessness of both as dinner guests; mice that creep out from under
sinks and leer up at horrified housewives; (at this point Mrs. Betts
lifted her skirts and barely suppressed a shriek.) Mice and cheese. The
kind of cheese mice eat, and the obscene sounds they make while eating;
the dumbness of mice and the dumbness of men.

By this time old Hanson was purple with rage. But before he could
interrupt Blashfield, whom the stenographer had given up trying to
follow, was well launched upon a burlesque of relativity, which rapidly
took form as a convention of mouse domestic science experts, presided
over by Minnie Mouse, and discussing the relative dietetic merits of
meat, cheese, caviar, etc. Even Harmsworth laughed, partly to cover his
confusion.

Then abruptly the wizard's mood changed. Come on fellows. Let's be
serious. What's the best way to sell cheese? Primrose Cheese?

With rapid logic he outlined the campaign that could, should and must be
conducted. The consumption of cheese in America was negligible compared
to its consumption in France, England, Germany, Switzerland---throughout
the world. The dietetic habits of America must be transformed. An
institutional campaign, then? No, a selling campaign, hard-boiled
selling copy that would boost the sales of Primrose Cheese from week to
week and from month to month. But the copy would be educational too. It
would show the things that Americans do eat and drink, and dovetail
cheese into the menu; Primrose Cheese for the cocktail party. Cheese for
dessert the continental idea. That's what all the best people are doing
and the rest of America must be shamed into imitating the Best People.
Style. Style in the copy. Style in the art. Jean Mazarin for the
art---he'll be in New York in two weeks and he'll love it.

Now, as to the trademark that some of you have been worrying about. What
is it? A primrose, crossed with a key. It looks a little like a
swastika, and a little like a Jewish candlestick. But look at it now.

Blashfield executed a few swift strokes on his sketching pad.

There's your solution. It's still a little like a swastika, and all the
patriotic Germans will notice it. It's also a little like a Jewish
candlestick, and all the Jews will notice that. But a second look will
convince anybody that it's neither one nor the other---and that's just
fine for everybody.

As usual, Blashfield had swept all before him. The conference broke up
after an assignment of preliminary tasks, all to be executed under his
supervision. The other Big Ideas, of course, were never removed from the
appropriate receptacle into which Blashfield, with surgical dispatch,
had consigned them.

Harmsworth had played polo all the next week, and when he returned was
assigned to a bank account. Hanson had groused for a while. His first
idea in twenty years. And on investigation it proved not to be his idea
after all. It was his secretary's idea, and for several weeks thereafter
the gossip of the women's room was enlivened by the lady's complaints
about how hard it was for a girl to get ahead in a big agency.

The campaign had consumed the time of eight or ten people for three
months. In the end, Blashfield had scrapped their efforts and done the
whole job himself in a last minute orgy of nerve-racking and expensive
nightwork by all and sundry.

Eddie Butts winced as he read a memorandum from the Treasurer,
protesting against so huge a bill for preliminary work on what was after
all, not a major account.

Well, there it was. And now if Bob Niemyer's steer was right, there
would be hell to pay tomorrow.

Eddie sighed, pushed his dictaphone into the corner, and helped himself
to a shot of the house liquor.


\subsection{2. THE FIRE ALARM}\label{the-fire-alarm}}

It was close to midnight, and Eddie Butts was in the middle of his third
pipe before Bob Niemyer, the space salesman, and his German friend,
stumbled through the darkened outer office and banged on his door.

They were not drunk; merely very formal and very, very earnest.

``Eddie, meet my friend Oscar Schleiermacher... Thanks, I guess I can
stand another... Eddie, I'm afraid this is serious. Oscar knows what
he's talking about, and he tells me that the big shot of the Primrose
Cheese Company, Hakenschmidt---

``Himmelschlussel, August B. Himmelschussel,'' prompted Oscar.

``All right, Himmelschlussel. Well, as I was saying, I was telling Oscar
about the swell presentation you'd worked up for Primrose
Cheese---naturally he wants a piece of it for his friends on the
Vortschrift---and when I got to the big idea, cheese and beer, cheese
and cigarettes, cheese for the cocktail party, why I'm telling you Oscar
almost passed out. Didn't you, Oscar?''

Oscar made an eloquent gesture, hitched his chair forward, and drained a
large glass of Scotch at a swallow.

``You see, Eddie, this bird Himmelschlussel runs his own business. And
how! He's got the o.k. on everything, see? What he says goes. And what
he's going to say when he sees this campaign of yours won't even be
funny.''

Mr. Schleiermacher nodded solemnly.

``Er ist ein Herrenhuter [\emph{sic}]. Sein Frau auch.''

``There,'' said Bob. ``What did I tell you? He's a Herrenhuter. What's a
Herrenhuter? That's what you're going to find out when old
Himmelschlussel gets an eyeful of that French night club art moderne
Blashfield has cooked up for him. A Herrenhuter is a Fundamentalist,
only worse. Let's be serious, Eddie. This Himmelschlussel is religious
as all hell. He's a prohibitionist. Some of his coin goes to the
Anti-Saloon League. What's more, Mrs. Himmelschlussel is one of the big
shots in the Anti-Cigarette League. Nobody that works for Primrose
Cheese can drink, smoke or forget to say his prayers. Isn't that right,
Oscar?''

``Ach, ja,'' said Oscar. ``Er ist ein Herrenhuter. Sein Frau auch.''

``His wife too,'' said Mr. Niemyer. ``So when Oscar gives me the
lowdown, I says to him: `Eddie Butts has got to know about this. Eddie
Butts is a friend of mine. Eddie and I are just like this'. Y' get me,
Eddie? What makes it worse, this Himmelschlussel has a bad case of shell
shock on advertising anyway. Ain't that right, Oscar?

``Schrecklich,'' confirmed Oscar with an expansive gesture.

``The story goes like this,'' continued Mr. Niemyer. ``The local team of
the League wins the pennant, see? And Himmelschlussel, he's a fan. Sure,
baseball, that's his only vice. It seems he has a nephew playing
shortstop on the team. That was eight years ago. Well, Old
Himmelschlussel, he's the proud uncle, and he's got to do something
about it, see? So what does he do? A big dinner for the team, see? Hell
with expense. Sauerbraten, Kartofelkloss, leberknudel, hasenpfeffer [\emph{sic}], the
whole works. No beer, no hard liquor. No cigarettes. Cheese. Boy, was
there cheese! Big camembert in the middle of the table. Four feet high,
weighs eighty pounds. Mottoes. Clock works. Imitation dugout. Birdie
pops out of dugout. Cuckoo, cuckoo, cuckoo---counts the score, see?
Fine. Swell. Cost a lot of money. Only thing is, you know camembert.
Eighty pounds of camembert. Ripe. Not so good. And those bush leaguers
thirsty as camels, and no beer. So they get tough. Bean the birdie with
pop bottles. Raise hell, see? That's bad enough, but next day the papers
get funny. Himmelschlussel don't advertise, see? They keep it up for
days. Himmelschlussel sore. Feelings hurt. You tell him, Oscar. ``Were
his feelings hurt?

``Vom herz, Herr Butts. Vom herz. Ach, schrecklich.'' Oscar held his
head and rocked in remembered sympathy.

``So Himmelschlussel goes Herrenhuter again, worse than ever. Ten
thousand simoleons that year to the Anti-Saloon League. And no more
advertising stunts. That contract of yours---how his sales manager got
that out of the old man I just can't imagine, unless they're in
trouble... What's that, Eddie. Don't want to rub it in. Just trying to
do you a favor, see? You and me are pals. As I says to Oscar, I
says---what d'you say, Eddie?''

``I said, Jesus H. Christ!''

Eddie Butts wasn't listening. The fire alarm had rung. He was busy
hunting numbers in the office telephone directory. Blashfield first.
Damn Blashfield. Damn Hanson. Why hadn't they found out about this big
shot?

``Thanks, Bob,'' said Eddie, as he led his visitors to the elevator.
``I'll let you know what happens. We got a day and a half. Maybe we can
pull out. Good-night. Good-night, Mr. Schleiermacher, and thanks for the
steer.''


\subsection{3. RESCUE PARTY}\label{rescue-party}}

After hours. The genius of advertising burns brightest after hours. When
the noise of traffic is stilled, when the stream of office time-servers
has flowed north into the Bronx, east and west under and over the rivers
to be blotted up by the vast and formless spaces of Long Island and
Jersey, light still lingers in the sky-scrapers of the mid-town
district.

Light and vision. Not money alone could buy the devotion of these
weary-eyed night workers. It is something else, something strange,
incredible, miraculous---perhaps a little mad. Is it for beauty that
they burn themselves? For truth? For some great cause? No, it is none of
these. It is like a perverse and blinding discharge of human
electricity, like athletes battling on the gridiron, or soldiers going
over the top.

In the Sargasso pool of quiet, high above the night-stricken city, what
toils, what genuine heart-breaks, what farcical triumphs are
consummated!

From the moment that Eddie Butts turned in the fire alarm, the wheels of
the Kidd, Kirby \& Dougherty agency never stopped turning. Blashfield
swooped in from Westchester, worked all night, and when his secretary
came in the next morning, turned over a basketful of new copy for
typing. Eddie Butts' dictaphone whirred continuously. Tense voices
barked into telephones. Printers, appalled by impossible demands, wailed
in anguish, achieved the impossible, and viciously pyramided the
overtime charges. Layout men never left their drawing boards. Typists
worked in relays. What had taken three months to do must be done again,
but this time in thirty-six hours.

It was done. Miraculously, it was done. Blashfield again. Blashfield the
magnificent. Never was the man so dangerous as when, with his back
against the wall, he was challenged by the impossible. A new Big Idea
had been conceived and was well on the way to birth before he reached
the office. Cheese and pie. New England stuff. Native American. Simple,
homey. The New England grandma. The Southern mammy. To hell with
Mazarin. Tell him, sorry, pay his bill or part of it, and charge it up
to profit and loss. Forsythe is our man. Forsythe, the best buck-eye
artist in America. He's busy? What of it? I said, get him.

Forsythe performed. Blashfield performed. Clerks, messengers,
typists---everybody performed.

By noon of the scheduled day for the presentation the miracle was
accomplished. Or almost. Typewriters still rattled and savage-lipped
production clerks still yapped into the telephone. One o'clock. No lunch
for anybody. Two o'clock, and the final pages of the revised plan were
bound into the portfolio. Three o'clock, and Himmelschlussel was
expected. Three-fifteen, and no Himmelschlussel. Had something gone
wrong?

Only Colonel Kidd himself---Calvin Kidd, author, editor and advertising
man---only Colonel Kidd remained calm. Back of his desk a framed motto
proclaimed the solid premise on which his professional imperturbability
was based: ``There is somebody wiser than anybody. That somebody is
everybody.'' It doesn't make sense, does it? Sure, that's just the
point. Calvin Kidd was a mystic. He remained calm. But his associates,
some of whom may have felt that their jobs were at stake, were less
philosophic. At the telephone switchboard, the battery of skilled
operators grew querulous striving to release the tide of out-going
calls. Himmelschlussel. Himmelschlussel! Where in hell is
Himmelschlussel?


\subsection{4. THE DELIVERY}\label{the-delivery}}

It wasn't Dorothy's fault. Afterwards, since it didn't matter---anyway
nothing mattered---everybody acknowledged that you couldn't fairly pin
it on Dorothy.

Dorothy was the reception clerk, stationed in the lobby of the offices
of Kidd, Kirby \& Dougherty, with a pad of forms before her and a
telephone receiver clasped over her lovely blonde hair. Dorothy knew her
r\^ole, which was to make quick and accurate judgments and translate them
into action.

So that when the little old man with the umbrella stepped out of the
elevator, she knew instantly what to do. The Primrose Cheese account was
in a jam. A messenger was expected from the printer, bringing revised
proofs. She had been warned to rush him through without delay to Mac in
the mechanical production department. Dorothy spotted him instantly and
beckoned him to the desk. The little old man advanced somewhat
diffidently.

``I am Mr. Himmelschlussel. I---

``From Hazenfuss, yes. You're just in time. Go right through the side
door and ask for Mac.''

Hazenfuss Brothers was the printing shop which at the stern behest of
Blashfield had performed the current typographical miracle.

The little old man hesitated, but Dorothy, gracious but imperative,
motioned him to the side door.

He vanished into a welter of comptometers, typewriters and proof
presses. Dorothy had just an instant to reflect that she hadn't seen
this particular messenger before. Also, wasn't it Hazenfuss that dolled
up their messengers in naval uniforms, so that they all looked like
musical comedy Commodores? This must be a new one. Come to think of it,
he did wear a kind of uniform, too---certainly was a funny old geezer.
Maybe Hazenfuss had thought up a new advertising dodge.

Meanwhile, Mr. Himmelschlussel was still trying to find Mac.
Successively, he was shunted to the shipping room, to the store room
clerk, to the purchasing clerk. Early in the ordeal, Mr. Himmelschlussel
began to lose things. First he lost his umbrella. Then he lost his hat.
Coincidentally with this second disaster, he completely lost his
English.

Alarmed by the clamor of what he took to be a minor riot in the
mechanical production department, Pfeiffer, the office manager, emerged
from his cubicle to see an elderly German-American gesticulating wildly
in the middle of a circle of bewildered clerks. At intervals, his gray
pompadour bristling, he would make a determined break for one of the
innumerable doors, only to be hauled back by an expostulating clerk.

Fortunately, Pfeiffer spoke German, for by this time Mr. Himmelschlussel
could speak nothing else....

When the perspiring Pfeiffer finally persuaded the long awaited client
to permit himself to be led into the presence of Colonel Kidd himself, a
strange quiet had descended upon the agency. Mr. Himmelschlussel himself
was quiet. He would speak neither English nor German. In response to
Colonel Kidd's urbanities he merely grunted. Blashfield's irresistible
wisecracks died unborn upon the desolate air.

Silently, the procession wended to the conference room. In silence, Mr.
Himmelschlussel listened to the reading of the plan. Upon the lavish
exhibit of layouts, charts, proofs, etc., he turned a cold Prussian eye.
Silence.

At last, Mr. Himmelschlussel spoke.

``Gentlemen, I haf joost come from de bank. Business is bad. We haf an
offer from de Universal Foods Corporation to buy Primrose Cheese. It is
a good offer. It is a very good offer. We have accepted that offer.

``Dese''---he gestured indifferently at the decorated walls of the
conference room---``dese iss very pooty pictures. De Universal Foods
people, maybe dey like to look at dem. I am sorry. I got to go now. My
wife and I, we have friends in Brooklyn. Good day, gentlemen.''

In the far corner of the lobby an elderly woman was waiting. She had
been waiting a long time. Dorothy thought she was perhaps a cleaning
woman, or the mother of one of the shipping room boys. She said nothing
and politely resisted Dorothy's gracious solicitudes. She had the corner
to herself now, and Dorothy noticed that the space salesmen had put out
their cigarettes.

Eventually Mr. Himmelschlussel emerged, escorted by Colonel Kidd. She
put her hand under his arm. They got into the elevator. They went to
Brooklyn...

Again that evening Eddie Butts worked late. He was tired, very tired. He
had missed lunch entirely and it was after seven. Eddie was hungry.
There, on the corner of the desk, was a left-over sample. Cheese.
Primrose Cheese.

Holding the package at arm's length, Eddie went to the open window. It
took a long time falling. You couldn't hear it strike, but you could
just barely see the yellow splotch it made on the pavement.

Eddie lingered at the window. Thirty-two stories. Every now and then an
advertising man jumps out of one of those high windows in the Grand
Central district. Usually, it is the follow-up man, the old reliable.
Usually, it is Eddie Butts.



% CHAPTER FIVE
\chapter[5 \hspace*{1mm} AS ADVERTISED: The Product of Advertising]{5 AS ADVERTISED: The Product of Advertising}
\chaptermark{5 AS ADVERTISED}

\newthought{The} foregoing fictionized account of what happens in a large advertising
agency will doubtless strike the lay reader as exaggerated. It will be
denounced, more or less sincerely, by advertising men who have lived and
toiled so long on the other side of the Advertising Looking Glass that
the barbarous farce-as-usual of advertising practice has become for them
the only reality, the only ``sanity'' with which their minds are
equipped to deal.

The account is nevertheless true in every essential respect. The fiction
is no stranger than many of the sober facts set forth elsewhere in this
volume.

We have now to consider what sort of product this advertising mill turns
out. Again, the writer's inclusions may seem at first thought too
sweeping.

The advertisement itself is the least significant part of this product.
The advertisement is an instrument, a tool, and the ad-man is a
toolmaker. In using these tools the newspapers, magazines and radio
broadcasters become something other than what they are commonly supposed
to be; that is one result. By operating as they must operate, not as
they are supposed to operate, these major instruments of social
communication in turn manufacture products, and these products are the
true end products of the advertising industry.

The most significant product, or result, is the effective dissolution of
practically all local or regional, autonomous or semi-autonomous
cultures based economically on functional processes of production and
exchange and culturally on the ethical, moral and aesthetic content of
such processes. The advertising-manufactured substitute for these
organic cultures is a national, standardized, more or less automatic
mechanism, galvanized chiefly by pecuniary motivations and applying
emulative pressures to all classes of the population.

In England, where the organic culture was older, richer and more
resistant, publicists and educators are more keenly aware of the
significance and potency of advertising, although there the business is
still relatively embryonic, lacking either the scale or the intensity of
the American phenomenon. \emph{Culture and Environment}, by F. R. Leavis
and Denys Thompson, best exhibits the 1933 English awareness of what is
happening, and this excellent book, representing the collaboration of a
literary critic and a schoolmaster will be referred to again in later
chapters.\footnote{{[}F. R. Leavis and Denys Thompson,
  \emph{\href{http://www.worldcat.org/oclc/188346}{Culture and
  Environment: The Training of Critical Awareness}} (London: Chatto \&
  Windus, 1933).{]}} Among English creative writers, D. H. Lawrence seems to
have grasped intuitively almost from the beginning, the nature and
causes of the disintegrative process.

In America, the most impressive testimony, both conscious and
unconscious, to the progressive disintegration of the organic American
culture is contained in the work of Sherwood Anderson. Anderson grew up
in a small Middle Western town during the period when the organic
relation between agriculture and small town craft-industry was being
shattered by the emergent forces of mass production, mass distribution,
and by the pseudoculture which the rapidly expanding apparatus of
advertising manufacture as a mechanical substitute for what it
destroyed. First as a manufacturer and later as an advertising man,
Anderson participated unwillingly in this dual process of destruction
and substitution.

This experience, in the view of the writer, provides the essential clue
to an understanding of Anderson's verse, short stories and novels. Much
of the brilliant early work was written on the marginal time of an
advertising copy writer employed by a large Chicago agency. It has a
single theme: the passionate rejection of the ad-man's pseudoculture and
the nostalgic search for the organic culture that was already dead or
dying. Anderson saw that the disintegration and sterilization of the
culture is reflected in the fragmentation and neutering of the
individual. In novel after novel, story after story, we see him
separating the quick from the dead and driving first backward, then
forward, into some terrain more habitable for the human spirit.

The reader will perhaps have been struck by the inhuman, hysterical,
phantasmagoric quality of advertising agency practice as described in
the preceding chapter. This is inevitable. The prime mover of the
advertising mill, the drive for profits, has no concern whatever for
human life. Without organic life itself, the advertising mill is fueled
by the organic cultural life which it disintegrates and consumes, but
does not restore or replace. On cultural as well as on economic grounds
it may be said that this organic social heritage is not inexhaustible.
Hence the advertising mill not only disintegrates and destroys all the
humanity that comes within the sphere of its influence but is
ultimately, like the modern capitalist economy of which it is a part,
self-destructive.

One sees this advertising mill as a coldly whirring turbine whose hum is
so loud, so continuous, so omnipresent that we no longer hear it. Its
force is centrifugal: all warm human life is expelled into the
peripheral darkness where it continues to revolve although the machine
can no longer use this nebula of burned-out dead and dying matter.

At the heart of the machine we see dim figures moving: the sort of
people whom the writer has tried to make real and credible in the
preceding chapter. They rush here and there, fiddling with levers,
filling the grease cups.... They are dead men. Against the blue light
their hands are lifted in queer, stylized gestures. They speak, but what
they say is without human meaning. It is the machine speaking through
them and the sound comes to us like the sound of a phonograph playing a
cracked record, hugely and hoarsely amplified. The lips of the robots
move and we hear: ... ``Advertising is the new world force lustily
breeding progress. It is the clarion note of business principle. It is
the bugle call to prosperity. But great force as it is, advertising must
seek all aid from literature and art in order that it may assume that
dignity which is its rightful heritage. Advertising is ... oom-pah!
oom-pah! Under the New Deal good advertising will become more essential
than ever. It will be in a position to help the business executive to
avoid those wasteful and excessive practices in selling which so often
add needless costs to needed products. Good advertising is opposed to
senseless price cutting and to unfair competition. Constructive sell ...
oom-pah! oom-pah! No sales policy is permanently beneficial that has its
roots in deception ... oom-pah! oompah! It costs a lot of money when a
community is to be attacked ... oom-pah! oom-pah! Remember that while a
shot-gun makes a lot more noise than a rifle it just messes things up.
Aim the rifle well and you get a nice clean hole ... oom-pah! oom-pah!
The most popular dinner guest in Jerusalem ... oom-pah! oom-pah! Every
occupation has its special satisfactions. The architect and the builder
see the product of their planning take shape in steel and stone. The
surgeon snatches life from the jaws of death. The teacher and the
minister give conviction and power to the things that are unseen. Our
calling is not less significant. We build of imperishable materials, we
who work with words.... All things perish, but the word remains ...
oom-pah! oompah! oom-pah! oom-pah! oom-pah! ...''

They are dead men. Their bones are bakelite. Their blood is water, their
flesh is pallid---yes, prick them and they do not bleed. Their eyes are
veiled and sad or staring and a little mad. From them comes an acrid
odor---they do not notice it, it may be only the ozone discharge of the
machine itself. When you ask them to tell you what they are doing, they
do not know, or at least they cannot tell you. They are voiceless,
indeed, self-less only the machine speaks through them.... Dead men tell
no tales.
\clearpage
Most are like that. But here and there among those dim wraiths is one
who still keeps some semblance of life. An artist, or perhaps one who
would have been a scholar or a scientist but that he has suffered the
spleen of an ill fate. Art and science are strong passions. Most of
these exceptional ones become in time like the others. But they are the
stronger spirits and now and then one of them escapes. They do not like
to talk of what they have seen and done there at the heart of the
machine. They like to pretend that it never happened; that it was a kind
of nightmare, as indeed it was. But when tales are told it is they who
tell them. From time to time Sherwood Anderson has told such tales.
Recently he has begun to tell more of them. They are quite horrible
tales. Artists find it difficult to use this material. The advertising
business is harder to write about than the war. It would perhaps bring
some of the dead back to life if more of such tales were told.

But the machine tenders are not the only dead. Great waves of force
shudder outward from the machine, and more and more this cold electric
force substitutes for the life-force of the people whom the waves
surround and penetrate. They too seem to lose the color and movement of
natural human life. They twitch with little fears and itch with little
greeds. They become nervous, jittery, mechanical. They can no longer
weep with spontaneous tears or rock with spontaneous laughter. They too
become in a sense self-less so that one cannot expect them to be true to
themselves or true to others. The waves which increasingly substitute
for their flagging organic will-to-live---the waves have indeed not
heard of this truth. For the prime mover from which the waves come is
beyond good and evil, truth and untruth, and the waves are everywhere.
They speak, these creatures, their lips move, but again it is the
machine speaking through them:

... ``He invented the foods shot from guns at the skin you love to touch
but your best friends won't tell you for three out of five are facing
calendar fear another day of suspense learn to be charming the smart
point of view without cost grandpa said I'll let you know my health to
Quaker Oats I owe upon my face came long ago the smile that won't come
off for skin eruptions need not worry you guard your dresses spare your
friends perspiration may cost you both who'd believe they called me
skinny 4 months ago I should think she'd notice it herself in closeups
you can trust Blick's Velvasheen a better mouthwash at a big saving
isn't it wonderful how Mary Ellen won the \$ 5,000 beauty contest and
Mrs. Jones wins her husband back at the foot of my baby's crib I made a
solemn promise the girl of his dreams but she almost lost him in a month
she didn't have a trace of constipation reports Dr. David of Paris what
color nails at Newport all shades I'll lose my job if this keeps up
can't make a sale can't even get people to see me I'd better ask the
sales manager what's holding me back couldn't take on that man you just
sent me seemed competent but careless about B. O. what a fool she is
takes pains washing a sweater gives no care to her teeth and gums and
she has pink toothbrush Mae West and the big hat she wore in ``She Done
Him Wrong'' who will be the first to wear it in Chicago if Mona Lisa
could have used these 4 Rosaleen eye beauty aids let's take a look at
the record toasting frees Lucky Strike cigarettes from throat irritation
William T. Tilden II steady smokers turn to Camels William T. Tilden II
did you hear the French nation decorates Campbell's soup chef for
sending the finest cooking throughout the civilized world Yeow! let's
run away to sea travel has its niceties....''

This sub-human or un-human jabberwocky saturates the terrestrial
atmosphere. It pours out of hundreds of thousands of loud speakers from
eight o'clock in the morning until midnight. Doubtless the biologists
will shortly inform us that this transformation of the auditory
environment has caused definite degeneration and malformation of the
average American ear. Certainly the eyes must have been affected, for
the same jabberwocky in print glares from the pages of billions of
copies of magazines and newspapers and other billions of posters,
carcards and mail communications. Is it any wonder that the American
population tends increasingly to speak, think, feel in terms of this
jabberwocky? That the stimuli of art, science, religion are
progressively expelled to the periphery of American life to become
marginal values, cultivated by marginal people on marginal time? That
these marginal people are prevented from exercising their proper and
necessary social functions except by permission of the jabberwock? That
many of them indeed compromise fatally with the creature and translate
what they have to say into its obscene jabberwocky?

Let us not forget that the jabberwock feeds on what it destroys and that
it restores and replaces nothing. It is fueled by the organic
will-to-live of the population, which it calls ``buying power.'' This
buying power is progressively exhausted---advertising as Veblen pointed
out, is a form of sabotage on production---just as our inorganic
resources of coal, oil and minerals are progressively exhausted. After
four depression years the jabberwock is hungry. It has devoured large
sections of the lower and lower middle classes and expelled their dry
bones, burned clean of their buying power, into the outer darkness.
There the electric breath of the jabberwock still plays on them, but
they are ash and slag. They cannot burn, they cannot feed the machine.
Fifteen million of them are dependent upon relief. Another thirty
million are so lean that they can fuel the jabberwock scarcely at all.
You see them dumped like mail sacks on park benches. You see them
fluttering like autumn leaves, magnetized into thin wavering lines job
lines, bread lines. They sit in chilly rooms listening as before to the
voice of the jabberwock, unwilling to believe that they have been
consumed, discarded. The waves still pulsate and the ash of the great
radio audience still glows a little---there is so little other food.
What is the jabberwock saying now? ... ``I will share.... Don't sell
America short.... Forward, America....''



% CHAPTER SIX
\chapter[6 \hspace*{1mm} THE MAGAZINES]{6 THE MAGAZINES}

\section{I. The Command to Buy}

\newthought{``FORWARD} America''; ``I have shared''; ``We do our part.''

The depression slogans of both the Hoover and the Roosevelt
administrations seem to imply a national unity, a culture. The people
are to be ``sold'' on this culture as a part of the task of
rehabilitating it. It is therefore proper to examine the content of this
culture, slightly down at the heels, as it is, in this fifth year of the
depression.

For this purpose the evidence provided by the editorial, article,
fictional and advertising contents of the contemporary mass and class
magazines is extraordinarily revealing. We have seen that the press,
including the magazine press, is used as an instrument of rule. The
rulers are the manufacturers, advertisers, distributors, financiers,
etc., who use not merely the magazine advertisements but the total
apparatus of this periodical press to enforce ``the command to buy.''
This rule is exercised both by direct injunction to buy and by the
promotion and stimulus of emulative and snob motivations, which in our
society must be largely satisfied through the purchase and display of
things.

With the motivations and technique of this rule clearly in mind, we
should expect to find a treatment of sex, economics, morals, philosophy,
science, etc.---designed to nourish and stimulate the buying motif. We
find all of this and more. We find what amounts to a conspiracy of
silence regarding all those aspects of the individual and social life
that do not contribute to the objective of the advertiser, which is
practically identical with that of the magazine itself. That objective
is to promote sales and to extend, complicate and consolidate sheer
emulative materialism as a way of life. We venture to say that no one
who has not attentively examined these magazines inch by inch can
conceive the astounding, sterile vacuity of these enormously expensive
and enormously read ``culture-bearers.''

The question that immediately arises is: do these magazines accurately
reflect the culture or are they merely trying to inflict a pseudoculture
on their readers? In a curious way both things are true. It would seem
that both the culture as lived and the culture as reflected by the
magazines are pseudocultures. Neither in life nor even in the
make-believe of the magazine fictioneer does this pseudoculture satisfy
anybody. It does not even satisfy the wealthy, who can afford to live
according to the snob, acquisitive, emulative pattern. The reductio ad
absurdum of the theory of a self-sufficient acquisitive culture is found
in \emph{Arts \& Decoration} which bullies and cajoles the rich into the
discharge of their function as the ideal human representatives of a
culture which has no content or meaning outside of acquisition and
display. In arguing for this way of life a writer in \emph{Arts \&
Decoration} is reduced to the following remarkable bit of philosophic
yea-saying: ``Chromium is more expensive than no chromium.''

These magazines are designed and edited with a view to making the
readers content with this acquisitive culture, but even a commercial
fictioneer has to put up a human ``front.'' He has to use models. He has
to exhibit, however superficially and shabbily the kind of people who
work in American offices and factories and on farms, and who walk the
streets of American cities and towns. In so doing he inadvertently and
inevitably gives the whole show away. He proves that these robots
galvanized by pure emulation are fragile puppets of glass. Mostly the
characters are faked. When they are at all convincing they are
definitely dissatisfied and unhappy.

This pseudoculture which is both reflected and promoted by the magazines
is evidently in a process of conflict and change. In fact it may be said
that there are two cultures: the older, more organic American culture,
and the new, hard, arid culture of acquisitive emulation pure and
simple. These cultures are in perpetual conflict. The emulative culture
is what the magazine lives by; the older more human culture is what the
reader wistfully desires. However, the magazines can afford to give the
reader only a modicum of these warm humanities.

The problem of the editor is essentially similar to that of the
advertising copy writer. The purpose of the advertisement is to produce
consumers by suitable devices of cajolement and psychological
manipulation, in which truth is used only in so far as it is profitable
to use truth. But the advertisement must be plausible. It must not
destroy the reader-confidence which the copy writer is exploiting.

In the same way the magazine editor may be thought of as producing, in
the total editorial and fiction content of the magazine, a kind of
advertisement. In this view the advertisement---say in issue of
\emph{The Woman's Home Companion}---must have some human plausibility;
it must contain some truth, some reality, otherwise the magazine would
lose circulation, i.e., reader-confidence. But the editor must never
forget that the serious business of the magazine is the production of
customers just as the writer of the individual advertisement must not
use either more or less truth and decency than will produce a maximum of
sales for his client.

We examined single issues of thirteen representative and large
circulation magazines in an attempt to determine the following facts:

\begin{enumerate}
\item
  Does the magazine promote buying, not only in the advertisements, but
  in the editorial, article, feature and fiction section of the
  magazine?
\item
  To what extent do the magazines permit criticism of the acquisitive
  culture?
\item
  Since literature, even popular literature, is supposed to reflect a
  culture, what kind of a culture, judged by the contents of these
  thirteen magazines, have we got?
\end{enumerate}

The thirteen magazines were chosen with the idea of having as many
different types of magazines represented as possible. The attempt was
also made to select magazines going to readers who belong to different
income classes. Eight of the magazines analyzed have over one million
circulation, and constitute over a third of the twenty-one magazines in
the United States having circulations of this size. The list of
magazines studied is as follows:

\vspace{5mm}

\begin{center}MAGAZINE STUDY\footnote{American Weekly, issue of Jan. 7, 1934; True Story, Dec. 1933;
  Household, Nov. 1933; Liberty, Dec. 23, 1933; Photoplay, Jan. 1934;
  American Magazine, Dec. 1933; Woman's Home Companion, Jan. 1934;
  Cosmopolitan, Dec. 1933; Saturday Evening Post, Dec. 16, 1933;
  Harper's Bazaar, Dec. 1933; Harper's Magazine, Jan. 1934; Nation's
  Business, Nov. 1933; Arts \& Decoration, Nov. 1933. Publisher's estimate}\end{center}
\tabulinesep=1.1mm
{\begin{longtabu} to 1.05\textwidth { X[2,l] X[1,r] X[1,l] X[3,l] } 
\emph{Name of Magazine} & \emph{Circulation} & \emph{Income Level} & \emph{Type} \\ [1.5ex]
\endfirsthead
\emph{Name of Magazine} & \emph{Circulation} & \emph{Income Level} & \emph{Type} \\ [1.5ex]
\endhead
American Weekly & 5,581,000 & Low & Illustrated Hearst Sunday supplement. \\
True Story & 1,597,000 & Low & Confession magazine. \\
Household & 1,664,000 & Low & Woman's magazine; rural type. \\
Liberty & 1,378,000 & Medium & White-collar class. \\
Photoplay & 518,000 & Medium & Largest circulation movie magazine. \\
American Magazine & 2,162,000 & Medium & Small town, small-city magazine. \\
Woman's Home Companion & 2,235,000 & Medium & Woman's magazine: urban type. \\
Cosmopolitan & 1,636,000 & Medium & Urban magazine: much sex fiction. \\
Saturday Evening Post & 2,295,000 & Medium & Greatest advertising medium in the world. \\
Harper's Bazaar & 100,000 & High & High style fashions. \\
Harper's Magazine & 111,000 & High & High-brow and sophisticated. \\
Nation's Business & 214,000 & High & Organ of the Chamber of Commerce of the U.S. \\
Arts \& Decoration & 23,000 & High & Interior decoration for the rich. \\
\end{longtabu}}


\subsection{Findings:}


Our analysis shows that buying is promoted not only in the
advertisements but in the fiction, articles, features, and editorials. A
\emph{Woman's Home Companion} story mentions a Rolls-Royce eighteen
times. \emph{Harper's Bazaar} gives free publicity in its article
section to 532 stores and products. The snob appeal, essentially a
buying appeal, since successful snobbism depends in the main on the
possession of things, appears in 68 per cent of the subject matter of
one magazine. To summarize: We find when the percentages for the
thirteen magazines are averaged, that 30 per cent of the total space of
the magazines is devoted to advertisements, and 13 per cent is devoted
to editorial promotion of buying. Hence 43 per cent of the space in
these magazines is devoted to commercial advertisements, and what may be
called editorial advertisements, combined. We find also that snobbism is
a major or minor appeal in 22 per cent of the subject matter of the
magazines.

There is a very striking correlation between the amount of space devoted
to promoting buying and the amount of space devoted to criticism of the
acquisitive culture. The more space a magazine devotes to promoting
buying the less space it devotes to instruction, comment or criticism
concerning economic and political affairs. Four of the thirteen
magazines do not mention depression or recovery at all. Only two
magazines, \emph{True Story} and \emph{Liberty}, question the
desirability of the capitalist economy. Only two magazines, the
\emph{American} and \emph{Nation's Business}, question whether it can be
permanently maintained. In summary we find that: (1) No criticism of
business appears in any editorial. (2) Some criticism of the acquisitive
culture appears in the fiction. (3) Most of the criticism of existing
conditions appears in articles and readers' letters. (4) The thirteen
magazines devote, on the average, 24 per cent of their editorial and
article space to supplying the reader with information about economics,
politics, and international affairs. (5) The women's magazines, which
rank highest among the thirteen magazines in respect to the editorial
promotion of buying, rank very low in regard to comment on economics,
politics, and international affairs. They devote, on the average, 27 per
cent of their space to editorial promotion of buying, and only 5 per
cent of their space to comment on affairs.

The following conclusions about the culture reflected in these magazines
may be drawn:

(1) This culture displays a surplus of snobbism, and a deficiency of
interest in sex, economics, politics, religion, art, and science.

(2) The United States does not have one homogeneous culture; it has
class cultures. Summarizing the findings of this study in relation to
class cultures, one may say that the culture of the poor shows a strong
bias in the direction of fear and sex, that the culture of the
middle-class displays less sense of reality than the culture of the poor
or the rich, and a higher degree of sexual frigidity, and that the
culture of the rich tends to be emulative and mercenary.

An analysis of 58 fiction heroines in 45 sex fiction stories in the ten
magazines containing fiction shows the following differences between the
heroines who appear in the magazines of the poor, the middle class, and
the rich. In the magazines of the rich, 5 per cent of the heroines are
mercenary. In the magazines of the middle class, 56 per cent of the
heroines are unawakened or unresponsive women. In the magazines of the
poor, 45 per cent of the women can be classified as being sexually
responsive. The number of babies appertaining to these fiction heroines
also throws interesting light on our class cultures. In magazine fiction
as in life the poor women have the largest number of babies. While the
41 fiction heroines of the middle-class magazines produce only three
children, the eleven fiction heroines of the magazines of the poor
produce nine.

Further distinctions between the classes appear in the statistics on
emulation. Emulation is the dominant appeal in the ads of six magazines
which go to readers on the upper income levels. In the remaining seven
magazines---the magazines of the lower income levels---fear is the
dominant appeal. Emulation is also much stronger in the fiction and
subject matter of the magazines of the upper income levels; it is, in
fact, almost twice as strong as in the magazines of the poor. In the
lower income group magazines, 17 per cent of the subject matter has
emulation as a major or minor appeal; in the upper income magazines, 31
per cent of the subject matter features emulation.

(3) The acquisitive culture, that is the culture which emphasizes things
and snobbism, battles, in the pages of these magazines, with an older
tradition and culture, in which sex, economics, politics, and sentiment
play major r\^oles. The acquisitive culture is dominant in five magazines,
the older culture in four magazines, while in the remaining four
magazines, the two cultures co-exist side by side. One may say, in
summary, that the acquisitive culture cannot stand on its own feet. It
does not satisfy. Except in the fashion magazines, and in some of the
women's magazines, it has to be offered to the reader with a
considerable admixture of the older traditional humanities.

(4) Correlating our various statistical findings, we note that the
acquisitive culture is not accessible to the majority of Americans; also
that it is not popular with the majority of Americans. The American
population apparently has a sturdy realism which the magazine editors
are forced to recognize. They do not want to spend their time reading
fairy tales about the lives of the rich. What they prefer, is to read
about heroes and heroines who are exactly one rung above them on the
economic and social ladder, a rung of the ladder to which they
themselves, by dint of luck, accident, or hard work, may hope to climb
to.

It would appear that the acquisitive culture reflected in these
magazines is a luxury product designed for women and the rich. The focus
upon women is because of their position as buyers for the family. The
success of the emulative sales promoting technique as applied to
middle-class women would appear to rest upon the fact that these women
are restless, that they suffer from unsatisfied romanticism, and that,
in many cases, they probably suffer also from unhappiness in their
marital relations. This is perhaps the most significant finding of the
study and we believe the reader will find it amply supported by the
detailed evidence adduced in the succeeding chapters.


\section{II. Chromium is More
Expensive}\label{ii-chromium-is-more-expensive}}

Culture is, by definition, the sum total of the human environment to
which any individual is exposed and the test of a culture, or
civilization, in terms of values is what kind of a life it affords, not
for a few but for all of its citizens.

The term culture, as used by anthropologists, ethnologists, and social
scientists generally, does not, of course, coincide with the use of the
word among the American working-classes, for whom it constitutes a
description of the middle-class culture to which they so devoutly
aspire. \emph{True Story Magazine}, the favorite magazine of the
proletariat, circulation 1,597,000, has a story about a poor boy, who
marries a banker's daughter and makes good. On first being introduced
into the banker's house he says: ``It was my first experience in a home,
where \emph{culture}, ease and breeding were a part of everyday life.''
\emph{Household Magazine}, circulation 2,006,000, which is read by farm
and small town women, has a page of advice to girls, conducted by Gladys
Carrol Hastings, author of \emph{As the Earth Turns}. Miss Hastings
describes how a daughter of the rich is forced because of the depression
to live on a farm and to do her own work. Miss Hastings says: ``I choose
not to stress how tired she was each night ... how she longed for the
ease and \emph{culture} of other associations, how little her few
neighbors satisfied her.''


\subsection{Class Cultures}

The popular and proletarian use of the word ``culture'' points to a
significant fact; the fact that, contrary to popular pre-war
conceptions, we do have classes in the United States, and that any
examination of our present American culture will, of necessity, break up
into an examination of a number of class cultures.

Two problems face the would-be examiner of contemporary American
culture. The first is to ascertain how many classes there are and the
second is to find a measuring stick for the culture of each of these
different classes. Both are nice problems.

It is noteworthy that there are no names, used in ordinary speech to
characterize social classes, unless ``racketeer'' and ``sucker'' can be
considered to be in this category. In which case we have not the Marxian
antithesis of the workers \emph{versus} the bosses, but the strictly
American antithesis of suckers \emph{versus} racketeers, complicated by
the fact that most Americans are racketeers and suckers at one and the
same time. Workers refer to themselves as ``the working-class of
people,'' executives discuss the white-collar class, ad-men refer to
mass and class publications, fashion analysts study the high, medium,
popular, and low style woman. Common speech is of little help in
differentiating such social classes as we have, nor are the professional
social scientists very useful. With the exception of Veblen's books and
of the magnificent study \emph{Middletown} made by the Lynds in 1927,
which describes minutely the culture of the working and business classes
of a typical American city, the social scientists have added very little
of any importance to what we know about the stratification of the
American population and about American culture.\footnote{{[}Robert S. Lynd and Helen Merrell Lynd,
  \emph{\href{http://www.worldcat.org/oclc/1001579439}{Middletown: A
  Study in Contemporary American Culture}} (New York: Harcourt, Brace
  and Co., 1929).{]}}

The most valuable sources of information we have about the economic and
cultural levels of the American population are such government
statistics as the Army intelligence tests and income-tax returns, and
the unpublished studies of consumer behavior on file in magazine offices
and in advertising agencies. One of the best of these studies is the
work of Daniel Starch. This study divides American families into income
groups, computed in multiples of one-thousand dollars. Since this
chapter expects to lean somewhat on Mr. Starch's researches, it will for
the sake of brevity divide Americans into three economic classes, each
of which proves on examination to have a fairly distinct cultural
pattern. Without bothering about exact names for these classes, since no
idiomatic or exact names exist, we may refer to them briefly as the
rich, the middle class and the poor.

The poor, those having incomes of less than \$2,000 a year, constituted
in 1925, seventy-seven per cent of the population. Most of them live
below the minimum comfort level. The richest members of this class can
afford a minimum health and decency standard of living; the poorer
members of this class cannot. During our most prosperous years, from
1922 to 1929, the majority of Americans were living on less than 70 per
cent of the minimum health and decency budgets worked out by the United
States Government bureaus. Lifelong economic security is rare. This
class is not of much interest to advertisers or editors. The Daniel
Starch studies show that only 34 per cent of the circulation of twenty
women's magazines goes to this group.

The middle class, those having incomes between \$2,000 and \$5,000 a
year can afford comforts. Severe ill-health or prolonged depression
periods, to mention only two of the most important causes, can ruin the
economic security of middle-class families. Nevertheless, it may be said
that lifelong economic security is within the grasp of some of the more
fortunate and thrifty members of this class.

The rich, those having incomes of over \$5,000 a year, are the class
that pays income taxes. Even the poorest enjoy comforts and a few
luxuries. With the richer members of this class, economic security
becomes a possibility, and is, in a considerable percentage of cases,
attained.

There remains the problem of finding a measuring stick with which to
measure the culture of these three classes; the poor, the middle class,
and the rich. Culture has many aspects; it is necessary within the space
of this book to select one of these aspects. Clark Wissler, the
well-known anthropologist, says in his book \emph{Man and Culture}:
``The study of culture has come to be regarded more and more, in recent
decades, as the study of modes of thought, and of tradition, as well as
of modes of action or customs.''\footnote{{[}Clark Wissler,
  \emph{\href{http://www.worldcat.org/oclc/849927361}{Man and Culture}}
  (New York: Thomas Y. Crowell Company, 1923), chap. 1.{]}} It is the modes of thought that
concern us in this chapter. It is more difficult to find out what people
are thinking than to discover what they are doing, but it is also more
fascinating.


\subsection{The Magazine Measuring-Stick}

The public's response to an art offers, perhaps, the best clue as to
what is going on in people's minds. There are, as it happens, three
popular arts in the United States, which are enjoyed to some extent by
all classes; they are the press, the talkies and the radio. The talkies
probably have most influence, but the press is for obvious reasons
easier to examine and measure; it is a better statistical foil.
Moreover, in our magazine-press, in which each magazine is to some
extent aimed at a particular class of readers, our class culture is more
accurately reflected than in either the talkies or in radio programs.

The only serious drawback to using the magazine-press as a measuring
stick for the culture of our three arbitrarily selected classes is that
a considerable section of the wage-earning class, who constitute over 75
per cent of the population, do not read magazines very much because they
cannot afford them. Mr. Starch's studies show that the most popular
magazine of the rich, \emph{The Saturday Evening Post}, is read by 67
per cent of all the families having over \$5,000 a year, while
\emph{True Story}, the most popular magazine among the proletariat, is
read by only 14 per cent of all the families having under \$2,000 a
year. Of the 14 per cent who read \emph{True Story}, over two-thirds
have incomes of \$1,000 to \$2,000 a year, while approximately one-third
have incomes of \$1,000 a year, or less.

The extent to which the magazines do and do not reflect the culture of
any specific economic class is shown in the following chart, based on
Mr. Starch's figures. The reader will observe that all of the magazines
cited have circulations in all three economic classes, and that most of
the circulation lies in the middle-class group. To find magazines which
represent the rich as \emph{versus} the middle class, it is necessary to
seek examples among the so-called class magazines. On this chart, three
magazines; \emph{Harper's Bazaar}, \emph{Harper's Magazine}, and
\emph{Arts \& Decoration}, belong to the class magazine group. Each of
these magazines has over 45 per cent of its circulation among the rich.
In order to strengthen our sample of magazines catering to the rich,
another class magazine, \emph{Nation's Business}, has been added to the
list of magazines to be studied.


\subsection{Who Reads the Magazines?}

The number of magazines which might be said to appeal in the main to the
poor, and which also have large circulations, is disappointingly small.
Only two magazines, \emph{True Story}, which is proletarian in flavor,
and \emph{Household}, which is not, have over one-third of their readers
among the poor. In seeking to fortify the number of magazines which
might be expected to reflect the culture of the poor, two magazines were
added to the list; \emph{The American Weekly}, the illustrated Hearst
Sunday supplement, which has one of the largest circulations of any
periodical in the country, and \emph{Photoplay}, the largest circulation
movie magazine. Examination proved however that \emph{Photoplay} is
probably to be considered as a middle-class magazine.

It might be noted in passing that, in the main, the poor have no press.
We have discovered no large circulation magazine which has over 45 per
cent of its circulation among the poor. One suspects that magazines like
\emph{True Story} cater to the one-tenth of the working-class consisting
of organized and skilled workers who can afford some comforts. One
suspects further that the other nine-tenths of the wage as \emph{versus}
salary earners, although they may read the magazines, have, strictly
speaking, no large circulation press at all.

\begin{figure}
\centering
\includegraphics{sixone}
\end{figure}

\subsection{The Editor-Reader Relation}

The advertising business has frequently been defined in this book as
consisting of the newspaper and magazine press, the radio, the
advertising agencies, and a considerable section of the talkie, paper,
and printing industries. To the magazine editor and the ad-man a
magazine consists of two parts: advertisements and filler. The filler is
designed to carry the advertisements. With rare exceptions, no way has
so far been discovered of getting the public to pay for advertisements
presented without filler. Hence the filler.

This strictly commercial point of view of the magazine editor, the
circulation manager, and the ad-man is not the reader's point of view.
The reader thinks of a magazine in terms of fiction, articles, features,
editorials, and advertisements. While he seldom buys the magazine for
the ads, he may enjoy certain ads even more than he enjoys the contents
of the periodical. In addition to hunting out the particular things in
the magazine which appeal to him as an individual, or which he hopes to
find tolerably palatable, he is more or (less aware of the personality
of the magazine. Its slant on things is as well known to him as the
slant of a family friend, and although he may not agree with the slant,
he enjoys savoring of it. From the reader's point of view, therefore,
one can add at least one more category to the commercial categories of
the editor and ad-man. One can say that the magazine consists not only
of advertisements and filler, but that it also has an editorial element,
that there is in fact, in most cases, a certain editor-reader relation,
which the reader is quite cognizant of.

That the editor-reader relation, just referred to, exists not only in
the mind of the reader, but in the mind of the editor as well, is shown
by the following statement made by Gertrude B. Lane, assistant editor of
\emph{Woman's Home Companion}. In a memorandum stating her objections to
the Tugwell Bill, Miss Lane says:

\begin{quote}
``I admit quite frankly that my selfish interests are involved. I have
spent thirty years of my life building up a magazine which I have tried
to make \emph{of real service to the women of America}, and I have
invested all my savings in the company which publishes this magazine.
The magazine business and the newspapers, rightly or wrongly, have been
made possible through national advertising. Great industries have been
developed and millions of people employed.''
\end{quote}

Miss Lane's angle is interesting. Is advertising perhaps the culture,
the swamp-muck, if you will, that exists to nourish this lily of
service? If Miss Lane is correct, the question that will interest the
magazine reader is not how thick is the muck, but how tall and fragrant
is the lily? An examination of the January, 1934, issue of \emph{Woman's
Home Companion} will perhaps answer this question.


\subsection{Service Versus Selling}

In looking for the service-angle suggested by Miss Lane, the writers
felt that a correct estimate of the amount of service rendered the
reader could perhaps best be found in editorials and articles, rather
than in the fiction. Fiction was also considered in relation to service,
and the results will be referred to later in this chapter. The
concentration on editorials and articles proved, however, to offer the
most useful index of service. The issue of the \emph{Woman's Home
Companion} examined contained in its editorials and articles three items
which could be listed under this head.

\begin{quote}
Item I. Article ``What Mothers Want To Know'' (5.5 inches). The writer,
a physician, starts out by saying: ``I wonder if we city doctors write
about the things that mothers want to know. At least sixty per cent of
the mothers' letters received by \emph{Woman's Home Companion} come from
small cities, towns, or rural communities, which have practically no
modern facilities, no hospitals or clinics for babies, well or sick, no
pediatrists. Many of the letters are pathetic.''
\end{quote}

\begin{quote}
Item II. Editorial ``The Mighty Effort'' (8 inches). This editorial
urges Americans to support President Roosevelt's program. The dangers of
this program can, in the opinion of the editors be avoided, ``if the
true American spirit prevails.'' The true American spirit consists in
moderation. Owen D. Young is quoted as saying: ``We must watch them that
threaten us, both from inaction and over-action, not that we may punish
them, but that we may prevent them from ruining us and themselves as
well. It is unnecessary for producers to unite into a trust ... it is
unnecessary for labor to unite in unions ... it is unnecessary for
consumers to unite in such a way as to threaten savings and labor
employed in production.''
\end{quote}

\begin{quote}
Item III. Letter. Signed, C. R. J., Oregon, entitled by the editors,
``Sensible Protest Against Frills'' (8.5 inches). Criticizes the home
economics classes attended by country and small town children, in which
the pupils are taught: ``How to give orders to a maid and butler ... to
put fancy frills on a chop bone, and to cook steaks.'' The writer notes
that most of the parents of these children afford steaks and chops very
rarely, and makes sensible suggestions as to what a home economics
course for country children should contain.
\end{quote}

Of the 1,404 inches devoted to editorials and articles, 22 inches, or
about two-thirds of a page, is devoted to service. But the lily of
service which raises its pure head in a naughty world should not be
measured in inches or percentages alone. What does the two-thirds of a
page devoted to service in the \emph{Woman's Home Companion} net the
reader? A reader makes a sensible statement, so sensible that one
concludes that it might be an excellent thing for editors to turn over
their editorial space to their shrewder readers. As far as the editors
are concerned they have only two things to say to the reader.

First: In a general editorial about recovery, they point out to their
readers, who are consumers, that ``it is unnecessary for consumers to
unite in such a way as to threaten savings and labor employed in
production.'' In suggesting that its readers do not become politically
active as consumers, the \emph{Companion} would seem to be serving its
own interests rather than those of its readers. Second: They promise in
the future to help the women living in small towns with their maternity
problems. Excellent as this is, a promise of service does not constitute
a service. If the \emph{Woman's Home Companion} fulfills its promise,
this fulfillment will constitute a genuine service to the reader.

Examination of the other twelve magazines selected for study is somewhat
more reassuring than examination of the \emph{Woman's Home Companion.}
The service element of the other magazines as measured by the editorials
and articles ranges as high as 88 or 79 per cent in contrast with the
\emph{Woman's Home Companion}'s 1.5 per cent. The complete list of space
devoted to service is as follows: \emph{Saturday Evening Post}, 88 per
cent; \emph{Nation's Business}, 79 per cent; \emph{American Magazine},
41 per cent; \emph{Harper's Magazine}, 37 per cent; \emph{Cosmopolitan},
28 per cent; \emph{Liberty}, 24 per cent; \emph{True Story}, 16 per
cent; \emph{Household Magazine}, 11 per cent; \emph{Harper's Bazaar}, 2
per cent; \emph{Woman's Home Companion}, 1.5 per cent; \emph{American
Weekly}, .7 per cent; \emph{Photoplay}, 0; \emph{Arts \& Decoration}, 0.


\subsection{Service as Sophistication}

To make sure that we are doing justice to the \emph{Woman's Home
Companion}, it might be well to state at this point what items the
writers have considered to have a service angle. An examination of the
thirteen selected magazines caused the writers to re-define service as
sophistication, and specifically sophistication about economic and
political affairs. Four kinds of items were included under
Sophistication:
\enlargethispage{\baselineskip}
\begin{enumerate}
\item
  Any reference to recovery or depression was considered to constitute
  sophistication, since it may be considered an index of interest in
  reality as opposed to fantasy.
\item
  Any recognition that an economic or political situation was complex
  rather than simple was also considered to constitute sophistication. A
  mention of three or four factors in a situation rather than one or two
  was considered to be complex as opposed to simple.
\item
  Any facts which did not bear directly on the financial or emulative
  interest of the specific class of readers to whom the magazine is
  addressed, were considered to constitute sophistication. Note: Only
  two or three examples were found.
\item
  Any criticism or satire of our contemporary culture and society which
  might be considered to apply not to a specific institution but to the
  society as a whole.
\end{enumerate}

The standards set up as sophistication are not high. Any truly
sophisticated presentation of an economic or political situation would
usually have to cover more than three or four factors in the situation.
Many of the articles in the \emph{Saturday Evening Post}, \emph{Nation's
Business}, and in such magazines as the \emph{Nation}, \emph{New
Republic}, and \emph{Fortune}, rate well above this
three-or-four-factors-in-a-situation level. It has been the effort of
the writers to include under sophistication everything which could
possibly be included under this category. Most if not all of the rays of
hope, inspiration or comfort extended to the readers by the editors it
has been possible to pick up under one of the four categories used.

When the results of the sophistication survey are averaged, it is found
that the average magazine devotes 24.4 per cent of its editorial and
article space to making the contemporary economic and political world
which so notably affects the destinies of its readers somewhat
comprehensible. The amount of sophistication is clearly one of the
important elements in the editor-reader relation of the magazine. The
extent to which the sophistication element in each of the magazines
studied has vitality or sincerity, will be considered when the contents
of individual magazines are described.

The sophistication survey shows one notable fact; that magazines
specifically for women are low in respect to sophistication. Remembering
that 24.4 per cent is the sophistication average for thirteen magazines,
consider the degree of sophistication of the following magazines
catering mainly to women: \emph{Household Magazine}, 11 per cent;
\emph{Harper's Bazaar}, 2 per cent; \emph{Woman's Home Companion,} 1.5
per cent; \emph{Photoplay}, 0; and \emph{Arts \& Decoration}, 0.
\emph{Harper's Bazaar}, a fashion magazine; \emph{Photoplay}, a movie
magazine; and \emph{Arts \& Decoration}, an interior decoration
magazine, are, of course, specialized magazines, with no interest in
economics or politics. Nevertheless, the line-up seems to have some
significance. Contrast the women's magazine sophistication record, for
example, with the sophistication record of the magazines which have an
exclusive or heavy male readership; \emph{Saturday Evening Post,} 88 per
cent; \emph{Nation's Business}, 79 per cent; and the \emph{American
Magazine}, 41 per cent. The claim that the contents of women's magazines
reflect the provincialism and low intellectual status of women was made
in an article in the December, 13, 1933, issue of the \emph{New
Republic}. This article provoked a spirited rebuttal from no less a
person than Carolyn B. Ulrich, Chief of the Periodicals Division of the
New York Public Library, New York City. Miss Ulrich says, among other
things:

\pagebreak \begin{quote}
``Who are the owners and editors of women's magazines? You will find
that men predominate in the executive offices and on their editorial
staffs. Would it not appear that we are still bound to what men think
desirable? Is that what most women want? And are not these magazines
really mediums for salesmanship, almost trade journals? Of the first
importance in these magazines is the advertising. The subject matter
comes second. The advertisements pay for the producing of the magazine.
The subject matter, aside from a few sentimental stories, covers those
interests that belong to woman's sphere. \emph{There, also, the purpose
is to foster buying} for the home and child. The entire plan of these
magazines is based on the man's interest in its commercial success.''
\end{quote}


\subsection{Perversion of Editor-reader Relation}

In one of Miss Ulrich's sentences, we find the clue to the nature and
character of our present women's magazines. Miss Ulrich says: ``The
subject matter ... stories aside, covers those interests that belong to
woman's sphere. There, also, the purpose is to foster buying.'' Miss
Ulrich is correct. If the contents of the women's magazines are
examined, it will be found that the editors devote from 48 to 15 per
cent of the total contents of the magazine to ballyhooing certain
classes of products or specifically named products; in short, to
peddling something over the counter, just as advertisements do. The five
magazines catering mainly to women, which rank very much below the
average in respect to sophistication, rank highest in respect to the
amount of editorial space devoted to salesmanship. The proportion of the
total space in the women's magazines devoted to editorial advertising is
as follows: \emph{Arts \& Decoration}, 48 per cent; \emph{Harper's
Bazaar}, 34 per cent; \emph{Photoplay}, 24 per cent; \emph{Household},
18 per cent; \emph{Woman's Home Companion}, 15 per cent. \emph{Harper's
Bazaar} devotes 26 of its non-advertising pages to mentioning the names
of 523 stores and products.

The nature and character of our women's magazines becomes clear if one
realizes that in these magazines the editor-reader relation has been
perverted. Where this relation has vitality and sincerity, the readers
get from the magazine something not wholly commercial. They do not
merely get enough filler or entertainment to make them swallow the
advertising; they are given something definite and humanly valuable, a
friendly relation to the editor, who is or should be, from the reader's
point of view, a person whose specific job it is to know more about
affairs in general than the reader can take time to know. An editor's
analysis of a situation, his judgment about it, have some weight with
the reader, just as a friend's analysis of a situation and judgment
about it have. However, where the editor-reader relation is perverted,
as in the women's magazines, the editor does not give the reader
something; he takes something away from the reader. It is a case of the
right hand giveth and the left hand taketh away. The left hand of the
editor takes away from the reader part of the non-advertising or subject
matter space of the magazine which is presumably what the reader pays
for, and devotes it to editorial advertising. The right hand of the
editor gives the reader something humanly valuable; sophistication. In
the five magazines catering primarily to women, as the accompanying
chart shows, the editorial left hand, the hand which takes, is the
active hand.

\enlargethispage{\baselineskip}
\subsection{Editorial Advertising}

Editorial advertising in the accompanying chart includes three
categories. In the order of their importance, that is, in the order of
the amount of space devoted to them, they are as follows:

\begin{figure}
\centering
\includegraphics{sixtwo}
\end{figure}

\begin{itemize}
\item
  Item 1: Pushing of advertised products.
\item
  Item 2: Pushing of sales of, or subscriptions to the magazine.
\item
  Item 3: Editorials or articles, pushing buying in general, or pushing
  the buying of certain classes of products, which may or may not appear
  in the magazine's advertisements.
\end{itemize}

Of the total space of the thirteen magazines, 10.9 per cent is, on the
average, devoted to pushing products; 2.6 per cent is devoted to pushing
the magazine; and one per cent to pushing buying generally. House ads,
pushing the sale of the magazine are familiar, and hardly need
illustration. The pushing of advertised products is also more or less
familiar. A few examples will probably suffice:

\vspace{2.5mm}

\begin{center}
\itshape{Artificial Skills}
\end{center}

\begin{quote}
``I sometimes think the women of today aren't sufficiently thankful for
or appreciative of the fabric marvels which are theirs.... As a miracle,
for instance, doesn't artificial silk answer every requirement of the
word?'' (\emph{True Story}: ``Sheer Fabrics That Would Make Cleopatra
Jealous.'')
\end{quote}

\begin{center}
\itshape{Oil Heater}
\end{center}

\begin{quote}
``Where lack of a basement makes installation of the usual type of
cellar plant impossible ... there are heat cabinets available.... With
one of these oil heaters in a room, the old fire-building,
stove-nursing, ash-carrying, half-warmed days are over.'' (\emph{True
Story}: ``Is Your Home Old-Fashioned in Its Heating Apparatus?'')
\end{quote}

\begin{center}
\itshape{Canned Meats
\end{center}

\begin{quote}
``In looking around to see just what I could discover in canned meats
and chickens, I found great variations in the size of their
containers.'' (\emph{Household Magazine}: ``A Short Cut to Meats---The
Can-Opener.'')
\end{quote}

\begin{center}
\itshape{Condensed Milk}
\end{center}

\begin{quote}
``She (my grandmother) tried cow's milk, the best she could obtain, but
without any improvement. In desperation she finally tried a spoonful of
the new condensed milk, a recent invention that a newcomer in the gold
camp had brought from the East. The baby loved it.'' (\emph{True Story}:
``From My Grandmother's Diary.'')
\end{quote}


\begin{center}
\itshape{Electric Lamps}
\end{center}

\begin{quote}
``She spent many months of patient searching for just the right lamps at
just the right prices. Lamps that would give the perfect angle of light
....'' (\emph{Woman's Home Companion}: ``A Healthful Luxury.'')
\end{quote}


\begin{center}
\itshape{Hotels}
\end{center}


\begin{quote}
``No place in the world has such sparkle as New York at this time of
year. Come for the fun of shopping ... to see the new ballets ... to
enjoy the restaurant life of these new days of the wine list .... For
help in choosing your hotel, write to the Travel Bureau.''
(\emph{Harper's Bazaar}: ``New York at Christmas.'')
\end{quote}

\vfill

\pagebreak


\begin{center}
\itshape{Tea Table Accessories}
\end{center}

\begin{quote}
``All of our social existence is tied up in a few familiar rituals. A
hostess is known by her tea tables and dinner tables. Marriages and
births and political victories and personal achievements are celebrated
there.... Occasionally something definite and permanent arises
phoenix-like from a passing mode. Lines that appeared as startling
innovations on the tea tray of some smart hostess gradually become
familiar in decorative treatment and in architecture. So a new style is
created.'' (\emph{Arts \& Decoration}: ``A Portfolio of Modern
Accessories.'')
\end{quote}

Somewhat more subtle and interesting are editorials and advertisements
pushing buying generally, or the buying of certain classes of products.


\begin{center}
\itshape{``A Call to Colors for the American Male''}
\end{center}

\begin{quote}
``The pioneering hard-fisted, hard-boiled American Male will cheer
campaign speeches on the benefits of rugged individualism and whistle
laissez faire, whenever he has to keep up his courage in a financial
crisis. He will grow turgidly eloquent on the benefits both to himself
and society of doing just as he sees fit when and if he pleases. He will
battle to his last breath against any code prescribing a uniform way of
running his business, auditing his accounts, educating his children or
divorcing his wives. Any form of regulation is to him a symptom of
Bolshevik tyranny. But the one moment when he is terrified of freedom is
when he buys his clothes. \emph{He is more afraid of wearing a bright
orange necktie to his office than of carrying a red flag in a communist
parade.}'' (\emph{Harper's Bazaar}.)
\end{quote}


\begin{center}
\itshape{``Bare Without Jewels''}
\end{center}

\begin{quote}
``To the great dressmakers and to the women who make fashion a matter
for prayer and meditation, and especially to foreign women, we Americans
are as incomplete as the vermilionless painting.... Lean back in a stall
in Covent Garden on a Ballets Russe night and compare the jewels you see
with those worn at the average American soiree. Foreigners cannot
understand our modesty in this regard. How extraordinary, they say, that
you Americans who have money are content with the small bracelet, the
one string of pearls, the nice ring or two....
\end{quote}

\begin{quote}
These simple molded gowns of black or jewel colored velvets, these dark
green sheaths, these brilliant columns of stiff white satin crave the
barbaric fire of emeralds, diamonds, rubies.... For the last twenty
years we have been genteel and timid about jewelry. It was not always
thus. Let those who feel shocked by this modern splendor remember that
their aristocratic grandmamas blazed with dog collars and tiaras.
\emph{And who are we to say that the Queen of Sheba was not a lady?}''
(\emph{Harper's Bazaar}.)
\end{quote}

\vfill

\pagebreak

\begin{center}
\itshape{``Contempora'}
\end{center}

\begin{quote}
``A contemporary chair or service plate can range as far in cost and
beauty as those of Louis the XIVth or any other period. \emph{Chromium
is more expensive than no chromium, beveled glass is more expensive than
glass that is not beveled}.'' (And a vote for Wintergreen is a vote for
Wintergreen.) \emph{Arts \& Decoration}.
\end{quote}

Perhaps it is because editorial advertising is newer than pure
advertising that the tone of editorial advertising is often so brash. In
\emph{Arts \& Decoration}, the magazine which has the highest percentage
of editorial advertising, the situation has gone so far that the
strident voice of salesmanship concentrates in the subject matter, while
the advertisements are comparatively dignified and serene.

The editor-reader relation is the vital core of the magazine. The study
of thirteen magazines shows that this relation has its credit and debit
side; that it is at once an Angel Gabriel and a Lucifer. In short, it is
a most human relation, in which the itchiness of the editor, eager to
attract more advertising and revenue, competes with his desire to be
humanly useful.

No description of the magazines would be complete without a reference to
the advertisements, which in contradistinction to the editorial
advertisements, are openly and unhypocritically concerned with selling.
Our statistics show that on the average 30.6 per cent, or a little less
than a third of the magazine is devoted to straight advertising, while
on the average 43.5 per cent, or a little over two-fifths of the
magazine, is devoted to straight advertising and editorial advertising
combined. This 43.5 per cent is the Selling-end of the magazine. The
other 54.6 per cent is devoted to what is generally known as filler and
what for the purposes of this study we have defined as Sophistication
and Entertainment.


\subsection{Major Advertising Appeals: Fear, Sex, and Emulation}

It is perhaps worth noting that the five magazines catering mainly to
women rank highest not only in respect to the proportion of space in the
total contents of the magazine devoted to editorial advertising, but
also in the proportion of space devoted to selling. The amount of space
devoted to selling averages 43 per cent in the thirteen magazines and 62
per cent in the case of the five women's magazines.

Advertisements are, to the student of a culture, one of the most
revealing sections of the magazine. A great many studies of advertising
have been made. First, they reflect, as in a mirror, the material
culture of a people. Second, they throw light on economic levels and
class stratification. With the material culture of the United States we
are not, in this chapter, primarily concerned. The extent to which
advertisements reflect class stratifications has already been mentioned,
and will be referred to again in more detail. For the moment, we shall
limit ourselves to asking one question: To what extent do the
advertisements in these thirteen magazines give the reader useful
information about the product? The success of the magazine,
\emph{Ballyhoo}, and its imitators, showed that many people found some
ads absurd, and perhaps annoying, and that they were glad to have them
kidded. Not all advertising, however, is of this character. The question
is what proportion of the ads are useful, and what proportion are
natural material for satire?

It was necessary to find a simple measuring stick. An analysis of the
advertisements showed that they appealed to many different instincts on
the part of the reader, to fear, to sex, to emulation, to the desire to
make money, the desire to save money, and so forth. Moreover, a single
advertisement often combines several appeals. It soon became apparent
that the three major appeals of the ads, those that appeared most
frequently, were fear, sex, and emulation. It was therefore decided to
break up the ads into two categories: 1) those that unmistakably
contained one of these three appeals, regardless of what other appeals
the individual ad might also contain; 2) ads which did not contain one
of these three appeals, and which were called straight ads. In the main,
it might be said that the straight ads contain more description of the
product than the fear-sex-or-emulation ads. This latter type of ad is
more concerned with creating atmosphere than with describing the
product.

What the writers mean by advertisements appealing to the instincts of
fear or sex hardly requires explanation. Emulation, however, needs to be
defined. As used in this chapter, emulation is equivalent to snobbism,
it is the keeping-up-with-the-Joneses motif, the desire on the part of
the individual to prove to his neighbors that his social status is
enviable. In short, it is a particular form of competitiveness, relating
not to personal charm or financial rating, but simply and strictly to
success in maintaining or achieving social status.

An examination of the ads showed that, on the average, 39 per cent of
the ads are fear-sex-and-emulation ads, while 61 per cent are straight
ads. The minimum percentage of fear-sex-and-emulation ads was 6 per
cent; the maximum, 66 per cent. Three out of the four magazines that
reflect the culture of the rich, the Class ``A'' magazines, were low in
respect to fear-sex-and-emulation ads. The statistics are as follows:
\emph{Harper's Bazaar}, 57 per cent; \emph{Nation's Business}, 28 per
cent; \emph{Arts \& Decoration}, 23 per cent; and \emph{Harper's
Magazine}, 17 per cent. No equally clear correlation appears in regard
to the magazines which rank high in respect to fearsex-and-emulation
appeals. Nevertheless, it may perhaps be said that a low percentage of
fear-sex-and-emulation ads is characteristic of the Class ``A''
magazines. This correlation may perhaps to some extent reflect the
sophistication of this class; what it probably reflects in the main is
the good manners of the rich; the desire for good tone, as \emph{versus}
vulgarity or stridency.

A further correlation between the fear-sex-and-emulation ads and class
stratification appears, when we consider the percentage of advertising
space devoted to each one of these three appeals in the various
magazines. The appeal to fear predominates in seven magazines, which
are, generally speaking, the magazines of the lower income-levels, while
the appeal to emulation predominates in six magazines of the upper
income-levels. In no magazine is the appeal to sex dominant over the
appeal to fear or to emulation. The following graph shows not only what
percentage of the total advertising space is devoted to appeals to fear,
sex, and emulation, but which is the dominant appeal in each magazine.

\begin{figure}
\centering
\includegraphics{sixthree}
\end{figure}

A little reflection shows that the dominance of the fear appeal in the
magazines of the lower income-levels and the dominance of emulation in
the magazines of the upper income-levels is quite natural. The poor
cannot afford emulation; the rich can. Moreover, the poor are used to
fear and insecurity, with them the reference to fear is not an alien
thing. As is the case with primitive peoples, they live surrounded by
fears.

The fact that sex proves in the advertisements of these typical American
magazines to be less powerful as an appeal than either fear or emulation
is interesting. One grants easily, without being able to prove it, that
fear is probably a stronger motivation than sex, in all societies. The
question remains whether emulation is in all societies a stronger motive
than sex, or whether it is merely in American society that emulation is
a powerful motivation, while sex is a weak motivation.

Before leaving the discussion of the ads to consider the section of the
magazines devoted to what we choose to call Entertainment, it may be in
point to make a few concluding but scattering comments concerning
advertisements.

First: We have seen that the majority of the ads, 61 per cent, are
straight ads, dealing in the main with the product, rather than
fear-sex-or-emulation ads, which are interested mainly in creating
emotion or atmosphere. A qualifying note is necessary at this point. It
would be inaccurate to assume that 61 per cent of the ads devote
themselves mainly to describing the product. The majority of these ads
devote more space to describing the effect upon the buyer of using the
product than to describing the product itself. Very elaborate
statistical work would have been necessary to document this observation,
and because of the difficulties involved, no work of this character was
done.

Second: With two exceptions, advertisements of products that appear in
the magazines of the rich, the middle classes and the poor, tend to be
the same; that is, to have the same words and copy, the assumption of
the ad-men being that we Americans are all brothers and sisters under
the skin. Of the two conspicuous exceptions, one has already been noted,
namely: the fact that fear appeals predominate in the lower
income-brackets, while emulation appeals dominate in the upper
income-brackets. The other exception is that the fear appeals in the
lower income-brackets are somewhat cruder than the fear appeals in the
upper income-brackets. Specifically, there is more appeal to fear of
parents for the safety and well-being of their children. Illnesses and
discomforts from which both adults and children may suffer are in many
instances embellished with photographs of wan, reproachful children.

\begin{enumerate}
\item
  ``Mother, Why Am I so Sore and Uncomfortable?'' (Waldorf Toilet Tissue
  ad in \emph{True Story}.)
\item
  ``Scolded For Mistakes That Father and Mother Made.'' (Postum General
  Foods ad in \emph{Household Magazine}.)
\item
  ``And Don't Go Near Betty Ann---She's a Colds-Susceptible.'' (Vick's
  ad in \emph{Women's Home Companion}.)
\end{enumerate}

Third: An examination of the advertising and also of the editorial
contents of the magazines shows that the commercial interests back of
the magazines treat women and the poor with scant respect, while men and
the rich have a somewhat better rating.


\section{III. The Ad-Man's Pseudoculture}

It is perhaps desirable once more to say what we mean by the ad-man and
what we mean by the pseudoculture. We have tried to show in the
preceding chapter that the commercial American magazines are essentially
advertising businesses. Hence the editors of these magazines may be,
with some minor qualification, correctly characterized as advertising
people motivated by considerations of profit.

But a society does not and cannot live solely by acquisitive and
profit-motivations. If this were possible the joint enterprise of the
advertising writer and the commercial magazine editor, which is, by and
large, to promote and construct a purely acquisitive culture, would be a
stable and successful enterprise.

It is nothing of the sort. Frankly the writers started with a
pessimistic hypothesis, viz.: that the acquisitive-emulative cultural
formula had so debauched the American people that they really liked and
approved this formula as worked out by the mass and class magazines. The
writers expected on examining the magazines to find the acquisitive
culture dominant in all of them, and to find that in the majority of
cases this culture existed undiluted by any admixture of the older,
traditional American culture. If they had found what they expected to
find, they would have been obliged to accept the conclusion that the
ad-man's acquisitive-emulative culture is an organic thing, something
capable of sustaining human life. The findings did not show this. On the
contrary, they showed beyond the possibility of a doubt that the
acquisitive culture cannot stand on its own feet, that it does not
satisfy, that it is, in fact, merely a pseudoculture.

The magazines live by the promotion of acquisitive and emulative
motivations but in order to make the enterprise in the least tolerable
or acceptable to their readers it is necessary to mix with this
emulative culture, the ingredients, in varying proportions, of the older
American culture in which sex, sophistication, sentiment, the arts,
sciences, etc., play major r\^oles. Only three of the thirteen magazines
examined are able to build and hold a circulation on the basis of an
editorial content consisting solely of acquisitive and emulative
appeals. All of these three are in one way or another special cases.
\emph{Arts \& Decoration}, \emph{Harper's Bazaar}, and \emph{Photoplay}
are all three essentially parasitic fashion magazines. The first two are
enterprises in the exploitation of the rich, who constitute over 50 per
cent of their circulation. \emph{Photoplay}, a middle class gossip and
fashion sheet, is, by and large, simply a collection agent for the
acquisitive and emulative wants built up by the movies which, as we have
seen, function predominantly as a want-building institution in the
American culture.

In other words the business of publishing commercial magazines is a
parasitic industry. The ad-man's pseudoculture parasites on the older,
more organic culture, just as the advertising business is itself a form
of economic parasitism; in Veblen's language, it represents one of the
ways in which profit-motivated business ``conscientiously withdraws
efficiency from the productivity of industry,'' this ``conscientious
sabotage'' being necessary to prevent the disruptive force of applied
science from shattering the chains of the profit system.\footnote{{[}Thorstein Veblen,
  \emph{\href{http://www.worldcat.org/oclc/752183}{Absentee Ownership
  and Business Enterprise in Recent Times: The Case of America}} (New
  York: B. W. Huebsch, 1923).{]}} It is, we
feel, important to note that this phenomenon of parasitism or sabotage
extends not merely to the economy considered as a mechanism of
production and distribution but to the culture considered as a system of
values and motivations by which people live.

But the American people do not like this pseudoculture, cannot live by
it, and, indeed, never have lived by it. The magazines analyzed, which
were published during this the fifth year of a depression, show that
fiction writers, sensitive to public opinion, often definitely repudiate
this culture. Americans tend, at the moment, if the magazine culture can
be considered to be a mirror of popular feeling, to look, not forward
into the future, but backward into the past. They are trying to discover
by what virtues, by what pattern of life, the Americans of earlier days
succeeded in being admirable people, and in sustaining a life, which, if
it did not have ease and luxury, did seem to have dignity and charm.
Although the main drift of desire is toward the past, there are other
drifts. Some editors and readers even envision revolution and the
substitution of a new culture for the acquisitive and the traditional
American culture.


\subsection{The Battle of the Cultures}

In the older, more humane culture, sex and sophistication are the major
elements. In the artificial profit-motivated pseudoculture by which the
commercial magazine lives and tries to make its readers live, emulation
tends to replace sex as a major interest, whereas sophistication
dwindles and ultimately disappears. The following table exhibits a
striking inverse ratio:

\vspace{5mm}

\begin{center}COMMERCIALISM \emph{VERSUS} SOPHISTICATION\end{center}
\tabulinesep=1.1mm
{\begin{longtabu} to 1.05\textwidth { X[l] X[c] X[c] } 
\emph{Magazine} & \emph{Per cent of editorial and article space devoted to sophistication} & \emph{Per cent of total magazine space devoted to editorial advertisements} \\ 
\endfirsthead
\emph{Magazine} & \emph{Per cent of editorial and article space devoted to sophistication} & \emph{Per cent of total magazine space devoted to editorial advertisements} \\
\endhead
Saturday Evening Post & 88\% & 3\% \\
Nation's Business & 79\% & 8\% \\
American Magazine & 41\% & 2\% \\
Harper's Magazine & 37\% & 7\% \\
Cosmopolitan & 28\% & 3\% \\
Liberty & 24\% & 4\% \\
True Story & 16\% & 6\% \\
Household & 11\% & 18\% \\
Harper's Bazaar & 2\% & 34\% \\
Woman's Home Companion & 1.5\% & 15\% \\
American Weekly & .7\% & 1\% \\
Photoplay & .0\% & 24\% \\
Arts \& Decoration & .0\% & 48\% \\
\end{longtabu}}

\vspace{2mm}

In the \emph{Saturday Evening Post} we find the maximum of editorial and
article space, 88 per cent, devoted to sophistication. By sophistication
we mean a realistic attempt by the editors to deal with the facts and
problems which constitute the everyday concerns of their readers. The
\emph{Post} devotes a minimum of space to editorial advertising. Yet,
paradoxically enough, the \emph{Saturday Evening Post} is the greatest
advertising medium in the world. This would seem to indicate that
editorial advertising is to a magazine what makeup is to a plain woman.
Not that the \emph{Post} is in any true sense a satisfactory and
creative cultural medium. The most that can be said for the \emph{Post}
is that it functions with some sincerity and effectiveness as the organ
of a specific economic and social class.

At the bottom of this dual ascending and descending scale, we find
\emph{Arts \& Decoration} with a sophistication rating of zero and 48
per cent of its total space devoted to editorial advertising. Obviously,
\emph{Arts \& Decoration} represents the phenomenon of pure commercial
parasitism. It is the organ of nothing and nobody except its publishers
and advertisers, and it holds its 18,000 readers by a mixture of
flattery and insult, which magazine publishers, it seems, consider to be
the proper formula to be used on the new-rich and the social climber.
The slogan would seem to be: Mannerless readers deserve a mannerless
magazine.

\begin{figure}
\begin{fullwidth}
\centering
\includegraphics{sixfour}
\end{fullwidth}
\end{figure}

There is another inverse ratio in which this battle of the cultures is
apparent. In the magazine literature of the prewar days, men and women
grew up, fell in love, married, had children, and lived more or less
happily ever after. Among current magazine examples we find that the
\emph{American Magazine} is still reasonably confident that this
biological pattern is fundamental to human life. In 78 per cent of its
fiction content sex---sentimental sex---is a major appeal.
Significantly, we note that only three per cent of the \emph{American
Magazine}'s non-advertising space is devoted to promoting emulative
motivations. With the \emph{Saturday Evening Post}, a magazine which
goes to a somewhat wealthier class of readers than the American, the
emphasis on sex has lessened, and the interest in the acquisitive
society is much more pronounced. Only 28 per cent of the \emph{Post}'s
fiction is devoted to sex, compared to the \emph{American}'s 78 per
cent. 45 per cent of the \emph{Post}'s subject matter space is devoted
to emulation. Still more extreme is the situation in respect to
\emph{Photoplay} and \emph{Arts \& Decoration}, where sex rates five and
zero per cent respectively, and emulation rates 20 and 43 per cent.

The magazine spectrum breaks down into three major categories; the five
magazines in which the acquisitive culture is dominant, the four
magazines in which the two cultures co-exist; and the four remaining
magazines in which the older culture is dominant. It is significant that
the first group of magazines caters exclusively to women; the second and
third groups to both men and women.

\begin{figure}
\begin{fullwidth}
\centering
\includegraphics{sixfive}
\end{fullwidth}
\end{figure}

There are two other women's magazines in which the acquisitive culture
is dominant. The \emph{Woman's Home Companion} is edited for the urban
woman, and \emph{Household Magazine}, the largest and most popular of
the rural women's magazines, caters to the small town and farm woman.
\emph{Woman's Home Companion} may be said to be typical of the six urban
women's magazines with over 1,000,000 circulation---\emph{Ladies' Home
Journal}, \emph{McCalls}, \emph{Woman's Home Companion}, \emph{Good
Housekeeping}, \emph{Pictorial Review}, and \emph{Delineator}; while
\emph{Household} is typical of the five rural women's magazines with
over 1,000,000 circulation---\emph{Household}, \emph{Woman's World},
\emph{Needlecraft}, \emph{Mother's Home Life} and \emph{Household
Guest}, and \emph{Gentlewoman}. These nine magazines alone distribute
239,000,000 copies of their product every year.

There is a distinct difference between the rural and the urban women's
magazines; the rural magazines being much closer to the older
traditional American culture. \emph{Household Magazine} is one of the
few magazines on our list that mentions God; the poetry is nai've and
sincere, and the editor is human, honest, and even imaginative about his
readers. The difficulty with \emph{Household} would seem to be that
there is a conflict between the editorial office and the business
office; the business office being intent on apeing the formula and
commercialism of the urban women's magazine group. In the urban women's
magazines, the older American culture has become so thin as to be hardly
visible. Even the interest in sex withers away in the \emph{Companion}.
While \emph{Household} devotes 58 per cent of its fiction to sex, the
\emph{Companion} gauges its readers' interest in sex at 22 per cent. The
sophistication element in \emph{Household} is 16 per cent; in the
\emph{Companion} it is 1.5 per cent.

The group of four magazines in which neither culture is dominant, but in
which both cultures exist side by side, includes the
\emph{Cosmopolitan}, \emph{Liberty}, \emph{True Story} and the
\emph{Saturday Evening Post}. The following table will show what
elements of the two cultures are present:

\vspace{2mm}

\tabulinesep=1.1mm
{\begin{longtabu} to 1.05\textwidth { X[l] X[l] X[c] } 
\emph{Magazine} & \emph{Older Culture} & \emph{Acquisitive Culture} \\
\endfirsthead
\emph{Magazine} & \emph{Older Culture} & \emph{Acquisitive Culture} \\
\endhead
Saturday Evening Post & Sophistication & Emulation \\
Cosmopolitan & Sex & Emulation \\
Liberty & Sex & Emulation \\
True Story & Sex & Emulation \\
\end{longtabu}}

\vspace{2mm}

In the magazines in which emulation is dominant, less than three-fifths
of the fiction is concerned with sex. But in \emph{Cosmopolitan},
\emph{Liberty} and \emph{True Story} over three-fifths of the fiction is
concerned with sex. The acquisitive culture is represented by a
considerable dash of emulation: \emph{Cosmopolitan} 13 per cent;
\emph{Liberty} 17 per cent; and \emph{True Story} 30 per cent. In
connection with \emph{True Story} it should be pointed out that the
emulative escape for the poor is crime and that this fact is quite
definitely recognized in the fiction content of this magazine.

The \emph{Saturday Evening Post} is in a class by itself. Its
sophistication content of 88 per cent is the highest of any of the
magazines examined, and its emulative content of 45 per cent is second
only to \emph{Harper's Bazaar}, which is 68 per cent. A third of the
\emph{Post}'s readers have incomes of over \$5,000 a year. They can
afford to play this emulative game and the \emph{Post} as a commercial
enterprise duly exploits this fact in its fictional content.

There are four magazines in which the older culture is dominant: the
\emph{American Magazine}, \emph{Harper's Magazine}, \emph{Nation's
Business} and the \emph{American Weekly}. In \emph{Harper's Magazine} we
find perhaps the most typical expression of the ``cultured''
upper-middle-class tradition, as it carries over from the nineteenth
century. The readers of \emph{Harper's} are given no emulative stimulus
whatever, except in the ads. The sophistication rating is 37 per cent.
\emph{Harper's} ranks fourth in this respect. In the \emph{American
Magazine}, the prewar, precrash culture persists. In particular, this
magazine continues to exploit the fictional formula of the prewar
culture. Its preoccupation with the pretty romantic aspects of courtship
reveals how strong is the cultural lag against which the hard, galvanic,
emulative culture battles. In its articles and editorials, the
\emph{American} appeals to the small city and small town American man,
who admires business success, bristles alertly about politics, and
believes that the world is inhabited by villains and kind people, with
the kind people in a position of dominance.

In the \emph{American Weekly} we encounter another emulation zero. Its
readers are urban proletarians, too poor to play the emulative game. The
Hearst formula realizes that they are strongly interested in sex: 65 per
cent, but that they are even more interested in science. Three times as
much space is devoted to science as to sex. True, the science is of a
primitive sort, like Paul Bunyan's ``Tales of the Blue Ox.'' Typical
\emph{American Weekly} titles are: ``The Sleeping Habits of the
Chimpanzee,'' ``The Growth of the Iron Horse Since the Six-Wheeled
Locomotive,'' ``Chicago Observatory Telegraphs to the Dead,'' ``Why Our
Climate Is Slowly Becoming Tropical,'' ``What the Tower of Babel Really
Looked Like.'' The \emph{American Weekly} is quite simply concerned with
serving a satisfying dish of weekly thrills. The technique is robust
since the modern world is full of wonders and the appetites of the
readers are not complicated.

The \emph{Nation's Business} is another very special case. This magazine
is the official organ of the United States Chamber of Commerce, while
the \emph{Saturday Evening Post} might be thought of as its unofficial
organ. The \emph{Nation's Business} ranks with the \emph{Saturday
Evening Post} in point of sophistication. Its editorial content is
devoid of emulative appeal and even the advertisements rate remarkably
low in these respects; only 9.6 per cent of the ads appeal to emulation.

It would be a commonplace to remark that most of the editorial content
of these magazines is quite ephemeral. Fifty years hence the literary
historian will probably have little difficulty in condensing the
creative contribution of our total commercial magazine-press during the
postwar period into a brief dismissive paragraph to the effect that the
fugitive literature of this period was ugly, faked and frail. After one
has diligently read this curious stuff over a period of weeks, one
begins to see our contemporary magazine pseudoculture as an almost human
creature. It is a robot contraption, strung together with the tinsel of
material emulation, galvanized with fear, and perfumed with fake sex. It
exhibits a definite glandular imbalance, being hyperthyroid as to
snobbism, but with a deficiency of sex, economics, politics, religion,
science, art and sentiment. It is ugly, nobody loves it, and nobody
really wants it except the business men who make money out of it. It has
a low brow, a long emulative nose, thin, bloodless, asexual lips, and
the receding chin of the will-less, day-dreaming fantast. The stomach is
distended either by the abnormal things-obsessed appetite of the
middle-class and the rich, or by the starved flatulence of the poor.
Finally it is visibly dying for lack of blood and brains.


\subsection{The Role of Emulation}

In anatomizing this pseudoculture we must refer again to our definition
of culture as the sum-total of the human environment to which any
individual is exposed, and point out again that the test of a culture is
what kind of a life it affords not for a few but for all of its
citizens. One grants immediately that emulation has a place in any
genuine culture. It is a question of balance, and the point here made is
that the quantity and kind of emulation exhibited by the magazine
pseudoculture is such as to affect adversely and probably disastrously
the viability of this synthetic creature that the magazines offer us.
Specifically, snobbism appears to be the antithesis of sex. Where the
first is dominant, the other tends to be recessive.

An analysis of the entire contents of the thirteen magazines shows that
sex and emulation are the principal appeals in the subject matter.
Sentiment occupies on the average only 1.8 per cent of the total space
in the magazines, humor only .9 per cent. In the advertisements there is
more emulation than sex. The average appeal to sex in the ads in the
thirteen magazines is 9.6 per cent, the average appeal to emulation is
14.7 per cent. In the subject matter sex continues to dominate
emulation. This is particularly true in the fiction where 55 per cent of
the stories have sex as the main appeal. Emulation, however, occupies no
inconsiderable place in the magazines. Twenty-two per cent or one-fifth
of the subject matter is concerned with emulation.

There is one generalization about emulation as it appears in these
magazines that can safely be made, emulation is not a commodity that can
be offered to the poor. Not even the lower middle-class can afford it.
It is distinctly for the well-to-do and for the rich. While fear is the
dominant appeal in the advertising sections of seven magazines which are
read by the lower income class, emulation is the dominant appeal in the
advertisements of six magazines which go to the upper income-levels. For
example: in \emph{True Story}, 42 per cent of the ads are fear ads. In
contrast, \emph{Harper's Bazaar} has no fear ads, and 35 per cent of
the ads are devoted to emulation.

Emulation is, of course, most apparent in magazines in which the
acquisitive, emulative culture is undiluted, like \emph{Harper's
Bazaar}, \emph{Arts \& Decoration} and \emph{Photoplay}. In the previous
chapter, ``Chromium Is More Expensive,'' we have already quoted
emulative editorial advertising taken from the first two of these
magazines. A few brief examples of snobbism, chosen not only from these
magazines but from the general list of magazines, will perhaps
illustrate the prevalence of snobbism and its character.

\begin{quote}
(1) ``It was a subtle satisfaction that no big social affair was
considered complete without us. 'Were the Roger Browns there?' was the
regular question in the aftermath of gossip.'' (\emph{True Story})
\end{quote}

\begin{quote}
(2) ``\,'She's one of the Mount-Dyce-Mounts.' 'One of the
Mount-Dyce-Mounts,' echoed John unbelievingly, and forgetting all about
Jean, he hurried down the steps ... and went up to where the old lady
had settled herself in a chair. John introduced himself with a charming
air.'' (\emph{Liberty})
\end{quote}

\begin{quote}
(3) ``\,'I keep only one groom so I help to look after my ponies myself
in the morning. I did not stop to take off my coat, because I was afraid
I might miss you. Excuse.' He removed his duster solemnly. In his tweed
coat and well-worn riding breeches, his costume conformed to type.''
(\emph{Woman's Home Companion})
\end{quote}

\begin{quote}
(4) ``He's a hotel aristocrat. You're a country gentlewoman. I'm so glad
it's all over. How wise Dr. Fancher was not to announce the
engagement.'' (\emph{Saturday Evening Post})
\end{quote}

\begin{quote}
(5) ``Now for the problem of the Christmas gift, for, despite the
pleasure we all must surely feel in giving gifts to our friends, the
choosing of gifts is indeed a problem, and the problem lies mainly in
avoiding the banal.'' (\emph{Harper's Bazaar})
\end{quote}

\begin{quote}
(6) ``Those who are demanding 'contempora' are in a sense the patrons of
modern design. Just as the Church was at one time, and the King at
another.'' (\emph{Arts \& Decoration})
\end{quote}

\subsection{The Role of Sex}

Before plunging into the jungle of our magazine sex fiction it will be
necessary to establish certain points of reference.

\begin{enumerate}
\item
  The biological norm of the sex relation tends to assert and re-assert
  itself against the religious and other taboos of the social
  environment, and against the limitations and frustrations of the
  economic environment. In other words, the readers of the magazines are
  both biological and social animals who would doubtless like to be
  human, to live balanced, vigorous and creative sexual and social
  lives.
\item
  Theoretically, the magazines, in so far as they deal with sex at all,
  are trying to instruct and aid their readers in solving their problems
  of sexual adjustment within the existing framework of the economy and
  of the mores. Since the writer of fiction or verse exhibits directly
  or indirectly a set of values, the verse and fiction writers are
  inevitably affecting, for good or ill, the values and attitudes of
  their readers in regard to sex. There are also the articles which deal
  with sex directly.
\end{enumerate}

Against this background, let us now attempt to describe what actually
goes on in these magazines. The exploitation of the sexual dilemmas of
the population by advertisers will be given consideration in the chapter
on ``Sacred and Profane Love.'' In the fictional and verse content of
the popular magazines we have another, less direct form of exploitation.
We know who writes the advertisements and why. It is necessary now to
ask: who writes the sex fiction and why?

The first point to note is that very little of it is written by literary
artists. There is a categorical difference between the equipment,
attitude and purpose of the literary artist who deals with sex
relations, and the equipment, attitude and purpose of the sex
fictioneer.

The work of the artist is a work of discovery, including self-discovery,
and of statement. In the field of sex the mature artist exhibits neither
timidity nor shame. True, the artist is often, like other human beings,
the victim of biological or socially acquired defects, inhibitions and
distortions, both physiological and psychological. Hence much genuine
literature in the field of sex must be characterized as in a sense
compensatory writing. It would seem probable, for example, that
practically all the work of D. H. Lawrence is of this nature, as well as
some, at least, of the work of Walt Whitman. But both these writers,
being genuinely gifted artists, are concerned only with the presentation
of the observed or intuitively perceived truth; they are concerned with
discovery. They are serving no ulterior purposes, and are in one sense
writing primarily for themselves. And being strong natures, they assert
their own values, attitudes, judgments, for value judgments are implicit
in the most ``objective'' writing.

In contrast, the commercial sex fictioneer is primarily concerned, not
with the discovery and statement of truth, but with the making of money.
If, as ordinarily, his is a tenth rate talent, his maximum service lies
in the telling of a tale; but in the telling he illuminates little or
nothing. At his worst the sex fictioneer is merely commercializing an
acceptable formula; he is ``selling'' the pseudoculture to itself; he
does nothing creative with the current sexual fact or with the current
sexual make-believe; he does not even achieve clear statement.

In this commercial sex fiction, the pattern is cut to the requirements
of the editor, who specializes in calculating what can and cannot be
said within the limits of a commercial enterprise designed to acquire or
hold a certain class or mass circulation. It is a fairly complex
calculation, and much study and experiment are required before the
apprentice sex fictioneer gets the editorial ``slant'' of a particular
magazine.

Of the thirteen magazines examined, \emph{True Story} is the only one
which definitely claims to offer sex instruction to its readers.

\begin{quote}
``Until five years ago,'' said a full-page advertisement, ... ``there
was nowhere men and women, boys and girls, could turn to to get a
knowledge of the rules of life.... Then came \emph{True Story}, a
magazine that is different from any ever published. Its foundation is
the solid rock of truth.... It will help you, too. In five years it has
reached the unheard-of circulation of two million copies monthly, and is
read by five million or more appreciative men and women.''
\end{quote}

While \emph{True Story} is certainly a commercial enterprise, and while
an unsympathetic commentator might well allege that it was specifically
designed to exploit the postwar relaxation of the sexual mores, it is
nevertheless true that \emph{True Story} is immeasurably closer to
reality than any of the other twelve magazines examined. This, in spite
of the fact that most of its ``true stories'' give internal evidence of
being fake stories, nine-tenths of which are written by formula and
perhaps one-tenth by high school graduates eager to become writers.

The distinction of \emph{True Story} rests on the fact that it admits
that sexual temptations sometimes occur and are sometimes yielded to;
also that it deals with matrimony rather than courtship. Its limitation
is its virtuous surrender to the Puritan conviction that an
extra-marital slip is a sin, inevitably followed by remorse and
retribution.

Of eleven stories and articles in the issue examined, six have sex for a
major theme and five of these stories deal with matrimonial
difficulties, i.e., sexual temptations not evaded. One must, of course,
point out that no true description of the sexual behavior of the poor is
to be derived from \emph{True Story}, although there are scenes in which
a married woman prepares the room for the reception of her lover and
receives him. What true descriptions we have must be looked for in the
work of such novelists as Edward Dahlberg, James T. Farrell, Erskine
Caldwell and Morley Callaghan. The \emph{True Story} formula, in its
negative and positive aspects, runs somewhat as follows: sinner
redeemed, sinner pays, sinner repents, saint sacrifices all; the beauty
of duty, of security after a narrow escape from losing one's reputation
and job; the beauty of being a true wife, the beauty of resignation, of
truthfulness, and of character.

After a particularly lurid escapade the \emph{True Story} heroine is
obliged to say something like this: ``If every silly, sentimental fool
in this sad old world could have witnessed that scene, it would have
done an enormous amount of good. Many a home would have been saved from
ruin. They would have known the tempting Dead Sea fruit of illicit love
for what it was, giving a bitter flavor to life for all who taste it.''

Obviously, the success of Mr. Macfadden's enterprise is based on the
profitableness of bearing witness.

An analysis of 45 sex stories from ten magazines, including \emph{True
Story}, yields much interesting material for speculation. But as regards
the technique of sexual behavior the harvest is meagre indeed. We were
able to discover only four items of premarital and two items of
postmarital technique.

Premarital technique: How a mother can recognize the first sign of love
in her adolescent son (\emph{Woman's Home Companion}). How to approach a
virgin (Data in a number of stories, but all very meagre and
questionable). How, if a girl is careful and smart she can take
everything and give nothing (\emph{American Weekly}). Why an unmarried
woman who wishes to seduce a youth should avoid tragic diversions such
as those incident to the mistake of taking along her pet goat
(\emph{Harper's Bazaar}).

Postmarital technique: How to commit bigamy. How to kill a drunken
husband and thereby improve one's social status.

In addition to the information about technique, the 45 sex stories
present the following conclusions about sex, sex and economics, and
morals:

\begin{quote}
\emph{Men:} ``All men are pretty dumb and clumsy. There might be men
somewhere who lived up to the things the poets, novelists and musicians
said of men. If so, she had never met them.''
\end{quote}

\begin{quote}
One man may be able to arouse a frigid woman, while another may not.
\end{quote}

\begin{quote}
A man will bet on his ability to pluck the bloom from a virgin, and then
not want it.
\end{quote}

\begin{quote}
A genius is not bound by the moral code of Puritanism.
\end{quote}

\pagebreak \begin{quote}
\emph{Marriage:} The sex revolution of the postwar era led to
unhappiness.
\end{quote}

\begin{quote}
After ``sleeping around,'' actually or mentally, a married couple's
chance of happiness is with each other.
\end{quote}

\begin{quote}
Through reading light, trashy stuff a woman may lose her husband.
\end{quote}

\vspace{0.1mm}

\begin{quote}
\emph{Sex and Economics:} Millions cannot buy love. A mercenary woman
cares more for her car than for her husband. A rich girl is smart if she
marries a poor boy who has brains. Since a poor girl is often no good,
it is safer to marry a rich girl.
\end{quote}

\vspace{0.1mm}

\begin{quote}
\emph{Morals:} Virtue is more attractive than vice. An ``indiscretion''
can strip a woman of her good name, rob her of her freedom, and cost her
every penny she has in the world. A common-law marriage may ruin a man's
social position years later. A married couple should be an example to
other married couples and to unmarried persons.
\end{quote}

These conclusions and the six technical points represent \emph{all} that
is to be gained from this magazine sex fiction.

Of the 45 sex stories examined, only 13 were straight sex stories. The
complications introduced in the remaining 32 are as follows:

\begin{quote}
Thirteen: economics plus sex; eleven: romance plus sex; five: the
American scene plus sex; two: the sex revolution; one: religion plus
sex.
\end{quote}

It is worth noting that although complications due to intermarriage of
races and nationalities might be expected, practically nothing of this
sort was encountered.

It should be emphasized that this magazine sex literature centers around
women rather than around men. The problems of men are considered in only
three of the 45 sex fiction stories. It is also significant that men
outnumber women in the cast of characters; a surplusage of men is
necessary properly to dramatize the feminine dilemma. This surplusage of
men is more pronounced as we ascend the class ladder. The woman of
\emph{True Story} hopes for no more than a single lover. The
middle-class heroine must have at least the choice of two. The grande
dame of \emph{Harper's Bazaar} requires a circle of adoring youths with
beautiful bodies, including at least one millionaire.

So frequently does the theme repeat itself in this magazine sex fiction
that we feel warranted in saying that the dominant desire of the woman
is to be freed from some situation in which she is bound or caught. But
in only two instances out of the 45 (the sex revolution stories) does
the heroine herself initiate positive action toward such liberation. The
most that the average heroine permits herself is to give some clue to
her prospective liberator. Out of a wealth of data we submit the
following quotations which serve best to reveal the typical heroine's
attitude:

\begin{quote}
``Restlessness, dissatisfaction possessed her. She wanted more---more,
somehow, than life was giving her. Other women were happy---sometimes
such stupid, plain, elderly women were happy, but she was continually
fretted and harassed by this sense of missing something---of being
cheated.'' (Kathleen Norris. ``Three Men and Diana.'' \emph{The
American})
\end{quote}

\begin{quote}
``I had Wanted Out. Always I had Wanted Out. Yet whenever I had tried to
find a door---when I had taken some great risk, like marriage, in order
to find the door---I had failed. There had been no door. Then, suddenly,
in some unexpected place the door would open!'' (Elsie Robinson. ``I
Wanted Out.'' \emph{Cosmopolitan}, April, 1934)
\end{quote}

All these fiction heroines want happiness, of course, but it is notable
that they get happiness only in the romantic moment which precedes
marriage. Stories of happy married life are entirely lacking in the
samples examined. Significant class differences characterize the
behavior of these heroines. The extravagance of the rich woman in the
matter of lovers has already been indicated. The shifting milieu of
these stories would also seem to show a class difference.

In Class ``A'' magazines the scene is always Europe, the Swiss Alps,
Scotland, England, the Riviera. America is ignored geographically. In
the Class ``B'' magazines the geography is mixed; Africa, London, the
Oregon of the gold rush, a fresh water college town, New England,
Chicago, New York and Hollywood. In the Class ``C'' magazines with only
a few exceptions the locale is America---the poor don't travel. The
typical scene is the country or small town, New England, Chicago, New
York and Hollywood. It would appear that Hollywood is the Riviera of the
proletarian as well as to a considerable extent the focus for the dreams
of the middle-class woman.

The following table indicates the range of fiction heroines encountered
by class categories. Note that the typical rich heroine is mercenary,
the typical middle-class heroine is an unawakened or unresponsive woman,
and the typical poor heroine is sexually responsive as well as
biologically more prolific. In magazine fiction as well as in life the
poor woman has the largest number of babies. While the 41 fiction
heroines of the middle-class produce only three children, the eleven
fiction heroines of the poor produce nine children.

\vspace{2mm}

\begin{center}
SEX FICTION HEROINES
\end{center}

\vspace{1mm}

\noindent MERCENARY WOMEN:

\addvbuffer[8pt 15pt]{\begin{tabu} to 0.9\textwidth { X[l] X[r] } 
 Class ``A'' Magazines & 51 per cent \\
 Class ``B'' Magazines & 10 per cent \\
 Class ``C'' Magazines & 10 per cent \\
\end{tabu}}

\vifll

\pagebreak

\noindent UNRESPONSIVE WOMEN:

\addvbuffer[8pt 15pt]{\begin{tabu} to 0.9\textwidth { X[l] X[r] } 
 Class ``A'' Magazines & 56 per cent \\
 Class ``B'' Magazines & 45 per cent \\
 Class ``C'' Magazines & 17 per cent \\
\end{tabu}}

\noindent RESPONSIVE WOMEN:

\addvbuffer[8pt 15pt]{\begin{tabu} to 0.9\textwidth { X[l] X[r] } 
 Class ``A'' Magazines & 45 per cent \\
 Class ``B'' Magazines & 34 per cent \\
 Class ``C'' Magazines & 17 per cent \\
\end{tabu}}


As to inter-class relationships the typical fictional device is the
Cinderella theme, either straight, Poor Girl Marries Rich Man, or in
reverse, Poor Boy Marries Rich Girl, the latter being apparently more
popular. Proletarian characters are frequently encountered in Class
``A'' sex fiction. It would appear that the readers of the Class ``A''
magazines like to parasite emotionally upon the richer sexual life of
the poor.

The bulk of American magazines are read by the middle class, the \$2,000
to \$5,000 income group. In the case of ten magazines which we have
selected as representative types, 51 per cent of the circulation goes to
the middle class. Twenty women's magazines, studied by Daniel Starch,
show about the same percentage; 57 per cent of them have middle-class
readers. The fact that the middle-class woman is the principal reader of
mass and class circulation magazines is important to keep in mind in
considering what we feel to be one of the significant findings of the
study. The editor of the typical mass circulation magazine, usually a
man, addresses himself primarily to the restless unhappy middle-class
woman. The fiction exploits rather than resolves this unhappiness, just
as the advertising exploits the emulative things-obsessed psychology of
this woman, which it would seem arises chiefly from her sexual
frustration. Here are two quotations which exhibit the condition of this
middle-class woman.

\begin{quote}
(1) ``Quite suddenly, without warning, Diana realized that her marriage
had been a losing fight. A mistake as far as her own interior happiness
was concerned.... She could still go on gallantly---picking
strawberries, heating rolls, brewing coffee. But somehow the glamour,
the excitement was gone. Neal seemed to be just a man, she just a woman,
there seemed no particular reason for their being together.'' (Kathleen
Norris. ``Three Men and Diana.'' \emph{American Magazine})
\end{quote}

\begin{quote}
(2) ``The second period in a woman's life is when, after many strenuous
years of adjustment toward husband and family, she feels entitled to let
her own personality have full scope. She wants to forget as much as
possible those difficult years, she wants to live her own life, to
entertain her own friends in her own background. By this time plain
Romeo has turned into Mr. Romeo Babbitt, \emph{but there is no Mrs.
Babbitt}. There is instead a gracious woman in the prime of life who has
matured in excellence like old wine and the cask must be adequate.''
(Daisy Fellowes. ``Home, Sweet Home.'' \emph{Harper's Bazaar})
\end{quote}

We have already noted the inverse ratio of sex deficiency and emulation.
Material emulation and snobbism are apparently substitutes for sexual
satisfaction. From the point of view of a commercial publisher
interested in achieving a maximum ``reader interest'' for his
advertisers the ideal subscriber to a middle-class woman's magazine is
the woman who has never experienced the full physical and emotional
satisfactions of sex; who is more or less secure in her economic
position and who determinedly compensates her sexual frustration by
becoming an ardent and responsive buyer.

One of the most frequent charges leveled against American culture is
that it is woman-dominated. Women, it is said, read the books, attend
the concerts and exhibitions, run the charities, figure increasingly in
politics, etc. The inference is that our cultural deficiencies are
caused by this domination of the woman, for which various explanations
have been offered.

Our examination of the magazine literature leads us to question the
accuracy of this picture. Is it women who have created this ad-man's
pseudoculture? Is it women who own and direct these commercial
enterprises of mass publications? No, it is predominantly men. It may
also be alleged that it is the stupidity of men which is largely
responsible for the sexual and emotional frustration of the typical
middle-class woman. The result of the middle-class woman's physical or
emotional frustration is not that she compensates by achieving a culture
superior to that of the man. A much truer statement would be that the
exploitation of the dilemma of these women by men has helped to bring
about the collapse of culture in the United States. It is significant to
note in this connection that it is precisely in the women's magazines
that sophistication tends to disappear. Of the five women's magazines
examined, four devoted less than three per cent of their article and
editorial space to sophistication.

In summarizing the sex content of the magazines it is sufficient merely
to note that it is almost incredibly thin and vapid, useless as
instruction, and deficient in thrills.


\subsection{Religion, Art, Science}

In the thirteen magazines examined, we find God mentioned once in a
fiction story and twice in poems. Art is mentioned only by \emph{Arts \&
Decoration}. Science, which gets full if crude treatment in Hearst's
\emph{American Weekly}, is encountered in only one other magazine,
\emph{Liberty}, which contains a story by Edgar Rice Burroughs, ``Tarzan
and the Lion Man,'' in which the author has a paragraph or two about the
imaginary genesis of his hybrid.

\subsection{The Role of Sophistication}

Of the four criteria for sophistication referred to in earlier chapters
only one, the treatment of the depression, proved to be important in
quantity or revealing in content. \emph{Photoplay}, \emph{Arts \&
Decoration} and \emph{Harper's Bazaar} do not mention the depression at
all. The negative response to the depression takes the form of a
repudiation of the acquisitive culture and a turning back in time to the
older American virtues and the older American pattern of life.

\begin{quote}
(1) ``Looking back {[}to the days when her husband, now a farm-hand, had
an \$8,000 a year salary{]} it seems as if we never found anything
very---very real to quarrel about. And the queer thing is I know we were
both rather clever then. We weren't stupefied with work, the way we are
now. I suppose that must be the answer. If I weren't too tired to think
clearly, I'd be able to see some sense to it. It actually seems as if
there were more dullness and stupidity in those smart squabbles about
books and plays and clothes and places to eat than there is in sitting
here---like dumb animals, too tired to talk, contented because we're
warm, and fed, and alive.'' (Hugh McNair Kahler. ``Winter Harvest.''
\emph{Saturday Evening Post})
\end{quote}

\begin{quote}
(2) ``Jonathan could not understand his sister's passionate loyalty to
the old house. He worshipped the modern, the technical, the efficient.
It was this that had made him persuade his brother to abandon the
leather factory, with its century-old reputation for honesty and fair
dealing and follow the will-o'-the-wisp of fortune with the vacuum
cleaners. Their story was the story of dozens of small industries.
\end{quote}

\begin{quote}
``\,`Listen to me, Jonathan,' said Charlotte coldly, `I want to read you
a few lines from this book.' She read, her voice trembling with the
intensity of her feeling:
\end{quote}

\begin{quote}
``\,`Never the running stag, the gull at wing,
\end{quote}

\begin{quote}
The pure elixir, the American Thing....'
\end{quote}

\begin{quote}
``\,`It's that---``The American Thing''---we've got away from it, from
everything we stood for. And now we're going back to it.... Look at the
farmers. They've got food they can't sell but no money. We'll take their
leather goods in exchange for food and hides.' ...
\end{quote}

\begin{quote}
``\,`But that's barter,' Jonathan gasped.
\end{quote}

\begin{quote}
``\,`Savagery.'
\end{quote}

\begin{quote}
``Bartlett looked at her steadily.... `Barter,' he said, at length.
`Ancient as the hills and modern as tomorrow'.'' (Francis Sill Wickware.
``The American Thing.'' \emph{Woman's Home Companion}.)
\end{quote}

In considering the positive response to the depression a brief summary
of the essential characteristics of these class cultures will be useful.
In magazines read by the poor, fear and sex are dominant and emulation
is negligible. The middle-class are immunized against fear, exhibit a
definite sex deficiency and are strong in emulation: they are the
climbers. In magazines going to the rich, fear reappears, and sex is
exploited chiefly for its mercenary or amusement value. Since these
magazines primarily exploit the climbing \emph{nouveau riche}, emulation
is very strong and is reinforced by a tremendous preoccupation with
``things.'' An example of the mercenary characteristic of the rich as
exhibited in the high income magazines is the following:

\begin{quote}
``\,`My dear Mr. Sherrard,' he said, `as a man of the world, you will at
once comprehend the situation. My wife and I are devoted to each other;
unfortunately, we have no money. Not-a-single-sou.' He paused to let
this sink in, then continued blandly as before. `Our tastes are what
might be described as traditionally extravagant. We can't help it, we
inherit them from our ancestors. Together, our life, save for a few
moments of bliss, is impossible. Apart, we simply cannot prevent---I
repeat, \emph{cannot prevent}---money coming to us in large quantities. It
is odd.'
\end{quote}

\begin{quote}
``\,`Very,' agreed Sherrard.
\end{quote}

\begin{quote}
``\,`I know what you are thinking: that it would be more noble to starve
than acquire such money. But then we are not noble-men that way'.''
(Margery Sharp. ``Immoral Story.'' \emph{Harper's Bazaar.})
\end{quote}

Where, in a transitional period, do the readers of magazines think they
are going? Before attempting to answer this question, it is worth noting
that the letters from readers warrant the belief that the readers are
going somewhere much faster than the editors would like.

The \emph{American Magazine} represents the lower middle-class male; the
\emph{Saturday Evening Post}, the upper middle-class male;
\emph{Nation's Business}, the rich. How do the men of these different
classes regard the future of business and of government? The
\emph{American Magazine} is behind the New Deal sturdily and
optimistically. None the less, in a pinch it is clear that the typical
\emph{American Magazine} reader would go fascist. This is revealed by
the general direction of the articles and by readers' letters. The
\emph{Saturday Evening Post} is belligerent and not frightened. The
creed of the \emph{Post} is to repel every invasion of business by the
government. It professes to believe that business is capable of running
the country without government aid. Whenever this illusion breaks down
the magazine alertly serves its readers by offering optimistic
adaptations to the necessities of the moment. The \emph{Post}'s high
point of sophistication is registered in the following quotation which
is the concluding paragraph of an article by Caret Garrett entitled
``Washington Miscellany.''

\begin{quote}
``The law of necessity hitherto acting {[}before the Roosevelt
Administration{]} was a law of nightmare. For that it is proposed to
substitute a law of the disciplined event. To say this has never
happened is not to say it cannot happen. But certainly it was by the
other way that the world grew as rich as it is, which is richer than it
ever was before.''
\end{quote}

The \emph{Nation's Business} is too near, perhaps, to the seats of power
not to have looked over the edge of the precipice and to have become
doubtful. ``Capital is Scared,'' it headlines, and in recording the
timidity of investors remarks: ``In other words they wonder whether or
not the days of private capitalism are numbered.'' Curiously the editor
of \emph{Nation's Business} seems to be less confident that Fascism is
our next phase than are the editors of the Communist \emph{Daily
Worker}. In reading the articles and editorials of \emph{Nation's
Business} one gets the impression that these frightened business men of
Wall Street, and of the provincial chambers of commerce, would not be
surprised if they awoke tomorrow morning to find the revolution on their
doorsteps.

With regard to the poor, our magazine indices are \emph{True Story} and
the famous Vox Pop of \emph{Liberty}. It seems clear that \emph{Liberty}
readers comprise a high percentage of war generation males, especially
Legionnaires. Their notion of a revolution would appear to be a
miraculous change of political administration whereby suddenly everybody
would get \$5,000 a year. In the lack of such miracles they advocate
homespun nostrums like the scrapping of machines, going back to the
land, etc. While it is clear that the readers of \emph{Liberty} are not
sophisticated radicals, labor legislation, technological unemployment,
and the revolution get mentioned in the Vox Pop pages. Whether the
\emph{Liberty} readers go fascist or communist would appear to depend
upon the energy and astuteness which one or the other party manifests in
proselytizing and mobilizing them.

\emph{True Story} is a mine of sophistication data regarding the poor.
The editors write about the family problems created by the depression
and invite contributions on the subject from their readers, but the
absorption with these problems is clearly evident in the fiction as
well. To the poor, poverty is a perpetual problem, in good as well as in
bad times. It is the unique distinction of \emph{True Story} among the
magazines examined that it is the only one which contains stories about
the poor. Despite the fakery which is apparent in much of this fiction,
there is also much genuinely revealing stuff. In the issue examined,
four of the nine fiction stories deal with the working class and two
deal with the very poor.

As already noted, the fiction writers for \emph{True Story} recognize
that the way out for the poor is crime. In the following quotation there
is presented a typical white-collar depression dilemma. The story
concerns a burdened father who, unwilling to seek the way out through
crime, kills himself in such a way that his family may collect the
insurance and pay their debts.

\begin{quote}
``\,`You know, Lois, the rottenest part of it all is Dad,' he said
slowly.... `Dad hasn't had much out of life. Mother's a swell person in
her way, but she's certainly made his life miserable. He's crazy about
us---about all his kids---but we've cost him an awful lot and I don't
think we've given him much in return. When I look at Dad and think of
all the years he's striven beyond his strength, of all the things he's
gone without to give us things---of how little he's had out of life, I
get sick inside. He's a man made for cheerfulness, and freedom and
happy-go-lucky ways. And he's been harnessed to routine and duties and
schedules all his life. And for what? He's ended in disgrace and
failure. No matter what we think---and we don't think he's a disgrace
and a failure---that's what it boils down to in the eyes of the world.
\end{quote}

\begin{quote}
``\,`A letter from Papa a letter.... He's going to commit suicide....
He's doing it for us.... You can see for yourself. He thinks he's no
good, and that he'll never land another job at his age. He wants to
leave us his insurance. He knows that'll wipe out every debt we have and
start us fresh. It's all he has to give and he's willing'.''
\end{quote}

\begin{quote}
(``Desperate Days.'' \emph{True Story}.)
\end{quote}

The alternative to crime as a way out would appear to be suicide. But
what happens when the poor do essay crime as a way out of their
dilemmas? The following quotation is taken from a story dealing with the
very poor.

\begin{quote}
``It was the first motion picture I had ever seen, despite the fact that
our little hamlet had boasted two shows weekly for many years.... We
walked ten miles to the next town.... Jimmie's pockets were bulging with
the life savings of his aunt, while he let me believe the money was
rightfully his.... In my talks with Jimmie, I came to see a change in
him. He laughed about the decencies of life, about the people who worked
hard for their bread, about the poor people who stood for oppression
from the rich.... The well defined line between right and wrong seemed
to grow fainter as the days passed. Sometimes I thought Jimmie was right
about the unfairness of things and our privilege to make up for it
outside the law....
\end{quote}

\begin{quote}
``Jimmie was sentenced first, and taken to prison several days before my
sentence was fixed. As he passed the women's cells, I could hear him
singing `Let the Rest of the World Go By.' He was trying to be a good
sport.... Club women called on me and tried in their mechanical way to
preach morals to me. Their visits served only to antagonize me. All the
time they were talking, my heart cried out `But you've had a chance in
life. You had love and home and friends. I didn't want to steal. Jimmie
was sick, and I was scared he'd die, if I didn't help him get the
stuff.' My lips did not form the words. In fact I hardly spoke to them
at all. I scowled my hatred at them, and saved my tears for my
pillowless bunk.''
\end{quote}

\begin{quote}
(``His Mother's Confession.'' \emph{True Story}.)
\end{quote}

The conclusion indicates that crime, that is theft, is no way out after
all since the wages of crime is jail. It is estimated that the poor,
that is to say, those having less than \$2,000 a year, constitute over
75 per cent of the total population. Where are they going in this
transitional period? It seems clear that a considerable percentage of
the readers of \emph{True Story} are desperate and cynical about the
possibility of escape from their dilemmas by any other route than the
crime route. Clearly that route is being increasingly followed as
Abraham Epstein notes in ``Insecurity, A Challenge to America,'' when he
points out that since the depression the total value of insurances
policies lapsed for inability to pay amounts to \$3,000,000,000, and
that the prisoners admitted to Sing Sing for robbery have increased by
70 per cent. It would seem apparent that here we have a nexus of
potential revolutionary material, inert at the moment, but capable of
mobilization by an able revolutionary leader who could show a practical
way out, other than the way of crime.

Recently in talking to a group of business men who were re-focusing
their advertising expenditures upon the narrowing sector of the
population which represents any exploitable buying power, I raised the
question as to what business intended doing with these extra-economic
men. The answer was ``Nothing.'' The assumption so far as I could gather
seemed to be that the surplusage of the population would starve
peaceably and eliminate itself. I recommended the reading of \emph{True
Story} to these bemused plutocrats. It seems very clear that the readers
of \emph{True Story} will not starve peaceably.

Here then we have the spectrum of the ad-man's pseudoculture as revealed
by its mass and class magazine literature.

Is it desirable to rehabilitate this ad-man's pseudoculture? The
question is somewhat beside the point since history does not evolve by a
series of moral or esthetic choices. A culture is rejected, not because
it is ugly and unjust, but because it is not viable. The more pertinent
question, therefore, is: ``Is it \emph{possible} to rehabilitate this
pseudoculture?'' The answer here is the same answer which must be given
to the question: ``Is it possible to rehabilitate the capitalist
economy?'' The capitalist economy can survive as long as it can validate
its rising mound of paper titles to ownership and income by the
enslavement of labor and by progressive imperial conquests. The
capitalist culture---the ad-man's pseudoculture---can survive as long as
it can give some substance to the traditional concept of individual
opportunity; the ability of the able individual to rise out of his
class. The economy and the culture are Siamese Twins; or rather, they
are aspects of the same thing. Examination of this magazine literature
reveals clearly that the democratic dogma is dying if not already dead;
that the emulative culture is not accessible to the poor and to the
lower middle-class; that the poor are oriented toward crime, and
potentially at least, toward revolution; that the middle classes are
oriented toward fascism. In short, the ad-man's pseudoculture is not
satisfying. To be effectively exploited it must be diluted with elements
derived from the older culture and with some measure of sophistication
and service, particularly with respect to the lower income groups. Its
decadence parallels rather strictly the decadence of the capitalist
economy. Historically, the ad-man's pseudoculture will probably be
regarded as a very frail and ephemeral thing.

We must therefore conclude that this culture, or pseudoculture, is not
viable, hence cannot be rehabilitated. This conclusion will be regarded
as optimistic, or pessimistic, depending upon the point of view of the
reader.



% CHAPTER SEVEN
\chapter[7 \hspace*{1mm} THE NATURAL HISTORY OF ADVERTISING]{7 THE NATURAL HISTORY OF ADVERTISING}

 \newthought{ASK} a child who is just beginning to read: ``What is a newspaper? What
is a magazine?'' He will speak of news and fiction and advertising as
integral parts of the same thing. Explain and argue as much as you like,
you will not be able to disturb his primitive conviction that the
advertising is not just as much a part of the paper as the news, and
that, if the thing is to make sense, it has to make sense as a unit.
Tell him that the news and editorials represent one thing, one
responsibility, one ethic, one function, one purpose; that the
advertising represents another thing, another responsibility, another
purpose. He nods vaguely and gives it up.

In other words, the child's instinct leads him to precisely the same
conclusion as that set forth and documented in the preceding study of
the magazines.

Advertising, in the broadest sense of the word, is as old as trade. The
definition offered by Frank Presbrey in his \emph{History and
Development of Advertising} would seem to be sufficiently broad and
accurate. To quote it again: ``Advertising is printed, written, or
graphic salesmanship deriving from oral salesmanship.''\footnote{{[}Frank Presbrey,
  \emph{\href{http://www.worldcat.org/oclc/700109}{The History and
  Development of Advertising}} (Garden City, NY: Doubleday, 1929).{]}} The modern
spread and intensified use of the instrument in America is made possible
by our almost universal literacy. But ancient graphic and written
advertising exhibits a functional relationship to the then current nexus
of economic and social fact which is strikingly similar to the
contemporary set-up.

The Babylonian temples were built of sun-baked bricks. Each brick was
stamped with the name of the temple and the name of the king who built
it. The temples were advertising, just as the Woolworth and Chrysler
Buildings are advertising. There is even some justice in Presbrey's
observation that these temples represented ``an institutional campaign
conducted by the kings in behalf of themselves and their dynasties.''

The Rosetta Stone is a eulogy of Ptolemy Epiphanes, dating from 136
B.C., in three languages: Coptic, hieroglyphs and Greek. It was erected
by the local priests in gratitude for a remission of taxes. The priests
were, in effect, the local satraps of Ptolemy and the Rosetta Stone was
functional with respect to the discharge of their responsibility. It was
necessary to ``sell'' Ptolemy to the people, and probably the priests
acted at the suggestion, certainly with the approval of their overlord.

When President Roosevelt was inaugurated he proceeded more directly.
Using the modern instrumentality of the radio, he sold the American
people on the closing of the banks and the incidental wiping out of
perhaps \$6,000,000,000 of their savings. The priests---the radio
broadcasters---contributed free time, and the other priests---the
newspapers---contributed enthusiastic approval and applause. With the
evidence of this and later triumphs of government-as-advertising before
us, those primitive Babylonian practitioners seem hopelessly outclassed.

Since literacy was the privilege of a minority, the Babylonian tradesmen
used barkers and symbols. Later, inscriptions were employed. Lead sheets
found in ancient Greek temples affirmed the rights of property by
cursing the sacrilegious people who did not return lost articles to
their owners. In ancient Greece the arts of elocution and music were
functional with respect to trade; the Greek auctioneer was an
elocutionist and was usually accompanied by a musician.

The word ``libel'' is Latin. In ancient Rome a libel was a public
denouncement of an absconding debtor.

It seems probable that advertising was more or less professionalized in
very ancient times. For example there is some reason for believing that
the walls of ancient Pompeii may have been controlled by a commercial
contractor. Early posters were inscriptions announcing theatrical
performances and sports, and commending the facilities of commercial
baths. Presbrey renders one such advertisement as follows: ``The troop
of gladiators of the sedil will fight on the 31st of May. There will be
fights with wild animals, and an awning to keep out the sun.''

With the break-up of the Roman Empire, advertising shared the general
obscuration of the middle ages. Says Presbrey, ``For nearly a thousand
years, following the decline of Rome, advertising made no progress.
Instead, it went backward, following the retreating steps of
civilization.''

When the profession re-emerges, it is under the changed conditions of
the medieval church-state. A decree of Philip Augustus in 1280
proclaims:

``Whosoever is a crier in Paris may go to any tavern he likes and cry
its wine, provided they sell wine from the wood and there is no other
crier provided for that tavern; and the tavern keeper cannot prohibit
him. If a crier finds people drinking in a tavern he may ask what they
pay for the wine they drink; and he may go out and cry the wine at the
prices they pay, whether the tavern keeper wishes it or not, provided
always that there be no other crier employed for that tavern.''

The ``just price'' for which the crier served was four \emph{dinarii} a
day. It was further provided that if the tavern keeper closed his door
against the crier, the latter might cry wine at the price of the king's
wine, and claim his fee.

Perhaps the last proviso gives a clue to the motivation of Philip
Augustus' proclamation. The king was in the wine business, too, and was
accordingly interested in the education and expansion of the market. The
king's wine was to be sold at a given price, which provided a measuring
stick for competition and was doubtless a factor in price maintenance.

As one might expect, the re-birth of advertising coincides with the
expansion of trade in Western Europe made possible by the suppression of
piracy and banditry by the Hanseatic League. In the sixteenth century
the chief form of advertising was the poster. It was called a
\emph{si-quis} (if anybody), the derivation being from the Roman lost
article posters. Most \emph{si-quis} were want advertisements. The chief
billboard in London was St. Paul's Cathedral, which was crowded with
lawyers, seamstresses, etc., seeking clients. Like the modern office
building or railroad terminal the sixteenth-century church also
contained tobacco shops and bookstalls. Tobacco, coffee and books were
among the first products advertised. It is in connection with the
exploitation of literature by advertising that one encounters, with a
glow of pleasure, no less a person than Ben Jonson, in his usual r\^ole of
objector and satirist.

In \emph{Every Man out of his Humor}, one of the characters is Shift,
who haunts St. Paul's ``for the advancement of a \emph{si-quis} or two,
wherein he hath so varied himself that if any of them take he may hull
himself up and down in the humorous world a little longer.'' By 1600
handbills and placards in behalf of books became so common that Jonson
enjoined his bookseller to use his works for wrapping paper rather than
promote them by the sensational methods then in use.

The objection is particularly interesting as coming from Jonson, who,
although he had been successively a bricklayer, a soldier and a
playwright, was by nature a scholar-poet, and an intellectual
aristocrat. He probably felt, like the modern historians Morrison and
Commager, that advertising had already ``elevated mendacity to the
status of a profession.'' He tolerated the noble patrons to whom he
dedicated his works because they helped to support him; but he clearly
despised the ``new people,'' the middle-class business men, who, having
tasted the sweets of profit in the expanding market, were marshaling
their forces for the later conquests of manufacturing and commerce.

Art was conscripted into the service of trade when Hogarth was employed
at making inn signs and illustrating handbills for tradesmen, including
one advertising himself as an engraver and another for his sisters, who
were designers of frocks.

By the end of the seventeenth century the apparatus of poster and
handbill advertising was functioning at full blast within the limits set
by the still primitive facilities of transport and communication.
Practically all the stigmata of the modern practice of advertising were
present. The greed and social irresponsibility of the advertiser
expressed itself in sweeping claims and cheerful misrepresentation; his
tastelessness in bad art and worse English. The seventeenth century
trader was a go-getting fellow---a low fellow coming up, with nothing to
lose in the matter of social status and a world of profit to gain. The
nobility and the princes of the church denounced him; city ordinances
were passed in London threatening with severe penalties tradesmen who
were so immodest as to advertise the prices of their wares. But the
advertiser met scorn with scorn and drove the logic of his acquisitive
opportunity always harder and higher. A French visitor to London in the
middle of the eighteenth century comments on the huge and ridiculous
ornamentation of the shop signs. As some of the early prints made us
realize, the streets of seventeenth century London were scarcely less
vulgar and commercial than the Great White Way of modern New York.

Business, however, still lacked its major tool, the press. It is upon
the evolution of this instrument that we must now concentrate our
attention.

It cannot be too strongly emphasized that the press begins and ends as
an instrument of government, whether official or unofficial, actual, or
potential and aspiring. What it is today it was in its earliest
beginnings. The invention of printing approximately coincided with the
early struggles for power of the rising middle class. In this long chess
game, with its shifting alliances, its victories, defeats and drawn
battles and its unstable truces, the press is the queen without whose
support the king, the official ruler, is helpless: a most bawdy,
promiscuous and treacherous queen, whose power is today threatened by a
new backstairs mistress, the radio. The press has played virtuous, even
heroic r\^oles in the past, and still does. But on the whole, she is like
Archibald MacLeish's poet in his \emph{Invocation to the Social Muse}:
She sleeps in both camps and is faithful to neither.

Although the press is and always was an instrument of government, it is
even more important to point out that the press came to birth as an
instrument of trade, which was aspiring to be government. From her
earliest memory the infant Messalina was rocked in the cradle of
business.

In 1594 the French philosopher Montaigne published an essay entitled
\emph{Of a Defect in our Policies} in which he urged the establishment
of exchanges for tradesmen and buyers. As a result a ``Bureau
D'Affiches'' was established in Paris. It functioned for only a brief
period and was followed by a quite obvious technical advance, the
publication of a \emph{Journal D'Affiches} (Journal of Public Notices)
which is said to be the first periodical in the history of Western
Europe. The first issue appeared Oct. 14, 1612. It was a want-ad medium,
no more and no less---newspaper of, by and for trade, and this it has
continued to be for more than 300 years. It is now called \emph{Les
Petites Affiches}, and is still a periodical of want-ads and public
notices. An humble and virtuous creature, \emph{Les Petites
Affiches}---the Martha of newspaperdom. Let us keep her in mind when we
come to study the careers of her successors and rivals, the Marys,
Ninons, Carmens and Messalinas who have relegated her to her present
comfortable and respectable bourgeois obscurity.

Trade, then, was news, and trade plus printer's ink became advertising,
but still news. Abortive public registers were chartered by James I and
Charles I in England. Henry Walker published his \emph{Perfect
Occurences} in 1649---this being a house organ for his Public Register
or \emph{Enterance}. But government was jealous of the emergent fourth
estate. \emph{Perfect Occurences} was suppressed in 1650 and Walker's
Public Registry, being deprived of advertising, soon died.

But the forces of the trading class, with God, as usual, conscripted
under their banner, were marching toward the conquest of power. In 1657
Marchmont Needham, Cromwell's official journalist, was publishing the
bi-weekly \emph{Mercurius Politicus} and \emph{Publick Intelligencer}.
He established eight offices of ``public advice'' in London and in 1657
obtained permission from Cromwell to issue, in addition to the news
letter, a weekly sheet called the \emph{Publick Adviser}. All the
advertisements, then called ``advices,'' were of the same size. The fees
were four shillings for a workman, five for a bookseller and ten for a
physician. Needham had a monopoly advantage and used it ruthlessly.
When, a little later, he raised his prices, the indignant tradesmen
denounced him as ``The Devil's Half-Crown Newsmonger.''

Since the news letter was a medium for the literate exclusively, it was
natural that booksellers were among the earliest advertisers. But the
medicine man and the realtor were also early on the scene. Since the
mass market for food and clothing was not yet literate, such advertisers
do not appear until later. At this point it is merely important to note
that trade, for its full development, required universal literacy, and
that the later use of public funds for school purposes was conceivably
motivated less by idealistic considerations than by the needs of trade.

Cromwell's Ironsides were business men out for power and marching under
the banner of God. They needed spiritual food, and when Cromwell marched
into Scotland, a newsbook was published for distribution to his army of
``Saints.'' Here are some specimen titles of the books advertised in
that publication, all of them obviously good selling copy for the
Puritan conquest of power, just as, nearly three centuries later, Bruce
Barton's \emph{Man Nobody Knows} became the bible of our modern Rotarian
saints, marching under the banner of ``Service'':

\begin{quote}
\emph{Hooks and eyes for Believers Breeches}\\
\emph{A Most Delectable Sweet Perfumed nosegay for God's saints to smell
at.}\\
\emph{The spiritual Mustard pot to make the Soul Sneeze with Devotion.}
\end{quote}

Upon the restoration in 1660 Charles II quickly put a stop to that. He
recognized the growing power of the press by suppressing it. Instead, a
two-page publication was issued called the \emph{London Gazette}. It
refused to carry advertising on the ground that commercial announcement
had no place in a ``paper of intelligence,'' that is to say, a newspaper
which presented non-commercial news. As a matter of fact the
\emph{London Gazette} was an official government newspaper and is still
published as such. Later in the reign of Charles II it did publish
advertisements, but in a separate sheet. The monarchy continued to
regard the press as a government function and privilege. In 1665 Roger
L'Estrange was given a patent as ``Surveyor of the Press'' which
included the exclusive privilege of ``writing, printing and publishing
advertisements.''

The amiable monarch was not averse to making a little money out of
trade, although he doubtless considered the upstart tradesmen as
permanently objectionable. The poet, Fleetwood Sheppard, who was one of
his favorites, doubtless expressed the royal view when he wrote the
following criticism of current advertising practice:

\begin{quote}
They {[}the current newsbooks of the year 1657 when this was written{]}
have now found out another quaint device in their trading. There is
never a mountebank who either by professing of chemistry of any other
art drains money from the people of the nation but these arch-cheats
must have a share in the booty, and besides filling up his paper, which
he knew not how to do otherwise, he must have a feeling to authorize the
charlatan forsooth, by putting him into the newsbook.
\end{quote}

Yet Charles II himself, shortly after his accession, was obliged to turn
advertiser, as witness the following plaintive appeal to his rascally
subjects:

\begin{quote}
We must call on you again for a Black Dog between the greyhound and a
spaniel, no white about him only a streak on his breast, and tayl a
little bobbed. It is His Majestie's own dog, and doubtless was stolen.
Whoever finds him may acquaint any at Whitehall, for the dog was better
known at Court than those who stole him. Will they never leave robbing
His Majesty? Must he not keep a dog?
\end{quote}

By the middle of the eighteenth century a considerable press, whose
principal support derived from advertising, was established in England
and on the continent. The essence of the modern phenomenon had been
achieved and its essence was clearly recognized by contemporary
commentators. We may therefore conclude this outline of the early
history of advertising with the following quotation from Dr. Samuel
Johnson, writing in the \emph{Idler} in the year 1759:

\begin{quote}
Advertisements are now so numerous that they are very negligently
perused, and it is therefore become necessary to gain attention by
magnificence of promises and by eloquence sometimes sublime and
sometimes pathetic. Promise, large promise, is the soul of an
advertisement {[}Promise them everything and blow hard, said my early
tutor, the sea lion{]}. The true pathos of advertisements must have sunk
deep into the heart of every man that remembers the zeal shown by the
seller of the anodyne necklace, for the ease and safety of the poor
toothing infants and the affection with which he warned every mother
that she would never forgive herself if her infant should perish without
a necklace.... The trade of advertising is now so near to perfection
that it is not easy to propose any improvement. But as every art ought
to be exercised in true subordination to the public good, I cannot but
propose it as a moral question to these masters of the public ear,
whether they do not sometimes play too wantonly with our passions.
\end{quote}

Dr. Johnson wrote as a good liberal of his period and his phrases have a
familiar ring. He might almost have been reviewing a volume by Stuart
Chase or applauding the demand of Messrs. Schlink and Kallet for a new
law to restrain the iniquities and hypocrisies of advertising. In
justice to these writers one must acknowledge both the value of their
exposures and the even more significant fact that all three have moved
steadily leftward in their political orientation.

What the good doctor did not see---and contemporary liberals seem
scarcely more acute---was that, given a literate population, the press
\emph{becomes} one of the instruments of government; that if the press
is financed by the vested property interests of business, then in the
end business becomes government. Finally, the good doctor should have
realized the futility of introducing moral and ethical values into a
trade relationship. The concepts of ``good'' and ``bad'' suffer a sea
change in this relationship; good advertising is advertising which makes
profits and bad advertising is advertising which does not make profits.
Neither the ``regulative'' attempts of government nor the idealistic
campaigns of reformers in and out of advertising will seriously affect
the economic determinants which operate in this relationship. At least
they haven't for over three hundred years.

Dr. Johnson felt that the art of advertising had reached approximate
perfection in the middle of the eighteenth century. In a sense he was
right. The archetypes of contemporary technical practice are almost all
to be found in the newspaper and handbill advertising of that period.
The later developments have been chiefly those of speed and spread,
with, however, this qualification: these developments have brought into
being a series of interlocking vested interests, which, while entailed
\emph{effects} of the underlying economic process, have also come to
function as important causes, influencing and even determining to a
considerable extent the subsequent evolution of our civilization.

The point of view adhered to in this book is that of regarding the
instruments of social communication as \emph{instruments of rule},
\emph{of government}. In this view the people who control and manage our
daily and periodical press, radio, etc., become a sort of administrative
bureaucracy acting in behalf of the vested interests of business. But
every bureaucracy becomes itself a vested interest; it develops its own
will to expansion and power. Bureaucracies are likely to be what
governments die of. In Russia a bureaucracy was set up, theoretically,
to solve the tasks of socialist construction, and gradually, with the
coming to birth of the classless society and the elimination of the
conflicts which the state power must adjust or suppress, to ``wither
away.'' The Russians are frank in confessing that they are obliged to
fight the tendency of their bureaucracy to propagate itself verdantly.
This struggle in fact has been and is one of the most difficult tasks of
the socialist construction.

In the following chapter we shall consider two other instruments of
rule, namely education and propaganda, and show how the use of these
instruments is frequently combined with the use of advertising.





% CHAPTER EIGHT
\chapter[8 \hspace*{1mm} THE THREE GRACES: Advertising, Propaganda, Education]{8 THE THREE GRACES: Advertising, Propaganda, Education}
\chaptermark{8 THE THREE GRACES}

\newthought{Modern} advertising reaches its highest expression in the United States
and under the political and social forms of our democratic institutions
and concepts: a free press, popular education, representative
government. It is important to note that the contemporary phenomenon is
an aspect of our so-called ``surplus economy,'' as is revealed by the
use of the phrase ``sales resistance'' in current advertising parlance.
``Sales resistance'' means an impedance of the distributive function. It
implies a lack of spontaneous demand for the product or service which
may be caused,

\begin{enumerate}
\item
  By the inferiority of the product as to quality or price with respect
  to competing products.
\item
  By the inertia of established buying patterns in the market at which
  the product is aimed.
\item
  By the inter-industrial competition, as for example, brick against
  lumber or meat against cheese.
\item
  By the inadequacy of the class or mass buying power with respect to
  the volume and price of commodities and services offered on the
  market.
\end{enumerate}

Although existing buying power is ultimately determinative, it is
possible to manipulate consumer preferences and the division of the
consumer's dollar within this iron limit. In other words the market can
be ``educated''---or propagandized---as you choose to put it, just as it
can be partially or wholly monopolized and the controls established with
respect to volume of production, distribution and price. These are,
perhaps, the two major factors in the obsolescence of the ``law'' of
supply and demand.

The education, or manipulation of the market may proceed directly
through the advertising of the product by the manufacturer or by a group
of manufacturers organized as a trade association; through unsigned
publicity prepared and issued by the manufacturer or his agent; through
the more or less influenced or coerced ``co-operation'' of the daily or
periodical press, radio and cinema; even through similar influences or
coercions focused upon our institutions of formal education. Sometimes
all four methods are used. A few typical examples will illustrate the
nature of the process, its detailed exposition being left for other
chapters.

It happens that a single manufacturer dominates the market for
automobile tire valves, caps and gauges. He stands to profit, therefore,
by any expansion of this market. Hence his advertising has tended to be
primarily ``educational''; that is to say, it tells motorists that
proper inflation adds to the durability of tires, that improper
inflation is dangerous; that the air pressure in tires should be
frequently tested, hence the motorist should own his own gauge; that the
valves require more or less frequent replacement.

Note that all this ``education'' is sound enough on the whole and in the
consumer's interest as well as that of the manufacturer and distributor.
Such education, or promotion, can be achieved more economically, on the
whole, by publicity than by advertising, since the publicizing of the
manufacturer's name and the brand name of his product, is, while
desirable in view of actual or latent competition, not essential.

Many newspapers and magazines carry columns of advice to motorists; the
editors of these automobile sections and pages can readily be persuaded
to publish small items urging motorists to keep the tires of their cars
properly inflated; especially if the manufacturer or his agent does the
whole column in which the advice about tires is mixed with other
standard bits of information and warning. This relieves the newspaper or
magazine staff of labor and expenditure; sometimes a staff member, or a
journalist having working relations with several publications, is
induced to do the job for a fee paid by the manufacturer, and then see
that the ``education'' or promotion is duly published. \emph{But} such
arrangements are precarious unless the newspaper or magazine gets some
quid pro quo. Hence an educational publicity campaign of this kind is
usually correlated with a minimum expenditure for paid advertising.
\enlargethispage{\baselineskip}
There is nothing unusual about such procedures, nor is any violation of
the current business code involved. True, the technique requires the
application of interested economic pressures. But so does the technique
of security promotion represented by the Morgan preferred list. In so
far as moral or ethical judgments are applicable to such procedures it
would seem futile to apply them to the individuals involved; rather,
they should be directed, not merely against the existing business code,
but against the system under which such codes naturally develop.
\clearpage
Another example. General Motors sells automobiles and advertises them in
the \emph{Saturday Evening Post}, which is one of the reasons why the
\emph{Post} can pay high prices for articles and fiction and yet sell
for a nickel. But the fact that General Motors and other automobile
manufacturers advertise in the \emph{Saturday Evening Post} also serves
to explain certain elements in the editorial content of the magazine.
The \emph{Post} by reason of its advertising lineage becomes an
important and profitable business property, one of a group of business
properties. Hence the editorial policy of the \emph{Post} is inevitably
conservative in its policies. With equal inevitability its editorial
management is favorably disposed toward the specific interests of its
advertisers. The \emph{Post} may or may not consider itself primarily an
advertising medium; it is so regarded by the advertiser and his agent.
The advertising manager of the \emph{Post} must be prepared to show that
the \emph{Post} is a profitable medium, a favorable medium; that the
editorial content of the magazine is favorable to, and supplements, the
message of the advertiser.

\emph{Saturday Evening Post} readers will perhaps recall that automobile
fiction stories appear recurrently in that magazine; that these and
other stories are often illustrated with happy and prosperous people in
automobiles. Naturally the artist is not permitted to make recognizable
a particular make of automobile.

The implication must, of course, be qualified before it can stand. It
would be expected in an automobile age that automobiles should figure in
much contemporary fiction. It would be impossible for the \emph{Post},
which solicits and publishes advertising of all kinds of products, to
emphasize unduly in its editorial columns the use of any particular
product.

But it would also be bad business not to utilize the editorial content
of the magazine to increase its value to advertisers, and that is
exactly what is done as a matter of course, not merely by the
\emph{Post}, but by many other newspapers and magazines of large
circulation, such as \emph{Good Housekeeping}, \emph{House and Garden},
\emph{Arts and Decoration}. It is inevitable, since the publication is a
business enterprise, that the business accounting should extend to the
editorial as well as the advertising management; the deciding vote in
any issue is naturally that of the advertising management.

American children, even a heavy percentage of the children of working
class parents, brush their teeth. They have been taught to do so. By
whom?

By the manufacturers and advertisers of toothbrushes and toothpastes,
operating directly through signed advertisements in newspapers and
magazines, indirectly through the co-operation of the dental profession,
indirectly through the more or less syndicated ``health talks''
published in newspapers and magazines, indirectly through the teaching
of hygiene in the schools. The co-operation of the dental profession is
secured by the distribution of free samples to dentists, the
solicitation of salesmen, etc: but also and more importantly it is
sought by ``constructive educational'' advertising in which the
advertiser urges the reader to ``visit your dentist every six months'':
such campaigns---that of the S. S. White Company, manufacturer of dental
chairs, mechanical equipment, supplies, etc., is an excellent
example---are in turn ``merchandized'' to the dental profession in the
professional publications. ``Merchandizing'' consists essentially of
advertisement of advertisements. The manufacturer points out to the
dentist how much he is doing to ``educate'' the public to patronize the
dentist, the implication being that in consideration for the
manufacturer's expenditure in such ``constructive'' publicity, the
dentist might well recommend the particular product to his patients. In
the case cited the product was a good one, made according to a formula
prepared by an eminent dentist, and the advertising copy more or less
aggressively de-bunked the unscientific ``talking-points'' of competing
dentifrices. A number of manufacturers, notably Colgate, have followed
this policy; others, such as Forhan's, Pepsodent, Ipana, etc., have
found it more profitable to select a particular half-true talking point,
exaggerate it, use the simple technique of fear appeal, and while
continuing to seek the co-operation of the dental profession, discount
the opposition of the more sensitive and ``ethical'' section of the
profession.

Education of another sort, secured through fostering the newspaper and
magazine propaganda of ``health talks,'' ``preventive dentistry,'' etc.,
can rarely be made to benefit the interest of any particular
manufacturer. In general such education is likely to be sound enough in
intent, and at least harmless in effect, although sometimes objected to
by dentists on the ground that it is insufficiently critical and
informative, and does not---could not, since the publication is an
advertising medium---take issue with the bunk which is spread on the
advertising pages. If the press were or could be a disinterested
educational instrumentality it might be expected to correct the
mis-education sponsored by its advertisers, but then, if the press
functioned in the interests of its readers rather than in the interests
of its advertisers, it would not publish pseudo-scientific, more or less
deceptive advertising. Again, the press is merely an advertising
``medium''; not until the ghosts which use this medium to materialize
their more or less sprightly profit-motivated antics---not until these
ghosts are exorcized can we expect the press to be anything except
precisely what it is. Ethical judgments are pretty much irrelevant. A
``good'' medium is not a medium which materializes only good ghosts; a
``good'' medium is a medium through which ghosts, good, bad and
indifferent can manifest themselves effectively. True, the more
respectable mediums are prejudiced against the more disreputable ghosts
and exclude them from their pages. But such prejudices and exclusions
are also likely to be economically rather than ethically determined; the
antics of the respectable ghosts require, for their maximum
effectiveness a decent parlor half-light, not the bawdy murk in which
the direct-by-mail peddlers of aphrodisiacs, abortifacients, and
contraceptives squeal and gibber. And the bigger and better ghosts spend
more, and more reliably.

Another form of indirect education---that which makes use of our public
schools---has both its positive and negative aspects. A familiar example
of the positive use of this ``medium'' of formal education is the
``toothbrush drill'' taught children in the primary grades.
Manufacturers of toothbrushes and of dentifrices have used and benefited
by this technique almost equally. They have enabled school boards to
economize by supplying free or at cost the literature used in teaching
dental hygiene, including various trick devices for making education
amusing to the young. Such education is neither very good nor very bad
in and of itself. But if a competent teacher or school nurse happens to
believe, as do many dentists, that the toothbrush is a dubious blessing;
that it should be used in strict moderation if at all; that the use,
say, of dental floss, is considerably more valuable hygienically---such
a school functionary is likely to encounter the pressures by which
heretics are disciplined---unless she can get the dental floss
manufacturers to spring to her aid. And finally, advertised toothbrushes
and dentifrices are likely to be absurdly overpriced; education which
results in teaching children to buy overpriced toothbrushes and
dentifrices when the use of ordinary table salt, with the occasional use
of dental floss, would constitute on the whole a more hygienic as well
as more economical regimen---such education has a certain unmistakable
ghostly quality. But the negative aspect of the advertising controls
operating on our publicly owned schools is vastly more important. In
recent years a new specialty has appeared in the teaching of economics;
it is called ``consumption economics'' and concerns itself with the
consumer as a factor in the economic scheme; how can the consumer best
serve his own interest? What is an intelligently balanced budget for a
given income level? What items should be bought and how can such items
be bought most economically? What are the possibilities and limits of
such developments---still embryonic in America---as consumers'
co-operatives, credit unions, consumers' research, etc.

On the surface there would seem to be merit in this idea of
``consumption economics.'' But ask the secretary of your local chamber
of commerce, or the business manager of the local paper, or any
prominent retailer what \emph{they} think about it. Or ask some of the
consumption economists, such as Robert Lynd, author of
\emph{Middletown}, just how far they have got in their attempts to
introduce such courses in the schools.\footnote{{[}Robert S. Lynd and Helen Merrell
  Lynd,~\emph{\href{http://www.worldcat.org/oclc/1001579439}{Middletown:
  A Study in Contemporary American Culture}}~(New York: Harcourt, Brace
  and Co., 1929).{]}} The writer asked such
questions; the answers were somewhat disheartening. In conclusion he
asked an even more naive question: to whom do these public schools
belong anyway? The answer, of course, is that they belong to the people,
since \emph{all} the people, directly or indirectly, pay taxes for their
support. But their use in the interest of \emph{all} the people is
simply impossible, because the interests of the people are divided and
conflicting. In the case of ``consumption economics,'' any attempt to
perform for the masses of the population even the modest service which
Consumers' Research performs for its 50,000 subscribers---an expert
measurement of the qualities and values of products and services offered
for sale---is and will be met by the united opposition of business and
the allies of business: manufacturers, distributors, bankers,
publishers---all the people who profit quite legitimately by selling
products and services in as great a volume as possible and for as much
more than they are worth as the traffic will bear: all these people and
all the people whose political voices they control: their employees,
wives, sisters, uncles, aunts and cousins---even perhaps some of the
cousins who would like to consider themselves disinterested school
superintendents and teachers serving the interests of all the people.
The opposition is unqualified and rigorous. Business men are also in a
sense educators. They use advertising and its related devices and
techniques to ``educate the consumer,'' to ``break down sales
resistance''; your earnest ``consumption economist'' would like to use
education to build up sales resistance. But let him try to do it.
Anybody who would want to cut the Gordian knot of this ``educational''
dilemma with the liberal sword of ``ethics'' is welcome to his pains.

In these few examples we have encountered advertising, propaganda and
education as parts of a single economic nexus. It becomes necessary at
this point to define these categories more sharply and to show their
interrelations.

The complex of phenomena is economic, institutional, technical,
psychological, whereas the tendency of current criticism by liberal
publicists has emphasized invidious ethical judgments. Yet it is only by
re-defining such value judgments that the play of forces can be
accurately described and analyzed. It is even more important to avoid
the artificial isolation of phenomena which superficial moral and
ethical criticism engenders. What we are dealing with is the
institutional and ideological superstructure of competitive capitalism.
Whether we take our cue from Marx or merely from the respectable social
ecologists, we may be sure that the mutual interaction of social
phenomena, whether categoried as economic, sociological or
psychological, is an immitigable fact; that when we seem to find
isolate, perverse and irreconcilable elements in the picture, we are
merely victims of our own thought patterns, for there can be nothing
mysterious or isolate about the phenomena. The contemporary French
historian, Andr\'{e} Siegfried, is obviously aware of the continuity and
mutual interaction of the social and economic phenomena we have been
describing when he writes, in \emph{America's Coming of Age}: ``Under
the direction of remarkably intelligent men, publicity has become an
important factor in the United States and perhaps even the keynote of
the whole economic structure.''\footnote{{[}Andr\'e Siegfried,
  \emph{\href{http://www.worldcat.org/oclc/639874}{America Comes of Age:
  A French Analysis}} (New York: Harcourt, Brace and Co., 1927).{]}}

Note that M. Siegfried is using ``publicity'' as an inclusive term to
denote all forms of advertising, propaganda and press agentry. The
writer would both widen and sharpen this inclusion by showing that the
apparatus of newspaper and periodical publishing, radio broadcasting,
motion picture production and distribution; with the conjoined apparatus
of advertising agencies, public relations experts, and dealers in
direct-by-mail, car card, and poster advertising, constitute in effect a
single institution; further, that the institutions and techniques of
formal education, both secondary and collegiate, are also closely
related and functional within the general scheme; that the purpose and
effect of these conjoined institutions and techniques is rule; the
shaping and control of the economic, social and psychological patterns
of the population in the interests of a profit-motivated dominant class,
the business class.

The necessity of such broad inclusions in any systematic analysis of the
phenomena becomes apparent when we come to define our major categories.
The definition of advertising offered by Frank Presbrey in his
\emph{History and Development of Advertising} is as follows:
``Advertising is printed, written, or graphic salesmanship, deriving
from oral salesmanship.''\footnote{{[}Frank
  Presbrey,~\emph{\href{http://www.worldcat.org/oclc/700109}{The History
  and Development of Advertising}}~(Garden City, NY: Doubleday,
  1929).{]}} This, of course, should be corrected to
include radio and motion picture advertising, but otherwise may be
allowed to stand. The point to be emphasized is that the practical
advertising man views all these instruments of communication
---newspapers, magazines, radio, motion picture---as \emph{advertising
media}; that this is in fact the accurate, realistic and significant
view to take of these instruments of social communication, whereas the
thought patterns of liberal laymen tend to make them appear to represent
some sort of ideal functional relationship between editor and reader, or
broadcaster and Great Radio Public---a relationship which these curious
parasitic growths, advertising and publicity, are insidiously, immorally
perverting. The layman sees that the tail is wagging the dog. The
advertising man knows that the tail is the dog and acts accordingly. He
knows that there is no real separation between the business and
editorial offices of a modern publication; that where such a separation
appears to obtain it is purely a management device, designed to insure
the more effective functioning of the publication as an advertising
medium. He knows, for he is called in as a ``publisher's consultant'' to
plan and execute the job---that the conception of a modern commercial
publication starts with the definition and segregation of a particular
buying public, which may be recruited and held together by a particular
type of editorial policy and content. The publisher's consultant sees an
unoccupied, or insecurely occupied niche in the crowded spectrum of
daily and periodical publishing. The publication is thereupon concocted
to the specifications necessary to entertain or inform that particular
section of the buying public. The objective is not attained, however,
until the circulation so recruited is sold to advertisers at so much per
head, the charge being based on the average buying power and the
demonstrated ``reader-interest'' of the readers. ``Reader-interest'' is
measured by response to advertising and the editorial content of the
magazine is carefully designed, as already indicated, to strengthen this
response. You pay your money and you take your choice, depending upon
the nature of your product or service and the methods by which it is
promoted. The readers of \emph{True Romances}, for example, are poor but
numerous and credulous, whereas the readers of \emph{The Sportsman} are
comparatively few, but very rich---and susceptible to the arts of
flattery and sycophancy. In both cases the collaboration of the
editorial and business managements is intimate and accepted as a matter
of course. Criticism of such arrangements by the more or less obsolete
criteria of an ideal reader-editor relationship is beside the point,
since the determinants are the objective forces of the competitive
capitalist economy.

In propaganda we encounter a phenomenon even more disturbing and
puzzling to liberal publicists and sociologists, especially since the
experience of the war demonstrated the dominance of this technique of
social control in modern societies. Again, contemporary students have
been frustrated by their tendency to view the phenomenon as isolate and
adventitious.

The latest book on propaganda, which digests and summarizes much that
has been written on the subject by contemporary sociologists and
publicists, is \emph{The Propaganda Menace} by Professor Frederic E.
Lumley, of Ohio State University. Professor Lumley experiences much
difficulty in reaching a satisfactory definition of propaganda. After
rejecting innumerable definitions offered by contemporary educators and
sociologists, he offers us the following:

\begin{quote}
Propaganda is promotion which is veiled in one way or another as to (1)
its origin or sources, (2) the interests involved, (3) the methods
employed, (4) the content spread and (5) the results accruing to the
victims---any one, any two, any three, any four, any five.\footnote{{[}Frederick E. Lumley,
  \emph{\href{http://www.worldcat.org/oclc/250273333}{The Propaganda
  Menace}} (New York: The Century Co., 1933), 44.{]}}
\end{quote}

In Professor Lumley's view the contrasting opposite to propaganda,
necessary in defining any term, is ``education.'' And it is precisely
there that his definition falls down, because of the highly conditioned
and shifting quality of the latter concept. More or less aware of these
confusions, aware that education must be related to some conception of
social change, Professor Lumley takes refuge in the relatively
sophisticated and acute definition of education offered by Professor
Bode as follows:

\begin{quote}
When formal education becomes necessary in order to fit the individual
for his place in the social order, there arises a need for reflection on
the aims and purposes of education and of life. Many aims have been
proposed, but if we view intelligence from the standpoint of
\emph{development}, the conclusion is indicated that aims are constantly
changing and that education is, as a matter of fact, the liberation of
capacity; or in Bagley's phraseology, it means training for achievement.
To make this liberation of capacity or this process of growth a
controlling ideal means the cultivation, of sensitiveness to the human
quality of subject matter by presenting it in its social context. The
fact that a given type of education is classed as liberal or cultural is
no guarantee that it fosters this quality of mind. Unless this
sensitiveness is deliberately cultivated, many human interests, such as
business, science and technical vocations, do not become decently
humanized. And to cultivate this sensitiveness deliberately means that
it is made the guiding ideal for education.\footnote{{[}Boyd Henry Bode,
  \emph{\href{http://www.worldcat.org/oclc/9622917}{Fundamentals of
  Education}} (New York: Macmillan, 1922).{]}}
\end{quote}

In this definition Professor Bode recognizes the necessity of relating
education to social change. He does not, in the passage quoted, take
account of the dynamics of social change. One does not need to insist
upon a strict Marxian interpretation to describe the essential nature of
social change. It will be readily granted by most readers that the
conflict of pressure groups within the social order results in shifting
balances of power; that these pressure groups tend to represent economic
classes; that the issues of conflict tend to be economic at bottom; that
the basic cause of change is the changing level of the productive
forces---in our day the machine technology. This is not to ignore the
equally real r\^ole played by pressure groups in the fields of the social
mores, religion, race, etc., but merely to emphasize the economic and
class roots of this perpetual conflict, where propaganda is so
powerfully instrumental.

If this is so, then there are certain crucial undefined terms imbedded
in Professor Bode's definition. What, for example, is meant by ``fitting
the individual for \emph{his place} in the social order''? Obviously the
students whom Professor Bode proposes to educate after this fashion
occupy not the same but different places in our social order, which,
while retaining a certain residual fluidity manifests an increasing
rigidity and class stratification. To fit a third generation Rockefeller
for his place in the social order is obviously a task different from
that of fitting Isidore Bransky, son of a radical East Side pants maker,
for \emph{his} place, which is a matter of strictly limited but crucial
choice, depending upon whether young Bransky leaves his class or
doesn't; whether he is fitted to become a labor organizer, legal defense
worker, radical journalist or merely an energetic legal ambulance
chaser, political fixer or other capitalist functionary in business or
in the professions. Should or can the educator remain above the battle
as respects this choice? Will not the educational means by which
capacity is liberated necessarily affect it? Finally, would Professor
Bode attempt to deny that education in a typical university does
inevitably indoctrinate and that on the whole it indoctrinates in the
direction of conformity to the existing order? In honesty, must not the
teacher tell his student that ordinarily he must save his body by
serving an exploitative system and, if possible save his soul by helping
to destroy this system?

What is meant by presenting subject matter ``in its social context''?
Whose social context? Does Professor Bode mean by social context the
contemporary class conflicts of American capitalism exacerbated by the
internal and international conflicts of our ``surplus economy''? Does he
mean the perhaps imminent ``freezing'' of the capitalist structure into
the corporative forms of Fascism?

Returning to Professor Lumley, it might well be alleged that in urging
``education'' as a preventive and cure of the propaganda menace,
Professor Lumley is really writing propaganda for a particular concept
of education: the concept of an objective, disinterested effort to
release capacity. Further, it might be argued that this concept is
doomed to remain in the field of theory, since it is observably
nonexistent in practice. Finally, it may be suggested that to erect a
purely conceptual theory of education, while ignoring the contemporary
practices and very real economic determinants of educators and the
institutions they work for, is itself a kind of propaganda: propaganda
by suppression which is one of Professor Lumley's recognized categories.

The necessity of such realistic clarifications cannot be evaded, and to
Professor Bode's credit it must be said that he, at least in his later,
more advanced position does not try to evade them. With Dewey, Counts
and other modern educators he acknowledges frankly that the theory of
education propounded in the passage quoted above is applicable only in a
classless society.
\clearpage
Behold, then, this precious absolute, education, the hope of democracy!
The more we turn it up to the light, whether we examine its practice or
even its theory as expressed by leading educators, the more it dissolves
in relativity. And our crucial problem remains with us: what is
education and what is propaganda with respect to the problems of the
individual in our society, faced as it is, with the self-preservative
necessity of fundamental social change?

If it were only possible to posit an ideal disinterested objectivity on
the part of the educator, and an absence of pressure controls operating
upon our educational institutions, the problem would be greatly
simplified. But, as we have seen, leading educators properly discard
such claims. The facts of class interest and individual subjectivity
must be and now are, generally admitted. The coercions of the social
order, for achievement in which the student is trained, these, too, are
frankly acknowledged. Recently Dr. Abraham Flexner has noted with proper
but perhaps futile indignation the tendency to vocationalize our
institutions of higher learning, that is, to make them functional with
respect to the requirements of business, and to the survival necessities
of students. And we have with us always the issue of ``academic
freedom'': the degree to which a teacher is permitted to express views
in conflict with the economic and social status quo. The underlying
fact, of course, is that in both privately and publicly supported
educational institutions the interest and prejudices of the ruling class
are ultimately determining, whenever education enters the field of
contemporary social and political struggle.

Many teachers, even of the social sciences, are quite unconscious of
these determinants and preserve the confident illusion of ``scientific
objectivity'' in the very act of asserting creedal absolutes which are
obviously a product of social and economic class conditioning. Professor
Lumley is himself a conspicuous example of this. In his concluding
chapter he writes: ``No sane person wants \emph{revolutionary}
communistic propaganda spread in this country.'' Is this the language of
an objective, disinterested educator? Professor Lumley urges that
instead of deporting and lynching Reds, their agitation be combatted (1)
by destroying the soil of gullibility through education and (2) by
removing desperate need through liberal reformism. Such recommendations
may seem relatively enlightened and civilized, but they are not quite
sufficient to rehabilitate Professor Lumley in his r\^ole of disinterested
educator.

The dubiousness of his position would quickly appear under circumstances
such as the following: suppose that because of the disinterested
teaching of Dr. Lumley one of his students had escaped the
class-conditioned thought patterns of his family and friends, or that,
because of the logical capacities released by education he had broken
through these patterns. Suppose that this student, having acquired some
acquaintance with Marx, Engels, Veblen, Lenin and others, should elect
as the subject of his doctor's thesis \emph{The Position of the Social
Scientist under American Capitalism}. The application of the Marxian
analysis to this material might well result in ``revolutionary
communistic propaganda.'' Would Professor Lumley pronounce his student
insane and withdraw his fellowship? If not, should he not have to
consider himself insane for permitting the spread of ``revolutionary
communistic propaganda''?

One thinks of a third solution for this imaginary academic dilemma:
shove the student back into the educational mill and trust that on his
re-emergence he would have more sense. Then suggest to him, as an
interesting subject for a thesis, \emph{Paranoiac Traits in Modern
Radical Leaders}.

It is indeed difficult to escape the conviction that the god of
education, like other gods, is not merely man-made, but made by a
particular group of men as a rationalization of their r\^ole in the
complex struggle of social forces---of ``pressure groups'': further,
that the institutions built up to exemplify and discharge this
r\^ole---our schools and universities---are similarly subject to such
rationalized determinants. The claim of disinterestedness, of
universality, is also made for the press, although Professor Lumley has
no difficulty in seeing that the latter institution becomes inevitably
an instrument of pressure groups. The same claim is even made for
business, the instrument of profit-motivated property owners. All of
these claims are of course equally invalid; none of these institutions
is separate or self-sufficient; all are swept into the struggle of
conflicting social forces; advertising, propaganda and education are
inextricably merged and intertwined.

The contemporary fact of this confusion is excellently illustrated by
the propaganda activities of the National Electric Light Association, to
which Professor Lumley devotes an indignant chapter. The investigation
of the Federal Trade Commission and the writings of H. S. Raushenbush,
Ernest Gruening and others have familiarized most readers with the
theory and practice of this propaganda campaign in behalf of our
privately owned light and power corporations. It will be sufficient here
to point out that the instruments of advertising, propaganda and
education were all used in such a way as to reinforce each other, all
contributing to the crude economic objective of protecting and
conserving the vested interests of private property in exploiting for
profit an essential public service.

Direct, explicit, signed propaganda by the National Electric Light
Association and its member companies was used in the form of paid
advertising. This provided an economic leverage for the control of the
news and editorial content of the press as effecting the interests of
the light and power companies. Note that the press was in a bargaining
position. Newspaper publishers could and did on occasion threaten to
expose the iniquities of the ``power trust'' unless the local companies
could be brought to see the propriety of buying advertising space in
their papers. Once this concession was made, the papers willingly
``co-operated'' with the NELA campaign, by printing the propaganda
furnished by the publicity directors in the form of mats, boiler plate
and mimeographed releases. One interesting and important point is
totally missed by Professor Lumley. In the case of the NELA campaign, as
of other propagandas by vested commercial interests, what was in effect
a method of control by bribery (blackmail from the point of view of the
NELA) was practicable only with respect to the smaller and less powerful
newspapers, just as it was only the less eminent professors who accepted
fees for making speeches and writing texts favorable to the power
interests. Integrity, as Stuart Chase has pointed out, is a luxury in
our civilization. It is, with certain qualifications, one of the
privileges of wealth and power. No evidence was produced to show that
the NELA had bribed the \emph{New York Times}. Attempts were made to
influence the Associated Press, but that is a mutual corporation, in
which the pressure upon individual members backs up inevitably upon the
directing officials.

On the other hand, it is equally important to note that it wasn't
necessary to bribe the \emph{New York Times}, and that, stupid as the
NELA publicity directors proved themselves to be, they probably had more
sense than to try to bribe either the \emph{Times} or other major
publishing corporations. Yet the editorials in the \emph{Times}, and its
handling of public utility news, especially with respect to the private
\emph{versus} public ownership issue, have been pretty consistently favorable
to the power interests. Why? Obviously, because the \emph{Times} is
itself a major capitalist property. It is part of the complex of
financial, business and social relationships which produces what is
called a ``conservative'' point of view. The owners and managers who
express and make effective this point of view are often not aware of the
economic and social pressures which influence them. They act
unconsciously, much as an experienced driver operates an automobile---he
is ``part of the car.'' The specific allegiance rarely becomes overt and
fully conscious.

Respectable and powerful newspapers and magazines cannot be expected to
swallow and approve the rawer aspects of contemporary commercial
propaganda. The \emph{Times} duly slapped the wrist of the National
Electric Light Association, following the exposures of the Federal Trade
Commission. It did not go down the line for Mr. Doheny and Secretary
Fall during the Teapot Dome scandal, though from time to time it
deprecates Congressional investigations as in general ``bad for
business.''

Some service---not only lip service but actual service---is due the
concept of a ``free press'' and a modicum of such service can usually be
obtained even by radical minority groups. The amount and quality of such
service is determined by the circumstances of the individual case. The
major determining factors are: the inherent news value of the incident
and its relation to other current news; the success with which current
liberal concepts of free speech, legal rights, etc., can be appealed to;
the class origin and political orientation of the reporter who covers
the story; the current pressures of local, national and foreign news;
the reputation of the radical propagandist as a reliable news source;
the mass pressure brought to bear upon the newspaper.

The writer has served as a commercial publicity man, an advertising man
and as a radical propagandist. All these techniques require careful
measurement and utilization of the forces operative in a given complex
of public relations. Neither as a commercial propagandist nor as a
radical propagandist is it intelligent to act on the assumption that the
capitalist press is ``kept,'' to use the familiar half-true radical
jibe. It must always be remembered that the press has to ``keep''
itself; that it has its own particular values, traditions and technical
requirements to conserve. Although, primarily because of the dominance
of advertising, the press functions in general as an organ of business,
it functions with relation to circulations which usually include a
variety of more or less organized and articulate pressure groups. Also,
journalism is a profession with an ethical tradition. Both the somewhat
eroded and romantic professional traditions of journalism and our
somewhat debilitated concepts of democratic freedom and fair play can
still be used to temper the winds of ``public opinion'' to the shorn
lambs of radical protest and agitation---especially when mass pressure
in the form of protests, strikes, and demonstrations is used to force
the issue.

Yet it must be confessed that these are all frail reeds to lean upon in
a pinch, especially if the pinch is local. To illustrate this last
point, it is sufficient to point to the contrast between the handling of
the 1931 disorders in the Kentucky coal fields by the Kentucky press, as
against the performance of the distant metropolitan journals and press
associations. The local editors editorialized against the ``Red
menace,'' and in their news reporting suppressed and distorted the
unquestionable facts of starvation of strikers, discrimination in the
administration of public and private relief, the capture of the
machinery of justice by the coal corporations and the violence of
middle-class mobs. True, on that occasion the Associated Press also
broke down, because the local A. P. reporter happened to be also one of
the leaders of the middle-class mob which illegally deported one of the
successive delegations of writers and students which entered the strike
area to bring relief to the strikers and to report the facts of the
situation to the country at large. \emph{But} the protests of Dos Passos
and others were effective on that occasion: the offending A. P.
correspondent was dismissed. And shortly afterward the \emph{New York
Times} sent a special correspondent, Mr. Louis Stark, to Harlan County,
where he did an honest and competent reporting job in a series of signed
articles.

A similar situation developed in connection with the Scottsboro case, in
which seven negro boys faced legal lynching in a situation growing out
of race prejudice and conflict fostered by ruling-class economic
interests. The evidence on which the boys were convicted, later shown to
have been largely perjured, was accepted pretty much without question by
almost the entire Southern press. The lynch atmosphere surrounding the
first trial was largely suppressed. The case was consistently ``played
down'' throughout the South. Citizens of New York learned more about the
Scottsboro case through the papers than citizens of Alabama. As a result
of the efforts of the International Labor Defense, a Communist-led
organization, a new trial was ordered by the United States Supreme
Court. The boys were again convicted by a jury obviously swayed by
anti-negro, anti-Jew and anti-radical appeals to prejudice. But the
\emph{New York Times} reporter, Mr. F. W. Daniell, reported the trial
with notable accuracy and fairness, whereas the Southern press for the
most part continued the policy of suppression and distortion, dictated
by the pressures of local and regional ruling-class prejudice and
interests. In this case the factor of professional pride entered also
into the equation. The prosecution made the mistake of treating Mr.
Daniell and other correspondents with scant courtesy. Promptly and
without trepidation, Mr. Daniell, both in his personal conduct and in
his dispatches, made it clear that the Alabama authorities were in no
position to bully and coerce the correspondent of the \emph{New York
Times}.

The press handling of the communist-led Hunger March to Washington in
the fall of 1932 provides another interesting example. In this case the
Hoover Administration broadcast appeals to State and local authorities
to ``stop the Hunger March.'' The evidence is overwhelming that the
press, actuated by the alarm of the administration and of business,
undertook more or less concertedly to play down and ridicule the
demonstration. The dispatches, both while the columns were enroute to
Washington and after their arrival, were so colored and so flagrantly
editorialized as to surprise even experienced radical organizers. The
demonstrators were ``neither hungry nor marching.'' The March was
treated as a Communist publicity stunt and both the leaders and the rank
and file were consistently ridiculed. Radio and news reels joined this
hostile chorus. But in the end, after the Washington police had executed
their melodramatic coup, and the 3,000 marchers were practically
imprisoned on a stretch of windswept highway on the outskirts of the
capital, the unity of the conservative press front began to crack.

There were several factors in this partial failure of the anti-communist
propaganda. In the first place, the Communist organizers of the
Unemployed Councils, hugely handicapped as they were by lack of funds
and by the terrified inertia of the destitute unemployed workers, had by
sheer drive and energy accomplished a notable feat in bringing the three
columns of marchers to a point of convergence on the capital within a
few hours of each other. In the second place the more radical working
class groups in the cities through which they passed had cheered the
marchers, aided them with contributions of food and shelter, and
otherwise counteracted the efforts of the authorities to disintegrate
and abort the enterprise. In the third place, Herbert Benjamin, the
Communist Director of the march, proved himself to be a cool,
resourceful, courageous and humanly appealing leader. He contrasted
favorably with Major (Duck-Legs) Brown, who directed the forces of the
District of Columbia police. The genuine discipline of the marchers
contrasted favorably with the provocative brutality and obvious
unfairness of the police. Protests, sponsored by more or less well-known
liberals, and invoking the rights of free speech, appeal to the
government, etc., were duly printed in the conservative papers. From the
publicity point of view, the most effective effort on the radical side
was the delegation of socially prominent New York women which came to
Washington and protested to Vice President Curtis and various
Congressmen and Senators. Known radicals, however prominent, are
comparatively useless for such purposes; their protests are not ``news''
and the conservative press virtuously plays them down as
``publicity-seekers.''

In the case of the Washington Hunger March the protests of the prominent
liberals and radicals helped, but what helped most was the fact that
Hoover, his official family and the brass hats of the army were
personally unpopular with the Washington correspondents and with the
staff members of the local papers. This unpopularity was a factor in the
forthright protests and the vigorous news writing which accompanied and
followed Hoover's expulsion of the Bonus Army a few months before. The
\emph{Washington News} printed the flagrant facts of police brutality
and provocation and editorially protested. (The \emph{News} is the local
Scripps-Howard paper and the city editor happened to be a liberal, as
well as personally popular with the newspaper fraternity.) At this point
the hitherto almost unanimous hostility of the capitalist press began to
falter. The disparity of forces, as between the microscopic army of
determined, but unarmed and unviolent marchers, and the armed might of
the government police and military made the administration's effort to
convert the demonstration into a Red scare seem a little ridiculous. The
climax came when Benjamin executed his hair-raising ``dress-rehearsal,''
after which he had said: ``Tomorrow we march.'' The next day came the
official order permitting the marchers to enter Washington.

What, by the way, was this performance? In its essence it was
propaganda, or if you like, education, in one of its highest
manifestations: that of strategic, dramatic action. It had its effect,
despite the effort of the conservative press to suppress and distort its
significance and muffle its reverberations.

With respect to this case there are a number of interesting points to be
noted. First, the Washington press, especially the \emph{News}, treated
the marchers more fairly on the whole than the New York papers. In some
instances the latter headlined the dispatches of their correspondents in
such a way as to distort, always in derision of the marchers, the true
bearing of the story.

The apparent reversal of the usual in such situations is simply
explained. In this case the pinch was not so much local as national. The
ruling-class and middle-class interests and prejudices served by the
capitalist press throughout the country were vigorously hostile to the
Communists and especially hostile to that particular demonstration. But
in Washington thousands of people had witnessed the inept and brutal
performance of the police. Although middle-class Washington public
opinion was in general hostile or indifferent to the marchers,
Washington didn't like Hoover, nor did it like the repetition, by a
defeated and discredited administration, of tactics rawer if anything
than those employed against the Bonus Army.

The Washington papers did nothing comparable to the exploit of a
\emph{Daily News} reporter who invented out of whole cloth and published
a speech alleged to have been made by Herbert Benjamin, violently
inciting his followers to a bloodthirsty attack upon Washington.
Theoretically, the \emph{News} couldn't do such a thing because it is a
mass paper sold to ``Sweeney,'' the working man---or at least its
promotion literature so alleges. It was the Struggle of Sweeney that
Benjamin was supporting. Actually, something of the sort was to be
expected. The \emph{News} uses sensational tabloid methods to exploit,
for purely commercial purposes, the economic illiteracy and the economic
and psychological helplessness of its readers. The \emph{News} is a
business property, a commercial, profit-making enterprise, and an
\emph{advertising medium.}

With the foregoing case histories in mind, let us return to our major
categories, advertising, propaganda and education, and examine once more
the liberal views of Professor Lumley and others. The thing to look for
in any system of social communication is the point of control.
Obviously, the key phenomenon is advertising, which is in turn merely an
instrument of competitive business. A commercial publication is an
advertising medium, that is to say, an instrument by which advertisers,
with the complex of interests and prejudices which they represent, shape
and control the economic, social and political patterns of the literate
population: directly through the signed advertisements themselves;
indirectly through the controlled or influenced editorial content of the
publication; indirectly through the controlled or influenced content of
formal education in the schools and colleges.

When a powerful vested interest, such as the electric power industry,
wishes by means of propaganda to shape public opinion favorably to its
interests, it is advertising that enables it readily to employ the
instruments of the daily and periodical press, radio, motion picture,
etc., for this purpose. Advertising is, of course, itself propaganda,
but more important, the granting or withholding of an advertising
contract offers a means of bribing or coercing indirect propaganda in
the editorial columns of the publication. Finally, where such bribery or
coercion is impracticable, as in the case of powerful publications like
the \emph{Times}, the same end is secured by reason of the fact that the
\emph{Times} is an advertising medium. As such it is an instrument of
business, and its editorial policies are conditioned by the pressures of
the dominant economic forces.

Professor Lumley exclaims at the omnipresence of propaganda. Our
civilization, he says is ``spooky'' with the ghosts of propaganda hiding
behind every bush. The professor has had nerves. Propaganda is no more
and no less omnipresent than the vested interests of competing and
conflicting economic and social pressure groups. The balance of power is
held by business, which, through advertising, controls the instruments
of social communication. There is nothing mysterious about it, nothing
moral, nothing ethical and nothing disinterested. How could there be?
Miracles don't happen in the body politic any more than they do in the
physical body of man.

Advertising is propaganda, advertising is education, propaganda is
advertising, education is propaganda, educational institutions use and
are used by advertising and propaganda. Shuffle the terms any way you
like, any one, any two, any three, to paraphrase Professor Lumley. What
emerges is the fact that it is impossible to dissociate the phenomena,
and that all three, each in itself, or in combination are
\emph{instruments of rule}.

Whether the use of these instruments is veiled or overt will doubtless
continue to be a matter of grave ethical concern to liberals like
Professor Lumley. But the majority of the propaganda to which he objects
is overt.

Every journalist knows this. The editors of \emph{The New Yorker} are
journalists, highly competent and sophisticated in that field, and they
take great pleasure in jibing at the bizarre efforts of the ``public
relations'' experts. On occasion they become as disgusted as any man
about town can permit himself to become without risk of rumpling his
hair. The following comment from \emph{Talk of the Town} in its issue of
Feb. 10, 1934, is an example. The note is headed \emph{Many Happy
Returns} and I quote the first and the concluding sentences:

\begin{quote}
The Quadruple-Screw Turbo-Electric Vessel Queen of Bermuda, Capt. H.
Jeffries Davis, was the scene last week of a novel birthday party for
President Roosevelt and the Warm Springs Foundation on behalf of the
Bermuda News Bureau, the Furness Bermuda Line, the Fashion Originators
Guild, and \emph{Island Voyager Magazine}, by special arrangement with
James Montgomery Flagg, Howard Chandler Christy, Carl Mueller, John
LaGatta, McClelland Barclay, forty mannequins, the six most beautiful
girls in America and Lastex. Mrs. James Roosevelt, mother of the
President, received....\\
Her son, Franklin, in whose honor the party was given, was fifty-two
years old; and there were moments... when we wondered whether the
country he has been working so hard to save was worth the effort.
\end{quote}

One is moved to ask Professor Lumley if there is anything insidious or
lacking in frankness about this extraordinary synthesis of personal,
political, philanthropic and commercial propaganda? Let us consider for
a moment, realistically, this question of the veiled or overt use of the
instruments of social communication as a problem in tactics. One admits
that the public which sees the end result only is frequently unaware of
the origins of propaganda. But ordinarily the propagandist himself
proceeds quite overtly in manipulating his instruments.

Advertising is overt enough as to its origin or sources because it is
signed by the advertiser. The interest involved is overt; the advertiser
wants to sell you something for more than it is worth, so that he can
make a profit on the transaction. The method is more or less tricky,
since it usually involves taking advantage of the economic, social and
psychological naivete of the reader. The results accruing to the reader
or to the advertiser are pretty much unpredictable as to either party.

The majority of successful propaganda practice, whether by commercial
``public relations counsellors'' like Edward Bernays and Ivy Lee or by
radical propagandists is overt; the name of the propagandist or the
company or organization he represents is typed or printed at the top of
his release. Sometimes commercial interests use dummy organizations as a
``front.'' For example, the munitions makers are more or less back of
the National Security League, just as the Communists are more or less
back of various peripheral organizations in the field of labor defense,
relief, etc. But to suppose that the hard-boiled publishers and editors
of the commercial press are taken in by these fronts is to be impossibly
na\"ive. Also, in the case of a powerful commercial client, such as, for
example, the Rockefeller interests, Mr. Lee has everything to gain by
having the release come from 26 Broadway. And in the case of the radical
propagandist, nothing makes the city desk so suspicious and sour as
clumsy attempts at indirection. As already pointed out, Benjamin's
``dress-rehearsal'' of the Hunger March into Washington was excellent
propaganda and surely that was overt enough. Admittedly, occasional
veiled publicity coups come off successfully; but the percentage of such
triumphs is relatively negligible and the backlash the next time you try
to make the papers more than wipes out your gains.

The publicity Machiavellis of the National Electric Light Association
were the laughing stock of the public relations profession and the
catastrophe which befell them was cheerfully predicted long before it
happened. They failed precisely because they were not sufficiently
overt. So far as the press was concerned, all they had to do was to walk
in the front door of the business office, sign their advertising
contracts and get pretty nearly everything they wanted. Expense? ``The
public pays the expense,'' to quote Deak Aylesworth's classic line.
Instead of which they employed the most extraordinary collection of
publicity incompetents that has ever been assembled under one tent. They
were equally stupid when it came to professors. All they succeeded in
hiring were cheap academic hacks who in the end did them more harm than
good.

As already pointed out, business can influence or control our schools
and universities when it wants to or feels that it has to. Professor
Lumley's ideal purification of the educational function falls down at
this point and at a number of others, suggested in the following
questions: how does an educator, unless one grants an inconceivably
psychological self-awareness, know whether or not be is ``veiling'' the
origin or sources of his instruction, the interests involved, the
methods involved or the content spread? How can he anticipate the
results accruing to the victims of either education or propaganda?
\enlargethispage{\baselineskip}

Apparently, what chiefly confuses liberals like Professor Lumley is the
residual ideological and institutional d\'ebris of ``democracy.'' The
thing becomes instantly explicit and forthright when rule is exercised
by a dictatorship and competition for rule is eliminated by force. The
liberal illusions of a free press, free radio, free speech,
constitutional rights, objective education, etc., all disappear almost
overnight. This has been happening under our eyes in Russia, Italy and
Germany. Do liberals have to be cracked on the head before they can see
it?

Pinkevitch, in his \emph{Education in Soviet Russia}, classifies
propaganda, and agitation as forms of education operating on somewhat
lower intellectual levels.\footnote{{[}A. P. Pinkevich,
  \emph{\href{http://www.worldcat.org/oclc/1635457}{The New Education in
  the Soviet Republic}} (New York: John Day Co., 1929).{]}} Press, radio, schools, colleges, are all
owned and operated by the state as instruments of \emph{rule} in behalf
of the ruling class, the class of workers and peasants. The purpose for
which these instruments are used is to make Communists, just as they are
used in Italy to make Fascists, and in America to make our curious
menagerie of capitalists, capitalist snuggle-pups, saps, suckers,
morons, snobs, pacifists, militarists, wets, drys, Communists, liberals,
New Dealers, double dealers and Holy Rollers.

In America the industry is hugely ramified but the underlying
motivations, controls and mechanisms are relatively simple, although, of
course, as in any transitional period of social conflict, the balance of
power is constantly shifting. A capitalist democracy is a state of
conflict almost by definition. Rather than to catalogue these conflicts,
expressing themselves in the form of propaganda, it would seem more
profitable to accept our instruments of social communication for what
they are: \emph{instruments of rule}; then to describe how these
instruments are used, in whose behalf and to what end.


% CHAPTER NINE
\chapter[9 \hspace*{1mm} TRUTH IN ADVERTISING]{9 TRUTH IN ADVERTISING}

\newthought{The} conception of ``Truth in Advertising'' is at once \emph{the least
tenable} and the \emph{most necessary} tenet of the ad-man's doctrine.
This contradiction arises from the fact that the advertising business is
essentially an enterprise in the exploitation of belief.

It is untenable because profit-motivated business, in its relations with
the consumer, is necessarily exploitative---not moderately and
reasonably exploitative, but exploitative up to the tolerance limit of
the traffic. This tolerance limit is determined not by ethical
considerations, which are strictly irrelevant, but by the ability of the
buyer to detect and penalize dishonesty and deception. This ability
varies with the individual, but in general reaches its minimum in the
case of the isolated ultimate consumer.

No manufacturer, in buying his raw materials or his mechanical
equipment, trusts the integrity of the seller except in so far as he is
obliged to do so. So far as possible, he protects himself by
specifications, inspections and tests, and by legally enforceable
contracts that penalize double-dealing.

But when the manufacturer or retailer turns to selling his finished
product to the ultimate consumer, the situation is reversed and the
elements are sharply different. In his natural state the ultimate
consumer is ignorant enough in all conscience. But he is not permitted
to remain in his natural state. It would be unprofitable,
unbusinesslike, to leave him in his natural state. Hence business has
developed the apparatus of advertising, which, as the editor of the
leading advertising trade publication has pointed out,\footnote{Roy Dickinson, president of \emph{Printers' Ink}, in ``Advertising
  Careers.''} is scarcely a
thing in itself, but merely a function of business management.

That function is not merely to sell customers, but to manufacture
customers. Veblen, with his customary precision, has indicated both the
object and the technique of this function:

\pagebreak \begin{quote}
The production of customers by sales publicity is evidently the same
thing as the production of systematized illusions organized into
serviceable ``action patterns''---serviceable, that is, for the use of
the seller in whose account and for whose profit the customer is being
produced.\footnote{{[}Thorstein
  Veblen,~\emph{\href{https://archive.org/details/AbsenteeOwnershipAndBusinessEnterprise}{Absentee
  Ownership and Business Enterprise in Recent Times: The Case of
  America}}~(New York: B. W. Huebsch, 1923), 306--7n12.{]}}
\end{quote}

What has honesty or truth to do with this business? A great deal,
because the \emph{idea} of truth is a highly exploitable asset. Always,
the customer must be made to feel that the seller is honest and truthful
and that he needs or wants the product offered for sale. Hence the
advertising business becomes an enterprise in the coincident manufacture
and exploitation of reader-confidence and reader-acceptance. In this
respect the ad-man's technique is not essentially different from that of
any vulgar confidence man whose stock in trade is invariably a plausible
line of chatter about his alleged ``trustfulness'' and ``honesty.'' The
writer has watched these gentry operating all the way from Los Angeles
to Coney Island. Their annual ``take,'' while less than that of their
respectable cousins in the advertising business, is still enormous.
Their techniques and successes, if studied by sociologists, would I am
convinced, yield valuable data regarding the contemporary American
social psychology.

Once, at Signal Hill, near Long Beach, California, the writer permitted
an oil stock salesman to give him transportation from Los Angeles to the
oil well, and to lead him through the successive steps by which the
``sucker'' is noosed, thrown and shorn. The prospects, consisting of
about a hundred more or less recently arrived Middle Western farmers,
their wives and children, seemed na\"ively appreciative of the hot dogs
and coffee, and of the genuinely accomplished sales histrionism which
they were getting free. One saw that they were devout believers in magic
of the cruder sorts, ranging from fundamentalism, through rugged
individualism, and spreading into the more exotic side-shows of Yoga,
the Apostle of Oom, numerology, spiritualism, etc., etc., which at that
time infested Los Angeles and still do. Their faces were weather-worn,
their hands were stubby. They were indeed enormously decent and
hard-working people---with less effective knowledge of their social
environment than any African savage [\emph{sic}].

At the climax of the performance, after an oil-smeared ex-vaudevillian
had rampaged up the aisle proclaiming that ``No. 6 had just come in at
ten thousand gallons,'' a scattering few came forward and signed on the
dotted line. They did so with a kind of hypnotized masochism---I am
convinced that many of them were instinctively aware that they were
being gypped [\emph{sic}].

In lieu of buying any of the promoter's exquisitely engraved optimism, I
took him aside afterward and explained that as an advertising writer,
engaged in advertising a nearby subdivision---a strictly legitimate
enterprise out of which many of the buyers made a good deal of
money---I, too, had a stake in the matter. He was only momentarily
embarrassed. Later, on the basis of our professional kinship, I got to
know him sufficiently so that, warmed by a little liquor, he became
approximately confidential.

``Brother,'' I remember his saying (He always insisted on calling me
``brother''), ``the technique of this racket is simple. Always tell the
truth. Tell a lot of the truth. Tell a lot more of the truth than
anybody expects you to tell. Never tell the whole truth.''

My colleague omitted one important element from his formula, the element
of emotional conviction, which I had seen him manipulate with
devastating effectiveness. It is observable that the most charlatans,
like the best advertising men, are always more than half sincere and
honest according to their lights. Sincerity is indeed a great virtue in
an ad-man, and if one has it not, one must at least feign it. In this
connection I recall the experience of a friend who took leave of the
advertising business after some years of competent and highly paid
employment in that business. Her employer, while acknowledging her
competence, had this to say on the occasion of her resignation:

``Miss ---------, you are an able person and a good worker. In my
judgment you have only one fault. You are not loyal to the things you
don't believe in.''

At first glance this statement would seem to plunge us into the deep
water of metaphysics. But the exegesis is simple. The possession of a
personal code of ethics is a handicap in the practice of
advertising-as-usual, the business being above all else impersonal, and
in fact so far as possible de-humanized. One must be loyal to the
process, which is a necessary part of the total economic process of
competitive acquisition. The god of advertising is a jealous god and
tolerates no competing loyalties, no human compunctions, no private
impurities of will and judgment.

The yoke of this jealous god chafes. How could it be otherwise, unless
one were to suppose that advertising men are a selected class of knaves
and rascals? They are, of course, nothing of the sort. They are average
middle-class Americans, a bit more honest, I suspect, than the average
banker or lawyer. In their personal lives they are likely to be kindly,
truthful, just and generous. They would doubtless like to be equally
truthful and just in the conduct of their business. But this, in the
nature of the case, is impossible. The alternatives are either a
cynical, realistic acceptance, or heroic gestures of rationalization.
Hence the tremendous pother that advertising men make about ``truth in
advertising''; or at least, that is half of the explanation. The other
half lies in the business-like necessity of keeping advertising in good
repute; of nursing the health of that estimable goose,
reader-confidence. Are they sincere, these advertising men who conduct
this ``truth-in-advertising'' propaganda which is echoed and re-echoed
by editors, publicists, economists, sociologists, preachers,
politicians? How can one tell, and does it really matter?

Quite obviously, advertising is an enterprise in special pleading
conducted outside the courts of law, with no effective rules of
evidence, no expert representation for the consumer, no judge and no
jury. To continue the analogy: in a court of law the accused swears to
tell ``the truth, the whole truth, and nothing but the truth,'' but if
he is guilty nobody expects him to do so. The attorney for the defense
is theoretically bound by his code of legal ethics, by penalties for
contempt of court, suborning of witnesses, etc. In practice he usually
makes out the best possible case for his client, using truth, half-truth
and untruth with pragmatic impartiality. Moreover, the judge and the
jury expect him to do precisely that, just as they expect the State's
attorney to use all possible means to secure a conviction. Judge and
jury are in theory, and ordinarily in practice, disinterested. They
balance one barrage of special pleading against the other, and so arrive
at a verdict based on the evidence.

It is generally recognized that a defense attorney does not tell the
truth, or permit the truth to be told, if he thinks this truth would
prejudice the case of his client. Why should it be supposed that an
advertising writer, employed to sell goods for a manufacturer or
retailer, can afford to tell the truth, the whole truth, and nothing but
the truth, and refrain from befuddling the judgment of his prospect? In
practice he tells precisely as much of the truth as serves the interest
of the advertiser, and precisely as much expedient half-truth and
untruth as he believes he can get away with, without impairing
``reader-confidence.'' If it seems profitable to scare, shame and
flatter his victim he does so unhesitatingly. If bought and-paid-for
testimonials will do the trick many agencies buy them. If the
fastidiousness or timidity of the publisher, the barking of the Federal
Trade Commission and the Food and Drug Administration, or the protests
of the reforming wing of the profession make it seem desirable to
conceal the fact that these testimonials were bought and paid for, such
a concealment is effected.

Privately, the cynics of the profession will tell you that this is the
prevailing practice, including their own practice. Having learned to
digest their ethical sins, they have no need of rationalizing them.
These cynics leave the ``reform of advertising'' to their more
illusioned colleagues of whom they tend to be coarsely contemptuous. The
plaint of the reformers---vulgarly referred to by the cynics as the
``Goose Girls''---runs somewhat as follows:

``The exaggerations, the sophistries, the purchased testimonials, the
vulgarities, the outright falsifications of current advertising are
quite intolerable. Such practices are destroying the faith, the
illusions, the very will-to-live of `reader confidence.' They constitute
unfair competition. The irresponsible agencies and advertisers who are
guilty of such practices are endangering the stability, the good repute,
and the profits of the advertising profession as a whole.''

To this plaint the cynics retort somewhat after this fashion:

``You fellows prate a great deal about `truth in advertising.' What do
you mean, truth, and what has the truth got to do with this racket? You
say we are killing that estimable goose, reader-confidence, the goose
that lays the golden eggs of advertising profits. Nonsense. It wasn't
the goose that squawked. It was you. And the reason you squawked was not
because you really give a whoop about `truth,' but because we, with our
more sophisticated, more scientific practice, have been chiselling into
your business. We can prove and have proved that bought-and-paid-for
testimonials sell two to one as compared to your inept cozenage, your
primitive appeals to fear, greed and emulation. Furthermore, the ethics
of advertising communications is relative and must be flexible. You have
to take into account both the audience to which such communications are
addressed and the object which these communications are intended to
achieve, and demonstrably do achieve.

``The audience, by and large, is composed of 14-year-old intelligences
that have no capacity for weighing evidence, no experience in doing so,
and no desire to do so. That goes equally for the readers of
\emph{Vogue} and the readers of \emph{True Romances}. They are
effectively gulled by bought-and-paid-for testimonials and even appear
to take some pleasure in being gulled. They buy on the basis of such
corrupt, false and misleading evidence, and this way of selling them
costs less than any other way we have discovered. It is, you will grant,
our duty as advertisers and as advertising agencies acting in behalf of
our clients, to advertise as efficiently as possible, thereby reducing
the sales overhead which must ultimately be charged to the consumer:
thereby, incidentally, safeguarding and increasing the profits of the
companies in which hundreds of thousands of widows and orphans, directly
or indirectly, have invested their all. It is our duty to use every
means we can devise, truthful or untruthful, ethical or unethical, to
persuade consumers to buy, since only by increased buying can the
country be pulled out of this depression. Ours is the higher morality.
The burden of restoring prosperity is on our shoulders. We have seen our
duty and we are doing it.''

Thus the cynics, in private. I must confess that I have derived far
greater intellectual pleasure from the utterances of such hard-boiled
devil's disciples than from the plaintive reproaches and lamentations of
the Goose Girls. One could wish, indeed, that the cynics were more
outspoken. Unfortunately, rationalization is the order of the day, in
business as in politics. Every week sees another proclamation of the new
order of probity upon which business is entering under the New Deal.
Even Kenneth Collins.... One is disappointed to see so able and
interesting an advertising man pledge himself to the Goose Girl
Sorority. But consider the recent advertising of Gimbels department
store in New York. Mr. Collins is Gimbels' advertising manager, having
recently transferred his talents from Macy's across the street, where he
had achieved a notable success by exploiting the slogan ``It's smart to
be thrifty.''

Mr. Collins, judged by his writings in the trade press, is something of
a realist. One can only conclude therefore, that when he assumed his new
duties, his survey of the situation convinced him that radical measures
were needed for the effective exploitation of belief. Here is the
advertisement in which the new ``slant'' was announced:

\begin{center}
GIMBELS\\
TELLS THE\\
WHOLE\\
TRUTH
\end{center}

Every intelligent person will join us in a great new campaign for truth
in advertising. And by truth we mean the whole truth, and nothing but
the truth---exactly what you demand of a witness before a Senate
Committee, or of your own children at home.

Let us tell you a straight story.

For years on end, we at Gimbels have been thinking that we were telling
the truth. We have been supported in our belief by ``the custom of the
business,'' by ``trade privilege,'' by reports from the Better Business
Bureau of New York and by the comments of our customers.

But what we have been telling was, so to speak, ``commercial truth.'' We
would tell you, quite honestly, that a certain pair of curtains had been
copied, in design, from a famous model, that the colors were pleasing,
that the price was very low. \emph{Every word of this was scrupulously
true}. But we may have failed to say that the curtains would probably
fade after one or two seasons of wear.

In the same way, we would tell you that certain dresses had materials of
good quality, that the styles were fresh, and the price very reasonable.
\emph{Every word of this was scrupulously true}. But we may have failed
to add that the workmanship was by machine rather than by hand, and
therefore the price was low.

We believe it is time to take a revolutionary step, in line with the
beliefs of the Administration, and of the opinions of intelligent people
everywhere. We believe that old-fashioned ``commercial truth'' has no
place in the New Deal. From now on, all Gimbels advertising (and every
word told you by a Gimbel salesman or saleswoman) will be---

The truth, the whole truth, and nothing but the truth.

How are we going to assure this? It is human to make mistakes, and we
may make them. If so, we want them called to our attention. We will
gladly and willingly print corrections. But we believe we will make few
errors, for this reason: as a check on our store buyers, and our
advertising writers, we have employed the services of a famous outside
research laboratory---

\begin{center}
THE INDUSTRIAL BY-PRODUCTS AND RESEARCH CORPORATION OF PHILADELPHIA
\end{center}

\noindent to make frequent scientific tests of the materials, workmanship and
value of the goods we offer for sale. This is one of the best equipped
laboratories of its kind in the United States, with a reputation of many
years of service to many of the largest industrial corporations in this
country. They are experts in textiles, and chosen for this reason,
because 80\% of our merchandise is either textile or dependent on
textile for wear.

They have no human or partisan reason to give us the benefit of any
doubt. They will give us impartial tests and reports.

Please read our advertising in today's \emph{News}, \emph{Journal},
\emph{Sun} and \emph{World}-\emph{Telegram}. Bear in mind when you read
the advertising of this firm that

\begin{center}
GIMBELS TELLS THE TRUTH
\end{center}

Note the astute dedication to intelligence, morality and unity of
interest which is implicit in the first paragraph. Just what is the
nature of this ``revolutionary step, in line with the beliefs of the
Administration, and of the opinions of intelligent people everywhere,''
which Gimbels, under the leadership of Mr. Collins, has taken?

Instead of contenting itself, as in the past, with telling that part of
the truth which might be expected to promote sales, and suppressing the
part which would tend to discourage or prevent sales, the store pledges
itself to ``tell the whole truth.'' For example, whereas it had
previously described a piece of cloth truthfully as being good value, it
would add in the future, the further truth that it would quickly fade;
it would say that a raincoat, while worth the price asked, would last
only a year.

One readily admits that this does represent a certain gain for the
consumer---a gain brought about either by the evangelical enthusiasm of
Gimbels and Mr. Collins for the New Deal, or, possibly, by the
coincident collapse of the consumer's confidence and the consumer's
pocketbook, and the consequent stiffening of his sales resistance. Mr.
Collins is to be credited for his astuteness in recognizing and dealing
with this condition. In fairness one should also credit him with a
personal, though far from unique preference for fair dealing, as against
the customary chicaneries of salesmanship and advertising.

But---and this but is important---Gimbels is a profit motivated
corporation, engaged, like any other business, in buying as cheaply as
possible and selling as dearly as possible. The Industrial By-Products
and Research Corporation of Philadelphia will undoubtedly tell the whole
truth and nothing but the truth to Gimbels, because it is employed by
Gimbels, and, in respect to this specific service at least, is
responsible to Gimbels and to Gimbels alone. But will Gimbels pass on
this whole truth and nothing but the truth to its customers and to the
readers of its advertising? The whole truth, including the truth about
Gimbels' profit-margin---\emph{all} the data which the customer would
require in order to measure value? No such proposal is made. At this
point the customer is protected by the competition of other merchants
and by that alone.

No, what we have here is a lot of the truth, more of the truth than
anybody expected Gimbels to tell, but not the whole truth. It is not in
the nature of profit-motivated business to tell the whole truth. Gimbels
is paid by its customers, but is responsible ultimately, not to its
customers, but to its stockholders. Hence the pressure of the economic
determinants is here as always and everywhere toward the exploitation of
the customer up to the tolerance limit of the traffic. Possibly this
tolerance limit is narrowing. I am not sure.

Mr. Collins' demarche is designed to produce customers by manufacturing
a ``systematized illusion'' to the effect that business is not business,
and that the customer, on entering Gimbels, can safely put aside and
forget the maxim, \emph{caveat emptor}, which is the only ultimate
protection of the buyer in a profit-motivated economy.

Suppose that Mr. Collins' readers are convinced; that they do stop
worrying about whether they are being cheated or not. They would like to
do this because it would certainly mean a great saving of time, money
and energy. But what happens if they do? They find that Gimbels' stock
in trade consists not merely of goods but of ``systematized illusions''
built up by decades of advertising and capitalized in trademarks which
add a considerable percentage to the cost of the product and a still
higher percentage to the price of the product. In the drug and cosmetics
department they would find that the price of the products offered for
sale frequently represents about 90\% of advertising bunk and 10\% of
merchandize. Will Gimbels, which is pledged to tell the truth, the whole
truth, and nothing but the truth, tell them that? No. Does the
Industrial By-Products and Research Corporation know this part of the
truth? It either knows it or could easily learn it. Is this truth of
interest and value to the customer? It certainly is. Then why doesn't
Gimbels buy this truth from its research company and pass it on to its
customers? Because Gimbels is a profit-motivated corporation responsible
not to its customers but to its stockholders. Because the manufacturers
of these absurdly priced and inadequately described products have by
advertising them, built up systematized illusions to the effect that
they are worth the price asked for them. Because Gimbels, which is not
in business for its health or for the health of its customers, is
obliged both to carry these products, and \emph{not} tell the truth
about them, or lose an opportunity for making a profit---usually a high
profit---on their sale. What would happen if Gimbels started telling the
truth about these products? The manufacturers would probably bring legal
or economic pressure to bear, sufficient to cause Gimbels to cease and
desist. Where can you learn the truth about such products? From
Consumers' Research, or for that matter, from almost any honest testing
laboratory you chose to employ. Why does Consumers' Research really tell
the truth, the whole truth, and nothing but the truth, to the best of
its ability? Because it is employed by and responsible to its
subscribing members its customers. Why doesn't Gimbels tell that kind of
truth to its customers? Because it is not responsible to its customers.
It is responsible to its stockholders.

It will perhaps be argued that the drug and cosmetic department is
exceptional. It is somewhat exceptional, but by no means unique. The
breakfast cereal business is also primarily an advertising business, and
many of the packaged ``values'' offered by Gimbels grocery department
are chiefly air, paper, cellophane and advertising.

It will be further argued that these areas of exploitation, entrenched
in the systematized illusions built up in the public mind by
advertising, are outside Gimbels' control. But is Gimbels completely
frank about its own ``house products''? If so, Mr. Collins can claim a
real revolution. The ordinary practice of the retailer in substituting a
house product for an advertised product is to take advantage of the
inflated illusion of value built up by advertising. The house product
may be, and frequently is, as good or better than the advertised
product. The price asked for the house product is ordinarily just enough
less than the price of the advertised product to make the substitution
acceptable to the customer. By crawling under the tent of inflated
``values,'' erected by advertisers, retailers are able to make excellent
profit margins through such substitutions---in the case of such a
product as face cream, margins running up to three hundred and four
hundred per cent. Wouldn't it be wonderful, Mr. Collins, if Gimbels made
up a list of such products and undertook to sell them for approximately
reasonable prices? Would this be in line with the beliefs of the
Administration, or would it come under the head of ``destructive price
cutting''? In any case, wouldn't it be nice for the consumer and---just
possibly---good business for you?

Sadly, one begins to suspect that the able, intelligent, hard-boiled Mr.
Collins has become just another Goose Girl. The morale of the geese is
terrible these days. Mr. Collins is responsible for a large flock, and
as a practical advertising man he realises that he must do something
about it. Hence, with his left hand, he launches ``a great new campaign
for truth in advertising.'' But his right hand is also busy. Alongside
this pronouncement ``Gimbels Tells the Whole Truth,'' we find another
Gimbel advertisement headed ``Sky's the Limit!'' In this advertisement
the reflexes of the reader are shrewdly conditioned to the need of and
purchase of a collection of beach chairs, outdoor tables, etc., for use
on the roofs of city apartments---a new market. This would seem to be
very competent advertising-as-usual in the modern chatty manner,
designed to compete with the interest of the adjoining news columns. It
is currently argued in the trade that this is good ``educational''
advertising because it manufactures customers. From the consumer's point
of view it would be possible to contend that what the consumer is
interested in is a concise description of the product and why it is
worth its price; that the chatter, being neither news nor information,
is a tiresome impertinence, intolerable in a civilized community. But
then, if the consumers had that much sense, they would no longer be
geese. So that Mr. Collins' big-hearted services as Goose Girl and
customer producer would no longer be required.

This example and that of the Gillette Safety Razor Company which is
examined in the following chapter, have been selected because in both
cases the claim of truth-telling is explicitly made. But for the fact
that the American pseudoculture is based on a structure of make-believe,
which, in turn rests on layer after layer of the accumulated
make-believe of past decades and past centuries, it would not even be
necessary to explode such claims for it would not pay to make them.
Sufficient to say that when an advertiser takes the \emph{name} of
Truth, it is in the nature of the case that he should do so in vain, and
with either conscious or unconscious hypocrisy; that the coincident
appeal to, and exploitation of, reader-confidence is merely one of the
necessary techniques of advertising mendacity-as-usual. The
documentation of this mendacity has been sufficiently attended to by
Messrs. Chase and Schlink in \emph{Your Money's Worth}, by Messrs.\footnote{{[}Stuart Chase and F. J. Schlink,
  \emph{\href{https://www.worldcat.org/title/your-moneys-worth-a-study-in-the-waste-of-the-consumers-dollar/oclc/229662}{Your
  Money's Worth: A Study in the Waste of the Consumer's Dollar}} (New
  York: Macmillan, 1927).{]}}
Schlink and Kallet in \emph{100,000,000 Guinea Pigs} and by the run of
the mill prosecutions by the Federal Trade Commission, by the seizures
of the Food and Drug Administration and by the exposures of quack
proprietaries by the American Medical Association.\footnote{{[}Arthur Kallet and F. J. Schlink,
  \emph{\href{https://www.worldcat.org/title/100000000-guinea-pigs-dangers-in-everyday-foods-drugs-and-cosmetics/oclc/546417}{100,000,000
  Guinea Pigs: Dangers in Everyday Foods, Drugs, and Cosmetics}} (New
  York: Vanguard Press, 1933).{]}}

The conclusion which these massive accumulations of data add up to in
the minds of good citizens in and out of the advertising business is
that the abuses of advertising should be corrected; that Congress should
pass another law; indeed, as I write this, Congress seems likely to pass
another law, which will be discussed in the concluding chapters of this
book. As a former advertising man, made familiar by years of practice
with the various techniques of the profession, the na\"ivet\'e of this
conclusion leaves me groaning with despondency. Congress can and
probably will legislate itself blue in the face, without changing an
iota of the basic economic and cultural determinants, and so long as
these determinants continue to operate the exploitation of the consumer
will simply, in response to criticism, spin the kaleidoscope of
technical adaptations. To put it more brutally, advertising will merely
find new ways of manufacturing suckers and trimming them. Mendacity is a
function of trade and observes no ethical limits just as military
warfare observes no ethical limits. Advertising is an exploitation of
belief. The raw material of this traffic is not merely products and
services but human weakness, fear and credulity. In the end, as Veblen
pointed out in the penetrating footnote already quoted, it becomes a
``trading on that range of human infirmities which flowers in devout
observances and fruits in the psychopathic wards.''

To do them justice, the Goose Girls---the reformers have come to
constitute almost a sub-profession of the profession itself---are in
many cases entirely sincere, since the ideas of a unified, functional
society is something undreamed of in their philosophies, or in the
textbooks of orthodox \emph{laissez faire} economists for that matter.
Few of them are as logical or as frank as the banker, Paul M. Mazur, who
in his book, \emph{American Prosperity, Its Causes and Consequences},
has this to say about the ``Truth in Advertising'' ballyhoo:

\begin{quote}
But should advertising ever limit itself under judicial oath to tell the
whole truth, unvarnished and unadorned, woe betide confidence in
America's products and industry.... If the whole truth were really told,
the career of advertising would degenerate from the impact of a powerful
hydraulic hammer to a mildly reproving weak slap on the wrist.\footnote{{[}Paul M. Mazur,
  \emph{\href{https://www.worldcat.org/title/american-prosperity-its-causes-and-consequences/oclc/289643}{American
  Prosperity, Its Causes and Consequences}} (New York: Viking Press,
  1928).{]}}
\end{quote}

So far as the writer is aware, the Better Business Bureau has never
denounced Mr. Mazur for this heresy---has never even given him a
``mildly reproving weak slap on the wrist.''


% CHAPTER TEN
\chapter[10 \hspace*{1mm} CHAIN MUSIC: The Truth About the Shavers]{10 CHAIN MUSIC: The Truth About the Shavers}
\chaptermark{10 CHAIN MUSIC}

\newthought{Some} time during the decade following the Civil War, and for reasons
unknown, whiskers began to go out in America. But this fashion mutation
ran counter to the conservatism of nature, according to which whiskers
continued to come in. Thus, by the mysterious power of fashion, a great
new industry was created, giving employment to millions of people, and
carrying the banner of progress to the most remote corners of the
inhabited globe.

It was the period during which the major vested interests of the
American capitalist economy were being parceled out and consolidated.
Railroads, coal, oil. And now, chins. Nude chins, or rather, the
dynamic, progress-generating conflict between biology and creative myth,
expressed in the man-made taboo on whiskers.

The ground-plan of this industry, as laid down by the founding fathers,
bears the unmistakable mark of genius, combining as it does subtlety and
a certain chaste and beautiful simplicity. The annual wheat harvest is
worth so much, in plus or minus figures---mostly minus in recent years.
The daily whisker harvest is worth so much---always plus, the market
being certain and the crop utterly reliable and independent of the acts
of God. Moreover, by an application on a grandiose scale of the Tom
Sawyer theory of business enterprise, the harvest hands actually pay for
bringing in a crop which in itself is worth nothing.

Nobody knows who started the taboo on whiskers. Not even a wooden cross
marks the unknown grave of this unknown soldier. But greatness was
indisputably his. He changed the face of the human race. He kept Satan
at bay by furnishing work for idle hands to do---all male hands, every
morning, three hundred and sixty-five days of the year. He mocked at
natural law. He refashioned the civilized ideal of masculine beauty. He
added uncounted millions to the wealth of this and other countries,
expressed in stock and bond ``securities,'' and in deeds and titles to
physical properties. The religion which he founded spread quickly into
all lands; it brought light and leading to the wandering tribes of
darkest Africa; the Igorrotes came down out of their trees and rejoiced
in the new gospel; even the Eskimos within the Arctic Circle ate less
blubber and turned to higher things.

No other religion can claim an equal number of adherents. Christianity,
Islam, Buddhism, Hinduism, Atheism have all slain their millions. But
the Shavers are as the sands of the sea, and death would be too good for
them. It would not be good business.

Moreover, as contrasted with the faltering faith of these other decadent
sects, the Shavers prove their loyalty by the punctilious observance of
a daily ritual and by regular tithes contributed to the coffers of the
True Church.

As already noted, the founder of this church is unknown. Quite possibly,
he died in poverty and obscurity. But the Great Apostle of the Shavers
was King C. Gillette. He became famous and rich. Quite probably his
portrait has been more widely disseminated than that of any other
religious leader in the history of the world. When he died he left a
large fortune, made out of nothing; made out of ``such stuff as dreams
are made on.''

The writer is a Shaver and will probably die a Shaver. Why, he does not
know. His father was a Shaver. The only whiskers in his immediate family
environment adorned the chin of his maternal grandfather, who was some
special kind of Shouting Methodist, I believe. At the age of ninety, he
was still stroking those whiskers and singing lustily ``There is a Green
Hill Far Away.'' There was also old Maginnis, the Celery King, but he
was a very dirty and eccentric old man, whom the Shavers used as a
Horrible Example.

The myth had been invented some years before I was born, and during my
childhood the taboo on whiskers became increasingly strict. The faces of
the young men especially were vigilantly watched for signs of heresy.
Whiskers were derided as a mark of effeminacy. Even mustaches were
considered a dangerous deviation from the Pure Faith.

On my sixteenth birthday, my father presented me with a Gillette Safety
Razor, and from that day on I observed the ritual punctiliously,
although during the early months the harvest was meagre. The blades, I
noted, were marked, ``Not To Be Re-Sharpened.'' This I took to be an
Article of the Faith, which I scrupulously obeyed, although it meant
that, since money was scarce, a package of blades had to last at least a
year. The first thirty days were the hardest. After that the frictional
heat generated by repeated scraping was sufficient to cauterize my
wounds.

I remember that my grandfather, observing the lamentable condition of my
chin, once urged that I see if his knife hone wouldn't help those blades
a bit. I repelled the suggestion with scorn. Grandfather and I were in
opposite camps. He was a Shouting Methodist and a bearded ancient. I was
an atheist and a Young Shaver.

The effects of this early religious training still linger. At various
times I have wondered what I would look like in my natural whiskered
state. But what would people say? And what would happen to my job? And
how would my best girl feel about it? So the next morning I would turn a
deaf ear to those perverse curiosities, and perform again the ritual of
the Gospel According to King C. Gillette.

I am reconciled now. Never while I live shall I look in the mirror and
see the image of myself as nature intended me to be. I am not myself. I
am not my own master. I am, like millions of my fellow men, a Shaver.

I remember that shortly before the war, there was a minor outcropping of
heresy. The Spirit of Doubt was abroad in the land, and the morals of
the young were being sapped by the insidious infection of a materialist
culture. Once I recall seeing a young man under thirty, doubtless in a
spirit of bravado, enter a public restaurant looking like the portraits
of Alexander Dowie. Quietly but firmly the waiters, with stern, set,
smooth-shaven faces, put him out into the night. Such devil's disciples
were rare, but unquestionably the minds of the people were troubled. One
shrinks from imagining what might have happened but that, just at the
crucial moment, the President declared war. Force without stint. The
Huns were at the gate. The whiskered Bolsheviks of Russia were attacking
the very foundations of civilization.

In the tremendous outpouring of religious faith and devotion that
followed, all doubts were swept aside. And the True Church did not fail
to do its bit. Sitting in solemn conclave in Boston the synod of the
Church decreed that not one American doughboy should lose his immortal
soul for lack of proper equipment to perform the ritual.

I have reason to know that the Church made good on this patriotic
commitment. Along with two million other Shavers I went, as a private
soldier, to France. I knew what I was fighting for. Those whiskered
Bolsheviks, and those bearded German professors who had signed the
manifesto pledging science and scholarship to the aid of the Huns!

Before I sailed I was presented at various times with eight separate and
complete shaving kits. Three of them were the official equipment of the
True Church; the others were put out by various dissenting sects which,
however, had made common cause with King C. Gillette in the Great
Crusade. Since I regarded these gifts as church property I preserved
them carefully, although the transportation problem was difficult for a
private soldier. My pack had only limited capacity, and in the aggregate
this plethora of equipment added quite a bit to the load I staggered
under on long hikes. With the best will in the world, I found that they
tended to drop out of the pack, to fall into mess kettles, and otherwise
disappear. But I still had six when I sailed.

Before I left the boat I was presented with three more. At the base
hospital the Y.M.C.A. secretary insisted on giving me another pair. I
attempted to protest, but his face froze, and I took them. This was
getting a bit thick I felt. My face was o.k. I shaved every morning, in
cold water at that. What was I expected to do with those eleven kits?
Then a great idea occurred to me. I would give them away. Besides doing
my bit at the front, I would enlist my services in the Propaganda of the
Faith, using the materials with which the Church had provided me.

I gave one to a bearded priest who was serving as brancardier with a
French ambulance section to which my unit was attached. He would only
take one, and I was a little saddened later when I found him, still
jolly and hirsute, using the blade as a nail cutter. Except for one
bearded old peasant woman who chased me out of her bistro, I had better
luck, curiously, with the women. I gave six separate shaving kits to six
different marraines, chiefly laundresses and barmaids in French villages
behind the lines. The women shaved too, it seemed. Obviously, the war
was going better than I had thought it was.

However, I made no real headway, because more shaving kits kept coming
in, from the Y.M.C.A. and through the mails from solicitous maiden aunts
at home. I broke down gradually and took to leaving them in the pig pens
where we occasionally lodged, and where, in the nature of things, they
would be of no service to the Cause. Even so, I reflected, I was better
off than the mules. The quartermaster's department, I was informed, had
been supplied with 300,000 branding irons for the mules. I wondered what
the mules would do with them. Provided there were any mules. In six
months at the front I never saw one; nothing but a herd of Algerian
donkeys, once, which rapidly disappeared into the French soupe. But if
there had been mules doubtless they would have branded themselves
thoroughly. The Church, I reflected, was not alone in its outpouring of
patriotic service. With all this I can testify that the morale of the
American troops was high. We shaved. We shaved almost every day. We
shaved with ditch water. We shaved with luke warm coffee. After
excusable omissions of the ritual, caused by duty at the front, we
shaved twice. For God. For Country. For King C. Gillette.

What happened after the Armistice was a different matter. As I look
back, it would seem that the whole magnificent structure of American
idealism crumbled almost overnight.

It was a fact, a regrettable fact, but a fact, that the chins of the
American doughboys were pretty sore. They started wagging. Some of the
things they said I hesitate even now to repeat.

They said they had too damned many shaving kits. They regretted that the
envelopes protecting the blades were not larger so that the paper might
be used for purposes for which the quartermaster's department provided
no regular supplies. They pointed out that whereas every soldier was
equipped with a dozen or so of shaving kits of assorted brands, none of
these kits was equipped with more than one blade. The Y.M.C.A. gave you
razors but no blades. You had to buy the blades. And the blades were
extraordinarily dull. I remember that one godless doughboy asserted in
plain words that they were made dull on purpose. Nothing happened to
him. In due time he was honorably discharged from the service and I met
him later in civil life. The doughboys talked a good deal about those
blades. Sometimes, in the evenings, there was enough chin music of this
sort to drown out the regimental band. Always, in such sessions, the
name of King C. Gillette would be intimately and often obscenely coupled
with the Y.M.C.A.

It was probably just shell shock---the reaction from the hardships and
dangers of the front. For myself, although a little disheartened, I
could excuse the talk. It was the things they did. They took to shying
shaving kits at truant pigs. The main street of a French village where
we were quartered became littered with them and the Mayor protested. The
lieutenant ordered out a detail, and a dozen men faced court-martial
rather than move a step. Nothing happened. The lieutenant, it soon
appeared, was growing a beard.

I am a good Shaver still, but naturally I did not go through this
experience unscathed. And in the years that followed the Armistice I
could not help observing that the Church seemed to be slipping. The
phrase ``not to be re-sharpened'' was no longer engraved on the
blades---a doctrinal concession to modernism for which the official
church was to pay heavily, for innumerable re-sharpening contraptions
were soon on the market and some of them were more or less effective.
Meanwhile, the chin music increased in volume and shrillness until at
last the Church was obliged openly to take the field against the growing
heresy.

In 1926 the Gillette Safety Razor Company spent nearly a million dollars
in newspaper and magazine advertising. The copy was moderate in tone,
attempting to reason the children back into the fold. The blades had
been improved. They were continually being improved. The mass production
process by which they were produced was incredibly accurate and was
checked by innumerable inspections of the finished product. The steel
used was the best and most expensive tool steel obtainable. True
believers should understand, when they experienced pain and consequent
doubt in connection with performing the daily ritual of the Faith, that
it wasn't the blade's fault. It might be the weather. Or the stiffness
of the communicant's bristles. Or the hardness of the water. Or the
temporary and excusable hardness of the communicant's heart, induced by
a late party the night before.

Reading this campaign I knew in my heart that it marked the beginning of
the end. Not so would old King C. Gillette have spoken in the great days
before that erratic genius sold out his interests to the bankers, and
went gaga as amateur economist and world -saver. The Church had become
rich and soft. Where the Great Apostle had once peddled his invention at
ten dollars a kit from door to door, the degenerate princes of the
Church now gave the razors away with a tube of shaving cream. True, the
empire was now huge, and rich tribute in the form of profits on blades
flowed in from every quarter of the globe. But godless men, actuated by
motives of material gain and without license of the True Church, had
actually ventured to manufacture blades suitable for the official razor
and offer them for sale in the marts of trade. And to such a low ebb had
the morale of the faithful sunk that more and more these blades were
purchased and used. So that the prestige of the True Church was shaken
and its tithes reduced.

Again the following year the Church struck out with a huge advertising
campaign. But again the note of authority was missing from its
pronouncements. The blades too, lacked edge, or at least the chins of
the faithful continued wagging to that effect. This heresy was
encouraged by a subversive organization known as Consumer's Research,
which informed its subscribers that some of the competing blades were
perhaps a little better than the official equipment---not much, but a
little. Other insidious rumors went forth; one to the effect that the
Church had even gone so far as to manufacture and sell, under a shameful
disguise from which the face of the Great Apostle had been removed,
cheap and inferior blades designed to compete both with the mavericks
and with the official product.

Day after day this subversive chin music gained in volume and in
ominousness. Meanwhile a major crisis approached in the internal economy
of the church. By virtue of the original patents issued by the State,
the gospel according to Gillette had become an Established Church and
the Gillette Company enjoyed a monopoly in the sale of the patented
razor. This greatly helped in keeping the ritual pure, as also in the
collection of tithes. But within a year these patents would expire.
Chaos, certainly would ensue unless somehow, somewhere, the officialdom
of the Church could muster a little statesmanship.

Long conferences were held, and at last a decision was reached. The
Church would apply for patents on an improved razor and an improved
blade, which latter would fit the old razor also. But since it would be
patented, the conscienceless mercenaries who already infested the market
would be stopped from imitating it. Meanwhile, the Church would put
forth huge quantities of the new razor, offered to the faithful free
with a tube of shaving cream. In a short space of time the new razors
would displace the old and since they required the new blades which only
the official Church would be entitled to make and sell, the elders of
the Church would once more sit at ease in Zion and further diversion of
the tithes would be prevented.

Everything went through as scheduled except those essential iron clad
patents. By some fluke or treachery, just before the Church's New Deal
for the Shavers was announced the market was flooded with blades which
fitted the new razor perfectly, as well as the old razor. And the State
remained neutral. And the Elders rent their garments. And the Shavers?
It is appalling to realize how little the Shavers cared about the whole
matter except that, finding the heretical blades to be of reasonably
good quality, they bought them in great quantities. So that the Elders
were obliged to seek out the heretic, and purchase his business for a
large, a very large sum of money. And a little later, after the stock
market crash, the stockholders of the Church questioned the
statesmanship of the elders---in fact raised hell. So did a hundred
circumcised and uncircumcised owners of production machinery that could
turn out blades, for countless new brands appeared on the market.

The later history of the Church is almost too melancholy to record.
Remembering the genius of the Great Apostle, the Elders sought out one
of the most famous Doctors of Advertising Homiletics in America and told
him to launch a new advertising campaign. He did so. He gave the
Faithful the old time religion plus a dash of Listerined Freud. ``Am I
losing my husband's love?'' (Picture of weeping wife; copy plucks at the
conscience of the husband who is forgetful of the morning ritual; the
cheek you love to touch.)

Too late. It didn't work. So then what did those dumb elders do? The
Truth! The Truth, no less, with the elders themselves beating their
breasts and crying ``Mea Culpa.'' The truth being a confession that for
a while the official blades were not so good, but now they're much
better, please, and we're honest men and need the money.

The truth, forsooth! Since when has a self-respecting church felt called
upon to defend its divinely inspired truth against the hecklers of the
market place?

The official blades \emph{are} better now, they say. And they cost just
about half what they formerly cost. I don't care. I am a Shaver, a
devout Shaver, if you like, but after all that has happened, I can no
longer be a faithful churchman. I buy any old blades. A while back I
bought a re-sharpening contraption and it worked more or less. And just
the other day I got out grandfather's hone which he specifically
bequeathed to me. It is a good hone. It has been a good hone since 1833.
In fact it does a better job, with less trouble than the contraption. I
suspect that there are by this time thousands like me. Ours is indeed a
faithless generation. And the Church does so little for us. Beards are
coming in again, I suspect. Some of my best friends are sporting
mustaches. And one of them has a red beard a foot long---says it
prevents colds.

Well, one man can't be expected to stand alone against these heresies.
And the Church is impotent, or at least silent, while the evil grows.
There is House of David, for example. And Senator J. Ham Lewis. And
Chief Justice Hughes. Old King C. Gillette would have known how to meet
that issue like a man and a Shaver. But if the Church has ever issued a
bull against Justice Hughes I have no record of it.

Now that the Church has lost its grip, I suppose it's a matter for the
NRA.

A great industry is at stake. The livelihoods of thousands of workers
hang in the balance. Congress ought to pass a law.




% CHAPTER ELEVEN
\chapter[11 \hspace*{1mm} BEAUTY AND THE AD-MAN]{11 BEAUTY AND THE AD-MAN}

 \newthought{We have} seen that, since advertising is essentially a traffic in belief,
the profession habitually takes the name of Truth, though usually in
vain. But since Beauty is Truth, Truth, Beauty, the profession is also
forever rendering vain oblations at the shrine of Beauty.

This worship has two major phases. The first is the manufacture, by
advertising, of successive exploitable concepts of feminine beauty, of
beauty in clothes, houses, furniture, automobiles, kitchens, everything.
The second phase of this worship has to do with the ad-man's view of his
own craft, and would appear to represent, in part at least, a perversion
of the normal human instinct of workmanship.

From some reason it is thought necessary for the ad-man, not merely to
sell the idea of beauty for profit, but to sell beauty beautifully. Why?
Is there not something excessive and pathological about advertising's
will-to-be-beautiful?

It is contended that an attractively designed advertisement of an
allegedly beautiful toilet seat is more effective than an ugly
advertisement of the same object. But this has never been proved
conclusively. On the contrary, there are many examples of very ugly
advertising which have been exceptionally effective. Yet the desire for
beauty in advertising is inextinguishable and has more or less had its
way. Fifteen years ago the well-designed newspaper or magazine
advertisement was the exception; today it is the rule. Has the
effectiveness of advertising increased proportionately? On the contrary,
it has decreased, and one of the factors in this decline is undoubtedly
the increased cost of producing this economically superfluous beauty in
advertising. In any case, beauty of design or text is only one of the
many variable, more or less unknown and unpredictable factors in the
selling relationship established by the advertisement. And finally, it
would be easy to show that even in 1929, when artists were often paid
\$2,000 for a single painting, photographers \$500 for a single print
and typographers equally fancy prices---even in the heyday of
art-in-advertising, cheap and ugly advertisements frequently sold goods
just as well or better. And today, what could be uglier than the inane,
story-in-pictures advertisements which sell Lux, Fleischmann's Yeast,
Lifebuoy Soap, and other products with demonstrated effectiveness?

There is, of course, a recognized and demonstrated commercial
justification for using expensive ``art'' and expensive typography in
the advertising of certain luxury products such as perfumes, de luxe
motor cars and the like. The principle is that of ``conspicuous waste,''
used to create an ambience, a prestige for the product, which will lift
it above the rational level of pride competition. The familiar snob
appeal, applied to such prosaic commodities as fifteen-cent cigarettes
and twenty-five-cent collars, also accounts for a good deal of
conspicuous expenditure in advertising ``art,'' and up to a certain
point, this is commercially justifiable. Yet it remains true, as many
hard-boiled professionals have pointed out, that beauty has been
permitted to run hog-wild in contemporary advertising practice. Carroll
Rheinstrom, Advertising Manager of \emph{Liberty}, was recently quoted
in \emph{Advertising and Selling} as believing that 90\% of current
advertising is waste because of the ad-man's pre-occupation with his own
techniques, to the exclusion of practical economic considerations.

No, the logical economic explanations don't make sense. Advertising
today, while anything but efficient, is far better designed and written
than it needs to be; obviously it costs far, far more to produce than it
ought to cost. Part of the explanation, I think, lies in a private
impurity of the advertising craftsman; he is more interested in beauty
than he is in selling. For him the advertisement is a thing-in-itself .
Highly developed craftsmanship in the graphic arts and in writing,
enormous expenditures of mechanical skill, are deposited at the shrine
not of Mammon but of Beauty. And all pretty much in vain. The art isn't
really art. The writing isn't really writing. And frequently the worst
``art'' and the worst ``writing'' sell products better than the best art
and the best writing.

\enlargethispage{\baselineskip}

Yes, the explanation of this curious phenomenon may well be that
advertising, since it doesn't make sense in economic, social or human
terms, jumps right through the Looking-Glass and becomes a
thing-in-itself!

It takes a na\"ive eye to see this. I had to have it pointed out to me by
a poet friend who makes his living writing prose for a very expensive
magazine. He picked up a copy of the publication and pointed to a Camel
cigarette advertisement in color. How much did that cost, he asked? I
estimated rapidly: \$1,000 for the drawing, add \$200 for the time of
the art director and an assistant, \$400 for the color plates, \$100 for
typography, \$100 more for miscellaneous mechanical charges, \$100 for
copy, \$300 pro-rated for executive and management charges. Total for
one advertisement, not counting the cost of the space, about \$2,300.

``Well,'' commented my poet friend, ``that's the end-result, isn't it?
That's why Kentucky planters go bankrupt growing tobacco, why negro and
white share croppers sweat, starve and revolt, why millions of men and
women diligently smoke billions of cigarettes all so that this
magnificent advertisement might be born and live its little hour.''\looseness=-1

My friend was treating himself to a little poetic license, of course.
But the more I stared at the phenomenon, the more I became convinced
that it made just as much sense upside down as right side up. And the
more I reflected upon the r\^ole of the ``creative worker'' in
advertising, the more I came to suspect skullduggery of an obscure,
unconscious sort. Ostensibly these craftsmen are employed to write words
and draw lines that will persuade their fellow man to buy certain
branded cigarettes, soaps, toothpastes, gadgets, etc. But do these
fellows really give a whoop about these gadgets and gargles or whether
people buy them or not? Did I, when I was a member in good standing of
the profession?

Never a whoop nor a whisper. What I cared about was my craft, and that
is what every genuine craftsman cares about---that and nothing else.
Each piece of copy was a thing-in-itself. I did a workman-like job, not
for dear old Heinz, or Himmelschlussel, or Rockefeller, or whomsoever I
was serving indirectly, but for myself; because it was pleasant to do a
competent job and unpleasant to do a slovenly job. I was aware, of
course, that Mr. Rockefeller, via the agency, was paying me, and I tried
not to get fired. But I never worried about my duty to Mr. Rockefeller
and to his oils and gadgets. The prospect, the customer? I was a bit
sorry for the customer, and tried to let him off with as little
bamboozlement as possible. But my real loyalty was to the Word, to the
materials of my craft. Loyalty to the Word---writing a competent
advertisement---sometimes meant being pretty rough and mendacious with
the customer. I couldn't help that. I was carried away by the fury of
composition, just as a good Turkish swordsman becomes carried away in
his professional dealings with the Armenians.

But chiefly, I think, my indomitable instinct of workmanship was hard on
my employer. Unconsciously I sabotaged his interests continuously. I
wrote clean, lucid prose, when the illiterate screed that the advertiser
wanted to print would probably have sold more goods. When my immediate
superior plaintively objected that what I wrote was too good for the
audience to which it was addressed, I was indignant and recalcitrant.
Ordered to rewrite the advertisement, I seized the opportunity to bring
it closer to my standard of craftsmanship, which had nothing to do with
commerce. If the client objected, I bullied him if possible, and
otherwise made a minimum of grudging concessions.

A percentage of the copy writers in advertising agencies are craftsmen.
I have known scores of them. They felt as I felt, and consciously or
unconsciously, they did what I did. The artists were even more obsessed
and obstreperous. As I knew them, their disinterestedness in the profits
of Mr. Rockefeller was extreme. They were interested in drawing pretty
pictures. They drew them as well as they could, regardless of whom and
of what? Regardless of the advertiser and what he had asked them to
draw. Naturally, the picture had to convey a sales message, and they
chattered a great deal about ``putting a selling punch'' into their
pictures. But I noticed that the best of them became so interested in
the design and the drawing that they frequently left no room for the
copy or even for the trade-mark of the manufacturer. (This last I
suspect was a trick of the Freudian unconscious; the trade-mark was
resented because it was the signature of the advertiser.) When account
executives and advertisers repined at such extravagant oblations at the
shrine of Beauty, the artists were haughtier even than the copy writers.
And since the average American business man has a puzzled and diffident
reverence for art, coupled with an enormous ignorance of the nature of
artists, their motivations and techniques, these so-called
``commercial'' artists did then and still do get away with an
astonishing amount of sheer mayhem and murder. The writers, too, though
to a less degree, because most advertisers can read and write. The
technique is less strange and the technician correspondingly less
formidable. All account executives in agencies, and worse still, all
advertisers, have an obscene itch to write themselves. Consequently the
copy writer must sternly and vigilantly keep these vulgarians in their
places. I always considered it to be my duty to stand on my dignity as a
``genius''---the word still goes big in the world of commerce,
especially on the West Coast---and \emph{epater} these bourgeois, partly
as a matter of self-respect, and partly as a practical measure of
professional and personal aggrandizement.

Commercial artists and writers indeed! Art for art's sake was our motto,
and to hell with the advertiser. I can remember not one, but half a
dozen times when an advertisement was written, illustrated, set up in
exquisite type, and deposited in proof form on the account executive's
desk almost ninety-nine and three-quarters per cent pure. True, the text
had more or less to do with the product which we were supposed to be
advertising, but the advertiser's ``message'' was merely a point of
departure for the copy writer's lovingly executed exercise in pure
design, and the typography was a study in black on white which made no
concessions whatever to readability. The advertiser's trade-mark and
signature were either carefully concealed or left out entirely. Usually,
of course, these pure triumphs, these pious oblations at the shrine of
Beauty, caused the account executive to yell bloody murder. He was right
and we knew he was right. We had gone too far. We would therefore
execute a careful retreat from such tactical excesses, grumbling dourly
for the sake of the record that the account executive was obviously an
ignoramus, and that his precious client was a misbegotten idiot whom we
would like to kill and stuff with his own Cheery Oats, or whatever it
was he sold; that, however, as loyal employees of dear old Kidder,
Bidder \& Bunkstein we would gladly give him what he wanted and hoped it
choked him.

We never did, of course, for that would have been to concede too much.
So that the client was kept in a constant, salutary state of baffled
rage, alarm and hope; and every now and then an unhappy account
executive would have a nervous breakdown. We never had nervous
breakdowns.

Does this seem exaggerated? But how can the honest chronicler record
fantasy except in the terms of fantasy? And the vast accumulation of
advertising during the post-war decade was fantastic in the extreme. It
is still fantastic. Look at it in the pages of any commercial magazine.
Does it make sense in terms of the sober, profit-motivated business that
advertising is supposed to be? Recently the investigators of the
Psychological Corporation discovered that the variation as between
advertisements of lowest and highest effectiveness runs as high as 1,000
per cent. An automobile assembly line is considered poor if it permits a
quality variation of more than 30 per cent. Is it sensible to believe
that a production technique which frequently shows 3,000 per cent
variation in the quality of the product is really aimed at its avowed
objective, namely the sale of products and services to customers? Well,
if I were out duck shooting and missed my duck by 1,000 per cent, I
should consider it open to question whether or not I was really trying
to hit that duck.

No. To understand this phenomenon we must employ a far subtler analysis,
giving all the factors their due weight, no matter how fantastic these
factors are, and no matter how seemingly irrational the conclusion to
which we are led.

Again, Veblen furnishes us with the essential clue. In the \emph{Theory
of Business Enterprise} and elsewhere in the whole body of his work,
Veblen notes that advertising is one element of the ``conscientious
sabotage'' by which business keeps the endlessly procreative force of
science-in-industry from breaking the chains of the profit system.\footnote{{[}Thorstein Veblen,
  \emph{\href{http://www.worldcat.org/oclc/220847741}{The Theory of
  Business Enterprise}} (New York: Charles Scribner's Sons, 1904).{]}} In
this view the business man figures as an art-for-art's-saker. His art is
the making of money, which has nothing to do with the use of the
productive forces by which a society gains its livelihood. The art of
making money is perhaps the purest, the most irrational art we know, and
its practitioners are utterly intransigent. Today these artists in money
making are prepared to starve millions of people, to plunge the planet
in war, to destroy civilization itself rather than compromise the purity
of their art.

Veblen saw all this clearly, and Stuart Chase has employed the Veblenian
apposition of business and industry in a sequence of useful books. But
one might well go further and assert that the contradictions of
capitalism persist even within the mental gears and pistons of its
exploitative functionaries.

Business sabotages industry by means of advertising. True. But we, as
advertising craftsmen, consciously or unconsciously motivated not by a
desire to make money but by an obsessed delight in the materials of our
craft---we in turn sabotaged advertising. We were and are parasites and
unconscious saboteurs. During the whole postwar decade we gathered
strength, inflated our prestige, consolidated our power. More and more
the ``creative worker'' became the dominant force in agency practice,
and advertising consequently became more and more ``pure.'' The shrine
of Beauty was buried under the fruits and flowers placed there by devout
artists and writers in advertising. We were no humble starvelings. We
caused the salaries and fees paid advertising artists and artists to
become notorious. Even I, who was always more or less aware of what I
was doing, and who was indifferent to money for its own sake---even I,
without particularly trying, because I never could keep more than a
fraction of my mind concentrated on the absurd business, managed to
triple my salary during the postwar decade. Agency production costs hit
the ceiling, broke through and sailed off into the empyrean. We
developed an esthetic of advertising art and copy, a philosophy, a
variety of equally fantastic creeds---a whole rich literature of
rationalization which should interest the psychiatrists greatly if they
ever get around to examining it.

I say ``we'' with poetic license. I speak for the profession, but I
speak out of turn, and I shall doubtless be roundly repudiated and
contemned by the menagerie of Cheshire Cats, March Hares, Mad Hatters
and Red Queens who still roam the scant pastures on the other side, the
\emph{right} side of the Advertising Looking-Glass. As a matter of fact
I contributed nothing to this literature of rationalization. I was too
busy making a living, trying to keep sane and do a little serious work
on the side, and wondering just how soon that beautiful iridescent
bubble would break, leaving us ``creative workers'' with nothing much in
our hands and a lot of soap in our eyes.

It broke. Came Black Thursday, and a chill wind blew through the
advertising rookeries of the Grand Central District. Advertising
appropriations were cut. That exquisite First Article of the Ad-Man's
Credo: ``When business is good it pays to advertise; when business is
bad you've got to advertise,'' was invoked with less and less effect. As
the months and years passed the whole structure of the industry began to
sideslip and sway. \emph{And advertising became less pure}. That
beautiful, haughty odalisque had to hustle down into the market place
and drag in the customers. She had to speak of price. She became dowdy
and blatant and vulgar. The primitive techniques of Hogarth in the
eighteenth century were resurrected via the tabloids, and the moronic
sales talk issued in ugly balloons from the mouths of ugly moronic
figures. Photography was cheaper than drawings and worked as well or
better. Testimonials were cheap and worked best of all.

Desperately, advertising began to step out of its part and tell the
truth a little. The customer got an occasional break. But advertising
lost her name, the poor girl. And it got worse. Every time car loadings
hit a new low, another big advertiser would go buckeye with testimonials
and other loathsome practices, and she would lose her name again.
Alarmed, the reformers of advertising started another vice crusade, and
their activities will be described elsewhere. They haven't accomplished
much, despite General Johnson's benediction pronounced on the ``good''
advertising that will be needed more and more under the New Deal. Their
voices become ever fainter and more faint.

Quite evidently the religion of Beauty-in-Advertising has entered upon a
period of decadence. The advertisers, being only one jump ahead of the
sheriff, or more often two jumps behind, are obliged to cut each other's
throats without benefit of Beauty. In fact many of them, having learned
wisdom from the tabloids, are openly blasphemous and vengeful with
respect to the art-for-art's-sakers. Pursued by their unforgiving
maledictions, the Priests of Beauty have fled to Majorca or Vermont,
where they nurse their wounds and wait. Not all of them, however. In
1932 and 1933, a few stalwarts attempted a counter-offensive against the
sansculottes who had laid waste the pleasant fields of advertising. The
more or less recognized leader of this gallant Lacedemonian band is Mr.
Rene Clarke, President of the firm of Calkins \& Holden, Inc., one of
the oldest, most ethical, and most respected advertising agencies in
America. Mr. Clarke is a genuinely gifted designer whose worship of
Beauty is without flaw or compromise. Among his many triumphs is that of
so glorifying Wesson Oil that millions of American housewives consume
tons of it, under the impression, doubtless, that it is a kind of
champagne.

When the evil days came, Mr. Clarke had no pleasure in them, and no
sympathy for the panic-stricken advertisers who with more or less
success were trying to lift themselves out of the spreading sea of red
ink by the balloon technique borrowed from the tabloids. Hence, after
the slaughter of the morons had proceeded without benefit of Beauty for
three depression years, Mr. Clarke, in 1932, published in
\emph{Advertising and Selling} the pronunciamento which is here quoted
in full:

\begin{center}
    \textit{Challenge}
\end{center}

\begin{quote}
\emph{Bring me Idealism:} I'm tired of things that look like things as
they are. Have you buried your hearts like pots of gold in the earth?
You who are entrusted with the responsibility of showing others what
they cannot see for themselves. If your eyes see only what is seen by
others, from where will the vision come? You who have been so disdainful
of the ordinary, will you stand aside now and let the ordinary lead you
back to the paths that stretch up to the heights?

You claimed to be the leaders, the gifted, sensitive few, who discerned
and brought into being the beauty that is truth. The quality of
leadership is tested by adversity. Because we have adversity, do you
renounce your leadership and hoard your visions against that time when
some one else has made a market for your talent?

Is your sense of beauty so delicate that it cannot be exposed to the
frost? Will you come out again like house flies at the first warm touch
of prosperity's spring?
\end{quote}

\begin{quote}
\emph{Bring me Courage:} I'm tired of conformity that hides behind the
general use. It is indeed a low level that parallels the taste of the
throng. If we all conform, wherein will the crowd find guidance away
from the common level? You say it narrows your market. Nothing of worth
has been created with one eye on one's market. One needs both eyes and
yet more to see into one's heart, and it is from there that truth is
born.

Courage walks alone, even in the market places. The crowd must follow
where the trail is blazed. Look at your idols. Did they hesitate because
no one had been that way before? Did they wait for acceptance before
they advertised their principles?
\end{quote}

\begin{quote}
\emph{Bring me Imagination:} I'm tired of today and want to see
tomorrow.

I need an image, not of what I am, but of what I hope to be. Put away
the mirror; set up the telescope. Was it not yesterday you boasted that
your souls had wings, that you could penetrate rare atmospheres where
the rest of us could not exist? Fly now, and bring us down a measure of
that ozone.

Bring us back from those excursions of the mind, which are the
responsibility of your guild, a portion of wine to wash down our dry
daily fare---wine from the vineyards of romance and imagination.

If you bring us only bread, you become mere housewives serving the needs
of the body, and we recede step by step from that estate which breeds
the very license of your occupation.

Have you no contacts with the gods that you only recite the
conversations of the world? What binds you to this circling round and
round? Can you not stretch your tether ever so little that the next
circle would be trod on untrampled ground?

Do you listen to those who counsel return to something which we had but
have lost. That is the creed of those who lack imagination or courage
and the refuge of those without plan. What we had we have not now. It
belongs to yesterday, not today nor tomorrow. Others may lean on and
borrow from the past, but you may not. Yours is the responsibility to
create the new, the fresh, the vital vision of tomorrow, what we hope to
be.
\end{quote}

Obviously Mr. Clarke has gone dada, and I trust no person in this
audience will be so ungracious as to ask what he is talking about. In
the old days, when, in the heat of copy and art conferences, advertisers
voiced such impertinent questions, we always boxed their ears and told
them to mind their own business, if any. Often there was little enough
by the time we got through with them.

I regard Mr. Clarke's manifesto as a classic of its kind, and not
without its historic interest; for Mr. Clarke himself is perhaps the
last of the art-for-art's-sakers in advertising. His manifesto is
illustrated by a most artistic photographic study of the artist himself,
standing with one hand resting on his hip, the other hand lifted and
placed upon a pillar of the temple of advertising, the clear, unsubdued
eyes gazing into the distance. The pose is suggestive, even ominous.
What does this Samson of Art-in-Advertising mean to do? Shorn of his
prestige, will he gird his loins once more, and bring the whole temple
roaring down upon the heads of the Philistines? It would be a fitting
end.

Let us turn now to a consideration of the primary phase of the Ad-Man's
worship of Beauty: the manufacture by advertising of successive
exploitable concepts of feminine beauty, of beauty in clothes, houses,
furniture, automobiles, kitchens, everything. One notes three major
points: first, that these concepts must be as rapidly obsolescent as
possible; second, that the connotation of beauty with expensiveness is
rigorously enforced; third, that beauty is conceived of as functional
with respect to profitable sales, rather than with respect to satisfying
beautifully and economically the living and working needs of the
population.

Most exploitation of the idea of beauty reduces in practical terms to a
promotion of sales and profits through the fostering of obsolescence.
This is most apparent in the field of women's fashions. Here the
exploitative apparatus includes not only advertising in the narrow sense
of the word, but also the editorial propaganda of the style magazines,
plus a more or less collusive hook-up with the rotogravure supplements
of the newspapers, with stage and motion picture actresses, and with
Junior League debutantes. This complex promotion apparatus is utilized
to achieve, first, the fundamentally false identification of beauty and
fashion. The acceleration of fashion changes during the postwar period
is an index of the textile industry's rapid emergence into the ``surplus
economy'' phase of capitalism, with its entailed crisis. The life-span
of a successful style was roughly about a year in 1920. Today, according
to the testimony of well-known stylists, this life-span has dropped to
less than six months. The mortality of the candidates for fashion's
favor has correspondingly increased.

Winifred Raushenbush, in an article in the \emph{New Freeman}, described
the dilemma of the dress manufacturer who knows that nine out of every
ten designs are doomed to ``take a bath,'' to use the trade jargon. This
mortality is about equally high throughout the fashion industry, whether
in hats, dresses or cloaks, and whether the manufacturer is serving the
high, medium or low style markets. Snobbism is, of course, the major
instrument of the promotion technique. The exquisite hauteur with which
both the advertisers in \emph{Vogue} and the editor of \emph{Vogue}
lecture their nouveaux riche readers is matched only by the slightly
burlesqued imitation of this manner to which indigent stenographers are
subjected when they look for bargains on Fourteenth Street. The
diffusion of a fashion change, both as to geography, and as between the
high, medium and low style levels has become almost instantaneous.
Emulative pressures are invoked all down the line. Women dress today not
merely for men, but for women as a form of social competition. So potent
is the style-terror that even during the depression the majority of
women would rather starve than risk the shame of nonconformity. They
save and scrimp, skip lunches, buy the latest mode, and four months
later are obliged to buy again---this time an ``ensemble,'' so that the
manufacturers of handbags and even cosmetics may also share in the
profits of style-obsolescence.

Deterioration of function fostered by advertising is especially
conspicuous in the field of fashion. Even in expensive high-style
apparel, the materials tend increasingly to be shoddy. And the crowning
joke is that for about fifty per cent of American women, the dress,
cloak and hat manufacturers do not produce, do not even attempt to
produce, clothes which have any relation to the physical type of the
women who are asked to buy them! This, at least, is the testimony of
Miss Raushenbush in another \emph{New Freeman} article entitled
``15,000,000 Women Can't go Nude.'' They don't go nude, of course. They
accept the ruthless prescription of the current fashion, which is
usually appropriate for the young flapper type. It looks and fits like
the devil on the mature woman, the short woman, the tall woman, the
``hippy'' woman. There are at least five major feminine types of these
``forgotten women'' the existence of whom the fashion industry has
barely deigned to notice, let alone serve adequately.

In recent years the attempt has been made to extend the sway of fashion,
\emph{i.e.}, profit-motivated obsolescence, into every conceivable field
of human purchase and use. Invariably this fashion offensive wears the
masque of beauty. Almost invariably, the net result is to increase the
tonnage of shoddy make-believe. One must say this at the same time that
one acknowledges in fairness that the industrial designers who have both
promoted and profited by this offensive, have tried to introduce some
slight measure of the substance and function of beauty, and in some
cases have measurably succeeded.

The motivation of this crusade is acknowledged in the title of an
article contributed by Earnest Elmo Calkins to \emph{Advertising and
Selling:} ``The Dividends of Beauty.'' One readily acknowledges that
nothing, whether beautiful or ugly, can be made under a profit system
unless it does pay dividends. The point is that under a profit system
both the guiding esthetic and its expression by a profit-motivated
industry are severely limited and distorted, so that the net product of
beauty is likely to be meagre indeed. Says Mr. Calkins:

\begin{quote}
The place of art in industry is becoming firmly established. A
restaurant arranges common vegetables in patterns in its windows, taking
full advantage of the different greens of peas, asparagus, cauliflower
and artichoke, and adds eye-appeal to appetite appeal. A railroad
landscapes its stations with grass plots and climbing roses and
transforms an unsightly utility into an attractive eye-catcher, builds
local goodwill, adds an esthetic touch to mere ordinary travel, and
creates a new sales argument.

Much has been accomplished in this new field, but the list is long of
manufactured articles waiting for that beautifying touch which costs but
little and adds so much to acceptance. The initial shape and color of
most machine-made articles are ugly. Why, I don't know. Nature does not
err that way. All her products are artistic and harmonious with each
other. Some appeal to several senses. An ear of corn is pleasant to
sight and touch.... Nothing but man with his filling stations, hot-dog
stands and automobile cemeteries strikes a discordant note.... A forest
grows unhelped and is forever beautiful. A town grows as it will and
looks like hell hit with a club. Beauty in man-made articles must be the
result of conscious thinking....
\end{quote}

Mr. Calkins, a veteran of the advertising profession, admits that he
doesn't know why most machine-made articles are ugly. By and large, the
writer must admit a similar ignorance. The glib radical answer would run
to the effect that it is not the machine, but the application of machine
technology to the making of profits that results in this ugliness. But
this answer doesn't cover all the facts by any means. Some machine-made
articles, even some machine-made consumer's goods, made for profit and
sold at Woolworth's, \emph{are} beautiful. Many handmade articles are
ugly---Elbert Hubbard's de luxe editions for example, and much of the
present flood of sweatshop toys, china, etc., coming out of Japan and
Germany; also the neoMayan design in pottery and textiles which results
when the primitive social-economic pattern of a Mexican village is
shattered and the native craftsmen are Taylorized by a capitalist
entrepreneur. Yet the burial urns and other art objects turned out in
quantity during some of the best Chinese periods, trade-marked, and
exported for profit to Persia, were and are extremely beautiful.
Production for use does not necessarily result in beauty, nor does
production for profit necessarily result in ugliness. Estheticians and
sociologists have striven vainly to discover the rationale of beauty in
the social context of production, sale and use. The best that the writer
can offer is a tentative observation to the effect that the American
genius, operating under the conditions of modern industrial capitalism
burns brightest, and gives the largest product of beauty in the field of
producer's goods: the machines themselves, turbines, electric cranes,
modern factory architecture and the product of these factories for
strict use seen in bridges, viaducts, etc. On the other hand the
American blind spot is in the field of economic and social organization;
hence we are likely to find that a machine product, designed for sale to
the ultimate consumer usually, though not always betrays the disorder,
the insanity, the ugliness of our decadent capitalist economy and our
chaotic distributive system. In general I think it may be said that
where the salesman and advertiser, rather than the craftsman and
producer, are in the saddle, what the consumer gets is likely to be
ugliness. In a fragmented civilization such as ours, art and the artist
tend to be tossed off to the periphery of a system which no longer is
organic. Mr. Calkins would like to bring the artist back to the center
of the system, where, as industrial designer, he can contribute ``that
beautifying which costs but little and adds so much to acceptance.'' The
attempt is in fact being made on a considerable scale, but without much
success, and for very good reasons.

A very good industrial designer---there are a number of highly talented
Americans at work in this field---can control some but not all of the
factors which determine whether a product is to be beautiful or ugly. He
can't control the profit-motive and that is precisely where he falls
down. As a matter of fact, who is it calls in the industrial designer?
The advertising agency, usually, or the sales manager of the
manufacturer. Why do they call him in? To make the product a beautiful
object? Incidentally, perhaps, but primarily to make the product a
\emph{salable} object. The designer hence must work not as an artist,
but as a showman, a salesman. If he were working as an artist, he would
make the form of the object express the truth of its function, not
merely in mechanical but also in economic and social terms, and it would
be beautiful. But his is perforce a one-dimensional art. Working as he
must, as a showman, he usually gives the object a novel flip of line or
color---he ``styles'' it in terms of the showman, not of the artist. As
a designer he finds himself frustrated and stultified by the false and
anti-social production relationships which condition his labor. The same
thing is true, of course, of the engineer, the educator, the doctor, the
architect, indeed of all creative workers in an acquisitive society.
Recently one of the best known and most highly paid industrial designers
in America came to me and asked what chance he would have of doing
serious work in Russia. He was fed up with the rootless frivolities that
sales managers had asked him to turn out.

\enlargethispage{\baselineskip}

It is in the field of package design that the artist has greatest
freedom and has scored his maximum of seeming successes. It is true that
simple, bold lettering, clear colors and good design produce more
sightly packages and that customers are attracted by such packages. It
is also true that these packages are likely to contain the same
overpriced, overadvertised and sub-standard content that they always
held. This package ``beauty'' is therefore skin deep, and its creation
the proper concern of business men and commercial dilettantes, not of
artists who have any conception of the social function of art. If these
packageers had any such conception they would probably feel obliged to
ask first, in three cases out of five, whether the product really ought
to be packaged at all.

It occurs to me that in discussing the r\^ole of the craftsman in
advertising I may have given the impression that his ``conscientious
sabotage,'' his interest in the materials of the craft rather than in
selling, his attempts to convert advertising into a thing-in-itself,
represent a genuine release of creative capacity. No such impression was
intended. If any genuine creation goes on in advertising agencies I have
never seen it. I have seen the sort of thing described: the crippled,
grotesque, make-believe of more or less competent craftsmen who played
with the materials of their craft but could never use them
systematically for any creative purpose. By and large there is no such
thing as art in advertising any more than there is such a thing as an
advertising literature.

The best of us, certainly, had more sense than to make any such
pretensions. I suppose that in some twelve years of advertising practice
I must have written some millions of words of what is called
``advertising copy,'' much of it for very eminent and respectable
advertisers. It was all anonymous, thank heaven, and I shall never claim
a line of it. True, half-true and false, the advertiser signed it, the
newspapers and magazines printed it, the radio announcer blatted it, and
the wind has blown it away. It was all quite without any human dignity
or meaning, let alone beauty, and it cannot be too soon forgotten.

No, we knew what we were. On the door of the art department of an agency
where I worked, a friend of mine, one of the ablest and most prolific
commercial artists in the business, once tacked a sign. It read: ``Fetid
Hell-Hole of Lost Souls.''

There are many hundreds of these ``fetid hell-holes'' in the major
cities of America. The inmates are, of course, dedicated to beauty,
beauty in advertising. Whether they knew it or not they are, as artists,
so many squeaking, tortured eunuchs. The Sultans of business pay them
well or not so well. They have made sure that they do not fertilize the
body of the culture with the dangerous seed of art.


% CHAPTER TWELVE
\chapter[12 \hspace*{1mm} SACRED AND PROFANE LOVE]{12 SACRED AND PROFANE LOVE}

 \newthought{In tracing} the pattern of the ad-man's pseudoculture, we come next to
the concept of love, which figures as an ingredient in most of the
coercions of fear and emulation by which the ad-man's rule is
administered and enforced. The theory and practice of this rule are
clearly indicated in the title of a comparatively recent advertising
text book by Mr. Kenneth M. Goode: \emph{How to Turn People into
Gold}.\footnote{{[}Kenneth M. Goode,
  \emph{\href{http://www.worldcat.org/oclc/612499286}{How to Turn People
  into Gold}} (New York: Harper \& Bros., 1929).{]}} As a practicing alchemist in his own right and also as an
agent of that purest of art-for-art's-sake gold-diggers, the business
man, the ad-man treats love pragmatically, using every device to extract
pecuniary gain from the love dilemmas of the population. The raw ore of
human need, desire and dream is carefully washed and filtered to
eliminate all impurities of intelligence, will and self-respect, so that
a deposit of pure gold may be precipitated into the pockets of the
advertiser.

The enterprise of turning people, with their normal sexual desires and
human affections, into gold, is greatly helped by the fact that our
Puritan cultural heritage is peculiarly rich in the psychopathology of
sex. This social condition is in itself highly exploitable, but it is
not enough. The ad-man is in duty bound not merely to exploit the mores
as he finds them, but further to pervert and debauch the emotional life
of our literate masses and classes. He must not merely sell
love-customers; he must also create love-customers, for, as we have
seen, the advertising profession is nothing if not creative.

The dominance of the love appeal in contemporary advertising must be
apparent to every reader of our mass and class magazines, as well as to
the Great Radio Audience. Curiously enough, it would appear that the
so-called ``higher'' manifestations of sex---its moral, ethical,
spiritual and romantic derivatives and sublimations, the domestic
affections and loyalties of husbands and wives, and of parents and
children, are more exploitable than the grosser sexual appetites. Love
rules the world, and the greatest triumph of modern advertising is the
discovery that people may be induced to turn themselves into gold simply
by a forthright appeal to their better natures, as a kind of public
duty, since it is recognized in all civilized communities that gold is
more beautiful and more valuable than people. Today, therefore, many of
our most successful advertisers stand, like John P. Wintergreen in ``Of
Thee I Sing,'' squarely upon the broad platform of Love, and when their
campaigns are conducted with proper vigor, skill and enthusiasm, their
election is almost automatic, as in the Third Reich. This, at least, is
the contention of many eminent members of the advertising profession.

The distinction between sacred and profane love is difficult to
maintain, and is in fact frequently blurred in current advertising
practice. For convenience in examining the evidence, perhaps the
following categories will serve:

\emph{Sacred Love}. The affections and loyalties of husbands and wives.
Maternal, paternal and filial affections. Religious and charitable
impulses. Respect for the dead. Idealism in romantic love, this being
closely related to the concepts of chastity and beauty.

\emph{Profane Love}. The physical intimacies of adolescents, such as
kissing, petting, etc. The problem created by sexual desire on the part
of both the married and the unmarried, as complicated by the desire not
to have children.

Illustrative material in both categories is so abundant that the
specimens cited in this exposition will necessarily fail to include many
of the most distinguished achievements of contemporary advertising. No
slight is intended, and any reader who wishes to do so can easily
correct the balance by a brief survey of the advertising pages of
current mass and class magazines.

The sanctity of marriage is a major item in the Christian idealism of
love. I quote at this point an advertisement by the Cadillac Motor
Company which exploits this idealism with all the resources of modern
advertising technique:

\vspace{1mm}

\begin{center}
I DO
\end{center}

\begin{quote}
It may have been but a decade ago ... or it may have been in the beautiful
90's ... but sometime, somewhere, a young man stood in the soft light of
a Junetime morning ... and repeated the words ... ``I do.'' ... Since that
time he has fought, without interruption, for the place in the world he
wants his family to occupy.... And it may be that, out of the struggle,
he has lost a bit of the sentiment that used to abide in his heart
 ... for success is a jealous master and exacts great servitude.... But
not when the Junetime comes ... and, with it, that anniversary of
\emph{another} June! ... Then the work-a-day world, with its many tasks,
is cast abruptly aside, and sentiment---pure and simple---rules in his
heart once more.... And, because there are literally thousands of him,
doorbells are ringing this June throughout America ... and smiling boys
in uniform stand, hats in hand, with the proof of remembrance.... And
along with the beautiful flowers, and the boxes of candy, and the other
tokens ... some of those brides of other Junes will receive the titles to
new Cadillacs... and for them there will be no other June like
this---save one alone.... There is a Cadillac dealer in your
community---long practiced in the art of keeping a secret.... Why not go
see him today? You can trust him not to tell!
\end{quote}

Note the exquisite, hesitant style. The copy writer knows he is treading
on sacred ground. Do not blame him for using the ``three dots'' device
invented by that fleshly Broadway columnist, Walter Winchell. Rather,
one should admire the catholicity of spirit by which profane techniques
are converted to sacred uses. Note that this tender message to fond
husbands, written not without awareness of its effect upon wives,
focuses upon the \emph{proof} that he has remembered his marriage
anniversary. Ladies, by their works ye shall know them. The more costly
the proof the more profound the sentiment. On that remembered June she
got a husband. This June she gets a Cadillac. Clearly the one was a
means to the other. Note too that only \emph{some} wives will get
Cadillacs, precious both in themselves and as emulative symbols in the
endless race to keep up with the Joneses.

In the original advertisement the photograph of orange blossoms was
reproduced in color. Beauty, sentiment, tact, effrontery---by means of
these reagents the advertising alchemist converts the pure and beautiful
devotion of husbands into something still more pure. Gold. Pure gold.

Advertisers believe enormously in children. They have lavished immense
sums upon the education of parents in matters of infant care and
feeding, the prevention of disease, etc. Much of that education is sound
enough, much of it is irresponsible and misleading, and all of it, of
course, is anything but gratuitous. I have before me an advertisement of
Cream of Wheat which shows the familiar scare technique used in
exploiting parental devotion. The headline, ``At the Foot of My Baby's
Crib I Made a SOLEMN PROMISE'' is melodramatic even as to typography.
What's it all about? The baby in the fable was shifted from milk to
solid food not Cream of Wheat and got sick. The doctor, who judging from
his photograph might well be a retired confidence man, tells the parents
to feed the baby Cream of Wheat. The inference is that if he'd been fed
Cream of Wheat from the beginning, he wouldn't have become sick, which
is itself an impudent enough non sequitur. Add the fact that semolina, a
non-trade-marked wheat product used by macaroni manufacturers, is in the
writer's experience of baby-feeding, an entirely satisfactory equivalent
for Cream of Wheat costing about a third as much, and you get a measure
of the advertiser's effrontery. Compute Cream of Wheat's share in the
huge annual levy of over-priced and de-natured breakfast cereals on
American food budgets, and you get a measure of the advertiser's service
to the American Home and the American Kiddy. The writer might add,
merely as his professional opinion, that without advertising the
breakfast cereal business would wither in a year, with very considerable
benefit to the health and wealth of American men, women and children.

\emph{Death}. It is probable that but for the ineffable mortician and
his confederate, the casket-maker, we might by this time have modified,
in the direction of decency, taste and economy, some of the grotesque
burial rites that we inherit from our savage ancestors. But no. It still
costs a tired, poverty-stricken American laborer about as much to die
and be buried as it does a high-caste Balinese, and the accompanying
orgies are, of course, infinitely more hideous. It is scarcely worth it.
Readers interested in this macabre traffic are referred to the study by
John C. Gebhardt for the Russell Sage Foundation. Advertising plays its
part, of course, and the appeal, in terms of menacing solemnity, is
invariably to the love of the bereft ones for the departed. New York
columnists still remember the maggoty eloquence of one Dr. Berthold E.
Baer in behalf of Campbell's Funeral Church, under such headlines as
``Buried with her Canary Bird,'' ``Skookum,'' etc. This series ran in
New York newspapers during the winter of 1919-1920. The current
advertising of the National Casket Company is scarcely less gruesome.

\emph{Romance}. When we enter the starry fields of romance, the
advertising lines begin to blur, and we can never be sure whether we are
dealing with love in its sacred or in its profane aspects. Of one thing,
however, we can always be sure. We are in the field of sex competition,
and the advertiser, with his varied stock of cosmetics, soaps, gargles
and deodorants, figures as Love's Armourer; also, perhaps, as schatchen;
also---well, the Elizabethans had a word for it. The advertiser's sales
patter runs somewhat as follows: ``You want a lover. Very well, gargle
with Blisterine, use such and such soaps and cosmetics, and let Cecilia
Bilson teach you how to be charming without cost.'' The exploitation of
love's young dream is by this time a huge industry in itself. Recently,
advertisers of such remotely serviceable products as radios and razor
blades have been trying to muscle in on it.

\emph{Profane Love}. When we come to the ``marriage hygiene''---n\`ee [\emph{sic}] 
``feminine hygiene''---advertisers it becomes clear that we are dealing
with the physical aspects of love. Physical love is taboo in our society
except when legalized by the State; taboo also, if one were to take our
various and tangled State and Federal statutes seriously (which
practically nobody does) except when having procreation as its object.
The d\'ebris of the law, reflecting as it does our obsolete mores, is
ridiculous enough---in Connecticut, for example, it is legal for a drug
store to sell contraceptive devices but illegal for a man or woman to
use them.

Very few people obey the law, of course. Birth control is today one of
the facts of American life. It is practiced, or at least attempted in
some form, almost universally.

But the laws remain on the statute books. The shadow of the taboo
remains, and in this shadow the advertising profession operates what is
probably the most flourishing racket in America, now that Capone is in
jail and prohibition is no more.

In the files of Consumers' Research I counted leaflets advertising some
fifty different antiseptics and other contraceptive products, and in the
files of the National Committee on Maternal Health, some hundred and
fifty more. Neither organization attempts to list them all; the total
probably runs into thousands. Each is represented either directly or by
implication to be a convenient, safe and reliable contraceptive.
Meanwhile the gynecologists of the world have been searching for
precisely such a thing and say they haven't yet found it. Meanwhile, the
leaders of the English Birth Control movement, in despair, are demanding
the legalization of abortion, and of sterilization as in Russia.
Meanwhile Margaret Sanger and her lieutenants in the American Birth
Control movement are pointing out that the existing legislation which
prohibits the dissemination of birth control information is really class
legislation. Upper and middle-class people whether married or not can
get advice from their doctors and buy contraceptives at drug stores. The
fifty per cent of the population which lives at or below a subsistence
level can afford neither doctors nor rubber goods. Only a few thousand
can be accommodated by the present capacity of the birth control
clinics.

But gynecologists are merely scientists and Mrs. Sanger is merely the
gallant and indomitable Mrs. Sanger. They scarcely rank with Doctor
Sayle Taylor, LL.D., now, because of the querulousness of the American
Medical Association. As the ``Voice of Experience,'' Doctor Taylor
comforts thousands of wounded hearts over the radio. In his personal
appearances before Men Only and Women Only he details the mysteries of
love and sells little booklets full of highly dangerous misinformation
and not lacking the address of a contraceptive manufacturer.

But how about the respectable drug houses whose annual ``take'' from the
contraceptive racket far surpasses that of the eloquent ``Doctor''?

The hired ad-men of these drug houses perform miracles of delicacy in
conveying to the magazine readers half-truths and outright deceptions.

Take Lysol, for example. In their monumental study ``The Control of
Conception,'' Dr. Robert L. Dickinson and Dr. Louise Stevens Bryant say
flatly that Lysol should be banned as a contraceptive. Not that it isn't
a good antiseptic. It is indeed, a powerful antiseptic---too powerful to
be used for contraceptive purposes except in weak solutions which the
average woman can scarcely be trusted to make with accuracy and not
reliable in any case. Further, the clinical evidence to date both in
England and in America, indicates that no antiseptic douche is at all
dependable as a contraceptive in and of itself.

In the earlier stages of the feminine hygiene campaigns, the language of
the ad-men was full of euphemisms, of indirection, of tender solicitude
for the sad-eyed wives pictured above such captions as ``The Very Women
who supposed they knew, are grateful for these enlightening facts.'' But
recently the pressure of competition has speeded up the style. ``Now it
Can be Told,'' they declaim, and ``Why mince words?''

Some of them don't; for example, the ad-man for Pariogen tablets, who
writes the following chaste communication, addressed presumably to the
automobile trade:

``Pariogen tablets may be carried anywhere in a purse, making hygienic
measures possible almost anywhere, no other accessories or water being
required.''

It has been argued that birth control education is a necessary social
job, and that the ad-men are doing it. The answer to that is that they
are doing it badly, irresponsibly and expensively, with a huge
by-product of abortion and other human wreckage and suffering. Thus far
birth control has been the obsession of a few honest crusaders like Mrs.
Sanger, Dr. Dickinson, and Dr. W. J. Robinson. For support, it has had
to let itself be made the plaything of philanthropic social
registerites, and say ``please'' to an organized medical profession so
divided in its counsels, so terrified of offending the mores, and so
jealous of its emoluments that it has dragged on the skirts of the
movement rather than assume the courageous leadership which is not
merely its right but its obvious duty. The medical societies of Michigan
and Connecticut are notable exceptions to this judgment.

Despite such handicaps, the labors of Mrs. Sanger, Dr. Dickinson and
others, aided by the gradual relaxation of the taboo since the war, have
achieved the following major results:

\begin{enumerate}
\item
  Some 144 clinics functioning in 43 States.
\item
  A technique, which while far from ideal or even completely reliable is
  successful in 96 to 98 percentage of cases.
\item
  An increasing penetration of the daily and periodical press with birth
  control propaganda. (Except for one or two liberal stations with
  negligible audiences, birth control is still barred from the air.)
\item
  Laboratory and clinical research at Yale, the Universities of London
  and Edinburgh, and elsewhere, which may at any moment yield
  revolutionary results. Russia, of course, has endowed such research
  heavily and may be first to solve the problem.
\item
  The establishment of birth control courses in practically all of the
  leading medical schools, and a considerable propagandizing of the
  profession through the \emph{Birth Control Review} which, however, was
  discontinued in July, 1933.
\end{enumerate}

What could be built now, on the foundations laid by the devotion of
these pioneers? The answer runs in terms of economics, politics and
sociology. A birth control clinic operated on a fairly large scale, such
as the Sanger Clinic in New York, can provide instruction, equipment and
clinical followup for about \$5.00 per year per patient. Multiply that
\$5.00 by about twenty million and you get \$100,000,000 a year as the
bill for a publicly administered contraceptive service of approximate
adequacy. Would it be worth \$100,000,000? Of course. Will anything of
the sort be done? Probably not. Why? The Pope and the Propaganda of the
Faith, which still, to paraphrase Veblen, ``ignores material facts with
magisterial detachment''---one of these facts being that wherever birth
control clinics have been opened they have been patronized by Catholics
in full proportion to the percentage of Catholics in the populations
served. The Fundamentalists are equally obstructive, although their
magazines cheerfully publish contraceptive advertising. Alas, of course,
the big drug houses, which doubtless would interpose objections on
purely moral, ethical and spiritual grounds. Also the Fourth Estate,
whose freedom to defend the sanctity of the home must not be impugned or
calumniated by any suspicion of a material interest arising out of the
advertising income received from the before-mentioned drug houses. Also
the medical profession, a small part of which feels itself obliged, like
the advertising profession, to turn human life into gold, a large part
of which is plain stupid and timid, and a part of which---a small
part---is magnificent and may be counted upon to go the limit at almost
any cost to itself.

In contrast to what is being done by the birth control clinics and what
might be done by an intelligent expenditure of public funds, let's have
one more look at how the job is being done by business men and
advertisers interested solely in ``service'' and ``truth.''

It is roughly estimated that the American people spend about
\$25,000,000 a year for contraceptive devices and materials. Largely
because of the failure of these commercially exploited hit-or-miss
techniques, Prof. F. J. Taussig of Washington University estimates that
there are about 700,000 abortions every year in this country. This
situation is, of course, highly exploitable, especially because of the
bootleg nature of the traffic. The most popular contraceptive sells at a
profit to the retail druggist of nearly 1000 per cent. According to Mr.
Randolph Cautley of the National Committee on Maternal Hygiene, the
advertising of abortifacients in the pulp magazines increased 2800 per
cent in one year---between 1932 and 1933. It is, of course, a
commonplace of medical knowledge that no abortifacient is effective and
that all of them are highly dangerous as well as illegal. In his survey
which was incomplete because of the limited funds at the disposal of his
organization---the three major contraceptive advertisers spent a total
of \$412,647 in 1933---Mr. Cautley counted 16 advertisers who were
obviously selling abortifacients, 35 who were selling contraceptives and
20 classified as ``uncertain.'' The abortifacient copy is especially
discreet. ``Use it when nature fails you,'' they advertise, and ``For
unnatural delay. Double strength. Rushed first class mail.'' Now and
then the Food and Drug Administration catches one of these rats, but it
is difficult, and will continue to be difficult even under the
strengthened provisions of the Copeland Bill.
 

% CHAPTER THIRTEEN
\chapter[13 \hspace*{1mm} SCIENCE SAYS: Come Up and See Me Some Time]{13 SCIENCE SAYS: Come Up and See Me Some Time}
\chaptermark{13 SCIENCE SAYS}

\newthought{The} mission of the ad-man is sanctified by the exigencies of our
capitalist economy and of our topsy-turvy acquisitive pseudoculture. His
mission is to break down the sales resistance of the breadlines; to
restore prosperity by persuading us to eat more yeast, smoke more Old
Golds and gargle more assorted antiseptics.

In fulfilling this mission it is appropriate that the ad-man invoke
divine aid. The god of America, indeed of the modern world, is the
scientist. Today it is only in the Fundamentalist, Sunday School
quarterlies that God wears long white whiskers. In the advertising pages
of the popular magazines He wears a pince-nez and an imperial; sometimes
He squints through a microscope; or, instead of Moses' rod, He
brandishes a test tube. The scripture which accompanies these pictorial
pluckings of modern herd responses is austere, erudite, and asterisked
with references to even more erudite footnotes. The headline, however,
is invariably simple and explicit. In it the god says that yeast is good
for what ails you.

The god is often a foreign god, resident in London, Vienna, Paris or
Budapest. That makes him all the more impressive---and harder for the
skeptical savants of the American Medical Association to get at and
chasten.

In response to a recent inquiry printed in the \emph{Journal of the
American Medical Association}, these savants remarked: ``Yeast is so
uncertain in laxative effect that it is hardly justified to classify it
among the cathartics.... That, among the hosts of persons taking yeast a
skin disorder clears up occasionally is not surprising. The association
might be entirely accidental. The history of yeast, the periodic waning
and gaining in favor, suggest that it has therapeutic value, but that
this value is slight indeed.''

Sometimes, as in the case of yeast, the god is appeased by appropriate
sacrifices: \$750, f.o.b. London, was the price offered to and declined
by one prominent English medico. Advertisers, however, have little
difficulty in rounding up plenty of less fastidious impersonators of the
deity, and the required honorariums are distressingly small---less than
half what is normally paid to society leaders. After being duly salved
and photographed, surrounded by the paraphernalia of his profession, the
``scientist'' gives his disinterested, expert, scientific opinion. But
sometimes he goes further. He \emph{proves} that the advertisers product
is the best.

The makers of Old Gold cigarettes have gone in heavily for this sort of
proof. A while back they proved that Old Gold is the ``coolest''
cigarette. This demonstration was made by Drs. H. H. Shalon and Lincoln
T. Work, for the New York Testing Laboratories. They proved, using the
``bomb calorimeter,'' the ``smokometer'' and other assorted abracadabra,
that an Old Gold cigarette contains 6576 B.T.U.'s; whereas Brand X
contained 6688 B.T.U.'s, Brand Y 6731 B.T.U.'s and Brand Z 6732
B.T.U.'s.

What, by the way, is a B.T.U.? It is an abbreviation for ``British
Thermal Unit''---a measurement of \emph{heat content}. If Old Golds
contain a fraction of a per cent less B.T.U.'s than the other tested
cigarettes, does that make them any ``cooler.'' Not by a jugful. What
does it prove? Nothing.

Scientists of this stripe are almost painfully eager to show that they
are good fellows---that they are prepared to ``go along.''
Intellectually, they are humble creatures---the altar boys and organ
blowers of the temple of science. They have wives with social ambitions
and children who need shoes. They lack advancement, and when
advertisers, who are often very eminent and respectable, make friendly
and respectful overtures, they are often very glad to serve the needs of
business.

Such friendships would doubtless be more general but for certain
unwarranted apprehensions, especially prevalent among the banking
fraternity. The strong men of Wall Street have been slow in realizing
that the glamorous Lady Lou and many of these stiff, spectacled earnest
creatures of the laboratory know their place in an acquisitive society;
that beneath that acid-stained smock there often beats a heart of gold.

Recently Mr. Kettering, vice president and research director of General
Motors, felt obliged to defend the engineer against the banker's charge
that he is upsetting the stability of business. Said Mr. Kettering, with
a candor which cannot be too much admired: ``The whole object of
research is to keep every one reasonably dissatisfied with what he has
in order to keep the factory busy in making new things.''

This definition of the object of engineering research may seem a little
startling at first. But it must be remembered that Mr. Kettering is not
merely an engineer, a scientist, but also a corporation executive and as
such a practical business man. In fact, it might almost be said that in
the statement quoted Mr. Kettering speaks both as a scientist and as an
advertising man; a scientific advertising man, if you like, or an
advertising scientist. Hence, when he says in effect that in our society
the object of scientific research is the promotion of obsolescence in
all fields of human purchase and use, so that profit-motivated
manufacturers may be kept busy making new things, his words, even though
they sound a little mad, must be listened to with respect. It would
appear that under the present regime of business, subject as it is to
the iron determinants of a surplus economy, the sales function must be
reinforced in every possible way. Hence the lesser departments of
science, with their frail purities, their traditional humanities, their
obsolete and obstructive idealisms, will be brought more and more under
the hegemony of the new ``science'' of advertising, than which no
department of science is more pure, more rigorous. The objects and ends
of this science are predetermined: they are, quite simply, to turn
people into gold, or to induce people to turn themselves into gold.

The medical experimenter may have qualms about vivisecting his guinea
pigs until he has first anesthetized them. The biologist may drop a tear
over his holocausts of fruit flies. But the young Nietzscheans who run
the advertising agencies observe a sterner discipline. The science of
advertising is the science of exploitation, and in nothing is the ad-man
more scientific, more ruthless than in his exploitation of ``science.''
He is beyond the ``good'' and ``evil'' of conventional morality. Not for
a moment can he afford to forget his motto: ``Never give the moron a
break.''



% CHAPTER FOURTEEN
\chapter[14 \hspace*{1mm} WHOSE SOCIAL SCIENTIST ARE YOU?]{14 WHOSE SOCIAL SCIENTIST ARE YOU?}

\newthought{As advertising} became more and more an essential part of the mechanism
of sales promotion, and as our newspapers and magazines took definite
form as \emph{advertising businesses}, the advertising profession became
highly respectable. It was part of the status quo of the acquisitive
society and could be effectively challenged only by persons and
interests standing outside this status quo.

As already indicated, the product of advertising was a culture, or
pseudoculture. Advertising was engaged in manufacturing precisely the
material which our economists, sociologists and psychologists are
supposed to study, measure and interpret---necessarily within some
framework of judgment. What framework? Where did our social scientists
stand during advertising's period of expansion and conquest?

They stood aside for the most part while advertising proceeded to play
jackstraws with the ``law'' of supply and demand, and other items of
orthodox economic doctrine. Thornstein [\emph{sic}] Veblen saw the thing clearly and
his brief treatment of advertising in \emph{Absentee Ownership} remains
today the most exact description of the nature of the advertising
phenomenon which has yet appeared.\footnote{{[}Thorstein
  Veblen,~\emph{\href{http://www.worldcat.org/oclc/752183}{Absentee
  Ownership and Business Enterprise in Recent Times: The Case of
  America}}~(New York: B. W. Huebsch, 1923), chap 11.{]}} But Veblen was a lone wolf all his
days. And it has been the journalists, publicists and engineers, rather
than the professors, who have made most effective application of
Veblen's insights. Stuart Chase, a disciple of Veblen, has worked
without academic sanctions, while the director of Consumers' Research,
Mr. F. J. Schlink, is an engineer, and Mr. Arthur Kallet, his
collaborator in the writing of \emph{100,000,000 Guinea Pigs} is
another.\footnote{{[}Arthur Kallet and F. J.
  Schlink,~\emph{\href{http://www.worldcat.org/oclc/546417}{100,000,000
  Guinea Pigs: Dangers in Everyday Foods, Drugs, and Cosmetics}}~(New
  York: Vanguard Press, 1933).{]}} For the most part, orthodox economists have either ignored
advertising, or in very brief and inadequate treatments, have complained
gently about its ``vulgarity,'' as if, in the nature of the case, it
could be anything but vulgar. A notable exception is the chapter on
``Consumers in the Market'' by Professor Corwin Edwards in the second
volume of \emph{Economic Behavior} by members of the Economics
department of New York University.\footnote{{[}Corwin Edwards, "Consumers in the Market," in
  \emph{\href{http://www.worldcat.org/oclc/337694}{Economic Behavior: An
  Institutional Approach}}, edited by Willard Earl Atkins and Donald
  William McConnell, vol. 2, 20--40 (New York: Houghton Mifflin Company,
  1931).{]}} Against this competent and
forthright analysis, however, must be set the sort of thing which
Leverett S. Lyon, economist of Brooking's Institute, contributes to
Volume I of the \emph{Encyclopedia of Social Sciences}. I quote here the
concluding paragraph of Mr. Lyon's article:

\begin{quote}
Consumer advertising is the first rough effort of a society becoming
prosperous to teach itself the use of the relatively great wealth of new
resources, new techniques, and a reorganized production method. Whatever
eventually becomes of advertising, society must provide some device for
this task. Some agency must keep before the consumer the possibilities
resulting from constant advance, for the world appears to be learning to
produce goods ever faster. Today the voices crying most loudly in the
wilderness of consumption are more concerned with noisily advertising
the weaknesses of advertising than with patient teaching of standards of
taste which will reform advertising by indirection. Other action is
possible. An increase of government specifications would help, although
not as much as is often thought, and they would require an enormous
amount of advertising. What is most needed for American consumption is
training in art and taste in a generous consumption of goods, if such
there can be. If beauty is profitable, no manufacturer is desirous of
producing crudity or vulgarity. Advertising, whether for good or ill, is
the greatest force at work against the traditional economy of an
age-long poverty as well as that of our own pioneer period; it is almost
the only force at work against puritanism in consumption. It can infuse
art into the things of life; and it will, if such an art is possible,
and if those who realize what it is will let the people know.\footnote{{[}Leverett S. Lyon, ``Advertising,''
  \emph{\href{http://www.worldcat.org/oclc/168443}{The Encyclopedia of
  the Social Sciences}}, edited by Edwin R. A. Seligman, vol. 1 (New
  York: Macmillan, 1930), 470.{]}}
\end{quote}

Intelligent and honest advertising men, at least, will have no
difficulty in recognizing this as a piece of advertising copy about
advertising. Like practically all advertising copy it is a piece of
special pleading and its appearance in an otherwise excellently edited
reference work is calamitous enough in all conscience.

It may be observed incidentally that Mr. Lyon is a frequent contributor
to the advertising trade press. He stands well within the status quo,
not merely of orthodox economic teaching, but of the advertising
business itself. It is natural enough that he should rationalize and
justify the r\^ole of advertising in our society, while making the usual
pretense of ``objectivity.''

\enlargethispage{\baselineskip}

The fact is, of course, that as advertising became powerful and
respectable it had a good many well-paid jobs to offer social
scientists, and that none of these jobs tolerated any degree of
``objectivity'' whatsoever: Jobs of teaching merchandizing and market
analysis in schools of business administration; jobs for statisticians
as directors of research in advertising agencies; jobs for psychologists
in testing new devices of cozenage, measuring ``consumer reactions,''
etc. There can be no doubt as to whom these social scientists belong.
They belong to the advertising business, and they can no more write
``objectively'' about that business than a copy writer can write
objectively about his client's gargles and gadgets.

With the rapid growth of the schools of business administration since
the war, these business-minded economists, psychologists, statisticians,
etc., came to rival in number and in influence their colleagues in the
departments of economics and psychology proper. But even the strictly
academic social scientists, practitioners of a ``purer'' discipline,
found increasing difficulty in sustaining their claim of ``objectivity''
and the younger ones, especially the economists, pretty much gave it up.
Both the motivation and the futility of this claim are well analyzed by
Mr. Sidney Hook in an unpublished manuscript:

\begin{quote}
The fascination of physical science for the social theorists is easy to
explain. Not only does it possess the magic of success, but what is
vastly more important, the promises of agreements and objectivity. In
the popular mind, to be objective and to be ``scientific'' are
practically synonymous terms. What is more natural, therefore, than the
fact that in a field in which prejudice, bias, selective emphasis are
notorious, there should be a constant appeal to a neutral point of view.
It is this quest for objective truth from a neutral point of view,
independent of value judgments, which has become the great fetich of
American social science.

It cannot be emphasized too strongly that the social activity which
contributes the subject matter of the social sciences is an activity
carried on by human beings in pursuit of definite ends. If we take these
ends as our starting point nothing is clearer than the fact that these
ends, whether they be of individuals or of classes, conflict. Social
conflicts are a real and permanent feature of the society in which we
live. Every attempt to develop an objective social science which will do
for social organization what science has done for technology must
grapple with the difficulty that there are as many directions in which
social reorganization may be attempted as there are social classes. The
attempt to evade this class conflict and to refuse to regard it under
existing conditions as fundamental is behind the strenuous effort to
emulate the ``exact sciences'' in which the only recognized conflict is
between the ``true'' and the ``false.''
\end{quote}

Taking, as Mr. Hook suggests, the ends sought by advertising as the
proper starting point for a consideration of the phenomenon, let us
return to Mr. Lyon's forensic summation and see what it amounts to. He
says: ``Consumer advertising is the first rough effort of a society
becoming prosperous to teach itself the use of the relatively great
wealth of new resources, new techniques and a reorganized production
method.'' In the first place, advertising is conducted by and for
advertisers, and the dissemination of a material culture which it
accomplishes is strictly in the interest of the advertiser, primarily,
and of the total apparatus of the advertising business secondarily. The
advertiser is concerned with ``teaching'' the consumer only in so far as
such teaching profits the advertiser and the routine product of
advertising is therefore pretty consistently mis-educational rather than
genuinely educational. This ``teaching'' involves not merely huge
economic wastes but a definite warping and conditioning of the
consumer's value judgments into conformity with the profit-motivated
interests of the advertiser.

Mr. Lyon proposes, by implication, a ``patient teaching of standards of
taste which will reform advertising by indirection.'' A teaching by whom
and for whom? Advertising is itself a tremendous ``educational'' effort
which operates in the interest of the advertiser with incidental profit
to the consumer only in so far as he can disentangle the truth from a
mass of special pleading, this incidental profit being vastly
overbalanced by the mis-educational pressures exerted not merely on his
pocketbook but upon his ``taste,'' that is to say, his value judgments.
Advertising, as Veblen said, is not merely an enterprise in sales
promotion, but an enterprise in the production of customers which
necessarily becomes an enterprise in ``creative psychiatry.''\footnote{{[}Veblen, \emph{Absentee Ownership}, 307n12.{]}} Does
Mr. Lyon propose that this huge \emph{interested} mis-educational and
anti-cultural activity be balanced and corrected by another educational
activity? In whose interest? Financed and conducted by whom? By
Consumers' Research, perhaps? By government? But why should any
government which pretends to govern in the interests of the people as a
whole proceed by ``indirection''; that is to say, educate consumers to
resist in their own interest the ``education'' which advertisers
disseminate in \emph{their} interest? Wouldn't it be simpler to
eliminate your negatives first and then see how much and what kind of
positive education is required?

Advertising, says Mr. Lyon, ``is almost the only force at work against
puritanism in consumption.'' By what right and in whose behalf does he
introduce this value judgment into his argument? Maybe our people would
prefer a little more puritanism in consumption, intolerable as such an
attitude may be to advertisers operating in the ``surplus economy''
phase of industrial capitalism. And does advertising really work against
puritanism in consumption? What do you mean, puritanism in consumption?
Buying wheat for what it is worth instead of ``puffed wheat'' at eight
times as much? Buying a radio instead of shoes for the baby?

Advertising, says Mr. Lyon, ``can infuse art into the things of life, if
such an art is possible, and if those who realize what it is will let
the people know.'' How? By more advertising, doubtless, along the lines
so frequently proposed by Mr. Bruce Barton and Mr. Walter Pitkin in the
interests, not of the ``people'' but of the advertiser and the
advertising business?

One gives space to such lamentable rationalizers as Mr. Lyon only
because he represents so typically the values, attitudes and motives of
the ad-man's pseudoculture as they are currently set forth by
advertising apologists. We shall encounter precisely the same kind of
logical jabberwocky when we come to consider the radio and the movies.
Meanwhile, let us have a look at the r\^ole of the psychologists. 



% CHAPTER FIFTEEN
\chapter[15 \hspace*{1mm} PSYCHOLOGY ASKS: How Am I Doing?]{15 PSYCHOLOGY ASKS: How Am I Doing?}
\chaptermark{PSYCHOLOGY ASKS}

\newthought{Advertising}, defined as the technique of producing customers, rather
than the technique of selling goods and services, employs well-known
psychological devices, and the advertising man is, in fact, a journeyman
psychologist. Academic and business school psychologists are therefore
naturally and properly interested in advertising as a field of study.
But when the quality and effects of this interest are examined, there
would appear to be a conflict between the layman's naive view of
psychology as a disinterested ``objective'' scientific discipline, and
certain current activities of academic psychologists in the field of
applied psychology.

In 1920, the founder of the American school of ``Behaviorism,'' Dr. John
B. Watson, resigned his professorship at Johns Hopkins and entered the
employ of the J. Walter Thompson Advertising Agency.\footnote{{[}See Kerry W. Buckley,
  ``\href{https://doi.org/10.1002/1520-6696(198207)18:3\%3C207::AID-JHBS2300180302\%3E3.0.CO;2-8}{The
  Selling of a Psychologist: John Broadus Watson and the Application of
  Behavioral Techniques to Advertising},''~\emph{Journal of the History
  of the Behavioral Sciences}~18, no. 3 (1982): 207--21; and Peggy J.
  Kreshel, ``\href{https://doi.org/10.1080/00913367.1990.10673187}{John
  B. Watson at J. Walter Thompson: The Legitimation of `Science' in
  Advertising},''~\emph{Journal of Advertising}~19, no. 2 (1990):
  49--59.{]}} Psychologists
have questioned the originality and value of Dr. Watson's contributions
to the young science of psychology. But his contributions, as a business
man, to the technique of advertising are outstanding.

The J. Walter Thompson Company is one of the largest and most
consistently successful advertising agencies in the world. Over the past
fourteen years the advertising which it has turned out has betrayed
increasingly the touch of the master's hand. It is good advertising,
effective advertising. It is also more or less unscrupulous, judged by
ethical standards, even the ethical standards of the advertising
profession itself. It is natural that this should be so, since ethical
considerations are irrelevant to the application of scientific method in
the exploitation of the consumer.

Consider the advertising of such products as Fleischmann's Yeast,
Woodbury's Facial Soap, Lifebuoy Soap, Pond's Vanishing Cream,
etc.---all J. Walter Thompson accounts of long standing. In this and
other advertising prepared by this agency, the fear-sex-emulation
formula is used systematically to ``condition the reflexes'' of the
reader into conformity with the profit-motivated interests of the
advertiser. By putting the bought-and-paid-for testimonial technique on
a mass production basis, this agency has doubtless achieved important
economies for the advertiser in the production of customers. Dr.
Watson's agency was also one of the leaders in the adaptation to
advertising of the story-in-pictures-balloon technique borrowed from
Hogarth via the tabloids. Objections on the score of ethics and taste
are met by the realistic argument that the market for these products
consists chiefly of fourteen-year-old intelligences, and that the
unedifying means used to convert these morons into customers are
justified by the ends achieved: the profits accruing to the advertiser,
the internal and external cleanliness of the moron, and the fixation of
systematized illusions in the minds of the public, necessary to the use
and wont of an acquisitive society.

Nothing succeeds like success. Probably Dr. Watson was never obliged to
ask his employers, ``How am I doing?'' His achievements were manifest,
and his present salary as vice president of his agency is reputed to be
four times the maximum stipend of a university professor.

Nothing succeeds like success. It may well be alleged that the prestige
of business dominates the American psychology, not excepting the
psychology of American psychologists. Veblen, whose approach to
economics was through social psychology and the analysis of
institutional arrangements, had an Olympian respect for himself, and no
respect whatever for business. But in terms of pecuniary aggrandizement
and academic kudos, Veblen got nowhere during his lifetime. Hence it was
natural that in the field of applied psychology, contemporary
psychologists would have chosen to follow Watson rather than Veblen.

In 1921, the year following the elevation of Dr. Watson's talents to the
realms of pecuniary accumulation, an organization called the
Psychological Corporation was incorporated under the laws of the State
of New York.\footnote{{[}See Michael M. Sokal,
  ``\href{https://doi.org/10.1002/1520-6696(198101)17:1\%3C54::AID-JHBS2300170108\%3E3.0.CO;2-R}{The
  Origins of the Psychological Corporation},'' \emph{Journal of the
  History of the Behavioral Sciences}~17, no. 1 (1981): 54--67.{]}}

The stock of the corporation is held by some 300 American psychologists,
all of them members of the American Psychological Association, and most
of them having the status of professor or assistant professor in
American universities and colleges.

The second article of the corporation's charter reads as follows:

\begin{quote}
The objects and powers of this corporation shall be the advancement of
psychology and the promotion of the useful applications of psychology.
It shall have power to enter into contracts for the execution of
psychological work, to render expert services involving the application
of psychology to educational, business, administrative and other
problems, and to do all other things not inconsistent with the law under
which this corporation is organized, to advance psychology and to
promote its useful applications.
\end{quote}

This article is quoted in one of the sales pamphlets issued by the
corporation and is supplemented by the following paragraph:

\begin{quote}
In the hands of those properly qualified, psychology can be applied
usefully to many problems of business and industry, and of educational,
vocational and personal adjustment. The purpose of the Psychological
Corporation is to promote such applications of the science and to
prevent, where possible, its exploitation by pseudo-scientists. A
portion of all fees for services rendered by the corporation is devoted
to research and the advancement of scientific knowledge of human
behavior.
\end{quote}

At a special meeting of the stockholders and representatives of the
corporation, held in conjunction with the 1933 convention of the
American Psychological Association at Chicago, Dr. Henry C. Link,
Secretary and Treasurer of the corporation, presented his report. In
effect Dr. Link was appealing to the value judgments of his colleagues.
He was saying: the corporation has been doing such and such things.
Business, especially the advertising business, thinks we have been doing
pretty well. How do you think we have been doing?

There was a row, a fairly loud row, judged by academic standards, and it
got into the papers. Some of the assembled psychologists, themselves
stockholders in the corporation, seemed to feel that Dr. Link had sold
the integrity, the purity of American psychology down the river to the
advertising business. Among the more forthright objectors was Dr. A. W.
Kornhauser, associate professor of Business Psychology at the University
of Chicago. It is interesting to note that the most strenuous objection
came, not from one of the science-for-science's-sake psychologists, but
from a business school professor. Perhaps it was because Dr. Kornhauser
is more aware of the nature and methods of business than some of his
less sophisticated associates. But before we discuss this row, it will
be necessary to describe briefly the sort of thing that the
Psychological Corporation had been doing.

Perhaps the most distinguished achievement to which Dr. Link pointed
with pride was co-operative study, carried on by sixty psychologists, of
the effectiveness of advertising, particularly among housewives. Dr.
Link's report of this study was published in the January, 1933, issue of
the \emph{Harvard Business Review}.\footnote{{[}Henry C. Link, ``A New Method of Testing Advertising
  Effectiveness,''~\emph{Harvard Business Review}~11 (1933): 165--77.{]}}

Between March 16 and April 4, 1933, 1,578 housewives in 15 widely
scattered cities and towns were interviewed by instructors and graduate
students of psychology working under the supervision of some fifteen
assorted Ph.D.'s and M.A.'s. They used a test questionnaire which asked
such questions as the following:

\begin{quote}
What canned fruit company advertises ``Just the Center Slices''? What
toothpaste advertises ``Heavens! Buddy must have a girl!''? What product
used in automobiles uses pictures of \emph{little black dogs} in its
advertising? What product asks ``What is the critical age of the skin''?
What toothpaste advertises ``Pink Toothbrush''? What product for use in
automobiles has been using advertisements showing pictures of fish,
tigers, flying geese and other animals? What do 85\% of dentists
recommend (according to an advertisement) for purifying the breath? What
soap advertises ``I learned from a beauty expert how to hold my
husband''? What does, for a product used in automobiles, \emph{what
butter does for bread}? What company or product advertised ``This is
Mrs. F. C. Adgerton of Spokane, Washington''? What company advertises
``Don't wait till the doctor tells you to \emph{keep off your feet}''?
What electric refrigerator is ``Dual-automatic''? What company
advertises a widely used toilet product as often containing ``harmful
acids''?
\end{quote}

There is a total of twenty-seven questions of this sort on the
questionnaire and the housewives had to answer all of them. The mind
shrinks from contemplating either the amount of high-powered
psychological persuasion required to hold them to their task, or the
sufferings endured by these 1,578 female guinea pigs in the cause of
``science.'' How many doorbells had to be rung before one willing
housewife was captured? Did they suffer? And how much? Dr. Link should
have answered those questions, too. I am sure the answers would prove
something, although I am not sure just what.

What \emph{was} proved, beyond question, when the questionnaires were
all turned in, collated, tabulated, analyzed, etc., by the most rigorous
scientific methods, was that, sure enough, housewives did read
advertising. I quote from Dr. Link's article:

\begin{quote}
The outstanding result of this test is the proof of the amazing
influence which advertising can and often does exert. For example, 1,090
or 69\% of the 1,578 housewives answered ``Chase \& Sanborn'' to the
question about the ``Date on the can.'' The correct answer, ``Ipana''
was given by 943 or 59.7\% of these women to the question regarding
``Pink Toothbrush.'' On the other hand, the themes of certain very
extensive campaigns registered correctly among only 15.65\%, 11.3\%, and
even 7\% of these housewives. In some cases, single advertisements,
appearing only once, registered better than campaigns which had run in
all the major magazines for six months, a year, or longer. That is to
say, some advertising was 50, 100 or 150 times more effective, as
measured by this test, than other advertising. The most conspicuous
example of this was the result of the question, What soap advertises
``Stop those runs in stockings''? This was the headline, explained in
the copy, of a full-page advertisement for Lux soap which had appeared
in just one of the leading women's magazines. Almost one half of the
housewives, 47.7\%, answered ``Lux.'' This one insertion, costing about
\$8,000, was found six times as effective as a year's campaign
advertising another article and costing about a million dollars, a ratio
of 750 to 1. The average of correct answers to the thirteen most
effective campaigns or advertisements was 36.3\%. The average for the
fourteen least effective was 8.8\%.
\end{quote}

The writer is not qualified to judge the scientific integrity of Dr.
Link's methods. But the findings of this study are manifestly highly
interesting and useful to advertisers, advertising agencies and
advertising managers of publications, \emph{who, incidentally got all
this research for nothing}. It was done gratuitously by the co-operating
psychologists, assistants and students, as a disinterested effort toward
the ``advancement of scientific knowledge of human behavior.'' ... Well,
perhaps not wholly disinterested. The published study was in effect, a
free sample and an advertisement of the sort of thing the Psychological
Corporation is equipped to do. Doubtless it was a successful
advertisement, since the corporation during 1933 conducted many
scientific investigations, sponsored and paid for by individual
advertisers, and conducted by its wideflung organization of psychology
professors, instructors and students.

In other words, what Dr. Link was presenting proudly to his assembled
colleagues was a successful advertising business, operating efficiently
according to current standards, and using advertising to sell its
services. Incidentally this business is in a position to cut the market
price for advertising research because public and philanthropic funds
help to support the co-operating professors, and they in turn are able
to use their students as Tom Sawyer labor, sustained wholly or in part
by the pure passion of science.

Whether ``scientific'' or not, that study of 1,578 housewives was
indubitably a contribution. To whom and for what end? Not to science,
but to the advertising business, to the end that it might conduct more
efficiently its effort to ``teach the use of the relatively great
wealth, of new resources, new techniques and a reorganized production
method.'' (L. S. Lyon's definition in the \emph{Encyclopedia of Social
Sciences}).\footnote{{[}Leverett S. Lyon,
  ``Advertising,''~\emph{\href{http://www.worldcat.org/oclc/168443}{The
  Encyclopedia of the Social Sciences}}, edited by Edwin R. A. Seligman,
  vol. 1 (New York: Macmillan, 1930).{]}}

This effort makes systematic use of techniques which are most accurately
characterized by Veblen's phrase: ``creative psychiatry.''\footnote{{[}Thorstein
  Veblen,~\emph{\href{http://www.worldcat.org/oclc/752183}{Absentee
  Ownership and Business Enterprise in Recent Times: The Case of
  America}}~(New York: B. W. Huebsch, 1923), 307n12.{]}} For
example, one of the advertising campaigns tested was that of Ipana
Toothpaste, which for the past ten years or more has been parroting
``Pink Toothbrush,'' in the effort to make people worry about their gums
and buy an expensive toothpaste, the use of which is alleged to prevent
the gums from bleeding, the advertising being the customary melange of
half-truth, inference and ambiguity.

When, therefore, Dr. Link appealed to the suffrages of his professional
colleagues, it was upon the following grounds: that the Psychological
Corporation has established efficient machinery by which its members
might sell their scientific abilities and the leg work of their students
to advertisers engaged, to quote Veblen once more, in ``the creative
guidance of habit and bias, by recourse to shock effects, tropismatic
reactions, animal orientation, forced movements, fixation of ideas,
verbal intoxication.... A trading on that range of human infirmities
which blossom in devout observances and fruit in the psychopathic
wards.''

What happened? The next annual meeting of the Board of Directors of the
Psychological Corporation was held in New York on\\ \noindent Dec. 1, 1933. The
managing director, Dr. Paul S. Achilles, explained that the objections
of Dr. Kornhauser and others may have arisen from insufficient knowledge
on the part of many psychologists of the charter and purposes of the
corporation and the nature and extent of its current activities. He said
that inasmuch as the corporation had never been subsidized nor conceived
as an organization to be supported by subsidies, his efforts for the
past three years had necessarily been concentrated chiefly on putting
the corporation on a self-sustaining basis.

It was Dr. Achilles' opinion that the two basic assumptions on which the
corporation was founded are: (1) That psychologists render services of
economic value; and (2) that a business organization of co-operative
psychologists rendering such services could not only be self-supporting
and useful to the science but could earn funds for research and
improvement of services. He felt that only as the corporation succeeded
first in demonstrating its capacity for self-support through rendering
creditable and marketable services such as it was now offering could it
hope to achieve its larger aims. In brief his feeling was that it was
equally if not more respectable for psychologists to earn their own way
and their funds for research than to depend on subsidies.

Dr. W. S. Woodworth, of Columbia, expressed the opinion that one of the
original aims of the corporation was to have frankly a commercial
standing so that it could do business with business men with more
freedom and directness than a university professor usually feels that he
can. Further, in regard to the corporation's market survey work, that
this seemed a legitimate field and that the mere fact that a market
study involved personal interviewing did not make it unworthy or
undignified.

The matter was clinched by the treasurer's report showing an 125\%
increase of gross receipts by the corporation over the preceding year,
and payments of \$7,000 to psychologists representing the corporation
and their students. The corporation, which had been in the red for some
time, was climbing out. Dr. Achilles (who incidentally has been serving
without salary) and Dr. Link were re-elected as managing director and
secretary-treasurer respectively. Other names on the present list of
officers and directors are J. McKeen Cattell, E. L. Thorndike, L. M.
Terman, Walter Dill Scott, W. V. B. Bingham, A. T. Poffenberger, R. S.
Woodworth and Rensis Likert.

So that is that, as we used to say when the client laid down the law at
an advertising conference. It looks bad for my old friends in the
research departments of the advertising agencies. If the Psychological
Corporation, under its present efficient management, continues to
progress, this sweated academic scab labor is going to take the bread
out of the mouths of a lot of families I know in Bronxville, Great Neck
and elsewhere. Doubtless, too, the standards of advertising research
will be greatly improved, when the job is taken over by psychologists
instead of the more or less irresponsible apprentices in the agencies to
whom such work is ordinarily assigned.

In the old days before the war I remember that advertising research was
considered to be something of a joke. You knew the answer before you
started out. Your job was to get the documents. We, too, went out with
questionnaires, were chased down the street by irate Italian green
grocers, and got our toes caught in doors closed energetically by
unco-operative housewives. It really wasn't so very dignified, Dr.
Woodworth, but it had its humorous compensations and it kept one in the
open air. I recall a two-hundred-pound football player who on graduation
drifted into an advertising agency where I worked and was assigned to
research. It was the middle of July, and he had to interview some fifty
housewives residing somewhere in the Oranges. I forget what he had to
ask them. Did they use Gypso, maybe, and if not why not?

His name was---call him Mr. Retriever. Two days later, Retriever
stumbled back into the office in a state of moral and physical
exhaustion. Somebody was callous enough to ask him how he had been doing
and how he felt.

``I've lost twenty pounds,'' said Mr. Retriever. ``I feel like the hobo
who started cross the continent by freight. He got aboard the car next
the engine and the brakeman kicked him off. He grabbed the next car and
got aboard. The brakeman kicked him off, but he scrambled back into the
third car. This ritual continued until the train stopped at a way
station, when the hobo walked to the front of the train and got aboard
the first car. The brakeman spotted him and in exasperation demanded:
``Brother, where in hell are you going?'' ``I'm going to Kansas City,''
replied the hobo, ``if my tail holds out.''

The sacrifices of dignity demanded of an advertising researcher are in
fact extreme. I recall a baby-faced collegian who rang a doorbell
somewhere in the wilds of Bergen County. There appeared in the doorway a
comely middle-aged German woman who listened silently to his patter,
meanwhile scrutinizing him shrewdly. When he finished, she gave him a
ravishing smile and said: ``I know what you want. You want a piece of
apfelkuchen.'' The collegian blushed, searched his conscience and said:
``Yes.'' This particular anecdote has a Rabelaisian sequel which the
writer feels obliged to withhold, in deference to the feelings of the
Better Business Bureau. In a contribution to the Nov. 9, 1933, issue of
\emph{Printers' Ink}, Dr. Link states that ``during the last two years
we have interviewed almost 12,000 women in their homes, in more than
sixty cities and towns.'' One is sure that the anecdotal literature of
advertising research has been greatly enriched by these investigations.

It is possible, of course, that the Psychological Corporation,
representing as it does the idealism and public spirit of American
psychologists, is secretly engaged in boring from within the advertising
business; one notes the repeated references ta the scientific research
which these pot-boiling activities are designed to finance. Possibly the
corporation intends to take as a point of departure Veblen's description
of advertising as an enterprise in ``creative psychiatry,'' and, using
the data obtained by its commercially sponsored investigations,
institute studies designed to show just what the advertising business
has done to improve or debauch the mental, ethical and moral level of
the average American. An attitude of suspended judgment is therefore
indicated. The difficulty is that a study such as that above suggested
would require some framework of value judgment, which would be most
unscientific. And if, in spite of this objection, the corporation
elected to make such a study, to whom would it report its results,
asking again, ``How am I doing?''


% CHAPTER SIXTEEN
\chapter[16 \hspace*{1mm} THE MOVIES]{16 THE MOVIES}

\newthought{Although} not a part of the advertising business proper, the movie
industry maintains and is maintained by a huge and efficiently operated
advertising apparatus---the dozen or so popular movie magazines whose
combined circulation of over 3,000,000 ranks next in volume to that of
the women's magazines.

These magazines serve in effect as house organs for the \$42,000,000,000
movie industry which every week spreads its wares before 77,000,000
American movie-goers, including 28,000,000 minors. But like other mass
and class publications these movie magazines are also house organs for
their advertisers---chiefly manufacturers of cosmetics, drugs and
fashion goods. How this dual r\^ole is worked out and how the movie
magazines articulate into the general economic scheme of the movie
industry becomes at once apparent when we examine their promotion
literature. I quote from a looseleaf promotion booklet issued by
\emph{Photoplay Magazine}, the largest and most successful of the movie
magazines:

\begin{quote}
\emph{Photoplay} offers you a concentrated, compact audience of 600,ooo
predominantly younger women the New Wanters ... \emph{Photoplay} ... is
outstandingly tributary to the great sales-making, want-building
influence of the screen.
\end{quote}

We begin to glimpse what is perhaps the major r\^ole of the movie in our
society, and a little later, in a signed statement by the editor, Mr.
James R. Quirk, we find this r\^ole explicitly stated:

\begin{quote}
It became increasingly apparent to the publishers of \emph{Photoplay}
that the vast public who spent millions through motion picture box
offices was interested in more than the stories flashed upon the screen;
that they were absorbing something beyond the vicarious emotions and
adventures of the screen folk.

The millions of young women who attended motion pictures began to
realise that, closely observing the stars and leading women of the
screen, they could take lessons to enhance their own attractiveness and
personality. Hollywood became the beauty center of the world....

Following closely the new interests which the motion picture provoked in
the minds of the audience, and the desires of millions of women to
profit by their achievement of beauty, the magazine sent experts on
beauty and fashions and famous photographers to Hollywood and reported
to its readers every new phase of the development of feminine
attractiveness. These subjects today share in basic importance with the
news of Hollywood pictures and personalities.

That made \emph{Photoplay} outstanding as a medium for
advertisers.... Its readers are inspired by the editorial pages to buy
the goods shown in its advertising pages. The editorial and advertising
interests dovetail perfectly.

Its fashion and beauty editors, all of whom have had training in actual
merchandising, are recognized by the trades as experts. Such stores as
Marshall Field \& Company of Chicago use its fashion pages in their
selections and merchandising, and credit \emph{Photoplay} in their
newspaper advertising, recognizing the combined style promotion power of
the screen and the magazine. Thousands of beauty shops throughout the
country receive and display its announcements of new Hollywood coiffures
and new beauty methods of the most beautiful stars.
\end{quote}

One somehow gets the impression that Mr. Quirk knows what the motion
picture industry is all about and what it is for. This impression is
confirmed when we note that \emph{Photoplay} lists over 80 well-known
manufacturers of drugs, cosmetics and fashion goods among its 1931--32
advertisers. It is further confirmed by the following even more explicit
statement of the nature of the business, quoted from the same source:

\begin{quote}
When women go to the movies they go to see themselves not in the mirror
but in the ideal world of fancy. During that hour or two in the romantic
world of make-believe, potent influences are at work. New desires are
instilled, new wants implanted, new impulses to spend are aroused. These
impulses may be at the moment only vague longings, but sooner or later
they will crystallize into definite wants.

When the American woman sees her favorite screen actress and notes with
very keen interest every detail of her attire ... she is immersed in that
mood which makes her most receptive to the suggestion that she must have
these lovely things for her own ... and she will scheme and plan to have
for her own the charming frocks and appealing millinery, the smart
footwear, the seductive furs and wraps---all the tempting possessions
which the silver screen has so seductively exposed to her view....

\emph{The motion picture paves the way.} Photoplay \emph{carries on,
renewing the impulses caught on the screen. It gives your product's
address and telephone number.}
\end{quote}

The facts are as stated, and the argument is logical and convincing. It
is clinched on the next page by a skillful reference to what is without
doubt the major asset of this movie-advertising coalition, which is
Youth.

\begin{quote}
Last year two million, next year two million, in the next ten years
twenty million, young men and women will come of age.... They will want
necessities, pleasures, luxuries. And they will get them---because their
\emph{buying temperature} is high.... It will pay you handsomely to find
the best point of contact with these millions of new wanters. It will
pay you to lay your wares before them in the atmosphere of enthusiasm
and romance in which the desire to own the good things of life is
engendered.... Photoplay's \emph{audience, 600,000 strong, is
predominantly with the younger women.}
\end{quote}

What is the nature of this admirable piece of promotion literature,
prepared under the direction of one of America's leading publisher's
consultants?

\enlargethispage{\baselineskip}

It is, quite evidently, by way of being applied sociology and
psychology. It is supplemented by tables and graphs showing the buying
power of \emph{Photoplay}'s readers, these being based on the research
of Daniel Starch, Ph.D., who operates a well-known and successful
commercial research bureau. Dr. Starch's figures seem startlingly high,
but there is really no good reason for supposing that his study was less
honest, less ``objective,'' than that of the group of sociologists,
psychologists and educators who conducted the Payne Fund study of the
motion picture with respect to its influence upon children and
adolescents. Dr. Starch was employed by the allied motion
picture-advertising business which has an axe to grind, and admits it.
The Payne Fund investigation was financed by a philanthropic foundation
and instigated by a middle-class reform organization, the Motion Picture
Research Council, which also has an axe to grind, a moral axe, if you
will. A little later we shall encounter another eminent sociologist and
psychologist operating in this arena, namely Mr. Will Hays, who also has
an axe to grind and more or less admits it, although in the nature of
the case Mr. Hays' operations require a lavish output of pragmatic
make-believe.

But first let us attempt to construct, on the foundations already laid,
a slow-motion picture of what this business is and how it works.

As in all other forms of advertising, the causal sequence traces back to
mass production as the most profitable technique of exploiting the ``art
and science'' of the motion picture. Mass production requires mass
distribution (including block booking and blind booking) and mass
advertising; also standardization of the product in terms of maximum
salability and a systematic ``production of customers by a production of
systematized illusions.'' The Payne Fund investigators discovered with
horror that between 75 and 80 per cent of current motion pictures deal
with crime, sex and love---obstinately refusing to merge the second two
categories.

Surely this is pretty much beside the point; an analysis of
Shakespeare's plays would probably show an even higher content of such
subject matter.

The \emph{Photoplay} promotion booklet, written by people who really
know something about the industry, hits the nail on the head in
emphasizing the standard content of romance, luxury and conspicuous
expenditure. This is not only the commodity of maximum salability, but
in the process of its manufacture and sale there emerges an important
by-product which is duly sold to advertisers by the movie magazines.

Why does the motion picture with a high content of ``romance,''
``beauty'' and conspicuous expenditure represent the standard movie
product of maximum salability? Because the dominant values of the
society are material and acquisitive. And because the masses of the
population, being economically debarred from the attainment of these
values in real life, love to enjoy them vicariously in the dream world
of the silver screen. The frustrations of real life are both alleviated
and sharpened by the pictures. As in the case of sex, the imaginative
release is only partially satisfying, and the female adolescent,
particularly, leaves the motion picture theatre scheming, planning ``to
have for her own ... all the tempting possessions which the silver screen
has so seductively exposed to her view.'' From this point
\emph{Photoplay} carries on, and renews the sweet torture in both its
editorial and advertising columns, so that the stenographer goes without
lunch to buy her favorite star's favorite face cream. The sales cycle is
now completed, and the following mentioned profit-makers have duly
participated: the producer, distributor and exhibitor of the motion
picture; the motion picture magazine; Dr. Starch, who helped to present
the merits of the motion picture magazine to the advertiser; the
advertising agency which got a 15 per cent commission on the cost of the
advertising space; the advertiser and all the distributive links ending
with the drug store that sold the stenographer the vanishing cream (net
manufacturing cost eight cents, retail price \$1.00).

But we are not through yet. The exploitative process as above outlined
runs counter to the residual Puritanism, both consumptive and sexual of
the American middle class, particularly the middle-class resident in
that section of America referred to in the shop talk of the industry as
``the Bible Belt.'' The movie industry is obliged, for honest commercial
reasons, to break down this Puritanism. But the Puritans feel obliged to
organize and effectuate their sales resistance, if only to protect their
children from the corruptive influence of the movie industry. They also
feel morally obliged to protect the children and adolescents of the
lower classes and prevent them from enjoying almost the only kind of
emotional release which their economic condition permits them.

So censorship movements spring up here, there and everywhere, usually
sponsored and financed by the church groups, women's clubs,
parent-teacher organizations, etc., through which the middle class
expresses its view of the morals, expenditure and conduct appropriate
for an eighteen-year-old proletarian typist. These movements provided
jobs and salaries chiefly for preachers without other ``calls'' and for
women's club leaders enjoying more eminence than income.

Naturally, the industry felt obliged to defend its vested interest in
the exploitation of the American masses, and specifically of the
American kiddy, sub-flapper and flapper. That made more jobs, and since
the industry was better organized and in a position to pay adequate
salaries to such genuinely gifted propagandists as Will Hays, the
industry invariably won. Mr. Hays makes use of a well-known principle of
applied sociology which is expressed in the formula: ``If you can't beat
'em, join 'em.'' With his characteristic evangelical enthusiasm, Deacon
Hays has managed in one way or another to ``join'' almost every
movie-reform movement which has appeared on the horizon during his long
tenure of office as President of the Motion Picture Producers and
Distributors of America, Inc., popularly known as the ``Hays office.''

The public relations machinery operated by the Hays office is in effect
a two-way system of diplomatic communication between the industry and
the various pressure groups which represent public opinion as applied to
the movies. Since Mr. Hays is employed by and responsible to the
industry, he is expected to see that these pressure groups interfere as
little as possible with the business as usual of the movies. But being a
man of talent, and a sociologist of parts, the good deacon does a lot
better than that. He strives always, and often with notable success, to
induce these reform groups to become propagandists for the Hays office
and salesmen of the Hollywood product, to the end that the Hays office,
far from being merely a defense against censorship, may become a
positive and useful sales promotion department for the industry as a
whole. With this in view he has built up three major instrumentalities:
(1) the National Board of Review, which clears and effectuates the
judgments of ten organized pre-viewing groups: The International
Federation of Catholic Alumnae, National Council of Jewish Women,
National Society of Daughters of the American Revolution, the Congress
of Parents and Teachers, National Society of New England Women, General
Federation of Women's Clubs, Women's University Club of Los Angeles, Boy
Scouts of America and Young Men's Christian Association. Note that these
are all middle-class organizations, chosen because it is in middle-class
pressure groups that censorship movements originate, although the bulk
of the industry's income is derived from the lower classes and lower
middle classes. In other words representatives of the ruling middle and
upper classes are invited to pass on what movies the masses are
permitted to see.

(2) The local Motion Picture Councils, Better Film Committees, etc.,
consisting usually of club women, church women and local parent-teacher
groups organized to deal with the 12,000 ``neighborhood theatre
situations'' into which Mr. Hays breaks down his field organization
problem. In 3,000 of these ``situations'' there is today a public group
of some kind working with the theatre manager, and the membership of
these groups is somewhere between 50,000 and 100,000.

(3) The Studio Relations Committee in Hollywood, which digests and
clears the data coming in from the field, determines broad lines of
production policy as it is affected by the organized opinion of these
groups, and enables each producer to learn from the mistakes of the
others.

Now watch what happens when this machinery goes into action. Some of
these pre-viewing groups pass some pictures; others pass other pictures.
In the end most of the pictures are likely to be passed by some one of
the groups. This permits Dr. Hays to announce in his annual report for
1932 that of 476 feature films reviewed by seven committees 413 (86.7\%)
were ``variously endorsed for family, adult and child entertainment ... by
one or more of these committees.'' There we have not merely censorship
reduced to innocuity, but a positive testimonial asset which the Hays
office duly capitalizes by spreading the glad news to his field
organization that ``unsophisticated films pay ... more than 8o per cent of
box-office champions of last year also endorsed in National Previewing
Groups selections.'' And the motion picture committee of the General
Federation of Women's Clubs sends out a statement of its program for the
year urging each local club committee to take as its slogan, ``Be Better
Film Buyers.''

But this isn't all. When the motion picture code hearings were held in
Washington a group of representative club women appeared to protest
against the evil of double features, which the producers also object to
for profit reasons. And when Henry James Forman's book, \emph{Our
Movie-Made Children}, appeared the \emph{Pennsylvania Clubwoman},
according to an article in the \emph{Christian Century}, attacked this
popularization of the Payne Fund studies and the Motion Picture Research
Council which instigated these studies.

So that a neutral layman, listening to the hue and clamor about the
movies, finds it a bit difficult to determine whether the Hays office
has joined the reformers or the reformers have joined the Hays office.
But the result is not in doubt. The industry has won every battle thus
far, including the battle of Washington at which the motion picture code
was signed. In this code the industry got practically everything it
asked for, including an undisturbed continuance of the blind booking and
block booking practices by which the big producers are enabled to ensure
a part of their market in advance of production. What did the reformers
get? They got President-Emeritus Abbott Lawrence Lowell, of Sacco and
Vanzetti fame, sitting on a committee with Eddie Cantor and Marie
Dressier to safeguard the morality of the movies and the interests of
the artists. This was supposed not to be funny, but Dr. Lowell couldn't
see it that way and resigned. Dr. Lowell is now president of the Motion
Picture Research Council, which instigated the Payne Fund studies of the
effects of the motion pictures upon children, and that was also a
serious matter.

Prior to the Payne Fund studies, the reform of the motion picture had
been almost the exclusive province of preachers, club women,
parent-teachers, Y. M. C. A. secretaries, Scout Masters, etc. Naturally
the sociologists, educators, psychologists and other academic savants
wanted in; there was a considerable overproduction of social scientists
during the late New Era, and the universities and colleges were not able
to absorb the surplus. Moreover, the Great Movie Argument, what with one
thing and another, and especially Will Hays, had become loud, raucous
and most unscientific. It was clearly up to the social scientists to
Establish the Facts.

The Facts, as determined by eighteen assorted sociologists,
psychologists and educators, are set forth in nine volumes published by
Macmillan, and are also summarized and popularized in a book by Henry
James Forman entitled \emph{Our Movie-Made Children}.\footnote{{[}Henry James Forman,
  \emph{\href{http://www.worldcat.org/oclc/504726021}{Our Movie Made
  Children}} (New York: Macmillan, 1935). On the Payne Fund studies, see
  Garth Jowett, Ian C. Jarvis, and Katherine H. Fuller,
  \emph{\href{http://www.worldcat.org/oclc/32390197}{Children and the
  Movies: Media Influence and the Payne Fund Controversy}} (New York:
  Cambridge University Press, 1996).{]}} It took four
years to dig up the Facts, which, however, turned out to be pretty much
what everybody knew all the time: that children who attend the movies
frequently are likely to be stupider than children who don't go to the
movies at all (this is also probably true of adults); that very young
children are frequently shocked and nervously injured by horror
pictures; that the movies not only reflect our changing sexual mores but
also \emph{affect} them---girls learn about men from John Gilbert and Clark Gable; boys learn
about women from Clara Bow and Greta Garbo. Life then proceeds to
imitate the art and pseudoart of the movies, in respect both to sex and
to other aspects of conduct. Other findings were that children do learn
from the movies and retain much of what they learn; that the movies
constitute in effect an independent, profit-motivated educational
apparatus rivalling and sometimes surpassing in influence the home and
the school; that the movies can be and are used as propaganda for and
against war, for and against different racial groups; that gangster
pictures, with or without moral endings, tend to teach gangsterism.
\clearpage
Although the investigators made much pother about the ``objective''
``scientific'' nature of this fact-finding study, they could scarcely
escape value judgments, and Mr. Forman frankly applies such judgments in
his popularization. They are middle-class value judgments, derived from
the conventional mores of the middle-class community, and applied to an
industry which is organized to serve not the classes, but the masses.
These value judgments crop out when Cecil De Mille's ineffable ``King of
Kings'' is cited as a ``good'' picture, and when Mr. Forman quotes the
testimony of high school and college youngsters, asked to describe what
effect the movies had on their lives. A college boy remarks sensibly
enough:

\begin{quote}
The technique of making love to a girl received considerable of my
attention ... and it was directly through the movies that I learned to
kiss a girl on her ears, neck and cheeks, as well as on the mouth.
\end{quote}

The implication is clear that such techniques are highly reprehensible,
whereas on purely objective grounds there would appear to be something
to be said for them.

But what the Payne Fund investigators didn't find is almost more
interesting than what they did find. For instance, they failed to remark
the r\^ole of the movie as commercial propaganda in promoting the
enterprise of the advertiser. The consistent class bias of the movies
also escaped attention although it is apparent enough both in the news
reels and in the feature pictures. During the 1932 Communist-led Hunger
March on Washington the newsreels were even more unfair than the press
in deriding and misrepresenting the marchers. And who ever saw an
American movie featuring as hero a successful strike leader?

As one of our three major instruments of social communication, the movie
is an instrument of rule. Naturally, in a business-ruled society, the
movie serves the propaganda requirements of business, both as to
commerce and politics. Why did the industry get what it wanted and the
reformers get nothing when the movie code was signed? Isn't it possible
that the administration felt that it needed the good-will of the
industry in order to stay in office?

Dr. W. W. Charters, director of the four-year study financed by the
Payne Fund, remarks in his introduction to Mr. Forman's volume: ``the
commercial movies present a critical and complicated situation in which
the whole-hearted and sincere co-operation of the producers with parents
and public is essential to discover how to use motion pictures to the
best advantage of children.''

One is tempted to ask ``What parents and what public?'' The
middle-class, more or less religious, more or less Puritan parents would
doubtless like a good deal less frank sex in the movies, more
``education'' and more ``wholesome'' romance of the \emph{Ladies' Home
Journal} variety. But the younger generation of the great cities might
be expected to assert, with some justice, that there is both more art
and more health in the sex movie at its worst than in the average
woman's magazine romance. There would probably be equally violent
disagreement concerning other varieties of social content. The radical
labor movement, if it were strong enough to have an effective voice in
the reform of the movies, would presumably demand that the producers
stop using news reels and feature pictures as anti-labor propaganda, and
even give them an occasional picture with a strike leader as hero. One
doubts that the middle-class reform groups would either make or support
such a demand.

The dilemma, which would have become apparent if, as originally planned,
a competent and sufficiently unorthodox economist had been included in
the group that made the Payne Fund study, is that the movie industry
represents Big Business operating in a cultural field, but for purely
commercial purposes. The industry will co-operate ``wholeheartedly and
sincerely'' with anybody and everybody for the good of the industry as
determined by box office receipts. Pressure groups, whether middle-class
or proletarian, which would like to see a different set of value
judgments, will in the end, one suspects, be obliged to shoot their own
movies and build their own audiences.

No mention has been made of the use of the movie for direct advertising
purposes. The ``sponsored'' movie---a more or less entertaining short
subject, advertising a commercial product or service and introduced into
a regular program---was tentatively tried out in 1929 and 1930. The idea
was to sell the advertiser a given run of his sponsored short in chain
theatres. The theatres ``owned'' their audiences, or thought they did,
and would have been glad to sell the ``fans'' at so much a head to the
advertisers. But the audiences proved restive and the idea was pretty
much abandoned. A certain modicum of two-timing is observable in the
current run of pictures, but it ordinarily takes the form of propaganda
rather than of advertising. The industry frequently needs to use the
paraphernalia of the army and the navy. It is therefore good business to
permit a percentage of army and navy propaganda in the pictures. As for
the use of the pictures and endorsements of movie stars in advertising,
that is merely a by-product of the industry and a part of its promotion
technique. Whether or not the public credits the sincerity of these
endorsements is unimportant; they sell goods and they advertise the
star.



% CHAPTER SEVENTEEN
\chapter[17 \hspace*{1mm} RULE BY RADIO]{17 RULE BY RADIO}


\newthought{Radio} broadcasting came into the world like a lost child born too soon
and bearing the birthmark of a world culture which may never be
achieved.

Her begetters, the physicists and engineers, didn't know what to make of
the creature. That she was wistful for a world not yet born did not
occur to them. Indeed her begetting was in a sense accidental. They had
been thinking of something else. And as for bringing her up, that was
scarcely their affair. Men of science are notoriously neglectful of
their technical progeny. Observing this neglect an American historian,
Vernon Parrington, was moved to remark that ``science has become the
drab and slut of industry.''

Radio had to belong to somebody. She couldn't belong to nobody. So one
day Business picked her up off the street and put her to work selling
gargles, and gadgets, toothpaste and stocks and bonds. What else could
have happened? Neither art nor education had the prestige or the
resources to command the services of this new instrument of
communication, even if they had had anything important to communicate,
which may be doubted. Government? But in America government was business
and business was government to a far greater degree than in any other
country. So that the development of the ``art and science of radio
broadcasting'' became in America a business enterprise, instead of a
government monopoly as in England and elsewhere in Europe.

About two years ago, Dr. Lee De Forest, one of the pioneers of
electronic science, and by general concession one of the begetters of
radio, encountered the lost child in his travels and was inexpressibly
shocked:

\begin{quote}
``Why should any one want to buy a radio or new tubes for an old set?''
declaimed the irate inventor, ``when nine-tenths of what one can hear is
the continual drivel of second-rate jazz, sickening crooning by
degenerate sax players, interrupted by blatant sales talk, meaningless
but maddening station announcements, impudent commands to buy or try,
actually imposed over a background of what might alone have been good
music? Get out into the sticks, away from your fine symphony orchestra
pickups, and listen for twenty-four hours to what eighty per cent of
American listeners have to endure! Then you'll learn what is wrong with
the radio industry. It isn't hard times. It is broadcasters'
greed---which is worse. The radio public simply isn't listening in.''
\end{quote}

One wonders why Dr. De Forest should have been so surprised to encounter
this Bedlam on the air. Surely he was familiar with its terrestrial
equivalent. At the moment, in fact he was engaged in fighting the Radio
Corporation of America in the courts.

The vulgarity and commercial irresponsibility of advertising-supported
broadcasting have been greatly complained about. Yet there is a sense in
which the defenders of the American system of broadcasting are right.
Radio is a new instrument of social\\ \noindent communication---that and nothing
more. In and of itself it contributed nothing qualitative to the
culture. It was right, perhaps, or at least inevitable that it should
communicate precisely the pseudoculture that we had evolved. Can any one
deny that it did just that? The culture, or pseudoculture, was
acquisitive, emulative, neurotic and disintegrating. Our radio culture
is acquisitive, emulative, neurotic and disintegrating. The ether has
become a great mirror in which the social and cultural anomalies of our
``ad-man's civilization'' are grotesquely magnified. The confusion of
voices out of the air merely echoes our terrestrial confusion.

This confusion becomes particularly apparent when attempts are made to
challenge exploitation of radio by business. In the van of such attacks
are the educators, marching under the banner of ``freedom'' and
``culture'' and invoking such obsolete political concepts as ``States'
Rights.'' Allied with the educators is the Fourth Estate. The appeal is
to ``public opinion,'' expressed and made effective through the
machinery of representative government in a political democracy where
one man's vote is as good as another's. But we have already had occasion
to examine the status of the Fourth Estate and of Education in our
civilization. The press is essentially an advertising business and as
such a part of the central acquisitive drive of the culture. Education
is a formal, traditional function which becomes increasingly peripheral,
decorative and sterile when it adheres to its ideals of disinterested
``objectivity'' and increasingly pragmatic and vocational when it
attempts to relate itself to the acquisitive realities of business as
usual. The press has a vested interest both in the purveying of news and
as a medium of advertising; commercial broadcasting chiselled into the
advertising income of the press and latterly began to compete in the
field of news purveying. Hence the interest of the press in
``reforming'' the radio was strictly competitive and pecuniary in
quality although, of course, the appeal to public opinion was not made
in those terms. It may fairly be alleged that the interest of the
educators was also, and not improperly, a job-holding and job-wanting
interest, although again the appeal to public opinion was not made in
those terms. As for the artists, the writers, poets, dramatists and
critics, who might claim a modicum of service from Radio---well, art is
scarcely an organized and independent estate in an acquisitive society.
The artists tend either to accept service as the cultural lieutenants of
business, to retreat into ivory towers or to become frank
revolutionaries claiming allegiance to a hypothetical future ``classless
culture'' and to the ``militant working class'' also more or less
hypothetical at the present stage of the social process.

The American system is quantitatively successful as judged by the rapid
extension of service---some kind of service---to about 15,000,000
American homes. Today the potential radio audience numbers over
60,000,000. In less than twelve years radio has become a cultural
indispensable and has introduced important new factors into the social
and political process.

The bill for this service is paid first by the set owners. Mr. H. O.
Davis of the Ventura Free Press estimates the annual amount of this
bill, covering the cost of power, new tubes, repairs and replacements of
radio sets, at \$300,000,000. The same authority estimates that the
maximum annual expenditures of all broadcasting stations and networks,
including the operation of enormously expensive advertising sales
departments, is not more than \$80,000,000 and that \$50,000,000 covers
the total expense for the actual production and transmission of all
programs.

The estimates are based on the technical and economic \emph{status quo}
of the ``art and science of radio'' as developed by business. Mr. Davis
undertook a reconnaissance study of this \emph{status quo}, which took
the form of an analysis of a typical day's output transmitted to the
listening public by 206 American broadcasting stations. The following is
quoted from his summarized findings:

\begin{quote}
The average number of interruptions for sales talk during a total of
2365 hours of broadcasting, sustaining programs included, was 5.28 per
hour per station.

The average number of interruptions for sales talks during 1195
program-hours sponsored by advertisers was 9.36 per hour. (Interruptions
for station announcements are not included in these figures.)

On 1195 hours of programs sponsored by advertisers the sales talks
consumed 174.7 hours, or 14.61 per cent of the total program time,
almost three times the maximum permitted on Canadian programs.

The number of ``spot ads,'' sales talks unaccompanied by entertainment
supplied by the advertiser, totaled 5092 and consumed 57 hours. Canada
prohibits the broadcasting of ``spot ads.''

Out of a total of 2365 broadcasting hours 789 hours, or 32.26 per cent,
were consumed by the playing of phonograph records. ``Electrical
transcriptions''---specially made records---consumed 30 hours or 4.82
per cent of the total broadcasting time.

A little more than 75 per cent of the entire number of hours was devoted
to music of some kind.

All musical programs consumed 1845 hours.

On the day of the survey the 206 stations under observation broadcast
9\( \frac{3}{4}\) hours of symphony-orchestra music, devoting .6 per
cent of the total music time to this type of entertainment. The same
number of hours was filled by the output of so-called haywire or
hill-billy orchestras.

Dance orchestras, on the other hand, filled 388 hours or 21 per cent of
the total music-time with jazz.

Other instrumental and vocal music of the popular variety, crooners
included, occupied 1219 hours, two-thirds of the total music-time.

From the quantitative standpoint vaudeville is next in importance to
music. It occupies almost half of the time not given over to music.
Vaudeville includes reviews, jinks, dramatic sketches, jamborees and
similar mixtures of entertainment.

The third largest portion of all broadcasting time is taken up by sales
talks of advertisers, which consume 8.5 per cent of all time on the air,
including both sponsored and sustaining time. In fact, commercial sales
talks consume as much of the broadcasting time as all news broadcasts,
all religious and political addresses and two-thirds of the lectures put
together....

On a typical day the average station will devote three-quarters of its
programs to some kind of musical presentation, but the highest class of
symphony-orchestra music will be heard during one-half of one per cent
of the total music time. And when music is on the air, four programs out
of ten will consist of the playing of phonograph records. More than five
times every hour the program will be interrupted for the delivery of a
sales talk lasting in excess of one minute. In addition there will also
be four breaks per hour in the program continuity for station
announcements, making a total of nine interruptions per hour.
\end{quote}

The reader, who is also probably a radio listener, will be able to dub
in the sounds that go with this statistical picture: the bedlamite
exhortations and ecstacies, the moronic coquetries and wise-cracks, the
degenerate jazz rhythms, punctuated by the ironic blats and squeals of a
demon from the outer void known as ``Static.'' An evening spent
twiddling the dials of a radio set is indeed a profoundly educational
experience for any student of the culture. America is too big to see
itself. But radio has enabled America to hear itself, and what we hear,
if closely attended to, supplies important clues to the present state of
the culture.

When we turn to the educators who have struggled for the uplift of radio
what we find is merely further proof of the cultural disintegration
which radio makes audible. It may be said without serious exaggeration
that the problem of the controlling and administering of radio
broadcasting is approximately coextensive with the problem of
controlling the modern world in the economic and cultural interests of
the people who inhabit it. Granted that the radio is socially and
culturally one of the most revolutionary additions to the pool of human
resources in all history---how does one go about integrating it with a
civilization which itself functions with increasing difficulty and
precariousness? Radio is potentially, even to a degree actually, an
instrument of world communication. But the interests of the world
population divide along racial, national and class lines. If these
terrestrial conflicts could be reconciled, presumably we should have
harmony on the air---even conceivably the communication of a world
culture. As it is, the great mirror of the other not only reflects the
conflicts of class and nation and race, but serves to expand the scale
and increase the intensity of these conflicts.

An adequate study of these conflicts, as they are reflected in the
current struggle for control of the microphone, would require a book in
itself. We have space here only for a brief description of what happens
when education and the arts encounter business-as-usual as represented
by the ``American system of broadcasting.''

The records of the Federal Radio Commission show that in May, 1927, when
the present radio law went into effect, there was a total of 94
educational institutions licensed to broadcast. By March, 1931, the
number had been reduced to 49. According to the National Committee on
Education by Radio, 23 educational broadcasting stations were forced to
close their doors between January 1 and August 1, 1930. At present, out
of a total of 400 units available to the United States, educational
stations occupy only 23.16 units, or one-sixteenth of the available
frequencies. In short, educators and educational institutions which
desire to make independent use of the radio as an educational
instrumentality are facing strangulation. They must either fight or
acquiesce in the present trend, which, if continued, will give the
commercial broadcasters complete control of the air---the educators
being invited to feed the Great Radio Audience such education as the
commercial stations consider worth broadcasting, at hours which do not
conflict with the vested interests of toothpastes and automobile tires
or with the careers of such established radio personalities as Amos 'n'
Andy, Phil Cook and Lady Esther.

The militant wing of the educators has chosen to fight and was organized
as the National Committee for Education by Radio. Represented on the
committee are the National Education Association, the National Council
of State Superintendents, the National Association of State
Universities, the Association of College and University Broadcasting
Stations, the National University Extension Association, the National
Catholic Educational Association, the American Council on Education, the
Jesuit Education Association and the Association of Land Grant Colleges
and Universities. Joy Elmer Morgan, editor of the \emph{Journal of the
National Education}, is chairman of this committee. Its work is financed
by the Payne Fund.

Let us turn now to the battalions of the opposition by which these
educational militants are confronted. On June 1, 1931, there were in the
United States 609 licensed stations divided in a ratio of one to sixteen
between the education and the commercial broadcasters. The strongest of
the latter group are affiliated in two great chains with the National
Broadcasting Company and the Columbia Broadcasting Company. N.B.C. is a
one-hundred per cent owned subsidiary of the Radio Corporation of
America, which manufactures radio equipment and pools the patents of
General Electric, Westinghouse and American Telephone and Telegraph.
Obviously the educational militants are facing a closely affiliated
group representing the dominant power and communications interests of
America. N.B.C. and Columbia represent big business, and what does big
business care for education and culture? But big business cares a great
deal, insist the commercial broadcasters, citing their cultural
sustaining programs and their repeated offers of free time on the air to
educators. There is, in fact, a group of educators who have accepted the
existing commercial set-up of broadcasting to the extent at least of
working with it and through it. They too are organized. The National
Advisory Council on Radio in Education is financed jointly by John D.
Rockefeller Jr. and the Carnegie Corporation. Its president is Dr.
Robert A. Millikan and its vice president is Dr. Livingston Farrand,
President of Cornell University.

Two years ago the educational militants were engaged in propaganda for
the Fess Bill, which would have assigned 15 per cent of the broadcast
band to educational broadcasting by educational stations. Latterly they
have turned more and more to the demand for congressional investigation
of radio with the hope that a congressional committee would recommend
government ownership and operation of radio facilities as in England and
more recently in Canada. The conservatives, as represented by the
National Council on Radio in Education, abstain entirely from political
propaganda and lobbying. The objectives of the council, as stated in its
constitution, emphasize fact-finding and fact-dissemination; it
undertakes to ``mobilize the best educational thought of the country to
devise, develop and sponsor suitable programs, to be brought into
fruitful contact with the most appropriate facilities in order that
eventually the council may be recognized as the mouthpiece of American
education in respect to educational broadcasting.'' Officially it still
suspends judgment on the question of private \emph{versus} public
ownership and operation of broadcasting facilities, remarking that, ``as
yet no one is prepared or competent to say whether or not this {[}the
announced educational program of the council{]} will eventually force
the council to discuss the mechanisms necessary for educational
broadcasting and whether their ownership should be in commercial hands,
in the hands of educational institutions, or in the hands of non-profit
co-operative federations, or perhaps in all.'' That statement was
written four years ago and the council is still busy ``finding the
facts'' by rigorously ``objective'' scientific procedures, meanwhile
sponsoring politically innocuous educational broadcasts on free time
contributed by the commercial chains.

In May, 1933, the National Council on Radio in Education held its annual
assembly. The Director of the Council, Mr. Levering Tyson, delivered a
report discussing various activities in broadcasting, research and
publication and urged the establishment of a National Radio Institute.
The writer participated in the discussion of this report and of the
prepared speeches which followed it, which are published in \emph{Radio
in Education}, 1933.\footnote{{[}\emph{Education by Radio} 3, no. 12 (1933).{]}} I was frankly puzzled by the attitude of the
educators as revealed at this conference.

In this view business, including the business of selling toothpastes,
laxatives, stocks and bonds, etc., by radio is assumed not to be
educative. The advertisers' sales talks (doctrinal memoranda in the
Veblenian terminology) and the jazz, vaudeville and other entertainment
by which they are made more palatable---all this is assumed not to be
educative. But obviously this business expresses the central acquisitive
drive of the culture. Obviously it influences the lives of the radio
listeners infinitely more than the relatively microscopic amount of
``education'' which the council had been able to put on the air---more
in all probability than the total output of American class rooms and
lecture platforms. Yet, by definition, it is not ``education,'' which is
conceived of as a meliorative something added to a secular process which
may be profoundly diseducational in that it contradicts and opposes at
practically every point the attitudes and ideals of the educator.

In arguing for a more realistic and more vital conception of the
educational function the writer pointed out that the end result of
American commercial broadcasting, as we have it, is demonstrably
diseducational; that radio advertisers are not interested in educating
the great radio audience in any true sense. What really happens is that
the advertisers are interested solely in promoting the sale of products
and services. Hence they tend to exploit the cultural inadequacies of
the radio audience and its moral, ethical and psychological
helplessness.

At this meeting, Mr. Henry Adams Bellows, LL.D., vice president of the
Columbia Broadcasting Company, made the usual formal offer of free time
on the air to the assembled educators. At the moment it happened that a
group of Communist ``fellow-travelers,'' organized as the League of
Professional Groups, was conducting a series of public lectures under
the general title ``Culture and Capitalism.'' The services of this group, which
included some well-known teachers and writers, were offered without
charge to Mr. Bellows but, as might have been expected, these radicals
clamored in vain for ``the freedom of the air.''

\enlargethispage{\baselineskip}

The issue of censorship was again raised at this meeting after Mr.
Hector Charlesworth, chairman of the Canadian Broadcasting Commission,
had declared that Communists and communist sympathizers were permitted
on the air in Canada. The position of the American commercial
broadcasters, as stated repeatedly by Mr. Bellows and others, is that
the American system provides more effective freedom for minority groups
than the system of government ownership as operated in England and in a
more modified form in Canada. The contention, of course, finds little
support in the experience of Communists and others who recurrently make
application in vain to the educational directors of the major chains.

It is difficult to write about the problem of radio censorship since all
our eighteenth century concepts of ``freedom'' are quite evidently made
obsolete by the technical nature of the instrumentality. Some form of
censorship and some form of international control is necessary. The
domestic problem is simplified under a political dictatorship. Both
Mussolini and Hitler promptly seized complete control of radio upon
assuming power and used it to consolidate and extend their rule. At the
moment Hitler's use of radio, which knows no political boundaries, is
perhaps his strongest weapon in his struggle to bring Austria under the
Nazi hegemony. It is safe to predict that in the next great war, radio
will constitute a major offensive weapon, second only in effectiveness
to the airplane.

Meanwhile, in America, the confusion brought about by our various and
sundry forms of censorship, both overt and concealed, is almost
indescribable. Miss Lillian Hurwitz, in a study of radio censorship
prepared for the American Civil Liberties Union, has no difficulty in
showing that despite the prohibition of censorship embodied in our
present radio law, The Federal Radio Commission ``has so construed the
standard of public interest, convenience and necessity as to enable it
to exercise an indirect censorship over station programs.''\footnote{{[}Lillian Hurwitz, \emph{Radio Censorship} (New York: American Civil
  Liberties Union, 1932).{]}} The very
assignment and withdrawal of radio licenses by the commission involves
an indirect censorship.

Meanwhile, as Miss Hurwitz abundantly proves, the stations themselves
are obliged to operate a systematic censorship, if only to protect
themselves against libel suits. They go much further than that, of
course. They not only impose their own conception of the ``public
interest, convenience and necessity'' but their own standards of taste,
morals and political orthodoxy. They protect their own source of revenue
by forbidding radio lecturers to attack radio advertising. When Mr. F.
J. Schlink, director of Consumers' Research, addressed the American
Academy of Political and Social Science on the subject of the New Deal
as it affects the consumer he was cut off the air by the Columbia
Broadcasting Company. Only after the issue was publicly posed by the
resulting newspaper publicity, was he permitted a week later to make the
same speech over Columbia facilities.

What will emerge from this welter of technical and commercial
necessities and political make-believe is quite impossible to predict.
Proposals to unify all communications services under a single government
control are now before Congress with the President's endorsement. A
non-partisan investigation of the broadcasting system has been
repeatedly urged and something of the sort is probably imminent.
Meanwhile, however, it should be pointed out that a tightened control of
the American Telegraph and Telephone Company would perhaps put the
government in a position to audit the wire charges which constitute a
heavy proportion of the overhead of the broadcasting chains. It has been
widely asserted that these charges are excessive; that both the
technical and economic problems of broadcasting could be solved by a
combination of ``wire and wax.'' By ``wax'' is meant wax records which
have been so perfected that an electrical transcription is now
practically indistinguishable from an original studio broadcast. By
``wire'' is meant wire chain hookups, the present cost of which is at
present almost prohibitive except for the two major chains. Then also
there is an assortment of more or less known technical potentialities,
such as wired radio, short wave and micro-wave broadcasting and
television, although the latter, according to competent technicians, is
at present to be classified as a stock-market development rather than an
electronic development. Taken together these various potentialities make
impossible any clear anticipation of what is likely to happen. With this
exception however: the trend of both technical and economic developments
point to the need of centralized control. This will be particularly true
if the Roosevelt Administration is forced, by the failure of the NRA to
increase buying power, to go left in the direction of a functional
reorganization of distribution.\footnote{{[}NRA refers to the National Recovery Administration.{]}} As we shall see later, when we come
to discuss the NRA program with respect to advertising, this cannot be
accomplished without a huge deflation of the advertising business,
affecting both the press and the commercial broadcasters.

A significant factor in the situation is, of course, Mr. Roosevelt's
immensely skillful and successful use of radio in building public
support for his administration. On the whole, it would seem only a
matter of time when Mr. Roosevelt, or whoever succeeds him, will be
obliged to say to radio broadcasting, ``You're mine! I need you to help
me rule!'' A faint intimation of this rather probable development
appears in the speech of Federal Radio Commissioner Harold A. LaFount at
the 1933 Assembly of the National Council on Radio in Education already
referred to. Commissioner LaFount said:

\begin{quote}
Educational programs could, and I believe in the near future will, be
broadcast by the Government itself over a few powerful short-wave
stations and rebroadcast by existing stations. This would not interfere
with local educational programs, and would provide all broadcasters with
the finest possible sustaining programs. The whole nation would be
taught by one teacher instead of hundreds, and would be thinking
together on one subject of national importance. Personally I believe
such a plan would be more effective than a standing army.
\end{quote}

The commissioner, who in view of his record, can scarcely be accused of
being unfriendly to the commercial broadcasters, was probably innocent
of dictatorial ideas. Yet his language is, to say the least, suggestive.

\emph{A more detailed discussion of the problem of radio is contained in
the writer's pamphlet ``Order on the Air!'' published by the John Day
Company.}\footnote{{[}James Rorty, \emph{Order on the Air!} (New York: John Day Company,
  1934){]}}



% CHAPTER EIGHTEEN
\chapter[18 \hspace*{1mm} RELIGION AND THE AD-MAN]{18 RELIGION AND THE AD-MAN}

\newthought{Weeks} before real beer came back, the beer gardens sprang into bloom
along Fourteenth Street. They are cheap. Fifteen cents buys a roast beef
sandwich, a portion of beans, a portion of potatoes and a slop of thin
gravy. You sit at an enamel table, look and listen. Imitation tile.
Imitation Alps. Imitation Bavarian atmosphere. Imitation beer. Three
people sit at the next table: an imitation pimp, an imitation stage
mother and an imitation burlesque show manager. Maybe the burlesque show
manager is real. He is gray-haired, red-faced, thickset and voluble. He
declaims:

``I'm a faker. God in his blue canopy above---that's out of\\ \noindent Shakespeare---God knows I'm a faker. When the priest baptized me, he
shook the holy water on my head (snap, snap) and said: `Taker, faker,
faker!'\thinspace''

\enlargethispage{\baselineskip}

I saw that. I heard that. If I had sat there long enough I am confident
I could have seen and heard anything. If one wishes to discover America,
all one has to do is to forget all the solemn and reasonable things that
solemn and reasonable people have spoken and written, and then go
listening and pondering into cheap restaurants, movie palaces, radio
studios, pulp magazine offices, police stations, five- and ten-cent
stores, advertising agencies. Out of this atomic, pulverized life, the
anarchic voices rise. They are shameless, these voices, and truthful,
and wise with a kind of bleak factual wisdom. Each atom speaks for
itself, to comfort itself, to assert itself against the overwhelming
nothingness of all the other atoms: each atom sending out an
infinitesimal ray of force, searching for some infinite reason, and
protesting obstinately against some infinite betrayal.

Fake. Baloney. Bunk. Apple sauce. Bull. There are over a hundred slang
synonyms for the idea which these words express, most of them coined
within the last two decades. No other idea has called forth such lavish
folk invention, and this can mean only one thing. It is the
pseudoculture's bleak judgment upon itself. It is possible for an
inhuman society to pulverize humanity, but the human essence is
indestructible. It is meek, or it is bitter; it remains human, truthful
and essentially moral, even religious.

What is religion, if it is not the framework of instinctively felt
values of truth and beauty and honor by which the race lives---if it is
to live? Reverse these thin worn coins of the folk argot---bunk,
baloney, etc.---and you find the true currency of the human exchange.
Honoring truth, the burlesque comedian pauses in his exit, shakes his
rear and says: ``Horsefeathers!''

But what we are concerned with here is not the deep human core of the
religious spirit, but the make-believe against which these atomic voices
are crying out: the fake religion, the moral, ethical and spiritual
make-believe of the acquisitive society, of the ad-man's pseudoculture.
If the inquiry were to be in any degree systematic and exhaustive, it
would lead us far back in time, back to the medieval synthesis of church
and state and its breakup by those Knights Templar of the rising trading
class, John Calvin and Martin Luther.

There are plenty of able and informed advertising men, and some of them
know this. Yesterday I was in the research department of a large agency
gathering certain statistical data. A former associate paused, greeted
me and we fell into conversation. Knowing me, he guessed what I was
doing---in fact I had never at any time tried to conceal anything---and,
helpfully, he offered his own explanations. He blamed Martin Luther. For
the long sequence of cultural disintegration, climaxed in our time by
the paradox of mass production and mass starvation and by the
development of the advertising agency as a mass producer of fakery,
human stultification and confusion, he blamed Martin Luther.

This man started life as a traveling salesman. He never went to college,
so that his mind remained fresh and avid, if cynical. And he had known
great charlatans in his time---notably Elbert Hubbard. He understood
them very well, and, being of a speculative turn, he had checked up on
their origins. He blamed Martin Luther. He was greatly interested when I
told him that the famous German scholar, Max Weber, author of \emph{The
Protestant Ethic}, also blamed Martin Luther a little, but John Calvin a
great deal more.\footnote{{[}Max Weber, \emph{\href{http://www.worldcat.org/oclc/535567}{The
  Protestant Ethic and the Spirit of Capitalism}} (New York: Charles
  Scribner's Sons, 1930 {[}1905{]}).{]}}

My friend had only a few minutes for gossip, however. He had to get back
to his desk and read proof on a new toothpaste campaign in which, by a
trick of pragmatic self-hypnosis, he had come to believe fervently. When
he had finished he would placidly stroll to the station, buy a paper,
and solve a cross-word puzzle en route to White Plains and his
comfortable and charming suburban family.

While somewhat exceptional, this man is far from being a unique figure
in the business. To those atomic voices heard above the clatter of
dishes in the Fourteenth Street beer gardens, we must add the voices of
the speakeasy philosophers of the Grand Central district---advertising
men, many of them, college men and more or less self-conscious fakers.
God in his blue canopy above knows they're fakers, but it is perhaps
somewhat to their credit that they know it too.

\vspace{2mm}

\begin{center}
\LARGE{2}
\end{center}

\noindent In discussing religion and the ad-man we are not concerned with the
sales publicity of the churches. There are plenty of texts on the
subject. What concerns us is the extent to which the culture of our
acquisitive society, as represented and publicized by the ad-man, has
become a rival of the Christian culture, represented by the Protestant
and Catholic Churches of the United States.

Since it is our purpose to compare these two cultures, it may be useful
to note what social scientists think culture and religion are. Culture
may be defined as the total social environment into which the individual
is born; religion is a behavior pattern which seeks to dominate the
culture. As sociological phenomena, religion, nationalism and
radicalism, although dissimilar in many respects, are categorically the
same. The sociologist would note the similarities between religions,
nationalism and radicalism, by calling them all behavior patterns. The
layman would call them religions. The name is not important. What is
important is the fact that they have common characteristics.

Each of these religions has an inclusive pattern for human life and
society. Each of them would prefer to be dominant and to exclude other
behavior patterns from the scene. Witness Russia and the Christian
Churches, or Nazi Germany and the Socialist and Communist Parties. As a
practical matter behavior patterns do succeed in living side by side,
but though the competition may not be overt, it is present. Every
behavior pattern has to be sold, more or less, continuously, to the
public. This is true, as the anthropologist, Malinowski, has pointed
out, even among primitive peoples. He says: ``The reign of custom in a
savage society is a complex and variegated matter just as it is in a
more civilized society. Some customs are very lightly broken; others are
regarded as mandatory.'' The more effective techniques used in selling
the public a behavior pattern may be considered techniques of rule.
Religious rituals belong in this category; so do the publicity engines
of Mussolini, and of Hitler. No proper perspective can be gained in
relation to such behavior patterns as religion, nationalism and
radicalism, unless one realizes that they are highly important in
relation to group survival. As Bagehot has said: ``Any polity is more
efficient than none.'' But the more shrewd and complete the polity, the
more efficient an instrument it is in the struggle for survival.
\clearpage
There are certain interesting parallelisms between the techniques of
persuasion and admonition used in religious rituals and those used in
contemporary advertising. Jane Harrison, the distinguished student of
Greek religions, notes that ritual in its beginnings has two elements:
the \emph{dromenon}, something which is done, and the \emph{legomenon},
something which is said.\footnote{{[}Jane Harrison,
  \emph{\href{http://www.worldcat.org/oclc/186690082}{Themis: A Study of
  the Social Origins of Greek Religion}} (Cambridge: Cambridge
  University Press, 1912).{]}} In the beginning, the words of the ritual,
according, to Miss Harrison, may have consisted of ``no more than the
excited repetition of one syllable.'' The action of the ritual is
something that is ``re-done, commemorative, or predone, anticipatory,
and both elements seem to go to its religiousness.'' The points at which
the techniques of religious ritual and advertising correspond are the
following: In both instances, there is repetition. In both instances the
symbols used in the ritual, or the ad, have the same meaning to the
audience. A symbol, which always has the same meaning, is called by
Durkheim, ``a collective representation.'' A number of social scientists
have pointed out that the Utopias of the radicals become comprehensible
if one realizes that they serve as collective representations. In
advertising, the name of the product, the slogan, the packaging and the
trade-mark, are obviously used as collective representations.

The net result of religious ritual is to leave the participants in a
religious ceremony more restless than soothed, simmering gently, or
boiling violently as the case may be, in an impressionable, emotional
state, which cannot find complete release in immediate action. (Note the
ritualistic function of the movies already described as a want-building
adjunct of the advertising business.) While the audience is in this
impressionable state, the minister or priest makes strong persuasive or
admonitory suggestions in regard to the action which the individual
should take in the future. In advertising, the admonitory or persuasive
voices of the priesthood are also present.

The close analogy between the sales publicity methods of the Christian
Church and those of the modern Church of Advertising was noted in 1923
by Thorstein Veblen, who missed little, if any, of the comedy of the
American scene. Veblen's long foot-note (p. 319, \emph{Absentee
Ownership}) should be read in its entirety in this connection. It is
particularly interesting as showing the rapid movement of forces during
the intervening decade.

\begin{quote}
The Propagation of the Faith is quite the largest, oldest, most
magnificent, most unabashed, and most lucrative enterprise in
sales-publicity in all Christendom. Much is to be learned from it as
regards media and suitable methods of approach, as well as due
perseverance, tact, and effrontery. By contrast, the many secular
adventures in salesmanship are no better than upstarts, raw recruits,
late and slender capitalizations out of the ample fund of human
credulity. It is only quite recently, and even yet only with a dawning
realization of what may be achieved by consummate effrontery in the long
run, that these others are beginning to take on anything like the same
air of stately benevolence and menacing solemnity. No pronouncement on
rubber-heels, soap-powders, lip-sticks, or yeast-cakes, not even
Sapphira Buncombe's Vegetative Compound, are yet able to ignore material
facts with the same magisterial detachment, and none has yet commanded
the same unreasoning assent or acclamation. None other has achieved that
pitch of unabated assurance which has enabled the publicity-agents of
the Faith to debar human reason from scrutinizing their pronouncements.
These others are doing well enough, do {[}\emph{sic}{]} doubt; perhaps
as well as might reasonably be expected under the circumstances, but
they are a feeble thing in comparison. ``Saul has slain his thousands,''
perhaps, ``but David has slain his tens of thousands.''\footnote{{[}Thorstein
  Veblen,~\emph{\href{http://www.worldcat.org/oclc/752183}{Absentee
  Ownership and Business Enterprise in Recent Times: The Case of
  America}}~(New York: B. W. Huebsch, 1923), 319--20.{]}}
\end{quote}

Within a year after this footnote was written, Mr. Bruce Barton
published \emph{The Man Nobody Knows}, in which the life and works of
the Saviour are assimilated into the body of the ad-man's doctrine, and
in which the very physical lineaments of the traditional Christ begin to
take on a family resemblance to those of the modern ad-man, so
excellently typified by Mr. Barton himself.\footnote{{[}Bruce Barton,
  \emph{\href{http://www.worldcat.org/oclc/70421692}{The Man Nobody
  Knows}} (New York: Grosset \& Dunlap, 1924).{]}} The discussion of this
brilliant job of rationalization must be reserved for a later chapter.
At this point it is sufficient to observe that today Veblen's ironic
patronage of the emerging priesthood of advertising sounds astonishingly
inept and dated. For it may well be contended that today the Propagation
of the Faith is relatively nowhere, while the religion of the ad-man is
everywhere dominant both as to prestige and in the matter of
administrative control. Granted that both religions are decadent, since
the underlying exploitative system which both support is itself
disintegrating by reason of its internal contradictions; none the less,
the ad-man's religion is today the prevailing American religion, and the
true heretic must therefore concentrate upon this modern aspect of
priestcraft. The ancient Propagation of the Faith continues, of course,
sometimes in more or less collusive alliance with the Church of
Advertising, sometimes in jealous and recalcitrant opposition. We can
give little space to the quarrels and intrigues of these competing
courtiers at the High Court of Business. Clearly the present favorite is
advertising, and we turn now to a brief resum\'e of the historic process
by which the priesthood of ballyhoo attained this high estate.


\vspace{2mm}

\begin{center}
\LARGE{3}
\end{center}

\noindent Starting, as any discussion of the economic and ideological evolution of
modern industrial capitalism must start, with the breakup of the
medieval church-state synthesis, we note that the Christian feudalism of
the Middle Ages did not live by buying and selling. As John Strachey
puts it in \emph{The Coming Struggle for Power}, ``what Western man
accomplished by some four hundred years of struggle, between the
fifteenth and the nineteenth centuries, was the establishment of the
free market.''\footnote{{[}John Strachey,
  \emph{\href{http://www.worldcat.org/oclc/473245572}{The Coming
  Struggle for Power}} (London: Victor Gollancz, 1932).{]}} The development of monopoly capitalism in the modern
period qualified this ``freedom'' of course; it also intensified the
fundamental contradictions of capitalism, and sharpened the ethical
dilemma which is concisely stated by the conservative philosopher, James
Hayden Tufts, in his \emph{American Social Morality}:

\begin{quote}
The impersonal corporation formed for profit represents in clearest
degree this separation of the modern conduct of commerce and industry
from all control by religious authority and by the moral standards and
restraints grounded in the older professedly personal relations of man
to man in kinship, neighborhood or civic community.... To turn over all
standards to the market was to lay a foundation for future conflicts
unless the market should provide some substitute for the older standards
when man dealt with his fellow and faced the consequences of his
dealing.\footnote{{[}James Hayden Tufts,
  \emph{\href{http://www.worldcat.org/oclc/570359384}{America's Social
  Morality: Dilemmas of the Changing Mores}} (New York: Henry Holt and
  Co., 1933), 125.{]}}
\end{quote}

The market did provide such a substitute, of course---a fake substitute.
It provided the religion of advertising and developed the forms and
controls of the ad-man's pseudoculture.

It is this utilitarian fakery with which we are here concerned, rather
than with the economic and political conquests of the trading class. We
are concerned with the ideological and religious rationalizations by
which these conquests were both implemented and justified. My former
advertising colleague who blamed this long history of serio-comic
rationalization on Martin Luther would seem to be somewhat in error,
just as Max Weber probably overemphasizes the r\^ole of the Protestant
Ethic, the Calvinistic doctrine of ``justification by works.''

In Weber's view the Calvinistic doctrine of worldly success in a
``calling'' as a means of winning divine favor constituted a necessary
theological counterpart of capitalism; without such reinforcement of the
normal lust for gain, he argues, the extraordinary conquests of
capitalism in England and in America would have been impossible.
Calvinism reconciled piety and money-making; in fact the pursuit of
riches, which in the medieval church ethic had been feared as the enemy
of religion, was now welcomed as its ally. It is important to note, as
does Tawney in his introduction to Weber's great essay, that the habits
and institutions in which this philosophy found expression survived long
after the creed which was their parent had practically expired. So that,
quoting Tawney, ``if capitalism begins as the practical idealism of the
aspiring bourgeoisie, it ends ... as an orgy of materialism.''\footnote{{[}R. H. Tawney, Foreword, in Max
  Weber,~\emph{\href{http://www.worldcat.org/oclc/535567}{The Protestant
  Ethic and the Spirit of Capitalism}}~(New York: Charles Scribner's
  Sons, 1930 {[}1905{]}), 3.{]}}

An orgy is an irrational affair. To the writer, the most interesting and
suggestive aspect of Weber's interpretation, as applied to the
contemporary phenomena of the ad-man's pseudoculture, is this divorcing
of the acquisitive drive from any control by hedonistic rationality. The
pursuit of wealth, for the Calvinistic entrepreneur, was not merely an
advantage, but a duty. And this sense of duty persisted long after the
Calvinistic sanctions had ceased to be operative. Moneymaking for
money-making's-sake, like art-for-art's-sake, supplied its own
sanctions. Both are self-contained disciplines, fields for the display
of an irrational and sterile virtuosity. Weber, in the concluding pages
of his essay, sets forth this consummation with moving eloquence:

\begin{quote}
In the field of its highest development, in the United States, the
pursuit of wealth, stripped of its religious and ethical meaning, tends
to become associated with purely mundane passions, which often give the
character of sport. (The advertising ``game.'' J. R.)

No one knows who will live in this cage in the future, or whether at the
end of this tremendous development entirely new prophets will arise, or
there will be a great rebirth of old ideas and ideals, or, if neither,
mechanized petrifaction, embellished with a sort of convulsive
self-importance. For of the last stage of this cultural development, it
might well be truly said: ``Specialists without spirit, sensualists
without heart; this nullity imagines that it has attained a level of
civilization never before achieved.''\footnote{[Weber,
  \emph{\href{https://archive.org/details/protestantethics00webe/page/182/mode/2up/search/nullity}{Protestant
  Ethic}}, 182.]}
\end{quote}

But note that this was written in 1905. What Weber saw with horror was
not ``the last stage,'' but the next-to-the-last stage---perhaps not
even that. The cage was kept spinning, not merely by its accumulated
momentum, but by the organized application, on a tremendous scale, of
the great force of emulation. Ten years before Max Weber wrote the
paragraph quoted, Thorstein Veblen had written \emph{The Theory of the
Leisure Class}, which gave currency to his fertile concepts of
``vicarious expenditure,'' ``conspicuous waste,'' etc.\footnote{{[}Veblen, \emph{\href{http://www.worldcat.org/oclc/222222126}{The
  Theory of the Leisure Class: An Economic Study in the Evolution of
  Institutions}} (New York: Macmillan, 1899)]} These
concepts, all revolving about the central motivation of emulation, are
the stock-in-trade of the modern advertising copy writer.

New prophets did arise in America---Elbert Hubbard for one, Bruce Barton
for another. America entered upon the ``surplus economy'' phase of
industrial capitalism, and the appropriate religion for this period,
which was interrupted, but also accelerated by the war, was the religion
of advertising, which did not reach full maturity until after the war.
The motion picture industry came along as an important adjunct of the
emulative promotion machinery, used as such both at home, and as an
``ideological export,'' to further the conquests of American imperialism
in ``backward'' countries. Peering out of the vistas ahead were radio
and television.

Seeing all this, Theodore Dreiser seized upon the great theme of
emulation---keeping up with the Joneses---and wrote \emph{The American
Tragedy}. And Carl Sandburg wrote, almost as a kind of sad ironic parody
of Weber: ``This is the greatest city of the greatest country that ever,
ever was.'' And the cage spun faster than ever. And Robert Frost wrote
\emph{West Running Brook}, in which he symbolizes western culture as a
stream disappearing in the barren soil of the American acquisitive
culture. And Robinson Jeffers wrote:

\begin{quote}
\justlastragged
Man, introverted man, having crossed\\
In passage and but a little with the nature of things this latter\\
century

Has begot giants; but being taken up\\
Like a maniac with self love and inward conflicts cannot manage his\\
hybrids.

Being used to deal with edgeless dreams,\\
Now he's bred knives on nature turns them also inward; they have\\
thirty points, though.

His mind forebodes his own destruction;\\
Act{\ae}on who saw the goddess naked among the leaves and his hounds\\
tore him.

A little knowledge, a pebble from the shingle,\\
A drop from the oceans; who would have dreamed this infinitely\\
little too much?
\end{quote}

When he wrote this, as a kind of an advance obituary of industrial
capitalism, Jeffers was an unknown recluse on the coast of California,
and the book in which it appeared was printed at his own expense. But
that same year the presses rolled out the four millionth copy of Elbert
Hubbard's \emph{Message to Garcia}, in which the big business cracks the
whip over the modern office wage slave.\footnote{{[}Elbert Hubbard, \emph{\href{http://www.worldcat.org/oclc/374531}{A
  Message to Garcia}} (East Aurora, NY: Roycrofters, 1903).{]}}

The cage spun faster still. On an August midnight in Union Square, New
York, a banner was flung out of the \emph{Freiheit} office reading
``Vanzetti Murdered!'' and, in the words of the New York \emph{World}'s
reporter:

\begin{quote}
The crowd responded with a giant sob. Women fainted in fifteen or twenty
places. Others too, overcome, dropped to the curbs and buried their
heads in their hands. Men leaned on one another's shoulders and wept.
There was a sudden movement in the street to the east of the Square. Men
began running around aimlessly, tearing at their clothes, and dropping
their straw hats, and women ripped their dresses in anguish.
\end{quote}

Thus the State of Massachusetts was killing the God in man. But Bruce
Barton still lived, and, having written \emph{The Man Nobody Knows},
went on to write \emph{The Book Nobody Knows}, and \emph{On the Up and
Up}. 



% CHAPTER NINETEEN
\chapter[19 \hspace*{1mm} EVOLUTION OF THE AMERICAN HERO]{19 EVOLUTION OF THE AMERICAN HERO}

\newthought{The} emergence of organized and incorporated salesmanship as the
characteristic phenomenon of the American society, the transmigration of
the soul of the Fourth Estate into the material body of the advertising
business---these developments can be viewed as logical sequences in the
evolution of industrial capitalism; they can also be studied as the end
products of a social philosophy. In this chapter we shall attempt to
outline the ideological evolution as it appears in the life and works of
significant American personalities. Benjamin Franklin, Jay Cooke, P. T.
Barnum, Henry Ward Beecher, Elbert Hubbard, Bruce Barton: what these men
thought and did and said was doubtless determined largely by the
economic environment in which they rose to power and influence. But
their attitudes, acts and utterances served to rationalize and thereby
to promote the material evolution, in the study of which the economist
specializes. What we look for, in the evidence of these lives, is the
religion of salesmanship which became more and more, after the turn of
the century, the religion of advertising. What we find is a kind of
sequence of crowd heroes, each modeling himself more or less on the ones
preceding. They are middle-class heroes, all of them, and the crown and
glory of the towering structure of rationalization which they erected is
the identification of the Christ mission with the mission of the
middle-class salesman and advertising man, which was accomplished by Mr.
Barton in \emph{The Man Nobody Knows}.\footnote{{[}Bruce Barton,
  \emph{\href{http://www.worldcat.org/oclc/70421692}{The Man Nobody
  Knows}} (New York: Grosset \& Dunlap, 1924).{]}}

Even today the masthead of the \emph{Saturday Evening Post} bears the
proud statement ``Founded by Benjamin Franklin.'' The statement is true
in spirit if not in fact. The \emph{Saturday Evening Post} is the most
influential advertising medium in America---in the world for that
matter. And the social and political philosophy of its publisher derives
clearly from the sly wisdom of that ineffable parvenu, that Yankee
all-right-nick of genius who signed himself ``Poor Richard.'' Franklin
serves as a point of departure because he was a business-minded
pragmatist. He was not a Babbitt and it is impossible to conceive of
Franklin, a man of genius, playing the r\^ole of a Hubbard or a Barton a
century later. But on the other hand it seems fair to credit Franklin
with laying the ground-work of the American acquisitive ethic.

\subsection{Benjamin Franklin}

``Remember, that time is money ... Remember, that credit is money. If a
man lets money be on my hands after it is due, he gives me the interest,
or so much as I can make of it during that time. Remember, that money is
of the prolific, generating nature. Money can beget money, and its
offspring can beget more, and so on. Remember this saying, '\emph{the
good paymaster is lord of another man's purse.}' He that is known to pay
punctually and exactly to the time he promises may at any time and on
any occasion raise all the money his friends can spare ... The most
trifling actions that affect a man's credit are to be regarded. The
sound of your hammer at five in the morning or eight at night, heard by
a creditor, makes him easy six months longer; but if he sees you at a
billiard table or hears your voice at a tavern, when you should be at
work, he sends for his money the next day.''

Remember, remember. Remember, in the matter of sex, its utilitarian
aspect; sexualize to promote health or for the sober procreation of
children; ``do not marry for money, but marry where money is.'' If as a
young man you cannot afford to marry, choose your mistress wisely,
preferably an older woman, since a pretty face adds nothing of utility
or substantial enjoyment to the transaction and moreover the older women
are so grateful.

Franklin was careful to be good because, honesty being the best policy,
it paid him to be good. And when he was not careful to be good, he was
careful to be careful.

``I grew convinced that truth, sincerity and integrity in dealings
between man and man were of the utmost importance to the felicity of
life ... Revelation had indeed no weight with me as such; but I
entertained an opinion that, though certain actions might not be bad
because they were forbidden by it, or good because it commanded them,
yet probably those actions might be forbidden because they were bad for
us, or commanded because they were beneficial to us in their own nature,
all the circumstances of things considered.''

The utilitarian point of view could scarcely be made more explicit. But
Franklin achieved a further logical extension of the utilitarian
philosophy, to which Weber calls attention in ``The Protestant Ethic.''

``Now, all Franklin's moral attitudes are colored with utilitarianisms.
Honesty is useful, because it assures credit; so are punctuality,
industry, frugality, and that is the reason they are virtues. A logical
deduction from this would be that where, for instance, the appearance of
honesty serves the same purpose, that would suffice, and an unnecessary
surplus of this virtue would evidently appear to Franklin's eyes as
unproductive waste. And as a matter of fact, the story in his
autobiography of his conversion to those virtues, or the discussion of
the value of a strict maintenance of the appearance of modesty, the
assiduous belittlement of one's own desserts in order to gain general
recognition later, confirms this impression. According to Franklin,
those virtues, like all others, are only in so far virtues as they are
actually useful to the individual, and the surrogate of mere appearances
is always sufficient when it accomplishes the end in view. It is a
conclusion which is inevitable for strict utilitarianism.''\footnote{{[}Max Weber,~\emph{\href{http://www.worldcat.org/oclc/535567}{The
  Protestant Ethic and the Spirit of Capitalism}}~(New York: Charles
  Scribner's Sons, 1930 {[}1905{]}), 50--52.{]}}

Compare this accurate characterization of Poor Richard's credo with the
attitude of the manufacturers of Creomulsion, a proprietary remedy,
expressed in a form letter designed to coerce newspaper publishers into
attacking the Tugwell Pure Food and Drugs Bill:

\begin{quote}
Gentlemen: You are about to lose a substantial amount of advertising
revenue from food, drug, and cosmetic manufacturers. Your pocketbook is
about to be filched and you will see how if you will personally study
... the enclosed copy of the Tugwell Bill. This bill was introduced by
\emph{two doctors}.... You publish your paper for profit and as a
service to your community. \emph{In most virile business organizations
the altruistic policies in the final analysis are means to the primary
end which is profit}. (My italics J. R.) ... An isolated editorial or
two will not suffice.... You need to take an aggressive stand against
this measure. You need to bring all personal pressure you can upon your
senators and representatives. You need to enlighten and thereby arouse
your public against this bill which is calculated to greatly restrict
personal rights. If this bill should become law we will be forced to
cancel immediately every line of Creomulsion advertising....
\end{quote}

Surely the italicized sentence expresses the essence of the Poor Richard
Philosophy and shows that the wisdom of Benjamin Franklin still lives in
the hearts and minds of his countrymen, especially those who, like the
manufacturers of Creomulsion, are engaged in manipulating the techniques
of rule by advertising.

\subsection{Jay Cooke}

Vernon L. Parrington, in the third volume of his \emph{Main Currents of
American Thought}, remarks that ``in certain respects Jay Cooke may be
reckoned the first modern American.''\footnote{{[}Vernon L. Parrington,
  \emph{\href{http://www.worldcat.org/oclc/759611323}{Main Currents in
  American Thought}}, vol. 3 (New York: Harcourt, Brace and Co., 1930),
  36.{]}} He financed the Civil War, and
in the course of his operations developed and used on the grand scale
most of the techniques of the modern advertiser and mass propagandist.
With the Liberty Loan drives in mind, compare Parrington's summary of
Cooke's pioneering achievements.

Under his bland deacon-like exterior was the mind of a realist.... If he
were to lure dollars from old stockings in remote chimney corners he
must ``sell'' patriotism to his fellow Americans; and to do that
successfully he must manufacture a militant public opinion. The soldier
at the front, he announced in a flood of advertisements, must be
supported at the rear.... To induce slacker dollars to become fighting
dollars he placed his agents in every neighborhood, in newspaper
offices, in banks, in pulpits patriotic forerunners of the ``one-minute
men'' of later drives.... He subsidized the press with a lavish hand,
not only the metropolitan dailies but the obscurist country weeklies. He
employed an army of hack-writers to prepare syndicated matter and he
scattered paying copy broadcast.... He bought the pressings of whole
vineyards and casks of pure wine flowed in an endless stream to
strategic publicity points. Rival brokers hinted that he was debauching
the press, but the army of greenbacks marching to the front was his
reply. It all cost a pretty penny, but the government was liberal with
commissions and when all expenses were deducted perhaps \$2,000,000 of
profits remained in the vaults of the firm to be added to the many other
millions which the prestige of the government agency with its free
advertising brought in its train.

Having successfully sold a war, Jay Cooke turned to selling railroad
stock---specifically, the Northern Pacific. He kept much of his war
publicity machine intact and used it both for this purpose and to shape
public opinion in regard to taxation funding, and the
currency---naturally in his own interests. But the outbreak of the
Franco-Prussian War smashed Cooke's European bond-selling campaign and
the fall of the house of Cooke precipitated the panic of '73.\looseness=-1

Jay Cooke carried into the realm of national finance and politics the
morals, ethics and philosophy of a frontier trader and real estate
speculator. Profoundly ignorant of social or economic principles he
wrote or had written for him contributions to economic theory which were
little more than clumsy and transparent rationalizations of a money
lender's greed. \emph{But}---he was successful in amassing great wealth;
hence he was, during his heyday, a popular hero whose opinions on any
subject were listened to with great respect by his fellow Americans.
Moreover, he was, as Parrington noted: ``Scrupulous in all religious
duties, a kind husband, a generous friend, benevolent in all worthy
charities, simple and democratic in his tastes, ardently patriotic.'' As
a man, he seems to have had neither blood nor brains---Franklin had
both---but in his life and work he applied the middle-class virtues of
Poor Richard to the acquisitive opportunities of the Gilded Age. So that
to a people given over to the worship of money-progress and
money-opportunity he was a kind of Moses, envied and revered in life by
all classes and worshipped by his biographer.

In brief, he was a mean and sorry little parvenu; and one of the
founding fathers of the religion of salesmanship and advertising. His
career marks a step in the evolution of the American crowd hero, and in
the evolution of the American pseudoculture.

\subsection{P. T. Barnum}
Salesmanship and showmanship are variants of the same technique and both
find their sanctions in Franklin's utilitarian ethic. America's greatest
showman belongs in the historical sequence of American crowd heroes for
a number of reasons. In him the doctrine of justification by works
receives its extreme pragmatic application in ``the people like to be
fooled'' and ``there is a sucker born every minute.'' That this greasy
faker, this vulgar horse-trading yokel could have successfully worn the
cloak of piety all his life; that his autobiography, the prototype of
the American success story, was for years an unrivaled best seller,
standing alongside of Franklin's \emph{Autobiography} and
\emph{Pilgrim's Progress} in many thousands of American homes; that he
was, for multitudes of his fellow citizens a model American---all this
is difficult to believe at this distance. Yet his biographer, M. R.
Werner, supplies impressive evidence that it was so.

\begin{quote}
When you give one of your daughters away in matrimony, advise her to
imitate Charity Barnum; when your son leaves home to try his luck on the
ocean of life, give him Barnum for a guide; when you yourself are in
trouble and misery and near desperation, take from Barnum's life and
teachings consolation and courage.
\end{quote}

Henry Hilgert, a Baltimore preacher, stood up in his pulpit and said
this to his congregation and there is every reason to believe that he
expressed with substantial accuracy the contemporary popular evaluation
of the great showman. The man he was talking about started his career in
a country store in Bethel, Connecticut, watering the rum, sanding the
sugar and dusting the pepper that he sold to his fellow townsmen,
cheating and being cheated, playing cruel practical jokes, all strictly
in accordance with the savage mores of that idyllic New England
community, where the public whipping post menaced the ungodly arid
suicides were buried at the crossroads. From this he advanced to running
a public lottery and with the profits went to New York, where the
advertising of Dr. Brandeth's Pills was helping James Gordon Bennett to
lay the foundations of the modern American newspaper.

At thirty-one Barnum was writing advertisements for the Bowery
Amphitheater at four dollars a week. He was a ``natural'' at the
business and used his skill to get control of the American Museum where
he began to advertise in earnest. When the posters of the negro
violinist didn't pull, he changed them to show the violinist playing
upside down. Then they pulled, and the customers didn't mind, because
Barnum gave them a flea circus and a pair of albinos as added
attractions. He advertised his theatrical performances as religious
lectures, and the best and most devout people flocked to them. His
advertisements of Joyce Heth, ``the nurse of George Washington,'' the
Japanese mermaid, the white whales, Jenny Lind and Jumbo drained the
dictionary of adjectives. Modern movie advertising has added nothing new
or better to the technique. He stood---with Tom Thumb---before kings. He
lectured to thousands on power of will and success through godliness. He
invested his money in factories and in real estate developments designed
to house religious working men who didn't drink, smoke or chew. He went
bankrupt, but with the \$150,000 his creditors couldn't get he ``came
back'' gloriously and made another fortune.

To the museum which Barnum gave to Tufts College there still come on
Sunday afternoons good people from the surrounding suburbs who stand in
awe before the stuffed carcass of Jumbo. And the college glee club still
sings:

\begin{quote}
Who was P. T. Barnum?\\
The first in tents\\
And consequently hence\\
The first in the realm of dollars and cents.\\
The first to know\\
That a real fine show\\
Must have a gen-u-ine Jumbo.\\
The first to come\\
With the needful sum\\
To found our college mus-e-um!\\
Pee Tee Barnum!!
\end{quote}

\subsection{Henry Ward Beecher}

Barnum had nerve, a kind of bucolic Yankee hardihood which enabled him
to trade in godliness with the same poker-faced effrontery that
characterized his circus barking. That, with a certain crude but
vigorous histrionism, would appear to be his contribution to the
evolution of the American crowd-hero type.

Henry Ward Beecher, in contrast, was a deplorable 'fraidcat all his
days. But he was a much more complex and interesting figure than the
great showman, and embodied more richly the conflicting strains of the
cultural heritage. He, too, was a middle-class crowd hero. Yet
curiously, his unrivaled eminence as a preacher and editor, in a period
when the influence of the church and the church press was enormous,
never quite gave him the mass influence which Barnum clearly had. One
reason for this, of course, was the scandal which clouded his later
years. But there is perhaps another and even more important reason.
Beecher, though a showman both by nature and by long training, had a
private impurity which is incompatible with pure showmanship, pure
salesmanship, pure money-making. Beecher took himself seriously. He was
a faker, a liar and a cheat, as was Barnum, and at bottom he was just
about as vulgar as Barnum. But Beecher had a personal mission---to
repudiate the harsh Calvinism of his father, the loveless despotism of
that barren Litchfield parsonage, and proclaim the gospel of love. So
Henry Ward Beecher struggled; a scared child, he begged the love of
women which he never earned; women whom he later repudiated. Seemingly
they loved him; at least they never gave him the hatred which his
cowardly betrayals richly deserved. Why? Perhaps because they pitied him
and saw that he was struggling genuinely after his fashion; struggling
to be himself, to defy the Calvinist God, to assert the Tightness of the
tremendous emotionality which was his greatest endowment. Victoria
Woodhull, that extraordinary woman, probably came close to stating the
truth about Beecher when she wrote:

\begin{quote}
The immense physical potency of Mr. Beecher, and the indomitable urgency
of his great nature for the intimacy and embraces of the noble and
cultured women about him, instead of being a bad thing as the world
thinks, or thinks it thinks, or professes to think it thinks, is one of
the noblest and grandest endowments of this truly great and
representative man. Plymouth Church has lived and fed, and the healthy
vigor of public opinion for the last quarter of a century has been
augmented and strengthened from the physical amativeness of Rev. Henry
Ward Beecher.
\end{quote}

How Beecher writhed when he read this! And with what maledictions the
brethren of Plymouth Church rejected this intolerable tribute to their
adored pastor! For it was not precisely Henry Ward Beecher's business to
revolutionize the sexual mores of his time. Not his the stuff of which
martyrs are made. Earlier in his career, Beecher had rejected this r\^ole.
When his brother, his father, most of his more courageous parishioners
had embraced the cause of abolition Beecher had played safe on the
slavery question. Instead he had chosen as his pulpiteering stock in
trade the denunciation of the liquor traffic. And the jibe of a
distiller whom he had attacked was well earned:

\begin{quote}
You cannot justify slavery by talking about the making of whiskey....
Why is thy tongue still and thy pen idle when the sentiments of thy
brother and thy church on slavery are promulgated? Thou idle
boaster---where is thy vaunted boldness? ... You are greatly to be
pitied, even by a distiller.
\end{quote}

Just what was Henry Ward Beecher's business, his ``usefulness'' to the
preservation of which he sacrificed friend after friend along with his
own honor and decency? It was the preaching business. It was also
indirectly the advertising department of the real estate business. In
\emph{Henry Ward Beecher: An American Portrait}, Paxton Hibben writes:

\begin{quote}
The investment character of his church was a matter that every
metropolitan minister of that day was expected to bear in mind. Pews
were auctioned off to the highest bidder and church scrip bore seven per
cent interest. A popular preacher was, also, a better real estate
advertisement than whole pages of publicity. Indeed, such a preacher as
Henry Ward Beecher proved, readily secured pages of publicity for the
neighborhood in which he officiated. For it was the day when church
going was the only amusement permitted the godly, and divine service
received the attention from the press later accorded theaters and social
activities.\footnote{{[}Paxton Hibben,
  \emph{\href{http://www.worldcat.org/oclc/1150829358}{Henry Ward
  Beecher: An American Portrait}} (New York: George H. Doran Company,
  1927), 107.{]}}
\end{quote}

Beecher had trained hard for this business. In a later lecture at Yale,
which took the form of a success story, he said:

\begin{quote}
I got this idea: that the Apostles were accustomed to feel for a ground
on which the people and they stood together; a common ground where they
could meet. Then they stored up a large number of the particulars of
knowledge, which belonged to everybody; when they get that knowledge
that everybody would admit, placed in proper form before their minds,
then they brought it to bear upon them with all their excited heart and
feeling.
\end{quote}

It is not difficult to recognize this as essentially the formula of Mr.
Barton's syndicated lay preachments. In fact, Beecher's pulpiteering and
Barton's syndicated essays are essentially advertisements designed to
``sell'' the acquisitive society to itself. Beecher's method was in all
important respects the method by which an advertising agency after
appropriate ``research'' arrives at the most effective ``copy slant''
with which to sell a new toothpaste or a new gargle. The basic
conviction which underlies all these enterprises in showmanship,
salesmanship and advertising is expressed in one of Mr. Barton's
favorite mottoes: ``There is somebody wiser than anybody. That somebody
is everybody.''

However, one must admit that although Beecher unquestionably had the
authentic Big Idea, he was too neurotic and too blundering ever quite to
come through as a successful advertising man. He was forever picking the
wrong theme song at the wrong time. Take his attitude toward Lincoln:

\begin{quote}
It will be difficult for a man to be born lower than he was. He is an
unshapely man. He is a man that bears evidence of not having been
educated in schools or in circles of refinement.
\end{quote}

Thousands of middle class American parvenues took that view of Lincoln
but it took a pompous blatherskite like Beecher to plump out with it
from the pulpit of a Christian church. And many of Beecher's
parishioners had sense enough to see that Lincoln was not merely a
better man but a better politician than Beecher. But there we run again
into Beecher's limiting private impurity. He was not merely a snob, but
a sincere snob.

Beecher was to achieve worse flops than this. In the year 1887, when
strikes were sweeping the country, Beecher undertook to rehabilitate his
smirched reputation by coming out as the defender of ``law and order''
and ``life, liberty, and \emph{prosperity}'' to quote his significant
revision of Jefferson. He said:

\begin{quote}
Is the great working class oppressed? ... yes, undoubtedly, it is ...
God has intended the great to be great and the little to be little ... the
trades union, originated under the European system, destroys liberty....
I do not say that a dollar a day is enough to support a working man, but
it is enough to support a man!... not enough to support a man and five
children if a man would insist on smoking and drinking beer.... But the
man who cannot live on bread and water is not fit to live.
\end{quote}

One can scarcely do better than to quote Paxton Hibben's comment on this
catastrophic muff, which the cartoonists exploited for years afterwards:

\begin{quote}
As the slogan of a great crusade in the leadership of which Beecher
could reconquer the esteem of the American public, this bread and water
doctrine somehow lacked pulling power.
\end{quote}

Beecher was not so much a cynic as a charlatan, and the limiting vice of
charlatans is that they tend to take themselves seriously. That is bad
business and the more sophisticated charlatans like Elbert Hubbard are
careful not to handicap their operations by private impurities of this
sort. Moreover, Beecher was sloppy and careless. Take his flier in
advertising in connection with Jay Cooke's Northern Pacific promotion
operations.

In January, 1870, Beecher received \$15,000 worth of Northern Pacific
stock for the express purpose of influencing the public mind to favor
the new railroad. Beecher's aid was to include the use of the
\emph{Christian Union}, a newspaper which he edited. The matter came to
light and Beecher was roundly denounced. The moral would seem to be that
Beecher should have been more careful as were some of his parishioners,
like ``Tearful Tommy'' Shearman, clerk of Plymouth Church, who were also
in on the proposition. The modern method of accomplishing the required
enlightenment of public opinion would have been for Jay Cooke to place a
substantial advertising contract with the \emph{Christian Union} and
then threaten to cancel it if Beecher failed to ``co-operate.'' Theodore
Tilton, the man whose wife Beecher begged love from and whom he ruined
and drove into exile---Theodore Tilton, also an editor, told Jay Cooke
to go to hell. But Tilton was a good deal of a man.

Beecher, like other divided souls, was not his own master. His physical
amativeness appears to have been genuine, and he was an authentic
sentimentalist, if there is such a thing. And he really did hate his
father and his father's Calvinism. So in the end, when it was fairly
safe to do so, Beecher came clean on one count. He denounced the
Calvinist hell, whose flames had been licking his conscience for all
those many years. Call it wish-fulfilment if you like, but Beecher stood
up in Plymouth Church and said:

\begin{quote}
To tell me that back of Christ is a God who for unnumbered centuries has
gone on creating man and sweeping them like dead flies---nay, like
living ones---into hell is to ask me to worship a being as much worse
than the conception of a medieval devil as can be imagined.... I will
not worship cruelty. I will worship love---that sacrifices itself for
the good of those who err, and that is as patient with them as a mother
is with a sick child.
\end{quote}

On the whole this was a pretty good negative-appeal advertisement. But
it wasn't entirely well-timed. Beecher had to alter the slant several
times before he hit the bull's eye of public opinion---that wise
``everybody'' to whom he dedicated his ``usefulness.''

Not a pleasant figure, Beecher. Half sincere and more than half neurotic
charlatans are never pleasant, nor are their lives at all happy. And in
age their faces look like the wrath of God.

\subsection{Elbert Hubbard}

In the sequence thus far we have seen a statesman, a financier, a
showman and a preacher, using the philosophy and techniques of
salesmanship and presenting themselves, with greater or less success, as
heroes for the admiration of the crowd. None of them was a professional
advertising man. But all of them were crowd leaders engaged in selling
themselves; also in selling the middle-class acquisitive ethic, and in
rounding out the body of rationalization which the expansion of American
industrial capitalism required. Advertising, as Mr. Roy Dickinson,
president of \emph{Printers' Ink}, has pointed out, is not an
independent economic or social entity. It is merely a function of
business management, and all these American crowd heroes were business
men, first, last and always.

In Elbert Hubbard, however, we encounter the advertising man \emph{per
se}, a professional of professionals. All the others had ``callings'' in
which, to earn divine favor, they were obliged to be successful. To be
successful they were obliged to employ the techniques of salesmanship,
of showmanship, of advertising, since these were the most effective
techniques of leadership and of rule in the system as they found it. But
Hubbard was called to the pure priesthood of advertising from the
beginning, and by his success in this ``calling'' became a crowd hero.
True, they called him a great writer, and a great printer, but the rose
of advertising smells the same by whatever name it is called; in effect
he never wrote or printed anything \emph{but} advertisements. This, as
we shall see, is equally true of that other great professional, Bruce
Barton.

Elbert Hubbard deserves much more careful and detailed study than he has
received at the hands of his biographers. He was born in 1856 in
Bloomington, Illinois, the son of a physician. At thirty he was already
a highly successful advertising man in the employ of a Buffalo, N. Y.,
soap manufacturer; among the sales techniques which he helped to develop
were the use of premiums and various devices of credit extension. In
1892, he had made enough money to retire and give himself a college
education. He entered Harvard as an undergraduate, but soon gave it up.
Obviously President Eliot and his academic co-workers didn't know what
America was all about. Hubbard wasn't sure himself, but he had a hunch.
It was the period of rococo enthusiasms in art, in economics, in
politics. Hubbard went to England, met William Morris, and cheerfully
appropriated all the salable elements of Morris's social and aesthetic
philosophy. He knew what he wanted, did Hubbard, and especially what he
didn't want. He wasn't having any of Morris's militant socialism for one
thing. As far as radicalism was concerned, Franklin's ``surrogate of
appearance'' was what Hubbard required---in other words a ``front.'' And
in his later career as strike-breaker and big business apologist he
discarded even that. As for art, Hubbard made haste on his return to
America to debauch everything that was good in the Morris aesthetic and
to heighten and distort what was bad to the proportions of burlesque.
The quantity of typographic and other sham ``craftsmanship'' spawned by
Hubbard's East Aurora workshop is too huge even to catalogue. Some of
the de luxe editions he got out sold for \$500 apiece. He knew his
American self-made business man, did Hubbard, and the cultural
``surrogates of appearance'' which the tycoons of the nineties required
for their libraries were hand-illuminated by a ``genius''---long hair,
flowing tie and everything---to the order of the patron.

The ``people like to be fooled'' said Barnum. But Hubbard was sharp
enough to see that the enterprise required none of the elaborate
paraphernalia of dwarfs, elephants and white whales that the pioneer
showman assembled. Hubbard was a one-man circus, and a one-man
Chautauqua. He edited and wrote a one-man magazine, \emph{The
Philistine}, and ran a one-man strike-breaking agency. A solo artist if
ever there was one. True he had helpers and disciples, but none was ever
permitted to share the limelight with the only original Fra Elbertus.
His point of view about the help was accurately expressed in \emph{A
Message to Garcia}.

\begin{quote}
It is not book learning young men need, but a stiffening of the
rertebrae which will cause them to be loyal to trust, to act promptly,
to concentrate their energies, do the thing, ``carry a message to
Garcia.''
\end{quote}

A hard taskmaster, the Fra, who got the efficiency idea early and gave
it its necessary ethical and moral rationalization. Carping critics
suggested that Hubbard's chief industry at East Aurora was working his
disciples. But big business seized upon \emph{A Message to Garcia} as a
revelation from Sinai, and the Fra simply coined money from then on.
Hubbard wrote in this classic manifesto which corporation executives
bought and distributed by the hundred thousand to their employees:

``He would drop a tear for the men who are struggling to carry on a
great enterprise, whose working hours are not limited by the whistle and
whose hair is fast turning white through the struggle to hold in line
dowdy indifference, slipshod imbecility and heartless ingratitude which
but for their enterprise would be both hungry and homeless.''

That, it would appear, was the only cause over which Elbert Hubbard ever
dropped a genuine tear. It was his own cause because he was a capitalist
in his own right. (By 1911, his plant at East Aurora included two
hotels, a group of factory buildings, and a farm, and he had five
hundred people on his payroll.) It was also the cause of the expanding
capitalist economy and the correlative acquisitive ethic, which came to
full maturity during the two decades preceding the war.

One wonders a little at the harshness with which Hubbard rebuffed the
craving of the white-collar slave for ``book-learning.'' He himself was
a kind of philosophical and literary magpie who lined his nest with
trifles gathered from the most recondite sources, both ancient and
modern. These, after having received a Hubbardian twist, were dished up
in the \emph{Philistine} and the \emph{Fra} as the authentic
pot-distilled wisdom of the sage of East Aurora.

He wrote so beautifully, sighed the newspaper critics of that May-tide
of commercial sentimentality which piled high and shattered itself upon
the realities of the war. In 1915, Elbert Hubbard went down with the
Lusitania, and the \emph{Literary Digest} in recording the event, quoted
this tribute by Agnes Herbert which appeared in the London \emph{Daily
Chronicle}:

\pagebreak \begin{quote}
Give me, I pray you, the magic of Elbert Hubbard. None of your Hardys,
your Barries, your Kiplings for me. The pen of Elbert Hubbard, an' it
please you.... Scoffers called him a literary faker. On occasion he was
so. He popularized his knowledge of the great philosophers and
transposed them so that the man in the street who would avoid the
original teachers as he would the plague, swallowed the carefully
wrapped up wisdom gratefully.... Everything Elbert Hubbard touched was
made beautiful by the magic of his mind. He was the greatest advertising
writer in the States and his methods turned the crying of wares into a
literary adventure. Each was a faceted gem not to be passed by.
\end{quote}

This seems a little lush, perhaps. The tribute of one Harold Bolce,
writing under the title ``Hubbard, the Homo, Plus'' in the
\emph{Cosmopolitan} for March, 1911, is more to the point.

\begin{quote}
Elbert Hubbard realized long ago that he was an heir of the ages and he
has foreclosed. He is rich, happy, healthy and wise. He has the woman he
loves.... He has struck pay dirt on Parnassus....

``In addition to factories and fields, the Fra has at least a quarter of
a million followers. Hubbard is not a crank. `Whom do you represent?'
was asked of Harriman when that great financier was beginning his
remarkable career. `I represent myself,' was the reply. Similarly
Hubbard does. He does not even constitute a part of the movements his
writings have helped to promote....

``A New Thought convention was in session at his inn, the delegates
paying full rates and getting their money's worth. `What is New
Thought?' asked a journalist. `Blamed if I know,' said Hubbard.... Mr.
Hubbard is sane---as sane as a cash register. In many ways he is,
perhaps, the most roundly gifted genius since Benjamin Franklin.''
\end{quote}

The Fra's production of advertising copy, not counting his \emph{Little
Journeys to the Homes of the Great}---including the home of Lydia
Pinkham's Vegetable Compound---was enormous. As Joseph Wood Krutch
pointed out in \emph{The Nation},

\begin{quote}
He is the spiritual father of all the copy which begins with an anecdote
about Socrates and ends with the adjuration to insist upon the only
genuine article in soapless shaving cream. He taught the merchant swank.
\end{quote}

Toward the end of Hubbard's career, he became overgreedy and
overconfident. The small change of lecture fees and book royalties was
not enough. He had become a pretty important fellow and felt that he was
worth important money. His price on one recorded occasion, for a job of
literary strike-breaking, was about \$200,000. Does this sound
excessive? It sounded a little high even to John D. Rockefeller and to
Ivy Lee, his public relations counsel in the lamentable affair of the
Colorado Fuel and Iron Company. The correspondence was reprinted in
\emph{Harper's Weekly}, Jan. 30, 1915, under the title, ``Elbert
Hubbard's Price,'' and the first letter is dated June 9, 1914.

\begin{quote}
Dear Mr. Rockefeller:

I have been out in Colorado and know a little about the situation there.
It seems to me that your stand is eminently right, proper and logical. A
good many of the strikers are poor, unfortunate, ignorant foreigners who
imagine there is a war on {[}the bullets that riddled the strikers'
tents at Ludlow were doubtless purely imaginary J. R.{]} and that they
are fighting for liberty. They are men with the fighting habit preyed on
by agitators....
\end{quote}

Hubbard went on to cite an article he had written about. the Michigan
copper country and said he was writing one about Colorado. He mentioned
his mailing list of 1,000,000 names of members of Boards of Trade,
Chambers of Commerce, Advertising Clubs, Rotarians, Jovians,
schoolteachers, judges and Members of Congress. He quoted a price of
\$200 a thousand for extra copies of the issue of \emph{The Fra} in
which his planned article would appear. He concluded:

\begin{quote}
Just here, I cannot refrain from expressing my admiration for those very
industrious, hard-working people, Bill Haywood, Charles Moyer, Mother
Jones, Emma Goldman, Lincoln Steffens and Upton Sinclair. Why don't you
benefit the world ... (by stating the Rockefellers' side of the case. J.
R.)?
\end{quote}

Elbert missed out on that one, although he was persistent enough. He
played golf with the elder Rockefeller. He wrote repeatedly to the
well-known Mr. Welborn, President of the Colorado Fuel and Iron Company.
``Do I make myself clear, boys?'' he seemed to be saying. He did. Ivy
Lee cannily suggested that Elbert be permitted every facility to gather
material for his article. Then if Mr. Welborn liked it, he could
doubtless arrange about the price. Ivy Lee knew his Fra. He wasn't
buying any pig in a poke from Elbert Hubbard.

The proposition, as the editor of \emph{Harper's Weekly} pointed out,
was in two parts:

\begin{enumerate}
\item
  The Fra offered to sell his opinion.
\item
  The Fra offered to make an investigation in support of his opinion.
\end{enumerate}

The Fra's one-man Chautauqua came to Middletown, N. Y. when the writer
was in high school and also working on the local daily paper. It came
twice in fact. The first year Elbert lectured on \emph{The March of the
Centuries}. It was a hodge-podge of Thoreau, Whitman, Emerson,
Michelangelo and who not. I recall being a bit puzzled, although I
reported the lecture respectfully enough, as was proper considering the
eminence of the lecturer. The next year, I was a year older and so was
the Fra. He was getting pretty seedy, in fact, I thought. Moreover, his
lecture, under a different title, was word for word the same balderdash
he had given us the year before. The next day in the columns of the
\emph{Middletown Daily Times-Press} I took out after him with shrill
cries of rage. The owner of the paper was away and I had fun. The piece
was picked up and reprinted widely. At the moment, as I remember it,
Hubbard had got himself a rating as a Bohemian immoralist, so that the
up-state editors had declared an open season on the Fra.

My employer, when he came back, was horrified. It was the first time in
the history of his management that the paper had printed an unkind word
about anybody. But the Fra didn't mind---it was just so much publicity
grist for his mill. The public likes to be fooled.

There seems to be nothing final to say about Fra Elbertus except that he
advertised and sold everything and everybody he could lay his hands on:
William Morris, Michelangelo, Thoreau, Emerson, Karl Marx, Socrates and
Paracelsus. And himself, Elbert Hubbard, a founding father of the
advertising profession---``the most roundly gifted genius since Benjamin
Franklin.'' 



% CHAPTER TWENTY
\chapter[20 \hspace*{1mm} THE CARPENTER RE-CARPENTERED]{20 THE CARPENTER RE-CARPENTERED}

\newthought{Although} Mr. Bruce Barton represents a logical projection of the rising
curve along which we have traced the evolution of the American hero, he
is, after Elbert Hubbard, rather an anti-climax. Mr. Barton's r\^ole in
the war, as director of publicity for the Y.M.C.A. was comparable, in a
way, to that of Jay Cooke in the Civil War. But Mr. Barton's r\^ole was
much smaller and the techniques employed were much more impersonal and
mechanized. Moreover, this mechanization and industrialization of sales
publicity became even more pronounced during the period of advertising
expansion that lasted from the Armistice to the fall of 1929. It would
seem that Mr. Barton's distinctive contribution to the evolution of the
American hero was the professionalization of advertising salesmanship
and its sanctification in terms of a modernized version of the
Protestant Ethic. The analysis is complicated by the fact that we are
dealing here with a contemporary figure whose career is not completed;
nor are the facts of his career readily available. This, however, is
perhaps not so important as it might seem. Mr. Barton has been a
prolific writer, and it is with the evolution of his thought that we are
primarily concerned.\footnote{{[}For a particularly perceptive treatment of Barton, see Lears, T. J.
  Jackson, ``From Salvation To Self-Realization: Advertising and the
  Therapeutic Roots of the Consumer Culture, 1880--1930,'' in
  \emph{\href{http://www.worldcat.org/oclc/654538769}{The Culture of
  Consumption: Critical Essays in American History 1880--1920}}, edited
  by Richard Wightman Fox and T. J. Jackson Lears, 1--38 (New York:
  Pantheon Books, 1983).{]}}
  
\enlargethispage{\baselineskip}

\emph{The Man Nobody Knows} was published in 1924, the year following
the publication of Veblen's \emph{Absentee Ownership}, which, in
general, has supplied the framework of theory for this analysis.\footnote{{[}Bruce Barton,
  \emph{\href{http://www.worldcat.org/oclc/70421692}{The Man Nobody
  Knows}} (New York: Grosset \& Dunlap, 1924); and Thorstein
  Veblen,~\emph{\href{http://www.worldcat.org/oclc/752183}{Absentee
  Ownership and Business Enterprise in Recent Times: The Case of
  America}}~(New York: B. W. Huebsch, 1923).{]}} It
was only with the publication of this book that Mr. Barton became, in
any sense, a national figure. In retrospect it is clear that the
ad-man's pseudoculture had already entered upon its period of decadence.
The far flung radiance of advertising during this period was a false
dawn; the fever-flush of a culture already doomed and dying at the
roots. But it is precisely in such periods that the nature of the
culture is most explicitly expressed and documented. \emph{The Man
Nobody Knows} is an almost perfect thing of its kind: more significant
and revealing as a sociological document, I think, than either Barnum's
autobiography or Hubbard's \emph{Message to Garcia}. We see the same
thing in the Athens of Pericles. As Euripides was to the more virile
poets of the Athenian rise to power, Aeschylus and Sophocles, so Bruce
Barton is to Barnum and Fra Elbertus.

When Mr. Barton published \emph{The Man Nobody Knows} he had already
achieved some standing as a writer of articles and fiction for the
popular women's magazines and his lay sermonettes were appearing in the
\emph{Red Book}. The advertising agency in which he was senior partner
was rapidly expanding and the chorus of applause which greeted \emph{The
Man Nobody Knows} was no small factor in enhancing the prestige and
profits of its author's more strictly secular activities in promoting
the sale of such products as Lysol, Hind's Honey and Almond Cream,
\emph{The Harvard Classics}, and a little later, the Gillette safety
razor and blades.

In 1924 the writer was in California, employed at part time by a San
Francisco advertising agency and for the rest, engaged in seeing the
country, writing poetry and participating in indigenous cultural
enterprises including the editing of an anthology of contemporary
California poetry. In connection with this latter enterprise, conducted
in collaboration with Miss Genevieve Taggard and the late George
Sterling, I encountered the poetry of Robinson Jeffers, whose work was
then almost unknown and who was living at Carmel on the California
coast. Greatly excited, I went to the editor of a magazine published in
San Francisco and devoted to the economic and cultural interests of the
Pacific coast. I informed this editor that California had a great poet
and that I should like to call attention to his work in the pages of the
magazine.

\enlargethispage{\baselineskip}

It is at this point that Bruce Barton enters the picture. Shortly
before, I had been approached by this editor, or his associate, with a
practical proposition. The lay sermons of Mr. Barton in the \emph{Red
Book} were considered highly edifying by the ex-Kansans and ex-Iowans
who had sold their farms and come to sun their declining years on the
California littoral. The editor felt that if he was to increase his
circulation, he must offer equivalent literary and philosophical
merchandize. I was an advertising man. Mr. Barton was an advertising
man. Couldn't I write something just as good as Mr. Barton's
sermonettes?

I tried. As I studied my model it seemed a simple enough task. I, too,
could quote Socrates, and Emerson, and Lincoln. I had the requisite
theological background---my grandfather had been a shouting Methodist.
And as for the style, I, too, I thought, could be simple though erudite,
chaste though human, practical but portentous.

Well, I wore out one whole typewriter ribbon on that job and produced
nothing but sour parodies. Some imp of perversity stood at my shoulder
and whispered obscenities into my ear. I quoted Marx when I had intended
to quote Napoleon or Benjamin Franklin. Desperately I tried to shake off
this incubus. Once I started with a quotation from Louisa May Alcott,
but when I pulled it out of the typewriter it read like a contribution
to Captain Billy's Whizbang.

Some of the least awful of my efforts I submitted to my prospective
employer. He shook his head. They lacked the human touch, he said. As a
matter of fact, they were human, all too human. My spirit was willing
but the flesh was weak.

The editor was kindly and told me to keep trying. I was still supposed
to be trying when I came in to bring up the matter of Robinson Jeffers.
As I recall it, there was some confusion on that occasion. The editor
was still hot on the trail of a Bruce Barton \emph{ersatz} and he
couldn't get it through his head that I was talking about something
else. When I finally managed to get within hailing distance of his
attention, he consented reluctantly to print an unpaid review of
Jeffers' privately printed \emph{Tamar}. The magazine did print part of
my review but the editor wrote a footnote in which he dissented strongly
from my enthusiasm.

About two years later, I saw a copy of a magazine published at the
Carmel artists' colony. The center spread was an advertisement of a
Carmel realtor headed ``Carmel, the Home of Jeffers.'' That gave me
pause. If I had only gone to the realtors in the first place, I
reflected, I might have made better headway with that Jeffers'
promotion. Later, too, I came to understand why I had failed so
miserably in my attempt to imitate those Barton sermonettes. The simple
fact that they were advertisements had never occurred to me. They were
and are advertisements, designed to sell the American pseudoculture to
itself.

I was used to writing advertisements. Maybe, if I had tried, I could
have written correct imitations of Mr. Barton's advertisements of
obscure but contented earthworms, of the virtues of industry and
diligence, of the vanity of fame. Also, maybe not. Mr. Barton may be
only a minor artist, but I suspect that he is inimitable.

The digression is perhaps excusable in that it reveals the early spread
of the Barton influence as compared with that of a major poet of the era
whom the average American has never heard of. By the time I returned to
New York, Mr. Barton had published \emph{The Man Nobody Knows} and was
soon a national figure comparable in influence to Henry Ward Beecher.
Instead of preaching in Plymouth Church, he was the honored guest at
luncheons of Rotary and Kiwanis clubs and Chambers of Commerce. Instead
of editing a religious journal---early in his career he had edited the
short-lived \emph{Every Week}---his syndicated sermonettes were
published in hundreds of newspapers. A professor of homiletics in a
well-known seminary has assured me that the influence of Mr. Barton's
writings upon the Protestant Church in America has been enormous. The
son of a clergyman, brought up in a small Middle-Western city, not
unlike the ``Middletown'' so ably described by the Lynds, he learned
early the lessons of pious emulation and of ``salesmanlike
pusillanimity'' which were the ineluctable patterns of behavior for all
young men of good family.\footnote{{[}Robert S. Lynd and Helen Merrell Lynd,
  \emph{\href{http://www.worldcat.org/oclc/1001579439}{Middletown: A
  Study in Contemporary American Culture}} (New York: Harcourt, Brace
  and Co., 1929).{]}} And Mr. Barton's family was excellent. His
father was not merely a popular and respected preacher but a scholar of
parts, author of a not undistinguished life of Lincoln. But this
distinction, in the early years at least, brought no proportionate
pecuniary rewards. So Bruce suffered the typical ordeal of the
minister's son. He had the entree to the best houses in the community
but no money with which to compete in the local arena of conspicuous
waste, pecuniary snobbism, etc.

Here we have the two opposing absolutes which, in his later creative
years, Mr. Barton undertook to reconcile: the quite genuine Christian
piety, and enforced asceticism of the parsonage, and the ``spirit of
self-help and collusive cupidity that made and animated the country town
at its best.'' The quotation is from Veblen's study of the country town
in \emph{Absentee Ownership}.\footnote{{[}Veblen, \emph{Absentee Ownership}, 156.{]}} The neo-Calvinist ethical
rationalizations described by Max Weber are brought into sharp relief by
Veblen's analysis. In the following passage he seems almost to be laying
down the ideological ground plan for Mr. Barton's subsequent career.
Says Veblen:

\begin{quote}
Solvency not only puts a man in the way of acquiring merit, but it makes
him over into a substantial citizen whose opinions and preferences have
weight and who is, therefore, enabled to do much good for his fellow
citizens---that is to say, shape them somewhat to his own pattern. To
create mankind in one's own image is a work that partakes of the divine,
and it is a high privilege which the substantial citizen commonly makes
the most of. Evidently this salesmanlike pursuit of the net gain has a
high cultural value at the same time that it is invaluable as a means to
a competence.
\end{quote}

One must not be misled into regarding Mr. Barton's specific contribution
as of the iconoclastic or creative sort. He found ready to hand the
ethical code and the theological rationalization of this code. His task
was merely that of the continuer, the popularizer. Here in Veblen's
words is a formulation, complete in all essentials, of the idealistic
code of the advertising agency business, of which Mr. Barton was to
become so distinguished an ornament:

\begin{quote}
The country town and the business of its substantial citizens are and
have ever been an enterprise in salesmanship; and the beginning of
wisdom in salesmanship is equivocation. There is a decent measure of
equivocation which runs its course on the hither side of prevarication
or duplicity, and an honest salesman---such ``an honest man as will bear
watching''---will endeavor to confine his best efforts to this highly
moral zone where stands the upright man who is not under oath to tell
the whole truth. But ``self-preservation knows no moral law''; and it is
not to be overlooked that there habitually enter into the retail trade
of the country towns many competitors who do not falter at prevarication
and who even do not hesitate at outright duplicity; and it will not do
for an honest man to let the rogues get away with the best---or any---of
the trade, at the risk of too narrow a margin of profit on his own
business---that is to say a narrower margin than might be had in the
absence of scruple. And then there is always the base line of what the
law allows; and what the law allows can not be far wrong.\footnote{{[}Veblen, \emph{Absentee Ownership}, 157.{]}}
\end{quote}

When Mr. Barton was going to high school and Sunday school, one of the
things he could scarcely help noticing was the characteristic red store
front of the A. \& P. Big Business was beginning to build the
distributive counterpart of the emerging system of mass production.
Veblen notes this transition as follows:

\begin{quote}
Toward the close of the century, and increasingly since the turn of the
century, the trading community of the country towns has been losing its
initiative as a maker of charges and has by degrees become tributary to
the great vested interests that move in the background of the market. In
a way the country towns have in an appreciable degree fallen into the
position of tollgate keepers for the distribution of goods and
collection of customs for the large absentee owners of the business.\footnote{{[}Veblen, \emph{Absentee Ownership}, 152.{]}}
\end{quote}

Mr. Barton's eminence both as advertising man and as an author became
established during the postwar decade. As most people realize by this
time, the catastrophic economic and cultural effects of the war were
deferred and postdated so far as America was concerned. This postdating
was accomplished by salesmanship and promotion applied to new
industries---notably automobiles, the movies and radio. It was without
doubt the rankest period of financial and commercial thievery in our
whole history. Salesmanship became a thing-in-itself, incorporated,
watered, reorganized, re-watered, aided and abetted by the state, and
then duly sanctified and validated under the Constitution. Veblen's
concept of \emph{Absentee Ownership} became less and less descriptive of
the actual situation in which the going rule became: ``never give a
stockholder a break.'' The more realistic terms were no longer owners
and managers but ``insiders'' and ``outsiders.'' One has only to refer
to the Insull affair, and to the exploits of Messrs. Mitchell and Wiggin
of the financial oligarchy, to establish the justice of this
description. The \emph{reductio ad absurdum} of the capitalist economy
was accomplished by the ``profitless prosperity'' of the New Era. It
will remain Mr. Barton's undying distinction that, in \emph{The Man
Nobody Knows}, he accomplished the \emph{reductio ad absurdum} of ``the
Protestant Ethic.''



\begin{center}
\LARGE{2}
\end{center}

With this background we are now in a position to give Mr. Barton's
masterpiece the sober and respectful attention which it should long ago
have received at the hands of sociologists and literary critics. It is
worth recalling that Henry Ward Beecher, too, wrote a life of Christ and
that Elbert Hubbard, albeit a free thinker, was also faithful after his
fashion in that he did not fail to exploit such elements of the
Christian tradition as suited his market. The Christs of Renan, of
Nietzsche, of Henry Ward Beecher, of Elbert Hubbard, of Giovanni Papini,
of Bruce Barton---these and other interpretations of the Christ figure
should provide an interesting and instructive gallery for the student of
human ecology. But in the space at our disposal here we must confine
ourselves to Mr. Barton's Christ. Clearly Mr. Barton felt that if the
Saviour was to live again in the mind and heart of the twentieth century
American business man, a radical though reverent reconstruction of the
legendary Christ was required.

The first point to note about \emph{The Man Nobody Knows} is that the
book is an advertisement. Mr. Barton is clearly engaged in ``selling''
the twentieth century American sales and advertising executive to the
country at large and to \emph{himself}. This secondary aspect of Mr.
Barton's unique promotion enterprise is very important. It must be
remembered that in terms of social prestige the big-time salesman, and
especially the advertising man, was still, in 1924, an upstart and a
parvenu; this in spite of the strategic, even crucial importance of the
salesman, the promoter, the advertising man in the struggle of business
to keep the disruptive force of applied science from destroying the
capitalist economy. In 1924 we were already face to face with the
tragi-comic social paradox which Stuart Chase describes in his
\emph{Economy of Abundance}. The only method of resolving that paradox
open to the business man was to sell more goods at a profit and, when
the ``sales resistance'' of a progressively dis-employed population
couldn't be broken down, to sabotage industry by monopoly control of
production and prices.

So Mr. Barton was the man of the hour on more than one count. Despite
the stout labors of P. T. Barnum, Elbert Hubbard and others, advertising
still bore the stigma of its patent medicine origins. In the callous
view of the crowd, the adman still wore the rattlesnake belt and
brandished the pills of the medicine man who, in the light of flaring
gasoline torches, had for many decades been giving the admiring citizens
of Veblen's ``country town'' practical lessons in the theory of business
enterprise and the uses of salesmanlike duplicity.

But times had changed. Advertising on the grand scale had become an
industry no less essential than coal or steel. It had become a
profession endorsed, sanctified and subsidized by dozens of
Greek-porticoed ``Schools of Business Administration'' in which a new
priesthood of ``business economists'' translated the techniques of mass
prevarication into suitable academic euphemisms. Advertising---in other
words, mass cozenage---had become a major function of business
management. The ad-man had become the first lieutenant of the new
Caesars of America's commercial imperium not merely on the economic
front but also on the cultural front.

The rattlesnake belt and the gasoline torch were no longer appropriate
for so eminent a functionary. They must be burned, buried, destroyed,
forgotten. The ad-man needed glorification and needed it badly.

\vspace{2mm}

\begin{center}
\LARGE{3}
\end{center}

It was to this task that Mr. Barton addressed himself with an elan, an
imaginative sweep and daring that can be adequately characterized only
by the word ``genius.'' Consider the magnitude of the enterprise. It was
necessary not merely to reconcile the ways of the ad-man to God, but to
redeem and rehabilitate a tedious and discredited Saviour in the eyes of
a faithless and materialist generation. Mr. Barton accomplished both of
these stupendous tasks in a single brief book. And he was able to do
this because, as a true son of his father, he had not fallen from grace.
Like a modern Sir Galahad, his strength was as the strength of ten
because his heart was pure. He was sincere.

I am aware that certain readers, who have not had the benefits of Mr.
Barton's strict upbringing, will probably question this statement. I can
only invite them to consider the evidence.

In the best homiletic tradition, Mr. Barton starts with a scriptural
text:

``Wist ye not that I must be about my Father's \emph{Business}?'' (The
italics are Mr. Barton's.)

The people settle back in their pews, the little boy in the second row
finds a safe cache for his gum, the rustle of garments ceases, and the
little boy hears the preface of Mr. Barton's great book entitled ``How
it came to be written.''

\begin{quote}
The little boy's body sat bolt upright in the rough wooden chair, but
his mind was very busy.

This was his weekly hour of revolt.

The kindly lady who could never seem to find her glasses would have been
terribly shocked if she had known what was going on inside the little
boy's mind.

``You must love Jesus,'' she said every Sunday, ``and God.''

The little boy did not say anything. He was afraid to say anything; he
was almost afraid that something would happen to him because of the
things he thought.

Love God! Who was always picking on people for having a good time, and
sending little boys to hell because they couldn't do better in a world
which He had made so hard! Why didn't God take some one His own size?

Love Jesus! The little boy looked up at the picture which hung on the
Sunday school wall. It showed a pale young man with flabby forearms and
a sad expression. The young man had red whiskers.

Then the little boy looked across to the other wall. There was Daniel,
good old Daniel, standing off the lions. The little boy liked Daniel. He
liked David, too, with the trusty sling that landed a stone square on
the forehead of Goliath. And Moses, with his rod and his big brass
snake. They were winners---those three. He wondered if David could whip
Jeffries. Samson could! Say, that would have been a fight!

But Jesus! Jesus was the ``lamb of God.'' The little boy did not know
what that meant, but it sounded like Mary's little lamb. Something for
girls---sissified. Jesus was also ``meek and lowly,'' a ``man of sorrows
and acquainted with grief.'' He went around for three years telling
people not to do things.

Sunday was Jesus' day; it was wrong to feel comfortable or laugh on
Sunday.

The little boy was glad when the superintendent thumped the bell and
announced: ``We will now sing the closing hymn.'' One more bad hour was
over. For one more week the little boy had got rid of Jesus.

Years went by and the boy grew up and became a business man.

He began to wonder about Jesus.

He said to himself: ``Only strong magnetic men inspire great enthusiasm
and build great organizations. Yet Jesus built the greatest organization
of all. It is extraordinary....''

He said, ``I will read what the men who knew Jesus personally said about
Him. I will read about Him as though He were a new historical character,
about whom I had never heard anything at all.''

The man was amazed.

A physical weakling! Where did they get that idea? Jesus pushed a plane
and swung an adze; He was a successful carpenter. He slept outdoors and
spent His days walking around His favorite lake. His muscles were so
strong that when He drove the money-changers out, nobody dared to oppose
Him!

A kill-joy! He was the most popular dinner guest in Jerusalem! The
criticism which proper people made was that he spent too much time with
publicans and sinners (very good fellows, on the whole, the man thought)
and enjoyed society too much. They called Him a ``wine bibber and a
gluttonous man.''

A failure! He picked up twelve men from the bottom ranks of business and
forged them into an organization that conquered the world.

When the man had finished his reading he exclaimed, ``This is a man
nobody knows.''

``Some day,'' said he, ``some one will write a book about Jesus. Every
business man will read it and send it to his partners and his salesmen.
For it will tell the story of the founder of modern business.''
\end{quote}

Note the ``action pattern'' suggested in the last sentence. It is a
recognized device of advertising copy technique: ``Mail the coupon
today!'' ``Look for the trade-mark!'' ``Send no money,'' etc. Business
men got the point and distributed thousands of copies of the book. In
fact no other lay sermon, save only Elbert Hubbard's \emph{Message to
Garcia}, has been so generously subsidized in this way.

Note, too, the evocation of the ``little boy'' who is, of course, Mr.
Barton himself. But he is also all the other little boys who had
squirmed in those straight pews of the Protestant Communion and now
ruled the church of business. Out of the mouths of babes. Mr. Barton,
who is, in fact, a remarkable example of arrested development, didn't
have to get down on his hands and knees to play church with these
children. Standing upright and fearless, he saw eye to eye with every
fourteen-year-old intelligence in the hierarchy of business.

The process of imaginative identification with the Saviour, suggested in
the preface, is continued in a sequence of logical and reverent
chapters: ``The Executive,'' ``The Outdoor Man,'' ``The Sociable Man,''
``His Method,'' ``His Advertisements,'' ``The Founder of Modern
Business,'' ``The Master.''

It is regrettable that space is lacking for extensive quotation. No
paraphrase of Mr. Barton's remarkable chronicle can do more than faintly
suggest the apostolic glow and conviction of the original. In the first
chapter he notes that the great Nazarene, like all successful business
executives, was above personal resentments and petty irritations. When
the disciples, weary at the end of the day, were rebuffed by
inhospitable villagers, they urged Jesus to call down fire from heaven
and destroy them. Here is Mr. Barton's imaginative rendering of the
Saviour's behavior on this occasion:

\begin{quote}
There are times when nothing a man can say is nearly so powerful as
saying nothing. Every executive knows that instinctively. To argue
brings him down to the level of those with whom he argues; silence
convicts them of their folly; they wish they had not spoken so quickly;
they wonder what he thinks. The lips of Jesus tightened; His fine
features showed the strain of the preceding weeks and in His eyes there
was a foreshadowing of the more bitter weeks to come.... He had so little
time, and they were constantly wasting His time.... He had come to save
mankind, and they wanted Him to gratify His personal resentment by
burning up a village!
\end{quote}

So, in later years, Mr. Barton, like Jesus, like Lincoln, knew how to
ignore the jeers of captious critics. He was a personage and knew it. He
had important work to do. He had to write with his own hand the
advertising message of important Christian advertisers, Jewish
advertisers and---just advertisers. And he had to direct the work of
others and endure, like Jesus, the stupidity and folly of his helpers;
like Elbert Hubbard, he was sometimes moved to cry out against the
``slipshod imbecility and heartless ingratitude which but for their
enterprise would be both hungry and homeless.'' Once in a symposium on
what the advertising agency business most needed, he wrote, ``God give
us men.''

It would seem probable, too, that Mr. Barton was not unmindful of the
career of his great predecessor, Fra Elbertus. Did Mr. Barton think of
himself as playing Jesus to the Fra's. John the Baptist? Probably not,
but the following passage suggests the comparison:

\begin{quote}
Another young man had grown up near by and was beginning to be heard
from in the larger world. His name was John. How much the two boys may
have seen of each other we do not know; but certainly the younger,
Jesus, looked up to and admired his handsome, fearless cousin. We can
imagine with what eager interest he must have received the reports of
John's impressive success at the capital. He was the sensation of the
season. The fashionable folk of the city were flocking out to the river
to hear his denunciations; some of them even accepted his demand for
repentance and were baptized.... A day came when he (Jesus) was missing
from the carpenter shop; the sensational news spread through the streets
that he had gone to Jerusalem, to John, to be baptized.
\end{quote}

Why boys leave home. Another bright young man digs himself out of the
sticks and goes to the big town to make his fortune.

In the chapter entitled ``The Outdoor Man'' Mr. Barton undertakes to
prove that Jesus was what is known as a he-man, somewhat resembling Mr.
Barton himself in stature and physique. In support of this contention he
points out:

\begin{enumerate}
\item
  He was a carpenter and carpenters develop powerful forearms. No
  weakling could have wielded the whip that drove the money-changers
  from the temple.
\item
  He was attractive to women, including ``Mary and Martha, two gentle
  maiden ladies who lived outside Jerusalem'' and Mary Magdalene, whose
  sins he forgave.
\end{enumerate}

In ``The Sociable Man'' Jesus is seen at the Marriage Feast of Cana. If
not the life of the party He is at least genial and tactful. The wine
gives out and Mr. Barton exclaims: ``Picture if you will the poor
woman's chagrin. This was her daughter's wedding---the one social event
in the life of the family.'' So Jesus, to uphold the family's
middle-class dignity turns the water into wine.

``His Method'' describes the selling campaigns of the obscure Nazarene
through which he climbed to the distinction of being the ``most popular
dinner guest in Jerusalem.'' Paul, especially, impresses Mr.
Barton---Paul, who was ``all things to all men'' and who became the hero
of Mr. Barton's latest book, \emph{He Upset The World}.\footnote{{[}Bruce Barton, \emph{\href{http://www.worldcat.org/oclc/383077}{He
  Upset the World}} (Indianapolis, IN: Bobbs-Merrill, 1932).{]}}

``Surely,'' remarks Mr. Barton, ``no one will consider us lacking in
reverence if we say that every one of the `principles of modern
salesmanship' on which business men so much pride themselves, are
brilliantly exemplified in Jesus' talk and work.''

The final conference with the disciples is presented as a kind of
``pep'' talk similar to those by which, during the late New Era, the
salesmen of South American bonds were nerved to go forth and gather in
the savings of widows and orphans.

``His Advertisements'' in Mr. Barton's view were the miracles. Here is
the way one of them, according to Mr. Barton, might have been reported
in the \emph{Capernaum News}:

\begin{center}

\textbf{PALSIED MAN HEALED}\\
\textbf{JESUS OF NAZARETH CLAIMS RIGHT TO}\\
\textbf{FORGIVE SINS}\\
\textbf{PROMINENT SCRIBES OBJECT}\\
\textbf{``BLASPHEMOUS,'' SAYS LEADING CITIZEN}\\
\textbf{``BUT ANYWAY I CAN WALK,'' HEALED}\\
\textbf{MAN RETORTS}

\end{center}

In the parables, especially, says Mr. Barton, the Master wrote admirable
advertising copy, and laid the foundations of the profession to which
Mr. Barton pays this eloquent tribute:

\begin{quote}
As a profession advertising is young; as a force it is as old as the
world. The first four words uttered, ``Let there be light,'' constitute
its charter.
\end{quote}

In ``The Founder of Modern Business'' Mr. Barton finds Jesus' recipe for
success in the following scriptural quotation:

\begin{quote}
Whosoever will be great among you, shall be your minister; and whosoever
of you will be the chiefest, shall be servant of all.
\end{quote}

Mr. Barton is quick to identify this as the modern ``Service'' creed of
Rotary. He says:

\begin{quote}
... quite suddenly, Business woke up to a great discovery. You will hear
that discovery proclaimed in every sales convention as something
distinctly modern and up to date. It is emblazoned on the advertising
pages of every magazine.
\end{quote}

One gets fed up with this sort of thing rather easily. Addicts of the
faith who find their appetite for the gospel according to Bruce Barton
unappeased by the foregoing quotations, are urged to consult the
original. The book ran into many editions and duly took its place on the
meagre bookshelves of the American Babbitry, alongside of the First
Success Story---Benjamin Franklin's \emph{Autobiography}, the Second
Success Story---P. T. Barnum's \emph{Autobiography}, and a de luxe
edition of Elbert Hubbard's \emph{The Message to Garcia}.

In due course, Mr. Barton's great book was made into a movie, which
enjoyed some success and further extended the popularity and influence
of the author. So far as I know, no attempt has been made to sculpture
Mr. Barton's re-carpentered Carpenter in wood, plaster or papier-mach\'e.
It would seem that the dissemination of the new icon might well have
been put on a mass-production, mass-distribution basis, like that of the
Kewpie doll, and Mickey Mouse. The neglect of this logical extension of
business enterprise is possibly attributable to the jealous opposition
of the vested interests concerned with the ancient Propagation of the
Faith, to which Veblen refers in a passage already quoted.

\emph{The Man Nobody Knows} was preceded by a relatively unsuccessful
lay sermon entitled \emph{What Can A Man Believe}? It was followed by
\emph{The Book Nobody Knows}, a volume of Old and New Testament
exegesis, done with Mr. Barton's characteristic unconventional charm,
which found much favor in church circles, and among Christian business
men. A collection of Mr. Barton's syndicated sermonettes has been
published under the title \emph{On the Up and Up}. One finishes the
reading of this volume convinced more than ever that Mr. Barton is
sincere. Take, for example, the quite charming little essay entitled
``Real Pleasures,'' in which the author describes his delight in
``walking along Fifth Avenue, looking in the shop windows, and making a
mental inventory of the things I don't want.'' This, from the head of
one of America's largest advertising agencies, is sheer heresy. But Mr.
Barton, being exempt from the ``vice of little minds,'' is full of
heresies. Elsewhere he praises the simple joys of primitive country
living. And when asked by ``Advertising and Selling'' to contribute his
professional credo to a running symposium which included the leading
advertising men in America, Mr. Barton went much farther than any other
contributor in recognizing, by implication at least, the inflated and
exploitative nature of the business, and in predicting the present drive
for government-determined standards and grades. It should be added that
his firm has for many years been considered rather exceptionally
``ethical'' in its practice; that it has never used bought or paid-for
testimonials; that it has declined much profitable business on ethical
grounds; that it has doubtless tried to give its clients a fair break
always, and the public as much of a break as considerations of practical
business expediency permitted. There are a number of agencies of which
this may be said, and it isn't saying much. Mr. Barton's firm, operating
well within the existing code of commercial morality, and even striving
sincerely to advance and stiffen that code, has sponsored and produced
huge quantities of advertising bunk, of expedient half-truths,
etc.---that being the nature of the business.

It is clear that in Mr. Barton we have at least four personalities:

\begin{enumerate}
\item
  The Sunday School boy who hated the Calvinist Christ (the Beecher
  complex);
\item
  The infantile, extraverted, climbing American who created that
  grotesque ad-man Christ in his own image, as a kind of
  institutionalized, salesmanlike tailor's dummy, to serve as a kind of
  robot reception clerk for the front office of Big Business.
\item
  The timid but talented minor essayist and dilettante who, given
  different circumstances, and subjected to a different set of social
  compulsions, might have produced a considerable body of charming and
  more or less scholarly prose; who might even have come to understand
  something of the meaning of the Christ legend and of the ethical
  values by which a civilization lives or dies.
\item
  The intelligent, acquisitive, informed man of affairs who knows a
  little of what it is all about, but lacks the nerve to do anything
  about it, except by intermittently adult fits and starts. Good old
  Daniel! Just what lions has Mr. Barton ever fought honestly and fought
  to a finish?
\end{enumerate}

An interesting figure, slighter on the whole than either Beecher or
Hubbard, but more complex, perhaps, than either. It was the
institutionalized and syndicated Barton that came to the fore again in
his last book \emph{He Upset the World}, which was excellently reviewed
by Mr. Irving Fineman, the novelist, in the magazine \emph{Opinion} for
April 25, 1932. Mr. Fineman notes that Mr. Barton has become a little
patronizing in his attitude toward The Man. He knows Him better now,
perhaps; certainly he recognizes that St. Paul was a better business
man. Says Mr. Fineman: ``It is a bit shocking, no later than the
twentieth page of this book, to find Bruce Barton censuring
Jesus---however gently! `He had no fixed method, no business-like
program.... He came not to found a church or to formulate a creed; He
came to lead a life.' So that, once having assigned to each his job---to
Jesus, as it were, the divinely pure genius, and to Paul, the hustling,
mundane \emph{entrepreneur}---it becomes a simple matter for Mr. Barton
to accept, indulgently, the impracticality of the one, who hadn't the
sense apparently to syndicate his stuff, and the go-getting tactics of
the other, who was frankly, `all things to all men.'\thinspace''

\enlargethispage{\baselineskip}

In his preface, Mr. Barton explains that he hadn't been interested in
St. Paul at first, but was induced by his publisher to re-examine the
scriptural sources and thereby converted to writing the book. Mr.
Fineman's parting jibe deserves recording:

``He should be warned however against the wiles of publishers, lest one
of them induce him to write a little book about Judas.''

The implied analogy would be more just if, in Mr. Barton, we were
dealing with an adult and fully integrated personality, but obviously
this is not the case. One does not accuse a child of betraying anything
or anybody. And Mr. Barton exhibits, more clearly, I think, than any
other contemporary public figure, the characteristic infantilism of the
American business man.

One suspects, however, that Mr. Barton has grown up sufficiently to
regret his masterpiece; indeed, that it is beginning to haunt him, like
a Frankenstein monster. The following episode, which I have slightly
disguised, out of consideration for the organization involved, would
appear to confirm this suspicion.

I was once visited in my office by a lady who represented a committee,
organized to serve a worthy, sensible, and admirable philanthropic
cause. The committee was getting out a new letterhead, of which she
showed me a first proof. She explained that she wanted a pregnant
sentence that would express the high aims of her movement. She had found
that sentence in \emph{The Man Nobody Knows}, by Bruce Barton, author
and Christian advertising man. She had learned that I knew Mr. Barton.
She knew that his books were copyrighted, but? Would I intercede for her
and obtain Mr. Barton's permission to use as the motto of her society
one of the most felicitous and beautiful sentences she had ever read?

Gladly, I replied, wondering what this was all about. But what was the
sentence?

She opened the Book. She pointed to the underlined sentence. It read:
``Let there be Light!''

I dictated a long memorandum urging Mr. Barton to grant her request. Mr.
Barton was not amused.




% CHAPTER TWENTY-ONE
\chapter[21 \hspace*{1mm} A GALLERY OF PORTRAITS]{21 A GALLERY OF PORTRAITS}

\newthought{No description} of the ad-man's pseudoculture can be considered complete
without some notation of the curious atrophies, distortions and
perversions of mind and spirit which the ad-man himself suffers as a
consequence of his professional practice.

I have heard it said of So-and-so and So-and-so in the profession:
``They are born advertising men.'' Obviously this cannot be true. Even
if one assumes the inheritance of acquired characteristics, the
phenomenon of advertising is too recent in biological time to have
brought about any substantial modification of human genes. Moreover,
although I have known many perverse and diabolical little boys, none of
these creatures was sufficiently monstrous to prompt the suspicion:
``This will grow up and be an advertising man.''

No, the ad-man is born not of woman, but of the society. He is the
subhuman or pseudohuman product of an inhuman culture. His insanities
are not congenital. They are the insanities of a society which, having
failed to embody in its growth process any valid economic, ethical or
moral concepts, is moronic in these respects. The ad-man seems
exceptional and terrifying merely because his whole being is given over
to the expression and dissemination of this moronism.

The ad-man is not necessarily an intellectual prostitute. As already
pointed out, if one accepts the economic and social premises of American
capitalism, the ad-man plays a logical and necessary r\^ole. The
production of customers, and the control of factory production in the
direction of profit-motivated obsolescence---these are functions in a
profit economy no less essential than the production of coal or steel.
Most advertising men feel this very strongly. It gives them confidence
and conviction, so that they are the more easily reconciled to their
habitual and necessary violation of the principles of truth, beauty,
intelligence and ordinary decency. They are profit-motivated producers
of customers, and they have the producer's psychology. It is right and
beautiful to make a customer out of a woman, even though this involves
making her into a fool, a slave and a greedy neurotic. \emph{It is so
right and so beautiful that the ad-man tends to make the same sort of
thing out of himself, his family and his friends.} I have had many
friends in the advertising business who have been solicitous about me,
because of my unorthodox views. At various times I have been put to some
embarrassment to keep them from trying, for the good of my soul, to make
me also a fool, a slave and a greedy neurotic. Your run of the mill
ad-man has no inferiority complex; indeed he is positively messianic
about his profession---there isn't a doubt in a carload of these
fellows.

This sounds quite mad, but it is also quite true. The inference, also
true, is that the society is mad; the ad-man is exceptional only in that
he carries more than his share of the burden of this madness.

Hence it is easy to absolve the ad-man on the ground that he knows not
what he does. This, I think, is a just acquittal for the vast majority
of the profession. But there are, of course, many exceptions. There are
many men and women in the profession who have explored worlds of the
mind and the spirit lying beyond this Alice-in-Wonderland world of the
advertising business. They are perhaps somewhat to be blamed, especially
those fallen angels who use their exceptional qualities of mind and
imagination actively to promote what they know to be a very dirty and
anti-social traffic. The distinction, while tenuous, is, I think,
genuine. It is between the intellectually sophisticated ad-man who sells
a part of himself to make a living, and the greedy cynic, often with a
will-to-power obsession, who sells \emph{all} of himself. I and most of
my friends in the business belonged to the first category, which is
fairly numerous. The will-to-power cynic is quite exceptional, and,
incidentally, he usually goes mad, too; he tends to believe in and
justify this acquired, distorted self; so that in the end we see this
ex-literary man or ex-artist as a Captain of Advertising, frothing at
the mouth at advertising conventions, or leading his hosts of devout,
iron-skulled ad-men into battle for God, for country and for Wet Smack
chocolate bars.

In the portrait studies which follow I have tried to include
proportionate representation of all three basic types. While these
studies are based on the writer's observation of real people, they are
all composite portraits; names, places and incidents have been
disguised. The writer is not interested in attacking individuals; rather
he permits himself the faint hope that some very likable ad-men who may
read this book may be freed from the coils of the ``systematized
illusions'' in which they have become entangled along with their
victims. When, as now, we are faced with the necessity of building a
civilization to replace the self-destroying barbarism which has hitherto
contented us, it is well to have as many people as possible know what
they are doing, even though what they are doing happens to be 

\pagebreak \noindent a mean and
dirty job. Most jobs are like that in our society, if that is any
comfort.


\section{ECONOMICS}\label{economics}}

Pete Sykes is the American University's great gift to advertising, and
perhaps the most typical advertising man I know.

In both the smaller and larger American colleges and universities,
during the period just before the war, the mindset of the average bright
young man was determined by the time he became a sophomore. Pete was
above the average as to energy and charm, but in all other respects he
was the perfect stereotype of the extraverted, emulative, career man in
his undergraduate phase.

He had some literary talent and made the staff of the college newspaper.
He had some executive ability and became assistant manager of the
football team. He was personable, his family was good enough, and he
made one of the snootiest fraternities. All this happened during his
first two years. As to his studies: in a moment of confidence he once
confessed to me that he could make nothing of Professor Ely's economics,
although he had studied hard in that course. He had determined to make a
million dollars after graduating, and he had been given to understand
that economics was the science of making a million dollars.

When Pete made this confession he was the managing head of a large
Middle Western agency. Although then only in his early forties he had
already made about half that million dollars. Without benefit of Ely,
however. I tried to explain. I cited the correspondence of a radical
editor with an engineer, exiled in Alaska, whose grown sons were in
college in Seattle and also studying Ely. The engineer became curious
and read Ely himself. He wrote: ``I think Professor Ely should have
married Mary Baker Eddy, for they are manifestly agreed as to the
non-existence of matter. And if they had married, I am confident that
their child would have been a bubble.''

Pete laughed and asked me what book he could read that did make sense. I
suggested Thorstein Veblen's \emph{Theory of Business Enterprise}.\footnote{{[}Thorstein
  Veblen,~\emph{\href{http://www.worldcat.org/oclc/220847741}{The Theory
  of Business Enterprise}}~(New York: Charles Scribner's Sons, 1904).{]}}
Fine! After lunch he stepped into a book store and ordered the book;
also a new detective novel.

I wasn't horsing Pete. He was and is a good fellow, with enough salt in
his nature to make him worth taking seriously, which is more than can be
said for most advertising men. After graduating he had been a newspaper
reporter, and he understood the surfaces of American life very well. He
was tolerant, too, if realistic. A year later he fired a friend of mine
on the ground that my friend's insistence on giving no more than half
time to the ``business nobody knows'' implied a lack of unmitigated
devotion to his profession, although in all other respects he was o.k.
My friend thought his point well taken and departed gracefully.

Pete had to fire over a third of his staff as the depression deepened,
and it bothered him. The civilization had put him on the spot, and it
wasn't fair, because he was still only a bright sophomore. His ambition,
his emulative obligation to himself, to his parents, to his classmates,
and later to a growing family, had never permitted him to achieve the
intellectual maturity which he secretly craved. What was he to do with
these stock-market-ruined surplus executives, these debt-burdened copy
writers---Smith's wife was going to have a baby, Robinson had
tuberculosis, etc., etc. Pete stalled, compromised, whittled, made
private unadvertised loans out of his own pocket, and in the end had to
fire most of them anyway.

Pete fought hard. To hold the business. To get new business. But he was
on the spot there, too. Pete was ethical, a power for ``truth in
advertising,'' and as sincere about it as practical business
considerations would permit. His agency turned out quantities of bunk,
of course. But respectable bunk. No bought-and-paid-for testimonials.
None of the gaudier and dirtier patent medicine accounts. His
fastidiousness cost him money and work. He had to prove that it was
possible to match the achievements of the testimonial advertisers by
using other, more ethical advertising methods. It wasn't easy, and
sometimes the ethical distinction between Pete's methods and those of
the testimonial racketeers seemed a bit tenuous. Particularly now that
the depression had forced advertisers to become increasingly
hard-boiled.

So Pete wasn't happy. He had worked terribly hard all his life. He was
moral. He had even cut out liquor so that he could work harder. After
failing and succeeding, failing and succeeding, half a dozen times, that
million dollars which he desired with such na\"ive emotional abandon was,
in 1929, almost within his grasp. But the stock market crash had
postponed the realization of that ambition indefinitely. And now the
iron collar of economics---Ely's, Veblen's, somebody's---economics was
not only choking him, driving down his standard of living, brushing
aside his pecuniary ambitions, but forcing him to be an advertising
faker, a slave driver, a hard-boiled executant of decisions written in
red ink and passed by vote of his board of directors.

It wasn't fair and Pete suffered. There he was, grimacing like the
gargoyle outside his skyscraper office, chilled by the winds of panic
that swept the country, watching the waters of prosperity recede, taking
with them first his profits, and now threatening the very continuance of
his profession. A tough spot. Out on the end of a limb. The buzz of the
Brain Trust in Washington worried him. Would they saw off the limb on
which he was sitting? But that would be outrageous! He was a hard,
competent worker. And a good fellow. He had fought like hell in behalf
of his employees. He had resisted the onslaughts of the advertising
vandals who were destroying reader-confidence. Economics? Damn
economics! Where did he get off in this beautiful American economic
scheme of things? And when would he get a little sleep?

You can see how hard it is to find effigies to burn, bad men to drive
out of office. I don't blame Pete. I blame the American university for
spawning so many sophomores, telling them that advertising was a
respectable career for an honest, intelligent person, and then walking
out on them as soon as the depression proved that the reverend
professors of ``economics'' were just as imbecile as any village
socialist had always said they were.... No, it's no use blaming the
university either. Let's blame Alexander Hamilton a good deal and Thomas
Jefferson, too. And John Calvin. And Daniel Shays for not being as good
a revolutionary engineer as Lenin....

I guess that let's Pete out. If I were Commissar in a Soviet
America---and I can think of few people less competent for the job---I
should want Pete at a desk around the corner. I'd have to watch him for
a while because he has a considerable will-to-power. But he's a good
fellow, and, given something serious to do, a good workman. The
depression has matured him. He isn't a sophomore any more. But there he
is, holding the bag for a staff of two hundred people, underpaying them
and overworking them because he has to, and occasionally obliged, for
business reasons, to strike those sophomoric attitudes he no longer
believes in. Pete is still one of the Kings of Bedlam. I think some
nights he prays for a revolution.


\section{BROADWAY IS SHOCKED}\label{broadway-is-shocked}}

A few years ago there came into an agency where I was working a tall
Westerner who had got himself a job in the publicity department. (Yes,
advertising agencies have publicity departments. They are quite
legitimate, although the newspapers don't like them much.)

His name was---call him Buck McMaster. He looked like a cowboy and had
been one in his youth in Oklahoma. He was a competent, facile
newspaperman and likable. The job paid more than most newspaper jobs and
it was easy. The smaller newspapers had to like the stuff and even the
desk men on the big ones were trained to say maybe, without meaning
maybe. The stuff had to tie up with the news, of course, and it had to
be competently written. But Big Business is news, and that agency was
doing jobs for Big Business. It was pap for Buck, even though they
loaded his desk with plenty of assignments.

He was happy as a lark at first. But within a couple of weeks that
cowboy was riding high and grabbing for the carriage of his typewriter.
Looking through the glass of my cubicle I could see him, scowling. And
from time to time I would hear him rip spoiled drafts out of his machine
and crunch them into the waste basket.

``Jesus Christ,'' he would bleat. ``Holy Mother, what next?''

At that time some of my signed writings were appearing in radical
magazines. He must have read something of mine and decided I was safe.

Late one afternoon he came into my cubicle and sat down.

``I'm going gaga,'' he said. ``This stuff is terrible. Do you mind
telling me---'' he leaned forward and whispered---``is this a racket,
too?''

I was startled. Newspapermen are supposed to be hard-boiled. And this
one was an ex-cowboy to boot, who looked tough enough for anything.

``Do you mind telling me,'' I asked, ``What was your last job?''

``Sure,'' he said. ``I was publicity man for---------.'' He named one of
the most salacious of the Broadway producers. ``It was a lousy job---you
know, cheap and nasty. I'd heard about the advertising business and
decided to get into something decent.''

He seemed hurt when I laughed.

``Well,'' he said morosely. ``Then I guess it's back to the bright
lights for me. I suppose you don't happen to know of anything in this
town a man can do and keep his self-respect?''

Buck got out finally by writing cheap fiction for the pulps. He was and
is a lot better than that. He has written honest, sensitive fiction
stories which he hasn't been able to sell. So he writes more pulp
fiction and is forever spoiling his business by writing it too well. He
lives in the country now, and has got himself elected justice of the
peace in his township. He's an honest judge, although he tells me the
local political pressures are considerable. He has a considerable local
reputation among the young people. When a couple arrives at his house,
wanting to get married, he first strives earnestly to dissuade them. If
he is over-ruled, he then leads them to an idyllic spot beside a brook
and reads them the Song of Solomon. Finally, he refuses to accept a fee.


\section{PURE GOLD}\label{pure-gold}}

There is a very scared man huddled back of his desk in a big Western
agency. He is one of the most gifted literary craftsmen I know. He is
something of a sophisticate, and I am confident has never believed a
word of the millions of words of advertising copy he must have written.
But he rarely says anything like that, even when drunk.

He is very scared. He is in his late fifties now, and has six children.
He is very eminent and successful, but he is scared just the same. As
the depression deepened, he saw to it that the people in his department
who stayed could be counted upon to protect his job. Just before the
bank holiday he put ten thousand dollars in gold coins in his safe
deposit box. Every now and then he would go in and make sure that the
gold was still there.

Mr. Gentroy. The brilliant Gentroy. Once he had literary ambitions. But
he was scared. And he is old now. A little of his light red hair is
still left. His face is red, too. When you ask him something he never
commits himself. And when you listen to him, you wonder who or what is
speaking.

There was something there once. A person. Possibly an artist. It is gone
now. For years he has been following Mr. Goode's prescription: he has
been turning people into gold. Now he is gold himself. Pure gold. Only
occasionally, when he is drunk, does a small bubble of laughter or anger
rise to the surface. The refining process is never quite complete. But
Gentroy, because he was so scared, has carried it farther than most.
Gold. Pure gold.


\section{POSTURE}\label{posture}}

Bodfish had asked the doctor about liquor, and the doctor had shrugged.
Bodfish had a leaky heart---the diagnosis was positive on that point.
Yet when Bodfish had asked him about liquor, the very Jewish, very
eminent and very expensive diagnostician had looked out of the window,
lowered his Oriental eyelids, and shrugged.

So Bodfish had gone directly from the doctor's office to the speakeasy.
In half an hour he was jolly. An hour later the Good Kid came in and
told him cheerfully that he was tight. He hadn't felt tight. On the
contrary, he felt himself to be the center of an immense, serene and
sober clarity. The experience was not unknown to him. The creative
moment. It was his ability to experience such moments that made him a
great advertising man. He had felt this way the night he had thought of
the Blisterine idea, which had revolutionized the advertising of
proprietary medicines. A sense of power, of marching analysis, of
kaleidoscopic syllogisms resolving into simple, original and utterly
right conclusions.

The sensation was similar, but this time his relaxed, athletic mind was
exploring strange territory. Himself. His life. The curious, strained,
phantasmagoric pattern of his days.

There was something he had wanted to tell the Good Kid, but she wouldn't
listen. He had felt a beautiful, paternal pity for the Good Kid. It
wasn't her fault, he had tried to tell her. It wasn't his fault, either.
They were both victims. As he said it, he had put forth a hand, the
wrist hairy, the flesh around the knuckles showing the first withering
of age, and attempted to lay it upon her brow in a gesture of chaste
absolution.

The Good Kid had laughed at him. ``You're drunk, B. J.'' she had assured
him briskly. And a little later she had gone off with the art director,
leaving him alone in the speakeasy in a corner facing the mirror.

The lamps of the speakeasy were heavily shaded. But there was
light---the mood of revelation persisted. It was as if his flashing mind
played against the mirror, and in that clear illumination the face of
Bodfish stared out at him in sharp relief. There were two Bodfishes now.
There was Bodfish, the ad-man, posing, gesticulating in the mirror. And
there was a new, masterful, illuminated Bodfish who smiled sardonically,
fingered his cigar, and continued the inquisition of that
Mephistophelian physician.

``Do you want the truth?'' the physician had asked, and Bodfish had said
he did.

Now, with the patient caught in the relentless reflection of the mirror,
Bodfish repeated the question.

``Do you want the truth?''

The lips in the mirror smiled. The head nodded. Yes, it was to be the
truth.

``Your posture is bad, Bodfish. Stand up!''

Bodfish stood up.

``Your nose is six inches ahead of your body. You're ahead of
yourself.''

\enlargethispage{\baselineskip}

The face in the mirror smiled deprecatingly. Bodfish's associates had
frequently made that flattering complaint. Bodfish was too bright. He
thought too fast. His mind was so active that---------

``Nonsense, Bodfish. I doubt very much that you have ever in your life
experienced the discipline of honest thought. That head and shoulder
posture what does it remind you of?''

The face in the mirror smirked.

``A hawk? Really, if I am to do anything for you, we'll have to dispense
with a few of these bizarre illusions. There are hawks in your business,
but not many of them. As it happens a number of my patients are
advertising men. Most of them are like yourself. Have you ever watched a
mechanical rabbit run around a race track pursued by whippets?''
\clearline
The doctor hadn't said that---not quite. But being something of a
histrion, as well as a good deal of a masochist, Bodfish enjoyed
exaggerating and refining the cruelties of the diagnosis.

``Posture, Bodfish, is not merely a physical thing. Yours is a moral, a
spiritual disequilibrium. Moreover, you embody, in your own psychic and
physiological predicament, the dilemma of the civilization. Its
acquisitive nose is ahead of its economic body. It is wobbling,
stumbling, about to fall on its face. Throw your chin in, Bodfish.
Think! Do you remember when you first got into the advertising
business?''

Bodfish remembered.

``You were an average youth, Bodfish; perhaps a little more sensitive
than the average, and with a frail talent for writing---not much, but a
little. You had an idea of yourself. It was that idea that held you
together that kept your shoulders back and your chin in. Posture,
Bodfish, is largely a matter of taking thought. You thought a good deal
of yourself in those days. Everything that happened to you mattered. It
mattered to the degree that it affected, favorably or unfavorably, your
idea of yourself. Tell me, Bodfish, in those days did you think of
yourself as a charlatan, a cheat and a liar? Did you think of yourself
as a commissioned maker and wholesaler of half-truths of outright
deceptions; a degraded clown costumed in the burlesque tatters of fake
science, fake art, and fake education, leering, cozening, bullying the
crowd into an obscene tent show that you don't even own yourself---that
by this time nobody owns?''

The reflected face became distorted as Bodfish advanced upon the mirror.

``Answer me, Bodfish! You wanted me to explain to you why you've got a
leaky heart, why your back hurts so you can't sleep, why none of your
office wives takes you seriously---not after the first week anyway. The
answer is that you've not only lost the idea of a society---you've lost
the idea of yourself. It's silly to speak to you as a sick person. As a
person you've practically ceased to exist. Long ago you stuffed yourself
into the waste paper basket along with all the other refuse of your
dismal trade. You went down the freight elevator in a big bale, back to
the pulp mill. What's left is make-believe. Why, you need three gin
fizzes before you can even take yourself seriously. You flap and rattle
like a prewar tin lizzie. And you come to me for repairs! Tell me,
Bodfish, why should any intelligent man waste his time rehabilitating
\emph{you}? Why, you're as obsolete as a Silurian lizard! ... Be sensible,
Bodfish, have a drink.''

Bodfish had a drink.

``To your great profession, Bodfish! To your billion dollar essential
industry! Fill up, Bodfish!''

Bodfish filled his glass.

``To your historic mission, Bodfish, the \emph{reductio ad absurdum} of
a whole era. Drink, Bodfish!''

Bodfish drank.

``To the 40,000 ewe lambs of American advertising, who, as the crisis
deepened, poured out their last full measure of devotion on the altar of
business as usual. To the vicarious sacrifice which history exacts of
the knave, the weakling, and the fantast. Drink, Bodfish!''

At three o'clock in the morning the push-broom of the negro roustabout
encountered an obstruction under the table next the mirror.

``Mistah Tony!''

The proprietor wiped the last glass, placed it carefully on the shelf,
and leisurely emerged from behind the bar.

``Get Joe and put him in the back room,'' instructed the proprietor
briefly.

His partner, the ex-chorus girl, returned from padlocking the front
door.

``They tell me he's lost the Universal Founders account.''

``Yes. His gal friend's quitting---told me so this evening.''

The proprietor frowned, opened the cash drawer and examined a check.

``Better take him off the list, Clara.''

It was late afternoon of the next day before Bodfish awoke.

He lay quietly staring at the painting of Lake Como on the opposite
wall. Then he closed his eyes. There was something he wanted to remember
something that had happened in the night. What was it? Oh, yes, posture!
That was the word, posture. Marvellous. A big idea. Never been used in
advertising before. Nine out of ten have posture defects.

Sitting up in bed he extracted pencil and an envelope and made hasty
notes. That was it. A cinch. That Universal Founders' account wasn't
lost. Not by a damn sight.

He rose, scrubbed briefly at the dirty sink, and inspected himself in
the mirror. Eyes clear. Face rested. Cured!

Great thing, posture. What the doctor ordered.

Bodfish straightened himself. That's it. Head up. Chin in. Posture.




% CHAPTER TWENTY-TWO
\chapter[22 \hspace*{1mm} G{\"O}TTERD{\"A}MMERUNG]{22 G{\"O}TTERD{\"A}MMERUNG: Advertising and the \newline Depression}
\chaptermark{22 G{\"O}TTERD{\"A}MMERUNG}

\newthought{The} evolution of the American salesman hero, climaxed by Mr. Barton's
deification of the salesman-advertising man in \emph{The Man Nobody
Knows} was rudely interrupted by the stock market crash in 1929. During
the depression years Mr. Barton's syndicated sermonettes struck more and
more frequently the note of Christian humility. It was an appropriate
attitude. For as the depression deepened it became apparent that the
ad-man could not carry the burden of his own inflated apparatus, let
alone break down the sales-resistance of the breadlines and sell us all
back to prosperity.

The ad-man tried. It is pitiful to recall those recurrent mobilizations
of the forces of advertising, designed to exorcize the specter of a
``psychological depression'': the infantile slogans, ``Forward
America!'' ``Don't Sell America Short!''; finally, the campaign of
President Hoover's Organization on Unemployment Relief, to which the
publications contributed free space and the advertising agencies free
copy.

One of these advertisements, which appeared in the \emph{Saturday
Evening Post} issue of Oct. 24, 1931, is headed ``I'll see it through if
you will.'' It is signed in type ``Unemployed, 1931,'' and the
presumptive speaker is shown in the illustration: a healthy, well-fed
workman, smiling and tightening his belt. The staggering effrontery of
these frightened ad-men in presuming to speak for the unemployed workers
of America can scarcely be characterized in temperate language. This
campaign signed by Walter S. Gifford, president of the American
Telephone and Telegraph Company, which was at the same time paying
dividends at the expense of the thousands of workers which it had
discharged and continued to discharge, and by Owen D. Young, chairman of
the Committee on Organization of Relief Resources, was designed to kill
two birds with one stone: first, to wheedle money out of the middle
classes, and second, to persuade the unemployed to suffer stoically and
not question the economic, social and ethical assumptions on which our
acquisitive society is based and out of which the eminent gentlemen who
sponsored the campaign were making money. The particular advertisement
already referred to understated the volume of unemployment about a
third, and then the ineffable ad-man, speaking through the masque of the
tailor's dummy workman said, ``I know that's not your fault, any more
than it is mine.''

It didn't work. The rich gave absurdly little. And the sales of
advertised products continued to drop despite the pleading, bullying,
snarling editorials printed by the women's magazines at the urgency of
the business offices which saw their advertising income dropping and
their ``books'' becoming every week and every month more svelte and
undernourished.

Nothing worked, and pretty soon the ad-men had so much to do, what with
the necessary firing and retrenchment that went on in the agencies and
publications, that they no longer even pretended that they could make
America safe for Hoover by advertising us out of the depression. The
worst of it was that the general public, and even the advertisers quite
evidently didn't give a whoop about the advertising business---that is
to say, the publisher-broadcaster-agency structure. Thousands of ad-men
were out of work---and the heartless vaudevillians of Broadway sat up
nights thinking up cracks about this unregretted circumstance.

The doctors, the architects, the engineers, even the lawyers were able
to command some public sympathy. But although from 1929 on the consumer
got less and less advertising guidance, stimulus and education, it was
apparent that anybody who had the money had no difficulty in buying
whatever he wanted to buy. So that when apprised of the sad plight of
the ad-men, the unsympathetic layman was likely to couple them with the
bankers and remark in Broadway parlance, ``And so what?''

And so the evil days came, and the profession had no pleasure in them.
And the priests of the temple of advertising went about the streets in
snappy suits and tattered underwear. And when they read their
\emph{Printers' Ink} in the public library they encountered some very
saddening statistical trends.

The Advertising Record Company uses a check list of 89 magazines and
gives dollar values, which increased from \$190,817,540 in 1927 to
\$203,776,077 in 1929. By 1932 the magazine lineage had dropped to
\$16,239,587 and the dollar value to \$115,342,606. Partial figures for
1933 are provided by \emph{Advertising and Selling}. They show magazine
lineage to be about 29 per cent under the 1932 figures for the first six
months of 1933. In July the descending curve began to flatten, so that,
what with beer and the NRA the September lineage is only minus 5.88 per
cent as against September, 1932---incidentally a reversal of the usual
seasonal trend.

The curve of national advertising in newspapers behaves similarly.
Starting with a dollar value of \$220,000,000 in 1925, it reaches a 1929
peak of \$260,000,000. Then it drops to \$230,000,000 in 1930,
\$205,000,000 in 1931, and \$160,000,000 in 1932. The drop continued in
the early months of 1933, but the recovery came sooner and has gone
higher; August newspaper advertising was 23.65 per cent above the same
month of the preceding year.

As might be expected, agriculture is the sore spot of the advertising
economy as it is of the economy in general. The Advertising Record
Company's figures show a slightly earlier incidence of distress in this
quarter. National advertising in national farm publications faltered
from \$11,092,342 in 1929 to \$10,327,956 in 1930, dropped suddenly to
\$7,775,415 in 1931, and slumped hopelessly in 1932 to \$4,921,514.

Radio advertising is unique in that it shows a continuous upward trend
during the depression years up to 1933. The combined figures of the two
major chain systems, National and Columbia, show an increase of
broadcasting expenditures by national advertisers from \$18,729,571 in
1929 to \$39,106,776 in 1932. But by April of this year radio
advertising was 42.71 per cent under the total for the same month of
1932. A reversal of this trend is indicated by the August total which is
off only 16.53 per cent as against August, 1932. In spite of their
increased income during the depression, however, the Wonder Boys of
radio have managed somehow to stay in the red---NBC, for example, has
yet to pay a dividend to its common stockholders.

So much for the statistical records of the advertising industry. The
summary is incomplete since it does not include the trade press,
car-cards, outdoor advertising and direct advertising. The trends,
however, have been similar.

The human records during these years of the locust have been even more
depressing. Certainly, the Golden Bowl of advertising is not broken. But
it has been badly cracked, and through that crack has leaked at least
half of the 1929 personnel of the profession and, probably, a bit more
than half of the profession's 1929 income. This is merely a rough
estimate, since no reliable figures are available. The writer is
indebted to a leading employment agency in the field for the estimates
here given. They are based on considerable evidence plus the best
judgment of an informed observer.

Advertising salaries were, of course, preposterously inflated during the
late New Era. A good run of the mill copy writer got \$150 a week,
whereas a newspaper reporter of equal competence would be lucky to get
\$50 a week. Practically any competent artist could choose between
starving to death painting good pictures and making from \$10,000 to
\$50,000 a year painting portraits of branded spinach, pineapple,
cheese, etc., so realistic that the publications in which they were
reproduced had to be kept on ice in order to arrest the normal processes
of nature. (The writer admits that the artists were not solely to blame
for this interesting phenomenom.)

The push-button boys, the high-power advertising executives, the star
agency business getters and publication space salesmen---all these were
similarly inflated as to salaries, and as to their conviction of their
own importance. Executive salaries of \$25,000 and \$30,000 were common
in 1929, and there were even a few \$50,000 a year men, not counting the
agency owners. Research directors and merchandizing experts had also
begun to come in on the big money. In some of the larger agencies, an
owlish, ex-academic or pseudo-academic type was in great demand as a
front for the more important clients. These queer birds got from
\$12,000 to \$40,000 a year. They specialized in the higher realms of
the advertising make-believe, being as statistical, psychological,
economico-psychological, statistical-sociological as Polonius himself.
Since there was indeed something rotten in Denmark, and advertising was
distinctly a part of that something, they, too, were pierced by the
sword of the depression and fell squealing behind the arras.

Eheu! Those were the happy days! Where are they now, those Pushers of
the Purple Pen, those pent-housed and limousined ``artists,'' those
academic prime ministers in their modern dress of double-breasted serge,
those industrial stylists and package designers, those stern, efficient,
young-old, button-pushing High Priests of the Gospel of Advertising?

A few, who didn't get caught in the stock market, are sitting and
drinking in Majorca, waiting for the waters to subside and the peak of
the advertising Ararat to reappear.

Some are doing subsistence farming in Vermont and elsewhere, with
perhaps a hot dog stand as a side line.

Some of them are on the receiving end of the formula of
salesmen-exploitation which many companies have adopted as a means of
conquering the rigors of the depression. You use your own car and your
own gas trying to sell a new gadget in a territory infested by other
salesmen for the same gadget. In two months you have sold two gadgets
and your commissions amount to \$58.75. Your business expense for the
same period amounts to \$79.85. That proves you're a poor salesman.

Some of the savants are back in the fresh-water colleges teaching the
same old stuff about scientific merchandizing to the Young Idea, from
whom they carefully conceal what's happened, assuming that they know
what's happened, which is doubtful.
\clearpage
A former copy writer of my acquaintance became business manager of a
radical monthly, on a theoretical salary. Another has gone to
California, where Life is Better, and the climate more suitable for
practicing his former craft of commercial fiction. He wasn't fired, by
the way. It was merely that he found he had no aptitude for the
brass-knuckled rough and tumble of current advertising practice.

One hears that some of the unemployed poets in advertising are writing
poetry and that some of the unemployed novelists in advertising are
writing novels. Perhaps that is one explanation for the increased
tonnage of manuscripts by which editors, publishers and literary agents
have been inundated.

For the so-called ``creative'' workers in advertising, the adjustment
has perhaps been a little easier than for the executives, ``contact
men,'' space salesmen, etc. A relief administrator told the writer about
an advertising man who had presented a difficult problem to her
organization. He needed money to feed his family, but he wouldn't
surrender his respectable address just off Park Avenue. He still hoped
to get back into the running, had a hundred ``leads'' and schemes.
Meanwhile, he must look prosperous, since an indigent, unsuccessful
advertising man is a contradiction in terms.

Many of the agencies started firing and cutting right after the stock
market crash. By the fall of 1930 wholesale discharges were frequent.
During the past year the havoc has been appalling. Agencies that
formerly employed six hundred people are operating with about half that
number. In the smaller agencies the staffs have been reduced from 150 to
30, from 30 to 8, from 16 to 4. Salaries have been cut again and again.
In some agencies there have been as many as four successive cuts. They
have hit the higher and middle brackets hardest---particularly the
``creative'' staffs. The employment agent already referred to has
recently placed copy writers at \$50 and \$70 a week who in 1929 were
getting \$10,000 and \$14,000 a year. Secretaries and stenographers have
dropped from \$40 and \$303 week to \$18 and \$15. In the entire agency
field there are perhaps a handful that have refrained from cutting
salaries or have restored cuts when business improved for that
particular agency.

Mergers have been numerous during the depression. The earlier trend
toward concentration of the business in the hands of a comparatively few
large agencies has been accelerated. In the process many well-known
names have disappeared from the agency roster.

As to the effect of the weeding-out process on the quality of the
residual agency staffs, it may be said that a percentage of sheer
incompetents has been dropped; that a percentage of incompetents has
been retained because through social or financial connections they
controlled the placing of valuable business; that in general, the trend
has been toward a more rigorous ``industrialization'' of the business,
with a lower average wage scale, and a progressive narrowing of
responsibility. The residual ad-men tend to be or at least to act
hard-boiled. They do what they are told, and they are told to get and
hold the business by any available means.

Competitive business is war. Advertising is a means by which one
business competes against another business in the same field, or against
all business for a larger share of the consumer's dollar. The World War
lasted four years. The depression has lasted four years. You would
expect that advertising would become ethically worse under the
increasing stress of competition, and precisely that trend has been
clearly observable. But, as already pointed out, ethical value judgments
are inapplicable under the circumstances. Good advertising is
advertising which promotes the sale of a maximum of goods or services at
a maximum profit for a minimum expense. Bad advertising is advertising
that doesn't sell or costs too much.

Judged by these criteria, and they are the only permanently operative
criteria, good advertising is testimonial advertising, mendacious
advertising, fear-and-emulation advertising, tabloid balloon-technique
advertising, effective advertising which enables the advertiser to pay
dividends to the widows and orphans who have invested their all in the
stocks of the company. It is precisely this kind of advertising that has
increased and flourished during the depression---this kind and another
kind, namely, price-advertising, which advertising men, including that
ad-man at large, General Hugh S. Johnson, view with great alarm. This
brings us to a consideration of various confused and conflicting aspects
of the New Deal which serve excellently to document the previously set
forth contentions of the writer concerning the nature of the advertising
business, its systematized make-believe, and its strategic position in
the capitalist economy.



% CHAPTER TWENTY-THREE
\chapter[23 \hspace*{1mm} NIRA: The Ad-Man on the Job]{23 NIRA: The Ad-Man on the Job}
\chaptermark{23 NIRA}

\newthought{When} President Roosevelt succeeded to the politically bankrupt Hoover
Administration, it was necessary not merely to legislate a New Deal but
to \emph{sell} this New Deal to the American People. Tribute has already
been paid to the President's extraordinary persuasiveness in his radio
addresses. It was natural that he should choose as his first lieutenant
a high-powered sales executive, General Hugh S. Johnson, who became
Director of the NRA.

The theory of the recovery, as outlined in the pronouncements of the
President, was to raise prices and wages, eliminate cut-throat price
competition, and thereby restore the solvency of the whole capitalist
fabric of production and distribution for profit. One of the businesses
that had to be rehabilitated was the advertising business.

Speaking before the convention of the Advertising Federation of America
held at Grand Rapids in June, 1933, General Johnson said in part:

\begin{quote}
Good advertising will become more essential than ever. It will be in a
position to help the business executive avoid those wasteful and
expensive practices in selling which so often add needless costs to
needed products. Good advertising is opposed to senseless price cutting
and to unfair competition. These are two business evils which we hope to
reduce under the new plan of business administration.

Constructive selling competition will be as strong as ever, and there
will be great need for aggressive sales and advertising efforts. The
only kind of competition that is going to be lessened is the destructive
cut-throat kind of competition which harms the industry and the public
as well. There should be more competition than ever in presenting
quality products to consumers, and in selling those products. What we
are going to need more than ever is energetic, honest efforts to sell
goods to people who are going to use them....

If there is one job for advertising men and women to carry through at
this moment, it is to study the implications and effects of the
industrial recovery act and then to apply their skill in assisting
business to gain fully from the planned results of the law.
\end{quote}

\pagebreak When General Johnson addressed the Advertising Federation of America, he
was speaking to the responsible heads of the advertising business,
including the owners and managers of major publishing properties.
Certainly these gentlemen realized very clearly that if any deliberate
deflation of advertising were included in the plans of the
administration, it would mean their bankruptcy. General Johnson
understood this as well as they did. He also must have realized acutely
that the administration could not afford to do anything of the sort,
since it is highly dependent upon press and radio support for the
execution of its program---even for its continuance in office. Hence,
the wings of the Blue Eagle were spread benignly over one of the most
fantastically exploitative and non-functional businesses in our whole
acquisitive economy. With this qualification, of course: ``\emph{Good}
advertising will become more essential than ever ... \emph{Good}
advertising is opposed to senseless price cutting and to unfair
competition.''

General Johnson knew his press and knew his politics. As a patriotic
savior of capitalism he was convinced that the advertising business was
one egg that couldn't be broken even to make the omelet of the New Deal.
But it was impossible to keep the recovery program pure, even if the
President had wanted to. Reform was bound to creep in. The investment
bankers got it first in the Securities Act. And eventually that
advertising egg did get cracked, or at least candled. It turned out to
be Grade ``C'' or worse.

It was Professor Rexford J. Tugwell, Assistant Secretary of Agriculture,
who started all the trouble by insisting that the Recovery Program
should include passage of a New Pure Food and Drugs Bill, designed to
protect the health and the pocketbook of the consumer. At this writing
this bill, as revised by Senator Copeland, under pressure from the
proprietary medicine, drug, food and advertising interests, is still
being fought by these interests although most of its original teeth have
been pulled. By the time this book is published it seems probable that
the revised bill or a substitute measure will have been passed. Since
the purpose of this book is not to analyze the legislative and other
developments incident to the New Deal but to describe the advertising
business, considered as an instrument of rule controlled and manipulated
by the American business hierarchy, we shall be content in the next
chapter with showing how and why the vitamin men, the medicine men and
especially the ad-men were successful in beating the Tugwell Bill. The
story is told in detail in the 500 page transcript of the hearings on S.
B. 1944, otherwise known as the Tugwell-Copeland Food and Drugs Bill,
held in Washington, \\ \noindent Dec. 7 and 8, 1933. It is one of the most
fascinating and revealing documents the Government Printing Office has
ever issued. Reading it is a sobering experience, even though Moliere [\emph{sic}]
himself could scarcely have conceived the rich comedy of the situation.
What emerges is a cross-section of the American pseudoculture. Benjamin
Franklin, Jay Cooke, Henry Ward Beecher and Elbert Hubbard were all
there in spirit, represented by slightly burlesqued reincarnations in
the bodies of statesmen, lawyer-lobbyists, medicine men and ad-men.
Bruce Barton didn't appear at the hearings, but did his bit in the field
by speaking against the bill. Dr. Walter G. Campbell, Chief of the Food
and Drug Administration, did an altogether magnificent job in explaining
the need for the bill so that by the time he had finished the assembled
lobbyists didn't have a leg to stand on. They did, however, have plenty
of money and an effective influence upon the daily and periodical press.
In the Sept. 18, 1933, issue of the \emph{Drug Trade News} appeared the
following frank statement of the strategy and tactics of the United
Manufacturers of Proprietary Medicines, as generalissimoed by
lawyer-lobbyist Clinton Robb.

\vspace{2mm}

\begin{center}
\textit{The 17 Plans}

\end{center}

\begin{enumerate}
\item
  Increase the membership of association at once to present a united
  front in combating the measure.
\item
  Secure co-operation of newspapers in spreading favorable publicity,
  particularly papers now carrying advertising for members of the
  association.
\item
  Enlisting all manufacturers and wholesalers, including those allied to
  the trade, and inducing them to place the facts before their customers
  through salesmen, and in all other possible ways, to secure their
  co-operative aid.
\item
  Secure the pledge of manufacturers, wholesalers, advertising agencies
  and all other interested affiliates to address letters to Senators to
  secure their promise to vote against the measure.
\item
  Line up with other organizations, such as Drug Institute, Proprietary
  Association, National Association of Retail Druggists and others, to
  make a mass attack on bill.
\item
  Appointment by the President of a committee to work in conjunction
  with Attorney Clinton Robb.
\item
  Co-operation of every member in forwarding to headquarters newspaper
  clippings and all available data as basis for bulletins and favorable
  publicity.
\item
  Co-operation of every member in doing missionary work in home
  districts to arouse public to the dangers of the legislation proposed.
\item
  Carrying to the public by every means available, radio, newspaper,
  mail and personal contact, the alarming fact that if the bill is
  adopted, the public will be deprived of the right of self-diagnosis
  and self-medication, and would be compelled to secure a physician's
  prescription for many simple needs.
\item
  Arrange for conferences between Association Committee and
  representatives of all other trade associations interested.
\item
  Enlist the help of carton, tube, bottle and box manufacturers.
\item
  Defeat use of ridicule by American Medical Association, proponents of
  the measure, by replying with ridicule.
\item
  Convince newspapers of justness of cause and educate public to same
  effect.
\item
  Setting up publicity department for dissemination of information.
\item
  Enlisting aid of Better Business Bureau in various cities.
\item
  Direct and constant contact with situation at Washington under
  leadership of Attorney Robb.
\item
  Pledge of 100 per cent co-operation on part of every member of the
  association present for continued and unremitting activity in every
  possible direction to defeat measure.
\end{enumerate}

Note plan No. 15, the mobilizing of the Better Business Bureaus, which
are agencies set up by the organized advertising business to expose and
penalize dishonest and misleading advertisers. We cannot stop here to
trace the history of the Better Business Bureau, except to point out
that its criteria of ``Truth in Advertising'' are the commercial
criteria already discussed in an earlier chapter; further, that even
these criteria cannot be applied to the disciplining of important
advertisers or powerful advertising agencies. The internal politics of
the advertising business is realpolitik. The Better Business Bureau can
point with pride to the scalps of numerous ``blue-sky'' stock promoters
and cheap and nasty patent medicine racketeers whom it has put out of
business. But in the nature of the case it cannot successfully hunt
bigger game, indeed it is not designed for this purpose. It is
essentially a ``Goose Girl'' organization which is concerned with the
maintenance of reader confidence, with keeping the methods and practices
of the advertising profession within the tolerance limit of an
essentially exploitative traffic.

But the Tugwell Bill attacked this traffic at several vital points : (1)
the clause declaring a drug to be misbranded if its labeling bears any
representation, directly or by ambiguity or inference, concerning the
effect of such drug, which is contrary to the general agreement of
medical opinion; (2) a similar clause leveled at false and misleading
advertising, which provided that the advertisement of the drug or
cosmetic be considered false ``if it is untrue, or by ambiguity or
inference creates a misleading impression''; (3) the clause authorizing
the Secretary of Agriculture to ``promulgate definitions of identity and
standards of quality and fill of container for any food.''

But ``ambiguity and inference'' is the stock-in-trade of the advertising
copy writer. And as for quality standards, it is the recognized task of
advertising to establish \emph{systematized illusions} of quality which
will lift the product above the vulgar level of price competition.

Being thus clearly attacked, it was to be expected that the ``reform''
pretensions of the advertising business would pretty much collapse; that
the profession would make a more or less united front with the patent
medicine racketeers, and with the drug, food and cosmetic industries;
that newspapers, magazines and radio stations would either actively
fight the bill or fail to support it.

In effect that is what happened, although the more respectable
advertisers and publications were considerably embarrassed by the rough
tactics of the patent medicine lobby, and certain partial cleavages
developed. At its annual convention the Association of National
Advertisers failed to pass resolutions attacking the bill, for the
reason, doubtless, that those advertisers who were affected slightly if
at all by its provisions felt that ``reader confidence'' would indeed be
somewhat rehabilitated if the patent medicine advertisers, the
``Feminine Hygiene'' advertisers, etc., were obliged to pull their
punches a little. \emph{Advertising and Selling} and \emph{Editor and
Publisher} attempted to play fair, gave much space to the proponents of
the bill and stoutly refused to ``go along'' with the campaign of abuse,
misrepresentation and press coercion laid down in Mr. Robb's ``17
Plans.''

To meet this attack Professor Tugwell and the officials of the Food and
Drug Administration had to rely upon their excellent and popular case,
upon the support of a handful of liberal and radical publications, which
carried little or no advertising, upon the far from active or organized
support of the medical profession, and upon the intermittent and poorly
financed help of a few women's clubs and consumer organizations. The
Food and Drug Administration had no propaganda budget; it did, however,
manage to stage its famous ``Chamber of Horrors'' exhibit at the Century
of Progress and later route this exhibit to women's clubs and other
organizations which asked for it. This pathetically inadequate attempt
to fight back was greeted by yells of rage from the patent medicine
lobby which was busy spending money lavishly in the execution of Mr.
Clinton Robb's ``17 Plans.'' United States Senators began getting
letters like the following from Mr. Daniel A. Lundy:

\begin{quote}
My dear Senator: It would seem, if Section 6 of the Deficiency
Appropriation Act, for the fiscal year of 1919 and prior year, is still
active, Walter Campbell may well be dismissed and prosecuted for his
alleged gross violations and abuse of authority, in spending government
money without permission of the Congress for radio, Paramount News Reel,
diversion of his employees' time for selfish purposes and other means to
influence passage of unconstitutional Tugwell-Copeland-Sirowich Food and
Drug Bills.

Walter Campbell, it would seem, has overridden all official propriety
and wisdom in his alleged overt act, and no public trust or confidence
once violated, as in this case, can be restored. There seems but one
road for Congress---the road in dismissing the Chief of the F\&D
Department, with penalties, if substantiated.

All others who have aided and abetted in these vicious and irregular
proposals, whether in lending their names or in actions, should come
under the same discipline.

Honest industry and a decent public prays for a thorough and speedy
investigation and not a white-wash of an alleged crime as despicable and
deplorable as the sell-out of the ``Teapot-Dome.''
\end{quote}

Mr. Lundy, as might be guessed, is a member of the Board of Managers of
the United Medicine Manufacturers Association. He is also connected with
the Home Drug Company, against which the Food and Drug Administration
has a case pending. But the Senator didn't know this. Nor was the Food
and Drug Administration empowered to tell him unless he specifically
asked; it had no means and no power to expose one of the most brazen and
vicious lobbies that ever disgraced Washington. In the \emph{Nation} of
February 14 the writer undertook to expose this lobby and the substance
of that article, which was entered in the record of the second hearing
by Mrs. Harvey W. Wiley.




% CHAPTER TWENTY-FOUR
\chapter[24 \hspace*{1mm} ALL FOR PURITY]{24 ALL FOR PURITY}

\newthought{There} are no interested, profit-motivated lobbyists at Washington; only
patriots, crusaders, guardians of our most sacred institutions, saviors
of humanity. If you doubt this, read the transcript of the public
hearings held December 7 and 8 in Washington on the Tugwell-Copeland
Food and Drug Bill. If, after that, you are still cynical, you should
read the mail the President, General Johnson, and Postmaster Farley
received from the patriotic medicine men, vitamin men, and cosmeticians
whose sole concern appeared to be the welfare of the present
Administration and the NRA. The names of these correspondents cannot be
divulged, but here are a few samples of their style:

\begin{quote}
With yourself and every other loyal citizen of the United States
endeavoring to assist in the relief of unemployment, it would seem that
any type of legislation that would retard the recovery of business would
be unfortunate at this time. Therefore, House Bill 6110 and the Copeland
Bill should be given serious consideration as their effect upon an
enterprise with an annual output of over \$2,000,000 would be serious
indeed....

We have no objections to regulation but ... here is no ordinary
regulator measure of the industry. Here is a bill known as the Tugwell
Bill ... that openly demands that the Secretary of Agriculture in
enforcement of regulations be final and absolute and without appeal to
the courts.... Now I'm no disgruntled manufacturer writing you; I'm
quite well able to take care of myself and have been doing it in this
business for many, many years....

Practically all the worth-while factors in proprietary cosmetic, drug,
food, and advertising industries are in accord that these Tugwell
measures are impossible of amendment and should be withdrawn....

I have recently been impressed with the danger to the Administration
that is resulting from the agitation created by what is known as the
Tugwell Bill....
\end{quote}

There are four main points to note about this huge correspondence, of
which only a few typical examples have been excerpted: (1) that the
names of most of the ready letter-writer firms are already familiar
through notices of judgment issued by the Food and Drug Administration
at the termination of cases brought under the present inadequate law, in
Post Office fraud orders or in the Federal Trade Commission
cease-and-desist orders; (2) that the writers invoke the principle of
``recovery'' as opposed to ``reform'' in order to defend businesses
which in most cases are demonstrably a danger and a burden to both the
public health and the public pocketbook; (3) that they do not hesitate
to misrepresent both the nature and effects of the bill, as for example
by asserting that Administration action would not be subject to court
review although such review would be easily available to defendants
under both the original bill and the present revised Copeland Bill; (4)
that the writers, by implication, threaten the Administration with a
political headache and political defeat, regardless of the merit of the
issues involved.

The nature and methods of this lobby can best be understood by examining
the following ``Who's Who'' of the leading lobbyists. A complete list is
as impossible, as would be any attempt to estimate the expenditure,
undoubtedly huge, of the proprietary drug, food, and advertising lobby.

\emph{Frank (Cascarets) Blair}. Mr. Blair represents the Proprietary
Association, the chief fraternal order of the patent-medicine group, but
even closer to his heart, one suspects, is Sterling Products. This is a
holding company for the manufacturers of such products as Fletcher's
Castoria, Midol, Caldwell's Syrup and Pepsin, and Cascarets, a
chocolate-covered trade phenopthalein and cascara laxative recently
seized by the Food and Drug Administration. The Proprietary Association
and Mr. Blair, plus the National Drug Conference, backed the Black Bill,
written by Dr. James H. Beal, chairman of the board of trustees of the
United States Pharmacopoeia. The Black-Beal Bill would further weaken
even the present inadequate law, make seizures practically impossible,
and permit nostrum-makers to get away with murder in their advertising.
In short, it is a sheer fake.

\emph{Hon. Thomas B. (Crazy Crystals) Love}. Mr. Love, a former
Assistant Secretary of the Treasury, is attorney for the Crazy Water
Company of Mineral Wells, Texas, manufacturers of Crazy Crystals, a
prominent exhibit last summer in the Food and Drug Administration's
well-known ``Chamber of Horrors.'' At the December hearings Mr. Love
said, ``No harm has ever resulted, or is likely to result, from the
misrepresentation of the remedial or therapeutic effect of naturally
produced mineral waters,'' which is a brazen enough falsification. Two
kinds of harm result from such misrepresentation---harm to the health of
the victim who takes a dose of horse physic under the illusion that a
dose of salts is good for what ails him; harm to the victim's
pocket-book because he paid about five times as much for that dose of
salts as it was worth.

\emph{H. M. (Ovaltine) Blackett}. Mr. Blackett is president of
Blackett-Sample-Hummert, a Chicago advertising agency. His pet account
is Ovaltine, that mysterious ``Swiss'' drink which puts you to ``sleep
without drugs'' and performs many miracles with underweight children,
nursing mothers, busy workers and old people. ``Food and drug
advertising,'' Mr. Blackett writes to magazine and newspaper publishers,
``is different from other classifications. It must actually sell the
product. It must put up a strong selling story---strong enough to
actually move the goods off the dealers' shelves.'' More briefly, Mr.
Blackett believes it would be impossible to sell a ``chocolate-flavored,
dried malt extract containing a small quantity of dried milk and egg''
for what it really is---at least for a dollar a can.

\emph{William P. (Jacob's Ladder) Jacobs}. Mr. Jacobs is president of
Jacobs' Religious List, which would appear to represent the alliance of
the fundamentalist business and the proprietary-medicine business. As a
publishers' representative of the ``official organs of the leading white
denominations of the South and Southeast,'' he offers a combined weekly
circulation of 300,317 to the God-fearing manufacturers of Miller's
Snake Oil (makes rheumatic sufferers jump out of bed and run back to
work), kidney medicines, rejuvenators (``Would you like to again enjoy
life?''), contraceptives (presumably for an equally holy purpose),
reducing agents and hair-growers. Mr. Jacobs is secretary and general
manager of the Institute of Medicine Manufacturers; he is, in fact, a
member of the old Southern patent-medicine aristocracy. His father, J.
F. Jacobs, was author of a profound treatise on ``The Economic Necessity
and the Moral Validity of the Prepared Medicine Business.''

\emph{J. Houston Goudiss}. Mr. Goudiss appears to be the missing link in
the menagerie of medicine men, vitamin men, and ad-men who crowd the big
tent of the Washington lobby and do Chautauqua work in the field. On
November 16th last he appeared before the convention of the New York
State Federation of Women's Clubs, donned the mantle of the late Dr.
Harvey W. Wiley, and begged his hearers to oppose the Tugwell Bill. He
said in part:

\begin{quote}
So far as I am known to the American public, I am known as a crusader
for the better health of our people.... Early in my career I came under
the benign influence of the late Dr. Harvey W. Wiley. I was privileged
to support him in his work ... Were Dr. Wiley alive today, I am sure
that he would be standing here instead of me. And if I presume to wear
his mantle, it is because I feel that the great urgency of the situation
calls upon me to do so.... When I was first informed that our Congress
was ready to consider a new pure food and drugs law ... I was
exultant.... Later when I read the proposed law ... my heart fell with
foreboding. I recognized it as only another overzealous measure like our
unhappy Eighteenth Amendment and the Volstead Act.... The Tugwell Bill
is fraught with danger....
\end{quote}

About that Harvey W. Wiley mantle the widow of Dr. Wiley, in the course
of an eloquent plea for the Tugwell Bill at the December hearing, said:
``I have never heard Dr. Wiley mention Mr. C. Houston Goudiss, and
inquiry at the Department of Agriculture discloses the fact that no
correspondence between Dr. Wiley and Mr. Goudiss between 1905 and 1911,
when Dr. Wiley resigned, is on file.''

And now about Mr. Goudiss himself: He publishes the \emph{Forecast}, a
monthly magazine full of vitamin chatter not unrelated to Mr. Goudiss's
activities as broadcaster over Station WOR for various and sundry food
products. He is author of \emph{Eating Vitamins} and other books also of
a signed advertisement for Phillips' Milk of Magnesia. His Elmira speech
was promptly sent out as a press release by the Proprietary Association,
and he also fought the Tugwell Bill over the radio.

The organizational set-up of the drug men, the food men, the medicine
men, and the ad-men is almost as complicated as that of the Insull
holding companies. At the top sits the High Council of the Drug
Institute, an association of associations, formed originally to fight
the cut-rate drug stores. The Proprietary Association, the Institute of
Medicine Manufacturers, and the United Medicine Manufacturers, all have
booths in this big tent. The last-named organization came right out in
the open, whooping, yelling, and rattling the wampum belt. The Food and
Drug Administration knows them well, and the public would know them
better if this department of government were authorized by law to
publicize its files. Here are a few of the most eminent and vocal
patriots and purity gospelers:

\emph{President J. M. (Toma Tablets) Ewing}. Toma Tablets are
innocuously labeled, but advertised for stomach ulcers, The advertising
clause of the Copeland Bill is what is worrying Mr. Ewing.

\emph{Vice President I. R. (Health Questions Answered) Blackburn}. Mr.
Clinton Robb, the legal magician for the U. M. M.\looseness=-1 A., fixed up the
labels of the Blackburn products, which rejoice in a string of notices
of judgment. These products are sold through an advertising column
headed ``Health Questions Answered.'' You write to Dr. Theodore Beck,
who answers the questions in this column, and the good doctor informs
you that one or more of the Blackburn products is good for what ails
you. It's as simple as that.

\emph{Vice President George Reese} is at present slightly handicapped in
selling venereal-disease remedies by the seizure by the Food and Drug
Administration a month ago of one of his nostrums---not the first action
of this kind, judging by the notices of judgment against this firm.

\emph{Vice President Earl E. (Syl-vette) Runner} can boast a dozen or
more notices of judgment against his many products, the most prominent
of which, Syl-vette, was seized only a short time ago. This ``reducing
agent'' is a cocoa-sugar beverage that keeps your stomach from feeling
too empty while a diet does the slenderizing.

\emph{D. A. (Gallstones) Lundy}, of the Board of Managers of the U. M.
M. A., advertises: ``Gallstones. Don't operate. You make a bad condition
worse. Treat the cause in a sensible, painless, inexpensive way at
home.'' But, alas, the proposed new law forbids the advertising of any
drug for gallstones, declaring the disease to be one for which
self-medication is especially dangerous. Perhaps this explains Mr.
Lundy's fervid letters to Senators demanding the dismissal and
prosecution of Chief Campbell of the Food and Drug Administration on the
ground that the latter has been improperly spending the Federal
Government's money for propaganda.

\emph{William M. (Nue-Ovo) Krause}, of the membership committee of the
U. M. M. A. Mr. Krause's Research Laboratories, Inc., of Portland,
Oregon, labeled Nue-Ovo as a cure for rheumatism until 1929 when the
Food and Drug Administration seized the product and forced a change of
the label. Nue-Ovo is still widely advertised in the West as a cure for
rheumatism and arthritis.

\emph{Kenneth (Vogue Powder) Muir}, of the Board of Managers of the U.
M. M. A. When Mr. Muir's Vogue Antiseptic Powder was seized in 1930, it
was being recommended not only for genito-urinary affections of men and
women but also in the treatment of diphtheria.

\emph{T. S. (Renton's Hydrocine Tablets) Strong}, of the Board of
Managers of the U. M. M. A., is a partner in Strong, Cobb \& Company of
Cleveland, pharmaceutical chemists who manufacture products for other
concerns. There are notices of judgment against venereal-disease
remedies and a contraceptive manufactured by them. This firm also makes
Renton's Hydrocine Tablets, a cinchophen product sold for rheumatism to
which, according to the American Medical Association, many deaths have
been directly traced.

\emph{C. C. (Kow-Kare) Parlin}. For months now Mr. Parlin, research
director of the Curtis Publishing Company, assumed much of the task of
mobilizing and directing the heterogeneous but impassioned hosts of
purity gospelers that fought the Tugwell Bill. Mr. Parlin is a
statistician, a highbrow, and no end respectable. Moreover, he
represents, indirectly at least, the \emph{Ladies' Home Journal} and the
\emph{Country Gentleman}. In their February, 1934, issues both of these
Curtis properties published editorials, written in language strikingly
similar to Mr. Parlin's recent speeches and signed writings, to the
effect that in their advertising pages they had struggled to be
pure---well, pure enough---and that the new bill was just painting the
lily.

How pure is pure? The February issue of the \emph{Country Gentleman}
contains advertisements of several products which would be subject to
prophylactic treatment if an effective law against misleading
advertising were passed. The February issue of the \emph{Ladies' Home
Journal}, which says that for more than a generation it has ``exercised
what we consider to be proper supervision over all copy offered for our
pages,'' contains advertisements of at least eight products whose claims
would require modification if the proposed bill became law. The
\emph{Ladies' Home Journal}'s ``pure-enough'' list includes Pepsodent,
Fleischmann's Yeast, Ovaltine, Listerine, Vapex, Musterole, Vicks Vapo
Rub, and Pond's creams. In addition to some of the foregoing, the
\emph{Country Gentleman} stands back of advertisements of Ipana, Toxite,
Sergeant's Dog Medicines, Bag Balm, and Kow-Kare. Concerning the
last-named product, the fact-minded veterinary of the Food and Drug
Administration comments as follows:

\begin{quote}
This used to be sold as Kow-Kure, which purported to be a remedy for
contagious abortion, until trouble threatened with the Pure Food and
Drug Administration. No drug or combination of drugs has any remedial
value in treating contagious abortion. The danger of these nostrums is
that the farmer relies upon them.
\end{quote}

There is one obvious lack in the foregoing list of purity gospelers. It
includes no women. We therefore hasten to present Gertrude B. Lane,
editor of the \emph{Woman's Home Companion}.

In the \emph{Woman's Home Companion}'s ``index of products advertised,''
the statement is made that ``the appearance in \emph{Woman's Home
Companion} is a specific warranty of the product advertised and of the
integrity of the house sponsoring the advertisement.'' Why, then, did
Miss Lane oppose the bill? Was she alarmed by the fact that the
\emph{Woman's Home Companion} publishes as pure some of the same
misleading advertisements that appear in the \emph{Ladies' Home
Journal}, already referred to, and that would be embarrassed by the
advertising provision of the Copeland Bill? It is a great industry:
women editors, publication statisticians, ad-men, vitamin men, medicine
men, cosmeticians, all in the same boat and rowing for dear life against
a rising tide of public opinion which demands that this grotesque,
collusive parody of manufacturing, distributing and publishing services
be compelled to make some sort of sense and decency no matter how much
deflation of vested interests is required.



% CHAPTER TWENTY-FIVE
\chapter[25 \hspace*{1mm} CALL FOR MR. THROTTLEBOTTOM]{25 CALL FOR MR. THROTTLEBOTTOM}

\newthought{The} inevitable conflict between the idea of capitalist ``reform'' and
the idea of capitalist ``recovery'' emerged most sharply in the drive
for commodity standards initiated by the more liberal members of Mr.
Roosevelt's official family. These liberals---loudly denounced as
``Reds'' by the patent medicine, drug and food lobby---achieved a
somewhat insecure footing in the Consumers' Advisory Board of the NRA
and in the Department of Agriculture under the leadership of Assistant
Secretary Tugwell.

It seems clear that in the beginning the Consumers' Advisory Board of
the NRA and the Consumers' Counsel of the AAA were conceived of as
decorative ingredients, designed to float around harmlessly in the
otherwise strictly capitalist alphabet soup of the New Deal. Under no
circumstances were they supposed to challenge the rule of business as
administered by the Industrial Advisory Board, backed by the Trade
Associations and the Chambers of Commerce. General Johnson's job was to
ride herd on the unregenerate forces of Big Business and induce them, by
alternate threats and pleadings, to save themselves and the country.

It was a tough assignment, and not the least of General Johnson's
embarrassments was the disposition of the Consumers' Advisory Board and
Professor Tugwell's group in the Department of Agriculture to clarify
and fortify the soup of the New Deal with some stronger functional plans
and programs.

The first blow-off came in the summer, when Professor Ogburn, of the
University of Chicago, resigned his appointment to the Consumers'
Advisory Board on the ground that a price- and wage-raising program,
unregulated by a statistical reporting service, was dangerous, and that
he had neither authority nor funds to establish such a statistical
control. This was followed by mutinous murmurs from the remaining
members of the Consumers' Advisory Board to the effect that their
carefully prepared and devastating briefs in behalf of the consumer
frequently got no further than General Johnson's desk; further, that
Charles Michelson, sitting at the publicity bottle-neck of the NRA, saw
to it that the press got only such denatured releases from the
Consumers' Advisory Board as would not disturb the equanimity of the
dominant business interests.

What the Consumers' Advisory Board and Professor Tugwell's group were
trying to do, of course, was to prevent the American people, as
consumers, from being ground between the lag of wages behind the
increase in prices---this trend being more and more apparent as the NRA
codes, with their open or concealed price-fixing provisions, went into
effect. The difficulty was that the consumer was a somewhat novel and
unsubstantial entity in the New Deal economics. Like Mr. Throttlebottom,
in ``Of Thee I Sing,'' he was the man nobody knows, although it was
precisely he whom business was theoretically set up to serve. If the
Labor Advisory Board had wished to do so, it might well have contended
that labor and the consumer are substantially identical. But it was
apparent from the beginning that the Labor Advisory Board represented
not the rank and file of labor, but the American Federation of Labor
officialdom, which was if anything less radical than Big Business
itself.

\enlargethispage{\baselineskip}

Hence the Consumers' Advisory Board was without allies at Washington and
without the support of an organized pressure group outside Washington.
One may doubt that the Chairman of the CAB, Mrs. Mary Harriman Rumsey,
had any notion of the political dynamite which any serious attempt to
discharge the ostensible functions of the board would explode. But on
the board were Dr. Robert Lynd, co-author of \emph{Middletown} and
author of a penetrating study of the economics of consumption
contributed to \emph{Recent Social Trends},\footnote{{[}Robert S. Lynd and Helen Merrell Lynd,
  \emph{\href{http://www.worldcat.org/oclc/1001579439}{Middletown: A
  Study in Contemporary American Culture}} (New York: Harcourt, Brace
  and Co., 1929); and Robert S. Lynd, ``The People as Consumers,''
  in~\emph{\href{http://www.worldcat.org/oclc/544930}{Recent Social
  Trends}}, vol. 2 (York, PA: Maple Press Company, 1933).{]}} Dr. Walton Hamilton, Yale
economist, and author of an iconoclastic dissenting opinion embodied in
the Report of the Committee on the Costs of Medical Care, and Dr. James
Warbasse, chairman of the Board of the Co-operative League. And the
staff of the CAB, headed by Dexter M. Keezer, formerly of the
\emph{Baltimore Sun}, assayed a rather high degree of sophistication
both as to economics and politics. For months both the board and its
staff were consistently rebuffed and slighted by General Johnson, and
their press releases were carefully censored by Publicity Director
Michelson. But they continued to submit briefs at code hearings, and
these briefs, although largely disregarded, kept the issues alive. And
in connection with the hearings on the Tugwell-Copeland Pure Food and
Drug Bill, there came another blow-off.

One of the most loudly mouthed charges of the patent medicine lobby was
that the Tugwell Bill was ``anti-NRA'', in that it would embarrass the
activities of nostrum makers, and reduce the income of newspapers,
magazines and broadcasters which sold advertising space and time on the
air to these nostrum makers. In the middle of the hearings, Dr. Lynd was
called over from the Consumers' Advisory Board to answer this charge.

Apparently it had never occurred to the assembled medicine men, drug
men, food men and cosmeticians, that the Consumers' Advisory Board could
be anything but the customary make-believe with which business-as-usual
cloaks its simple acquisitive motivations. Hence the consternation of
these lobbyists as Dr. Lynd proceeded deftly and suavely to invoke the
pale ghost of the ultimate consumer---to bring Mr. Throttlebottom to
life.

``Do you see what I see?'' said the ad-men to the patent medicine men.
And the drug men, the cosmeticians, the vitamin men of the food
industry, and the Fourth Estate all chimed in on a chorus of
denunciation that became more and more hysterical as the hearings
proceeded.

They saw that the drive of the Consumers' Advisory Board of the NRA to
get consumer representation on the Code Authorities and quality
standards inserted in the codes, the effort of the Consumers' Counsel of
the AAA (headed by Dr. Fred C. Howe) to insert quality standards in the
food processing and other agreements which it was then negotiating, and
the controls and penalties embodied in the Tugwell Bill, especially the
quality standards provisions, were all co-ordinate elements in the
attempt of the President's left-wing advisers to do right by Mr.
Throttlebottom, Mrs. Throttlebottom and the children.

From the point of view of business-as-usual, this sentimentalism about
the consumer is the sin against the holy ghost, nothing less. Business,
especially the interlocked drug, cosmetic, food and advertising
businesses, is organized to do Mr. Throttlebottom right, and the
difference is more than a matter of phrasing.

Amidst audible grinding of teeth by the assembled ad-men, Dr. Lynd
argued from the premises of ``quality merchandise,'' ``service'' and
``truth in advertising'' to which \emph{Printers' Ink} and other organs
of the advertising business have long proclaimed allegiance. Today, he
pointed out, in view of the elaborate fabrication of commodities, the
widespread use of synthetic materials and current packaging processes,
fair competition, the avowed objective of the NRA, must include both
quality competition and price competition. For example, the AAA had
found that the milk agreements, in order to quote price at all, had also
to quote butter fat. In nearly every line of merchandizing, a similar
need exists for quality standards on which to base price competition. In
fact, some of the producers and growers, such as the citrus fruit, rice
millers, and cling-peach canners, had actually asked for quality grades
in the AAA agreements.

The object of the NRA, continued Dr. Lynd, is to increase net buying
power, which means that it must not only increase wages but stop losses
through substandard buying. Both government and industry avoid such
losses by buying on specification. Should not consumers---the 30,000,000
families who in 1929 spent 60 per cent of the national income over
retail counters---know what they are buying? Under the New Deal, labor,
the consumer and government are recognized as co-partners in American
industry. The proposed Food and Drug Bill, like the demand for quality
standards in the recovery codes, represents a simple and necessary aid
to the isolated consumer in his difficult and largely helpless effort to
compete on an equal footing with the massed resources of industry.

Note how carefully Dr. Lynd kept within the theoretical zone of
agreement. None the less the ad-men and their allies lost no time in
putting him on the spot. The December 14th issue of \emph{Printers'
Ink}. headlined a mangled version of his statement: ``Opposes NRA, SAYS
Lynd'', and in the Dec. 21st issue Mr. Roy Dickinson, president of
\emph{Printers' Ink}, declared:

\begin{quote}
... it is my firm belief that Professor Lynd's plans in the Consumers'
Advisory Board, in connection with the Consumers' Board of the AAA, are
a definite threat to the success of the whole NRA program. His scheme of
attempting at this time to change the whole system of distribution of
trade-marked, advertised merchandise, is a distinct menace to the whole
industrial machine out of which wages, profits and government taxes must
come. Both President Roosevelt and General Johnson have publicly
expressed themselves that increased advertising of quality branded
merchandise is an integral and essential part of the whole recovery
program. Professor Lynd ... would attack over a wide front the whole
system on which not only advertising but profits depend. Which viewpoint
is truly representative of the Administration attitude? It is time that
advertisers, publishers and all other industries dependent on
advertising were told what they may expect, and get ready to fight for
their existence if the Lynd viewpoint is representative.
\end{quote}

One gathers from this that Mr. Throttlebottom just mustn't know too
much, and that any attempt to inform him must be scotched before it
starts. So Mr. Dickinson called out the advertising mob, and with
similar warning tocsins, the medicine men called out their macabre
guerrillas. The impression one gains from reading the trade press during
this period is much like that made by the final reel of a gangster
melodrama, in which the good-bad gangsters draw their rods and ``blast
their way out.'' This ferocity becomes understandable when we add up
what was at stake.

It has been roughly estimated that about \$350,000,000 a year was at
stake for the advertising business alone. This money is paid by
advertisers, chiefly through advertising agencies which collect
commissions, to newspaper and magazine publishers, broadcasters,
car-card and direct by mail companies for the advertising of foods,
drugs and cosmetics theoretically designed to inform and instruct Mr.
Throttlebottom, eliminate his halitosis, pep him up with vitamins, and
otherwise make him a better and more popular fellow.

But we have already seen that modern advertising represents not so much
a competitive selling of goods and services as a competitive manufacture
of consumption habits, the technique of this manufacture being
essentially a technique of ``creative psychiatry.'' What was attacked by
the Tugwell Bill, and even more, by the attempt to embody quality
standards in the codes, was this enterprise in ``creative psychiatry,''
and the largely irrational and un-economic consumption habits which
advertisers manufacture and capitalize. In \emph{Recent Social Trends},
Dr. Lynd notes that the Maxwell House Coffee habit of the American
people was bought in 1928 for \$42,000,000, and the Jell-O habit in 1925
for \$35,000,000. The asking price for the Listerine habit and the
``Crazy Crystal'' habit would also doubtless be impressive if we knew
them.

When the ad-men, the food men, and the drug men howl about the brain
trust's attack on ``the whole system on which not only advertising but
profits depend,'' that is the system they are howling about, and the
loudness of the howl is directly proportioned to the size of the
howler's stake in the matter. The capitalized claims of the food, drug
and cosmetic advertisers upon the creatively psyched Mr.
Throttlebottom's shrinking dollar would probably run into billions if
accurately computed. The stake of the advertising business, otherwise
known as the newspaper, magazine and broadcasting business, is smaller,
but even more indispensable. Newspapers and magazines derive about two
thirds of their income from advertisers, and somewhere between 40 per
cent and 50 per cent of this advertising income is contributed by the
food, drug, proprietary medicine and cosmetic advertisers. Naturally the
publishers and broadcasters and their allies want this creative psyching
of Mr. Throttlebottom to go right on. Naturally, when they contemplate
what would happen if quality standards were systematically introduced
into the codes, they become hysterical and incoherent.

In contrast, the functionalists in Washington have been almost
excessively lucid. In fact, one fears that for all their suavity and
sweet reasonableness, they have made themselves all too clear. For
example, they sponsored the work of a committee, headed by Dr. Lynd,
which has recommended the establishment of a Consumers' Standards Board
under the joint control of the Consumers' Advisory Board of the NRA and
the Consumers' Counsel of the AAA, with a technical director, and a
technical staff of commodity experts and an interdepartmental advisory
committee drawn from Federal Bureaus. The budget asked for provided
\$65,000 for the first year for administrative expenses, plus \$250,000
for research and testing. Dr. Lynd's report quotes that devastating
sentence from the impeccable Hoover's 1922 report as Secretary of
Commerce:

\begin{quote}
The lack of ... established grades and standards of quality adds very
largely to the cost of distribution because of the necessity of buying
and selling upon sample and otherwise, and because of the risk of fraud
and misrepresentation and consequently larger margins of trading.
\end{quote}

Still keeping on the safe, sane and conservative territory of economic
and technical truisms, Dr. Lynd's report goes on to quote a 1930 report
of the Bureau of Standards:

\begin{quote}
Producers are experts in their own commodity fields, but seldom does the
consumer get the full benefit of this knowledge. Under present
conditions this group knowledge is suppressed and the tendency is all
too frequent to give the buyer merely what he asks for.
\end{quote}

Moreover, as F. J. Schlink, director of Consumers' Research, points out
in his ``Open Letter to President Roosevelt,'' ``it is impossible for a
private consumer to secure access to the immensely valuable findings of
the Bureau of Standards, paid for in every major respect by general
taxation of \emph{consumers}.'' In this letter Mr. Schlink urges a
Department of the Consumer, with Cabinet representation and equal status
with other Federal Departments. But even the less sweeping
recommendations of Dr. Lynd's committee were calculated to freeze the
blood of the embattled ad-men, drug men, cosmeticians, vitamin men, etc.
According to Dr. Lynd, the standards promulgated by the Consumers' Board
would not stop at the point at which the commercial standards of the
Bureau of Standards must now stop, \emph{i.e.}, at the type of standards
to which 65 per cent of the industry is ready to agree, but would go on
beyond this to a thoroughly satisfactory set of consumer grades and
labels. Past experience has shown that the official promulgation of
definite consumer standards, even though they go beyond current
practice, operates as a norm to which competitive business tends to
approximate.

It requires but little imagination to see that what is here envisaged is
a fundamental reorganization of distribution in the direction of
function. This would entail a huge deflation of the vested interest of
advertisers, and of the advertising business, in the exploitation of the
American consumer; also huge economies in both production and
distribution.

Even poor old Throttlebottom should be able to see this if there were
any way of getting the word to him. There isn't, for the reason that our
instruments of social communication, the daily and periodical press, the
radio, are in effect the advertising business.

Anybody who wants to fight Mr. Throttlebottom's battles in America had
better hire a hall or write a book. Advertising is the Sacred and
Contented Cow of American journalism. Any irresponsible naturalist who
attempts to lead that cow into the editorial office of any
advertising-sustained American publication is greeted by hoots of
derision. The writer knows, because he has tried. Here are a few typical
hoots:

\begin{quote}
This is an admirable article. Why don't you hire a hall somewhere in the
Bronx and read it to a lot of people?
\end{quote}

\begin{quote}
This subject is the Sacred Cow herself and you know it damned well. Yet
you seem to want old Bossie to commit hara-kiri just because she's not a
virgin. And what would happen to the kiddies then, including yours
truly? Sure, I know: man does not live by bread alone. There is also
butter. I see I've got to teach you the facts of life all over again,
starting with the bees and the flowers. Meanwhile, as one professor of
animal husbandry to another, go sit on a cactus.
\end{quote}

\begin{quote}
Sorry that this article is not adapted to our present needs. Have you
any child's verse?
\end{quote}



% CHAPTER TWENTY-SIX
\chapter[26 \hspace*{1mm} CONCLUSION: Problems and Prospects]{26 CONCLUSION: Problems and Prospects}
\chaptermark{26 CONCLUSION}

\newthought{``There} is nothing the matter with advertising,'' Bruce Barton once
protested, ``that is not the matter with business in general.''

Since advertising is, in the end, merely a function of business
management, Mr. Barton's statement is true, broadly speaking. It might
be added that there is nothing the matter with business that is not the
matter with the professions; also, that there is nothing the matter with
business \emph{and} the professions except that they are obsolete as
practiced under the limiting conditions of an obsolete capitalist
economy. Finally, there is nothing the matter with the machine, with
industry, except that its productive forces cannot be released, and its
dehumanizing effects controlled, under a profit economy.

All these qualified acquittals must be rendered lest the edge of
criticism seems to bear too sharply and too invidiously upon the ad-man.
Invidiousness is, of course, the bread of life in a competitive
capitalist society. It is inevitable, in a fragmented civilization, that
the fragments should quarrel. It is curiously unsatisfying for a man to
be honorable and respectable in the sight of God. No, his sense of
virtue and status must be fortified by the conviction that he is
\emph{more} honorable, more respectable than other men.

I have been greatly amused, more than once, by the complacent na\"ivet\'es
of architects, engineers, doctors, dentists, ``pure'' scientists, and
``objective'' social scientists, who were quite prepared to agree with
me that advertising is a very dirty business. They regarded me,
apparently, as a reformed crook who was prepared, like a mission
convert, to testify concerning the satanic iniquities that I had put
behind me. I have noticed that my replies tend to chill the sympathetic
interest of such people. I say, first, that I have not wholly reformed.
Since I intend to maintain myself economically in an exploitative
economy while it lasts, I expect to enjoy the luxury of integrity in
strict moderation. I say, second, that I am not interested in pouring
invidious moral and ethical comfort into their pots by telling them how
black my particular kettle undoubtedly is.
\clearpage
This invidiousness, these differential judgments, came to the surface
with a rush when, in the aftermath of the 1929 stock market crash, the
magazine \emph{Ballyhoo} was launched. This development, revealing as it
did the catastrophic collapse of ``reader-confidence'' in advertising,
deserves some detailed consideration.

Whereas the stock in trade of the ordinary mass or class consumer
magazine is reader-confidence in advertising, the stock in trade of
\emph{Ballyhoo} was reader-disgust with advertising, and with
high-pressure salesmanship in general. Initially the magazine carried no
paid advertisements. It directed its slapstick burlesque primarily at
the absurdities of current advertising. By October, 1931, its
circulation had passed the million and a half mark and a score of
imitators were flooding the news stands.

The editor of \emph{Ballyhoo}, Mr. Norman Anthony, was formerly one of
the editors of \emph{Life}, and had at various times vainly urged that
humorous weekly advertising medium to bite the hand that fed it by
satirizing advertising. The stock market collapse, and the consequent
reaction against super-salesmanship of all kinds, gave Mr. Anthony his
opportunity, which he seized in realistic commercial fashion.

In style, \emph{Ballyhoo} is a kind of monthly Bronx Cheer, bred out of
\emph{New Yorker} by \emph{Captain Billy's Whizbang}. It expresses the
lowest common denominator of sterile ``sophistication,'' and it is still
successful, although its circulation, at last reports, had dropped to
approximately half of its 1931 peak. And for at least two years it has
taken advertising---advertising designed to sell goods, although adapted
to the pattern of \emph{Ballyhoo}'s burlesque editorial formula.

What had apparently happened was this: the frantic excesses of the
ad-man in the production of customers by ``creative psychiatry'' had
created a new market in which Mr. Anthony established a pioneering
vested interest. This new market consisted of a widespread popular
demand to have advertising burlesqued. Hence \emph{Ballyhoo} became what
might be called an enterprise in tertiary parasitism. In the present
period of capitalist decline, business, as Veblen has shown, parasites
on the creative forces of industry. Advertising, as the writer has tried
to show in this book (c.f. the chapter on ``Beauty and the Ad-Man'')
parasites to a considerable degree on business. \emph{Ballyhoo}, in
turn, parasites on the grotesque, bloated body of advertising.

Mr. Anthony's enterprise is, of course, strictly commercial. When, after
its initial success, the owners of the magazine desired to two-time
their readers in the conventional manner of publishing-as-usual, it is
reported that Mr. Anthony at first objected. But he was over-ruled, and
in due course an advertisement appeared in \emph{Printers' Ink} offering
advertising space in \emph{Ballyhoo}.

Without serious injustice the sales talk of \emph{Ballyhoo}'s
advertising manager may be paraphrased as follows:

``Advertisers: Buy space in \emph{Ballyhoo}. Of course we burlesque you
and shall continue to do so, whether you buy space in the magazine or
not. But these burlesques don't hurt your business. They help it. True,
the saps laugh, but they also buy. Think of it! A mob of a million and a
half saps, laughing and buying! Here they are, packaged and ready to
deliver. How much do you offer?''

After this, the hostility with which many advertisers and many
advertising-supported publications had regarded \emph{Ballyhoo} began to
subside. What if Mr. Anthony's publication was, in a sense, a parasitic
enterprise? He was smart. \emph{Ballyhoo} had got away with it. And
forthwith they proceeded to imitate him.

More and more, advertising began to step out of its part and kid itself.
The single column, cartoon-illustrated campaign for Sir Walter Raleigh
Smoking Tobacco is an early example of this trend. The early copy,
particularly, was an obvious burlesque of the Listerine halitosis-shame
copy. Other advertisers picked up the idea, especially radio
advertisers. Ed Wynn's kidding of Fire Chief gasoline is an excellent
example of the application of burlesque to the production of customers.
More and more, it is the fashion to make radio sales talk allegedly more
palatable by infecting the whole program with burlesque advertising
asides.

Even the preview advertising in the motion picture theatres is beginning
to betray a similar infection. For example, the preview promotion of
\emph{George White's Scandals} consisted of a genuinely amusing satire
of the hackneyed extravagances of motion picture advertising. The Jewish
comedian who played the r\^ole of assistant impresario was sternly
forbidden by Mr. White to use the words ``stupendous,'' ``gigantic''
and``colossal'' in describing the wonders of the new show. Driven to
desperation by this cruel stifling of commercial enthusiasm, the
comedian threatened to shoot himself, and did so. His dying words are:
``George White's Scandals is a stupendous, gigantic, and colossal
show.''

It is contended by the broadcasters, and doubtless also by the movie
producers, that this burlesque sales promotion takes the curse out of
sales talk, and this is probably true to a degree. But the prevalence of
the trend gives rise to certain ominous suspicions. In every decadent
period, satire and burlesque tend to become the dominant artistic forms.
When the burlesque comedian mounts the pulpit in the Church of
Advertising, it may be legitimately suspected that the edifice is
doomed; that it will shortly be torn down or converted to secular uses.

Confirmation of this suspicion appears in the current r\^ole of the
advertising trade press, indeed of the trade press in general. The
writer has had occasion to note that his contributions on the subject of
advertising were not welcomed by consumer publications supported by
advertising. In contrast, the trade press has given space to forthright
radical attacks upon the advertising business both by the writer and by
other critics of advertising such as Dr. Robert Lynd, F. J. Schlink and
others.

This is less surprising than it might seem at first sight. Both
\emph{Advertising and Selling} and \emph{Printers' Ink} have at first
times built their circulations by crusading for ``truth in
advertising,'' the prohibition of bought-and-paid-for testimonials, and
other items of pragmatic advertising morality. Moreover, their readers
want to know what the dastardly enemies of advertising are doing and
thinking, and who is in a better position to tell them than these very
miscreants themselves?

This brings us to a consideration of the agitation for government
grading of staple products, which is the chief threat by which the
advertising business is now menaced. It met and defeated this threat by
deleting the standards clause from the original Tugwell Bill. But the
same threat popped up at every code hearing and in Dr. Lynd's report
urging the establishment of a Consumers' Standards Board, which was
followed by F. J. Schlink's more sweeping demand for a Department of the
Consumer with representation in the Cabinet.

To defeat the raid of the New Deal reformers on the advertising
business, the food, drug, cosmetics and advertising interests
concentrated in Washington a lobby reliably estimated to be from three
to four times as big as any other Washington lobby in history. Yet in
spite of this huge effort the Copeland Bill, after successive revisions
by the Senate Commerce committee, emerged with a number of its smaller
teeth still intact, and conceivably it may be passed by the time this
book appears.

An ironic aspect of the matter was the dual r\^ole played by Senator
Copeland, as broadcaster for Fleischmann's Yeast and Nujol, and as
sponsor of a bill which would, if passed, have definitely limited the
advertising activities of his commercial employers. On March 31st,
Arthur Kallet, Secretary of Consumers' Research, who, with F. J.
Schlink, had ably and energetically defended the consumer interest in
Washington in connection with the Tugwell and Copeland Bills, the
censorship and suppression of the Consumers' Advisory Board, etc.,
signed a circular letter urging the defeat of the emasculated Copeland
Bill and the mobilizing of consumer support of the Consumers' Research
Bill (H.R. 8313). Enclosed was the following statement by the Emergency
Conference of Consumer Organizations.

\begin{quote}
``The Fleischmann Yeast Company, probably to an extent greater than
almost any other national advertiser, would be affected adversely by the
original Tugwell Food and Drug Bill. This bill has been twice revised by
Senator Royal S. Copeland, who is employed by the Fleischmann Yeast
Company at a high fee in connection with its weekly advertising
broadcast.

``The original Tugwell Bill was far too weak to afford adequate consumer
protection, and the Copeland-revised Bill is so much weaker from the
consumer viewpoint that it should be thrown out entirely and new
legislation substituted. This cannot be accomplished unless it is driven
home to the public that there is probably only one man in Congress who
is and has been employed by manufacturers of dubious drug products, and
that this person has, for some curious reason, been placed in charge of
food and drug regulatory legislation. The twice revised bill shows that
Dr. Copeland has taken excellent advantage of the opportunity thus
afforded him to emasculate the original bill.

``The Tugwell Bill was introduced by Dr. Copeland at the last session of
Congress. It was turned over to a sub-committee of the Senate Interstate
Commerce Committee (where consumer-protective legislation certainly does
not belong). The sub-committee consisted of Senator Copeland as
chairman, Senator McNary (a fruit grower who would also be adversely
affected by the bill) and Senator Hattie Caraway. This sub-committee
held public hearings early in December. During the two-day hearings,
Senators Copeland and McNary's antagonism to the best features of the
bill was manifest. Moreover, while representatives of the manufacturers
whose fraudulent and dangerous activities were to be controlled were
given every opportunity to attack the proposed bill, not a single
consumer was given a hearing until within two hours of the close of the
session. Senator Copeland's commercial connections were pointed out by
representatives of Consumers' Research, and new hearings under an
impartial chairman were demanded, but this demand was ignored. It is
noteworthy that at the end of the first day's session, Dr. Copeland went
from the hearings to a broadcasting studio to speak on behalf of
Fleischmann's Yeast.

``The Senator is now and has in the past been employed by other
advertisers who would be adversely affected by the Tugwell Bill, among
them the Sterling Products Company, and the makers of Nujol.

``The broadcasts for Fleischmann's Yeast were begun after the Senator
introduced the Tugwell Bill. For a Senator to accept compensation from
an organization affected by pending legislation is a violation of a
criminal law, if there is any intent to affect the legislation. While
intent cannot in this case be proved, there is clearly a violation of
the spirit of the law.''
\end{quote}

Supplementing this statement, it may be noted that a business
organization known as the Copeland Service, Inc., occupies the office at
250 W. 57th Street adjoining the office of Senator Copeland. The
president of this organization is Mr. Ole Salthe, who in an interview
with the writer on April 5th undertook to describe the nature of this
business. A brief advertising folder issued by Copeland Service, Inc.,
offers the following services:

\pagebreak \noindent \textit{Laboratory Service}

\begin{quote}

Including chemical and bacteriological examinations. Clinical and
biological tests, particularly in relation to the improvement of present
products or the development of new products.
\end{quote}

\noindent \textit{Radio Programs and Lectures}

\begin{quote}

Dr. Royal S. Copeland and a staff of experienced radio speakers are
available to manufacturers of meritorious food and drug products. These
speakers can talk authoritatively on health, food, diet and nutrition,
and insure broadcasts that are interesting and productive of sales.
\end{quote}

\noindent \textit{Labels and Printed Matter}

\begin{quote}

Wide experience in the revision and preparation of labels and printed
matter concerning claims made for food and drug products so as to
conform to municipal, State and Federal Laws.
\end{quote}

\noindent \textit{Special Articles}

\begin{quote}

Relating to health, food, diet and nutrition written in a popular style
for general distribution.
\end{quote}

\noindent \textit{Market and Field Surveys}

\begin{quote}

Staff of experienced investigators in the food and drug industries are
available.
\end{quote}

Dr. Salthe was for twenty years in the employ of the New York City
Department of Health, being director of the division of foods and drugs
when he retired in 1924. In 1925 he became president of Copeland
Service, Inc., with which Royal S. Copeland Jr. is also now connected.
Dr. Salthe declared that aside from broadcasting services for
Fleischmann's Yeast and Stance, makers of Nujol and Cream of Nujol,
Copeland Service, Inc., had no clients. Did I know of any prospects? Dr.
Salthe earnestly denied any connection whatever between the Senator's
sponsorship of the food and drug bill and his r\^ole as a radio artist for
Yeast and Nujol. Copeland Service, Inc., he said, was trying to put on a
sustaining program over N.B.C. stations in which the Senator would give
``constructive educational talks on food buying, including the
mentioning of worthy products.''

Consumers of foods, drugs and cosmetics are invited to decide what is
wrong with this picture and to extract whatever wry amusement they can
from it.

Obviously, neither the emasculated Copeland Bill, nor the original
Tugwell Bill, nor even the Consumers' Research Bill represent a direct
functional approach to the economic and social problems involved,
because no such approach is possible within the framework of the
capitalist economy. All that is possible is to set up more and more
rigid legal and administrative controls over the exploitative activities
of business. The Consumers' Research Bill goes the limit in this
direction. Under its provisions manufacturers of drugs and cosmetics,
and of food products potentially dangerous to health, would be licensed
and bonded; only approved products could be manufactured; all labels and
advertising claims would have to be approved by a board of experts.

The bill is well calculated to freeze the blood of the ad-men, drug men,
vitamin men and cosmeticians. Incidentally, it constitutes an
\emph{excellent reductio ad absurdum} of the whole idea of progress by
reform, capitalist planning, etc. Obviously, it would be much simpler to
socialize pharmacy, medicine and the production and distribution of
foods, and, also obviously, no such socialization could be achieved
without a social revolution.

The most serious challenge to advertisers, and to the advertising
business is, of course, embodied in the agitation for government grading
conducted by the Consumers' Advisory Board, the Consumers' Counsel of
the AAA, and from the outside by Consumers' Research. Here, too, the
maximum result to be attained within the framework of the capitalist
economy would still leave untouched the major contradictions of
capitalism. The agitation is none the less important and fruitful. The
demand for government grading of consumers' goods cannot be successfully
argued against, even from the premises of competitive capitalism. The
promulgation of quality standards and their control would be necessary
government functions in any economy. Significantly, the agitation for
standards has brought to light serious cleavages between the vested
interests affected.

Between the manufacturers and the consumer stand the big distributors,
the mail order houses, the department stores, and the chain stores. They
tend increasingly to sell house products rather than advertised brands.
They represent the more nearly efficient and functional agencies of
distribution under capitalism. They are powerful, and they object to
being squeezed by manufacturers, either through high prices or lowered
standards.

In the course of General Johnson's field day for critics last March,
Irving C. Fox, secretary of the National Retail Dry Goods Association,
in addition to protesting against price rises, revealed that within a
week or two after the codes went into effect, with provisions
prohibiting returns after five days, the quality of merchandise became
much lower than prior to the adoption of these provisions. Chain store,
mail order and department store buyers, and buyers for municipal, State
and Federal departments, have been, in all probability, the most
effective allies of the Consumers' Advisory Board in the fight against
high prices and lowered standards. Not that the consumer standards
movement has got anywhere to date. In one of the reports of the
Consumers' Advisory Board, Prof. Robert Brady testified that

\begin{quote}
``Of the first 220 codes, which cover the most important American
industries, only about 70 contain clauses having anything to do with
standards, grading or labeling. Most of these clauses are absolutely
worthless from the point of view of the consuming interests. In some
cases they are so vague that they permit anything and condone
everything. In some cases they are positively vicious in that they may
be used covertly for price fixing purposes and even practically to
compel the lowering of quality. In four cases, for example, the code
authority is instructed to declare that the giving of guarantees beyond
a certain point is an unfair trade practice, whereas most of the
industries affected have long been accustomed to give and live up to
guaranties far beyond these points.''
\end{quote}

For confirmation of this statement we have only to turn to the
\emph{Journal of Commerce} for April 13, 1934, from which the following
quotation is taken:

\begin{quote}
``Substitution of lower quality for standard products continues on a
substantial scale and prevents consumers from realizing the full import
of price increases that have taken place.

``Retail prices in many lines have been arrived at after study and
experience with mass buying habits. Merchants conclude, therefore, that
they must preserve these established price levels even at the cost of
sacrificing quality, to maintain their physical volume of sales.

``This reasoning has been found so practical and effective in many
instances that manufacturers of branded and trade-marked merchandise
have been adopting the same policy in increasing numbers, it is
reported. In some cases, manufacture of the previous standard quality is
being given up altogether. In some other instances goods meeting the old
specifications are being sold under a new branded name at a higher
price.''
\end{quote}

\vspace{2mm}

\begin{center}

\LARGE{\textit{2.}}

\end{center}

In the light of all these developments, the advertising profession is
bound to contemplate its future with alarm and foreboding. Where
business in general fears the still remote prospect of social
revolution, the advertising business faces deflation through the
inevitable and already well-begun processes of industrial cartelization,
of capitalist ``rationalization,'' which here, as in Italy, Germany, and
in England are bound to enforce a lower standard of living upon the
masses of the population.

At the last convention of the Association of National Advertisers, Dr.
Walter B. Pitkin, Professor of the School of Journalism at Columbia
University, played Cassandra to the assembled ad-men by adding up the
costs of the depression to advertising. ``To begin with,'' said Dr.
Pitkin, ``we are left with between 60 and 64 million people at or below
the subsistence level.'' These are ``extra-economic men'' as far as the
advertising business is concerned. The arts of ``creative psychiatry''
are wasted on them because their buying power is negligible. The average
annual per capita income is down to \$276. If from this is subtracted an
average of \$77 for fixed debt charges, we are left with an average of
per capita expendible income of \$199. Multiply this by four and we have
\$800 as the family average.

But Dr. Pitkin had worse horrors than this to reveal. He believes that
even if we have recovery sufficient to bring about a return of the
pre-depression income levels, this recovery will not be accompanied by
similar spending. Not only are there between 60 and 64 million
``extra-economic Americans---outside the money and profit system,'' but
they don't want to get back into this system. Dr. Pitkin cited examples
of middle class professional people, who, having become adapted to the
shock of having to live on eighteen dollars a week, were content with
what they had; at least they were unashamed, since so many of their
friends were in a similar condition. Dr. Pitkin sums up the problem
confronting the advertising profession as follows:

\begin{quote}
``You have got not merely the problem of scheming to get people's income
up, but you have got the problem of breaking down what you might call a
degenerate type of social prestige, and that is a new problem in
advertising and selling, it is a new problem in merchandising which not
one manufacturer in the United States has yet attempted to face.''
\end{quote}

In passing it might be noted that as a result of the ``scheming to get
people's income up'' as conducted by the industrialists who wrote the
codes, some of whom were in Dr. Pitkin's audience, the volume of goods
sold in February, 1934, was apparently from 6 to 8 per cent less than in
February, 1933.

The assembled advertising men fired questions at Dr. Pitkin. They begged
this earnest savant for some hope, for some way of ``meeting the
issue.'' This is what they got:

\begin{quote}
``We have seen advertising in the last twenty-five years develop from
local commodity advertising, next to trade advertising, then
institutional advertising of a whole domain of businesses....Those are
merely the first movements in a direction toward which we must go a long
way further. You have got to go beyond institutional advertising to some
new kind of philosophy of life advertising. I don't know any better
expression for it than that, but what I mean is that you have got to
sell an enormous number of people in the United States, people of power,
people of intelligence as well as the down-and-outs; you have got to
sell them the conception very clearly of the American standard of living
as we used to think of it, and have a return to it with all that it
implies.''
\end{quote}

If this seems fantastic under the circumstances, I can only point out
that among advertising men in general, Dr. Pitkin is regarded as a
top-leader intellectual. The ad-men were made pretty unhappy on this
occasion, for they couldn't see how they were going to carry out Dr.
Pitkin's recommendations. In effect, what he said was: ``What you need
is more advertising.'' And they knew that before.

Advertising men are indeed very unhappy these days, very nervous, with a
kind of apocalyptic expectancy. Often when I have lunched with an agency
friend, a half dozen worried copy writers and art directors have
accompanied us. Invariably they want to know when the revolution is
coming, and where will \emph{they} get off if it does come.

The other day I encountered a very eminent advertising man indeed,
emerging from an ex-speakeasy. He hailed me jovially and put the usual
question: ``How's the revolution coming?''

``Rather badly,'' I replied. ``Although I think you and your crowd are
certainly doing your bit.''

``You're damned right,'' replied this advertising magnifico. ``I've got
a big white horse. I call him `Comrade.' And when the revolution comes,
I'll be right out in front: `Comrade Blotz'.''

With a sudden chill I reflected that, given the sort of mass moronism
which the advertising business has been manufacturing for these many
years, something of the sort might conceivably happen. What that eminent
ad-man thought of as ``revolution'' was, of course, Fascism. I venture
to predict that when a formidable Fascist movement develops in America,
the ad-men will be right up in front; that the American versions of
Minister of Propaganda and Enlightenment Goebels [\emph{sic}] (the man whom
wry-lipped Germans have Christened ``Wotan's Mickey Mouse'') will be
both numerous and powerful.



\end{document}
