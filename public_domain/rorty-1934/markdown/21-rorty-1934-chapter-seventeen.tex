\textbf{DOI:} 10.21428/3f8575cb.be0fd269

\textsuperscript{\emph{Seventeenth chapter of James Rorty.}}
\textsuperscript{Our Master's Voice: Advertising\emph{. New York: John
Day Company, 1934.}}

\textsuperscript{\href{https://github.com/mediastudiespress/singles/raw/master/public_domain/rorty-1934/pdfs/21-rorty-1934-chapter-seventeen-msp.pdf}{PDF}}
\textsuperscript{\textbar{}}
\textsuperscript{\href{https://github.com/mediastudiespress/singles/raw/master/public_domain/rorty-1934/pdfs/21-rorty-1934-chapter-seventeen-original.pdf}{ORIGINAL
PDF}}

\begin{center}\rule{0.5\linewidth}{\linethickness}\end{center}

\hypertarget{chapter-17}{%
\section{CHAPTER 17}\label{chapter-17}}

\hypertarget{rule-by-radio}{%
\subsection{\texorpdfstring{\emph{RULE BY
RADIO}}{RULE BY RADIO}}\label{rule-by-radio}}

RADIO broadcasting came into the world like a lost child born too soon
and bearing the birthmark of a world culture which may never be
achieved.

Her begetters, the physicists and engineers, didn't know what to make of
the creature. That she was wistful for a world not yet born did not
occur to them. Indeed her begetting was in a sense accidental. They had
been thinking of something else. And as for bringing her up, that was
scarcely their affair. Men of science are notoriously neglectful of
their technical progeny. Observing this neglect an American historian,
Vernon Parrington, was moved to remark that ``science has become the
drab and slut of industry.''

Radio had to belong to somebody. She couldn't belong to nobody. So one
day Business picked her up off the street and put her to work selling
gargles, and gadgets, toothpaste and stocks and bonds. What else could
have happened? Neither art nor education had the prestige or the
resources to command the services of this new instrument of
communication, even if they had had anything important to communicate,
which may be doubted. Government? But in America government was business
and business was government to a far greater degree than in any other
country. So that the development of the ``art and science of radio
broadcasting'' became in America a business enterprise, instead of a
government monopoly as in England and elsewhere in Europe.

About two years ago, Dr. Lee De Forest, one of the piojieers of
electronic science, and by general concession one of trie begetters of
radio, encountered the lost child in his travels and was inexpressibly
shocked:

\begin{quote}
``Why should any one want to buy a radio or new tubes for an old set?''
declaimed the irate inventor, ``when nine-tenths of what one can hear is
the continual drivel of second-rate jazz, sickening crooning by
degenerate sax players, interrupted by blatant sales talk, meaningless
but maddening station announcements, impudent commands to buy or try,
actually imposed over a background of what might alone have been good
music? Get out into the sticks, away from your fine symphony orchestra
pickups, and listen for twenty-four hours to what eighty per cent of
American listeners have to endure! Then you'll learn what is wrong with
the radio industry. It isn't hard times. It is broadcasters'
greed---which is worse. The radio public simply isn't listening in.''
\end{quote}

One wonders why Dr. De Forest should have been so surprised to encounter
this Bedlam on the air. Surely he was familiar with its terrestrial
equivalent. At the moment, in fact he was engaged in fighting the Radio
Corporation of America in the courts.

The vulgarity and commercial irresponsibility of advertising-supported
broadcasting have been greatly complained about. Yet there is a sense in
which the defenders of the American system of broadcasting are right.
Radio is a new instrument of social communication---that and nothing
more. In and of itself it contributed nothing qualitative to the
culture. It was right, perhaps, or at least inevitable that it should
communicate precisely the pseudoculture that we had evolved. Can any one
deny that it did just that? The culture, or pseudoculture, was
acquisitive, emulative, neurotic and disintegrating. Our radio culture
is acquisitive, emulative, neurotic and disintegrating. The ether has
become a great mirror in which the social and cultural anomalies of our
``ad-man's civilization'' are grotesquely magnified. The confusion of
voices out of the air merely echoes our terrestrial confusion.

This confusion becomes particularly apparent when attempts are made to
challenge exploitation of radio by business. In the van of such attacks
are the educators, marching under the banner of ``freedom'' and
``culture'' and invoking such obsolete political concepts as ``States'
Rights.'' Allied with the educators is the Fourth Estate. The appeal is
to ``public opinion,'' expressed and made effective through the
machinery of representative government in a political democracy where
one man's vote is as good as another's. But we have already had occasion
to examine the status of the Fourth Estate and of Education in our
civilization. The press is essentially an advertising business and as
such a part of the central acquisitive drive of the culture. Education
is a formal, traditional function which becomes increasingly peripheral,
decorative and sterile when it adheres to its ideals of disinterested
``objectivity'' and increasingly pragmatic and vocational when it
attempts to relate itself to the acquisitive realities of business as
usual. The press has a vested interest both in the purveying of news and
as a medium of advertising; commercial broadcasting chiselled into the
advertising income of the press and latterly began to compete in the
field of news purveying. Hence the interest of the press in
``reforming'' the radio was strictly competitive and pecuniary in
quality although, of course, the appeal to public opinion was not made
in those terms. It may fairly be alleged that the interest of the
educators was also, and not improperly, a job-holding and job-wanting
interest, although again the appeal to public opinion was not made in
those terms. As for the artists, the writers, poets, dramatists and
critics, who might claim a modicum of service from Radio---well, art is
scarcely an organized and independent estate in an acquisitive society.
The artists tend either to accept service as the cultural lieutenants of
business, to retreat into ivory towers or to become frank
revolutionaries claiming allegiance to a hypothetical future ``classless
culture'' and to the ``militant working class'' also more or less
hypothetical at the present stage of the social process.

The American system is quantitatively successful as judged by the rapid
extension of service---some kind of service---to about 15,000,000
American homes. Today the potential radio audience numbers over
60,000,000. In less than twelve years radio has become a cultural
indispensable and has introduced important new factors into the social
and political process.

The bill for this service is paid first by the set owners. Mr. H. O.
Davis of the Ventura Free Press estimates the annual amount of this
bill, covering the cost of power, new tubes, repairs and replacements of
radio sets, at \$300,000,000. The same authority estimates that the
maximum annual expenditures of all broadcasting stations and networks,
including the operation of enormously expensive advertising sales
departments, is not more than \$80,000,000 and that \$50,000,000 covers
the total expense for the actual production and transmission of all
programs.

The estimates are based on the technical and economic \emph{status quo}
of the ``art and science of radio'' as developed by business. Mr. Davis
undertook a reconnaissance study of this \emph{status quo}, which took
the form of an analysis of a typical day's output transmitted to the
listening public by 206 American broadcasting stations. The following is
quoted from his summarized findings:

\begin{quote}
The average number of interruptions for sales talk during a total of
2365 hours of broadcasting, sustaining programs included, was 5.28 per
hour per station.

The average number of interruptions for sales talks during 1195
program-hours sponsored by advertisers was 9.36 per hour. (Interruptions
for station announcements are not included in these tigures.)

On 1195 hours of programs sponsored by advertisers the sales talks
consumed 174.7 hours, or 14.61 per cent of the total program time,
almost three times the maximum permitted on Canadian programs.

The number of ``spot ads,'' sales talks unaccompanied by entertainment
supplied by the advertiser, totaled 5092 and consumed 57 hours. Canada
prohibits the broadcasting of ``spot ads.''

Out of a total of 2365 broadcasting hours 789 hours, or 32.26 per cent,
were consumed by the playing of phonograph records. ``Electrical
transcriptions''---specially made records---consumed 30 hours or 4.82
per cent of the total broadcasting time.

A little more than 75 per cent of the entire number of hours was devoted
to music of some kind.

All musical programs consumed 1845 hours.

On the day of the survey the 206 stations under observation broadcast
9{{{{\(\frac{3}{4}\)}{{{}{{}{{{{{{}{{{4}}}}{{}{}}{{}{{{3}}}}}{​}}{{{}}}}}{}}}}}}}
hours of symphony-orchestra music, devoting .6 per cent of the total
music time to this type of entertainment. The same number of hours was
filled by the output of so-called haywire or hill-billy orchestras.

Dance orchestras, on the other hand, filled 388 hours or 21 per cent of
the total music-time with jazz.

Other instrumental and vocal music of the popular variety, crooners
included, occupied 1219 hours, two-thirds of the total music-time.

From the quantitative standpoint vaudeville is next in importance to
music. It occupies almost half of the time not given over to music.
Vaudeville includes reviews, jinks, dramatic sketches, jamborees and
similar mixtures of entertainment.

The third largest portion of all broadcasting time is taken up by sales
talks of advertisers, which consume 8.5 per cent of all time on the air,
including both sponsored and sustaining time. In fact, commercial sales
talks consume as much of the broadcasting time as all news broadcasts,
all religious and political addresses and twothirds of the lectures put
together....

On a typical day the average station will devote three-quarters of its
programs to some kind of musical presentation, but the highest class of
symphony-orchestra music will be heard during one-half of one per cent
of the total music time. And when music is on the air, four programs out
of ten will consist of the playing of phonograph records. More than five
times every hour the program will be interrupted for the delivery of a
sales talk lasting in excess of one minute. In addition there will also
be four breaks per hour in the program continuity for station
announcements, making a total of nine interruptions per hour.
\end{quote}

The reader, who is also probably a radio listener, will be able to dub
in the sounds that go with this statistical picture: the bedlamite
exhortations and ecstacies, the moronic coquetries and wise-cracks, the
degenerate jazz rhythms, punctuated by the ironic blats and squeals of a
demon from the outer void known as ``Static.'' An evening spent
twiddling the dials of a radio set is indeed a profoundly educational
experience for any student of the culture. America is too big to see
itself. But radio has enabled America to hear itself, and what we hear,
if closely attended to, supplies important clues to the present state of
the culture.

When we turn to the educators who have struggled for the uplift of radio
what we find is merely further proof of the cultural disintegration
which radio makes audible. It may be said without serious exaggeration
that the problem of the controlling and administering of radio
broadcasting is approximately coextensive with the problem of
controlling the modern world in the economic and cultural interests of
the people who inhabit it. Granted that the radio is socially and
culturally one of the most revolutionary additions to the pool of human
resources in all history---how does one go about integrating it with a
civilization which itself functions with increasing difficulty and
precariousness? Radio is potentially, even to a degree actually, an
instrument of world communication. But the interests of the world
population divide along racial, national and class lines. If these
terrestrial conflicts could be reconciled, presumably we should have
harmony on the air---even conceivably the communication of a world
culture. As it is, the great mirror of the other not only reflects the
conflicts of class and nation and race, but serves to expand the scale
and increase the intensity of these conflicts.

An adequate study of these conflicts, as they are reflected in the
current struggle for control of the microphone, would require a book in
itself. We have space here only for a brief description of what happens
when education and the arts encounter business-as-usual as represented
by the ``American system of broadcasting.''

The records of the Federal Radio Commission show that in May, 1927, when
the present radio law went into effect, there was a total of 94
educational institutions licensed to broadcast. By March, 1931, the
number had been reduced to 49. According to the National Committee on
Education by Radio, 23 educational broadcasting stations were forced to
close their doors between January 1 and August 1, 1930. At present, out
of a total of 400 units available to the United States, educational
stations occupy only 23.16 units, or one-sixteenth of the available
frequencies. In short, educators and educational institutions which
desire to make independent use of the radio as an educational
instrumentality are facing strangulation. They must either fight or
acquiesce in the present trend, which, if continued, will give the
commercial broadcasters complete control of the air---the educators
being invited to feed the Great Radio Audience such education as the
commercial stations consider worth broadcasting, at hours which do not
conflict with the vested interests of toothpastes and automobile tires
or with the careers of such established radio personalities as Amos 'n'
Andy, Phil Cook and Lady Esther.

The militant wing of the educators has chosen to fight and was organized
as the National Committee for Education by Radio. Represented on the
committee are the National Education Association, the National Council
of State Superintendents, the National Association of State
Universities, the Association of College and University Broadcasting
Stations, the National University Extension Association, the National
Catholic Educational Association, the American Council on Education, the
Jesuit Education Association and the Association of Land Grant Colleges
and Universities. Joy Elmer Morgan, editor of the \emph{Journal of the
National Education}, is chairman of this committee. Its work is financed
by the Payne Fund.

Let us turn now to the battalions of the opposition by which these
educational militants are confronted. On June 1, 1931, there were in the
United States 609 licensed stations divided in a ratio of one to sixteen
between the education and the commercial broadcasters. The strongest of
the latter group are affiliated in two great chains with the National
Broadcasting Company and the Columbia Broadcasting Company. N. B. C. is
a one-hundred per cent owned subsidiary of the Radio Corporation of
America, which manufactures radio equipment and pools the patents of
General Electric, Westinghouse and American Telephone and Telegraph.
Obviously the educational militants are facing a closely affiliated
group representing the dominant power and communications interests of
America. N. B. C. and Columbia represent big business, and what does big
business care for education and culture? But big business cares a great
deal, insist the commercial broadcasters, citing their cultural
sustaining programs and their repeated offers of free time on the air to
educators. There is, in fact, a group of educators who have accepted the
existing commercial set-up of broadcasting to the extent at least of
working with it and through it. They too are organized. The National
Advisory Council on Radio in Education is financed jointly by John D.
Rockefeller Jr. and the Carnegie Corporation. Its president is Dr.
Robert A. Millikan and its vice president is Dr. Livingston Farrand,
President of Cornell University.

Two years ago the educational militants were engaged in propaganda for
the Fess Bill, which would have assigned 15 per cent of the broadcast
band to educational broadcasting by educational stations. Latterly they
have turned more and more to the demand for congressional investigation
of radio with the hope that a congressional committee would recommend
government ownership and operation of radio facilities as in England and
more recently in Canada. The conservatives, as represented by the
National Council on Radio in Education, abstain entirely from political
propaganda and lobbying. The objectives of the council, as stated in its
constitution, emphasize fact-finding and fact-dissemination; it
undertakes to ``mobilize the best educational thought of the country to
devise, develop and sponsor suitable programs, to be brought into
fruitful contact with the most appropriate facilities in order that
eventually the council may be recognized as the mouthpiece of American
education in respect to educational broadcasting.'' Officially it still
suspends judgment on the question of private \emph{versus} public
ownership and operation of broadcasting facilities, remarking that, ``as
yet no one is prepared or competent to say whether or not this {[}the
announced educational program of the council{]} will eventually force
the council to discuss the mechanisms necessary for educational
broadcasting and whether their ownership should be in commercial hands,
in the hands of educational institutions, or in the hands of non-profit
co-operative federations, or perhaps in all.'' That statement was
written four years ago and the council is still busy ``finding the
facts'' by rigorously ``objective'' scientific procedures, meanwhile
sponsoring politically innocuous educational broadcasts on free time
contributed by the commercial chains.

In May, 1933, the National Council on Radio in Education held its annual
assembly. The Director of the Council, Mr. Levering Tyson, delivered a
report discussing various activities in broadcasting, research and
publication and urged the establishment of a National Radio Institute.
The writer participated in the discussion of this report and of the
prepared speeches which followed it, which are published in \emph{Radio
in Education}, 1933. I was frankly puzzled by the attitude of the
educators as revealed at this conference.

In this view business, including the business of selling toothpastes,
laxatives, stocks and bonds, etc., by radio is assumed not to be
educative. The advertisers' sales talks (doctrinal memoranda in the
Veblenian terminology) and the jazz, vaudeville and other entertainment
by which they are made more palatable all this is assumed not to be
educative. But obviously this business expresses the central acquisitive
drive of the culture. Obviously it influences the lives of the radio
listeners infinitely more than the relatively microscopic amount of
``education'' which the council had been able to put on the air---more
in all probability than the total output of American class rooms and
lecture platforms. Yet, by definition, it is not ``education,'' which is
conceived of as a meliorative something added to a secular process which
may be profoundly diseducational in that it contradicts and opposes at
practically every point the attitudes and ideals of the educator.

In arguing for a more realistic and more vital conception of the
educational function the writer pointed out that the end result of
American commercial broadcasting, as we have it, is demonstrably
diseducational; that radio advertisers are not interested in educating
the great radio audience in any true sense. What really happens is that
the advertisers are interested solely in promoting the sale of products
and services. Hence they tend to exploit the cultural inadequacies of
the radio audience and its moral, ethical and psychological
helplessness.

At this meeting, Mr. Henry Adams Bellows, LL.D., vice president of the
Columbia Broadcasting Company, made the usual formal offer of free time
on the air to the assembled educators. At the moment it happened that a
group of Corr munist -fellow-travelers,'' organized as the League of
Professional Groups, was conducting a series of public lectures under
the general title ``Culture and Capitalism.'' ices of this group, which
included some well-known teachers and writers, were offered without
charge to Mr. Bellows but, as might have been expected, these radicals
clamored i vain for ``the freedom of the air.''

The issue of censorship was again raised at this meeting after Mr.
Hector Charlesworth, chairman of the Canadian Broadcasting Commission,
had declared that Communists and communist sympathizers were permitted
on the air in Canada. The position of the American commercial
broadcasters, as stated repeatedly by Mr. Bellows and others, is that
the American system provides more effective freedom for minority groups
than the system of government ownership as operated in England and in a
more modified form in Canada. The contention, of course, finds little
support in the experience of Communists and others who recurrently make
application in vain to the educational directors of the major chains.

It is difficult to write about the problem of radio censorship since all
our eighteenth century concepts of ``freedom'' are quite evidently made
obsolete by the technical nature of the instrumentality. Some form of
censorship and some form of international control is necessary. The
domestic problem is simplified under a political dictatorship. Both
Mussolini and Hitler promptly seized complete control of radio upon
assuming power and used it to consolidate and extend their rule. At the
moment Hitler's use of radio, which knows no political boundaries, is
perhaps his strongest weapon in his struggle to bring Austria under the
Nazi hegemony. It is safe to predict that in the next great war, radio
will constitute a major offensive weapon, second only in effectiveness
to the airplane.

Meanwhile, in America, the confusion brought about by our various and
sundry forms of censorship, both overt and concealed, is almost
indescribable. Miss Lillian Hurwitz, in a study of radio censorship
prepared for the American Civil Liberties Union, has no difficulty in
showing that despite the prohibition of censorship embodied in our
present radio law, The Federal Radio Commission ``has so construed the
standard of public interest, convenience and necessity as to enable it
to exercise an indirect censorship over station programs.'' The very
assignment and withdrawal of radio licenses by the commission involves
an indirect censorship.

Meanwhile, as Miss Hurwitz abundantly proves, the stations themselves
are obliged to operate a systematic censorship, if only to protect
themselves against libel suits. They go much further than that, of
course. They not only impose their own conception of the ``public
interest, convenience and necessity'' but their own standards of taste,
morals and political orthodoxy. They protect their own source of revenue
by forbidding radio lecturers to attack radio advertising. When Mr. F.
J. Schlink, director of Consumers' Research, addressed the American
Academy of Political and Social Science on the subject of the New Deal
as it affects the consumer he was cut off the air by the Columbia
Broadcasting Company. Only after the issue was publicly posed by the
resulting newspaper publicity, was he permitted a week later to make the
same speech over Columbia facilities.

What will emerge from this welter of technical and commercial
necessities and political make-believe is quite impossible to predict.
Proposals to unify all communications services under a single government
control are now before Congress with the President's endorsement. A
non-partisan investigation of the broadcasting system has been
repeatedly urged and something of the sort is probably imminent.
Meanwhile, however, it should be pointed out that a tightened control of
the American Telegraph and Telephone Company would perhaps put the
government in a position to audit the wire charges which constitute a
heavy proportion of the overhead of the broadcasting chains. It has been
widely asserted that these charges are excessive; that both the
technical and economic problems of broadcasting could be solved by a
combination of ``wire and wax.'' By ``wax'' is meant wax records which
have been so perfected that an electrical transcription is now
practically indistinguishable from an original studio broadcast. By
``wire'' is meant wire chain hookups, the present cost of which is at
present almost prohibitive except for the two major chains. Then also
there is an assortment of more or less known technical potentialities,
such as wired radio, short wave and micro-wave broadcasting and
television, although the latter, according to competent technicians, is
at present to be classified as a stock-market development rather than an
electronic development. Taken together these various potentialities make
impossible any clear anticipation of what is likely to happen. With this
exception however: the trend of both technical and economic developments
point to the need of centralized control. This will be particularly true
if the Roosevelt Administration is forced, by the failure of the NRA to
increase buying power, to go left in the direction of a functional
reorganization of distribution. As we shall see later, when we come to
discuss the NRA program with respect to advertising, this cannot be
accomplished without a huge deflation of the advertising business,
affecting both the press and the commercial broadcasters.

A significant factor in the situation is, of course, Mr. Roosevelt's
immensely skillful and successful use of radio in building public
support for his administration. On the whole, it would seem only a
matter of time when Mr. Roosevelt, or whoever succeeds him, will be
obliged to say to radio broadcasting, ``You're mine! I need you to help
me rule!'' A faint intimation of this rather probable development
appears in the speech of Federal Radio Commissioner Harold A. LaFount at
the 1933 Assembly of the National Council on Radio in Education already
referred to. Commissioner LaFount said:

\begin{quote}
Educational programs could, and I believe in the near future will, be
broadcast by the Government itself over a few powerful short-wave
stations and rebroadcast by existing stations. This would not interfere
with local educational programs, and would provide all broadcasters with
the finest possible sustaining programs. The whole nation would be
taught by one teacher instead of hundreds, and would be thinking
together on one subject of national importance. Personally I believe
such a plan would be more effective than a standing army.
\end{quote}

The commissioner, who in view of his record, can scarcely be accused of
being unfriendly to the commercial broadcasters, was probably innocent
of dictatorial ideas. Yet his language is, to say the least, suggestive.

\emph{A more detailed discussion of the problem of radio is contained in
the writer's pamphlet ``Order on the Air!'' published by the John Day
Company.}

\begin{center}\rule{0.5\linewidth}{\linethickness}\end{center}

\textsuperscript{\href{https://github.com/mediastudiespress/singles/raw/master/public_domain/rorty-1934/pdfs/21-rorty-1934-chapter-seventeen-msp.pdf}{PDF}}
\textsuperscript{\textbar{}}
\textsuperscript{\href{https://github.com/mediastudiespress/singles/raw/master/public_domain/rorty-1934/pdfs/21-rorty-1934-chapter-seventeen-original.pdf}{ORIGINAL
PDF}}

\textsuperscript{\emph{Seventeenth chapter of James Rorty.}}
\textsuperscript{Our Master's Voice: Advertising\emph{. New York: John
Day Company, 1934.}}
