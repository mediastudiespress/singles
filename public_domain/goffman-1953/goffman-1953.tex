\documentclass[openany,nobib]{tufte-book}

\hypersetup{colorlinks=true,allcolors=[RGB]{97,15,11}}

% For print version
    % 1. change document class \documentclass[twoside,symmetric [?],justified]
    % 2. change full width
        %\begin{fullwidth}
        %\end{fullwidth}
    % 3. get rid of margin notes & implement endnotes | replace footnote \usepackage{endnotes}
        %\endnote{text here}
        %\theendnotes
    % 4. geometry %\usepackage{geometry}
        %\geometry{paperheight=8.5in,paperwidth=5.5in,top=.5in,right=0.5in,left=0.5in,bottom=.75in,includehead,includemp} 
    %5. remove hyperlinks and comment out \hypersetup above      
    %6. change title font sizes
    %7. (figure out how to change the endnote formatting)


%%
% Book metadata
\title[Communication Conduct in an Island Community]{Communication Conduct}
\date{i\kern+0.04emn a\kern+0.04emn I\kern+0.04ems\kern+0.04eml\kern+0.04ema\kern+0.04emn\kern+0.04emd C\kern+0.04emo\kern+0.04emm\kern+0.04emm\kern+0.04emu\kern+0.04emn\kern+0.04emi\kern+0.04emt\kern+0.04emy}
\author[Erving Goffman]{Erving Goffman}
\publisher{a mediastudies.press public domain edition}


%%
% If they're installed, use Bergamo and Chantilly from www.fontsite.com.
% They're clones of Bembo and Gill Sans, respectively.
%\IfFileExists{bergamo.sty}{\usepackage[osf]{bergamo}}{}% Bembo
%\IfFileExists{chantill.sty}{\usepackage{chantill}}{}% Gill Sans

%\usepackage{microtype}

\usepackage{hanging}

% restart footnotes each chapter
\let\oldchapter\chapter
\def\chapter{%
  \setcounter{footnote}{0}%
  \oldchapter
}

%my addtion for toc
\newcounter{oldtocdepth}

\newcommand{\hidefromtoc}{%
  \setcounter{oldtocdepth}{\value{tocdepth}}%
  \addtocontents{toc}{\protect\setcounter{tocdepth}{-10}}%
}

\newcommand{\unhidefromtoc}{%
  \addtocontents{toc}{\protect\setcounter{tocdepth}{\value{oldtocdepth}}}%
}
\usepackage{hyperref}
\usepackage{bookmark}

%%
% Just some sample text
\usepackage{lipsum}

%%
% For nicely typeset tabular material
\usepackage{booktabs}

%%
% Another table package
\usepackage{tabu}

\usepackage{longtable}

%%
% For table spacing
\usepackage{verbatimbox}

%%
% For graphics / images
\usepackage{graphicx}
\setkeys{Gin}{width=\linewidth,totalheight=\textheight,keepaspectratio}
\graphicspath{{graphics/}}

% The fancyvrb package lets us customize the formatting of verbatim
% environments.  We use a slightly smaller font.
\usepackage{fancyvrb}
\fvset{fontsize=\normalsize}

%%
% Prints argument within hanging parentheses (i.e., parentheses that take
% up no horizontal space).  Useful in tabular environments.
\newcommand{\hangp}[1]{\makebox[0pt][r]{(}#1\makebox[0pt][l]{)}}

%%
% Prints an asterisk that takes up no horizontal space.
% Useful in tabular environments.
\newcommand{\hangstar}{\makebox[0pt][l]{*}}

%%
% Prints a trailing space in a smart way.
\usepackage{xspace}

%%
% Some shortcuts for Tufte's book titles.  The lowercase commands will
% produce the initials of the book title in italics.  The all-caps commands
% will print out the full title of the book in italics.
\newcommand{\vdqi}{\textit{VDQI}\xspace}
\newcommand{\ei}{\textit{EI}\xspace}
\newcommand{\ve}{\textit{VE}\xspace}
\newcommand{\be}{\textit{BE}\xspace}
\newcommand{\VDQI}{\textit{The Visual Display of Quantitative Information}\xspace}
\newcommand{\EI}{\textit{Envisioning Information}\xspace}
\newcommand{\VE}{\textit{Visual Explanations}\xspace}
\newcommand{\BE}{\textit{Beautiful Evidence}\xspace}

\newcommand*{\justlastragged}{%
\leftskip=0pt plus 1fil
\rightskip=-\leftskip
\parfillskip=\leftskip
\parindent=0pt}

\newcommand{\TL}{Tufte-\LaTeX\xspace}

% Prints the month name (e.g., January) and the year (e.g., 2008)
\newcommand{\monthyear}{%
  \ifcase\month\or January\or February\or March\or April\or May\or June\or
  July\or August\or September\or October\or November\or
  December\fi\space\number\year
}


% Prints an epigraph and speaker in sans serif, all-caps type.
\newcommand{\openepigraph}[2]{%
  %\sffamily\fontsize{14}{16}\selectfont
  \sffamily\large
  \begin{doublespace}
  \noindent\allcaps{#1}\\% epigraph
  \noindent\allcaps{#2}% author
  \end{doublespace}
}

\usepackage{enumitem}
\setlist[enumerate]{itemsep=0mm}

% Inserts a blank page
\newcommand{\blankpage}{\newpage\hbox{}\thispagestyle{empty}\newpage}

\usepackage{units}

% Typesets the font size, leading, and measure in the form of 10/12x26 pc.
\newcommand{\measure}[3]{#1/#2$\times$\unit[#3]{pc}}

% Macros for typesetting the documentation
\newcommand{\hlred}[1]{\textcolor{Maroon}{#1}}% prints in red
\newcommand{\hangleft}[1]{\makebox[0pt][r]{#1}}
\newcommand{\hairsp}{\hspace{1pt}}% hair space
\newcommand{\hquad}{\hskip0.5em\relax}% half quad space
\newcommand{\TODO}{\textcolor{red}{\bf TODO!}\xspace}
\newcommand{\ie}{\textit{i.\hairsp{}e.}\xspace}
\newcommand{\eg}{\textit{e.\hairsp{}g.}\xspace}
\newcommand{\na}{\quad--}% used in tables for N/A cells
\providecommand{\XeLaTeX}{X\lower.5ex\hbox{\kern-0.15em\reflectbox{E}}\kern-0.1em\LaTeX}
\newcommand{\tXeLaTeX}{\XeLaTeX\index{XeLaTeX@\protect\XeLaTeX}}
% \index{\texttt{\textbackslash xyz}@\hangleft{\texttt{\textbackslash}}\texttt{xyz}}
\newcommand{\tuftebs}{\symbol{'134}}% a backslash in tt type in OT1/T1
\newcommand{\doccmdnoindex}[2][]{\texttt{\tuftebs#2}}% command name -- adds backslash automatically (and doesn't add cmd to the index)
\newcommand{\doccmddef}[2][]{%
  \hlred{\texttt{\tuftebs#2}}\label{cmd:#2}%
  \ifthenelse{\isempty{#1}}%
    {% add the command to the index
      \index{#2 command@\protect\hangleft{\texttt{\tuftebs}}\texttt{#2}}% command name
    }%
    {% add the command and package to the index
      \index{#2 command@\protect\hangleft{\texttt{\tuftebs}}\texttt{#2} (\texttt{#1} package)}% command name
      \index{#1 package@\texttt{#1} package}\index{packages!#1@\texttt{#1}}% package name
    }%
}% command name -- adds backslash automatically
\newcommand{\doccmd}[2][]{%
  \texttt{\tuftebs#2}%
  \ifthenelse{\isempty{#1}}%
    {% add the command to the index
      \index{#2 command@\protect\hangleft{\texttt{\tuftebs}}\texttt{#2}}% command name
    }%
    {% add the command and package to the index
      \index{#2 command@\protect\hangleft{\texttt{\tuftebs}}\texttt{#2} (\texttt{#1} package)}% command name
      \index{#1 package@\texttt{#1} package}\index{packages!#1@\texttt{#1}}% package name
    }%
}% command name -- adds backslash automatically
\newcommand{\docopt}[1]{\ensuremath{\langle}\textrm{\textit{#1}}\ensuremath{\rangle}}% optional command argument
\newcommand{\docarg}[1]{\textrm{\textit{#1}}}% (required) command argument
\newenvironment{docspec}{\begin{quotation}\ttfamily\parskip0pt\parindent0pt\ignorespaces}{\end{quotation}}% command specification environment
\newcommand{\docenv}[1]{\texttt{#1}\index{#1 environment@\texttt{#1} environment}\index{environments!#1@\texttt{#1}}}% environment name
\newcommand{\docenvdef}[1]{\hlred{\texttt{#1}}\label{env:#1}\index{#1 environment@\texttt{#1} environment}\index{environments!#1@\texttt{#1}}}% environment name
\newcommand{\docpkg}[1]{\texttt{#1}\index{#1 package@\texttt{#1} package}\index{packages!#1@\texttt{#1}}}% package name
\newcommand{\doccls}[1]{\texttt{#1}}% document class name
\newcommand{\docclsopt}[1]{\texttt{#1}\index{#1 class option@\texttt{#1} class option}\index{class options!#1@\texttt{#1}}}% document class option name
\newcommand{\docclsoptdef}[1]{\hlred{\texttt{#1}}\label{clsopt:#1}\index{#1 class option@\texttt{#1} class option}\index{class options!#1@\texttt{#1}}}% document class option name defined
\newcommand{\docmsg}[2]{\bigskip\begin{fullwidth}\noindent\ttfamily#1\end{fullwidth}\medskip\par\noindent#2}
\newcommand{\docfilehook}[2]{\texttt{#1}\index{file hooks!#2}\index{#1@\texttt{#1}}}
\newcommand{\doccounter}[1]{\texttt{#1}\index{#1 counter@\texttt{#1} counter}}

% Generates the index
\usepackage{makeidx}
\makeindex

\titleformat{\section}%
  {\normalfont\LARGE\scshape}% format applied to label+text
  {\llap{\colorbox{orange}{\parbox{1.5cm}{\hfill\color{white}\thesection}}}}% label
  {1em}% horizontal separation between label and title body
  {}% before the title body
  []% after the title body

\titleformat{\subsection}%
  {\normalfont\large\scshape}% format applied to label+text
  {\llap{\colorbox{orange}{\parbox{1.5cm}{\hfill\color{white}\thesection}}}}% label
  {1em}% horizontal separation between label and title body
  {}% before the title body
  []% after the title body

%%%% Kevin Goody's code for title page and contents from https://groups.google.com/forum/#!topic/tufte-latex/ujdzrktC1BQ
\makeatletter
\renewcommand{\maketitlepage}{%
\begingroup%
\setlength{\parindent}{10pt}

{\fontsize{24}{24}\selectfont\textit{\@author}\par}

\vspace{2in}{\fontsize{24}{52}{\allcaps{\@title}}\par}

\vspace{0.35in}{\fontsize{24}{52}{\allcaps{\@date}}\par}

\vfill{\fontsize{14}{14}\selectfont\smallcaps{\@publisher}\par}

\thispagestyle{empty}
\endgroup
}
\makeatother


\titlecontents{chapter}%
    [4em]% distance from left margin
    {}% above (global formatting of entry)
    {\contentslabel{2em}\textit}% before w/ label (label = ``Chapter 1'')
    {\hspace{0em}\textit}% before w/o label
    {\qquad\thecontentspage}% filler and page (leaders and page num)
    [\vspace*{0.5\baselineskip}]% after

%%%% End additional code by Kevin Godby

\begin{document}

% Front matter
\frontmatter\pagenumbering{roman}\setcounter{page}{3}

% ONE full title page
\begin{fullwidth}


\maketitle


% TWO copyright page
\newpage

~\vfill
\thispagestyle{empty}
\setlength{\parindent}{0pt}
\setlength{\parskip}{\baselineskip}
\emph{Communication Conduct in an Island Community}, originally deposited in 1953 at the \smallcaps{University of Chicago}, is in the public domain.

\par Published by \smallcaps{mediastudies.press} in the \smallcaps{Public Domain} series

\par Original formatting, spelling, and citation styles retained throughout, with occasional {[}\emph{sic}{]} to indicate an uncorrected error.

\href{http://mediastudies.press}{mediastudies.press} | 414 W. Broad St., Bethlehem, PA 18018, USA

\par New materials are licensed under a Creative Commons Attribution-Noncommercial 4.0 (\href{https://creativecommons.org/licenses/by-nc/4.0/legalcode}{\smallcaps{CC BY-NC 4.0}})

\par \smallcaps{Cover design}: Mark McGillivray | Copy-editing \& proofing: Emily Alexander

\par \smallcaps{Credit for scan}: Internet Archive, \href{https://archive.org/details/GOFFMAN1953CommunicationConductInAnIslandCommunity}{2015 upload}

\par \smallcaps{Credit for LaTeX template}: \href{https://www.overleaf.com/latex/templates/book-design-inspired-by-edward-tufte/gcfbtdjfqdjh}{Book design inspired by Edward Tufte}, by \href{https://ctan.org/pkg/tufte-latex}{The Tufte-LaTeX Developers}

\par \smallcaps{ISBN} 978-1-951399-09-2 (print) | \smallcaps{ISBN} 978-1-951399-10-8 (pdf)

\par \smallcaps{ISBN} 978-1-951399-08-5 (epub) | \smallcaps{ISBN} 978-1-951399-07-8 (pdf)

\par \smallcaps{DOI} \href{https://doi.org/10.32376/3f8575cb.baaa50af}{10.32376/3f8575cb.baaa50af}

\par\textit{Edition 1 published in December 2022}

% THREE dissertation page
\newpage
\thispagestyle{empty}
\setlength{\parindent}{10pt}

\begin{center}
\vspace*{.1in}

{\fontsize{18}{24}\selectfont\textit{The University of Chicago}\par}

\vspace{.5in}{\fontsize{24}{28}\selectfont\allcaps{Communication Conduct in an Island Community}\par}

\vspace{.5in}{\fontsize{20}{24}\selectfont\textit{A Dissertation Submitted to the
Faculty of the Division of the Social Sciences in Candidacy for the
Degree of Doctor of
Philosophy}\par}

\vspace{.5in}{\fontsize{20}{24}\selectfont{Department of Sociology}\par}

\vspace{.5in}{\fontsize{16}{24}\selectfont\textit{By}\par}

\vspace{.5in}{\fontsize{20}{24}\selectfont{Erving Goffman}\par}

\vspace{.5in}{\fontsize{16}{16}\selectfont\textit{Chicago, Illinois}\par}

{\fontsize{16}{24}\selectfont\textit{December, 1953}\par}


\end{center}

\end{fullwidth}

% BLANK PAGE

\newpage
\thispagestyle{plain} % empty
\mbox{}

% TOC

\begin{fullwidth}

\newpage
\thispagestyle{empty}


\vspace*{.1in}

{\noindent\fontsize{32}{24}\selectfont\textit{Contents}\par}

\vspace{.75in}

\setlength{\parindent}{25pt}
{\fontsize{14}{12}\selectfont{\hyperref[ch:The Cradle: Introduction to the mediastudies.press edition]{The Cradle: Introduction to the mediastudies.press edition}\hfill x\par}}

\vspace{.25in}

{\fontsize{14}{12}\selectfont{\hyperref[ch:Introduction]{Introduction}\hfill 4\par}}

\vspace{.5in}

{\noindent\fontsize{20}{24}\selectfont\textit{Part One: The Context}\par}

\vspace{.25in}

{\fontsize{14}{12}\selectfont{\hyperref[ch:Chapter I: Dixon]{Chapter I: Dixon}\hfill 11\par}}

\vspace{.5in}

{\noindent\fontsize{20}{24}\selectfont\textit{Part Two: The Sociological Model}\par}

\vspace{.25in}

{\fontsize{14}{12}\selectfont{\hyperref[ch:Chapter II: Social Order and Social Interaction]{Chapter II: Social Order and Social Interaction}\hfill 23\par}}

\vspace{.5in}

{\noindent\fontsize{20}{24}\selectfont\textit{Part Three: On Information About One’s Self}\par}

\vspace{.25in}

{\fontsize{14}{12}\selectfont{\hyperref[ch:Chapter III: Linguistic Behavior]{Chapter III: Linguistic Behavior}\hfill 31\par}}

\vspace{.25in}

{\fontsize{14}{12}\selectfont{\hyperref[ch:Chapter IV: Expressive Behavior]{Chapter IV: Expressive Behavior}\hfill 35\par}}

\vspace{.25in}

{\fontsize{14}{12}\selectfont{\hyperref[ch:Chapter V: The Management of Information About Oneself]{Chapter V: The Management of Information About Oneself}\hfill 46\par}}

\vspace{.25in}

{\fontsize{14}{12}\selectfont{\hyperref[ch:Chapter VI: Indelicate Communication]{Chapter VI: Indelicate Communication}\hfill 56\par}}


\vspace{.25in}

{\fontsize{14}{12}\selectfont{\hyperref[ch:Chapter VII: Sign Situations]{Chapter VII: Sign Situations}\hfill 60\par}}


\pagebreak

\thispagestyle{empty}

\vspace*{.1in}

{\noindent\fontsize{20}{24}\selectfont\textit{Part Four: The Concrete Units of Conversational\\\noindent Communication}\par}

\vspace{.25in}

{\fontsize{14}{12}\selectfont{\hyperref[ch:Chapter VIII: Introduction]{Chapter VIII: Introduction}\hfill 66\par}}

\vspace{.25in}

{\fontsize{14}{12}\selectfont{\hyperref[ch:Chapter IX: Social Occasion]{Chapter IX: Social Occasion}\hfill 77\par}}

\vspace{.25in}

{\fontsize{14}{12}\selectfont{\hyperref[ch:Chapter X: Accredited Participation and Interplay]{Chapter X: Accredited Participation and Interplay}\hfill 83\par}}

\vspace{.25in}

{\fontsize{14}{12}\selectfont{\hyperref[ch:Chapter XI: Expression During Interplay]{Chapter XI: Expression During Interplay}\hfill 89\par}}

\vspace{.25in}

{\fontsize{14}{12}\selectfont{\hyperref[ch:Chapter XII: Interchange of Messages]{Chapter XII: Interchange of Messages}\hfill 98\par}}

\vspace{.25in}

{\fontsize{14}{12}\selectfont{\hyperref[ch:Chapter XIII: Polite Interchanges]{Chapter XIII: Polite Interchanges}\hfill 105\par}}

\vspace{.25in}

{\fontsize{14}{12}\selectfont{\hyperref[ch:Chapter XIV: The Organization of Attention]{Chapter XIV: The Organization of Attention}\hfill 114\par}}

\vspace{.25in}

{\fontsize{14}{12}\selectfont{\hyperref[ch:Chapter XV: Safe Supplies]{Chapter XV: Safe Supplies}\hfill 119\par}}

\vspace{.25in}

{\fontsize{14}{12}\selectfont{\hyperref[ch:Chapter XVI: On Kinds of Exclusion from Participation]{Chapter XVI: On Kinds of Exclusion from Participation}\hfill 125\par}}

\vspace{.25in}

{\fontsize{14}{12}\selectfont{\hyperref[ch:Chapter XVII: Dual Participation]{Chapter XVII: Dual Participation}\hfill 132\par}}

\vspace{.5in}

{\noindent\fontsize{20}{24}\selectfont\textit{Part Five: Conduct During Interplay}\par}

\vspace{.25in}

{\fontsize{14}{12}\selectfont{\hyperref[ch:Chapter XVIII: Introduction: Euphoric and Dysphoric Interplay]{Chapter XVIII: Introduction: Euphoric and Dysphoric Interplay}\hfill 139\par}}

\enlargethispage{\baselineskip}

\vspace{.25in}

{\fontsize{14}{12}\selectfont{\hyperref[ch:Chapter XIX: Involvement]{Chapter XIX: Involvement}\hfill 141\par}}

\vspace{.25in}

{\fontsize{14}{12}\selectfont{\hyperref[ch:Chapter XX: Faulty Persons]{Chapter XX: Faulty Persons}\hfill 148\par}}

\vspace{.25in}

{\fontsize{14}{12}\selectfont{\hyperref[ch:Chapter XXI: Involvement Poise]{Chapter XXI: Involvement Poise}\hfill 156\par}}

\vspace{.25in}

\pagebreak

\thispagestyle{empty}

\vspace*{.1in}

{\fontsize{14}{12}\selectfont{\hyperref[ch:Chapter XXII: On Projected Selves]{Chapter XXII: On Projected Selves}\hfill 171\par}}

\vspace{.25in}

{\fontsize{14}{12}\selectfont{\hyperref[ch:Chapter XXIII: The Management of Projected Selves]{Chapter XXIII: The Management of Projected Selves}\hfill 190\par}}


\vspace{.5in}


{\fontsize{14}{12}\selectfont{\hyperref[ch:Interpretations and Conclusions]{Interpretations and Conclusions}\hfill 198\par}}

\vspace{.25in}

{\fontsize{14}{12}\selectfont{\hyperref[ch:Bibliography]{Bibliography}\hfill 210\par}}

\end{fullwidth}



% THE CRADLE: INTRODUCTION TO THE MEDIASTUDIES.PRESS EDITION
\chapter[THE CRADLE: INTRODUCTION TO THE MEDIASTUDIES.PRESS EDITION]{The Cradle: Introduction to the mediastudies.press\\ edition}
\label{ch:The Cradle: Introduction to the mediastudies.press edition}
\chaptermark{THE CRADLE: INTRODUCTION TO THE MEDIASTUDIES.PRESS EDITION}

\emph{\smallcap{\LARGE{Yves Winkin}}}

\vspace{0.5in}

\newthought{Erving Goffman's dissertation}
\marginnote{\href{https://doi.org/10.32376/3f8575cb.21a77b51}{doi}}is the Rosetta stone for his entire work,
which, as time goes by, appears to be more and more groundbreaking. Why
was \emph{Communication Conduct in an Island Community} not published
earlier? Why did commentators not exploit it more systematically? Why
did Goffman himself not try to have it published? All those questions
are unanswerable today. But here is the gem. Not much of a frame is
needed to appreciate it---only the circumstances of Goffman's fieldwork
in the Shetlands, and then some highlighting. When Goffman defended his
dissertation in the early summer of 1953, his committee members were
none too pleased, according to legend. Seventy years later, the piece
appears luminous, extraordinarily mature, as if Goffman were already a
fully professional sociologist from day one.\footnote{Special thanks to
  Wendy Leeds-Hurwitz for her graceful editing job.}

\hypertarget{maybe-he-was-a-spy-goffman-in-unst-19491951}{%
\section[Maybe He Was a Spy: Goffman in Unst
(1949--1951)]{\texorpdfstring{Maybe He Was a Spy: Goffman in Unst\\
(1949--1951)\footnote{I am relying on data collected in Unst in 1988
  (August 25--September 2), and the two papers derived from that brief
  stint of fieldwork: Winkin, ``Goffman à Baltasound, 1949--1951,''
  \emph{Politix} 3--4 (1988): 66--70; Winkin, ``Baltasound as the
  Symbolic Capital of Social Interaction,'' in \emph{Erving Goffman},
  ed. Gary A. Fine and Gregory W. H. Smith (London: Sage, 2000),
  193--212. For a recent analysis of the dissertation, see Karl Lenz,
  ``Dissertation: \emph{Communication Conduct in an Island Community,}''
  \emph{Goffman Handbuch,} ed. Karl Lenz and Robert Hettlage (Berlin: J.
  B. Metzler, 2022), 257--65.}}{Maybe He Was a Spy: Goffman in Unst (1949--1951)}}\label{maybe-he-was-a-spy-goffman-in-unst-19491951}}

``Out of the blue,'' mumbled Charlotte Mouat, when I asked her about
Erving Goffman's arrival in Baltasound in December 1949.\footnote{Interview
  on August 31, 1988, with the help of her nephew, Tony Mouat, and a
  home nurse.}

Baltasound is the main community on the island of Unst, all the way to
the north end of the Shetland archipelago. Charlotte Mouat was the owner
and the manager of the Springfield Hotel, which served as Goffman's
headquarters during his fieldwork period, between 1949 and 1951. I spent
nine days on the island in late August 1988, trying to meet as many
people as possible who remembered him almost forty years later. Many
did, actually, but they still could not figure out why he came and
stayed for so long near them. Yes, near them, not with them.

There are plenty of small mysteries to unravel. Why would a foreign
young man come to Unst in December and ask for a room at the hotel?
There was absolutely nothing to do on the island at that time of the
year. The weather was awful; the sun barely showed up between 9:00 a.m.
and 4:00 p.m. There were no birds to watch, and the newly revived
``Up-Helly-Aa,'' the Viking-looking celebration, was not due before the
end of February. Maybe he was a spy---so apparently suggested some
people, according to Mary Priest, who was one of the waitresses at the
Springfield Hotel.\footnote{Mary Priest, interview by the author, August
  26, 1988.} After all, the island had been strategic during World War
II, with thousands of soldiers in barracks, many boats and submarines in
the harbor, and refugees from Norway.

In addition, the young man, always in a khaki army jacket with many
pockets and in boots laced up to the knees, just walked around a lot.
What could he be doing all day? He lived for some time in the annex of
the hotel and then bought a small cottage from Wally Priest, a few
hundred yards from the Springfield. Priest was engaged to Mary and
needed the money to buy a new house in time for the wedding. As Miss
Sutherland, the eighty-something daughter of the former local policeman,
wrote to me:

\begin{quote}
He, as I remember, was not a very big person; somehow one felt that he
was rather aloof, a kind of solitary figure in a world of his own. He
was said to be an ``anthropologist'' who was writing a book on the
subject. This was a kind of deterrent to those of us who weren't very
well educated. One often wondered if he wasn't lonely, sitting by
himself in that bare little cottage but his need for privacy would be
respected.\footnote{L. J. Sutherland, email message to author, August 4,
  1988.}
\end{quote}

\noindent Goffman was thus a mystery for many islanders. But this is also a
mystery for the biographer: Why Unst, and more specifically, why Unst in
December? There are partial answers, or at least plausible answers. One
has to do with Lloyd Warner, who supervised Goffman's master's thesis in
sociology at the University of Chicago. It happened that he had become
friendly with anthropologist Ralph Piddington when they were both doing
fieldwork in Australia in the late 1920s. Piddington moved to the
University of Edinburgh in 1946 and envisaged the creation of a
department of anthropology. By 1949 there was money available for a
graduate student to do tutorials and to conduct fieldwork in the
Shetland Islands. Warner suggested the job to Goffman, who applied and
got it. But how to explain that Goffman decided to move beyond the
United States for his dissertation, the only one of his cohort to do so?
One may only conjecture that the idea of an island ethnography, à la
Malinowski in the Trobriand Islands or Radcliffe-Brown in the Andaman
Islands, was seductive. There may also have been some pressure on
Warner's part, who probably wanted to repeat a ``community study'' in
Europe, a few years after the work of his students Solon Kimball and
Conrad Arensberg in Ireland.\footnote{Conrad Arensberg and Solon
  Kimball, \emph{Family and Community in Ireland} (Cambridge, MA:
  Harvard University Press, 1940). In the preface, Warner wrote: ``The
  book that has grown out of their experience there is an excellent
  contribution to our ever-growing body of knowledge of the communities
  of the world. From such a knowledge we may sometimes expect a
  comparative science of the social life of man'' (ix).} And a third
possible reason: Goffman may have wanted some time away from Chicago, in
spite of the fact he was engaged to Angelica Schuyler Choate, a master's
student in human development at the University of Chicago. But she could
visit him in Edinburgh---she certainly had enough money on her own to
afford such a trip.\footnote{Yves Winkin, ``Life and Work of Goffman,''
  in \emph{Goffman Handbuch,} ed. Karl Lenz and Robert Hettlage (Berlin:
  J. B. Metzler, 2022), 3--11.}

The fact that he arrived in Unst at a bad time of the year, if there is
any good one on that rough island, may well have to do with his duties
as an instructor in Edinburgh. He completed his term before taking the
boat from Aberdeen. Between December 1949 and May 1951, a stretch of
eighteen months, he totaled twelve months on the island. The remaining
six months were probably spent in Edinburgh for his classes, and in
London, where he visited his old partner Liz Bott, who was then
completing her doctorate at the London School of Economics.

There are more mysteries, but there are at least partial answers
available to solve them. Could we suggest that Goffman arrived on the
island with a clear mandate from Warner to undertake a community study?
We can answer positively on the basis of three leads. First, the
psychological toolkit: As Goffman became comfortable with the hotel's
two maids, he often asked them to look at ``drawing sets and tell him
what we saw in them,'' as Mary Priest told me.\footnote{Priest,
  interview.} At first, she hesitated, because she did not know what he
would write about the answers, but finally she went ahead. (``I was told
it came from Germany. Do you think it is true?'') Clair Auty (\emph{n\'ee} Anderson) was even
more explicit: ``All too often'' Goffman would give her ``these stupid
cards'' with blots and spots of colors and ask her to tell him what she
saw. ``He told me I had a vivid imagination.'' There were also
``triangles and circles,'' and he would ask her what the odd one was.
``He said I had a fair brain if I was not so idle.''\footnote{Clair
  Auty, interviews by the author, August 1988. The administration of
  the psychology tests was a recurrent theme in the many conversations I
  had with Clair Auty.} Clearly, Goffman was applying what he had
learned for his master's thesis, during which he had asked fifty
upper-middle-class women to take the Thematic Apperception Test (TAT).
He worked under the supervision of Warner, who always considered
psychological tests an integral part of the anthropologist's tool kit.

Then there were constant queries about social class. According to my
informants, Goffman was ``obsessed'' with social class on the island,
and kept asking them questions about the ``gentry'' and the differences
they perceived between the gentry and themselves---that is, the
commoners, especially the ``crofters'' (small farmers). He wanted to
know everything about the Saxbys and the Spences, the two upper-class
families of the island. ``He made you talk more than he did,'' as Claire
Auty put it. Goffman was clearly adopting Warner's approach to society,
although he apparently never developed strong ties with members of the
gentry, except for the Guthries, the new doctor and his wife. As Tony
Mouat, Charlotte's nephew, drove me by the Saxby house, he noticed the
older Saxby on his bike and stopped to ask him about Goffman. I wrote in
my diary: ``But Saxby, apparently, only met him at New Year's Eve and
had nothing more to say. See how a filter appears: people I can/I can't
see.''

Finally, although this is anecdotal, Warner delivered the Munro Lectures
(ten of them!) at the University of Edinburgh in April--May 1950, on
``The Application of Social Anthropology to Contemporary
Life.''\footnote{The lectures were turned into a book, \emph{Structure
  of American Life}, published in 1952 by University of Edinburgh Press
  and republished in 1953 in an augmented version by University of
  Chicago Press under the title \emph{American Life: Dream and Reality}.}
In the memoir that his widow, Mildred Hall Warner, published many years
later, there is no mention of Goffman being in the audience.\footnote{Mildred
  Hall Warner, \emph{W. Lloyd Warner Social Anthropologist} (New York:
  Publishing Center for Cultural Resources, 1988). Chapter XII (pp.
  163--74) is devoted to the Munro Lectures.} But maybe he was, and
maybe he met and discussed with Warner his fieldwork in progress. It is,
at least, certain that Goffman did not meet Radcliffe-Brown (who may
also have been in the audience). Recall his famous dedication of
\emph{Relations in Public}: ``Dedicated to the memory of A. R.
Radcliffe-Brown whom on his visit to the University of Edinburgh in 1950
I almost met.''\footnote{Goffman, \emph{Relations in Public:
  Microstudies of the Public Order} (New York: Basic Books, 1971).}

Now where do we go from here? Clearly, Goffman's dissertation is not
another community study à la Warner, despite hints that his original
intention was to write in that vein. So, we must ask: What happened? An
interpretation may be offered: Goffman made necessity a virtue---and in
the process provided the groundwork for a new subfield in sociology.

In 1984, the American sociologist Michael Schudson scrutinized
\emph{Communication Conduct} and stressed the fact that Goffman spent
most of his observational efforts on three sites: the hotel, the
billiards, and the ``socials.'' Those were selective places: Few local
people would ever visit the hotel, the pool room was restricted to men,
and the activities of the socials were either ``by invitation only'' (as
with whist) or by age only (as with a dance). So Schudson concluded:

\begin{quote}
So far as one can tell from Goffman's dissertation, he had no intimate
contact with crofter family life. There is no indication that he made
any friends; there is no special ``informant'' that anthropologists have
often discussed with such feeling. Indeed, Goffman is intentionally
anti-anthropological. He claims that he was not doing a study ``of a
community'' but a study ``in a community.'' But putting aside a concern
for the macrosociological features of the community he studied and
putting aside any interest in features that distinguished this community
from others, he inadvertently wound up examining primarily the social
interactions that most resembled interactions in the most detached and
impersonal settings of modern life.\footnote{Michael Schudson,
  ``Embarrassment and Erving Goffman's Idea of Human Nature,''
  \emph{Theory and Society} 13 (1984): 640.}
\end{quote}

\newpage\noindent Schudson could not have known that Goffman cultivated a close
relationship with one ``special informant,'' the postman James (Jimmy)
Johnson, who was sixty-six in 1950. He was Claire Auntie's uncle.
According to his nephew Bob Anderson, Claire's brother, he was
well-travelled and well-read; he knew local dialects and folklore. He
was often seen walking around the island with Goffman.\footnote{Bob
  Anderson, interview by the author, August 30, 1988.}

But Schudson is right about Goffman having ``no intimate contact with
crofter family life.'' Indeed, he never lived with a family; he lived by
himself in a tiny cottage and took his meals at the nearby hotel. But
could he have done otherwise? Schudson suggests that Goffman selected
those three observational sites. In response, I would like to suggest
that these were the only three semi-public places that were open to him,
along with the local store, the church, and the reading room (adjacent
to the billiards room). He could also hang around the harbor and a few
other public places. But private houses were off limits, except for an
occasional meal, and people were most taciturn. He was stuck. Goffman
explained at the very beginning of his dissertation that he tried to
participate in as many situations and social occasions as possible. He
also explained that he did not conduct formal interviews, did not employ
questionnaires, and did not use tape-recorders or ``motion-picture
cameras,'' all methods which would have been out of place. As he put it:
``In order to observe people off their guard, you must first win their
trust.''\footnote{Goffman, \emph{Communication Conduct in an Island
  Community}, 5. Hereafter \emph{CC}. Page numbers refer to the original
  manuscript.} This is all quite fine and respectable, but one could be
forgiven for suspecting that this was a rationalization of an impossible
situation. There was no way he could have deployed a Warner-inspired
community study, which would have involved home visits, questionnaires,
and in-depth interviews. So, instead, he turned to the one thing
available: ``conversational interaction,'' as the first sentence of the
dissertation says. For this, he ``just'' needed to look and listen
nearby, and to write notes down once back at the cottage. The islanders'
taciturnity led him to make the best out of skimpy materials.
Ultimately, the results turned out to be, quite simply, revolutionary.
The dissertation is incredibly innovative. Goffman's entire oeuvre
cannot be properly understood if one does not read the dissertation
first. It provides the matrix for the following ten books.

\hypertarget{birth-of-a-sociology}{%
\section{Birth of a Sociology}\label{birth-of-a-sociology}}

The radical nature of Goffman's dissertation begins with the title.
There is no subtitle, and no reference to a theoretical frame or
methodology. It was likely the first time ever that ``communication''
was used for a dissertation title in sociology, and probably one of the
first times the term was used in the singular in a dissertation in any
discipline. Moreover, ``communication,'' which was often used as a
modifier in those days, was here associated with ``conduct.''
``Conduct'' was not regularly used in the social sciences in the early
1950s---and still isn't today. What is most remarkable in the
association between ``communication'' and ``conduct'' is that the very
meaning of communication is transformed from a means to an activity. At
the time, the dominant usage of ``communication'' (again, most often
used in the plural) referred to means, first to physical facilities,
such as roads and railways, and later to media, especially the press and
broadcasting. As Raymond Williams has pointed out, this use (of
``media'') ``is not settled before mC20 {[}mid-twentieth
century{]}.''\footnote{Raymond Williams,~\emph{Keywords: A Vocabulary of
  Culture and Society} (Oxford: Oxford University Press, 1976), 72.} But
the singularization of the term was not completed until the early
1970s\footnote{Elvira M. Arcenas, ``\,`Communication' in the Making of
  Academic Communication'' (PhD diss., University of Pennsylvania,
  1995), \url{https://repository.upenn.edu/dissertations/AAI9543043}.}
and may be related, in parallel with notions such as ``society,''
``culture,'' or ``language,'' to a progressive
conceptualization.\footnote{George W. Stocking, \emph{Race, Culture, and
  Evolution: Essays in the History of Anthropology} (New York: Free
  Press, 1969), 195--233.} In any case, ``communication conduct'' sets
the tone: Goffman intends to break with then-current vocabulary and ways
of thinking. A source of inspiration must have been the 1951 book by
Jurgen Ruesch and Gregory Bateson, \emph{Communication: The Social
Matrix of Psychiatry}, explicitly mentioned in Chapter II.\footnote{Jurgen
  Ruesch and Gregory Bateson, \emph{Communication: The Social Matrix of
  Psychiatry} (New York: W. W. Norton \& Company, 1951). See \emph{CC},
  40.} Ruesch and Bateson used ``communication'' to refer to
``interpersonal'' and ``intrapersonal'' exchanges of messages. That was
congruent with Goffman's approach to communication as interaction
practice.\footnote{See Wendy Leeds-Hurwitz and Yves Winkin, ``Goffman
  and Communication,'' in \emph{The Routledge International Handbook of
  Goffman Studies}, ed. Michael Hviid Jacobsen and Greg Smith (Abingdon,
  UK: Routledge, 2022), 184--94.}

The redeployment of ``interaction'' is certainly the major theoretical
breakthrough of the dissertation. At the time, the notion of interaction
was frequently used in social psychology, especially in small group
research, with the underlying assumption that it was somehow the
equivalent of ``mild, short-term, mutual impact.'' Papers were full of
``feedbacks'' and ``effects.''\footnote{See H. J. Leavitt and R. A. H.
  Mueller, ``Some Effects of Feedback on Communication,'' \emph{Human
  Relations} 4 (1951): 401--10. Seen again in P. Hare, F. F. Borgatta,
  and R. F. Bales, eds., \emph{Small Groups: Studies in Social
  Interaction} (New York: Knopf, 1955), 414--23.} Goffman rejected all
that from page one of his dissertation:

\begin{quote}
The research was not designed to determine thoroughly or precisely the
history of any interaction practice, the frequency and place of its
occurrence, the social function which it performed, or even the range of
persons among whom it occurred.
\end{quote}

\noindent Many years later, he would return to his rejection of social psychology
and its use of ``interaction'':

\begin{quote}
My belief is that the way to study something is to start by taking a
shot at treating the matter as a system in its own right, at its own
level, and, although this bias is also found in contemporary
structuralism, there is an unrelated source, the one I drew on, in the
functionalism of Durkheim and Radcliffe-Brown. It is that bias which led
me to try to treat face-to-face interaction as a domain in its own right
in my dissertation, and to try to rescue the term ``interaction'' from
the place where the great social psychologists and their avowed
followers seemed prepare to leave it.\footnote{Erving Goffman, ``A Reply
  to Denzin and Keller,'' \emph{Contemporary Sociology} 10, no. 1
  (1981): 62.}
\end{quote}

\noindent Note how Goffman repeats the formula ``in its own right'': It seems to
be borrowed from Durkheim's own lexical fetish: ``society as a reality
\emph{sui generis.}'' Indeed, what Goffman is after is the reality
\emph{sui generis} of interaction. This is (again) clear from the very
first page of his dissertation: He wants to ``build a systematic
framework useful in studying interaction throughout our society.'' And
here comes the crucial justification: ``As the study progressed,
conversational interaction came to be seen as one species of social
order.'' We can observe the birth of the ``interaction order'' (the
title of his 1982 American Sociological Association presidential
address) right here.\footnote{Published as Goffman, ``The Interaction
  Order,'' \emph{American Sociological Review} 48, no. 1 (1983): 1--17.}
Goffman is indeed taking interaction away from social psychology and
reinstalling it ``as a domain in its own right'' within sociology,
thanks to the idea that it is ``one species of social order.'' For
thirty years, Goffman pursued the same argument, but he often buried it
under other explorations. Here, in the dissertation, it is crystal
clear.

\hypertarget{such-a-luminous-piece-of-work}{%
\section{Such a Luminous Piece of
Work}\label{such-a-luminous-piece-of-work}}

The dissertation is structured in five parts: description of the
fieldwork site (one chapter), presentation of the theoretical model (one
chapter)---and three more parts of theoretical developments (five, ten,
and six chapters, respectively). It is highly unusual for a dissertation
to devote only one chapter to the description of the field and
twenty-two chapters to theoretical elaborations. And those chapters bear
titles as odd as ``Indelicate Communication,'' ``Safe Supplies,'' or
``Involvement Poise.'' Imagine how puzzled the members of the committee
must have been. Goffman was surely aware of the ``false impression''
that his unorthodox presentation might produce since he tried to correct
it in the ``Introduction'':

\begin{quote}
. . . the beginning of each chapter is phrased in terms of a general
discussion of particular communication concepts, and only later in each
chapter are field data introduced. This stylistic device is employed as
a way of rendering the data easy for use in the development of a general
communication framework. In consequence, a false impression is sometimes
given that the field data has been brought in as an afterthought, merely
to illustrate concepts earlier arrived at. I should like to make it
quite clear that the terms and concepts employed in this study came
after and not before the facts.\footnote{\emph{CC}, 9.}
\end{quote}

\noindent This is not the place to offer a full analysis of the dissertation. Let
me just highlight a few outstanding innovations.

Once ``Dixon'' (the pseudonym for Baltasound) is presented, Goffman
bluntly offers his ``sociological model,'' which consists of a
nine-point parallel between social order and social interaction---or
interaction order, as he called it later in the dissertation. This is
the first and only time in his entire work he so explicitly shows how
the interaction order proceeds from the social order. This is also one
of the rare occurrences of a clearly acknowledged debt to Talcott
Parsons (\emph{The Social System}, 1951) and, even more strangely, to
Chester Barnard (\emph{The Functions of the Executive},
1938).\footnote{Talcott Parsons, \emph{The Social System} (Glencoe, IL:
  Free Press, 1951); Chester Barnard, \emph{The Functions of the
  Executive} (Cambridge, MA: Harvard University Press, 1938).} It may be
worth noting that, at the end of the 1920s, both Parsons and Barnard
attended the Harvard seminar of Lawrence Henderson, a physiologist who
was fond of the work of Vilfredo Pareto.\footnote{L. J. Henderson,
  \emph{Pareto's General Sociology: A Physiologist's Interpretation}
  (Cambridge: Harvard University Press, 1935).} Between Pareto,
Henderson, Barnard, Parsons, and Goffman, there is a common denominator:
the notion of system, loosely defined as a set of interdependent
elements. For Parsons and Goffman, regulating mechanisms maintain the
system in operation. The system may be society as whole or a single
interaction. As systems, society and interaction work the same way: When
they are under pressure, coping mechanisms intervene to maintain the
balance, i.e., to maintain social order or interaction order. Goffman is
thus going to introduce, almost in passing, notions that will be
developed later, notably in \emph{Presentation of Self in Everyday Life}
(1959), such as embarrassment and working acceptance.\footnote{Goffman,
  \emph{The Presentation of Self in Everyday Life} (New York: Anchor,
  1959).} He is also going to offer his vision of interaction ``not as a
scene of harmony but as an arrangement for pursuing a cold
war.''\footnote{\emph{CC}, 40.} As Parsons would not have dared to say,
an open war is too costly, but a cold war is affordable---here Goffman
is already adumbrating his work of the 1960s on strategy, as discussed
with conflict specialists such as Thomas Schelling, Albert Wohlstetter,
and Daniel Ellsberg.

Once his model has settled, Goffman turns to conversational interaction.
This is stunning. Goffman is the only Chicago sociologist who finally
paid his due to the linguist and anthropologist Edward Sapir. As Everett
Hughes later admitted, Chicago interactional sociology never managed to
deal empirically with language as social behavior.\footnote{See Stephen
  O. Murray, \emph{Group Formation in Social Science} (Edmonton:
  Linguistic Research, 1983), 79, 243 (letter from Hughes to H. D.
  Duncan).} Moreover, Goffman foreshadows the sociolinguistics movement
of the 1960s, led by Dell Hymes and John Gumperz. It is even more
surprising to call into being an ethnography of speaking in the early
1950s, when the field was dominated by a descriptive linguistics which
claims that texts indicate their own structures, a position leaving
little room for the speakers or the context.\footnote{Dell Hymes,
  ``Linguistics: The Field,'' in \emph{International Encyclopedia of the
  Social Sciences}, ed. David L. Sills (New York: Macmillan, 1968), 356.}
Goffman was the only sociologist of his generation to break away from
such an attitude and, instead, provide an early argument for an analysis
of language as interaction. Only in his 1964 paper ``The Neglected
Situation'' would he return to language, and then again, much later, in
\emph{Forms of Talk}, his last book (1981).\footnote{Goffman, ``The
  Neglected Situation,'' \emph{American Anthropologist} 66, no. 6
  (1964): 133--36; Goffman, \emph{Forms of Talk} (Philadelphia:
  University of Pennsylvania Press, 1981).} But he opened the field
thirty years ahead of everyone.

Goffman does not reduce his scope to verbal exchanges. He discusses in
Chapter IV the old notion of ``expressive behavior,'' quoting Darwin and
Sapir as well as Gordon Allport and Philip Vernon (\emph{Studies in
Expressive Movements}, 1933) and, more surprisingly, the French
psychologist Charles Blondel (\emph{Introduction à la psychologie
collective}, 1928).\footnote{Gordon W. Allport and Philip E. Vernon,
  \emph{Studies in Expressive Movements} (New York: Macmillan, 1933);
  Charles Blondel, \emph{Introduction à la psychologie collective}
  (Paris: A. Colin, 1928).} He swiftly navigates between the traditional
characterization of gestures as spontaneously revealing the ``soul'' and
the culturally oriented approach stressing the learned, and thus
intentional, aspects of the gestural repertoire:

\begin{quote}
. . . the member is obliged to obey the rules of expression, once
learned, in a sufficiently automatic and unselfconscious way so that
observers will in fact be partly justified in their assumption that the
emotion conveyed to them is a dependable index of the actor's emotional
state.\footnote{\emph{CC,} 59--60.}
\end{quote}

\noindent This is the necessary platform for developing the notion of ``impression
management'' that will be so central in \emph{Presentation of Self}. By
then, citations to the background literature will have disappeared. But
it is worth stressing that Goffman had already laid the theoretical
groundwork in his dissertation. And he did not need the metaphor of the
theatre to build his case---a ``dramaturgical model'' that seduced
superficial commentators for years.

When we read Chapter V, ``The Management of Information about Oneself,''
we realize that the quintessential Goffman we all know, the sociologist
deciphering ``members''---be they members of a rural community, of a
mental hospital, or of a casino---is already fully evident in his
dissertation, at age thirty. Intentional linguistic behavior and
supposedly impulsive expressive behavior are combined to produce
interactions based on mutual ``exploitations'' of information. The
receiver exploits the expressive behavior of the sender ``as a source of
impression about him,'' while the sender ``may attempt to exploit the
fact that this exploitation occurs and attempt to express himself in a
way that is calculated to impress the recipient in a desired
way.''\footnote{\emph{CC}, 85--86.} We all know this, yet it is still
impressive to find it so clearly expressed in Goffman's dissertation.

It is also impressive to find Goffman shifting, by the end of Chapter
VII, from an exploitative view of interaction to a ritual view: ``Even
more than being a game of informational management, conversation
interaction is a problem in ritual management.''\footnote{\emph{CC},
  103.} This is the model later developed in ``The Nature of Deference
and Demeanor'' (1956).\footnote{Goffman, ``The Nature of Deference and
  Demeanor,'' \emph{American Anthropologist} 58, no. 3 (1956): 473--502.}
Here, in the dissertation, only two pages are needed to reshuffle
Durkheim's \emph{Elementary Forms of the Religious Life}.\footnote{Émile
  Durkheim, \emph{The Elementary Forms of the Religious Life} (London:
  G. Allen \& Unwin, 1915).} But they are sufficient to break away with
the rather paranoid vision of social life developed in the
dissertation's first hundred pages. Goffman then distills this ritual
view of interaction in the remaining two hundred pages.

The fourth part of the thesis consists of ten chapters dissecting ``the
concrete units of conversational communication.'' It would be
anachronistic to speak of ``conversation analysis,'' but Goffman is
almost there already. In pages replete both with data collected in situ
and ad hoc concepts, he is going to build not a sociology of language
but a sociology of speaking. Along the way, an enduring theme would
develop: the idea that conversationalists must be present, physically
but also psychologically. Goffman speaks of ``accredited participation''
(Chapter X) and discusses the ways to cover ``improper lulls'' and to
display an appropriate ``attention quota'' (Chapter XI). Chapter XV
deals with ``safe supplies,'' i.e., ``stores of messages that persons
can fall back upon when they are in a position of having to maintain
interplay and yet not having anything to say'':\footnote{\emph{CC}, 213.}
small talk, joking, or just watching the open fire. Chapter XVI is
devoted to the ``kinds of exclusion from participation,'' and Chapter
XVII borrows the notion of ``away'' from Bateson and Mead's
\emph{Balinese Character}: ``The participant keeps his face more or less
in a position to convey attention signs to the speaker, but his thoughts
and eyes turn inward or come to focus on some object in the
room.''\footnote{\emph{CC}, 232--33; Gregory Bateson and Margaret Mead,
  \emph{Balinese Character: A Photographic Analysis} (New York: New York
  Academy of Sciences, 1942).}

All in all, those chapters lead to the notion of ``involvement,''
developed in the dissertation's fifth and last part, but also in several
later papers, such as Chapter III of \emph{Behavior in Public Places}
(1963) or Chapter 10 of \emph{Frame Analysis} (1974).\footnote{Goffman,
  \emph{Behavior in Public Places: Notes on the Social Organization of
  Gatherings} (New York: Free Press, 1963); Goffman, \emph{Frame
  Analysis: An Essay on Organization of Experience} (Cambridge, MA:
  Harvard University Press, 1974).} One may venture to say that the
notion is one of those secret keys that make Goffman's entire work more
intelligible.

Involvement is the interface between the exploitative view of
interaction developed in the dissertation's early chapters and the
ritual view developed in the later chapters. In order to be ``euphoric''
(i.e., fluid), an interaction has to be a mix of calculation and
deference, Goffman says in Chapter XVIII. Calculation without deference,
and deference without calculation, will only lead to a dysphoric
interaction, to the point of rupture. To quote the final words of
Chapter XIX, precisely titled ``Involvement'':

\begin{quote}
If rules of tact are followed, often boredom sets in. If rules of tact
are broken, often embarrassment sets in. Apparently, a fundamental
source of involvement consists of the slight infraction of tactful
rules; either the infraction is committed in an unserious way or care is
taken to bend the rule but not break it.\footnote{\emph{CC}, 257.}
\end{quote}

\noindent The notion of involvement leads to Goffman's concept of \emph{self},
which is crucial to understanding all of his work. In the last two
chapters of his dissertation, he elaborates the idea of ``projected
selves.'' Goffman does not simply say that participant A wants to
project a certain image of herself into other participants. He also does
not say that participant A fits into a predefined role needing to be
accomplished (as with a role in a play). To the contrary, the
participant produces a situational self, produced through their
involvement in the interaction:

\begin{quote}
At the moment of coming together, each participant---by his initial
conduct and appearance---is felt by others to ``project'' a self into
the situation. . . . The participant may be non-committal and
indefinite; he may be passive, and he may act unwittingly. None the
less, others will feel that he has projected into the situation an
assumption as how he ought to be treated and hence, by implication, a
conception of himself.\footnote{\emph{CC}, 300.}
\end{quote}

\noindent In other words, A's self is actually what B thinks A projects into the
situation. And they are going to maintain the initial understanding of
their projected selves: ``If the minute social system formed by persons
during interplay is to be maintained, the definition of the situation is
not to be destroyed.''\footnote{\emph{CC}, 302.}

Goffman goes on to describe precisely the sorts of situations which ran
temporarily out of control during his stay in Dixon. Those are the funny
pages of the dissertation.\footnote{\emph{CC}, 304--27.} They are all
based on some misunderstanding of the situation by one of the
participants, which is followed by embarrassment on the parts of all
those involved. Embarrassment would later appear in Goffman's work as an
important concept, as in his ``Embarrassment and Social Organization''
(1956).\footnote{Goffman, ``Embarrassment and Social Organization,''
  \emph{American Journal of Sociology} 62, no. 3 (1956): 264--71.}

The dissertation's final chapter appears, from its anodyne title
(``Interpretations and Conclusions''), to promise mere summary. What is
more interesting is the subtitle, ``The Interaction Order.'' The phrase
would reappear as the title of Goffman's final contribution, his 1982
ASA presidential address.\footnote{Goffman, ``The Interaction Order.''}
It was as if he had wanted to come full circle, as if he had been
consistent and systematic throughout his intellectual career. Actually,
he was far more consistent than many commentators have recognized. Many
Goffman scholars take it as given that his work jumps from one topic to
another uncommonly often. As this all too brief reading of the
dissertation has shown, he was, on the contrary, quite orderly in
planting seeds to be nurtured later. Orderliness turned out to be a key
word for both his vision of the world and for his work. A final quote
sums it up:

\begin{quote}
In this study I have attempted to abstract from diverse comings-together
in Dixon the orderliness that is common to all of them, the orderliness
that obtains by virtue of the fact that those present are engaged in
spoken communication. All instances of engagement-in-speech are seen as
members of a single class of events, each of which exhibits the same
kind of social order, giving rise to the same kind of social
organization in response to the same kind of normative structures and
the same kind of social control.\footnote{\emph{CC}, 345.}
\end{quote}

\hypertarget{return-from-the-field}{%
\section{Return from the Field}\label{return-from-the-field}}

In May 1951 Goffman left Baltasound for Edinburgh, where he completed
his contract with the university, which ran until that fall. At some
point during the summer, he was joined by Angelica, or ``Sky,'' as she
was called by her friends and relatives. They went to London and then to
Paris, where they probably stayed for several months on rue de Lille, in
the apartment that Sky's aunt (her mother's sister) kept as a
pied-à-terre while she lived in Italy.\footnote{Born Mary Phelps Jacob,
  nicknamed ``Polly,'' she was given the sweet name of Caresse by her
  husband, Harry Crosby, in 1924. He committed suicide with his mistress
  in 1929 and his widow continued to run the Black Sun Press they had
  created together.} Goffman started to draft his dissertation in
Paris---``the best place to write,'' as he put it many years later to
Dean MacCannell.\footnote{Dean MacCannell, interview by the author, May
  13, 1987.} Together they returned to Chicago in the spring of 1952.

In May, Goffman completed his ``PhD Thesis Statement,\emph{''} a
fifteen-page document overview of the dissertation to come.\footnote{See
  Goffman, ``Draft of PhD Thesis Statement'' (unpublished manuscript,
  May 1952), 3--4,
  \url{http://cdclv.unlv.edu/ega/documents/eg_thesis_statement_52.pdf}.}
By that time everything was in place, except that the focus was on the
self rather than the interaction order, \emph{viz} the tentative title:
``The Social Rules Regarding Expression of Oneself to Others.'' Although
Goffman explains that he went to Unst to ``study the rules of conduct
which islanders adhered to while engaged in social interaction with one
another,'' he mentions that ``after some data had been collected and
partly analysed {[}\emph{sic}{]}, it became apparent that a shift in
original emphasis would be required.''\footnote{Goffman, ``Draft of PhD
  Thesis Statement,'' 3--4.} Is that the shift from a Warner-like
community study to the study of a terra incognita? It would be difficult
to say, but at least there is, for the first and last time, the
recognition that a shift happened at some point. Goffman would go on to
work on the dissertation for almost a year, but he at least took time to
get married to Angelica Schuyler Choate in July 1952.

A year later he publicly defended his dissertation. Here is another
small mystery: While the ``PhD Thesis Statement'' mentions Warner,
Everett Hughes, and Daniel Horton as advisors, the dissertation's first
page thanks Warner, Horton, and Anselm Strauss. Hughes has disappeared.
What happened? Was he mad at Goffman for his disruptive dissertation,
which literally hid field data under the rug? There is actually a
simpler explanation for the mystery: Hughes was in Germany at the time,
as a visiting professor at the University of Frankfurt. He may have been
happy to escape from Goffman's defense in this most legitimate way, but
we will never know.

The rule at the time was that the entire department faculty, and not
just the committee, could ask questions during a dissertation defense.
Apparently, there were rough questions. As Strauss has described, ``I
remember it was a warm spring day, and Goffman under the usual heavy
attack had good control of himself, such good control that when a bead
of perspiration rolled down his brow to his nose, he did nothing about
it!''\footnote{Anselm Strauss, letter to the author, October 13, 1985.}

Goffman certainly got his degree, but for his entire life he maintained
a grievance against his committee, who ``did not understand what I was
after,'' as he put it to me.\footnote{Goffman, discussion with the
  author, April 1980.} While he never published his dissertation as a
book, he exploited it throughout his career, not so much in
\emph{Presentation of Self} (1959) as in \emph{Behavior in Public
Places} (1963). His final book, \emph{Forms of Talk} (1981), may be seen
as another late result of his conversational investigations, as I have
argued above. There is now a major endeavor to be undertaken: to recast
Goffman's oeuvre in the light of his now-published dissertation. A new
generation of Goffman scholars is invited to get to work.

% BLANK PAGE

\newpage
\thispagestyle{plain} % empty
\mbox{}

\mainmatter\pagenumbering{arabic}\setcounter{page}{1}

% REPEAT TITLE PAGE
\newpage
\begin{fullwidth}
\maketitle




% THREE dissertation page
\newpage
\thispagestyle{empty}
\setlength{\parindent}{10pt}

\begin{center}
\vspace*{1in}

{\fontsize{18}{24}\selectfont\textit{The University of Chicago}\par}

\vspace{.5in}{\fontsize{24}{28}\selectfont\allcaps{Communication Conduct in an Island Community}\par}

\vspace{.5in}{\fontsize{20}{24}\selectfont\textit{A Dissertation Submitted to the
Faculty of the Division of the Social Sciences in Candidacy for the
Degree of Doctor of
Philosophy}\par}

\vspace{.5in}{\fontsize{20}{24}\selectfont{Department of Sociology}\par}

\vspace{.5in}{\fontsize{16}{24}\selectfont\textit{By}\par}

\vspace{.5in}{\fontsize{20}{24}\selectfont{Erving Goffman}\par}

\vspace{.5in}{\fontsize{16}{16}\selectfont\textit{Chicago, Illinois}\par}

{\fontsize{16}{24}\selectfont\textit{December, 1953}\par}


\end{center}



% Simmel quote
\newpage
\thispagestyle{empty}




~\vfill
\noindent{\fontsize{13}{20}\selectfont\textit{\ldots{} there exists an immeasurable number of
less conspicuous forms of relationship and kinds of interaction. Taken
singly, they may appear negligible. But since in actuality they are
inserted into the comprehensive and, as it were, official social
formations, they alone produce society as we know it. To confine
ourselves to the large social formations resembles the older science of
anatomy with its limitation to the major, definitely circumscribed
organs such as heart, liver, lungs, and stomach, and with its neglect of
the innumerable, popularly unnamed or unknown tissues. Yet without
these, the more obvious organs could never constitute a living organism.
On the basis of the major social formations---the traditional subject
matter of social science---it would be similarly impossible to piece
together the real life of society as we encounter it in our experience.
Without the interspersed effects of countless minor syntheses, society
would break up into a multitude of discontinuous systems. Sociation
continuously emerges and ceases and emerges again. Even where its
eternal flux and pulsation are not sufficiently strong to form
organizations proper, they link individuals together. That people look
at one another and are jealous of one another; that they exchange
letters or dine together; that irrespective of all tangible interests
they strike one another as pleasant or unpleasant; that gratitude for
altruistic acts makes for inseparable union; that one asks another man
after a certain street, and that people dress and adorn themselves for
one another---the whole gamut of relations that play from one person to
another and that may be momentary or permanent, conscious or
unconscious, ephemeral or of grave consequence (and from which these
illustrations are quite casually chosen), all these incessantly tie men
together. Here are the interactions among the atoms of society. They
account for all the toughness and elasticity, all the color and
consistency of social life, that is so striking and yet so
mysterious.}

\vspace{.5in}

\begin{center}

\singlespacing{\fontsize{10}{1}\selectfont{Georg Simmel, \emph{The Sociology of Georg Simmel}, trans. and ed.\\ Kurt\ B. Wolff (Glencoe, Ill.: The Free Press,
  1950), pp. 9--10.}}


\end{center}

\end{fullwidth}

% MAINMATTER


% INTRODUCTION
\chapter[INTRODUCTION]{Introduction}
\label{ch:Introduction}
\chaptermark{INTRODUCTION}

\newthought{This is a report}
\marginnote{\href{https://doi.org/10.32376/3f8575cb.778546a7}{doi}}on a study of conversational interaction. It is based
on twelve months of field work carried on between December, 1949, and
May, 1951, in a small community in Great Britain.\footnote{I am very
  grateful to the Department of Social Anthropology and the Committee on
  Social Science Research of the University of Edinburgh, who financed
  and sponsored the study, and to Professors W. Lloyd Warner, Donald
  Horton, and Anselm Strauss of the University of Chicago, who served as
  thesis advisors} The community is located on a small island, one in an
isolated group of islands that supports a subsistence rural economy.

The aim of the research was to isolate and record recurrent practices of
what is usually called face-to-face interaction. The research was not
designed to determine thoroughly or precisely the history of any
interaction practice, the frequency and place of its occurrence, the
social function which it performed, or even the range of persons among
whom it occurred. The project was concerned with a more elementary
question, namely, the kinds of types of practices which occurred.

I was especially concerned with those social practices whose formulation
and analysis might help to build a systematic framework useful in
studying interaction throughout our society. As the study progressed,
conversational interaction came to be seen as one species of social
order. The social order maintained through conversation seemed to
consist of a number of things: the working in together of messages from
different participants; the management by each participant of the
information about himself conveyed in his messages; the show of
agreement maintained by participants; and other things.

I settled down in the community as an American college student
interested in gaining firsthand experience in the economics of island
farming. Within these limits, I tried to play an unexceptional and
acceptable role in community life. My real aim was to be an observant
participant, rather than a participating observer.

During the full period of study, an effort was made to guide
participation in two directions. First, I tried to participate in as
many as possible of the different situations in which members of the
community entered into face-to-face interaction with another (e.g.,
meals, types of work, schooling, shop-loitering, weddings, parties,
socials, funerals), and to do this with as many different sets of
participants as possible. The aim here was to ensure experience with the
full range of variation. Secondly, I participated regularly and for an
extended period of time in a few daily and weekly social occasions, each
time with the same set of participants. Here the aim was threefold: to
minimize for at least some islanders the inhibitory effect of having a
stranger present; to ensure observation of the kinds of interaction
crises which occur infrequently but which throw light on conduct which
occurs regularly; and, finally, to ensure observation of occasions in
which factors usually present were for some reason absent, thus
providing a makeshift way of experimentally varying one factor while
keeping others constant. My attempt to ensure range and depth of
participation was facilitated by two fortunate social facts: much of the
recreational life in the community is formally organized as an
undertaking open to any resident of the island, and there is a strong
tradition of neighborly assistance with farm tasks, whereby offers to
help are readily accepted and give to the helper a traditional right to
eat a day's meals with those he has helped.

During the first few months of the study, it was possible for me to take
a running record at large-scale gatherings, noting down verbatim bits of
conversation and gestures, and sketching ecological movements, as these
events occurred. Later, and especially in the case of small-scale
gatherings, recording of this kind would have been considered offensive,
improper, and inconsistent with relationships I had established. It then
became necessary to record daily observations at the end of each day or
at moments of privacy during the day.

While in the field, I tried to record happenings between persons
regardless of how uninteresting and picayune these events seemed then to
be. The assumption was that all interaction between persons took place
in accordance with certain patterns, and hence, with certain exceptions,
there was no \emph{prima facie} reason for thinking that one event was a
better or worse expression of this patterning than any other event. I
want to confess, however, that I found indiscriminate recording very
difficult to do, especially in situations where a written note of the
event could not be made until some hours after the event had occurred.
There was a constant temptation to record only those events which found
at the time a neat place in my conceptual organization, either as
conforming or radically disconfirming instances. (Thus, as the
conceptual organization changed, so also did the kinds of facts that
were recorded.) There was also a temptation to concentrate on those
vents which struck me as bizarre, dramatic, or entertaining---events
likely to mark a reader feel that the data were interesting and
meaningful. Mechanical devices such as tape recorders and motion-picture
cameras, or rigid techniques such as time-sampling, would have provided
a desirable check on these recording biases. These corrective devices,
however, were not practical for social, economic, and technical reasons.

When a spate of interaction is observed in a small isolated community,
it is possible for the observer to place the event in a wider context of
information concerning the occupation, socio-economic status, friendship
and kinship ties, and personality characteristics of the participants.
The observer obtains part of this information by direct observation,
part by properly timed offhand inquiry, and part of it is thrust upon
him by members of the community in order that he may participate without
awkwardness in conversational interaction which makes no sense without
such information. Therefore it was not necessary to carry on formal
interviews, or to employ schedules and questionnaires in a
systematically way, in order to collect basic social facts. Nor were
these formal techniques employed in order to collect data that might
bear directly upon conversational interaction. Members of the community
seemed to have few notions of a well-formulated kind concerning social
interaction, and I came to feel, by the hints conveyed to me when I
first settled down in the community, that residents would not readily
accept as a friend and neighbour someone who asked formal questions
about interactions or someone who showed an unnatural interest in
matters of the kind. In order to observe people off their guard, you
must first win their trust. Had the island culture been the kind in
which it is possible for outsiders to ask odd sorts of questions, I
still could not have employed questionnaires because I did not know
about interaction then, either from my own experience or from the
literature then available, to ask the right questions. In order to learn
what the right questions were, I had to become taken for granted by the
community to a degree and in a way that made it unsuitable for me to ask
these questions. Interviewing was carried on however, on matters related
to the history of the community and to its civic and economic
organization, these being matters which the islanders felt were proper
subjects for interviews. And interviewing was carried on wherever and
whenever questions could be disguised as the ordinary curiosity of an
ordinary outsider.

I personally witnessed almost all the behavior and events described in
this report. There was, therefore, no need to make use of the
sophisticated techniques employed by students who study what people do
by carefully analyzing what they say they do. However, I cannot prove
that any event recorded in my field notes had, for those who
participated in it, the subjective significance and meaning that I claim
it had. I cannot even prove that any particular event had the outward
objective form that I attribute to it. In order to ensure that a wide
range of interactive situations were observed in their natural contexts,
and in order to ensure that some interactions were observed deeply and
intimately, as an ordinary participant would observe them, it was
necessary to sacrifice other kinds of assurances and controls.
Nevertheless, a reasonable number of checks upon observation did seem
available.

By being present with some---and only some---of the participants before
and after an observed interaction occurred, it was possible to confirm
and disconfirm my own interpretations and reactions by asking leading
questions and by conversations of the preparatory and post-mortem kind.
Confirmation and dis confirmation were also obtained by participants in
the kind of furtive communication which occur during an
interaction---communication of the kind that ordinarily allows
participants to convey secretly an unofficial running comment and
judgment on the proceedings in which they are officially involved.

Further, I was allowed to participate informally to the degree to which
islanders could rely on me to observe correctly what was occurring in
the interaction. Errors on my part were corrected by means of informal
sanctions administered by members of the community themselves; correct
observation was rewarded by increasing permission to participate
informally and by increasing capacity to know what was likely to happen
next and to react appropriately. To participate in interaction without
causing others to feel embarrassed and ill at ease requires that one
exercise, almost unthinkingly, constant tact and care concerning the
feelings of others; to exercise this discretion it is necessary to
perceive correctly the indications others give of what they are feeling.

Also, the study was concerned with communication; unlike factors such as
attitudes, motives, allegiances, etc., there is a sense in which this
factor cannot function at all unless the meaning intended by the actor
is similar to the meaning that his observers place upon his acts.

Finally, a constant check upon observations was provided by the
informational conditions that prevail in a small isolated community. The
observations made during a particular interaction could be placed into
and checked against a context of information concerning the social
reputation of each of the participants, their momentarily inactive
social roles, and---since most islanders played out the full circle of
their social relationships within the geographical confines of the
island---the other kinds of interactions in which they participated. In
fact, the availability of this background information, coupled with the
relatively wide range of interaction that occurs in a community,
provided the two reasons for seeking an isolated community as a
convenient place in which to study social interaction.

\newpage While these several checks upon observation were available, it was not,
of course, claimed that statements made in this study have the kind of
reliability that is to be found in counts that are made of durable
physical objects that other students can go back and recount. Since this
study was concerned with the kinds of things that occurred, and not with
more advanced problems such as measures of frequency or intensity, the
observational technique employed seemed adequate for the purpose.

The framework developed here attempts to cover a range of data
systemically and uniform all. This means that the preliminary terms have
been designed to lay the foundation for terms that come later, and
therefore that these preliminary terms may have very little interest in
their own right. This also means that special terms have had to be given
to types of events which are almost but not quite covered by terms
existing in the literature already. The effort to be systematic has also
caused me to make formal and ponderous explications of notions that form
part of common sense understanding of social events. I would like to
apologize in advance for these sources of irritation, but I do not see
how a current study of interaction can be made without first defining
one's terms. However, while there is an inescapable need to define one's
terms, it was often not possible, it must be admitted, to do this in a
satisfactory way, or to refrain in certain places from falling back on
common sense language.

This report attempts to exclude information which might positively
identify the community in which the study took place. All names have
been changed; sources of historical and statistical information have not
been identified. This is not a study \emph{of} a community; it is a
study that occurred, \emph{in} a community, of behavior with which no
living person ought to be publicly identified.

The study falls into five parts. The first part consists of a brief view
of the social life of the community, with special reference to certain
recurrent situations for which a relatively extensive interaction record
was kept. Here an attempt is made to provide a context for some of the
events described later, while at the same time not prejudicing the
anonymity of the community. The second part outlines, very tentatively,
a conceptual model for viewing interaction as a form of social order.
The third part deals with the management of information-about-self. Part
four deals with interaction units. The final part deals with conduct of
persons while engaged in conversation. The ratio of substantive material
to analytical discussion is low in Part Two and increases with each
succeeding part.

Except for the introductory chapters (I and II), the beginning of each
chapter is phrased in terms of a general discussion of particular
communication concepts, and only later in each chapter are field data
introduced. This stylistic device is employed as a way of rendering the
data easy for use in the development of a general communication
framework. In consequence, a false impression is sometimes given that
the field data has been brought in as an afterthought, merely to
illustrate concepts earlier arrived at. I should like to make it quite
clear that the terms and concepts employed in this study came after and
not before the facts. The framework of terms presented in this study was
developed in order to identify regularities observed in the
communication conduct of the islanders, or in order to make explicit the
assumptions which seemed to underlie some of these identifications of
regularities.

% Part One

\newpage
\thispagestyle{empty}
\begin{fullwidth}

\begin{center}
\vspace*{3in}

{\fontsize{35}{24}\selectfont{Part One}\par}

\vspace{1in}

{\fontsize{35}{24}\selectfont\textit{The Context}\par}

\end{center}

\end{fullwidth}


% CHAPTER I: DIXON
\chapter[CHAPTER I: DIXON]{Chapter I: Dixon}
\label{ch:Chapter I: Dixon}
\chaptermark{CHAPTER I: DIXON}

\newthought{More than a hundred}
\marginnote{\href{https://doi.org/10.32376/3f8575cb.4b831ce5}{doi}}miles off the coast of Britain there is a cluster of
islands containing about twenty thousand inhabitants. These persons are
supported by a poor economy of small-scale sheep farming and fishing.
Less than ten per cent of the five hundred square miles of land on the
islands is under cultivation, and, except for home-knitting for a luxury
market, almost nothing is manufactured. Until recently, the population
had been declining very rapidly. The policy of the national government,
for various reasons, has been a protective one, helping to maintain
British standards of living by means of agricultural subsidies, statutes
governing the rental price of small holdings, and an extremely high
\emph{per capita} payment to the islands for upkeep of required social
services.

The persons who live on the islands are drawn together by a distinctive
dialect, a rich cultural heritage, and what amounts to a thousand years
of shared historical identity and development. The name for the cluster
of islands---let us call it Bergand---is the name that an inhabitant of
any one of the particular islands in the cluster is likely to identity
himself by. Bergand has been under British rule for only three
centuries. Until the last war, respectable Englishmen thought of Bergand
as a source of seamen and servants, and the islands enjoyed---along with
many other clusters of people in Britain---the status of a subordinate
minority group. These factors making for distinctiveness are, of course,
reinforced by the natural barrier of water between the mainland of
Britain and the islands. In many ways, then, Berganders form a society
unto themselves.

A fourth of the population of Bergand is concentrated in one town,
hereafter called Capital City, which is located on the largest island in
the cluster. There is a twice-weekly steamboat contact between the
mainland of Britain and Bergand, as well as daily air service. These
contacts with the outer world are tunneled through Capital City, and all
formal lines of communication on the islands also lead into this point.
Capital City is also the center for institutions which serve the whole
cluster of islands, and in general it has something of the ethos and
something of the role of a national capital. Fashions travel outward
from this town to all the islands in the cluster; people on the road to
success or retirement travel in the other direction.

The island on which the study took place is a rectangular piece of rock
nine miles long and four miles wide; it is covered by a thin skin of
poor soil. The end link in a chain of islands, it is cut off from its
only neighbor by a channel of fast water a mile wide. The island is
linked with Capital City by a thrice-weekly boat service and a
thrice-weekly ferry-overland service.

The typical farm holding an the island consists of five or ten acres
under intensive cultivation, a similar number of acres of improved
grassland, and hill grazing-rights for fifty or sixty sheep. Subsistence
holdings of this kind may be called crofts. The average crofter has four
or five cows and a score of ponies. The island grows not quite enough
grass to feed the stock and not quite enough vegetables to feed the
inhabitants. Some milk has to be imported for the school lunches. The
principal sources of cash income are typical for the island cluster: the
export of sheep and cattle tor slaughter; the export of raw wool,
hand-knitted goods, and work ponies; government payments in the form of
agricultural subsidies, pensions, and unemployment relief. The size of
individual holdings is limited by government policy---policy that is
apparently designed to encourage land cultivation by individual family
units. There are only three agricultural holdings on the island that
make use of a full-time hired hand.

There are about three hundred dwelling units in use on the island. The
division is based on ecological clustering, trade area, and conscious
identification. Each community is centered in a fan-shaped way around a
nucleus of service institutions. Each nucleus or service center is
located on a part of the coast line that can serve as a harbor, and
contains a community hall, a post office, a school, one or two churches,
three or four stores, and a relatively dense grouping of houses. The
three center points of service form a line, not a triangle, because of
the narrow shape of the island. This study took place in the middle
community, hereafter called Dixon. The communities lying to the north
and south of Dixon will be called, respectively, Northend and Southend.
For their size these communities are probably the most isolated in
Britain.

Fifty years ago there were additional foci of settlement. There is some
evidence that some of these contained local concentrations of extended
kin. Today these settlements can be clearly seen in Northend, where
economic and social consolidation is not yet complete. In general, a
rapid shift of internal population is bringing persons closer and closer
to the three centers of service, so that now most persons live within
two miles of one of them.

In certain ways the center of service in Dixon is a center for the other
two communities as well. The only usable freight pier on the island is
located in the long narrow bay that serves as Dixon's harbor. Coal and
gasoline supplies for the island are located at this pier and delivered
from it. The only bakery on the island is attached to the principal
Dixon store. This store is of the ``general'' kind; it is the largest on
the island and to some extent provides an informal social center for all
three communities. The island's chief business family, its sole
practising doctor, and its resident ``squire'' all live in Dixon. A
school that will serve all the secondary school students on the island
is coming into operation in Dixon. Neither of the other two communities
plays a role of like importance for the island as a whole.

The three hundred residents of Dixon are all white, Protestant (of three
different denominations), and most of them have lived on the island for
as many generations as those without special interest can trace.
Regardless of occupation, almost all the residents are sufficiently
rural in spirit to keep at least a garden of vegetables, some chickens,
and a few sheep.

\vspace{.2in}
\begin{centering}

\Large{* * * * *}

\end{centering}
\vspace{.17in}

\noindent The deepest social division in Dixon---as perhaps in most small British
communities---is the one which separates persons who have gone to
``Public Schools'' from those who have gone to free government
schools.\footnote{Public Schools in Britain form a national training
  system. Their pupils are recruited from families all over the country
  who have high socio-economic status or aspirations in that direction.
  These schools provide a foundation for a nation-wide network of
  ``personal contacts.'' They also foster a single set of manners,
  attitudes, and speech habits which can be easily distinguished from
  the many different local patterns of behavioral that are possessed by
  rural and urban persons of other social classes.} On the whole,
persons of the kind that go to Public Schools think of themselves as
being different from and superior to other kinds of Britons; in many
areas of social intercourse other kinds of Britons (hereafter called
''commoners'') overtly accept the low status that Public School persons
proffer to them. In country districts, where members of one class are
likely to have known members of the other class all their lives, the
division is often phrased, as it will be occasionally in this study, as
one between ``gentry'' and ``locals.''

In Dixon there are two families of the gentry class; the only other
family of this class on the island lives in Southend. There is the
``Alexander'' family, whose forebears came to Dixon over two hundred
years ago from the mainland of Britain. They have been the principal
resident squire of the island ever since. The second Public School
family consists of ``Dr.~Wren'' and his wife, who moved to Dixon from
the mainland of Britain only a few months before the study began.
Dr.~Wren is the only practising doctor on the island. All the islanders
are registered with him under the British free medical-service plan. He
works and is worked very hard. The Alexanders and the Wrens, and the
Public School family in Southend, the ``Huntleys,'' form a friendship
group. With certain limits and variations, they maintain the style of
life and the social distance from locals that is characteristic of the
gentry everywhere in Britain.

The stereotype that the gentry have of the locals seems to be similar to
the one that prevails throughout Britain. The gentry, when by
themselves, spend a good deal of their time recounting the latest action
of a crofter which shows how impossible crofters really are. This sort
of talk is accompanied by much hilarity and by the applications to the
crofters of a standard of judgment which condones the behavior of the
crofters on the ground that nothing better is to be expected of persons
who are not quite human. Even when crofters referred to are ones whom
the speaker knows well, the general term ``they'' may be used with a
special intonation suggesting that the term ``they`` is not quite a
human term of reference. Frequently when a specific name is mentioned it
is given a special pronunciation or twist to suggest that the person
does not qualify to be referred to by ordinarily statement of name.
Sometimes the gentry refer to crofters (in their absence) as the natives
or locals. One woman, who had lived all her life on the island, in
talking to newly-arrived class-members said,''They're awfully
good-tempered, you know, you have to say that about them.'' In her
absence, another member of the gentry said, ``She's awfully good with
them, you know; she goes fishing with them, and goes into their kitchens
and cooks with them.''

Approximately two thirds of the families in the commoner class derive
their principal source of income from crafting. The remaining sources of
main income derive in part from the County and National governments (in
the form of wages for the maintenance of roads, schools, postal
services, vital statistics registration, and customs inspections, and in
the form of unemployment benefits and pensions), and in part from
private enterprise (in the form of wages and profits for shop owners and
workers, quarry workers, lorry drivers, bakers, hotel operators, skilled
craftsmen, and fishermen).

The commoners in Dixon (as apparently elsewhere in the island cluster)
seem to be a patient, mannerly people with a great deal of self-control.
Towards outsiders they show considerable social reserve; towards
fellow-commoners who live on the island they show equalitarianism
respect and a deep sense of mutual concern.

The ``household`` papers to be the basic social unit; while it usually
contains a single immediate family, it tends to be regarded as the
proper home for lineal and affinal kin who are in need of a place at
which to work or in which to live. Members of a household show a great
deal of kindly solicitude and affection for each other, regardless of
age, sex, or kinship relation.

There are two wider social units based on the household. Each household
has a ''neighbourhood circle,'' consisting of the four or five crofts
that immediately surround it. Each household also has a ``kin circle,''
consisting of the close relations, affinal and lineal, of the male and
female heads of the household, excluding those relations who are on
``bad terms'' with the household. Both of these social units---sometimes
separately, sometimes together---constitute an organization of mutual
aid and informal social intercourse. Both circles are expected to play a
role in funerals and in work crises, such as harvest.

Within the commoner class there is a growing differentiation in style of
life between those who operate small crofts and those who have other
kinds of full-time employment. A locally-recruited middle class is
emerging, based on families that have not engaged in full-time crafting
for one or two generations. Commoners show a strong resistance, however,
to the tendency for this cultural split to become a consciously
recognized social one. Functional explanations for this resistance can
be easily suggested. The social guild between Public School people and
the commoners is sufficiently great to embarrass any division that may
occur within the commoner class. This is reinforced by the strong
tendency for Public School people to treat all commoners in the same
way, for apparently the gentry feel that once a single informal bridge
is created to the commoners, the whole pattern of social distance and
superordination will collapse. Kin circles are stressed as units of
informal social life, and these prevent recognition of the potential
class line by cutting across it. Further, an important element in the
self-conception of Dixon commoners is based on their beliefs concerning
the difference between natives of the island cluster and all other
Britons. This mode of self-judgment undermines the attempt of some
commoners to construct a basic image of themselves in terms of invidious
distinctions between themselves and other commoners. Finally, the chief
merchant family of the island has, up to the present, held itself apart
''socially`` from the commoners, thus failing to play the important role
in class formation that families of this kind typically play in an
island cluster.

In Dixon there are two families which are not placed socially either
with the commoners or with the Public School class. One of these
families consists of the island's previous doctor, now a much-respected,
aged, and ailing man; his son; and his daughter. Both daughter and son
are in their early middle years. Both are unmarried.

The other marginal family, the ``Allens,'' are the chief merchants on
the island. The family came as ordinary commoners from another island in
the cluster three generations ago, and for two generations they have
been the most economically powerful family on the island. One branch of
the family ran a larger sheep farm at the time of the study but has
since moved to the mainland of Britain. The remaining unit of the family
consists of a man, his wife, and a son, ``Ted,'' in his twenties. They
own the principal shops in all three communities, the pier, the bakery,
the mineral rights for the island. They have the coal agency for the
island. They hire two craftsmen to build boats. They operated a woolen
mill for a time, with the aid of a son-in-law. They give full-time
employment to about thirty persons in the community. The potentially
autocratic position of the Allen family is not felt nearly so strongly
as it might be. Several explanations can be suggested. Government
regulations regarding employment and prices provide one kind of
limitation; alternate channels of supply (especially mail-order) provide
another. In addition, the recent generation of Allens seems to have
genuine paternalistic feelings of responsibility toward the economic
welfare of the islanders.

It should be added that the crofters, on the whole, seem to approve of
the Allens. Stories are often told and retold of the times that the
Allens kept men at work even thought it had to be ``made'' work, of the
time they helped to re-establish a family that had been burned out, of
the fact that their prices may not be low but that they are not higher
than prices in other places, of the fact that the Allens have invested
in industry which brought employment for the men of the island but
losses to the owners. Less explicitly, there is the feeling that the
Allens have chosen to remain on the lonely island, and that the
gratifying present relative poverty of the squire is due to the
financial cunning of the previous Allen generation. This positive
attitude seems to prevail even though the social distance that the
Allens have maintained for two generations from their fellow-islanders
is not usual on Bergand islands.

The Allen family has always recruited wives and friends from outside the
local commoner group. For the last two generations they have maintained
the style of life of Public School people. Commoners treat the Allens as
if this family were of the Public School class, and the three families
in that class are on intimate ``social'' terms with the Allens and treat
them almost as equals. It is interesting to note that Ted Allen is so
far an exception to this social pattern. He is treated with social
acceptance by the Public School class, while at the same time he seeks
and finds social equality and intimacy with some commoners in Dixon. His
orientation towards commoners seems to be both cause and effect of the
emerging middle class.

In addition to commoners and Public School people, there are certain
other categories of persons who play a role in the social life of Dixon.
Throughout the year, but especially from May to September, flushing
boats dock at the Dixon pier, remaining from an hour to a week. Crews
from the boats buy supplies at the local stores and exchange fish for
money or fresh eggs with local residents. Also, tourists come to the
community during the summer to fish for trout or watch birds for a week
or two, or to spend one evening of a steamboat excursion on a remote and
rugged island. Finally, throughout the year commercial travellers visit
the island to make a round of its stores, and officials come on
government business.

\vspace{.2in}
\begin{centering}

\Large{* * * * *}

\end{centering}
\vspace{.17in}

\noindent This study is mainly concerned with events which occurred in the social
life of commoners that are native to Dixon. The study is particularly
concerned with three of the social settings in which events of this kind
regularly occurred: socials, billiards, and the hotel.

\hypertarget{socials}{%
\subsection{Socials}\label{socials}}

Every year from September to March a social is held in the Dixon
community hall approximately every second week. Each social is
advertised in the stores and post office of all three communities. In
most cases, anyone who reads the advertisements is free to come. Bus
service from central points all over the island is provided two or three
times during the evening. Attendance varies from sixty to two hundred
persons, most of whom live in Dixon. Northend and Southend hold similar
fortnightly socials. A third of the population of Dixon attends socials
regularly. Direct observations were made at almost all of the socials
held in Dixon and Northend during the twelve months of study.

In Dixon (as in the other two communities on the island) two formal
voluntary associations exist which provide committee machinery for most
of the community undertakings. There is the ``Dixon Workingmen's
Association,'' which owns the community hall and is open for membership
to all male residents of the community, and the ``Women's Rural
Institute,'' which is open to all resident females in the community. The
management of a social usually involves formal cooperation by both of
these associations.

The pattern of organization for socials is well established and helps to
solve the organizational problems for many of the other large-scale
social undertakings that occur in the community. The first part of a
social consists of planned entertainment, starting at eight o'clock in
the evening and lasting for about three hours. A short intermission is
observed for tea and buns, and then the second part of the social, a
dance, is held. The dance lasts until about two-thirty in the morning,
depending, it is said, on the energy and spirit of the dancers.
Admission price for the whole evening is usually two shillings and six
pence, but separate tickets at a lower rate may be bought for the
entertainment or the dancing.

The entertainment part of a social usually consists of ``progressive''
whist played for seven or eight small prizes. Twice a year instead of
whist there is a ``bring and buy sale'' where contributed goods, usually
home-produced, are auctioned. Twice a year the entertainment is provided
by a ``concert'' consisting of a short play, vocal and instrumental
solos, and recitations---all performed by local talent. Persons from six
to eighty participate actively in all of these forms of entertainment.

The dance part of the social consists (with certain seasonal exceptions)
exclusively of ``country-style'' dancing: ``quadrilles,'' ``lancers,''
``gay gordons,'' ``St.~Bernard's waltz,'' ``old fashioned waltz,'' and
``Boston two-step.'' Almost all the dancers are between thirteen and
forty-five, although it is expected that persons outside this age range
will watch the dancing from a single row of benches that line the walls
of the dance floor. Music from the hall stage is provided by a piano and
an accordion or violin. Musicians are recruited from the audience on a
volunteer basis; in one evening three or four different sets of players
may be used. Intermission teas are served once or twice during the
evening's dance. As the evening wears on, the age range narrows until
persons of courting age are almost the only ones that remain.

In Dixon there are some large-scale social occasions at which attendance
may be regarded as an obligation and responsibility: the annual
Christmas party and concert; the two or three ``church socials'' held
each year in each of the churches; the semi-annual ``bulb show'' of
flowers; the annual summer regatta and annual ``gala day.'' Public
School people join with the commoners in these kinds of occasions.
However only commoners and occasionally an island visitor appear at
ordinary socials. On the whole, socials seem to express and consolidate
the feeling that all native commoners are socially accessible to and
socially equal with one another, and that no one will desert the hard
life of the island or their identification with crofters. Lately there
has come to be a twice-monthly showing of a 16 mm. motion picture, but
attendance seems more to divide the community---into those who go and
those who do not---than to integrate the community.

It should be added, perhaps, that elaborate community machinery for
carrying out community-wide social events is not given allegiance by
some members of the community. Many of the male crofters take the view
that they are too tired or too busy to attend the socials. On the whole,
it is those of crofters class who have non-crafting sources of income
who form the hard core of attendants and officers. There is thus one
community composed of persons who have gone to school with each other,
known each other all their lives, intermarried, and another one, set
within the first, composed of all those who are ``active'' in formally
organized community affairs. In certain respects, each of these
communities gives the appearance of being the only one.

\hypertarget{billiards}{%
\subsection{Billiards}\label{billiards}}

In the Dixon community hall there is a room called the ``reading-room.''
It is about twenty feet wide and thirty feet long, and one wall is
shelved with about four hundred books. About twenty-five persons in
Dixon regularly make use of this library

In the center of the room is a standard three-quarter length billiard
table. The table top is geared to an axle, and with the top turned over
the table is used for most of the formal meetings that are held in the
community. Except for the cushions, which are almost dead, the table is
in good condition. There are four cues, one of them short, one of them
cracked, and a set of snooker and billiard balls. Snooker is rarely
played. English billiards is played in accordance with standard rules
for this game. Billiards is officially open to any member of the Dixon
Workingmen's Association and any guests of a member. According to a
formal rule that is often broken, the billiard season is from October to
May, and play is held on Monday and Saturday nights from seven to
eleven-thirty. At the end of each evening of play, each player deposits
three pence for every game he has played, in order to defray the cost of
fuel, lights, and servicing on the part of the hall caretaker.

Observations were made during almost every evening of billiards that
occurred during the period of field work. During this period
approximately fifteen persons came to be recognized as billiard players.
Half of these were steady players who could be expected to appear almost
every night that billiards was held; the other half consisted of
occasional players who spent two or three nights a month at the game.
Most of the players were of the oldest fully-active generation, from
about fifty to sixty-five years old. A third of the players were of the
youngest generation that was fully active in community life, that is,
persons in their late twenties and early thirties. One or two players
fell between the two age groups.

It was apparent that those who have acquired the habit of playing
billiards in Dixon do not represent a socially haphazard selection from
the total population of the community. The players do not share a
particular social characteristic that is not also shared by some
non-players. The \emph{set} of characteristics which most of them
possessed, however, was almost exclusively theirs: residential proximity
to the hall; full-time employment other than crofting; official role in
the Dixon Workingmen's Association.

Billiards in Dixon can be understood in terms of the social functions it
performs for the community. It provides a place where some of the
organizational business of the community can be carried on under
conditions which ensure informality. It provides an opportunity for some
of the older community leaders to give informal training to some of the
future community leaders. It ensures a wide channel of communication of
a point of strong solidarity between the oldest and the youngest active
adult generation.

It is interesting to note that the managers of Allen stores in Dixon and
Southend are players. Ted Allen and the chief assistant of the Allen
store in Dixon also play. All four of these persons play regularly
except the manager of the Southend Allen Store. In general it appears
that billiards provides one of the ways in which the Allen business
organization ensures lines of solidarity with the commoner class---the
class from which it draws its employees and customers.

In this study, attention is not directed to the social functions of
billiards for Dixon. The study is more directly concerned with aspects
of the game that can be described with very little reference to anything
beyond the room in which the game is played.

\hypertarget{the-hotel}{%
\subsection{The Hotel}\label{the-hotel}}

In Dixon there are two hotels, the only ones on the island. One of the
hotels is very little used and used only in the summer. In this study no
attention is focused on it. The other hotel has fourteen guest rooms
which are filled by tourists from the mainland of Britain during July
and August. This hotel is kept open all year round for the overnight
convenience of occasional travelling salesmen and governing agents. It
has also been used during the winter by families of the Public School
class whose houses were being remodeled. It is referred to in this study
as ``the hotel.''

The hotel is owned, operated, and lived in by the ``Tates,'' a couple of
the island-born commoner class in their early middle years. During the
busy summer season a staff of six is maintained, all of whom usually
live on the premises during the period of their employment. During the
long winter there will be no hired staff, or a staff of one or two
maids, depending on whether or not the hotel has any semi-permanent
winter guests.

The hotel was once the home of the Allen family. During the last war the
Royal Air Force rented the building from its present owners and added an
extension for use as a dormitory. Some years ago the Allen family ran a
hotel in Dixon, where Mrs.~Tate received some experience in hotel
management. The Allens no longer operate a hotel. The Tate hotel now
represents perhaps the biggest operation undertaken on the island by
commoners; its success is a symbol for many persons on the island of the
potential ability of commoners.

Mr.~and Mrs.~Tate play leading roles in the organized social life of the
community. They also play a role in the maintenance of solidarity
between Dixon and Northend, since Mrs.~Tate was born and raised in
Northend; she is still known as a ``Northend lass.''

It is customary for the leading belles of Dixon to spend a summer or two
at the hotel in the capacity of kitchen maid, upstairs maid, or
waitress. The pay is good, and the girls frequently claim that they take
the work mainly in order to earn money for especially attractive
clothes. For most of the twelve months of study, two of these girls,
``Jean Andrews'' and ``Alice Simon,'' worked at the hotel. Both were in
their early twenties. During the summer the hotel always hires a Dixon
commoner, ``Bob Hunter,'' as cook; he is about thirty years old,
unmarried, and lives during the winter on his family's croft. For the
last few years, Jean Andrews, Alice Simon, and Bob Hunter have formed
the core of the hired hotel staff.

The hotel itself plays an important role in the community. During the
summer the hotel buys a great deal of food from the local stores and
nearby crofters and is important economically in this way. It plays a
major role in maintaining Dixon as a practical place for tourist
interest. The annual influx of tourists serves, apparently, as
comforting evidence to local residents that Dixon has a place of value
in Britain. The immediate presence of middle and upper-class guests
serves the entire staff as a learning situation for approved patterns of
conduct. The hotel serves in this way as a center of diffusion of higher
class British values.

During the first two months of the study, Dr.~Wren and his wife were
permanent guests of the hotel. They were waiting for the county to
purchase and remodel the building that was to become their house. During
these two months I stayed in the hotel in the capacity of a guest and
took my meals with the Wrens and all occasional hotel guests at a small
dining room table. When the Wrens moved, I moved into a vacant cottage,
returning to the hotel kitchen for meals with the staff. I ate one meal
with them almost every weekday for six months. During a summer I also
worked part time in the hotel scullery as second dishwasher. It was
therefore possible to make a long series of mealtime observations both
as a guest of the hotel and as a member of its kitchen staff, in this
way getting two different views of the same process.

% Part One

\newpage
\thispagestyle{empty}
\begin{fullwidth}

\begin{center}
\vspace*{3in}

{\fontsize{35}{24}\selectfont{Part Two}\par}

\vspace{1in}

{\fontsize{35}{24}\selectfont\textit{The Sociological Model}\par}

\end{center}

\end{fullwidth}

% CHAPTER II: SOCIAL ORDER AND SOCIAL INTERACTION
\chapter[CHAPTER II: SOCIAL ORDER AND SOCIAL INTERACTION]{Chapter II: Social Order and Social Interaction}
\label{ch:Chapter II: Social Order and Social Interaction}
\chaptermark{CHAPTER II: SOCIAL ORDER AND SOCIAL INTERACTION}

\newthought{In the study of}
\marginnote{\href{https://doi.org/10.32376/3f8575cb.9e2c6ac1}{doi}}social life, it is common to take as a basic model the
concept of social order and to analyze concrete behavior by focusing on
the ways in which it conforms to and deviates from this model. In the
present study, I assume that conversational interactions between
concrete persons who are in each other's immediate presence is a species
of social order and can be studied by applying the model of social order
to it.\footnote{The classification of social interaction as a species of
  social organization or social order derives from Talcott Parsons,
  \emph{The Social System} (Glencoe, Ill.: The Free Press, 1951). My
  view of the criteria for social order derives mainly from Chester I.
  Barnard, \emph{The Functions of the Executive} (Cambridge, Mass.:
  Harvard University Press).} The applicability of this model to
conversational interaction is suggested below. In this and in other
chapters dealing with the conceptual framework, conversation in Western
society is assumes as the data for which the framework is to be
relevant.

\hypertarget{the-model}{%
\section{The Model}\label{the-model}}

\begin{quote}
\begin{enumerate}
\item
  Social order is found where the differentiated activity of different
  actors is integrated into a single whole, allowing thereby for the
  conscious or unconscious realization of certain overall ends or
  functions.
\end{enumerate}
\end{quote}

\noindent In the case of conversational interaction, the acts that are integrated
together are acts of communication, or messages. The flow of messages
during a conversation is continuous and is uninterrupted by competing
messages, and any one message from a participant is sufficiently
meaningful and acceptable to the other participants to constitute a
starting point for the next message. Continuous and uninterrupted
interchange of message is the work flow of conversational interaction.

\begin{quote}
\begin{enumerate}
\setcounter{enumi}{1}
\item
  The contribution of an actor is a legitimate expectation for other
  actors; they are able to know beforehand within what limits the actor
  is likely to behave, and they have a moral right to expect him to
  behave within these limits. Correspondingly, he ought to behave in the
  way that is expected of him because he feels that this is a morally
  desirable way of behaving and not merely an expeditious way of
  behaving.
\end{enumerate}
\end{quote}

\noindent This criterion of social order can be applied without modification or
elaboration to the case of conversational interaction.

\begin{quote}
\begin{enumerate}
\setcounter{enumi}{2}
\item
  Proper contribution from participants is assured or ``motivated'' by
  means of a set of positive sanctions or rewards and negative sanctions
  or punishments. These sanctions grant or withdraw immediately
  expressed social approval and goods of a more instrumental kind. These
  sanctions support and help to delineate social rules that are both
  prescriptive and proscriptive, enjoining certain activity and
  forbidding other activity.
\end{enumerate}
\end{quote}

\noindent The relation between conversational order and the sanctions that
regulate it seems somewhat different from the relation between other
types of social order and the sanctions which regulate them. Unlike
other kinds of social order, the sanctions employed in conversational
order seem to be largely of the kind where moral approval or disapproval
is immediately expressed and felt; little stress seems to be placed on
sanctions of a more instrumental kind. Further, in conversational order,
even more than in other social orders, the problem is to employ a
sanction which will not destroy by its mere enactment the order which it
is designed to maintain.

\begin{quote}
\begin{enumerate}
\setcounter{enumi}{3}
\item
  Any concrete social order must occur in a wider social context. The
  flow of action between the order and its social environment must come
  under regulation that is integrated into the order as such.
  Maintenance of this regulated relation depends on the maintenance of
  social order in the environment. On the whole, the stress here is on
  the negative sanctions enjoining non-interference, as opposed to the
  positive sanctions enjoining specific contributions exchanged between
  the order and its environment.\footnote{This factor has recently been
    described by George C. Humans in \emph{The Human Group} (New York:
    Harcourt Brace, 1950), under the term ``external systems.'' See
    especially pp.~86--94.}
\end{enumerate}
\end{quote}

\noindent This element in social order can be applied directly to the case of
conversational interaction.

\begin{quote}
\begin{enumerate}
\setcounter{enumi}{4}
\item
  When the rules are not adhered to, or when no rules seem applicable,
  participants cease to know how to behave or what to expect from
  others. At the social level, the integration of the participants'
  actions breaks down and we have social disorganization or social
  disorder. At the same time, the participants suffer personal
  disorganization and anomie.
\end{enumerate}
\end{quote}

\noindent In the case of conversational interaction, weakening of rules results in
disorganization that is usually experienced as embarrassment. The
occurrence of embarrassment marks a point of confusion and
disorientation; participants sense a false note in the situation.
Embarrassed participants are said to be flustered, ill at ease, or to
have lost countenance.

\newpage

\begin{quote}
\begin{enumerate}
\setcounter{enumi}{5}
\item
  A person who breaks rules is an offender; his breaking of them is an
  offense. He who breaks rules continually is a deviant.
\end{enumerate}
\end{quote}

\noindent In the case of conversational interaction, he who breaks the rules is
said to be \emph{gauche}, \emph{de trop}, or out of place. Offenses, or
in other words acts which cause embarrassment, are said to be bricks,
howlers, \emph{gaffes}, \emph{faux pas}, boners. (These acts,
incidentally, provide us with an opportunity for studying the kinds of
assumptions which underlie proper interaction behavior. These
infractions of proper behavior provide us with a sort of situational
news for directing our attention to the requirements of ordinary
situations which would otherwise have gone unnoticed.) If an actor
continuously breaks interaction rules, and especially if he does this in
a wide variety of different interaction situations, we say he is a bore,
a hopeless person, impossible. In the present study, deviants of this
persistent kind will be called faulty persons.


\begin{quote}
\begin{enumerate}
\setcounter{enumi}{6}
\item
  When a rule is broken, the offender ought to feel guilty or
  remorseful, and the offended ought to feel righteously indignant.
\end{enumerate}
\end{quote}

\noindent In the case of conversational interaction, the guilt that the offender
feels is described as shame. Shame will also be felt by those
participants who have identified themselves with the offender, or who
have defined themselves as personally responsible to theirs for the
maintenance of order. Those who have been offended feel shocked,
affronted, impatient.

\begin{quote}
\begin{enumerate}
\setcounter{enumi}{7}
\item
  An offense to or infraction of the social order calls forth emergency
  correctives which reestablish the threatened order, compensating for
  the damage done to it. These compensatory actions will tend to
  reinstate not only the work flow but also the moral norms which
  regulated it. Some of these correctives will also serve as negative
  sanctions against the offender.
\end{enumerate}
\end{quote}

\noindent In the case of conversational interaction, there is a set of adaptations
to offense which protects the offended but which, in doing this,
destroys the interaction order in which the protective action occurs.
Thus, offended participants can react by withdrawing from the offender,
or by ignoring him completely, or by shifting radically the
understanding and social distances upon which the interaction is based.
(All of these lines of adaption, incidentally, must rely upon the
offender or improper actor to behave in a proper way as an object for
these kinds of action; otherwise they can note be applied to him.)

Usually none of the drastic lines of action mentioned above are
employed. Participants usually respond with toleration and forbearance
to acts which offend against the interaction order. However tentative
this accommodative response may be, it allows the interaction to be
maintained, while corrections, if they are to be applied, can be applied
in a tactful way without destroying the interaction itself.\footnote{Talcott
  Parsons makes the same point in \emph{The Social System}, p.~303:
  ``When we turn to the consideration of normal social interaction
  within such an institutionalized framework as a process of mutually
  influenced and contingent action we see that a process of social
  control is continually going on. Actors are continually doing and
  saying things which are more or less `out of line,' such as by
  insinuation impugning someone's motives, or presuming too much.
  Careful observation will show that others in the situation often
  without being aware of it, tend to react to these minor deviances in
  such a way as to bring the deviant back `into line,' by tactfully
  disagreeing with him, by a silence which underlines the fact that what
  he said was not acceptable, or very often by humor as a
  tension-release, as a result of which he comes to see himself more
  nearly as others see him. These minor control mechanisms are, it may
  be maintained, the way in which the institutionalized values are
  implemented in behavior. They are, on a certain level, the most
  fundamental mechanisms of all, and only when they break down does it
  become necessary for more elaborate and specialized mechanisms come
  into play.''} Accommodative behavior takes the form of apparent
acceptance as appropriate of the behavior of others; it gives rise to
what might be called a working acceptance. Injuries to the working
acceptance are avoided by means of protective strategies and haled by
means of corrective strategies. Exercise of the strategies may be called
tact.

\begin{quote}
\begin{enumerate}
\setcounter{enumi}{8}
\item
  Given the rules of the social order, we find that individual
  participants develop ruses and tricks for achieving the private ends
  that are proscribed by the rules, in such a way as not to break the
  rules.
\end{enumerate}
\end{quote}

\noindent In the case of conversational interaction, individuals employ what might
be called gain strategies. These designs for action allow the individual
to alter the working acceptance to suit his own ends, providing the
alteration is sufficiently small or concealed so as not to jeopardize
the working acceptance itself. Usually the strategist, in these cases,
is interested in raising the definition that others present have of him
and/or in lowering the definition they have of someone else who is
present. In these situations, the working acceptance ceases to be an end
or a means of action and becomes instead a framework of limiting
conditions and boundaries of actions.

\vspace{.2in}
\begin{centering}

\Large{* * * * *}

\end{centering}
\vspace{.17in}

\noindent 

As a model, the concept of social order perhaps does not lead us to give
sufficient stress to a crucial characteristic of conversational
interaction, namely, the forbearance maintenance of a working
acceptance. Let us explore this characteristic for a moment.

When persons find it necessary to exercise forbearance they usually feel
hostile and resentful towards the person who requires this treatment.
Those who forbear must accept, for a moment anyway, a public threat to
interaction norms as well as to the evaluation of self which these norms
help to protect. Certain defenses and strategies of a covert kind are
employed through which the offended but forbearance actor may come to
terms with his ``real'' feelings and with the public threats to them.

The forbearant actor may accept the injury to his private or real
valuations, repress the experience, or keep it as separate as possible
from the rest of his conscious life. He may sincerely try to redefine
his private conceptions in order to make his demands consistent with the
treatment he and the interaction receive. He may, at least to himself,
define forbearance as an opportunistic means to the end of manipulating
the offender, thus proving at least to himself that his public
accommodative behavior is not a real expression of his valuations. He
may covertly impute disqualifying attributes to the offender so that the
behavior of the offender and the treatment accorded the offense need not
be taken seriously.\footnote{An extreme example is found among
  primary-school children who behave themselves in the required manner,
  while crossing their fingers or muttering to themselves denials and
  ritual profanations of the person to whose standards they must show
  forbearance.} He may tell himself that he will withdraw from
communication and from the social relationship that gives rise to it as
soon as it is polite to do so---thus allowing himself to feel that his
forbearance is a sign of forbearance and nothing more. He may, finally,
decide to tolerate the offensive behavior with the object in mind of
sharply correcting the offender at a later time---a time when the
offender will be obliged to accept the criticism in good grace.

The defenses we have been considering represent a form of what has been
called intrapersonal communication.\footnote{Jurgen Ruesch and Gregory
  Bateson, \emph{Communication} (New York: W. W. Norton, 1951),
  pp.~199--203 and 278--279.} They may be effective even though they
seldom give rise to overt action and interpersonal communication except,
perhaps, during some subsequent interaction.

In conversational interaction, as opposed to many kinds of social order,
offense is quite common; hence, forbearance is almost a constant
requirement. The dissensus that forbearance conceals, as expressed in
the many intrapersonal communications to which the necessity to exert
forbearance gives rise, should be considered as part of the model for
conversational interaction and not as something which occurs as a
deviation from the model. For example, the exercise of gain strategies
is so common a thing that it is f often better to conceive of
interaction not as a scene of harmony but as an arrangement for pursuing
a cold war. A working acceptance may thus be likened to a temporary
truce, a \emph{modus vivendi} for carrying on negotiations and vital
business.

It is interesting to note that a desire to maintain a working acceptance
is, paradoxically enough, one of the few general bases of real consensus
between persons. Individuals regularly act on the assumption that others
are the sort of person who would attempt to maintain a working
acceptance, and this imputation of an attribute is usually justified by
consequent behavior. Persons, on the whole, can be relied upon to make
every effort to avoid a ``scene.'' In this context it may be added that
many so-called empty gestures seem to serve primarily as signs that the
sender is ``responsible'' and can be counted upon to play the social
game of maintaining a surface agreement with and an acceptance of the
others.

The very general tendency for persons to maintain a working acceptance
during immediate communication must not lead us to make narrow
assumptions concerning the motivation of this behavior. An actor may
attempt to maintain the appearance of agreement in order to save the
situation and minimize embarrassment, or in order to be genuinely
indulgent to the offender, or in order to exploit the offender in some
way.

We must also be careful to keep in mind the truism that persons who are
present are treated very differently from persons who are absent.
Persons who treat each other with consideration while in each other's
immediate presence regularly show not the slightest consideration for
each other in situations where acts of deprivation cannot be immediately
and incontestably identified as to source by the person who is deprived
by these acts. The kind of consideration shown for persons who are not
present is a special problem and is not dealt with in this study.

The use of the social order model in studying conversational interaction
is inadequate in certain other ways, to be considered later.

% Part Three

\newpage
\thispagestyle{empty}
\begin{fullwidth}

\begin{center}
\vspace*{3in}

{\fontsize{35}{24}\selectfont{Part Three}\par}

\vspace{1in}

{\fontsize{35}{24}\selectfont\textit{On Information About\\ One’s Self}\par}

\end{center}

\end{fullwidth}

% CHAPTER III: LINGUISTIC BEHAVIOR
\chapter[CHAPTER III: LINGUISTIC BEHAVIOR]{Chapter III: Linguistic Behavior}
\label{ch:Chapter III: Linguistic Behavior}
\chaptermark{CHAPTER III: LINGUISTIC BEHAVIOR}

\newthought{In common sense usage,}
\marginnote{\href{https://doi.org/10.32376/3f8575cb.a6e896ec}{doi}}the term ``communication'' seems to be used
chiefly to refer to the transmission of information by means of
configurations of language signs, either spoken or written. This kind of
sign behavior has certain general characteristics:

\begin{quote}
\begin{enumerate}
\item
  The vocabulary of terms employed can be defined or specified with
  tolerable clarity and interpersonal agreement, and is relatively
  independent of the context or medium in which it occurs. Hence,
  messages framed in one language can be translated without great loss
  into other language systems.
\item
  Messages that are put together by means of language signs can be
  discursive, involving a long sequence of interdependent links of
  meaning. These messages can also be abstract in character.
\item
  The language or linguistic component of a message is not merely
  consensually understood but the meaning is, in some sense, officially
  accredited. The sender may be made explicitly responsible for having
  sent it and the recipient may be made officially responsible for
  receiving and understanding it. Given the social situation in which
  the message occurs, its linguistic meaning is the formally sanctioned
  one.
\item
  Linguistic behavior is thought (by the everyday user) to be merely and
  admittedly a means employed in order to convey information. There is
  an obligation to value and judge the message on no other basis. It is
  felt that a linguistic message is conveyed intentionally for the
  purpose of conveying the meaning of it. Speech or writing is a
  goal-directed act, and communication is the goal. It is a voluntary
  act, in the sense that the interpretation the recipient ought to make
  is foreseeable by the sender before he sends his message, at a time
  when it is possible for him to modify his message and within his
  capacity to do so.
\item
  A linguistic message---more technically, the semantic component of a
  message---has an explicitly stated object of reference or direction of
  intent. As a recent student of conversational interaction has
  suggested:

  ``The direction of intent is operationally defined as the object
  toward which the remark is made. Remarks may be directed toward the
  self, toward the group relationships, toward the issue being discussed
  and toward aspects not involved in the immediate
  grouping.''\footnote{B. Steinzer, ``The Development and Evaluation of
    a Measure of Social Interaction,'' \emph{Human Relations}, II,
    103--121 and 319--347. Steinzer attempts to work out a set of
    general intent categories for the content analysis of conversation.}

  The ``meaning'' of a message resides, then, in what is said about its
  object of reference. Meaning is thus officially independent of the
  actor who sends the message and of the conditions under which the
  message is sent.\footnote{Recipients almost always qualify the
    information in a message by observations concerning the state of the
    sender, his calmness, nervousness, seriousness, social
    qualifications, and the like. Strictly speaking, these qualifying
    sources of information pertain to the response of the recipient, not
    to the content of the message. Qualifying information does not
    officially provide us with biographical material concerning the
    sender, as do his statements about himself. The signs which convey
    qualifying information are not linguistic signs and do not have an
    object of reference; they are natural signs or expressions and are
    an intrinsic part of the very thing about which they convey
    information.}
\end{enumerate}
\end{quote}

No doubt the most important kinds of linguistic behavior consist of
spoken and written communication. There are, of course, clear cut
examples of linguistic behavior which involve performances of other
kinds. The Morse code and the semaphore system provide cases in point.
These are, as Sapir suggested, language transfers, and anything that can
officially be expressed by means of spoken words can be conveyed by
them.

In addition, there are cases where formal and official language-like
status is given to certain behaviors of a gestural kind but where a
limitation exists as to the size of the vocabulary and as to the number
of different statements that can be made in the language. In considering
this kind of behavior, Sapir says:

\begin{quote}
In the more special class of communicative symbolism one cannot make a
word-to-word-translation, as it were, back to speech but can only
paraphrase in speech the intent of the communication. Here belong such
symbolic systems as wig-wagging, the use of railroad lights, bugle calls
in the army, and smoke signals.\footnote{Edward Sapir, \emph{Selected
  Writings of Edward Sapir}, ed.~David G. Mandelbaum (Berkeley:
  University of California Press, 1951), p.~107.}
\end{quote}

\noindent Sapir also suggests that this kind of language behavior usually
subserves a technical process where spoken and written language is
impractical because of transmission conditions or because there is a
desire to rigidly limit the possible response to the message. In Dixon,
for example, a shepherd rounding up sheep can manage his dog by means of
about six gestures that are significant to his dog and for which the
dog, in a sense, is held officially responsible. There is a signal to
make the dog stop in his tracks, to make him lie down, to make him come
back to the shepherd, to make him cannily come up behind a sheep, to
make him cut far in back of a stray so as to round it up. Perhaps there
is also a type of signal employed to make the dog vary the rate at which
these commands are obeyed, but this tends to be a less official part of
the vocabulary and is played down in the competitions called dog trials
that are formally designed to test the discipline and linguistic
capacity of dogs. Another technical language can be found in use in
Dixon on Friday nights by the eight or ten men regularly employed to
unload the steamboat. Due to noise level and the restriction of angle of
vision caused by equipment, a small vocabulary of terms consisting of
full arm gestures is employed to instruct the hoist engineer as to rate
of movement of the hoist rope, as to lateral and vertical movements, and
as to stopping and starting.

We must consider, finally, an even more simple kind of language
behavior. There are certain gestures which are employed as an official
part of linguistic communication but which can only be officially used
to convey a single piece of information.\footnote{Many examples of
  conventionalized gestures of this kind are considered in Levette J.
  Davidson, ``Some Current Folk Gestures and Sign Languages,''
  \emph{American Speech}, XXV, 3--9.} In Dixon, for example, pupils in
the classroom gain permission to speak to the teacher by first holding
up their right hand. Pupils are taught that holding up one's hand is the
formally correct and recognized way of obtaining the attention of the
teacher. Similarly, when one student has the eraser that has been
assigned to his block of seats, other pupils in this block can request
the eraser by tapping the student who has it on the shoulder. Tapping
for this purpose has been explicitly and officially assigned a meaning,
although, of course, the teacher has difficulty in preventing students
from loading the sign with meanings of an unformalized kind. Throughout
our society, beckoning gestures signifying ``come here,'' and shrugging
of shoulders signifying ``I do not know'' tend also to be messages whose
meaning is clearly understood and to a degree formally
accredited.\footnote{The art of miming provides a very exceptional case
  of employing sequences of understandable gestures in order to convey
  information of the same complexity as can be conveyed by the spoken
  word. Certain types of deaf and dumb language which do not employ
  language transfer provide another example. These symbol systems are
  not an adaptation to special technical conditions of communication, as
  in the case of train signals, but rather an adaptation to lack of
  usual communication capacities on the part of communicators.}

The general characteristics of linguistic behavior have been mentioned,
and four types of this kind of behavior have been described: language
proper; language transfers; technical symbol systems; and official
signals. These types of linguistic behavior vary in complexity and
formality, but all share the essential characteristic of linguistic
behavior: they carry a message for which the sender can be made
responsible, and they are properly usable in an admittedly intentional
way for purposes of communication.

When we examine linguistic behavior, a clear difference can be found
between the object \emph{at which} the message is \emph{directed} and
the object \emph{to which} the message \emph{refers}. When an individual
says something about a person he is talking to, these two objects
coincide, but this fact should not lead us to confuse the role of
recipient with the object of reference.

In analyzing linguistic behavior, it is convenient to distinguish types
of recipient. A linguistic message may be directed at one or more
specific persons who are immediately present to the sender. This
immediate linguistic communication is sometimes called face-to-face
interaction or conversational interaction. It is, perhaps, the classic
or type-case of linguistic communication, other kinds being
modifications of it. Linguistic communication may, of course, occur
between specific persons who are not immediately present to one another,
as in the case of telephone conversations and exchanges of letters.
These mediated kinds of communication contact vary, of course, in the
degree to which they restrict or attenuate the flow of information and
the rapidity of interchange. In the literature, this has been called
point-to-point communication. Linguistic communication may also occur
between a source of sign impulse and all the persons who happen to come
within range of it. This has been called mass-impression in the
literature. We are accustomed to consider this weakened kind of
linguistic communication in reference to advertising and the mass media,
radio, press, etc., but it also plays an important role in rural
communities. In Dixon, for example, two of the three shops and the post
office have bulletin boards on which notices of all kinds are posted.
These notices are usually of the ``open'' or ``to whom it may concern''
kind. Anyone seeing a notice is automatically considered to be an
appropriate recipient for it. Invitations to community socials, to
auctions, to funerals, etc., are posted in this way.\footnote{In
  European cities that have a \emph{quartier} type of social
  organization, the death of a resident is sometimes advertised by means
  of a card placed in a shop window or on the door of a house, the card
  inviting all those who wish to attend the funeral.} Correspondingly,
the kinds of ``social occasions'' that are organized by means of
conversational or point-to-point communication of invitations are not
stressed, although they are becoming more common. In previous periods in
the social history of Dixon, invitations to weddings were also posted in
an open way, a practice which is still observed in a few small Bergand
communities. Mass impression is often a weakened form of linguistic
communication because recipients are usually not obliged to accept
responsibility for having received the message, although this is not
always the case.\footnote{Recently we have been witnessing a series of
  conflicts concerning the use of certain bounded spaces for this kind
  of communication. Legal questions have arise as to whether advertisers
  have the right to make use of the sky above a city for skywriting, the
  sidewalks in front of stores for loudspeaker blasts, and bus and rail
  coaches for piped advertising. This is referred to as the problem of
  the ``captive audience.''} Orders posted on a barracks bulletin board
usually render all persons in the barracks legally accountable for
having read the message. Government proclamations in official newspapers
or even on the radio can also carry this kind of responsibility. Posting
of banns is another case in point.\footnote{Ruesch and Bateson
  (\emph{op. cit.}, p.~39) have suggested ``many-to-one'' as another
  type of linguistic communication. In such cases, a large number of
  different persons are able to convey, in a relatively anonymous way,
  particular messages to a single recipient. Instead of a
  mass-impression this is, in a sense, mass-impressing communication.
  Another marginal type of linguistic communication occurs when one
  sender directs a message to a specific recipient in such a way that
  many other persons can equally well receive the message. This occurs
  in the case of personal message boards, as found in some European
  cities, and when a child comes to the window of a house and yells for
  his companion to come out. Perhaps these kinds of communication
  arrangements can best be considered in another context, in an analysis
  of sender and recipient responsibility.}


% CHAPTER IV: EXPRESSIVE BEHAVIOR
\chapter[CHAPTER IV: EXPRESSIVE BEHAVIOR]{Chapter IV: Expressive Behavior}
\label{ch:Chapter IV: Expressive Behavior}
\chaptermark{CHAPTER IV: EXPRESSIVE BEHAVIOR}

\newthought{Whenever an individual acts}
\marginnote{\href{https://doi.org/10.32376/3f8575cb.17a480e0}{doi}}in any way, we can assume that something
about him is conveyed, even if it is only the fact that he did not act
in a given way. In the style of the act, in the manner in which the act
is performed, in the relation of the act to the context in which it
occurs---in all these ways something about the actor is presented in the
character of his act. The tendency for the character of the actor to
overflow into the character of his acts is usually called the expressive
aspect of behavior.\footnote{Perhaps the best study of expressive
  behavior is to be found in Gordon Allport and Philip Vernon,
  \emph{Studies in Expressive Behavior} (New York: Macmillan, 1933).}

Behavior which is not expressive may be called instrumental.
Instrumental behavior consists of activity which is officially of no
value in itself but only of value in so far as it serves as a means to
another end. Linguistic communication is a type of instrumental behavior
and is officially valued only because it can serve as a means of
conveying information. It must be clearly understood that expressive
behavior is not a form of instrumental behavior; it is not intended as
an admitted means to the end of transmitting information, or, in fact,
as a means to any other end. Expressive behavior is not, primarily,
rational behavior that can find a place in a voluntaristic means-ends
scheme; rather, it is part of the behavioral impulse associated with any
act.\footnote{An effort is sometimes made in the literature to say that
  a logic can be found in expressive behavior; it may be
  ``understandable'' to others, through a process of emotional empathy;
  it may ``hang together'' ``as a whole where the form of each of the
  parts reinforces and repeats the form of the whole; it may serve a
  psychological or social function; etc. However, the possibility of
  making many different kinds of''sense'' out of expressive behavior
  does not alter the status of that behavior as a non-rational,
  non-instrumental type of action.}

In distinguishing between expressive and instrumental behavior, a manner
of speaking has been employed which carries certain kinds of danger.
Instead of speaking of instrumental and expressive behavior, it might be
more accurate to speak of the instrumental and expressive components of
a given concrete behavior. It might be still more accurate and stills
after to speak rather of the instrumental and expressive functions of a
given concrete behavior, this last usage minimizing the tendency to
refit into concrete entities what are merely analytical aspects or
abstractions of concrete entities. Purely for reasons of style, all
three usages will be employed interchangeably.

When we examine the components of behavior in situations, it will be
apparent that in one situation the instrumental component will be
dominant and in another situation the expressive component will be
dominant. One usually says, for example, that the performance of a
manual task is predominantly instrumental and that our exclamation when
we stub our toe is predominantly expressive. It will also be apparent
that a situation which we expect to find defined as predominantly
instrumental may take on extra expressive significance until the latter
component becomes the dominant one. Thus, when a worker on the line
becomes concerned with the kind of time-rating that has been accorded to
his job by management, both he and management may become more concerned
with the spirit in which he performs his job and with his marginal
productivity than with his production as a whole. The expressions he
conveys may suddenly become more important than the operations he
performs. In all of this there is no conceptual problem.

The distinction between expressive and instrumental components of action
has been recognized by many students. An aspect of the distinction
appears in an essay by Durkheim written in 1906.\footnote{Emile
  Durkheim, ``Determination du Fait Moral,'' reprinted in
  \emph{Sociologie et Philosophie} (Paris: Presses Universitaires de
  France, 1951), especially pp.~60--61.} At that time he suggested that
some acts have concrete consequences and that other acts have social
consequences. In the first case we deal with acts only because they have
consequences; in the second case we deal with acts because they express
something about the actor and his relation to the moral world.
Radcliffe-Brown and Talcott Parsons make a similar
distinction.\footnote{A. R. Radcliffe-Brown, ``Taboo'' (Frazier Lecture,
  Cambridge, 1939), \emph{Structure and Function in Primitive Society}
  (London: Cohen and West, 1952), pp.~143--144; Talcott Parsons,
  \emph{The Structure of Social Action} (New York: McGraw Hill, 1937),
  pp.~430--433.} Lately, Bales has given us a thorough characterization
of the different between the two components of action:


\begin{quote}
When we wish to make a distinction regarding a predominant weight of
emphasis on the backward or forward reference of action, we shall use
the terms ``expressive'' and ``instrumental'' respectively, to designate
the proper weight of emphasis. If the act is judged by the observer to
be steered by cognitive orientation primarily to the past, or if it is
felt to be caused in a nonmeaningful manner by some existing state of
emotion or motivational tension in the self, and if the results which
follow it are judged not to have been specifically anticipated by
symbolic manipulation, we shall speak of the act as primarily
expressive. On the other hand, if the act is judged to be steered by a
cognitive orientation to the future as well as the past and to be caused
in part by the anticipation of future consequences, we shall speak of
the act as instrumental. This distinction is recognized in our everyday
habits of speech: in what we have called primarily expressive activity,
the individual is said to act ``because'' of some immediate pressure,
tension, or emotion. In the instrumental act, the individual is said to
act ``in order to'' realize certain ends. Thus, we might drum our
fingers on the table \emph{because} we are nervous or tense, or we might
raise our eyebrows \emph{in order to} summon the waiter. The difference
lies in the degree to which anticipated consequences enter in as a
steering factor. All instrumental activity is also expressive, as we
view it, but not all expressive activity is necessarily instrumental.
All behavior is considered to be at least 

\newpage expressive, as viewed by the
other and as apprehended and scored by the observer.\footnote{Robert F.
  Bales, \emph{Interaction Process Analysis} (Cambridge, Mass.:
  Addison-Wesley Press, 1950), pp.~50--51.}
\end{quote}

\noindent The distinction between expressive behavior and instrumental behavior
has been elaborated and at the same time confused by many current
writers who contrast expressive behavior with linguistic behavior. In
making use of these efforts, one always runs the risk of forgetting that
linguistic behavior is merely one sub-type of instrumental behavior, and
that the proper contrast is between the two general classes of behavior,
expressive and instrumental, and not between one class and a member of
the other class. We can partly correct for this error by keeping in mind
that our interest here is the contrast of one kind of instrumental
behavior, namely, linguistic behavior, with one kind of expressive
behavior, namely, the kind that is apt to occur when persons are engaged
in conversational interaction.

Sapir provides us with a good statement of the intermingling of
expressive and linguistically-instrumental behavior in speech:

\begin{quote}
Gesture includes much more than the manipulation of the hands and other
visible and movable parts of the organism. Intonations of the voice may
register attitudes and feelings quite as significantly as the clenched
fist, the wave of the hand, the shrugging of the shoulders, or the
lifting of the eyebrows. The field of gesture interplays constantly with
that of language proper, but there are many facts of a psychological and
historical order which show that there are subtle yet firm lines of
demarcation between them. Thus, to give but one example, the consistent
message delivered by language symbolism in the narrow sense, whether by
speech or by writing, may flatly contradict the message communicated by
the synchronous system of gestures, consisting of movements of the hands
and head, intonations of the voice, and breathing symbolisms. The former
system may be entirely conscious, the latter entirely unconscious.
Linguistic, as opposed to gesture, communication tends to the official
and socially accredited one; hence one may intuitively interpret the
relatively unconscious symbolisms of gesture as psychologically more
significant in a given context than the words actually used. In such
cases as these we have a conflict between explicit and implicit
communications in the growth of the individual's social
experience.\footnote{Sapir, \emph{op. cit.}, p.~105.}
\end{quote}

\noindent Another good description is found in Pear, in his discussion of
conversational tact:

\begin{quote}
Let us for a moment regard conversational tact objectively, as a mere
matter of movement-patterns. Gramophone records of some tactful
conversations would give a very imperfect impression, for many signals
of tact are visual. Raising or refraining from raising the eyebrows,
presenting a sympathetic or inscrutable face, settling into one's chair
as if to invite the vis-à-vis to make a long speech; rising suddenly as
if to indicate its termination; no one of these events is transmissible
by radio without television. Subtler, however, and often less easy to
study are speech-sounds made tactfully. The words and phrases,
intonation, speech-melody, are all important; yet their choice depends
so much upon local convention, the relative social status of the
conversants, the district in which the phrase is used, that to interpret
them requires expert knowledge. At times, an important feature of
conversational exchange may be a momentary physical contact of the
conversers. A touch, a hand on the shoulder, a hand-shake or its
omission, when meeting or parting---all these gestures, especially the
hand-shake, need to be translated and the translation should be an
up-to-date one.\footnote{T. H. Pear, \emph{Psychology of Conversation}
  (London: Nelson, 1929), p.~48.}
\end{quote}

\noindent The distinction between the linguistic and expressive components of
speech is often pointed up by reference to the logically discursive
character of language proper in contrast to the ``emotional'' character
of the expressive or gestural components of speech. As Park suggested:

\begin{quote}
In the fist case {[}symbolic language{]} the function of language is
purely `referential,' as in scientific discourse. It points out its
object, identifiers, classifies, and describes it. In the second case
{[}expressive language{]}, language, modulated by accent, intonation and
inflection, tends to be expressive merely. In that case the function of
words seems to be to reveal the mood and the sentiments of the person
who utters them, rather than to define and express an idea.\footnote{Robert
  Ezra Park, \emph{Race and Culture} (Glencoe, Ill.: The Free Press,
  1950), pp.~38--39.}
\end{quote}

\noindent Ogden and Richards, of course, have given us the terms ``referential''
and ``emotive'' to designate the linguistic and expressive components of
speech.\footnote{C. K. Ogden and I. A. Richards, \emph{The Meaning of
  Meaning} (New York: Harcourt Brace, 1938), pp.~152--158.} Lasswell has
suggested the terms ``purport'' and ``style'' to designate the same
difference in written communication.\footnote{Harold Lasswell,
  \emph{Language and Politics} (New York: Stewart, 1949), chap.~ii,
  ``Language of Politics.''}

Common sense understanding of the phrase ``expressive behavior'' seems
to be closely tied to commonsense notions concerning the identity and
character of the so-called ``natural expression'' of the emotions. If
one is to use the term ``expressive behavior'' or ``expression'' in a
consistent and technical way, it is helpful to go back to the
commonsense conception of emotional expression and to make explicit some
of the assumptions and limitations of this everyday concept.

Critchley, in his discussion of expressive behavior, provides us with a
useful summary of emotional signs. He includes among them:

\begin{quote}
\ldots{} those cutaneous phenomena of a primitive and protective nature,
subserved by the autonomic nervous system and which are almost entirely
outside the control of volition; the manifestation of blushing, pallor,
horripilation, goose-flesh and sweating belong here. Tremor of the
hands, dryness of the mouth, increase or decrease in the muscular tonus,
alteration in stance and attitude, are also regarded as expressive
movements of a more automatic and less voluntary character.\footnote{Macdonald
  Critchley, \emph{The Language of Gesture} (London: Edward Arnold,
  1939), pp.~11--12.}
\end{quote}

\noindent Another is by Blumer:

\begin{quote}
Expressive behavior is presented through such features as quality of the
voice---tone, pitch, volume---in facial set and movement, in the look of
the eyes, in the rhythm, vigor, agitation of muscular movements, and in
posture. These form the channels for the disclosure of feeling. It is
through these that the individual, as we say, reveals himself as apart
from what he says or does. Expressive behavior is primarily a form of
release, implying a background of tension. It tends to be spontaneous
and unwitting; as such, it usually appears as an accompaniment of
intentional and consciously directed conduct.\footnote{Herbert Blumer,
  ``Social Attitudes and Nonsymbolic Interaction,'' \emph{J. of
  Educational Sociology}, IX (515--523), 520.}
\end{quote}

\noindent The commonsense understanding is that these emotional expressions are
instinctive and not subject to voluntary control;\footnote{Psychologists
  have provided some rational elaborations of the voluntary-involuntary
  dichotomy. Voluntary behavior is said to consist of movement subject
  to the conscious control of the subject. These movements are said to
  be activated by the striped muscles under control of the cerebospinal
  nervous system. Involuntary behavior is said to consist of movements
  not subject to the conscious volition or control of the subject. These
  movements are thought to be activated by the smooth muscles under
  control of the autonomic nervous system. A qualification recognized by
  psychologists is that many movements over which persons have no
  conscious control can be brought under voluntary control by special
  training; the eye-blink is a favorite example. This view of the
  dichotomy is inadequate in many different ways, but I am not able to
  provide an adequate analysis of the concepts involved. For an
  interesting preliminary treatment see Gilbert Ryle, \emph{The Concept
  of Mind} (London: Hutchinson's University Library, 1949), pp.~69--74.}
that the form of the expression is somehow an iconic image of the mental
state or emotion that gives rise to the expression; that signs of
emotion provide a trustworthy index of how and what the action is really
feeling. Let us examine these assumptions.

When we examine instances of emotional expression, we frequently find
that these signs are not iconic and do not portray or delineate by their
structure the structure of their reference. Since these signs are
symptoms, not symbols, they frequently form a highly differentiated part
of the causal complex that gave rise to them.\footnote{Apparently there
  is some ground for claiming that emotional expressions are vestigial
  remains of acts and states once useful to the organism as an
  adaptation to crises. See Charles Darwin, \emph{Expression of the
  Emotions in Man and Animals} (London: John Murray, 1872).} One also
finds that it is not helpful to refer to these expressions as
``instinctive.'' By now it is well understood that the same group of
persons uses the same expression, e.g., tears, in quite different
emotional contexts, and that there are very great differences from group
to group as to where, how, and how much the emotions will be
expressed.\footnote{These differences have been well described in Weston
  Labarre, ``The Cultural Basis of Emotions and Gestures,'' \emph{J. of
  Personality}, XVI, 49--68. A model empirical study is found in David
  Efron, \emph{Gesture and Environment} (New York: King's Crown Press,
  1941), where a description is given of differences in conversational
  gestures between Italians and Jews in New York.} In referring to
expressive behavior as forming a collective texture, Blumer suggests
that:

\begin{quote}
\ldots{} expressive behavior is regularized by social codes much as is
language or conduct. There seems to be as much justification and
validity to speak of an affective structure or pattern of meanings.
Almost every stabilized social situation in the life of a group imposes
some scheme of affective conduct on individuals, whose conformity to it
is expected. At a funeral, in a church, in the convivial group, in the
polite assemblage, in the doctor's office, in the theater, at the dinner
table, to mention a few instances, narrow limits are set for the play of
expressive conduct and affective norms are imposed. In large measure,
living with others places a premium on skill in observing the affective
demands of social relations; similarly, the socialization of the child
and his incorporation into the group involves an education into the
niceties of expressive conduct. These affective rules, demands, and
expectations form a code, etiquette, or ritual which, as suggested
above, is just as much a complex, interdependent structure as is the
language of the group or its tradition.\footnote{Herbert Blumer,
  \emph{op. cit.}, pp.~522--523.}
\end{quote}

\noindent And even if one wished to argue that the emotions themselves are
somewhat instinctive, as opposed to the form in which they are
conventionally expressed, it would still be necessary to appreciate that
an event which arouses our emotions must derive its significance from
the world of learned social values in which we live.

Further, it is a fact that there is an important expressive component in
behavior which is thought to be in no way emotional, in the ordinary
sense of that term. For example, in making a statement that is felt to
be the kind which requires a great deal of careful consideration,
deliberation, and freedom from emotional bias, the conviction that one
is, in fact, making a thoroughly voluntary statement of this kind, is
conveyed by certain expressive behaviors of an involuntary kind. If
listeners feel that this expressive component is deliberately feigned
and controlled, then the capacity of the statement to convince the
listeners that it is a sincerely deliberative one may be destroyed.
Similarly, all our so-called voluntary behavior, such as walking, or
talking, involves a degree of unselfconsciousness and could not be
smoothly executed were one to become too conscious of what one is doing.

The commonsense assumption that emotional expression is a reliable index
to the state of mind of the actor appears to be partly valid, but
perhaps not for the reasons commonsense would supply. In this study it
is assumed that the emotional expression practiced by the members of a
particular group is determined by the moral rules recognized in the
group regarding social interaction. The member must not only learn how
and when to express his emotions, but is morally obliged to express them
in this approved way.\footnote{An excellent treatment of this question
  may be found in Charles Blondel, \emph{Introduction a la Psychologie
  collective} (Paris: Armand Colin, 1927), chap.~iii, ``La Vie
  affective,'' pp.~152--158.} Further, the member is obliged to obey the
rules of the expression, once learned, in a sufficiently automatic and
unselfconscious way so that observers will in fact be partly justified
in their assumption that the emotion conveyed to them is a dependable
index of the actors emotional state. It is suggested here that emotional
expression is a reliable index because persons have been taught to act
in which a way as to make it a reliable index and are more ally obliged
to act in such a way as to confirm the fiction that emotional response
is an unguarded instinctive response to the situation.

We see, then, that if we focus our attention on emotional behavior, we
shall arrive at too narrow a conception of the concept of
``expression.'' Some further, and even more fundamental, limitations are
produced by undue concern with emotional expression. We maybe begin to
examine these limitations by noting Morris' definition of expressive
behavior.

\begin{quote}
\ldots{} the manner of production of signs and the kinds of signs
produced may themselves be to the producer of the sign or to other
persons signals of the state of the producer of the sign. This is a
common situation, and such signs can be called expressive signs. A sign
on

\newpage this usage is \emph{expressive} if the fact of its production is
itself a sign to its interpreter of something about the producer of the
sign.\footnote{Charles Morris, \emph{Signs, Language, and Behavior} (New
  York: Prentice-Hall, 1946), pp.~67--68.}
\end{quote}

\noindent Here Morris seems to be suggesting that the expressive aspects of sign
behavior, such as rapidity or smoothness of conversational flow, may
express something about the emotional state of the talker in exactly the
same way as might other holiday movements, such as nervous movements
with fingers and eyes. None of these signs are symbols instrumentally
designed as a means to the end of communication; they are natural signs
or symptoms of a causal complex, the individual's emotional state.
However it seems to be reasonable to extend the concept of expression
and say that certain aspects of a given body of speech may be in one
sense of the symbolical order and yet expressive. For example,
considerable work has been done by psychologists,\footnote{See, for
  example, F. H. Sanford, ``Speech and Personality: A Comparative Case
  Study,'' \emph{Character and Personality}, X, 169--198; Stanley S.
  Newman, ``Personal Symbolism in Language Patterns,''
  \emph{Psychiatry}, II, 177--184, and ``Behavior Patterns in Linguistic
  Structures,'' in \emph{Language, Culture and Personality}, eds.~Leslie
  Spier, A. Irving-Hallowell and Stanley S. Newman (Manasha, Wis.: Sapir
  Memorial Publication Fund, 1941), pp.~94--106.} linguistic
anthropologists,\footnote{See, for example, Benjamin Lee Whorf, ``Four
  Articles on Metalinguistics,'' reprinted from \emph{Technology Review}
  and \emph{Language, Culture, and Personality} (Washington, D.C.:
  Foreign Service Institute, Department of State, 1950).} and
psychoanalysts,\footnote{A clear treatment of the different order of
  things that can give rise to expressions is given by Roland Dalbiez,
  \emph{Psychoanalytical Method and the Doctrine of Freud}, trans. T. F.
  Lindsay (New York: Longmans, Green, 1941), Vol. II, chap.~iii, ``The
  Methods of Exploring the Unconscious.'' See especially p.~94 ff.,
  where he considers the fact that psychic states, like physiological
  ones, can give rise to symptoms of a psychic kind. He employs the term
  ``psychic Expression'' to refer to a natural sign of mental
  phenomenon.} in illustrating the notion that a patter of thought or a
way of organizing phantasies can act as a causal complex and give rise
to expressions of a symptomatic natural-sign type, even though the
events that are patterned or organized consist of conventional
linguistic symbols which carry an object of reference.

Once we allow that a causal complex which produces expressions can be
something other than the emotional state of a particular actor, and even
something of a different order, namely, images and symbols, we are in a
position to take a further step. It greatly simplifies thinking if we
assume that a set of persons in actual interaction with one another
constitutes a casual complex which can give rise to
expressions.\footnote{I am not concerned here with arguing the
  nominalist-realist problem; interaction as a system of integrated acts
  may ultimately be best analyzed from the point of view of each
  participant, taken successively, and not from the interactive system
  as a whole. Whether fiction or not, the treatment of an interaction
  system as a reality \emph{sui generis} greatly simplifies the
  conceptual problem.} When we classify interaction systems along with
emotional states as something which can give rise to natural signs, then
we are in a position to appreciate more clearly the great number of
events which are ``expressive'' and to remove from our focus of
attention from gestures which pertain to the physiological equipment of
particular actors and bring it to bear on events which express
relationships between persons or between persons and the social context.

Regardless of what casual complex one is interested in---be it the
emotional state of the actor, his mode of organizing experience, or the
interaction as a unit---the meaning of an expression does not lie in the
relation between the expressive act and the actor but in the relation of
the actor, through time and space, to the social context in which the
expressive act occurs. Bales provides a good statement of this:

\begin{quote}
A great many of the qualitative distinctions we feel in the observation
of interaction, and the verbal terms by which we designate these
distinctions, rest essentially on our conception of the nature of the
established social relationship between the participants. For example,
approximately the same kind of concrete behavior might be called
``rewarding the other'' if the status of the actor is assumed to be
higher, or ``congratulating the other'' if the status is assumed to be
equal, or ``admiring the other'' if the status of the actor is assumed
to be lower. Other-distinctions are based on a combination of this kind
of assumption plus an assumption about the nature of the preceding act,
that is, on temporal sequence. For example, a given kind of concrete
behavior might be called ``submission'' if it follows an aggressive
attack by the other, or ``agreement'' if it follows a tentative
proposal.\footnote{Bales, \emph{Interaction Process Analysis},
  pp.~68--69.}
\end{quote}

\noindent Once we see that the commonsense assumptions concerning emotional
expression involve limitations, we can go on and attempt to introduce a
set of assumptions that are more helpful for sociological purposes. We
can see expressive behavior as one sub-class of a more general category,
expressive events. We can define expressive events as signs that are
symptomatic of the structure of a social situation, this structure
involving the relation of the participations to one another and to the
situation. The emotion (as this term is commonly understood) that is
involved in these relations will only be one variable, and for the
source of these expressive events we will look not to the physiological
machinery of a particular actor but to the general characteristics of
the physical and social scene in which the interaction occurs.

The scene in which interaction occurs seems constantly to provide us
with a sort of expressive field, a supply of events so well designed to
portray the conceptions and evaluations that persons have of one another
that after a process of social learning we unselfconsciously and
uncalculatingly employ them in this way. Let us attempt to outline these
general sources of expressive signs.

Persons, like other physical objects, are uniquely located in time and
space. Therefore they are necessarily ordered in the transitive relation
of priority (both temporal and spatial) with respect to any particular
point of reference. This provides---whether desired or not---a readily
available means of expressing social precedence. Similarly, degrees of
physical closeness or separateness between persons are inevitable on
physical grounds and incidentally provide vehicles for expressing social
intimacy and social difference.\footnote{The role of ``presence of one's
  body'' as a vehicle for carrying signs expressive of social intimacy
  and equality has been given important consideration by W. Lloyd Warner
  in the Yankee City series, especially in the treatment of the social
  role of clique structures. The phrase ``informal participation'' has
  been used in this connection. Perhaps the limiting case of this sort
  of thing is found in the use of the term ``to have smallpox'' that is
  found among American criminals. A person wanted for arrest is said to
  have ``smallpox;'' ``smallpox'' is ``catching'' because anyone found
  in the intimate presence of a person wanted for arrest is himself
  subject to arrest.} This provides us with a sort of ``expressive
ecology.''

The process of linguistic communication, as a physical process, has many
preconditions, characteristics, and concomitants which can, and
regularly do, serve as expressions of the attitudes and evaluations that
participants have regarding one another. Delicate shadings of approval
and disapproval, inclusion and exclusion, are typically conveyed in this
way.\textsuperscript{25}

The formal organization of persons for the pursuance of a given overall
task requires---due to the nature of organization as such---that orders
be given, that actions be initiated by one person to another, and that
individuals\marginnote{\textsuperscript{25}\setcounter{footnote}{25} An example can be found in William F. White and Burleigh
  B. Gardner, ``Facing the Foreman's Problems,'' \emph{Human
  Organization}, IV, 1--18. In this article the writers describe the
  care that management must take not to talk with one worker more than
  with another, lest this be taken as an expression of favoritism (see
  p.~10, the section on ``Communication and Favoritism''). They also
  consider the fact that ``one-way'' communication may express one kind
  of social evaluation or relationships, and ``two-way'' communication
  another relationships.} actively cooperate with one another.\footnote{A group of
  sociologists influenced by the work of E. D. Chapple have stressed,
  perhaps too much so, the expressive overtones usually found in
  situations where one worker must habitually initiate action of a
  purely instrumental kind to another worker.} Many of these
requirements of organization provide vehicles which are employed as
signs expressive of the valuation that the members of the organization
make of one another. These valuations pertain to matters such as
equality-inequality, subordination-superordination, dependency, etc.

The performance of a particular individual at a given task differs at
least to some degree---on physical grounds alone---from the performance
those present have come to expect of the task in general and of the
individual in particular. Inescapable deviations of this kind provide a
ready sign for conveying the attitude of the performer to those for whom
and among whom the performance occurs.\footnote{An important body of
  data illustrating this possibility is found in the literature on
  restriction of output, as, for example, Donald Roy, ``Quota
  Restriction and Goldbricking in a Machine Shope,'' \emph{Amer. J.
  Sociol.}, LVII\textless{} 427--442. Another body of data is found in
  the psychological analysis of ``feeding tantrums'' on the part of
  children, where refusal to east serves as a way-in which attitude to
  one's parents is expressed; see, for example, Emmy Sylvester,
  ``Analysis of Psychogenic Anorexia and Vomiting in a Four-Year-Old
  Child,'' The \emph{Psychoanalytic Study of the Child}, I, 167--187.
  Accidents at work are perhaps an extreme example; see Karl Menninger,
  ``Purposive Accidents as an Expression of Self-Destructive
  Tendencies,'' \emph{Int. J. Psycho-analysis}, XVII, 6--16. The
  tendency for a given task performance to take on a ``project value''
  having to do with early experiences of the actor is illustrated in D.
  D. Bond, \emph{The Love and Fear of Flying} (New York: International
  Universities Press, 1952).}

Finally, acts which are traditionally taken, in a particular situation,
as expressive of the conceptions that persons have of one another can
themselves take on an extra superimposed layer of expressive
significance. Thus, for example, ceremonies such as greetings and
farewells, which usually express our approval of one another, may be
performed in a snide or fawning fashion, expression different additional
evaluations.

From all the events which might be employed as expressions, it is
apparent that one cultural group will stress the use of one type of
event and make little use of another type, while a different cultural
group will distribute its stresses and omissions in a different
way.\footnote{See, for example, Gregory Bateson and Margaret Mead,
  \emph{Balinese Character} (New York: New York Academy of Science,
  1942), pp.~74--83, where the apparent tendency of the Balinese to
  place special emphasis on the cardinal points and on differences in
  elevation as sources of sign-vehicles is considered.}

It is also apparent that social change can bring to a group an
alteration in the signs that are stressed by it. Further, it is to be
noted that as a consequence of social change, there may be a radical
change in the expression carried by a particular sign vehicle. This
possibility can be illustrated from the social history of Dixon.

In Dixon, patterns of social visiting and mutual aid regarding crucial
croft tasks have traditionally followed kinship and neighborhood ties,
so that informal participation, while an expression of lines of
solidarity, does not convey any information that has not long been taken
for granted. Failure to channel one's social participation along these
lines expressed the fact that persons once close to each other had had a
personal quarrel, a ``falling out.'' However, with the growing
importance of internal cleavages along social class lines, informal
participation is coming to take on a new meaning. Information
participation is coming to express class equality. Since class position
is subject to kinds of change and ratification that are not
characteristic of position in a kinship system or neighborhood circle,
informal parties in Dixon are coming to take on the ethos that is
characteristic of these gatherings in middle class Western society.
Where before these gatherings were taken in a calm way, as a matter of
course, they are now taking on a tone of excitement at the upper levels
of the class system and a tone of disappointment at the lower class
levels.

Another illustration of the shift in significance of social
participation is found in the case of the twice-monthly whist-socials
held during the winter months. Until about 1950, invitations to these
socials were ``open;'' anyone wanting to come to the first half, which
consisted of progressive whist, was welcome; anyone wanting to come to
the second half, which consisted of a dance, was also welcome, whether
or not he had come for the first part. During intermissions at whist,
tea and sandwiches would be served as a collective social operation; the
eight or nine persons acting as organizational hosts would bring food
from the kitchen and serve everyone in the hall in rotation from
platters of sandwiches and single pots of tea. Seating during the tea
was of no great importance and expressed kinship ties, neighborhood
ties, and age-grade intimacy. Since the tables were arranged in a
continuous chain around the hall, choice extorted to the right of one
was sometimes not accompanied by choice exerted to the left of one. In
any case, each participant had enough ties with any other participant to
sustain informal interaction for the period of the intermission. During
1950 a new institution was introduced from the other and more
``advanced'' islands. It was called a ``hostess social'' and entailed a
radical alteration in the traditional invitation and catering pattern.
The dance during the second half of the social remained open to
everyone, but participation in the first part, the whist, was by
personal invitation only. Invitations were extended by about fifteen
women selected by the organizing committee as ``hostesses.'' Each
hostess invited enough guests for two or more ``tables'' of whist,
i.e.e, two or more sets of four persons. As usual, the tables were
arranged in a continuous chain around the hall, but at meal time the
chain was broken and each hostess was given her own area in the hall and
her own tables. Hot water was collectively organized, but the rest of
the food was handled separately for each cluster of tables, the hostess
being responsible for bringing food for her own set of tables. This
pattern for organizing food distribution brought hostesses into
competition and comparison with one another regarding number of tables
invited, elaborateness of spread, etc. It also provided the community
with a new way of seeing at a glance the cleavages in the community. By
and large, a hostess still fills her quota with members of her family or
neighborhood circle, and by and large anyone who wants to obtain an
invitation can readily get one, but a tendency is apparent to select
guests on a basis of class equality ties. New participation patterns
such as these are, of course, both cause and effect of the emergence of
class cleavages within the crofter population.

We are now in a position to summarize the characteristics of expressive
behavior. In doing this a contrast will be made with the
characteristics, as previously reported, of linguistic behavior.

\begin{enumerate}
\item
  Expressive behavior provides information that cannot be precisely
  formulated or defined, and, in an important sense, the persons of whom
  the behavior is expressive cannot be made officially and formally
  responsible for the information they have made available about
  themselves. (Linguistic behavior, on the other hand, can be precisely
  defined, and the person who communicates it can be made responsible
  for his communication.)
\item
  Information conveyed by expressive signs is not discursive and does
  not form part of an extended logically integrated line of reasoning.
  Typically, only certain general facts can be conveyed by expressive
  behavior, these having to do with the actor's general alignment or
  attitude toward whatever instrumental activity he happens to be
  engaged in at the time or toward the social situation which he happens
  to be participating in at the time. (Linguistic behavior, on the other
  hand, can form an extended line of argument, and the object of
  reference which it has may, but need not, concern the actor's general
  alignment to the situation.)
\item
  Expressive behavior is ``uncalculated,'' or, to use a dubious term,
  ``involuntary;'' the expressive aspect of behavior is felt to be the
  sort of thing that one ought not to modify out of a desire to
  influence the response to oneself that the recipient will make because
  of the information carried by it. (Linguistic behavior is one type of
  instrumental behavior, and it is felt proper to have employed it with
  the consequences in mind that it is likely to call forth.)
\item
  Expressive behavior is an intrinsic part of the object which it
  carries information. The object may be a characteristic of a person or
  a characteristic of a set of persons in interaction with one another.
  Expressions are not conventional signs, i.e., symbols; they are
  natural signs or symptoms. Logically speaking, the structure of
  expressive sign relationships is relatively simple, involving only two
  elements, a causal complex and a symptom of this complex. An
  expressive sign remains a sign even though there may not be an
  interpreter present who makes use of it as a source of information. If
  an expressive sign is made use of, however, then it is essential that
  the interpreter be able to identify the causal complex which is
  responsible for the presence of the sign. (Linguistic behavior, on the
  other hand, is not an intrinsic part of the object to which it refers,
  but a conventional symbol of it. Linguistic sign relationships are
  logically complex, involving a minus of three elements: sign, object
  of reference, and interpreter. The causal complex responsible for the
  sign, namely the sender, is not an essential part of the relationship,
  although a frequent one. If a linguistic sign is not interpreted, it
  ceases to be a sign.)
\end{enumerate}


% CHAPTER V: THE MANAGEMENT OF INFORMATION ABOUT ONESELF
\chapter[CHAPTER V: THE MANAGEMENT OF INFORMATION ABOUT ONESELF]{Chapter V: The Management of Information About\\ Oneself}
\label{ch:Chapter V: The Management of Information About Oneself}
\chaptermark{CHAPTER V: THE MANAGEMENT OF INFORMATION ABOUT ONESELF}

\begin{quote}
There are some additional qualifications necessary, in the practical
part of business, which may deserve some con­sideration in your leisure
moments---such as, an absolute command of your temper, so as not to be
provoked to passion upon any account; patience, to hear frivolous,
impertinent, and unreasonable applications; with address enough to
re­fuse, without offending; or, by your manner of granting, to double the
oglibation;---dexterity enough to conceal a truth, without telling a
lie; sagacity enough to read other people's countenances; and serenity
enough not to let them discover anything by yours---a seeming frankness,
with a real reserve. These are the rudiments of a politician; the world
must be your grammar.\footnote{\emph{Letters of Lord Chesterfield to His
  Son}, Everyman'ss ed. (New York: Dutton, 1929), p.~41.}
\end{quote}

\newthought{In social life,}
\marginnote{\href{https://doi.org/10.32376/3f8575cb.a475d25e}{doi}}an actor commonly finds that very basic ends, of both an
ultimate and intermediate kind, are furthered by gathering information
about those with whom he interacts, es­pecially information about the
conceptions that these persons have of themselves and of him. With
information about others, the actor can predict in general their likely
behavior, and pre­pare for it. With information or this kind, he can
determine how best to shape his own behavior in order to call forth a
desired action from others. (The exploitation of the indicated likely
response of others to his own behavior is required, of course, whether
the actor wishes to please or to displease the others.) With information
of this kind, the actor can learn what is expected of him and ``where he
stands'' with respect to the others, helping thus to determine for
himself who and what he is. We find, then, a whole complex of ends, any
one or more of which may motivate the actor to the same kind of
activity, i.e., an effort to find out as much as possible about the
persons with whom he interacts.

The expressive function of behavior has to do with the tendency of
events associated with the actor to carry information about the actor.
This process has, intrinsically, nothing whatsoever to do with
communication of a linguistic and intentioned kind. Expressive
information is there whether or not anyone realizes this to be the case.
Specialists in ``human relations'' recognize that they must exploit all
sources of information, linguistic and expressive, that are available to
them, since it is appreciated that the subject's linguistic version of
the situation may be a highly biassed {[}\emph{sic}{]} one. Thus, we are
given advice of the following kind:

\begin{quote}
When you talk with the patient, you should listen, first, for what he
wants to tell you, secondly, for what does not want to tell, thirdly,
for what he cannot tell.\footnote{L. J. Henderson, ``Physician and
  Patient as a Social System,'' \emph{New England J. of Med.}, CCXII
  (819--823), 822.}
\end{quote}

\noindent It is interesting to note that certain patients also make use of
expressive cues as a source of information. Fromm-Reichmann, for
instance, makes the following suggestion concerning schizophrenic
patients:

\begin{quote}
The schizophrenic's ability to eavesdrop, as it were, on the doctor
creates another special personal problem for some psychiatrists. The
schizophrenic, since his childhood days, has been suspiciously aware of
the fact that words are used not only to convey but also to veil actual
communications. Consequently, he has learned to gather information about
people in general, therefore also about the psychiatrist, from his
inadvertant {[}\emph{sic}{]} communications through changes in ges­ture,
attitude and posture, inflections of voice or expres­sive
movements.\footnote{Frieda Fromm-Reichmann, ``Notes on the Development
  of Treatment of Schizophrenics by Psychoanalytical Psychotherapy,''
  \emph{Psychiatry}, XI (263-273), 273.}
\end{quote}

\noindent But, of course, this kind of detective work goes on constantly in
non-professional situations; in every interaction, each participant is
both patient and doctor. In Ichheiser's terminology, sources of
\emph{expression} of one person comes to be sources of \emph{impression}
of him for other persons.\footnote{Gustav Ichheiser provides a clear
  statement of the difference between expression and impression on
  pp.~6--7 of his mono­graph, ``Misunderstandings in Human Relations,''
  Supplement to \emph{The American Journal of Sociology}, Sept.~1949
  (Chicago: University Chicago Press, 1949).} Of all the actu­al sources
of expression that exist concerning any actor, those which occur while
the actor is engaged in linguistic communication are perhaps the most
important. These sources are important because if a recipient is in a
position to receive a linguistic communication in a face-to-face
context, he is also thereby in an excellent position to observe the
sender closely. It should be noted that a linguistic message involves a
certain amount of ac­tive communication, transmission, or ``sending'' on
the part or the sender and a certain amount of passive receptivity on
the part of the recipient. In the case of expressive information, on the
other hand, the impression or message is not so much sent as it is
taken, the message is not so much communicated as it is conveyed; here
the recipient must in many respects play a more active role than the
sender.

It has been suggested that a whole complex of ends is served for the
actor when he obtains information about the other. Consequently,
recipients of a linguistic message tend to scruti­nize the expressive
behavior of the sender of the message. In many cases, of course, the
sender has as good a motive for try­ing to prevent the flow of
information about himself as his ob­servers have for seeking this
information. The only end that cannot be served for the sender by the
exercise of information control is that of free and spontaneous
self-expression, and this is not so much an end of action as it is a
characteristic of the behavioral impulse. The sender typically exerts
tactical con­trol, therefore, over his linguistic communication. He also
tends to exert control over the expressive component of his be­havior in
an attempt to influence the response that recipients are like to have to
it.

In general, a person who wishes to exert control over the information
about self which others are able to acquire about him may communicate
misinformation, inadequate information, or unserious
information.\footnote{Professional rules in service trades sometimes
  explic­itly deal with the degree to which different kinds of
  mis-communications are officially permissible. See, for example, E. C.
  Hughes, ``Study of a Secular Institution: the Chicago Real Estate
  Board,'' (Unpublished Ph. D. dissertation, Department or Sociolo­gy,
  University of Chicago, 1928), p.~85: ``The line between
  misrepresentation and truth is hard to draw. The code deals with three
  types of questions regarding truth, (1) over-statement, (2)
  understatement of unfavorable facts, and (3) silence regarding
  significant facts which have not been asked for.''}

Misinformation may be communicated by linguistic signs. This may be
called deceit. In immediate communication, deceit does not occur
frequently except under special circumstances, as in the ``white lie.''
Persons who practise more serious deceit, e.g., the ``bare-faced lie,''
and who are detected, place them­selves in an almost indefensible
position. Misinformation may be conveyed by expressive signs. This may
be called feigning. Feigning occurs quite regularly, partly because the
signs em­ployed may refer to mental or emotional states which no one can
completely prove that the sender does or does not possess.\footnote{In
  the case of feigning or dissimulation, the sender appreciates that his
  expressions are ``false'' and misinformative; they are employed, in
  fact, precisely in order to throw the observer ``off the scent.'' A
  more important communication be­havior is that of affectation, where a
  sender's expressive ges­tures are seen to be a product or calculation
  and design, while at the same time the impression is given that the
  sender is at least partly taken in by his own act and partly convinced
  that he is in fact the sort or person that his affectations attempt to
  establish. Affectation or posing is a crucial communication
  possibility and will be considered in chap.~xix.}

Inadequate information may be communicated by linguistic signs. A person
who acts in this way is said to lack candor and frankness, to be
close-lipped. When this reticence pertains to specific issues, as it
may, for example, in the criminal world, we speak of clamming up. When a
person provides inadequate ex­pressive information, we sometimes think of
him as being ``cold'' or disdainful. Some games of chance are
specifically designed to give play to the faculty for expressive
constraint, and the term poker-face, starting as a word applying to a
game, has become widely used in ordinary social contexts.\footnote{It is
  an interesting fact that in some cultures the practice or making no
  expressive response in some situations is institutionalized in the
  form of a slight smile, this sign be­coming a way of openly
  communicating that one is not expressively communicating.}

The transmission of misinformation and inadequate in­formation appears to
be a very general practice, although we have little systematic knowledge
as to where in a given social structure it is practised the most and
with what degree of success in carrying conviction.\footnote{For a
  consideration of a social situation in which there arises the use of
  statements that are technically true but by im­plication false or
  insufficient, see Fritz J. Roethlisberger, ``The Foreman: Master and
  Victim of Doubletalk,'' \emph{Human Factors in Management}, ed.~S. D.
  Hoslett (New York: Harper, 1946), pp.~51--73, especially pp.~58--59.}
Recently Margaret Mead has given us an extreme example:

\begin{quote}
With this requirement, that all behavior be controlled and directed
toward Party goals, goes the requirement that the Party member treat
himself as a tool to carry out the wishes of the Party, but that he be
at all times a conscious tool, voluntarily submitting himself to the
discipline of the Party. And the discipline must be minute and detailed,
over himself and over his every movement. So an informant reports an
encounter with a Soviet professor in Berlin, who told her that he smoked
a pipe ``because while smoking a pipe the fact does not reveal so
much.'' Then he added: ``See, this we learned during the Soviet period.
Before the revolution we used to say: `The eyes are the mirror of the
soul.' The eyes can lie---and how. You can express with your eyes a
devoted attention which in reality you are not feeling. You can express
serenity or surprise. I often watch my face in the mirror before going
to meetings and demonstrations and \ldots{} I was suddenly aware that
even with a memory of a disappointment my lips became closed. That is
why by smoking a heavy pipe you are sure of yourself. Through the
heaviness of the pipe the lips become deformed and cannot react
spontaneously.''\footnote{Margaret Mead, Soviet Attitudes toward
  Authority (New York: McGraw-Hill, 1951), pp.~65--66.}
\end{quote}

\noindent In Dixon, the practice of conveying misinformation or inadequate
information seems well developed. Frustrations that occur in the pursuit
of everyday tasks are rarely a cause of outbursts of anger and are
usually taken with apparent calmness and coolness, as a matter of
course. When a housewife accidentally burns her finger on a hot pot or
puts too much salt in a soup, when two men have to try time and time
again to move a cow from one field to another, when a fish net gets torn
on the rocks of the floor of the inlet and must be drawn up for a day's
work of mending---in these and many other daily frustrations very little
emotional expression is allowed to escape. Occurrences which call forth
frustration and deprivation are said, merely, to be
``awkward.''\footnote{British gentry have a somewhat similar affective
  stress, labelling extremely deprivational events ``a bore.''} So, too,
when a young man leaves Dixon for a year or two to work as a seaman, or
when he returns after having been away for this reason, his womenfolk
will bid farewell to him at the pier, or greet him there, without
kissing him and with very little show of emotion, yet family ties seem
to run extremely deep. And, finally, persons who are not Berganders are
treated with politeness and distance, and any opinions which they
express are often answered with very mild agreement; so smooth is this
treatment that outsiders frequently never realize how little they have
learned.\footnote{It is sometimes claimed that ``expressive'' withdrawal
  on the part of crofters in Bergand is related to their historical
  status in relation to the Lairds. Until the crofting act of 1895, a
  landlord had the right to increase rent without warning and in any
  amount. Widely appreciated tradition has it that any show of
  prosperity on the part of a crofter would have immediately led to an
  increase in his rent. Hence, it is felt that crofters had no motive
  for ``bettering'' themselves, and that there was a natural reason to
  conceal, physically and verbally, the slightest gain in wealth and
  one's plans and feelings in general.}

It has been suggested that a sender may convey misin­formation, both
linguistically and expressively, and that he may convey insufficient
information, both linguistically and expres­sively. In the interests of
completeness, a final possibility must be considered. The sender may
convey unserious information. A complex relation between context,
expressive cues, and linguis­tic content of the message establishes the
assumption that recipi­ents are not to give credence to the sender's
message and that the sender is not to be made responsible for what he
has said. Recipients are officially meant to understand that what is
con­veyed to them, especially the linguistic components of the mes­sage
they receive, is exactly what the sender does not believe; what the
sender does believe is left an open question. The right to be unserious
is a right to play at communication\footnote{The best treatment of
  unseriousness that I know of may be found in Kurt Riezler, ``Play and
  Seriousness,'' \emph{J. of Philosophy}, XXXVIII, 505--517. A sender
  may (for many different motives) play at play, or, in this case, treat
  unseriousness unseriously, by attempting to keep the note of levity
  from his voice as long as possible and by attempting to mimic
  completely the serious expressive tone normally associated with the
  linguistic message being sent. This communication game will be
  considered later.}; it repre­sents an important communication license
and it is employed in many different ways for many different purposes.
It should be understood that almost any particular piece of information,
linguistic or expressive, can be communicated seriously or unseriously,
depending on the context and the spirit of the communica­tion. A decision
on the part of a sender to treat a matter unseriously is, of course, a
quite serious thing in most cases; this decision can be employed by
recipients as a source of impression about the sender. It would seem
that efforts on the part of a sender to define his message unseriously
are usually acknowledged and accredited by recipients. If a sender's
at­tempt to maintain an unserious definition of the situation is
unacceptable to recipients, they cannot hold him responsible for the
linguistic component of his message but they can hold him responsible
for a breach of good taste and for improper joke­ making. In Dixon, the
art of unseriousness seemed to be highly developed. Three special ways
of talking are set aside for this purpose: simulated baby-talk;
simulated Public School ac­cents; and formulation of a message in
no-longer-current forms of Bergand dialect. ``Straight-faced'' teasing
and mock affront are also employed extensively.\footnote{Another
  favorite mode of unseriousness, but of a some­what more formal kind, is
  play-acting---a very popular form of amusement on the island. During
  the year, skits and plays are put on at least two socials in each of
  the communities. Cer­tain kinds of skits, such as ones involving
  ministers, are felt to be in bad taste by some of the more
  ``fundamentalist'' of the crofters. In Capital City, where an annual
  drama festival is held, plays that are too realistically dramatic,
  such as Synge's \emph{Riders to the Sea}, are also felt by some to be
  improper vehicles for the stage.}

It has been suggested that a sender may attempt to exert control over
the other's response to him by inhibiting his spontaneous response to
the situation and conveying, instead, misinformation, inadequate
information, and unserious information. Thus an actual message may
contain information which purposely obscures from view the real feelings
and thoughts of the sender. Recipients may, of course, come to realize
that the sender is interested in controlling the impression given, and
they may come to anticipate a distortive or tactical element in the
mes­sages they receive. In order to get through the screen of
dis­tortions, evasions, omissions, etc., to the ``real'' feelings and
conceptions of the sender, recipients may have to examine each message
to find out what can be accepted at ``face value'' and what has to be
analyzed and translated so as to reveal the real information that is
hidden by it.\footnote{One decoding method is to treat the sender's
  choice from the many possible modes or distortion as an expression or
  the sender.} In the terminology of crypto­graphy, the recipient may
find that part of the message is ``clear,'' that is, it can be taken at
its face value, and that another part or the message is ``coded,'' that
is, it is distor­tion of some kind and must be decoded before providing
truthful information.

We usually feel that clear information is conveyed by expressive
emotional behavior during times of crisis and that coded information
comes to us through linguistic messages when a person is ``on his
guard.'' This may be usually the case but it is not necessarily so.
Occasionally a sender communicates linguistically his real feelings and
thoughts. Thee significant point here is that moral norms seem to
develop regulating the amount and the place for clarity and coding in
messages in a particular situation. Further, the sender and the
recipient each develops his own version as to what part of the message
is coded and what part is clear; they each develop a version of the
discrepancy between clear and coded parts of the message. In
con­versational interaction, then, we find an interesting set of
discrepancies: a socially permitted discrepancy defined as ap­propriate
for situations like the given one; the ``objective'' discrepancy which
in fact exists between the coded and the clear information in the given
situation; the opinion of the sender and of the recipient as to what is
in fact the objective discrepancy between the clear and coded parts of
the message. Discrepancies between these several discrepancies provide
one way of describing certain kinds of interpersonal communication
problems.

When persons are engaged in conversational interaction, those who are
recipients seem to participate in two streams of signs, linguistic signs
and expressive signs. At the sane time, those who send messages of a
linguistic kind seem to participate chiefly in the purely linguistic
aspect of their own behavior. Thus, if the term communication be
employed broadly to cover the process by which a recipient acquires both
streams of signs (re­ceiving one, taking the other), then we see that
communication is usually asymmetrical; the sender is involved in one
stream of signs, the recipients in two. As Simmel suggests:

\begin{quote}
\ldots{} all of human intercourse rests on the fact that every­body knows
somewhat more about the other than the other voluntarily reveals to him
\ldots{}\footnote{Simmel, \emph{op. cit.}, p.~323.}
\end{quote}

\noindent When the expressive stream of signs is cut off, as it can be in mediated
communication, then real problems of understanding arise. This is nicely
described by Whyte in his discussion of the difficulties arising in
restaurants from the use of mechani­cal devices for transmitting orders
from waitresses to cooks:

\begin{quote}
To build better teamwork in the supply system, management needs to think
in terms of communication and status. We have seen clearly that
mechanical devices are not an adequate substitute for face-to-face
communication. Nor is this simply because the particular words that come
over the pub­lic address system, teleautograph, or phone are sometimes
misunderstood. We make our judgment as to other people not alone through
the bare words we utter but through the way we express ourselves,
through our gestures and facial expres­sions, and through our past
experience with this relation­ship, which tells us how to interpret the
other man's be­havior. Nearly all of this background for adjustment,
understanding, and cooperation is lost when people are separated so that
interaction is filtered through mechanical devices.\footnote{William F.
  Whyte, \emph{Human Relations in the Restaurant Industry} (New York:
  McGraw-Hill; 1948), p.~60.}
\end{quote}

\noindent In his essay entitled, ``Behind Our Masks,'' Park makes much the same
point:

\begin{quote}
It is curious and interesting that this character that we call human
should be so intimately connected with expres­siveness. Human interest,
as we ordinarily use that phrase, attaches to anything that is
``expressive''; that is, to any­thing that suggests, symbolizes or
reveals sentiments and passions in others of which we are immediately
conscious in ourselves. The faces we know have no secrets for us. For
that reason, if for no other, we feel secure and at home with them as we
do not among less familiar faces\ldots. One of the first and most
important discoveries the one who meets an alien people for the first
time is likely to make, is that, different as they seem, most strange
people, when you come to know them, turn out to be human, like
our­selves. It always requires an effort of imagination to realize this.
It is because their faces are for us not ex­pressive; and we, in turn, do
not respond to sentiments whose expression we are not able to
read.\footnote{Robert Ezra Park, \emph{op. cit.}, pp.~252--253.}
\end{quote}

\noindent We have suggested that a sender often has reason to at­tempt to control
the response that his messages evoke and hence has reason to attempt to
control ``with malice aforethought'' the expressive component of his
behavior. A limited amount of this instrumental use of an essentially
non-instrumental aspect of behavior is socially permissible, especially
in certain situa­tions, as in greetings and farewells, where a certain
amount of ungenuine expressive behavior is permissible, although strong
sanctions are exerted against those who are felt to have affected
expressive behavior at inappropriate times. Further, a certain
additional amount of calculated employment of one's expressive behavior
is no doubt accomplished without detection. In general, however, it
seems that a sender cannot tamper with what ought to be the expressive
component of his communication, or even became aware of the probable
effect on recipients of this component, without this concern itself
being communicated in an expressive way. The asymmetrical character of
the communication process thus remains, but it occurs at a somewhat
different level. The recipient checks up on the linguistic component by
means of what ought to be the genuinely expressive halo of signs
associated with it, and then checks up on this presumed expressive
behavior by examining the fugitive stylistic features of it that are
al­most impossible to feign.

It has been suggested that a sender is often motivated to restrict the
information which he advertantly or inadvertantly makes available about
himself, and that a recipient is motivated to acquire as much
information about the sender as possible. We often find, then, that
conversational interaction involves a constant game of concealment and
search, and that in this game a given player will usually be better at
the task of discovering things about the other than at the task of
concealing things from him.\footnote{Perhaps here it should also be
  noted that in circles conversant with Freudian doctrine, ``slips'' are
  taken, jokingly or otherwise, as a revelation of ``real'' feelings;
  since slips are apparently in no way subject to conscious control,
  they may con­vey embarrassing information, or what is taken to be
  embarrassing information. The doctrine that slips are ``significant''
  adds new hazards to the concealer, gives new power to the searcher.}
For example, a crofter's wife has admitted to me (and I have also
observed) that in order to find out whether a guest ``really'' likes the
food he is being served, she does not listen to his words of praise,
which courtesy demands of him, but observes the rapidity with which he
raises the food to his mouth and the zest with which he chews it. Such
cues to the attitude of an actor are extremely difficult to distort.
Extreme applications of this game of concealment and search, operating
at an institutionalized level, are found in the application of modern
projective testing and the use of laboratory police methods.

For a close analysis of conversational interaction, it is useful to give
consideration to the level of care that a re­cipient feels he must exert
in regard to the reception of a par­ticular message. It is also useful to
know to what lengths the recipient feels he must go in order to find
clear information in the message or in order to decipher the coded
information. At the lowest level of sophistication in the game of
concealment and search we find this: the sender transmits a message
which he implicitly assumes the recipient will take at face value, as
consisting wholly of clear information; at the same level the recipient
assumes that the message contains clear and coded com­ponents and that
the sender is unaware that this discrimination is being made. At the
next level of sophistication, the sender takes into consideration the
recipient's double assumption, namely that the message has both clear
and coded components and that the sender is not aware that this
discrimination is being made by the recipient. In poker and in other
social contexts, this is called bluffing. At the same level of
sophistication, the re­cipient maintains the usual asymmetry of the
communication proc­ess by perceiving the bluff and guiding his response
accordingly. At a third level, we have the practice whereby a sender
bluffs that he is bluffing and a recipient bluffs that he is being taken
in by a bluff. An infinite regress is imaginable, but three levels in
this process seem to be all that we need to con­sider in most situations.

\vspace{.2in}
\begin{centering}

\Large{* * * * *}

\end{centering}
\vspace{.17in}

\noindent In Chapter IV, a clearcut analytical distinction was made between
linguistic behavior, as an intentioned and instru­mental activity, and
expressive behavior, as an impulsive, non-rational aspect of behavior,
having nothing to do with communica­tion in the narrow sense of that
term. However, in the present chapter it has been necessary to recombine
these two modes of behavior in an intricate way. For one thing,
recipients exploit the expressive behavior of a sender as a source of
impression about him. Secondly, the sender may attempt to exploit the
fact that this exploitation occurs and attempt to express himself in a
way that is calculated to impress the recipient in a desired way. What
on the surface is expressive behavior becomes, then, in a sense,
instrumental behavior.

Let us restate and amplify slightly the ways in which ex­pressive
behavior and linguistic behavior intermingle and com­plicate each other.
First. we find that the recipient, unbe­knownst to the sender, derives
information about the sender by examining his expressive behavior. This
expressive stream of signs does not involve the sender in an active
communicative role; it adds to his active role as a linguistic
communicator a passive role as an expressive object. If the sender then
becomes aware that he has conveyed two streams of information, one
linguistic and one expressive, his role as an expressive object becomes
a little less passive. Further, the sender may realize beforehand that
the expressive component of his behavior will be audited, and attempt,
in a surreptitious way, to enact the kind of expressive behavior which
is likely to call forth the kind of response that he wishes to evoke
from his recipients. The sender may be successful in his bluff or
detected at it; in either case, ex­pressive behavior takes on an
instrumental function and becomes an activity more closely akin to what
we think or as communication. Finally, the sender may modify his
expressive behavior ``with malice aforethought,'' at the same time
openly conveying that this calculated display of expressive behavior is
intentional. This sometimes seems to be the case with certain
conventionalized gestures of respect and approval and with ``etiquette''
in general.\footnote{It is extremely difficult to make judgments
  concerning the degree of unselfconscious spontaneity involved in the
  per­formance of a given piece of social ritual or ceremony. It seems
  that one can say, however, that forms of etiquette which seem
  arbitrary, empty gestures at the time they are consciously learned may
  easily come, with the passage of years, to be unthinking and genuinely
  expressive aspects of one's behavior.} Here a certain amount of
feigning (and deceit) is felt to be socially permissible. In these
cases, expressive behavior becomes an active form of communication.
However, in­tentionally employed expressive behavior does differ from
lin­guistic behavior in significant ways. Intentionally employed
expressive behavior, however ``conventionalized'' it becomes, must take
the form of behavior that could possibly be unselfconsciously or
spontaneously expressive, and, as in the case of genuinely expressive
behavior, the sender cannot be made officially re­sponsible for having
conveyed the information carried by it. Thus we are able to see that
official signals, such as raising one's hand for attention, which form
part of linguistic behavior, may be quite similar in appearance to
intentionally employed ex­pressive gestures, such as raising one's hat to
a woman, and yet the first sign is part of an officially accredited
aspect of communication---what might be called the formal aspect---while
the second sign is part of an unofficial or informal aspect of
communication.

We find, then, when we examine persons engaged in con­versational
interaction, that a very complex dialectic is in progress between the
formal or linguistic component of communication and the informal or
expressive component of communication and that the latter itself
contains a host of messages which differ among themselves in the degree
to which they approach what we usually think of as active communication.

\newpage It has been suggested that both sender and recipient can be made
explicitly responsible for a linguistic message, the sender for having
transmitted it and the recipient for having received it. Playing in and
around this major theme, there are many variations and melodies, which
now embellish, now oppose, now develop, now reinforce, the dominant
theme. These varia­tions are conveyed by expressive behavior and provide
for a some­what irresponsible kind of communication; the sender cannot be
explicitly and specifically held accountable for the content of these
messages, and the recipient has the right to act as if he has not
received them. The contrapuntal interweaving of responsible transmission
with irresponsible transmission seems to make for flexibility. Persons
can give lip service to a message that has been accepted or that is
acceptable, while at the same time conveying by informal expressive
means information that would disrupt the working acceptance if it were
conveyed lin­guistically.

Perhaps the most obvious example of the flexibility pro­vided by the
interweaving of the two components of communication is to be found in
what is called in everyday terms ``innuendo.'' Innuendo occurs when the
informal expressive component of communication carries information that
is radically opposed to or different from the information carried by the
linguistic component and when, at the same time, the sender conveys the
fact that he expects the recipient to be impressed by this divergent
expressive component. Sometimes innuendo is used by a sender to convey
compliments which he is not in a position to deliver explicitly;
usually, apparently, innuendo is employed as a means of conveying
disagreements, criticisms, and depreciatory judg­ments which might put
the working acceptance in jeopardy were they conveyed linguistically.
Often innuendo is conveyed by statements made
``unseriously.''\footnote{In general there is a sense in which the two
  forms of communication license, unseriousness and innuendo, are the
  re­verse of each other, even though they may both add flexibility and
  adaptability to conversational interaction. In the case of
  unseriousness, the recipient is obliged to overlook the lin­guistic
  component of the message, which, if taken seriously, would disrupt the
  working acceptance. In the case of innuendo, the recipient is obliged
  to overlook the expressive component, which, if conveyed
  linguistically, would disrupt the working acceptance. Innuendo
  provides an explicit agreement while con­veying the tact that one
  really doesn't exist; unseriousness provides an explicit disagreement
  while conveying the fact that one really exists. When unseriousness is
  pressed into the serv­ice or innuendo we obtain a serious use of
  unseriousness; the great frequency with which this communication
  arrangement is em­ployed should not lead us to underestimate the
  complexity and subtlety of the arrangement.}



% CHAPTER VI: INDELICATE COMMUNICATION
\chapter[CHAPTER VI: INDELICATE COMMUNICATION]{Chapter VI: Indelicate Communication}
\label{ch:Chapter VI: Indelicate Communication}
\chaptermark{CHAPTER VI: INDELICATE COMMUNICATION}

\enlargethispage{\baselineskip}

\newthought{It was suggested that}
\marginnote{\href{https://doi.org/10.32376/3f8575cb.08d178df}{doi}}the actor, as a participant in the game of
concealment and search, exerts self-control over in­formation about
himself which he provides to others. Whether properly or improperly,
whether he is or is not detected in his effort, the actor guides some of
his communications by an appre­ciation of their likely effect upon the
persons who receive them, this appreciation being guided in turn by the
indications that recipients make of the response a proposed line of
action will evoke from them. Spontaneous expression of feelings is
partly inhibited and approriate feelings are, in part, conveyed.
Ac­commodation, working acceptance, and tentative harmony are the usual
result. If a working acceptance cannot be managed, embarrassment,
ill-ease, and confusion are often the result. With­drawal, conflict, or
abrupt alteration in relationships may also occur. In all of these
cases, however, the individual knows that he is communicating and knows
to whom he is communicating. Although he may be unaware of all that he
communicates, he is in a position to exert discretion over a part, at
least, of what he conveys, and he is in a position to make use of what
he can learn by examining closely the indicated response of recipients
to him. If he is not able to exert prior calculation over all that he
conveys, he is at least in a position to benefit from a similar
incapacity on the part of those who respond to him. Thus, whether we
examine cases of working acceptance, withdrawal, con­flict,
embarrassment, or shift to alternate bases of treatment, we find the
general factors of awareness and partial control.

There are a number of marginal situations, however, in which the general
factors of awareness and partial control can­not operate, or are not
allowed to operate. In these situations, the actor finds himself under
direct observation of others but is not in a position to modify his
behavior by means of indica­tions of the response he is calling forth
from them; corrective feedback is not possible. The strategies which the
actor usually employs to protect himself, to protect those about whom he
is talking, and to protect the interaction itself, cannot be employed.
We shall refer here to indelicate communication arrangements.

One form of indelicate communication occurs in those ­professional-client
relationships for which certain forms of social irresponsibility are
heavily institutionalized. Priests, psychiatrists, and lawyers
convincingly guarantee a client that certain kinds of reprisals and
moral judgments will not be made, no matter what the client conveys to
them by word or deed. It is in the client's self-interest to be honest
and frank, while at the same time the professional defines himself as
someone who cannot and does not take offense. In such a context, it is
pos­sible for the client to maintain, of his own free will, a
com­munication situation which is so counter to his ordinary protective
strategies as chronically to cause him embarrassment.

Another form of indelicate communication occurs in those cases where a
person is explicitly obliged to speak honestly if not frankly and at the
same time accept the social consequences of having done so. Prenuptual
exchanges of confidences provide one example. Another example is found
in the technique employed in everyday conversation of turning on a
sender, interrupting him, and asking him in a special tone of voice if
something he has just said is actually and really true. Evidence given
under oath at a trial or hearing is a formal example. (This suggests one
reason why these scenes are frequently embarrassing.) In all these
cases, the sender is given special warning that any deviation from truth
will be fully held against him; he is thus forced to be untactful, to
contradict the image he has projected of himself or the image others
have projected of themselves.

The two forms of indelicate communication that have been
con-\\\noindent sidered---the first where a person is allowed to be untactful and the
second where a person is not allowed to be tactful---in­volve
communication situations where the usual amount of stra­tegic control
over the 1inguistic components of messages does not prevail. There are a
number of allied forms of indelicate communication which differ from
those mentioned in that the sender is not aware that he is not employing
the usual amount of calculation and control, or, if he is aware of his
lack of con­trol, he is free to exert control yet incapable of doing so.

We have the case where a person involuntarily frees him­self from the
inclination toward concealment, as in narco-hypnosis (and anesthesia
generally), and the case where he more or less voluntarily does the same
thing, as in inebriation. An inter­esting indelicate situation arises
when an actor comes under direct observation of a person of whose
capacity in this regard the actor is not aware. In everyday terms,
recipients of this kind are said to be spying. We are familiar, for
example, with the use of dark glasses and veils as a means of concealing
from an observed person the fact that he is being observed or as a means
of concealing the kind of response that observation of him is arousing.
So also, wall mirrors are sometimes used as a means of observing persons
who think they are not being observed or, at least, not being observed
from that angle. A few examples may be given from the field.

\begin{quote}
In Dixon it is a very common practice for persons in a cottage to look
out of the kitchen window every ten or fifteen minutes or when they hear
the croft collies barking. In this way, the inhabitants have ample
warning of the approach of anyone, and they have time to arrange
themselves and the room so that the image of thenselves which they wish
to communicate to the visitor will not be contradicted by what the
visitor sees. This mechanism of forewarning is apparently possible
because there are no trees and frequently no neighboring buildings to
obstruct the view from a window. A visiting crofter therefore feels that
the warning of a knock on the door is not necessary, and frequently
neglects to knock. (The visitor may, however, make a slight pause or
shufflilng sound before entering.) Since the rooms in a crofter's
cottage have very little light, it is possible to observe the approach
of a visitor without the visitor being able to discover that he is being
observed. Crofters enjoy the practice of comparing tbe facial expressive
behavior of a visitor just before he enters the cottage with his
behavior just after he enters.
\end{quote}

\begin{quote}
In the primary grades in the Dixon school, pupils some­times have to
withstand scrutiny by visitors. In some cases the stimulus of ``personal
interaction'' with strangers is too much for the pupils to handle with
equanamity; at the same time the ``news value'' of a visitor to the
school may be too great to allow a pupil to turn his attention
elsewhere. Some pupils em­ploy the solution of covering their faces with
their hand to shield themselves from the gaze of the stranger, while at
the same time examining him through the small openings between their
fingers.
\end{quote}

\begin{quote}
In Dixon there is a common practice of using a pocket telescope for the
purpose of observing one's neighbors without being observed in the act
of observing them. In this way it is possible to keep a constant check
upon what part of the annual cycle of work a neighbor is engaged in and
how rapidly he is progressing with it. It is also possible in this way
to keep informed as to who is visiting whom. (This use of telescopes is
apparently related to the physical distance between crofts, to the
absence of trees, and to the strong maritime tradition of the island.)
\end{quote}

\begin{quote}
In the hotel, the maids would observe the arrival of new guests through
the kitchen window. Differences in light intensi­ty made this a one-way
possibility and gave the maids a chance to arrive at an initial
appraisal of the new guests and to communicate this to one another
before it was necessary to have any actual contact with the guests.
\end{quote}

\noindent Just as persons can be observed without their knowing it, so also their
verbal exchanges can be audited without their knowing it. We are
familiar with the practice of eavesdropping on conversations, the
practice of reading another's mail, and the practice of listening in on
a party telephone line.

In the case of spying and the several kinds of eaves­dropping, the sender
is prevented from modifying his communication in accordance with its
probable effect upon the person who receives it because the sender is
not aware (or is not aware in time) that the person \emph{is }receiving
it. Senders may attempt to guard against such an eventuality by trying
to behave in such a way as to give no offense to the image any
unobserved observer might have of himself or of the sender.\footnote{In
  Western society there is amoral rule, often followed, that a person
  who happens into a situation where he can observe others or ovehear
  them, without this fact being apparent to the others, ought to warn
  them in some way by means of a tactful cue. This warning allows the
  others to take precautions against communicating anything that will be
  offensive in any way.} This kind of communication super-ego is
especially effective in guiding the conduct of persons who are in fact
unobserved by anyone.

An interesting complication in these indelicate communication
arrangements occurs when an individual is being spied upon or overheard,
knows this to be the case, and tries not to shatter the illusion of the
irregular recipient. In this case, the sender may feel that he is in an
excellent position to de­lude the recipient in any desired direction,
presumably on the grounds that the recipient will not exercise customary
scepticism in regard to what he oversees or overhears. This constitutes
a kind of bluff.

% CHAPTER VII: SIGN SITUATIONS
\chapter[CHAPTER VII: SIGN SITUATIONS]{Chapter VII: Sign Situations}
\label{ch:Chapter VII: Sign Situations}
\chaptermark{CHAPTER VII: SIGN SITUATIONS}

\newthought{When persons are in}
\marginnote{\href{https://doi.org/10.32376/3f8575cb.eba9bbba}{doi}}each other's immediate presence, and especially when
they are engaged together in conversational interaction, important
informational conditions obtain. Each participant is in a position to
convey information about himself both linguistically and expressively,
especially information having to do with his conception of himself and
his conception of the others present. An embarrassingly rich context of
events is available to serve as vehicles for signs of this kind; some of
these events are simultaneously part of the task organization in which
the persons find themselves, while some of these events serve no
explicitly recognized task.

When persons are in each other's presence in a given situation, a
definition usually prevails as to how each is to be treated. This
definition of the situation is made possible by the fact that each
participant possesses known determinants or qualifications which select
out for others which of the different possible categories of treatment
is to be accorded him. A corollary of this definitional process is that
all qualifications which a person possesses which act as selective
determinants of treatment in other situations but which are officially
denied as irrelevant in the current situation must be ignored. The
information which these irrelevant determinants or social
characteristics carry may be received, but there is an obligation on the
part of the recipient to act as if the information is in no way a
determinant or a selector of behavior. This involves, on the part of
recipients, suppression of response to information. Further, persons
must not bring forth such irrelevant determinants as are not already
apparent.

When persons are in each other's immediate presence and are engaged in
activity which provides vehicles that are well designed to express
actual or possible conceptions that persons have of one another, it may
inadvertently become difficult to suppress the flow of information that
is false or that has been defined as irrelevant. For reasons outside the
personal aims of any of the participants, and outside their personal
control, events may occur which are so aptly expressive of important
irrelevant valuations, or important potential but not actual valuations,
that participants will feel either that an improper evaluation has been
made or that other participants might feel this has occurred. Attention
is therefore drawn away from the activity that has been in progress and
is brought to bear, at an inappropriate time, on matters of status, and
especially on expressive signs by which relative rank is conveyed. At
times like these, tension over signs seems to develop and we have what
might be called a sign situation. In so far as any particular
participant is forced, through the flow of ordinary action in the
situation, to act in such a way as to produce a sign vehicle that is
accidentally and incidentally well suited to convey irrelevant or
incorrect social information, we may say that he is faced with a sign
situation.

Sign situations are constantly occurring; on the occasions when they do
not occur, they must constantly be guarded against. There are a fairly
large number of strategies of a preventive kind for avoiding the
occurrence of these difficulties and there are a fairly large number of
strategies of a corrective kind for resolving these difficulties when it
has proven impossible to avoid them. These strategies are so widely
known and used that we may think of them as institutionalized.

Perhaps the most obvious technique for handling a sign situation and
resolving the dilemma that it introduces is to employ a principle of
randomization. By means of this technique, an indulgence or a
deprivation can be differentials distributed (in time or by amount)
among participants in a way such that none of them is likely to
interpret the differential allocation as an expression of differential
status. Randomization, then, is a way of basing distribution on a
principle that is patently independent of differential status. Like
other such strategies, it is a way of ensuring that no offense will be
taken where none is meant but where offense is likely. Flipping coins,
cutting for high card in order to determine priority of play during a
game are common examples of randomization. Serial ordering of persons on
a basis of alphabetical priority is another case in point.

In the case of randomization, an extra-social principle is utilized as a
means of demonstrating that officially irrelevant qualifications have
not been employed. Another way of solving the same problem is to
distribute differential treatment in accordance with a social principle
of precedence which involves social qualifications in which no one
present is very actively concerned or in which there is very little open
to dispute. Thus those who approve of protocol claim that it is a device
not for expressing social distinctions but rather for preventing the
occurrence of such expressions. By taking note of every event which
might be taken by some as an expression of relative status or relative
approval, and establishing an order for these events based on
distinctions in rank established beforehand, assurance can be given that
nothing not already taken for granted will be expressed. Another
important example derives from service relations, where the principle of
``first come, first served'' is commonly employed. Thus customers are
induced to interpret the order in which they are served as expressive of
nothing more significant than their order of arrival. In the same way,
the principle of seniority is often invoked in formal organizations as a
tactful means of distributing differential rewards. (These practices
have the important incidental function of stressing the reality and
importance of the situation at hand as opposed to the reality of the
participants' irrelevant statuses.)

In this connection it should be noted that we have statuses of an almost
ceremonial kind, such as the very old, the very sick, the young, and the
``weaker sex''; and that in certain contexts the incumbents of these
statuses can be given preferences which are of very little value in
themselves but which might otherwise be allocated in an offending way.
The potentially troublesome privileges that are neutralized in this way
convey respect that is more akin to light pity than to envy. It should
be added that there are occasions when these statuses provide a
disturbing issue and at these times they cannot, of course, play a
merely ceremonial role. A woman who is an ardent feminist may be
offended if her sexual status is not allowed to remain irrelevant as a
determinant of treatment in certain kinds of situations, even though she
may appreciate that little significance is attached to the differential
treatment she is accorded. So also, a man who is not quite old may be
offended if he is given the empty privileges sometimes accorded to the
aged.

A very general way of dealing with a sign situation is that of apology
or exorcism, a verbal technique for convincing a potentially offended
person that no offense is meant or an offended person that the offense
was not intentional. Apologies frequently take the form of a
well-patterned interchange between offender and recipient of the
offense. In this way an act can sometimes be cleared of the expressive
function that has been or might be imputed to it.

By merely entering a conversation or place where conversation may occur,
an individual performs an act that is well designed to serve as a
vehicle for expressive interest, involvement, or approval. Similarly, by
leaving a place where communication is occurring, the individual
performs an act that could easily be taken as expressive of
disinvolvement, lack of interest, and disapproval. However, on many
occasions an individual may desire to enter or leave the communication
presence of others for instrumental reasons unrelated to the expressive
use to which these acts lend themselves, or for social reasons which he
desires to conceal. The individual is thus faced with a sign situation.
If the technical motive for the act is sufficiently clear and urgent,
then this fact alone usually seems enough to resolve the tension, giving
the individual license to more or less neglect the potential
expressiveness of his act. Everyone is willing to make allowances for
emergencies; the chief problem is to convince others that one's behavior
has been suddenly determined by one.\footnote{According to
  book-etiquette, the expressive implications of refusing an invitation
  must counteracted, at least formally, by reference to a real or
  fictive prior engagement or obligation of some kind. An invitation,
  once accepted, may be broken tactfully only on grounds of sudden
  business, ill-health, or death of a relative. Similarly, on leaving a
  party early, some ``legitimate'' excuse must be given.}

A related strategy is based on the use of ``natural breaks'' in
communication. Persons frequently postpone their arrival or departure
until such time as its potential expressive value is minimal.

The social life of Dixon is full of ways in which persons may be given
offense unintentionally and of ways in which the likelihood of doing
this can be avoided or neutralized. In the shops, customers, regardless
of sex, age, kin or class status, are served in order of priority of
presence in the shop; a shopper who breaks this order must broadcast a
very good reason for doing so. Those who organize the annual concerts
make sure that all three communities are represented as performers so
that no one community will take offense. At billiards, right of play is
determined by the number of games one has waited, and first play at the
start of any particular game is determined by flipping a coin or
guessing which hand contains the ball. There seem, in fact, to be few
situations where the actor does not have to ask himself the questions:
am I being tactful; will I be thought unfair. The larger the number of
events that can serve as signs, the more difficult the problem. Perhaps
it is in face-to-face communication that the greatest number of events
occur which might possibly be taken as expressive. Hence face-to-face
communication can be seen as a scene of diplomatic labor, where
participants must expend a great deal of effort ensuring that others do
not receive the ``wrong'' impression.

It was suggested earlier that conversational interaction may be viewed
as an informational game, the goal of the game being to learn as much as
possible about the other while at the same time controlling as much as
possible the information about oneself that the other obtains. The rules
of the game establish permissible times, places, and amounts of deceit
and feigning, and provide negative sanctions for players who are caught
breaking the rules. It is a game of informational management.

This view must be broadened, however. The instrumental setting in which
the actor enters into conversational interaction which others is
constantly providing, or threatening to provide, situations and events
that can easily be accepted by others as expressions of the actors'
sentiments and conceptions. In many cases, these impressions do more
than ``give the actor away.'' These impressions often make the actor
responsible for conceptions which are offensive to recipients or
unfavorable to the actor, and often these impressions are either not
justified or have to do with sentiments which the actor possesses but
which he had definitely been trying to suppress. For the actor, others
may come to be seen as sacred objects. The social attributes of
recipients must be constantly honored; where these attributes have been
dishonored, propitiation must follow. The actor must be on his guard
almost all the time and carefully poised in his action. He must conduct
himself with great ritual care, threading his way through one situation,
avoiding another, counteracting a third, lest he unintentionally and
unwittingly convey a judgment of those present that is offensive to
them. Even more than being a game of informational management,
conversational interaction is a problem of ritual management.

The ritual model for social interaction has been poorly treated in the
literature, perhaps because of the stress given by G. H. Mead and by
Weber to the fact that a social relationship, and hence social
interaction, was a product of \emph{two} persons taking \emph{each
other's} actions into consideration in pursuing their own action. This
stress seems to have given an instrumental flavor to our thinking about
the kinds of considerations we show in regard to others: the implication
is that we \emph{take into consideration} the actions of others (the
better to achieve our personal ends, whatever these may be) and not so
much that we \emph{give consideration to} other persons. By
``consideration'' we have tended to mean calculation, not
considerateness.

A case may be made for the view that the best model for an object to
which we give consideration is not a person at all, but a sacred idol,
image, or god.\footnote{This general view I base on Emile Durkheim,
  \emph{The Elementary Forms of the Religious Life}, trans. J. W. Swain
  (New York: Macmillan, 1926), pp.~240--272.} It is to such sacred
objects that we show in extreme what we show to persons. We feel that
these objects possess some sacred value, whether positive or purifying,
or negative and polluting, and we feel disposed to perform rites before
these objects. These rites we perform as frequently and compulsively as
the sacred value of the object is great. These worshipful acts
expression our adoration, or fear, or hate, and serve for the idol as
periodic assurances that we are keeping faith and deserve to be in its
favor. When in the idol's immediate presence we act with ritual care,
appreciating that pious actions may favorably dispose the idol toward us
and that impious actions may anger the idol and cause it to perform
angry actions against us. Persons, unless they are of high office, do
not have as much sacred power or \emph{mana} as do idols, and hence need
not be treated with as much ceremony. An idol is to a person as a rite
is to etiquette.


% Part One

\newpage
\thispagestyle{empty}
\begin{fullwidth}

\begin{center}
\vspace*{3in}

{\fontsize{35}{24}\selectfont{Part Four}\par}

\vspace{1in}

{\fontsize{35}{24}\selectfont\textit{The Concrete Units of Conversational Communication}\par}

\end{center}

\end{fullwidth}

% CHAPTER VIII: INTRODUCTION
\chapter[CHAPTER VIII: INTRODUCTION]{Chapter VIII: Introduction}
\label{ch:Chapter VIII: Introduction}
\chaptermark{CHAPTER VIII: INTRODUCTION}

\newthought{At this point it}
\marginnote{\href{https://doi.org/10.32376/3f8575cb.18454129}{doi}}seems proper to provide a more systematic statement of
some of the assumptions and definitions which underlie some of the terms
and usages appearing in the first three parts. At the same time it is
also necessary to provide some very elementary definitions as a
background for what is to follow.

A body of information that is transferred from one place to another is
commonly called a message. A message involves a configuration of signs
and the transmission of physical carriers, or what have come to be
called vehicles, of these signs. We usually think of vehicles as issuing
from a sign-source, and we think of vehicles as being impelled form a
source with a sign-impulse of given force or intensity, and as being
impelled in a particular direction. The process of impulsion is usually
called transmission, and arrangement of vehicles for the purpose of
transmission is usually called encoding. A physical field in which
vehicles of a given kind can be transmitted may be called a medium. We
usually think of a medium in reference to the particular type of
equipment---human and non-human---which must be employed if the signs
transmitted in the medium are to be received. The source of sign-impulse
may be called a transmitter and the agency which receives the signs a
receiver. It is to be noted that the terms so far defined pertain to the
physical aspects of communication, not to the social setting in which
communication occurs.

When an individual exercises his physical capacity as a transmitter or
receiver, we find that he may also take on communication roles of a
social kind. The individual may be held to be personally responsible for
the content of a given message and for having initially transmitted it;
the message in question is thought to be \emph{his} message. Where these
conditions prevail, we shall say that the individual has the social role
of sender. On the other hand, the individual may be held to be the one
for whom the message is specifically and admittedly intended; the
message in question is thought to terminate properly with him. Where
these conditions prevail, we shall say that the individual has the
social role of recipient.

The social roles of sender and recipient seem to be the most basic ones
in communication, but certain additional ones are clearly defined and
heavily institutionalized in our society. A person may perform the role
of drafter, this role entailing the rough formulation of messages that
are later checked over and authorized by the sender who will be made
responsible for the content of the message. Specialists of this kind may
be found in large-scale formal organizations, and are not responsible,
in certain ways, for the message which they help to formalize. A person
may perform the task of relayer, receiving a message from its sender (or
from another relayer), encoding it for retransmission and retransmitting
it to a recipient (or to another relayer). This is the case with
stenographers and telegraphers. Finally, a person may perform the task
of courier, conveying a message from one point to another without
knowing what is in the message. This is the role that postmen take.

Persons who have the task of drafting messages, or relaying them, or
carrying them, have the social duty of acting as if they were merely
instruments, not persons. They operate under a strong moral obligation
not to take advantage of the position in which their occupational duties
place them.

It is apparent that persons who are employed merely to assist in the
task of communicating may abuse their position and make inappropriate
use of the information their occupational role has put them in a
position to receive. This would seem especially likely where those who
assist in the process of communication happen to be in additional
relationships to those whom they assist. Under these circumstances,
effective segregation between the role of communication assistant and
other roles would presumably be difficult. For example, on the island,
persons who use the telephone and telegraph tend to allow for the fact
that messages may not remain a secret. There is a cautionary tale about
a previous telegrapher's agent who held up government notification of a
rise in the price of fish for one night so that a relative could buy up
the island's catch at a low price and sell it, off the island, at
enormous profit.

As sender, then, is the person who initially transmits the authorized
version of a message, and the recipient is the person for whom the
message is intended. Usually these two roles are the basic ones,
regardless of how many persons have helped to prepare, retransmit, or
carry the message.

The physical capacity of persons to communicate with one another in one
or more media is obviously related to the fixed physiological
characteristics of man as an instrument for transmitting and receiving
messages. For example, sounds with a frequency over sixteen hundred
cycles per second cannot be directly used for signalling between
persons, but these sounds can be used for signalling from a person to an
animal and from animal to animal. The capacity to communicate is also
related to variable factors which effectively increase or decrease
natural human capacity. Three of these factors may be mentioned. First,
there are behavioral devices such as whispering, shouting, ``encoding,''
and focusing of attention. For example, parents frequently exclude
children from communication by spelling messages out or using a language
not known to their children. Governments employ similar devices in order
to send messages while at the same time maintaining ``security.'' When
Berganders meet on the mainland of Britain, they sometimes make use of
the Bergand dialect in order to talk to each other in a way which can be
heard but not understood by those around them. Secondly, there are
transmission barriers such as walls, intervening persons, and noises.
For example, persons in a crowded city street can come close to each
other physically without realizing that this has been the case, whereas
persons who live or work where there are few intervening barriers to
communication can engage in certain kinds of communication over
relatively great distances. For example, most of those who fish in the
ocean waters around Dixon are acquainted with one another and can
identify each other's crafts (and are known to be able to do so) from a
great distance. Thus when two boats come within about half a mile of
each other, recognition and greeting is given by means of hand waving or
tooting, and this courtesy is an expected thing. Thirdly, there are
mechanical aids, such as telephones and mail services. For example,
shepherds on the island often make use of a whistle and a staff as a
means of increasing signalling power with respect to their sheep dogs;
without these mechanical adjuncts certain kinds of land could not be
readily utilized for grazing.

In order for communication to occur, certain minimum physical conditions
must be satisfied. The person who transmits the message (whether in the
capacity of sender or relayer) generally must be allowed to complete a
meaningful unit of communication. His message must not be ``jammed'' by
competing messages nor by disturbances which distract recipients who
wish to be attentive. The recipient, obviously, must be close enough to
the source of the sign-impulse to receive the message and must focus
enough of his attention to make effective use of this position. Further,
the signs transmitted to the recipient must be of the kind that the
recipient's equipment is prepared for or geared for; in other words, the
signs must be ``meaningful'' to the recipient. These conditions are
imposed by the extra-social characteristics of the human condition, yet
these conditions must be satisfied by habitual social arrangements.

\newpage In all communication situations the possibility arises that the
recipient will not correctly interpret or understand the message that is
conveyed to him. Whenever a relayer and/or a courier are involved as
mediated agents in the conveyance of a message, additional opportunities
for confusion arise. The message may become (advertently or
inadvertently) lost or modified in transit.\footnote{In formal
  organizations, practices are sometimes employed to minimize these
  kinds of confusion. Important words in telegrams are repeated,
  duplicate messages are sometimes sent in alternate media, and
  recipients may be required to send return evidence of having correctly
  received the message.} Also, the message may be ascribed to someone
who did not in fact send it. We may then say that communication
situations vary according to the degree to which the sender can be
assured that his message has been received by the persons for whom it
was intended and has been correctly received by them. We may also say
that situations vary according to the amount of usable proof they offer
a recipient that the message received by him is the message transmitted
by the sender. When communication between two persons is mediated by a
rigorously institutionalized relayer, such as a telegrapher, or when
there is no mediating agent, as in the case of face-to-face interaction,
we frequently find that neither sender nor recipient is in a position
with respect to the other to deny the existence and character of the
message. When communication between two persons is mediated by an
informal relayer, as in the important case where a sender tells a
recipient what an absent person has said about him, we find that the
absent person is usually in a position to deny that he made the
statement in question. We also find that the recipient is usually in a
position to act as if there were some doubt that the absent person
really sent the message in question. These objective communication
characteristics may account in part for the social fact that persons in
our society exercise much less care in their treatment of individuals
who are absent than in their treatment of individuals who are present.

The framework of this study is chiefly concerned with the kind of
communication which is unassisted by mechanical devices or by persons
acting in the mediating capacities of relayer or courier. We are
concerned with communication between persons who are immediately present
to each other, where the sender is at the same time the transmitter or
physical source of the sign-impulse.\footnote{In recent literature the
  term ``small group'' has been widely used to designate the research
  area of face-to-face interaction. There is little excuse for this
  usage. The term group has a relatively distinct meaning in sociology.
  Face-to-face interaction regularly occurs among the members of small
  groups, but frequently the kind of behavior which students describe
  when they study interaction in this context is not characteristic of
  persons in their capacity as members of small groups but rather of
  persons in their capacity as immediate interactants. The latter
  behavior occurs between persons who regularly have dealings with each
  other but who do not constitute a small group, and it even occurs
  between persons who come into each other's presence only once.} The
justification for this limitation of scope rests on the empirical fact
that persons present are treated very differently from persons absent.
It appeared that a study focused on one kind of treatment could not
easily deal with the other kind of treatment.

Interaction between persons who are immediately present to each other
possesses some crucial communication characteristics. One of these
characteristics---the ``non-deniable'' nature of messages---has already
been mentioned; others will be considered later. None the less, the
criterion of immediate presence provides a heuristic delimitation of
scope, not an analytical one. From the point of view of communication,
face-to-face interaction does not seem to present a single important
characteristic that is not found---at least within certain limits---in
mediated communication situations.

In the study of immediate communication, we deal with signs of which the
sender is the actual physical source: sender and transmitter are one. On
the whole we deal with signs transmitted by gesture and with authors
signs transmitted by speech. (The olfactory medium, as in the case of
perfumes and body odors, and the tactile medium, as in the case of
nudges, do not seem to play a major role in immediate communication.)
And for the most part, we do not deal with lines or channels of
communication which are open to reception or transmission at either end
but rather with zones, anywhere within which a message may be received.
Persons who are within the zone where reception of a given impulse is
possible are usually said to be in the range of the impulse. The shape
of zones in cases of immediate communication is an important factor and
is dependent upon a complex set of interdependent elements.\footnote{The
  body of a sender is a transmission barrier from the point of view of
  those who are behind the sender; it is a focusing reflector from the
  point of view of those who are in front of him. This gives a
  conical-shaped stress to zones of communication, with the source of
  impulse at the apex of the cone. This directional effect is especially
  apparent in the case of visual signs, since they are propagated by
  waves that are lateral, not spherical. \textbar{} Some important
  sign-vehicles involve facial expressions which can be ``read'' only if
  all parts of the surface of the face are seen in relation to one
  another. In such cases the apex of the cone is very narrow, so that
  the communication zone is effectively reduced to a narrow cylinder.
  This is also the case where the source of impulse is set into the face
  and shielded more than ordinary sources of impulse, as in the case of
  eye expressions. \textbar{} In the case of relatively intense sound
  impulses, the angle of propagation approximately describes a circular,
  and the length of the cone can be taken as the radius of a sphere
  within which reception of the sign-impulse is possible. With sounds
  that are less intense, the effectiveness of the body as a barrier
  apparently varies with the frequency of the sound as well as with its
  intensity.}

In considering the factors which influence the shape of communication
zones, we must also consider the factors which influence the
effectiveness of a recipient once a zone of communication has been
established. The volume of a visual or auditory zone can be increased
slightly by a recipient's concentration on the source of the impulse,
and the volume can be radically decreased by lack of attentiveness on
the part of the recipient. In the case of visual signs, a recipient can
easily shut off his receiving equipment or can redirect it so as to
remove himself completely from the reception opportunities of a zone in
which he finds himself. This fact, as will be seen later, underlies the
practicability of certain kinds of tact. In general, the factors which
influence the shape of a zone and the receptivity of a recipient play an
important role in habitual communication arrangements.

Communication zones possess an obvious characteristic: as the distance
between source of impulse and recipient increases, the ability of the
recipient to receive the message gradually decreases. There is an
obvious limit to this.

We frequently find that communication occurs in regions where there is a
sharp limit to the simple inverse linear relation between distance from
source and receptivity. The region may be bounded to a varying degree by
transmission barriers which abruptly reduce the intensity of any
sign-impulse that passes through them. The ordinary room perhaps
provides the most important example of this kind of communication
arrangement, where walls, ceiling, and closed door act as transmission
barriers. Regions of this kind may be called bounded regions, and they
may be said to vary in the degree to which they are bounded. In certain
cases, then, receptivity decreases gradually up to a point and then
decreases sharply.

Bounded regions, of course, vary widely according to size, and the kind
of communication behavior to be found in a particular community will
vary with the kind of bounded regions found in it.

\vspace{.2in}
\begin{centering}

\Large{* * * * *}

\end{centering}
\vspace{.17in}

\noindent The chief focus of attention of this study is conversational
interactions among persons immediately present to one another. As
previously suggested, the exclusion of mediated linguistic communication
is somewhat arbitrary and is justifiable on practical grounds rather
than on theoretical ones. It is convenient here to make another
delimitation of the scope of this study.

It appears that the information a person conveys (whether he does this
in an active or a passive way) can play two different roles and be
organized in two different ways. The terms ``directed'' and
``undirected'' will be used to refer to these two organizational forms.

Directed information is information which is directed at particular
recipients and hearts upon a particular conversational issue, a
particular object of reference, that is current at the time. This
information may be conveyed by linguistic behavior. It may also be
conveyed by expressive behavior of the kind we use in qualifying our
linguistic statements or in responding in a truncated form to linguistic
statements---in other words, the kind of expressive behavior which can
be cut sufficiently short so as to add information about and only about
the conversational topic of the moment. Obviously, we cannot receive
directed information unless we are at the time engaged in actual
conversations or in overhearing actual conversation.

Undirected information may be defined as information which is conveyed
between persons who are within perceptual range of one another but who
are not necessarily involved in actual conversation with one another.
When, for example, an individual is engaged in conversational
interaction with one cluster of persons, he does not thereby cease to
provide a source of impression to those not in the cluster. By the
loudness of his voice, or the extravagance of his gestures, he conveys
how willing he is to allow persons in clusters other than his own
continue their own conversation without undue distraction. By his
demeanor and his choice of clothes he conveys the degree of respect he
feels for the persons in the region and the social occasion that has
brought them together. By means of the same behaviors, the individual
both intentionally and unwittingly communicates something about his
statuses in the wider social worlds which lie beyond the present region
and occasion. The individual, further, conveys his status relative to
those who are present in the region by participating in one cluster as
opposed to another or by not participating in any cluster.\footnote{At
  parties hosts tend to feel responsible for seeing that guests
  ``enjoy'' themselves and that they ``fit'' socially with one another.
  Each type of party has its own rules as to how long a guest can remain
  unattached to a conversational cluster (or how frequently he may
  become detached from one) before this lack of involvement in directed
  communication becomes a sign (transmitted by undirected communication)
  of the guest's improper relation to the party. At formal dinners there
  is a rule regarding fair distribution of conversational attention to
  partner on the right and partner on the left; at dances there are
  ``duty'' dances; at informal parties the host frequently has the
  obligation of engaging in engaged persons in conversation so as to
  stop undirected communication of the isolate's status.}

Undirected information, it appears, can only be communicated by means of
the expressive component of behavior. Related to this is the fact that
undirected information cannot easily be formulated, precisely and
consciously, into a specific message; participants have strong feelings
about this kind of message (perhaps because it conveys overall
conceptions that the sender has about himself and others), but only
vague ideas as to what exactly is being communicated.\footnote{Perhaps
  it should be noted that the medium which relies on a sense of smell
  seems employable for only undirected messages. There are a few
  exceptions, as, for example, the use of mercaptan in the air tunnels
  of mines as a sign of a linguistic order that danger is present. The
  medium which relies on a sense of touch, however, can be employed to
  convey directed messages---as when one recipient nudges another
  recipient as a means of commenting upon a particular directed message
  conveyed by a third person, the sender.}

It is to be noted that in the kind of directed message where a sender
makes a verbal statement to a recipient, the sender can ``catch
himself'' half way through his message and try to modify it in
accordance with what he perceives to be the response it is eliciting. If
this is not possible, and he finds that his message has elicited a
response that he did not wish to elicit, he can hastily add another
verbal statement that is calculated to repair the damage. Corrective
feedback is possible. Typically, however, this repair work is less
possible in the case of inappropriate undirected messages. If a person
appears at a social occasion in inappropriate attire, or intoxicated, or
in the company of an undesirable person, he cannot hastily correct the
unfavorable impression he may make. Often he cannot even ``shut up,'' as
he can after conveying an inappropriate directed message, but must go on
transmitting the unfavorable message until he leaves the place where
interaction is going on.

Undirected messages, unlike directed ones, are not supposed to be
conveyed with any particular recipients in mind. Of course, a sender may
employ a particular undirected message for the special purpose of
influencing a specific recipient, as when a woman ``dresses up'' for the
effect it will have on the man who is courting here. Even in such cases,
however, the message remains the kind that cannot be strictly
formulated, and it retains a disguise as a message that is relevant for
all persons who happen to come within range of it.

Directed communication frequently takes the form of a rapid exchange,
between two talkers, of statements and replies. Rapid and continued
give-and-take is, in a sense, the conversational thing about
conversation. Undirected communication, on the other hand, is not so
clearly a part of \emph{inter}action and \emph{inter}change. Frequently
a person's undirected messages are merely absorbed into the accumulated
impression we have of him, and then by themselves they frequently do not
elicit an overt response from us or, at least, an immediate overt
response. It is to be noted, however, that certain undirected messages
give rise to an attenuated and sluggish forms of ``conversational''
exchange.\footnote{Interaction systems of this attenuated kind are not
  considered in this study.} If an actor annoys his neighbor by making
too much noice or by burning ill-smelling garbage, the neighbor may,
when the time is ripe, negatively sanction the actor, conveying the
sanction by means of directed or undirected signs; in return the actor
may answer with a reprisal or a reparation. However, unlike
conversational interaction, exchanges of reprisals and counter-reprisals
may be protracted affairs and usually involve only a few exchanges of
messages. Parallels between conversational systems and undirected ones
will be suggested throughout this report but will not be developed.

Undirected communication plays an important role in our social life, and
yet it has been very little studied. Three important sources of
undirected signs may be mentioned. First, there are clothing
patterns.\footnote{A recent illustration of the role of clothing as
  undirected communication is given by Miller and Form, \emph{Industrial
  Sociology} (New York: Harper, 1951), p.~356, in their description of
  status symbols in a garage: ``This factor of clothing may be carried
  to a ludicrous degree. In a small garage that the authors studied a
  wide gamut of clothing symbolized gradations of status. The owner
  worked in this `business' suit. The stock and order clerk wore no
  special uniform but had to remove his coat and worked in his shirt
  sleeves. The supervisor of the mechanics in the shop also removed his
  coat, but he wore a very non-functional piece of clothing, a white
  smock. The mechanics wore full-length blue jumpers, and the
  apprentices and cleanup men wore overalls or discarded clothing, of
  darker hues. Although this hierarchy of garb was not formally
  instituted, it was nonetheless scrupulously observed. No one could
  presume to rise above his status by wearing the costume
  ''inappropriate'' to his job.''} This means of conveying one's
conception of oneself and one's opinion of the social surroundings has
the interesting characteristic of continuous transmission. Exception
when the sender is taking a bath, in our society, his body is covered
(or significantly uncovered) with materials which convey a message to
anyone who comes within visual range. Secondary, there are participation
patterns. The persons in whose presence an actor is seen, or the persons
in whose conversations he could be participating but is not, provide
sources of information about the actor. These sources, too, tend to be
in continuous transmission, for there are many social situations in
which the actor conveys something about himself simply by appearing in
the company of no one. Thirdly, there are what might be called
``location'' patterns. the furnishing and decor of a person's room,
office, or place of work; the size, style, and upkeep of his house; the
appearance of the land immediately surrounding his house---all these are
important sources of undirected signs which tell us significant things
about him. Like the first-mentioned source of signs, these are in
continuous transmission. Unlike clothing and participation patterns,
location patterns do not follow the sender wherever he goes. A potential
recipient must come to the place where the sender is habitually located
and upon which he has left his mark, and frequently the recipient must
gain permission from the sender to do this, before the recipient can
avail himself of the information that can be found in such
locations.\footnote{In Britain a very important source of undirected
  information is found in speech patterns. Range of vocabulary, volume
  and pitch of sound, dialect, intonation, accent---all of these signs
  help to place a person socially and regionally, even though the
  recipient overhears only a snatch of the linguistic message that is
  being communicated. Like clothing patterns, this form of undirected
  communication follows the sender wherever he goes; unlike clothing
  patterns, speech patterns are not in continuous transmission---a
  speaker can shut up.}

In Dixon, there are many sources of undirected information that seem to
be typical in Western society. For example, almost all adult male
crofters have four levels of clothes-finery: rough work clothes;
informal indoor clothes; clothes for socials and small parties; clothes
for the most important occasions, such as weddings and funerals. Each of
these levels is deemed appropriate for a certain range of social
occasions (although much the same set of persons may meet each other at
occasions in all of the ranges); inappropriate dress is considered an
affront to those who perceive it and to the social occasion in which it
occurs. So, also, a man in Dixon who does not have a ``clean'' shave
conveys thereby a slight disrespect for the persons and institutional
with which he has dealings.

Because Dixon is a rural community that is small in geographical size
and uncluttered with communication barriers such as trees, some
communication problems arise in connection with undirected signs. The
state of one's crops and the size of one's stock are physically present
for everyone to see. This possibility is increased by the custom of many
crofters of carrying around or having in the house a small pocket
telescope. Thus an important aspect of one's wealth cannot be concealed
from others. This makes it difficult for crofters to practice strategies
that members of other occupational groups often employ, namely,
overestimating one's wealth in some circumstances, concealing it in
others, and underestimating in still others. Nor can one crofter conceal
from another the state he has reached in the annual work cycle or the
tactics he is employing in performing the basic croft tasks. Thus, for
example, errors of judgment or lack of work skill cannot be concealed.
The two fishing crews find themselves in the same position. More than
half the members of the community are in a position to observer directly
the times when the boats put out and the times when the boats do not.
Crews that go out in bad weather have no protection from residents who
judge such actions to be unwise. Crews that do not put out in bad
weather have no way of concealing this face from residents and must face
the scorn of ex-sailors. So, also, the exact catch of one boat can be
seen and compared with the catch obtained by another boat, or the catch
that should have been made. Large catches bring claims from creditors
and friends; small catches bring judgments of low ability. The crofter
and the fisherman thus have little informational control over their
work.\footnote{The importance of this kind of control for the protection
  of the worker is brought out nicely in Donald Roy, \emph{op. cit}.}

The fact that a crofter must do much of his work in the open, before the
eyes of the community, as it were, tends to throw into clear relief any
social change that occurs in regard to crofting customs. As soon as one
crofter makes an innovation, others find that they become identified as
persons who do or do not use the new technique. Thus the last fifteen
years has seen a radical shift in plowing techniques, a shift from the
use of horses to the use of tractors. Apparently those in the vanguard
of the change fifteen years ago were as clearly marked out in the
community for this fact as are those who today still use horses. Since
the crofter, during the last fifty years, has been moving from the
status of a peasant-tenant to the status of an independent progressive
farmer, the visible ownership and use of costly modern capital goods is
a means of saying something and having something said as well as a means
of doing something.

Another illustration of the interplay between social change and
conditions of undirected communication in Dixon may be seen in the
changing conceptions as regards the work day. Traditionally the number
of hours worked in a day was determined by the urgency of the task in
the current phase of the croft work cycle, by the availability of light
to work by, and by the fitness of the weather. During the darker winter
days, when the sun rises at about ten and sets at about three-thirty,
crofters sometimes stay in bed, where it is warm, until eleven or twelve
in the morning, and, except for regular chores such as feeding the stock
and doing repairs, not much is done by the menfolk. During the other
seasons, basic croft tasks, such as lambing, casting of peats, plowing,
and sowing, which have to be accomplished within the right calendar
period, may keep a crofter and his family working as many hours as they
are physically capable of. It was not rare for crofters to rise at three
o'clock during the lambing season, and at four o'clock during the
peat-casting season. It was not rare for hay to be raked and stacked by
moonlight. However, apparently in connection with increasing government
employment and increasing government regulation of working hours, an
eight-hour-day conception of work is becoming more prevalent. This day
stops on Sunday, holidays, and Wednesday afternoon, but it does not vary
according to the season or the clemency of the weather. Fewer and fewer
crofters are now working after their six o'clock supper, although,
during June and July, there is frequently enough light to work all night
long. There is a feeling that it is improper for persons to work in the
evening. Similarly, there seems to be a tendency to be more and more
selfconscious about staying abed all through the morning in wintertime.
Winter mornings are coming more and more to be defined as times when it
is improper to not be up and around. In other words, the hours between
eight in the morning and six at night on weekdays are coming not only to
be more and more common as the period when one is working, but this time
period is coming more and more to be defined as the time when, and only
when, men ought to be engaged in work. Failure to be seen working during
this time, or perceived attempts to work during other times, are coming
more and more to be felt as something which gives the community a bad
name.

\enlargethispage{\baselineskip}

Interestingly enough, Wednesday afternoon off, which those who work in
the shops or for the government enjoy, is apparently still felt to be a
slightly improper luxury; the young clerks who choose to spend that
afternoon in visible recreational pursuits seem to do so with feelings
of selfconsciousness and even feelings of guilt. So, too, Wednesday
night, which has traditionally been a time for socials and
festivities---a sort of duplication of what also occurs on Saturday
night---does not yet seem to have succumbed to standard Anglo-American
definitions of Wednesday night.

It may also be noted that specialized communication services are also
influenced by the perception range characteristic of undirected
communication in Dixon. If a telegram is delivered to anyone, many
persons in the neighborhood can see that this has taken place, and soon
the whole community knowns that an important event has occurred in the
family to which the message was delivered. Similarly, the shape of
packages delivered by the postman can be seen by many persons, although
the package itself is not violated. If a package shaped like a bottle of
whiskey is delivered to someone, many others soon know that this has
occurred.\footnote{For a handful of persons in the community, drinking
  is thought by others to be a ``problem.'' The chief postman cooperates
  with the disapproved drinkers by delivering their whiskey packages by
  car personally, thus eliminating undirected signs conveyed by
  packages. Of course, this double delivery service is known of, and the
  appearance of the postman's car outside certain cottages is itself
  taken as a sign of a whiskey delivery.} All of these sources of
undirected communication contribute to the feeling that many crofters
express that Dixon is a fine place but everyone knows too much about
everyone else.

% CHAPTER IX: SOCIAL OCCASION
\chapter[CHAPTER IX: SOCIAL OCCASION]{Chapter IX: Social Occasion}
\label{ch:Chapter IX: Social Occasion}
\chaptermark{CHAPTER IX: SOCIAL OCCASION}

\newthought{In Dixon, as, apparently,}
\marginnote{\href{https://doi.org/10.32376/3f8575cb.387f8616}{doi}}elsewhere in English-speaking society, the
term ``social occasion'' is often given to events such as a
whist-social, a picnic, a public political meeting, etc. When we examine
events of this kind, we can isolate a set of common characteristics:

\enlargethispage{\baselineskip}

\begin{enumerate}
\item
  Regulations usually exist as to who may and may not participate, and
  all those participating do so in capacities defined as relevant.
\item
  The event is felt to have a beginning and an end (even though in some
  cases it may not be possible to define precisely the moment of
  beginning or ending) and is felt to be in continuous existence between
  these points, even though lulls and intermissions may occur. Further,
  between the beginning and the end of an occasion there is what might
  be called an involvement contour, a line tracing the gradual initial
  involvement of the participants in the occasion, the peaks and low
  points of the involvement of the participants during the occasion, and
  the path by which the participants come finally to reemerge from their
  psychological commitment to the activity of the occasion and leave the
  interaction.
\item
  Participants recognize that the event involves a ``main'' or ``chief''
  activity and that this activity takes place in a very small number of
  bounded regions which are usually connected with one another. Main
  regions are recognized. In addition, recognition is given to a number
  of other regions, usually smaller than the main ones, where activity
  functionally related to the main activity but different from and
  subordinate to it takes place. Thus, at a whist-social in Dixon, whist
  is defined as the main activity and the large room in the community
  hall is defined as the main bounded region; the kitchen, the cloak
  rooms, and the entrance hallway are recognized as places where related
  but secondary activity occurs. These regions, whether main or
  subordinate, are of course the scene of other kinds of social
  occasions at other times.
\item
  One or more participants are usually defined as responsible for
  getting the occasion under way, guiding the main activity, and
  terminating the event.
\end{enumerate}

Events which may be classified as social occasions themselves vary in
certain ways. Some of these dimensions of variations will be suggested
here.

1. Social occasions vary according to the degree to which participants
recognize that the goal or object of the occasion is realized within the
occasion itself.\footnote{Simmel, of course, makes this point, \emph{op.
  cit}., p.~45, where in comparing sociability to play he says:
  ``Inasmuch as in the purity of its manifestations, sociability has no
  objective purpose, no content, no extrinsic results, it entirely
  depends on the personalities among whom it occurs. Its aim is nothing
  but the success of the sociable moment and, at most, a memory of it.
  Hence the conditions and results of the process of sociability are
  exclusively the persons who find themselves at a social gathering. Its
  character is determined by such personal qualities as amiability,
  refinement, cordiality, and many other sources of attraction.'' In his
  lectures Professor Shils has made the same point in reference to
  primary groups.} Thus, in Dixon, a political rally may be attended in
order to obtain the opinion of the speaker; attendance in such cases is
an admitted means to an end, and the end is something that falls outside
the meeting itself. A party, on the other hand, is not attended as a
means to some end lying outside the party itself; to say that
participants go for recreation seems only an attempt to put into an
instrumental mode of thinking what really does not belong there.
Occasions which are, in a sense, their own and are variously described
in the literature as convivial, informal, recreational, or social in
nature; the other kind of occasion is sometimes called ``serious'' or
``formal.'' Obviously a recreational occasion may have small periods
within it devoted to serious activity, and serious occasions may have
small parts devoted to recreation. Also, we find that persons attend
supposedly serious occasions just for the convivial pleasure of being
with people and that persons attend supposedly convivial occasions for
what we call ``ulterior'' motives; in both cases, however, the person
who attends for improper reasons gives lip service to the socially
defined nature of the occasion and acts as if he were attending for
proper reasons. For example, during billiards at the Dixon hall, it
seemed that at least one steady player, the manager of Allen's Dixon
shop, played because he thought it was a good thing for himself and for
the business to be represented at the occasion. He admitted privately to
me that he really didn't care what kind of a score he was able to build
up during a shot and was only concerned to keep the teams as evenly
balanced as possible so as to ensure the interest of the players; if he
found himself getting more points than his side needed to keep a little
ahead, he would ``let up'' and not really try. What ought to have been
an end in itself was for him a means to an end.

Of course, a social occasion that is properly defined as recreational
for one person may be defined as serious for another. For example, the
job of the caretaker of the community hall during billiard nights was to
close the hall at night and see that the lights were kept in working
order. The caretaker was supposed to spend the evening among the players
but as a worker, not as a player. Interestingly enough, on many
occasions he found himself unable to treat the occasion as a means to
his livelihood; he continually got caught up in the occasion and found
himself wanting to play even though he ought to have been present not as
a player but as a worker. In joining the play, the caretaker found it
necessary to give constant assurance that he was merely filling in until
others came or that he really didn't want to play at all. This effort on
the part of the caretaker to stay within his role, and his ability to do
so, became a standing joke with the steady players.

A final qualification must be made concerning the recreational-serious
polarity. It sometimes seems that some participants obtain enjoyment and
spontaneous involvement in an occasion to the degree to which the
occasion provides a lowering of social barriers between themselves and
persons of relatively high status. A ``successful'' party in Dixon, as
in many other places, is often one in which a person who has previously
been distant and superior to those present ceases, at least for the
duration of the occasion, to maintain his usual social distance. In this
sense, the occasion is a means to an external end. But in these cases,
participants who are given this means are not supposed to define it as
such or recognize it as such. A social occasion, it seems, can actually
function as a means to an external end for a participant, and yet he may
sincerely feel that all he gets from the occasion is recreation and
enjoyment. The instrumentality of a recreational occasion may be
unconscious, and hence the person for whom the occasion is instrumental
in this sense need not feign the absence of an ulterior motive.

2. Occasions vary in the degree to which they are organized by means of
preestablished explicit directives, giving us on one hand occasions
which tend to be what are often called ``informally organized,'' and on
the other hand occasions which tend to be ``formally organized.''
Formality-informality, as regards organization, is found in various
factors. Three examples may be suggested.

First, a plan of operation may be explicitly specified beforehand,
setting out a detailed agenda for the occasion, or, on the other hand,
the plan of proceedings may tend to be implicit, with the participants
deciding at any one stage in the undertaking what they will do in the
next stage. In Dixon, for example, the semi-annual concert is fully
programmed, performers knowing beforehand the sequence in which they
will appear; family picnics, in contrast, tend to be informally
organized and decisions, in contrast, tend to be informally organized
and decisions as to what to do at any particular time tend not to be
arrived at until it is time to act upon the decision.

Secondarily, some of the participants may be explicitly designated as
officers who have the right and obligation to direct proceedings, or, on
the other hand, leadership may either be inessential or develop
spontaneously as a consequence of interaction during the occasion. For
example, during a sheep ``cawing,'' when shareholders in grazing rights
to a particular stretch of hill work cooperatively to bring the sheep
together for dipping or shearing, one man is designated to give commands
to the herders so that the sheep cannot find a weak point in the closing
ring of herders and break for the hills. His word is the authorized
signal for beginning or ending each phase in the operation. On the other
hand, during billiards no one has the official right to say when the
players ought to quit and go home; the decision comes in what looks to
be a spontaneous way, although in fact it must usually be informally or
implicitly authorized by the ``informal'' leader.

Thirdly, rights and obligations may tend to be explicitly specified in
detail beforehand, with rewards and punishments specified in detail as a
means of guiding behavior, or rights, obligations, and sanctions may be
taken for granted and not determined explicitly until the moment arrives
for exerting them. For example, at billiards, which tends to be
informally organized, there is none the less a specific explicit rule
that each player place two pence in an ``expenses'' box for each game
played; at most parties in the community, no explicit duties are placed
upon guests.

3. Social occasions appear to vary in the degree to which they are
conducted in what has come to be thought of as a formal or an informal
way. In occasions which are formally conducted, participants are obliged
to restrict their activity to roles that are explicitly or implicitly
defined as the main and proper ones for the occasion. In occasions which
are informally conducted, participants are allowed to interact in
capacities other than those defined as relevant for the occasion. Thus,
in Dixon, at birthday parties, participants are fairly strongly obliged
to stay within the ethos of a party and not separate themselves off,
individually or in small clusters, for activity in whose spirit all
participants cannot share. On the other hand, when a few friends ``drop
in,'' without special reason, the occasion tends to be informal,
participants moving in and out of their role as party guests, as
interest at the moment dictates.

Observations in Dixon suggest that the degree to which a given occasion
is serious or convivial cannot tell us the degree to which it will be
formally or informally organized, and that neither of these factors can
tell us whether it will tend to be formally or informally conducted.
Hence it seems useful to distinguish among the three variables, although
all pertain in some way to the commonsense notion of
formality-informality, a notion that has been used with little further
refinement in much sociological literature.

4. Social occasions may vary according to the number of different lines
of action which are defined as the main activity of the occasion. For
example, in Dixon during the natural ``Gala Day,'' several competitive
sports events (such as the running broad jump, the hundred yard dash)
and several farm competitions (such as produce judging and sheep dog
trials) may be defined as main activities and be in progress at adjacent
places at the same time. On the other hand, the evening ceremony, during
which the prizes are awarded, is part of the Gala Day's stage
performance which allows for only one main activity at a time.

5. Social occasions vary according to the degree to which persons look
forward to them as coming concrete entities and/or look back at them,
after they are past, as things to be separated out from the flow of
events in which they are embedded and seen as independent units.
Regardless of what occurs at an occasion, persons tend to think of some
as distinct entities and of other occasions as not. For example, an
employee may know that he will be at work all day in a given place two
weeks from a given moment, but he will not single out this attendance at
work, or, rather, the occasion which he thereby attends, and think of it
as a distinct and special thing; it will be just another work day. On
the other hand, the day at work which is given over to the Christmas
party may for him constitute a special occasion, to which he looks
forward and to which he looks back. A party which was begun on the spur
of the moment may be an occasion to which no one looked forward but to
which all participants look back.

6. Social occasions seem to vary according to the degree to which they
constitute ``regular'' occasions and form part of a series of occasions.
A regular occasion is often thought of as one which occurs at the same
place, at the same point in a daily, weekly, or annual time cycle, and
with the same participants, as the other occasions in the series. For
example, in Dixon the social occasion provided by the accidental burning
down of a shop does not recur in any periodic sense; the twice-monthly
showing of the rural film unit does form part of a series of recurrent
showings. Recurrent or regular occasions the smiles seem to differ in
subtle ways. Some series of occasions are recognized as a series; the
series is looked forward to and back upon as a series, and behavior at
one regular occasion may have some explicit or implicit carryover and
consequence for a later similar occasion. We sometimes use the term
``sessions'' to refer to a series of this kind. In Dixon, there is an
annual sailing boat competition that awards a cup to the boat that makes
the best total score in a series of about eight races. Each race is held
on Saturday night during eight successive weeks. The eight races and the
eight Saturday nights are felt in certain ways to be a single unit. On
the other hand, daily dinner in a Dixon household involves the same
participants in the same activity at the same place, but little social
recognition seems to be given to the series as a series.

\newpage In the research reported in this study, social occasions and series of
occasions were not, as such, the focus of attention. The concept of
social occasion has been considered because it is helpful to give some
attention to what one is not, specifically, studying in order to speak
more clearly about what one is studying. Furthermore, it will now be
possible to talk about the context or setting in which social
interaction occurs in terms that are not completely undefined. It should
be noted, however, that no attempt has been made to consider other kinds
of contexts which provide a setting for interaction, such as diffuse
definitions of the situation that prevail in a given place and time and
that lead us to feel that certain interaction is appropriate on Saturday
night downtown that is not appropriate Tuesday afternoon in the factory,
and that permissible behavior on New Year's Eve may everywhere be a
little different from what is considered permissible at other times.

% CHAPTER X: ACCREDITED PARTICIPATION AND INTERPLAY
\chapter[CHAPTER X: ACCREDITED PARTICIPATION AND INTERPLAY]{Chapter X: Accredited Participation and Interplay}
\label{ch:Chapter X: Accredited Participation and Interplay}
\chaptermark{CHAPTER X: ACCREDITED PARTICIPATION AND INTERPLAY}

\newthought{When two or more}
\marginnote{\href{https://doi.org/10.32376/3f8575cb.6c0c9ce2}{doi}}persons are engaged in linguistic communication with
one another, in Dixon and apparently elsewhere in Western society, there
is a tendency for each participant to extend to himself and to all other
participants the like privilege of ``accredited'' attendance. Briefly,
each person not only participates in the interaction but does so, and is
allowed to do so, with legitimacy; his manner conveys that he is openly
and admittedly involved in the conversation and that his presence in the
conversation is a proper and justifiable thing.

Accredited or legitimated attendance may be thought of as a kind of
status. It is perhaps one of the broadest of statuses; persons of
extremely discrepant social position can find themselves in a situation
where it is fitting to impute it to one another. The status carries the
right and the obligation to receive the message at hand, and the status
implies the judgment that the participant is worthy and capable of
receiving the message. It should be noted that incumbents of the status
are obliged to be engaged at that very moment in exercising their status
and that the status does not carry over from one period when it is being
exercised to another period, as, for example, in the case of
occupational statuses. There is no interspersing of times during which
the status is exercised with times during which it is latent.

Additional communication statuses may be imposed on participants in an
unequal and differential way. For example, only certain participants
among those present may be allowed to send linguistic messages as well
as receive them. These limitations on one's rights as sender
nevertheless do not alter the fact of an underlying like status of
accredited participation which all participants equally enjoy.

While a person may have only the right to receive a linguistic message,
and not the right to send one, and still be an accredited participant,
it must be made quite clear that mere reception of the message in
question does not necessarily imply recognition as a legitimate
participant. A person may overhear a conversation without the conversers
knowing that this is the case.\footnote{This kind of communication
  arrangement was considered in chap.~vi.} Further, a person who is
known by the participants to be in a position to audit their
conversation may be given the status of a ``non-person'' and treated
from the point of view of the conversation as if he were not present and
therefore as if certain kinds of consideration need not be given to
him.''\footnote{This kind of communication arrangement is considered in
  chap.~xvi.}

Reference has been made, in the chapter on indelicate communication
arrangements, to some obvious kinds of reception and participation that
are socially unrecognized or unaccredited. The difference between
accredited participants and unaccredited participants can be much more
subtle than was suggested there. A person can overhear a conversation
and know that the accredited participants know he is overhearing the
conversation---and yet not be a legitimate participant. This may occur
whether or not the conversers make an effort to feign that they are not
aware that they are being overheard. Further, the conversers may convey
by their manner to the eavesdropper that they realize that he is
overhearing them, while at the same time the eavesdropper may convey
back to them that eh knows they know he is overhearing them---and still
the intruder need not be taken into the conversational circle as an
accredited participant. In all of these marginal types of communication,
we may have an exchange of action and reaction between accredited
participants and the intruder, and yet this by-play is not part of
conversational interaction in the strict sense of the term.\footnote{Students
  of social interaction have sometimes confused the issue by attempting
  to study a limited type of interactive system, namely conversation, by
  means of very abstract criteria, e.g., the action of two persons when
  each knows he is under observation by the other. Abstract criteria
  such as this are equally satisfied by a whole range of interesting but
  minor communication arrangements. The crucial criterion of
  \emph{accredited} participation seems to have been consistently
  overlooked. The presence of this factor would seem to serve as a means
  of isolating a natural area for sociological study.}

It seems that among the accredited participants of a given spate of
linguistic communication, one participant is usually given the role of
accredited sender and the remaining participants are accorded the role
of accredited recipients. The thoughts of all the participants are
usually brought to bear on a particular subject-matter of reference,
while at the same time the recipients focus their visual attention on
the sender for the duration of his message.\footnote{Cf. R. F. Bales and
  others, ``Channels of Communication in Small Group Interaction,''
  \emph{Amer. Sociol. Rev.}, XVI (461--468), 461. ``The conversation
  generally proceeded so that one person talked at a time, and all
  members in the particular group were attending the same conversation.
  In this sense, these groups might be said to have a `single focus,'
  that is, they did not involve a number of conversations proceeding at
  the same time, as one finds at a cocktail party or in a hotel lobby.
  The single focus is probably a limiting condition of fundamental
  importance in the generalizations reported here.''} Accredited
recipients have the obligation of granting their attention to the sender
and the right to expect him to convey a meaningful, acceptable message;
the accredited sender has the right of receiving the concerted attention
of the other participants and the obligation to fulfill their
expectation that a meaningful, acceptable message will be forthcoming.

When a number of persons recognize one another as accredited
participants, turning their minds to the same subject-matter and their
eyes to the same speaker, a shared definition of the situation
apparently comes to prevail. A shared understanding arises as to what
judgments are to be openly stated concerning the topic under
consideration, and a working acceptance or surface consensus is achieved
concerning the complex social valuation that is to be provisionally
accorded each participant. A mental set is established and particular
attitudes are encouraged. A culture, a climate of opinion, a group
atmosphere tend to arise.\textsuperscript{5}

It is possible,\marginnote{\textsuperscript{5}\setcounter{footnote}{5} A statement of this is provided by
  Gregory Bateson in his discussion of ethos in \emph{Naven} (Cambridge:
  Cambridge University Press, 1936), pp.~119--120. ``When a group of
  young intellectual English men or women are talking and joking
  together wittily and with a touch of light cynicism, there is
  established among them for the time being a definite tone of
  appropriate behavior. Such specific tones of behavior are in all cases
  indicative of an ethos. They are expressions of a standardised system
  of emotional attitudes. In this case the men have temporarily adopted
  a definite set of sentiments toward the rest of the world, a definite
  attitude toward reality, and they will joke about subjects which at
  another time they would treat with seriousness. If one of the men
  suddenly intrudes a sincere or realist remark it will be received with
  no enthusiasm---perhaps with a moment's silence and a slight feeling
  that the sincere person has committed a solecism. On another occasion
  the same group of persons may adopt a different ethos; they may talk
  realistically and sincerely. Then if the blunderer makes a flippant
  joke it will fall flat and feel like a solecism.''} presumably, for the thoughts and visual attention of
recipients to come together into a focus in order to receive a single
message from a speaker, and then for this common orientation to break
down completely once the message has been received. Apparently, however,
when a number of individuals join one another in a state of mutually
accredited participation, there is a tendency for the
social-psychological alignment of the participants to remain intact even
though a sender's message has been terminated and even though there may
have been a shift in the spatial position of the participants and
fluctuation (within limits) in the number of accredited participants. As
one participant ceases to play the role of sender and falls back into
being merely a recipient, another participant takes on the role of
sender. The definition of the situation that provided the context for
one message is maintained and provides a context for the next message.
The focus of visual attention in a sense is also maintained, for while
it passes from one speaker to another, it tends to pass to a single
speaker. We may refer to the total communication which occurs on the
part of accredited participants during the time that they are aligned
together in one definition of the situation and one focus of visual
attention as an interplay.\footnote{It would be less troublesome to use
  the term ``a conversation'' instead of the term ``an interplay.''
  However, certain interplays, as for example political speeches, can
  hardly be called ``conversations.''} The persons who maintain a
particular interplay are not thereby a group; they have merely extended
to one another a certain kind of temporary communication status.

An interplay may last for a moment, as in the case of strangers who are
forced to pass each other on a narrow walk and who glance at each other
in order to make sure that difficulties or misunderstandings will not
arise. An interplay may last hours, as in the case of organized debates.
An interplay may include only two participants (no doubt the most common
arrangement); it may include many participants, e.g., a mass meeting. A
particular social occasion is usually the scene for more than one
interplay at any given moment, but this is not necessarily the case;
some social occasions encompass or incorporate only one interplay.
Finally, it is often convenient to characterize an interplay by the
character of the social occasion in which the interplay occurs.

The statement has been made that participants in an interplay focus
their thoughts on the same subject-matter and direct their visual
attention to a single speaker, although this attention may pass from one
speaker to another. Some qualifications of this statement must be
suggested.

\enlargethispage{\baselineskip}

1. The focus of attention in an interplay may momentarily pass to
objects which can serve in this capacity but in no other relevant one.
During informal conversation, for example, the focus of attention may
momentarily pass to infants, or animal pets, or even to cultural
artificers.

2. A group of persons may play together the role of a single affective
sender. Choral singing at a church social provides an example.

3. A participant may attempt, sometimes successfully, to take over the
focus of attention before the currently recognized sender is ready to
relinquish his role. In addition to the phenomenon of interruption, we
also find the phenomenon of ``heckling,'' that is, the practice of
capturing the focus of attention for a brief moment in an unrecognized
way, so that the recognized sender does not officially terminate his
message and is obliged to act as if the focus of attention has not
really left him. And we find, especially during large formally-organized
occasions, that a knot or cluster of participants may furtively engage
in an informal interplay of their own while ostensibly involved in the
formally organized one.\footnote{This communication arrangement is
  considered in chap.~xvii.} In all of these disruptive acts, however,
the disruptive sign-impulse is modulated so as to allow in some way for
the dominance and effective transmission of the accredited message. In
Dixon, for example, the only observed exception to the rule that
unaccredited messages ought to be modulated in favor of the accredited
message occurred in the case of a sixty-five year old man, an orator of
wide repute in the community. In his cottage, within his family circle
(and only there), he would interrupt a conversation with a request for
the focus of attention and then launch into a long statement, whether or
not his request for attention was granted. His family developed a rare
tolerance for hearing the full sound of two conversations while being
engaged in only one.

The meaning and significance of interruption will, of course, vary. In
formally organized interplays explicit and specific sanctions may exist
for curbing interruption. In court trials, for example, we have contempt
of court actions. Simmel has referred to the practice in some medieval
guilds of imposing a fine upon those who interrupted an alderman in his
speech.\footnote{Simmel, \emph{op. cit.}, ftn. p.~349. It has been said
  that in Nazi Germany persons in a cafeteria or other semi-public place
  would be fined if they did not stop their conversation when the voice
  of Hitler came over the radio loudspeaker. This is a case of legal
  sanctions being imposed on the interruption of mass-impression
  messages and is no doubt rare.} Miller, in considering what happens
when persons come to be on increasingly informal terms, suggests that a
record of their speech would show changes in rules regarding
interruption:

\begin{quote}
Such a record of the timing of their conversation will show that at
first they are quite polite. Neither interrupts, both wait for the other
to finish. As they get to know each other, the rate of interaction
increases and interruptions become more frequent. The proportion of the
time that each person spends talking usually settles down after several
interviews to a relatively constant value.\footnote{George A. Miller,
  \emph{Language and Communication} (New York: McGraw-Hill, 1951),
  p.~254.}
\end{quote}

\noindent None the less, if interruption becomes too frequent and both sender and
receiver talk at the same time, the interplay usually becomes
disorganized.

\newpage 4. It has been suggested that interruptions may occur but that some
limitation will exist concerning them. As a fourth qualification to the
original definition of interplay, another basic possibility must be
mentioned. When one sender terminates his message, it may happen that no
other participant immediately volunteers to take on the role of sender
and contribute a messages. A lull may occur and yet the interplay may
not, sociologically speaking, have ended. In general, brief lulls are
permissible between messages, and somewhat less brief lulls are
permissible between interchanges. A lull of some kind, for example, is
often required in order to give recipients a chance to consider the
message they have received and prepare a response to it. But if a lull
occurs that is too long, relative to the norms of the interplay,
interactional disorder and feelings of shame and uneasiness may
result.\footnote{American broadcasting has contributed the term ``dead
  air'' to refer to situations where listeners have, in a sense, given a
  station their accredited attention and then found that sound suddenly
  ceases. Apparently stations operate on very slender norms of
  toleration for dead air. A consideration of the silences during
  conversation, from the psychological point of view, is given in J. A.
  M. Meerloo, \emph{Conversation and Communication} (New York:
  International Universities Press, 1952d), pp.~114--119. The role of
  silence in the psychoanalytical interview is illuminated by Edmund
  Berger in his article, ``On the Resistance Situation: the Patient Is
  Silent,'' \emph{Psychoanalytic Review}, XXV, 170--186.}

\enlargethispage{\baselineskip}

5. During the time that persons are accredited coparticipants, the
attention of one or more participants may wander momentarily from the
sender. Some of the ways in which this can occur are considered later.
Here it must be noted that different kinds of interplay have different
standards of tolerance regarding this threat of interaction. Interaction
in Dixon tended to confirm the commonsense notion that where an
interplay is small, containing two or three persons, rules seem to be
strict regarding withdrawal of attention, and where an interplay is
large, as in the case of formally organized community socials, greater
leeway seems to be accorded to individual participants in momentarily
withdrawing attention from the accredited sender. The commonsense
explanation for this seems to be valid: if the disaffection of one
participant is likely to destroy the interplay (as in the case of
two-person interplays or in the case of multi-person formal interplays
where the recognized sender withdraws his own attention) then it is
strongly tabooed; if it is not likely to destroy the interplay then
withdrawal is only mildly disapproved.

6. There are times when the definition of the situation established in
an interplay may evolve, develop, or shift rather markedly, so that it
becomes reasonable to ask whether or not two different interplays have
not been grafted onto each other, the same set of participants and the
same focus of attention serving one interplay up to one moment and
another interplay afterwards. One often finds, however, that when one
participant ``changes the topic'' completely, and has not done so to
save the situation from even graver tensions, then his insensitivity to
the prevailing mood and topic is felt to be somewhat improper. Those who
do want to change the tone or topic frequently feel obliged to effect a
smooth transition by means of messages that meaningfully link the
interplay as it was up until then with the interplay as it will become
under the direction of the individual initiating the transition.

7. Sometimes interplays of more than two persons may involve
differential recipient roles. If a sender has more than one recipient,
he may address his message to all of them together as a unit. In public
speaking this possibility is frequently an obligation, and speakers work
out devices for giving their hearers the impression that they are all
equally included. The sender may, on the other hand, address his message
to only one or two of his recipients, on the assumption that the
unaddressed recipients are none the less recognized as participants.
This communication arrangement is typical in small informal interplays.
In discussing the question, Bales has used the term ``target'' to refer
to the addressed recipient.\footnote{Bales and others, ``Channels of
  Communication in Small Group Interaction,'' p.~462.} The addressed
recipient is usually given the visual attention of the sender, this act
providing both symbol and source of preferential recipient
status.\footnote{By definition, of course there can be no one with
  unaddressed status in two-person interplays. Miller (\emph{Language
  and Communication}, p.~251), in making a similar distinction between
  addressed and unaddressed recipients, suggests that telephone
  conversations necessarily provide for no unaddressed recipients.
  Miller's illustration fails to distinguish between unaddressed
  recipients each of which the sender knows is present and unaddressed
  recipients who the sender does not know are present.}

8. Recipients may enjoy many different kinds of clearly defined
privilege with respect to assuming the role of sender. In some
interplays, the addressed recipient may be accorded more right to take
over the role of sender than is accorded to the unaddressed recipients.
As previously suggested, in some interplays, certain categories of
recognized recipients may not be given the right to become senders. In
formally organized meetings, for example, one category of participant
may have the right to raise questions during the formal discussion,
another category may have the lesser right to raise questions only after
the formal discussion has ended, and a third category may have no right
in this respect at any time. Similarly, children at the dinner table are
sometimes allowed to listen but forbidden to talk;\footnote{J. H. S.
  Bossard, ``Family Modes of Expression,'' \emph{Amer. Sociol. Rev.}, X
  (226--237), 229.} if not forbidden to talk, they be ``helped out'' and
in this way not permitted to finish a sentence by themselves.\footnote{\emph{Ibid}.,
  p.~228.}

Differential sending status is often expressed in terms of restrictions
placed upon the kind of message that can be sent. At public meetings,
for example, the audience may be restricted to sending the kind of
message that can be conveyed by upward or downward modulation of
terminal applause. During certain interplays, one category of sender may
only be allowed to say, ``Yes, sir,'' or, ``No, sir.''

9. If a sender addresses his message to some recipients and not to
others, his unaddressed recipients may shift the focus of their
attention so that it falls, in part, upon the addressed recipients as
well as upon the sender. An extreme example of this occurs in the case
of activities involving by-play between two performers that are staged
in front of an audience. In such cases, the audience may tend to focus
its attention on an interplay of staged messages instead of upon a
single message.

% CHAPTER XI: EXPRESSION DURING INTERPLAY
\chapter[CHAPTER XI: EXPRESSION DURING INTERPLAY]{Chapter XI: Expression During Interplay}
\label{ch:Chapter XI: Expression During Interplay}
\chaptermark{CHAPTER XI: EXPRESSION DURING INTERPLAY}

\newthought{In an earlier part}
\marginnote{\href{https://doi.org/10.32376/3f8575cb.1dc2eaf1}{doi}}of this study, it was suggested that communication,
seen as a physical process, provides many events that are well adapted
to serve as expressions, witting or unwitting, especially expressions of
the evaluative judgment that participants make of one another. As one
type of communication arrangement, an interplay provides many vehicles
for carrying information about the judgments participants make of one
another. Of course, an event which is well designed to express such
evaluations may not come to act in this way, and an event which does
come to be expressive in this way may not be employed by anyone as a
source of information. Furthermore, a vehicle which commonly carries
information of one kind in one culture may carry a different meaning in
another culture. In this chapter some of the frequent sources of
expression in interplay will be considered.

1. One source of expression during interplay is to be
found in the manner in which recipients attend to the sender.
Chesterfield's view of this matter is interesting:

\begin{quote}
There is nothing so brutally shocking, nor so little forgiven, as a
seeming inattention to the person who is speaking to you; and I have
known many a man knocked down for (in my opinion) a much slighter
provocation than that shocking inattention which I mean. I have seen
many people who, while you are speaking to them, instead of looking at,
and attending to you, fix their eyes upon the ceiling, or some other
part of the room, look out of the window, play with a dog, twirl their
snuff-box, or pick their nose. Nothing discovers a little, futile,
frivolous mind more than this, and nothing is so offensively ill-bred;
it is an explicit declaration on your part that every, the most
trifling, object deserves your attention more than all that can be said
by the person who is speaking to you. Judge of the sentiments of hatred
and resentment which such treatment must excite in every breast where
any degree of self-love dwells, and I am sure I never yet met with that
breast where there was not a great deal. I repeat it again and again
(for it is highly necessary for you to remember it) that sort of vanity
and self-love is inseparable from human nature, whatever may be its rank
or condition; even your footman will sinner forget and forgive a
beating, than any manifest mark of slight and contempt. Be therefore, I
beg of you, not only really, but seemingly and manifestly, attentive to
whoever speaks to you; nay more, take their tone, and tune yourself to
their unison. Be serious with the serious, gay with the gay, and trifle
with the triplets. In assuming these various shapes, ends our to make
each of them seem to sit easy upon you, and even to appear to be your
own natural one. This is true and useful versatility, of which a
thorough knowledge of the world at once teaches the utility, and the
means of acquiring.\footnote{\emph{Letters of Lord Chesterfield to His
  Son}, pp.~261--262.}
\end{quote}

2. In the literature, some attention has been given to the
fact that lulls in conversation or frequent interruptions express
something significant about the relation of the participants. Chapple
and Coon have suggested that:

\begin{quote}
The degree of adjustments between two individuals may be measured in
terms of the amount of synchronization between their action and
silences. When two persons are able to interact, within the normal
limits of their interaction rates, in such a way that they do not
interrupt each other frequently and that neither fails to respond when
the other stops talking, they are well adjusted, . . . the disturbing
effects of interruptions and failures to respond produce changes in the
sympathetic nervous system which the physiologists describe as pain,
fear, and range.\footnote{E. D. Chapple and C. S. Coon, \emph{Principles
  of Anthropology} (New York: Holt, 1942), p.~39}
\end{quote}

\noindent And a clinical study by Chapple and Lindemann shows that ``double
action'' and double silence occur very little among normals but much
more frequently among the disordered.\footnote{E. D. Chapple, and E.
  Lindemann, ``Clinical Implications of Interaction Rates in Psychiatric
  Interviews,'' \emph{Human Organization}, I, 111} In the case of
improper lulls in the interaction, it is to be noted that the impression
made by lulls on those who must experience them varies a great deal from
one type of interplay to another. In Dixon, in formally organized
interplays such as those occurring during a concert, lulls created by
the failure of one performer to follow another rapidly enough, or the
lull caused by the failure of volunteer musicians to appear at the time
dancing was to have begun, caused some disorder and strain, but on the
whole such lulls were taken in stride as an expression of the
incompetence of those who had been chosen to run the concert. On the
other hand, lulls which occurred during informal ``ad hoc'' interplay
seemed to be a more serious thing; they tended to be taken as an
expression of the fact that the participants had too little ``in
common'' to justify informal social intercourse.

3. Of the many different sources of expression in
interplay, students of interaction seem to have given most consideration
to forms of what might be referred to as ``attention quota,'' that is,
the relative amount of time during which a given participant acts as a
sender, or the relative number of messages he sends. Chapple and Coon
have suggested that each person has a demand level for attention which
is peculiar to him and which he tries to establish in all of his
interplays.\footnote{Chapple and Coon, \emph{op. cit}., p.~39.} One
student, in discussing the casual coming together of persons in brief
conversation, says, ``Our earlier experience had indicated a very strong
relationship between decision-winning or leadership and talking-time in
ad hoc groups for four persons.''\footnote{F. L. Strodbeck. ``Husband
  and Wife Interaction,'' \emph{Amer. Sociol. Rev}., XVI (468--473),
  469.} Another student has reported a correlation of .93 between the
time a participant in an eight-man ``group'' spent talking and the votes
he received from observers for having demonstrated
leadership.\footnote{B. M. Bass, ``An Analysis of Leaderless Group
  Discussion,'' \emph{J. of Applied Psych}., XXXIII, 527--533,
  especially pp.~531--532.}

Students interested in the expressive significance of attention quota
have quite frequently employed this factor as an index of a rather
complex variable, namely, ``informal status within the
group.''\footnote{Leon Festinger and John Thibaut, ``Interpersonal
  Communication in Small Groups,'' \emph{Theory and Experiment in Social
  Communication}, by Leon Festinger and Others (Ann Arbor: Edwards
  Bros., 1952), pp.~37--49, especially p.~44, claim to have shown that a
  large volume of communication may be directed to and originate from a
  participant who violently disagrees with other participants. Attention
  quota in such cases would probably not be an index of ``status'' in
  the interplay. Bales, ``The Equilibrium Problem in Small Groups,''
  \emph{Working Papers in the Theory of Action}, by Talcott Parsons,
  Robert F. Bales, and Edward A. Shils (Glencoe, Ill.: Free Press,
  1953), p.~131, cogently argues that the supposition that high
  attention quota is related to status is at least a good working
  assumption, for it causes us to examine critically any deviation from
  this rule.} The drawback of this approach is that often participants
also realize that attention quota is a significant expression and
attempt to increase or decrease the number or length of the messages
they send, in an effort to control the impression that they feel their
actions give. Perhaps a less famous expression of rank within the
interplay, such as the quote of time or times during which a given
participant is the addressed recipient, would provide a more reliable
index.\footnote{There are, of course, many other expressions of
  differential evaluation within an interplay, some of these stressing
  the rank of the participant within the interplay and some stressing
  more the rank of the participant in the wider social world. For
  example, when two participants attempt to reply at the same time to a
  sender's message, the participant who is the more highly esteemed of
  the two is often according the right to proceed. Sometimes, of course,
  this introduces an untactful show of superiority, and an attempt may
  be made to resolve the sign tension by allowing the first of the
  respondents to have the floor. In Dixon, when two persons start to
  answer a third and appear to have started at exactly the same time, a
  brief moment of disorganization follows, often terminated in laughter.}

In any case, it is convenient to think of the granting of attention as a
kind of indulgence, for in this way we can better appreciate that esteem
for the sender is merely one of the reasons we might have for granting
him our attention.

In Dixon, the use of attention quote as a general measure of informal
leadership or esteem was grossly inadequate in certain contexts. Three
of these may be mentioned.

First, the occurrence of something special to a particular
participant---a birthday, a minor accident, an achievement,
etc.---tended for a time to place the participant in the focus of
attention and make him the central object of reference. (Of course, it
may well have been that the lower a participant's usual position, the
more drastic must be the special event that enables him to monopolize
attention.)

Secondly, persons who were too far removed from the commoners to find a
place within their ranking structure were frequently accorded long
period of ungrudging attention. Small children, strangers, gentry,
kittens, chronic misbehaviors---all these qualified for attention
indulgence. This patter of treatment, incidentally, seemed also to be
extended sometimes to persons who ``ought to have known better,'' but
who none the less attempted to obtain more attention than was fitting
for them. On such occasions the offender was led into taking even more
attention than he may have wanted, for which unknowingly paid the price
of being classified along with children, cats, and strangers.

Thirdly, during formally organized recreation, persons frequently seemed
to act in capacities which they did not judge as important and hence
seemed not much concerned over the allocation of attention at these
times. Thus, at the semi-annual concerts, the esteem in which a
particular performer was held in the community at large (as a person,
not as a performer) did not seem to influence very much the willingness
of the audience to accord the performer her attention. In fact, games
such as whist or ``beetle'' formally incorporated the right for each
participant to have either equal attention indulgence or an equal chance
of receiving a large amount of this indulgence.

When the above mentioned qualifications were not operative, attention
time tended to be an indulgence in Dixon, and an indulgence that persons
were felt to deserve according to their rank in the interplay. It
seemed, however, that in addition to the relative factor an absolute one
was operative. In two-person interplay, no cases were observed where the
subordinate did not have the right to convey some messages. The
statement-reply nature of communication would itself have operated in
this direction. As the number of participants increased, however, the
number of messages thought proper for a particular participant seemed to
decrease more than proportionately. The indulgence involved in receiving
the attention of more than three or four persons seemed to be considered
so great a thing that more than a moment of it was frequently thought to
be a presumption on the part of the person who obtained. A point seems
to be reached where even those of high status in the interplay feel that
it is presumptuous or dangerous for them to accept the focus of
attention for more than a moment.\footnote{Apparently one solution for
  this problem is for a sender rigorously to direct his message to a
  particular participant, often the informal leader, as a means both of
  obtaining extra legitimacy for his demands and at the same time
  providing a simulation of a two-person interplay.} Perhaps this may
help us to account for the fact that has been widely cited in the
literature, namely, that informal interplays of more than five persons
tend to be unstable and tend to suffer a cleavage into two or more small
interplays.\footnote{See, for example, John James, ``A Preliminary Study
  of the Size Determinant in Small Group Interaction,'' \emph{Amer.
  Sociol. Rev}., XVI, 474--477, especially p.~476.} Perhaps this may
also help to throw light on some of the social functions of organizing
some interplays in a formal way, suggesting that a formally selected
sender acting in a formally designated and limited capacity does not
have to rely on his own personality and general status as a warrant for
the attention he receives and can therefore accept with impunity the
attention of many persons.

During informal interplay in Dixon, when the sender in one small
conversational cluster suddenly received the attention of members of a
neighboring cluster, embarrassment frequently resulted and the sender
frequently terminated his message in a rapid and somewhat disorganized
way. With one class of exceptions, only one instance was observed where
a talker was willing to accept the attention of a relatively larger
number of listeners for more than a few seconds. At a dinner party of
twenty-five, a man made a comment to his neighbors on the political
situation, and after answering a question raised by a person at the
other end of the table went on to air his views to the whole room.
However, he was a man famous in the community for playing communication
tricks; he seemed to have sensed, on this occasion, that many listeners
felt embarrassed, but he talked to the whole table in spite of it.

The class of exceptions observed regarding size limitation on informal
interplay pertains to the institution of ``story telling'' which is
found in Bergand and a few other Britain's islands. When from about six
to about fifteen men gather in a room, a person reknowned {[}\emph{sic}{]} as a
story teller may be persuaded to settle back and tell a tale. Tales
usually have to do with more heroic days, when sailing vessels were
still employed and when the harbor in Dixon was filled with crafts. The
story tellers seemed to be able to handle the attention they received
with exquisite poise and balance, injecting enough personal involvement
and reference to keep attention alive, and yet doing this in such a way
that the indulgence of the listeners was transferred from the story
teller to the past about which he was talking. In Dixon, the idea that
there might be a communication arrangement half way between informal
interplay on one hand and formally organized interplay on the other
seemed to be dying, and only a few old men still seemed to appreciate
that the institution of story telling required a special skill and
manner and involved a special communication license with respect to
attention indulgence.

4. One source of expression in interplay is to be found in
a person's entrance into or initiation of an interplay and in his manner
of leaving or terminating it.\footnote{There are analogous rules for
  guiding initiation and entrance, withdrawal and termination in the
  case of social occasions. For example, in some social circles in our
  society it is felt that early leavetaking is a possible affront to
  those remaining, and there is a formalized rule that no one may leave
  until the highest-ranking person makes a visible move to do so. In
  other circles in our society, it is understood that the more intimate
  the relationship between a particular guest at a party and his host,
  the longer it is proper for him to stay, and that guests on more
  distance terms with the host ought to leave in time to give more
  intimately related guests an opportunity of being alone with the host.}
An illustration may be taken from an early American etiquette book,
where conduct with respect to conversational clusters at parties is
considered:

\begin{quote}
If a lady and gentleman are conversing together at an evening party, it
would be a rudeness for another person to go up and interrupt them by
introductions a new topic of conversation. If you are sure that there is
nothing of a particular and private interest passing between them, you
may \emph{join} their conversation and strike into the current of their
remarks; yet if you then find that they are so much engaged and
entertained by the discussion that they were holding together, as to
render the termination or the change of its character unwelcome, you
should withdraw. If, however, two persons are occupied with one another
upon what you guess to be terms peculiarly delicate and particular, you
should entirely withhold yourself from their company. If you are talking
to a lady with the ordinary indifference of a common acquaintance, and
are only waiting till some one else comes up, for an opportunity to
leave her, you should not move the instant another reminds, for that
would look as if your previous tarrying had been compulsory, but you
should remain a few moments and then turn away.\footnote{\emph{The
  Canons of Good Breeding: or the Handbook of the Man of Fashion}
  (Philadelphia: Lee and Blanchard, 1839), pp.~68--69.}
\end{quote}

\noindent It is to be noted that participation status in an interplay involves
important rights and obligations. In general, accredited status in the
same interplay puts persons in an extremely good position to convey
linguistic and expressive information to one another. In a sense, shared
status of this kind opens persons up to one another. We can therefore
appreciate why persons are usually interested in seeking or avoiding
accredited participation with specific other persons. Perhaps,
therefore, we can also understand why it is that any alternation in the
likelihood or probability of two persons entering into interplay with
one another tends to be marked by ceremony of some kind.

It is to be noted, further, that persons who enter into interplay with
each other tend to show more accommodative consideration of one another
than they would have been had they not entered together into the
interplay.\footnote{Knowledge of this fact is exploited by ``stemmers''
  or street salesmen who force persons into conversational interaction
  without waiting for a justifiable or proper pretext for doing so. The
  stemmer then phrases his salestalk or ``pitch'' in such a way that the
  potential customer must open contradict the salesman if a sale is to
  be avoided. In order not to have to contradict someone ``to his face''
  and in order to terminate what is in any case an improper interplay,
  potential customers frequently agree to the sale.} This fact seems to
be especially true of sender and recipient.\footnote{Since a sender
  need be more careful, ritually speaking, of an addressed recipient
  than an unaddressed recipient, senders sometimes attempt to convey a
  remark for which there are unaddressed recipients, or even persons who
  are forced into the role of effectively excluded overhearers, but for
  which there is pointedly no addressed recipient. We sometimes call
  this communication arrangement ``talking into the air.'' On the
  island, when an individual wished to expression an opinion which eh
  could not quite bring himself to convey to an addressed recipient, he
  would sometimes address his remark to the kitchen cat, or to a small
  child, or ``into the air'' in a ruminative, editorializing and
  inwardly directed spirit. Similarly, on the island as elsewhere in
  Wester society, a closely related pair of persons, such as husband and
  wife, will sometimes wait for the presence of a third person before
  voicing criticism or approval of the other member of the pair. A third
  person can be used as an addressed recipient and be told things that
  the talked-about member of the pair can accept as an unaddressed
  recipient but not as an addressed recipient. The presence of this kind
  of leeway is one of the factors which distinguishes three-person
  interplay from two-person interplay.}

Since joint participation makes persons available to each other it is
not surprising that we find that rules exist for determining who may
break into conversational interplay with whom, and under what
circumstances this may be done. In general, in our society, it seems
that we have a right to bring a person into interplay or a right to
enter an interplay that is already in progress to the degree that our
action cannot be construed as an effort to reduce social distance or
improperly acquire strategic information. If the interplay is patently
going to be brief, then strangers can accost each other, as when one
persons asks another for directions, or a match, or the time. If visible
proof is not available that the interplay will not involve participants
in entangling alliances, then, at least in our cities, strangers do not
quite have the right to engage each other in conversation. It may be
noted that the institution of ``introduction'' in our society
establishes between persons, in many cases, the right and the
obligations of entering into interplay with each other whenever this
becomes a physical possibility, even if the possibility is quite
unexpected. It may also be noted that certain social occasions in our
society, such as informal social parties, give all those present, by
virtue of their presence, the right to enter or be called into any
interplay in progress or to initiate interplay with anyone present.

In Dixon, as in many other rural regions in Western society, all adults
of like sex have the right and obligation of momentarily entering into
interplay with each other when passing on the road or field.\footnote{The
  few exceptions to this rule are considered later in another context.}
These interplays are required to be positively toned and accommodative.
A difficulty frequently arose, therefore, between persons who are
antagonistic to one another. To enter into interaction with an enemy
tended to call forth more accommodation than one wanted. To refuse to
enter into interaction when obviously in a position to do so tended to
signify too great an insult.\textsuperscript{16} Hence, between gentry and
crofter, and between crofters who had ``fallen out'' with each other,
avoidance relationships were sometimes practised as a solution to the
problem. Persons hostile to one another tended to avoid each\marginnote{\textsuperscript{16}\setcounter{footnote}{16} This is what is known as a
  ``cut.'' One gets the impression that this communication arrangement
  is less frequently employed today than in previous periods. We avoid
  interaction by avoiding persons' eyes, but we seem to be inclined to
  allow the person whose eyes have been avoided to retain the belief
  that he has been accidentally overlooked.} other's
eyes if possible and not to frequent the same place at the same time.

Once persons have entered together into interplay, termination of the
interplay commonly becomes a delicate matter. If one participant
withdraws before the others do, this act is often taken, justifiably or
not, as an expression of the departed one's attitude to those remaining.
This possibility causes some persons to be leary about initiating
interplays which they do not have a ready means for terminating. In
official circles, where highly sacred participants must be protected,
termination of an interplay (and a social occasion) is signaled by the
leavetaking of the highest-ranking participant, others not leaving until
that participant does.

The eventual necessity for every interplay to terminate constitutes a
sign situation; whether or not participants desire it, something will
probably be taken to have been expressed. There are several standard
strategies for resolving this sign dilemma.

First, allowances are made for clearcut extenuating circumstances. Thus,
on the island, a housewife whose soup boiled over could rely on the
noise and smell to make good her leavetaking and could be sure that her
hasty departure would be tolerated. Messages bearing painful tidings
could also be relied upon as a sufficient pretext to leave a
conversation or an occasion.

Secondly, the leavetaker may strongly confirm to the participants that
the self that is leaving is not, in a sense, the self that the
leavetaker thinks most highly of. By apologizing profusely, or by
offering an excuse which clearly puts him on the side of those remaining
and in opposition to the obligation that calls him away, or by joking to
suggest that the self that is hurrying away is not a serious self, the
leavetaker can leave tactfully.

It is interesting to note that informal interplay is frequently
terminated, or at least that the termination is frequently confirmed, by
the participants moving away from one another. In Dixon an interesting
difficulty arose in this connection. When two persons met in a field and
engaged for a moment in informal interplay, they would attempt to close
out the interplay with the usual signals, such as ``good day,'' and the
like. This was satisfactory as long as the two persons happened to be
going in different directions. But if their paths diverged only slightly
from a single point, separation could only be gradually effected. In
such cases, the persons found themselves still close to one another even
though official good-byes had been made. Frequently the difficulty
seemed to be resolved by one participant either breaking into a run as
soon as paths started to diverge, perhaps offering an excuse for doing
so, or by one or both participants taking a path that involved more
clearcut divergence.\textsuperscript{17}

Thirdly,\marginnote{\textsuperscript{17}\setcounter{footnote}{17} A similar problem is sometimes found in
  our society when two persons, little acquainted, find themselves
  seated or standing close to one another for a long period of time.
  After a moment of ``small talk'' they find themselves with nothing to
  say and yet not in a position to terminate the interplay. Newspaper
  reading is often used as a thin excuse to break from the interplay in
  this situation. Newspaper reading seems to be the minimum activity by
  which an individual can withdraw from doing nothing and hence being
  open for interaction.} the withdrawing participant may stat as long as possible in
order to show that he is genuinely involved in the interaction. Thus,
during billiards, almost all the players followed the practice of not
leaving the hall immediately on completion of the game they were playing
in but rather waited out a few minutes of the next game, which did not
involve them as players. This was a final gesture that the evening's
play as such, and not merely their own turn, had involved them. This
kind of tactful delicacy was very common in Dixon.

As a final note on termination of interplay, it is to be suggested that
a very common strategy for ending an interplay is for all parties to
withdraw simultaneously. This seems to be accomplished by an exchange of
very minimal cues among the participants so that each becomes areas that
the conversation is about to be terminated and makes necessary
allowances. No one in such cases is left holding the interplay. On the
island, this kind of natural termination frequently occurred, especially
where the participants had been together before at occasions of similar
interplays and when some feeling of solidarity and mutual approval
existed.\footnote{Bales also seems to have noted this phenomenon. ``We
  note joking and laughter so frequently at the end of meetings that
  they might almost be taken as a signal that the group has completed
  what it considered to be a task effort, and is reading for disbandment
  or a new problem. This last-minute activity completes a cycle of
  operations involving a successful solution both of the task problems
  and social-emotional problems confronting the group.'' Bales, ``The
  Equilibrium Problem in Small Groups,'' \emph{op. cit}., p.~143.}

In the hotel kitchen natural termination of mealtime interplay became
linked with feelings of work control and self-respect on the part of the
employees. After a meal, everyone would linger for a time over teat and
cigarettes, talking, and allowing a margin of time to elapse, even
during the busiest days, between when the meal was technically finished
and when it was sociologically finished. The managers, who wanted the
employees to return to work as soon as possible, often felt uneasy about
waiting for natural termination and tried to hurry up the ending.
Mr.~Tate was often untactful in these matters and would withdraw
psychologically from the interplay and, in a changed tone of voice, tell
the employees that there was much work to be done. On many occasions
this command was openly overlooked (often, apparently, unconsciously)
and the participants would have an extra cup of tea, or bring a chair
that had been removed from the table when they had gotten up for
something back to the table, on the assumption that persons sitting
around a dinner table could not be openly commanded to do work.
Mrs.~Tate seemed more subtle in her approach and would try to terminate
the interplay from within, as it were, saying in the tone of voice being
used in the interplay at the time that she guessed she had to get back
to work. This often succeeded. Interestingly enough, she felt a little
uneasy at using this technique and once admitted to those present, by
means of a half-guilty smile, that she had been trying to affect a
spontaneous reaction.

% CHAPTER XII: INTERCHANGE OF MESSAGES
\chapter[CHAPTER XII: INTERCHANGE OF MESSAGES]{Chapter XII: Interchange of Messages}
\label{ch:Chapter XII: Interchange of Messages}
\chaptermark{CHAPTER XII: INTERCHANGE OF MESSAGES}

\newthought{In terms of the}
\marginnote{\href{https://doi.org/10.32376/3f8575cb.3bc11117}{doi}}framework of this study, the minimal concrete unit of
communication consists of the sign behavior of a sender during the whole
period of time through which a focus of attention is continually
directed at him. This unit of communication may be called a ``natural
message,'' or, where no confusion is likely, simply a ``message.'' Short
messages such as words or single sentences apparently have distinctive
phonetic features; long messages, such as an uninterrupted thirty-minute
talk, seem less neatly describable in phonetic terms. A single natural
message may, of course, involve different pieces of information, but
these differences are not relevant in terms of this study.

The concept of natural message has been employed in other studies of
social interaction and has apparently been independently hit upon. One
student gives the following definition for the basic unity of his study:

\begin{quote}
The unit of verbal behavior chosen was arbitrarily defined as the entire
statement a person made that occurred between the statements of
individuals immediately preceding and following the person's
expression.\footnote{Steinzer, \emph{op. cit}., p.~109}
\end{quote}

\noindent Two other students of interaction, in a work on attention quota or what
they call ``participation rates in small groups,'' give the following
statement:

\begin{quote}
The basic unit of participation labeled by the observers is the word,
sentence, or longer statement of an individual that follows such a
participation by one member and continues until it is terminated by an
appreciable pause or by the participation of another member. In other
words, an individual's uninterrupted contribution is taken to be one
participation.\footnote{F. F. Stephen and E. Y. Mishler, ``The
  Distribution of Participation in Small Groups: An Exponential
  Approximation,'' \emph{Amer. Sociol. Rev}., XVII (598--608), 600.}
\end{quote}

\noindent It is interesting to note that the acting profession employs a similar
unit, called ``a speech.'' In Mencken's definition, a speech is ``A unit
of an actor's spoken part; it may be one word or a
thousand.''\footnote{H. M. Mencken, \emph{The American Language},
  Supplement II (New York: Knopf, 1948), p.~691.}

Our commonsense view of linguistic communication, especially of the
conversational kind, leads us to expect that when one message
terminates, one of its recipients will take over the role of sender and
convey another message. We expect a statement made by one person to be
given an answer by another person. It is also to be expected that any
particular message, whether statement or answer, will have two
components, an expressive one and a linguistic one. Recipients will be
concerned with what the sender says and also with the way in which he
says it. However, when we examine conversational interaction closely, we
find that a more complicated process frequently occurs.

First, we find that the recipient, in paying attention to the message,
expresses the fact that he is doing so, by means of posture and facial
expression. Also, as a larger and larger fraction of the message becomes
transmitted, the recipient comes to be more and more in a position to
know what the message will contain and what consequence it will have for
him. The state of being in possession of this information seems to flow
over, impulsively and spontaneously, into expressive behavior. In
addition, the recipient seems to conduct an incipient internal
monologue, conveying to himself or to anyone who is close enough or
perceptive enough to hear, a rehearsal of how he is going to respond to
the message when it is finally terminated, or (and this would seem to be
more frequently the case) how he would like to respond to it were there
no reason to exercise forbearance and restraint. We shall refer to this
responsive expressive impulse and this self-communication as ``take,''
following a Hollywood usage which has precisely recognized this element
in communication. Of course, wittingly or unwittingly, the recipient's
take is conveyed to the sender as a source of impression, giving the
sender an opportunity of constantly checking up on the probably
consequence of his message. When the sender has just terminated his
message, the recipient perhaps reaches a culminative point in the
judgments and considerations which he addresses to himself regarding the
message. Once the recipient starts transmitting his considered reply to
the previous sender, his take may diminish. Thus, if we modify our
original decision of message to include the take along with the reply
that emerges from it, we see that the expressive weighting in a message
may decline in importance as the importance of the linguistic component
increases.

In conversational interaction where messages are very brief and where
participants feel they need not exercise much control over the
expression of their responses, the take and reply in a message may
become merged and may overlap considerably. In other kinds of
communication situations, the two components---take and reply---may be
distinctly separated. For example, we may explode when we receive a
letter and an hour later answer it in a friendly and polite way. In
cases where an individual overhears others talking and is not himself in
a position to answer them, we may get from the eavesdropper a take and
no reply.

In general, then, a take is expected to be genuinely expressive,
although in fact it is often feigned. It is not officially directed to
the sender.\footnote{A sender may negatively sanction a recipient by
  openly asking the recipient to explain linguistically his take. This
  forces into the realm of accredited messages what was meant to be
  unaccredited.} Sometimes it involves words of the class that are
called expletives. A reply is expected to be more linguistic in nature.
It is officially directed to the previous sender, not to oneself. This
differentiation is also a temporal one; a take precedes (whether partly
overlapping or not) a reply.

It is interesting to note that when a message is long, recipients
frequently employ their take in a fairly open way as a signal of their
attitude toward the message. Thus, during a lengthy political speech,
cheers, hoots, and boos may be quite openly conveyed by recipients while
they are still in the process of receiving the very message to which
their take is a response.

On the island, adults in talking to other adults attempted on the whole
to suppress any signs of their take to a message, except signs conveying
the fact that they were attending to the message and were generously
receptive to it. A frank take to the message seemed to be indulged in
only when the recipient could have assurance that the sender would not
see it. Among young people, who were presumably not yet obliged to treat
each other with the delicacy required between adults, take was often not
suppressed and was often given an important role in the communication
process. Temper tantrums and sulking, while apparently relatively rare
among island children, illustrate this. Further, on occasions when a
person was being teased or was having his leg pulled, it was thought
cooperative of him to evince as great a surprise take as possible, both
in response to the lie he was being told and in response to being told
that he had been told a lie. Among pre-adults, an explosive take in
teasing situations seemed spontaneous and not put on; girls, especially,
had the bait of attacking male teasers with their fists and feet out of
desperation at not having any other means of response under
control.\footnote{In our society the so-called ``surprise party''
  features a take as the high point of the ceremony. The person for whom
  the party is given repays the givers by openly expressing a highly
  emotional take, one that mixes gratitude with loss of poise, before
  collecting himself for a linguistic response. Similarly, children are
  made to close their eyes until a present can be arranged so that they
  will see it suddenly and as a whole, thus allowing parents to observe
  their children in what is taken to be an unguarded reaction of pure
  delight.}

It may be noted that in Dixon, as apparently elsewhere, unaddressed
recipients sometimes neglect to prevent a frank expression of their
valuation of the sender from appearing on their faces; feeling in
observed, they take the opportunity of spontaneously expressing for
their own private consumption what they really feel. Persons in Dixon
sometimes suddenly turn on their unaddressed recipients in order to
catch them for a moment in inadvertent sincerity.

\vspace{.2in}
\begin{centering}

\Large{* * * * *}

\end{centering}
\vspace{.17in}

\noindent When we examine an interplay we often find that the messages which occur
within it are not evenly spaced out in time but occur, rather, in
temporal clusters or groupings. Messages within one of these temporal
groupings or spurts of communication are usually more closely related in
content and culture than are messages which occur in different temporal
groupings of the same interplay. Frequently the first message in one of
these groupings presents a ``statement'' of some kind and the following
messages in the grouping provide a reply, then a reply to the reply, and
s on. A communication spurt of this kind may be called an
interchange.\footnote{This concept derives in part from Chapple's
  monograph, ``Measuring Human Relations,'' \emph{Genetic Psychology
  Monographs}, XXII, 3--147, especially pp.~26--30, ``Definition of an
  Event.'' See also A. B. Horsfall and C. A. Arensberg, ``Teamwork and
  Productivity in a Shoe Factory,'' \emph{Human Organization}, VIII
  (13--25), 19, where the following statement is given: ``Thus the
  simplest observed event might have been a simple interaction, one
  which took place only between two persons on our record. The first
  action was initiation, the second was response; together the two
  actions established an initial, simple unit of interaction. We could
  label such a simple interaction''a.'' It might have been merely a
  single exchange such as greetings in passing, e.g., A: ``Hello B,'' B:
  ``Hello yourself,: or it might have continued for some time with many
  exchanges between the two. The measure of the time from its initiation
  in an act of A's to the last response in the last act of B's for the
  time gave the''duration'' of an event. Speaking in general terms, an
  event terminates with a change of two, or with the entrance of a third
  person upon the scene, to whom the others, or one of them, act or
  respond.''}

An interchange may involve several persons but ordinarily it is
restricted to two persons who alternately take the role of sender and
addressed recipient while all the other participants in the interplay
restrict themselves to the role of unaddressed recipient. The two
persons may question or answer each other, or engage each other in
parries and thrusts, while the unaddressed recipients merely watch. The
two persons who are actively engaged in the interchange may not, of
course, have equal sending rights. It should be added that in large
formally-organized interplays of the actor-audience kind, the audience
not only limits itself to conveying a few kinds of recognized messages
but also comes to serve for the performer as a single addressed
recipient. We thus tend to get interchanges between two actors, one of
whom is the whole audience.\footnote{This can be very clearly seen in
  the question and answer interchanges between a revivalist preacher and
  his congregation.}

After a particular interchange is completed, certain communication
behaviors are possible: two other participants may provide the
subsequent interchange of the interplay; the same participants may
initiate a new interchange; one of the original participants may
initiate a new interchange with a previously unaddressed recipient; or
the interplay may be terminated.

It should be noted that a sender and his addressed recipient seem to
accept greater obligation towards each other with respect to mutual
responsibility, forbearance, and accommodation than do a sender and his
unaddressed recipients. When there are no more than two recognized
participants in an interplay, then, of course, the heightened
responsibility between sender and addressed recipient necessarily
applies to both participants.\footnote{It may be noted that this
  responsibility is specifically counteracted in certain situations and
  specifically exploited in others. Confessors such as priests and
  psychoanalyst is sometimes arrange matters so that the confessing
  cannot see the involuntary gestural response which the confession
  evokes from the confessor. In this way the confessant is less likely
  to feel restrained in the presence of the confessor from talking of
  things which ordinarily constitute improper topics of conversation.
  Simi-}

The unit of the interchanges has been considered so far chiefly in
reference to its physical characteristics, namely, a rapid exchange of
messages between two participants. In this chapter an attempt will be
made to account for the nature of the unit by reference to two
explanatory principles having to do, first, with communication as a
ritual system, and, secondly, with communication as an informational
system.

\newpage Explicitly\marginnote{larly, in disciplinary and highly structured situations, where
  superordinates may be required to look straight ahead and not into the
  eyes of the sender. Third-person forms of address, found in extremely
  ritual situations, are perhaps attempts to deny the fact that
  face-to-face interaction is actually in progress.} or indirectly, any message involves, or may be taken by
participants to involve, an evaluation or judgment of all persons who
receive it. Hence the sender necessarily runs the risk of giving offense
to the image that his recipients have of themselves and of him or to the
image they have of things with which they feel identified. The rapid
sequence of messages which follows after the initial message of an
interchange either conveys to the sender an acknowledgment that his
valuations are provisionally acceptable or modifies these valuations
until provisional agreement is reached by all participants. The lull
which follows an interchange is permissible because the working
acceptance conveyed by the interchange ensures that a brief silence will
not be taken as a sign that someone has been offended. A silence
\emph{during} an interchange usually conveys the fact that the
recipients cannot frame a reply that is workable consistent with their
own valuations and the valuations projected by the sender.

In Chapter VII it was suggested that a ritual model might well be
fruitful in the study of interaction. Instead of employing a
rationalistic bias, claiming that we perform our tasks strategically
taking into consideration the probable response of others, we can employ
a ritualistic bias, claiming that we interrupt our tasks in order to
worship and placate the gods around us. Offerings and placations are, of
course, a consequence of our taking into consideration the gods' likely
response to us, but this likelihood is not established by indications
that the gods make concerning their future effect upon us but rather by
the religious tenets and norms which guide our treatment of them and the
idols which represent them. In human interaction, however, the idol
which we are ritually careful of is also ritually careful of us. If we
offer him up a prayer or perform a gesture of obeisance, he, unlike
other kinds of idols, can answer us back, blessing us, or returning the
compliment of worship. Thus, instead of a single act by which a devotee
expresses his attitude toward a graven image, we get a double act, a
statement and reply, for the graven image is in a position to respond to
the offering that has been made to him.

The ritual model of the interchange is suggested by Park in his
introduction to Doyle's \emph{Etiquette of Race Relations in the South}:

\begin{quote}
Etiquette is concerned primarily with personal relations. It grows up in
the first instance, perhaps, as the spontaneous expression of one person
in the presence of another, of a sentiment of deference. Under
ordinarily circumstances such an attitude of propitiation of one
individual implies and is likely to evoke a corresponding expression of
benevolent recognition on the part of that other. Expression and
response rather than stimulus and response are the natural termini of
every instance of social interaction.\footnote{R. E. Park, \emph{op.
  cit}. , p.~182.}
\end{quote}

\newpage\noindent By employing this model we can see that an interplay is not a continuous
flow of communication; it proceeds by discontinuous jerks or steps, an
interchange at a time.

Perhaps most interchanges---as Park implied---are limited to only two
messages, expression and response, as he called them. An actor usually
has a fairly clear understanding of the expectations of his recipients
and is sensitive for many reasons to the rule that requires persons to
be treated with tactful concern for their attributes or sacred
qualities. We may therefore expect him to restrict his messages, in most
cases, at least, to ones to which it is possible for a recipient to
express acceptance. The recipient, in turn, must express something or
else become responsible for the ambiguities conveyed by silence. In this
way we can understand the prevalence of two-message interchanges.

We have attempted to account for the character of interchanges by
reference to the fact that persons are ritually delicate objects which
must be treated with care, with ceremonial offerings and propitiations.
Working acceptance often marks the termination of an interchange, and a
working acceptance is required in order to keep in control the eddies of
insult and offense, of reprisals and counter-reprisals that persons can
involve themselves in. A second explanatory principle for the nature of
the interchange may be suggested now.

No matter what it is that a sender wishes to communicate, it would seem
that his object is to communicate successfully; he wants the recipient
to receive the message and to receive it correctly. If a recipient
replies to the message he has received and gives some form of answer,
then the original sender can use this reply as a test of whether or not
his original message has been correctly received. Whether the reply is
one of agreement or disagreement may be of secondary importance, as long
as it is a meaningful reply, for if the reply is meaningfully related to
the original message, then the original sender can be sure that the line
of communication between himself and the other is in effective
operation. The working acceptance that is achieved by the time the
interchange terminates---whether the achievement of this measure of
consensus has required only one exchange of messages or a long series of
exchanges---may serve chiefly to signify that the participants
understand one another, not that they agree with what they understand.
And, in fact, many working acceptances seem to be limited to a meager
consensus of this kind.

We have attempted to account for the fact that an interplay proceeds by
steps, an interchange at a time, by reference to two factors, a ritual
factor and an informational factor. Taken together, these two factors
seem to supply a partial explanation for why an interchange takes the
form that it does. The aim of this study, however, is to describe, not
to explain, and in the next and following chapters many varieties of
interchanges will be illustrated. Before proceeding to this, it will be
convenient to raise four questions about the nature of interchanges.

A message has been defined as the sign behavior of an individual during
the continuous period when he is the focus of attention. However,
sometimes during a long message a sender will pause, obtain a momentary
take of approval from his recipients, and then launch into what appears
to be a different message. Interestingly enough, both Steinzer, and
Stephen and Mishler, qualify their definition of a message in this way.

\begin{quote}
If the person stopped talking for five or more seconds, then continued,
the statement was counted as two units.\footnote{Steinzer, \emph{op.
  cit}., p.~109}
\end{quote}

\begin{quote}
However, if there was a clear change of content during the course of a
lengthy contribution, it was taken to be the beginning of a new unit of
participation.\footnote{Stephen and Mishler, \emph{op. cit}., p.~600}
\end{quote}

One way to account for these double messages is to say that they are in
fact separated by a fleeting message on the part of the recipient.
Another explanation is possible. Often a particular sender contributes
the last message in an interchange and then happens to be the sender who
contributes the first message in the next interchange. We would then
expect that there would be little if any connection between the two
parts of the sender's message and that at the same time there would be
little if any interruption of them on the part of others. Thus we have
an analytical explanation of what originally appeared to be an awkward
qualification that commonsense observation forced upon the definition of
a message.

Secondly, it is apparent that while many interplays are wholly made up
of clearly articulated interchanges, this is not always the case. For
example, when men in Dixon gather to talk about their exploits as seamen
in distant ports, one participant will tell his tale and the moment he
has finished (or a moment before he has quite finished) another
participant will tell his, and then another participant, and so on. In
these conversations, one message will cling to the topic established by
the previous one, but the statement-and-answer character of
communication will be muted. In a certain sense it would be more
realistic to say that each participant was merely waiting out his turn
to take the floor and obtain a share of the attention indulgence. Thus,
while this study is limited to a consideration of the interchange as a
concrete unit, it should be understood that there are other natural
units in interplay as well.

Thirdly, in making a distinction between interplay and interchange, it
must be admitted that it is possible for an interplay to contain only
one interchange and hence be coterminous with it. Thus, there are many
kinds of interchange taking the form of courtesies that are coterminous
with the interplay that incorporates them. This fact may be placed
alongside the fact that a social occasion may be coterminous with the
interplay it incorporates.

Finally, as suggested in a previous chapter, an entire interplay may
function as one message in a prolonged series or exchange of messages.
An interplay may take a form other than the rapid exchange of concrete
messages. The overall treatment of an individual during an interchange,
or interplay, or social occasion may function as a single message in an
extended exchange of messages. The individual may respond by
appropriately adjusted behavior during the next interchange, or
interplay, or social occasion, and this may in turn give rise to
retaliatory or compensatory behavior on the part of others at the next
meeting. Similarly, the tendency of one set of persons to come together
and form a conversational cluster may be taken as a message and
responded to by other persons forming their own cluster or by other
persons attempting to disrupt or expand the one that is formed.
Exchanges of messages of this kind may take the form of interchanges,
the moves or messages of which are themselves complex interactive
systems. These protracted or higher-order interchanges often rely partly
upon messages which are solely expressive, but this need not be the
case. Thus, a linguistic message on the part of one participant may
carry implications for a particular recipient and not be responded to or
answered by him until several interchanges have occurred or even until
another social occasion. These protracted interchanges are often less
neatly brought to a conclusion than are the simple concrete ones we have
been discussing but they none the less provide an important area of
study. For practical reasons, however, the present study is only
concerned with the interchange of concrete messages.

% CHAPTER XIII: POLITE INTERCHANGES
\chapter[CHAPTER XIII: POLITE INTERCHANGES]{Chapter XIII: Polite Interchanges}
\label{ch:Chapter XIII: Polite Interchanges}
\chaptermark{CHAPTER XIII: POLITE INTERCHANGES}

\newthought{In Dixon, as apparently}
\marginnote{\href{https://doi.org/10.32376/3f8575cb.05fe323e}{doi}}in other Bergand communities, there are many
occasions when persons make a special effort to show respect and concern
for each other. If a person become sick, neighbors offer to help out,
and all the adult members of the community will make a point of asking
anyone who might know about the current status of the sick person. When
a person of any age has a birthday, the occasion will usually be marked
by a party held for upwards of fifteen people; the immediate family,
favorite relations and neighbors, and close friends. Invited persons all
show their regard by bringing gifts. When a couple marries, a hundred or
more persons will usually attend the wedding party held in the community
hall, and many gifts will be given. When someone dies, males who are
immediate neighbors, friends, and close relations will accompany the
body to the burial ground. There are many other ceremonies of a similar
kind. The islanders account for the ceremonial concern they show to one
another by saying by saying that nothing much happens on the island so
that persons are forced to turn to themselves as topics of conversation
and as excuses to congregate. In any case, the ceremony seem to confirm
a change in status of one or a few of the community's members, or to
reaffirm community support of a member who is injured.

Ceremonial respect is frequently expressed or conveyed by the offering
of gifts or assistance and the like. This study is not concerned with
these ceremonies as such. However, some ceremonial offerings rely upon
communication itself as a vehicle for conveying the offering. These
provide clear examples of concrete interchanges and will be considered
here.\footnote{While in general it did seem that islanders showed
  ``genuine'' respect and courtesy to one another, frequent
  demonstrations of considerateness cannot be taken, in themselves, as
  evidence of mutual regard. For example, if tradition makes islanders
  sensitive to high standards of mutual concern, and at the same time
  islanders have come to feel dislike for one another, then elaborate}

\hypertarget{road-salutations}{%
\subsection{Road Salutations}\label{road-salutations}}

There are no sidewalks in Dixon, and anyone going to or from a center of
organized social activity is usually obliged to travel part of the way
on roads. These roads are never very crowded. Therefore when\marginnote{politeness may merely represent an effort to conceal or hold back real
  hostilities that are considered to be improper. On much the same
  grounds, Bales has argued that gestures of solidarity shown during
  interaction may be evidence not of deep solidarity but of the concern
  shown by participants that breakdown of the interaction is threatened
  and that something must be done to bolster it. See his discussion of
  what he calls the ``flip-flop'' problem, \emph{Interaction Process
  Analysis}, pp.~117--118.} persons
pass each other during the day, whether on foot, bicycle, cart, motor
bicycle, or car, they cannot convincingly act as if the other has not
been seen.\footnote{One exception is allowed by the islanders. When a
  car passes a pedestrian going the same way, it is appreciated that the
  task of driving on a curving, narrow road may constitute a legitimate
  excuse for not turning from the steering wheel to salute the
  pedestrian who has been passed. On the other hand, there are so few
  cars in Dixon---approximately fourteen---that when two cars pass each
  other going in opposite directions in the night, it is assumed that
  drivers can identify each other, and honking of the horn is a required
  salutation.} The overlooking of someone cannot be rationalized as
having arisen from accident or communication barriers; overlooking can
only be taken as an expression of the attitude of the overlooker to the
overlooked.

Adult residents of the island who pass each other on the road,
regardless of the community, class, or the sex from which they come,
whether they are personally acquainted or ``know of'' each other, or
neither, or obliged to enter into interplay with one another. Minimally
this consists of a momentary meeting of the eyes in the exchange of an
nod or verbal salutation, with no other interruption of their current
ongoing activity.\footnote{In Capital City persons unacquainted with
  each other tend not to offer a salutation on passing each other on the
  sidewalk. At the same time, there are enough people on the street so
  that it is possible not to see persons that one passes close to.} If
the persons are acquainted with each other, and especially if they have
not engaged in interplay with each other for a long period, or if a
ritually significant event has occurred to one of them recently, then a
mere salutation or recognitional interchange is followed by a chat which
can last for minutes.

Minimal salutation between adult commoners on the road involves the use
of a few set interchanges consisting solely of a statement on the part
of one actor and a reply on the part of the other, both delivered with a
specific and quite standardized tone. The following are perhaps the most
frequent.

\begin{quote}
Actor: ``Ae, ae.`` \\
Other: ``Ae, ae.`` (Used only by men.)
\end{quote}

\begin{quote}
Actor: ``Foine day.'' \\
Other: ``Foine day.''
\end{quote}

\begin{quote}
Actor: ``Better day.'' \\
Other: ``Ae,'' or, ``Grand day.''
\end{quote}

\begin{quote}
Actor: other's Christian name \\
Other: actor's Christian name
\end{quote}

\begin{quote}
Actor: ``Voo ist du?'' \\
Other: ``Nae sae bad.''
\end{quote}

\noindent When one of the persons is on a cart, or bicycle, or motor bike, then
each usually waves or nods his head. When one of the persons is in a
car, then he may lift one hand off the wheel and smile,\footnote{This
  seems to be the only occasion when men regularly use smiles as
  salutations.} or only one of these, and receive a similar sign in
return.

In Dixon, salutations seem to confirm and symbolize the right of all
islanders to have certain kinds of access to all other islanders. More
important, apparently, these salutations provide an opportunity of
acknowledging allegiance to the island and to the commoners, in general,
who live on it. In these interchanges, each participant seems to
symbolize for the other not a particular person but the whole island,
and it is to the whole island, via its momentary representative, that
the salute is give. Thus, a very standardized tone is employed on these
occasions, as if to express the fact that individual differences between
one person whom an individual may salute and another are at the moment
irrelevant.

\enlargethispage{\baselineskip}

When a commoner is working in a field and is close to the road, and
another commoner passes on the road, then a recognitional interchange
occurs, or the person working interrupts his task and comes to the fence
by the road for a brief chat. The ceremonial care that commoners on the
island exert in each other's behalf is illustrated by the readiness of
persons to interrupt their work for these reasons.\footnote{In Dixon,
  readiness to interrupt work in the fields for the sake of social
  interaction can also be partly explained by the fact that crofters
  apparently feel that their repetitive tasks are tedious and that any
  ``break'' is welcome.} When the person in the field is not near the
road, but near enough to be able to determine who the walker is, then a
salutation occurs but without an interruption in work unless there is a
very special reason for having a chat. However, there is a point not
close to the road, but not too far away, where the willingness of the
worker to come up to the road for a chat becomes, in a sense, optional,
and not dictated by custom. Work at this middle distance from the road
places the worker in a sign situation, for it becomes difficult to
handle particular passers-by by means of conventionalized courtesy due
anyone. The decision of the worker to come to the fence for a talk, or
not to come to the fence, and the decision of the walker to invite this
move, or to inhibit it, becomes an expression of the particular feelings
between the two persons, an expression that is writ too large not to
become an in opportune or embarrassing source of impression.

Meetings on the roads between members of the gentry and a commoner are
characterized, minimally, by a salutation. On the part of the gentry
this consists of a nod, a smile, a comment about the weather, or mention
of the commoner's Christian name. On the part of the commoner this
consists of mention of the surname of the other, ``Doctor'' or ``Doctor
Wren'' in the case of the physician, comment about the weather, and, to
a decreasing extent, ``sir'' to the laird. Very occasionally a male
commoner will doff his hat to male gentry. These exchanges of
recognition, by their linguistic and expressive content both, signify a
relation of inequality.\footnote{In general, solutions or recognitional
  interchanges seem to be possible between persons of widely different
  statuses. For example, in many army systems, all persons of lower rank
  have the right to salute, and receive a return salute from, officers
  of the highest rank. It may be noted that at cease-fire parleys, rules
  may bar opposing representatives from saluting each other or shaking
  hands, and that soldiers under disciplinary detention may be excluded
  from the right to give and receive salutes.}

The salutations which occur when gentry meet each other almost always
form part of a longer interplay. If their meeting is planned, then an
interplay of some length is inevitable; if their meeting is accidental
then---due to the fact that there are so few of them---the accident
itself is grounds for a small celebration. First-naming is symmetrically
indulged in, apparently as a symbol of the mood of equality, intimacy,
and differentiation from gentry-commoner communication that
characterizes these situations. A mere recognitional interchange would
be a sign that the participants were on very bad terms.

Until the age of approximately fourteen, children of commoners are, in
certain senses, not obliged to conduct themselves in a socially
responsible manner. In a sense they are neither sacred nor profane, but
rather ritually neutral; in some ways they cannot give serious offends
nor ought they to take it. One way in which this capacity of being a
``non-person'' is illustrated is by their meeting behavior. When they
pass an adult on the road, they need not be given recognition by him and
they seldom give recognition. Their eyes tend to meet the eyes of others
less than is the case with adults. When this does occur they often
become ``shy'' or a little embarrassed. Similarly, when recognition is
given to a child in a home, this recognition often takes the form of
play and often is not returned. Thus, too, an adult may sit next a child
at a social or at dinner and never break the activity with a moment of
recognition, which almost always occurs when adults are thus situated.

Seamen who put in at the Dixon pier are divided by residents into
British and Foreign, usually on the basis of appearance. All seamen have
the right to shop at the local stores, to attend the dances and dance
with local girls, to attend the bi-monthly movies, and to use the local
post office and trunk line. They also have the right to receive free
medical attention. (All of these rights are also enjoyed by tourists.)
Further, foreign and British seamen who have used the pier for years
have friends in the community with whom they may upon occasion spend an
evening or who visit with them for a while on their boats. The seamen
recognize the obligation of receiving the local customs officer and
allowing inspection.

It is the opinion of some commoners that foreign seamen are of the
lowest type, uncouth and uncivilized. In any case, foreign seamen are
rarely given salutations by residents whom they may pass on the road or
stand next to in the shops. Frequently residents will look at these
seamen, as they pass them or stand near them, but not recognize them as
persons with whom interplay is to be initiated.

British seamen and tourists share a mixed status with respect to
salutation rights. Sometimes they will be treated as foreign seamen are,
as if not there in the capacity of persons but there merely in the
capacity of objects to be looked at. Knowing the language, however, they
sometimes initiate a salutation to residents whom they may pass.
Residents then usually reply, although with a gesture in which they
patently put very little feeling. Sometimes, however, residents will
proffer a brief nod to these strangers, or even a truncated reference to
the weather. A similar gesture will then be returned to them. Perhaps
such interchanges sometimes assure the outsider that within limits he is
safe. In any case, the nod seems to convey equality coupled with extreme
distance.

There are on the island at least three persons whose faces are deformed
in such a way as to affect speech patterns. They are by ``aesthetic''
standards ``ugly,'' and ugly to such a degree that looking at them
throws attention off. They tend to remove themselves from contexts where
recognitional interchanges would ordinarily occur and to restrict their
communication to situations which are clearly defined in technical as
opposed to social terms. These persons have developed a pattern of
withdrawal and even in the circle of their own family play an atypical
ritual role.\footnote{The communication problem presented by these
  persons is considered in chapter xx.}

\hypertarget{social-occasion-salutations}{%
\subsection{Social Occasion
Salutations}\label{social-occasion-salutations}}

During community-wide social occasions, when up to two hundred persons
may be gathered together in the community hall, it is not expected that
each person present will enter into a salutation army interchange with
every other person present. There are sufficient intervening barriers to
supply excuses for neglect. During intermissions, in moving from the
hall to the hallways or smaller rooms, it is necessary of persons to
pass each other in close quarters. On these occasions a minimum
interchange of some kind is required. Similarly, in sitting down on a
bench, whether during teatime or during a dance, persons on either side
are usually acknowledged in some way. (This general pattern of showing
responsibility to those closest to one also obtains in shops and
outdoors on occasions when crowds collect for an auction or the like.)
Salutations in these circumstances may contain the same words as are
found in salutations between persons passing on the road, but the
intonation appears to be different, apparently giving less weight and
seriousness to the hall salutation. Typically, interchanges involving
two messages will occur. For example:

\begin{quote}
Actor: ``Good crowd.'' \\
Other: ``Aye, fine crowd.''
\end{quote}

\begin{quote}
Actor (during whist): ``Good score?'' \\
Other: ``Aye.''
\end{quote}

\begin{quote}
Actor: head nod. \\
Other: nod returned. (Between male adults.)
\end{quote}

\begin{quote}
Actor: smile. \\
Other: smile returned. (Between women.)
\end{quote}

\begin{quote}
Actor: ``So.'' \\
Other: ``so so.''
\end{quote}

\begin{quote}
Actor touches arm of other. \\
Other: Christian name of actor. (A woman to another woman or to a
child.)
\end{quote}

\begin{quote}
Actor: ``Well, well.'' \\
Other: ``Well, well.''
\end{quote}

\begin{quote}
Actor: other's Christian name. \\
Other: actor's Christian name.
\end{quote}

\noindent Since a person can reasonably take the stand that it is feasible to
salute only one person at a time, the possibility arises in the hall
that while an individual is engaged in saluting one person he will pass
immediately in front of another person and not be able to salute him in
doing so. This kind of overlooking is not justified on the road.

The smiles and nods that persons in Dixon used as a brief recognitional
interchange also occurred in houses where interplay might lapse during
domestic activity. Thus, the female head of the household tended to
involve those present who were not out of the immediate family in
occasional smile interchanges. This occurred especially when eyes
happened accidentally to meet. Friends were given assurance in this way,
throughout the period of their stay in the house, of the welcome and
approval given them. In the hall, often no one was quite in a position
to offer anyone else assurance as to the propriety of his presence and
so the kind of smile interchange characteristic of household activity
was not possible.

When a group of five or six persons worked on a particular piece of
land, pulling weeds, clearing stones, planting potatoes, spreading
manure, raking hay, or any of the other croft tasks, the work would be
interrupted every fifteen or twenty minutes when two workers happened to
find themselves close to each other. The interruption would consist of a
brief interchange in which the workers would affirm to each other that
the work was getting done or make a comment about the weather. These
pauses, and the longer ones for ten o'clock tea, one o'clock lunch, four
o'clock tea, and supper, seemed to express the fact that the workers
were not merely animals engaged in routine labor all day long but were
persons, capable and desirous of conducting social interaction with
other persons.

\hypertarget{minor-propitiatory-interchanges}{%
\subsection{Minor Propitiatory
Interchanges}\label{minor-propitiatory-interchanges}}

Salutations provide examples of very brief interchanges, many of them
reduced to the bare minimum of two short messages. For another set of
illustrations we may turn to occasions when persons feel they must
exercise ``etiquette,'' or ``manners,'' or ``courtesy.''

The role of etiquette is clearly seen when persons impinge upon each
other in some accidental and incidental way. At such times, they
frequently make use of stereotyped social formulae as a means of
handling the situation and ensuring that no offense is given. These
formulae cover requests for small favors, apologies for accidents,
misdemeanors, and the like.

Forms of etiquette and courtesy represent, in a sense, model
interchanges. These forms of communication plainly recognize that
persons are objects of value that must be treated with ritual care in an
environment that is full of potentially offensive signs. In the case of
many of these interchanges, the number of messages and the approximate
content for each message have been formally laid down in books on
etiquette.\footnote{Thank-you notes, and ``bread and butter'' letters
  (``roofers'') provide nice examples as to specification, paragraph by
  paragraph, of courtesy letters. The round of correspondence called
  forth by a letter of introduction provides an example of specification
  as to number of messages.}

When one person in Dixon passes in front of another in such a way that
this can be interpreted as an act of precedence, or when one person
touches another in a way that may be interpreted as an aggression, then
the actor in many cases offers a pardon to the other person. A pardon
begged is a corrective strategy, a way of exorcising a possible slight
already committed and neutralizing a sign situation. Interchanges
involving pardons and apologies frequently have a very simple structure:

\begin{enumerate}
\item
  Actor performs potentially offending act.
\item
  Actor says to other: ``Sorry,'' or ``Oh, oh,'' or ``Pardon,'' or
  ``Excuse me.''
\item
  Other terminates the interchange by saying, ``O.k.,'' or ``That's all
  right,'' or merely by smiling.
\end{enumerate}

When one person in Dixon wants to ask another for a minor assistance of
some kind---an assistance which might be interpreted as an act of
servant-like subordination on the part of the individual offering the
assistance---then an interchange of messages is frequently required in
order to ensure that offense will not be taken. These interchanges are
not corrective, since the potentially offending act has not yet
occurred; they are, rather, preventive. Thus, if one person wishes to
have something passed to him by another, or wishes to have the other
change position a little, etc., the following interchange frequently
occurs:

\begin{enumerate}
\item
  Actor formally makes a request in a supplicating tone of voice, e.g.,
  ``Will ya please pass the . . .,'' or ``Do ya mind moving a little so
  I can . . . ``
\item
  Other agrees to perform act and at the same time states that he has
  not been offended by the request, e.g., ``Surely,'' or ``With
  pleasure,'' or ``Uh hm.''
\item
  Other then performs the service.
\item
  Actor offers some kind of thanks to the other.
\item
  Other terminates the interchange by a brief not or by saying ``Aye,''
  etc.
\end{enumerate}

If the favor is considerable or is of the kind that is very likely to be
taken as an expression of relative ``status,'' then the interchange may
include an extra round of statement and reply, in which the actor, after
the other agrees to perform the favor, asks again if the other is quite
sure that he will not mind performing the service. To this the other
usually gives a second assurance that no offense is being taken, and the
interchange terminates in the usual way.

\hypertarget{terminative-echoes}{%
\subsection{Terminative Echoes}\label{terminative-echoes}}

As suggested in the last chapter, during informal conversational
interplay it is possible for one participant to introduce a message and
then for the next speaker to give very little attention to what has been
said but rather to make use of his opportunity as sender to contribute a
message involving mention of his own experience. The second speaker will
allow the first speaker to finish his message, and will take a cue as to
what range of things ought to be talked about from the first message,
but he will go on to put his own oar in, as it were, not bother, really,
to answer the first statement. The first message seems to establish a
license as to what sort of self-reference can be made, and the following
speakers in the interplay exercise the license in their own behalf.
Interchanges of this kind---if it is proper to refer to them as
interchanges---will have as many messages in them as the participants
have personal experiences that can be mobilized descriptively for the
occasion.

Another favorite conversational interchange is one in which the first
message makes a claim which the other participants cannot quite let
pass, and some qualifying messages are directed to the first one until
matters have been sufficiently put aright to allow the interchange to
end.

In Dixon, apparently more so than in more argumentative and less polite
subcultures, many claims made by a participant in the interplay were not
duplicated by other participants or disputed by them but rather politely
honored in a rapid if half-hearted way. Whether the sender implicitly
asks his recipients to respond with shock, surprise, laughter,
agreement, or approval, and whether or not recipients were genuinely in
sympathetic tune with the speaker's implied request, they tended to
comply.

Every member of the community seemed versed in the use of an extensive
set of brief phrases, by which an expectation introduced by a person's
message could be fulfilled, the interchange quickly terminated, and the
respondent freed from further need to act out what was perhaps not
felt.\footnote{In American subcultures there is a corresponding list of
  accommodative echoes, for example, ``Tsk-tsk,'' ``You don't say,''
  ``My gosh,'' ``What d' ya know,'' ``Really,'' etc., each of which
  terminates a two-message interchange. Without command of these pat
  replies, it is difficult to be at ease in the conversation of a
  particular group.} So smooth was this technique that on many occasions
speakers did not become aware that their face had just been saved.
Especially useful were proverbs, which committed the respondent to
nothing and could usually be used to terminate an interchange regardless
of the message with which it had been started.

If the first message in an interchange implied that the recipients ought
to be shocked at why they had just heard, phrases such as, ``My
feeder,'' ``Such o ting,'' would be chorused in response. If a point of
view had been voiced that recipients could not agree with, they would
guardedly respond with, for example, ``There's something in what you
say,'' or ``I donna kin.'' A few of the island's favorite terminative
echoes are given below, following instances of interchanges in which
they were employed.

\begin{quote}
A commoner of wide repute, known as one given to bragging: ``I've been
in every house and know everyone on the island.'' His host: ``Dat'll be
right, dus du kin.''
\end{quote}

\begin{quote}
Commoner, proud of his Ford car: ``I've been driving it for seven years,
mind you, and never a part I've had to buy.'' Passenger: ``I hear you,
boy.''
\end{quote}

\begin{quote}
Hotel maid, speaking about scullery boy having stayed out till six
o'clock. ``It's no right now, is it.'' Cook: ``Past spaekin about.''
\end{quote}

% CHAPTER XIV: THE ORGANIZATION OF ATTENTION
\chapter[CHAPTER XIV: THE ORGANIZATION OF ATTENTION]{Chapter XIV: The Organization of Attention}
\label{ch:Chapter XIV: The Organization of Attention}
\chaptermark{CHAPTER XIV: THE ORGANIZATION OF ATTENTION}

\newthought{We have considered the}
\marginnote{\href{https://doi.org/10.32376/3f8575cb.03040491}{doi}}interplay and some of the units of interaction
which may occur within it once accredited participation has been
established. Let us now return to consider some of the structural
characteristics of interplay as such.

It has been suggested that an interplay characteristically involves a
focussing of the attention of the listeners upon the speaker. The
initiation and maintenance of this organized attention, the transfer of
it from one speaker to another, and its final dissolution all involve
problems in attention management.

Persons who wish to be accorded attention as senders---to be given the
floor, as it were---frequently precede or initially accompany their
messages with signs conveying a specific request for attention. These
signs consist of speech infections, interjections such as ``oh'' or
``hello,'' calling out of a recipient's name or ``catching'' his eye. It
is possible to distinguish between those signs which request accredited
participants to focus their attention on a particular participant and
those which request individuals to enter into interplay and become
accredited recipients. These signs (whether initiating a message or an
interplay) frequently impress the potential recipient in such a way as
to prepare him for the length of the message that is to come, for its
urgency, and to some extent for its character. Very frequently, an
immediate reply is given to these signs by the recipient, assuring the
sender that his message will be received and that, in a sense, it will
not be taken as an offense for him to proceed with it.\footnote{It may
  be worth noting that persons in highly structured subordinate
  positions, for example butlers, may be required to call the other into
  interplay by means of signs which preserve the illusion that the
  subordinate is not initiating interplay.} Sometimes the reply may
contain an explicit request to hold off for a moment so that the
individual can prepare himself for the interaction he is being called
into. In addition to this information, the reply may also provide the
original sender with some idea as to how willing the other person really
is to become involved in interaction. Typical replies in answer to a
request for attention consist of interjections such as ``yes'' and ``uh
hm,'' a pause in he recipient's ongoing behavior, and orientation to the
recipient's eyes in the direction of the sender. In other words, before
a potential sender launches into his message he may signify a request
for clearance and wait for clearance to be given him before proceeding.

Where interplay is limited to a few persons who are not at the moment
related to each other in such a way as to make offense or
misunderstanding likely, a potential sender may not wait for an actual
reply to his request for clearance but merely mark the point where it
should have occurred by introducing a slight pause and change of tone
between his initial request for clearance and his message. In many
formally-organized interplays, however, clearance becomes a codified
practice. At a formal meeting, for example, a participant who wishes to
become the accredited sender often must first stand up or get
authorization from the chairman.\footnote{Clearance signs can also be
  important in mediated communication. Many organizations have a policy
  of answering each letter with two letters, one to say that the letter
  has been received and read, and a later one to give the answer
  originally requested. So, too, the receipt in the case of financial
  transactions serves as lasting proof for the sender that his financial
  communication has been received.}

It will be apparent that the ability to refuse, overlook, or postpone a
request for clearance gives the potential recipient an important way of
exerting control over participation obligations that important senders
place upon him. Clear cases of this can be found in situations where
there is some doubt as to which of several requests to initiate an
interplay a recipient will honor. For example, in order to obtain
service, a patron may attempt to catch the eye of a waitress or store
clerk, thus initiating an interplay in which requests and orders can be
presented.\footnote{So commonly do we employ eye-to-eye looks as a means
  of initiating an interplay or addressing a message that when we
  suddenly find ourselves in this relation to someone with whom we are
  not communicating at the time, we frequently feel flustered and look
  away or enter into momentary interplay to regularize the situation.
  Those who look into another's eyes without acknowledging this in
  either of the above ways are sometimes thought to be ``cold'' or
  ``hard.''} Service personnel may wish to avoid ill-timed involvement
of this kind and can do this by averting their own eyes.\footnote{In
  many societies, averting of eyes is apparently an institutionalized
  way of conveying a modest and tactful self-restraint from entering
  into the intimacy of an interplay.} Similar cases occur in situations
where a potential recipient can choose from among several accredited
participants the one to be given clearance. The power of choice, in this
case, may be an official right, as in the case of a chairman at a
meeting,\footnote{H. S. Elliott provides an example of this in his
  consideration of the problems of management that a chairman of a
  meeting faces, in \emph{The Process of Group Thinking} (New York:
  Association Press, 1929), pp.~73--74: ``The chairman would, on the one
  hand, get every person to take part and see to it that all points are
  represented and, on the other hand, restrain the inveterate talker and
  keep him from monopolizing the discussion. Just to look encouragingly
  toward those who are not taking part and not to look toward the ones
  who want to participate in essentially is a help.} or an unofficial
right, as in the case of an ``informal leader'' during
interplay.\textsuperscript{6}

Once clearance has been accorded to a potential sender, and he begins to
send his message, both sender and recipient may continue to convey their
involvement in communication by means of what we might call attention
signs. These constitute a minor but significant communication courtesy.
Attention signs are frequently conveyed by a medium other than the one
employed for the message, thus ensuring that jamming does not occur
between the two streams of signs. Direction of the eyes in the case of
both sender and recipient is a typical attention sign during spoken
communication.\textsuperscript{7}

It does not seem to be usual for a sender to lose interest in his
communication role. Therefore attention signs which pass from sender to
recipient do not play a vital role in the organization of communication,
except, of course, as a means of distinguishing addressed recipients
from unaddressed recipients. Attention signs from recipients to a
recognized sender seem to play a more important role. They act as an
``informative feedback,'' telling the sender of the effect of his
message in time for him to modify his behavior in a direction calculated\marginnote{Frequently the
  expression of the face indicates that a person is on the point of
  taking part and just recognizing this desire will bring timid ones
  into the discussion. Sometimes the chairman may call upon certain ones
  by name. If a person persists in monopolizing the discussion he may
  find it necessary to restrain the talkative member. He can do this
  easily by tactfully saying, ``Wait a minute, Mr.~- - -, we want to
  hear what Mr.~- - - thinks about this question!''}
to\marginnote{\textsuperscript{6} W. F. White, ``Small Groups, Large Organizations,''
  in Rohrer and Sherif, eds., \emph{Social Psychology at the Crossroads}
  (New York: Harper, 1951), pp.~297--312, provides on p. 300 an
  illustration of how an informal leader can serve as a sort of
  sanctioned for those who should take over the focus of attention.
  ``Several men are standing in the club room in groups of two, two, and
  three. Individual X comes in and the three little groups immediately
  reform into one larger group, with the seven men remaining silent
  while X talks, and each man seeking to get the attention of X before
  he himself speaks.''} obtain\marginnote{\textsuperscript{7}\setcounter{footnote}{7} This is apparently not a universal practice. An
  early report on the Northwest Coast Amazons claims that: ``When an
  Indian talks he sits down, no conversation is ever carried on when the
  speakers are standing unless it be a serious difference of opinion is
  under discussion; nor when he speaks does the Indian look at the
  person addressed, any more than the latter watches the speaker. Both
  look at some outside objects. This is the attitude also of the Indian
  when addressing more than one listener, so that he appears to be
  talking to someone not visibly present.'' See Thomas Whiffen,
  \emph{The North-West Amazons} (London: Constable, 1915), p.~254.}
 a desired response from recipients. These signs also warn a
sender if there is a danger that the focus of attention is about to
break up or pass on to a new sender.

It has been suggested that clearance signs and attention signs play an
important role in the organization of interplay. These signs, especially
attention signs, provide the sender with a continuous indication of the
stability of the communication structure. They constitute what Ruesch
and Bateson refer to as communication about communication, or
``metacommunication.''\footnote{Ruesch and Bateson, \emph{op. cit.},
  pp.~23--24.}

It is customary for a sender to close his message with a gesture or
speech inflection conveying the fact that the message has ended and that
the sender is now ready to relinquish his role and himself become a
recipient. These termination-of-message signs may sometimes be used by a
recipient as evidence that clearance has been given to him. The most
clearcut sign of this kind, perhaps, is the intonation and word-order we
employ when we ask a question.\footnote{As Bales points out, in ``The
  Equilibrium Problem in Small Groups,'' \emph{op. cit.}, asking a
  question is an effective way of signalling a desire to
  relinquish the role of sender with the expectation that the addressed
  recipient will then take up the role of sender. Of course the
  effectiveness of this sign depends upon the addressed recipient
  accepting the obligation and responsibility of the role that is being
  proferred him. During informal conversation, recipients usually accept
  this obligation so automatically}

It is to be noted that clearance signs which signify the termination of
a message may be distinguished in practice from those clearance signs
that are employed to terminate an interplay,\textsuperscript{10} and these, in turn, from signs which signify the
termination of a social occasion. For example, a resolution may be
required to terminate a committee meeting, and a special song, or a
special lighting and staging effect, may be employed to terminate a
social occasion. It may be noted that in the Dixon primary schools,
where some of the pupils have not yet learned to interpret or respect
cues which signify the termination of an interplay, the teacher, after
calling a pupil up to her desk to check over an exercise, sometimes had
to propel the pupil back to his or her seat (usually in an affectionate
or joking way) in order to bring the hearing to a close.

\vspace{.2in}
\begin{centering}

\Large{* * * * *}

\end{centering}
\vspace{.17in}

\noindent Let us use the term ``sending position'' to refer to the spatial or
ecological point at which any participant in an interplay is or could be
located, relative to the other participants. Senders find it useful to
be at a sending position at which it is possible and convenient to
receive attention cues from all the recipients. The degree to which any
particular location permits this kind of reception is a measure of its
favorability as a sending position. Whenever there are more than two
participants in an interplay, one participant is likely to be in a
position that is more favorable than the position held by any of the
other participants. During informal interplay in Dixon there was a
certain amount of surreptitious (and even unconscious) jockeying for
favorable sending positions, while, on the other hand, participants
frequently\marginnote{that they seldom realize that an
  obligation has been fulfilled---although, of course, they may give a
  false answer, or an insufficient answer, or an unserious answer.
  However, at formal meetings, a guest speaker may request questions
  from the floor at the end of his speech, and he and the chairman may
  expect that a period of questions and answers will follow, and yet no
  one in the audience may take up the role of sender which the guest
  speaker has, by his request for questions, attempted to relinquish. At
  such times we can see more clearly that our ordinary willingness to
  make at least some answer to a question is the fulfillment of an
  obligation. It is interesting to note that high-placed political
  figures who are asked questions by newsmen may find themselves in the
  dilemma of giving no answer and thus failing to fulfill the obligation
  of communicators, or giving an ambiguous answer those subtle
  expressive overtones may be examined for implications that are
  embarrassing. The dilemma is sometimes resolved by the curt phrase,
  ``No comment,'' by which the individual can acknowledge the fact that
  a question has been put to him, that he has correctly received it, and
  that he accepts, in general, the role and obligations of communicator} arranged\marginnote{\textsuperscript{10}\setcounter{footnote}{10} In mediated
  communication, clear differences can sometimes be found between signs
  which terminate a message and signs which terminate an interplay. In
  one-way wireless communication, for example, a word such as ``over''
  may be given as a sign that a message has ended and that the other
  participant has clearance, and a word such as ``out'' may be given as
  a sign that the interplay has been terminated. In the case of wireless
  communication, the initiation of an interplay may call for signs that
  are similarly explicit and specific, e.g., ``calling . . ,'' ``come
  in, . . .''} to sit or stand so as not to block any other
participant's line of vision too much.

It is apparent that interplays will vary according to the disposition of
favorable and unfavorable sending positions established in it. At one
extreme we have cases where only one or two points provide good sending
positions, so that a participant who wants to become a sender must first
move into position to do so. This is the case in platform-audience
communication, where a potential sender must first come to the front of
the audience and preferably stand on a raised platform. A characteristic
of this kind of communication arrangement is that one physically closed
interplay can be maintained even though an extremely large number of
persons, relative to the size of the region, are packed into the region.

At the other extreme we have the ecological arrangement by which all
participants are in a favorable and more or less equally favorable
sending position. The typical case is where three or four persons have
come together and face each other in a circle for the purpose of
informal conversation. A sending circular of this kind provides the one
important exception to the rule that one participant in an interplay
usually has the best sending position.\footnote{This point has, of
  course, been recognized in the literature. For example, Elliott,
  \emph{op. cit}., p.~64, in talking about a discussion club called a
  Bible Circle, says: ``This is a good name because some circular
  arrangement gives the best results in discussions. The important thing
  is that just as far as possible the members have a chance to look into
  the face of the other members.'' Circle organization, of course,
  solves a ceremonial problem; no one need be given the head or the foot
  of the table and the invidious evaluation which such positions may be
  taken to express.}

The case of sending circles, where all participants are in a favorable
sending position, provides some interesting complications. If all the
persons in a bounded region are to be involved in the same sending
circle---that is, if there is to be only one interplay, and all
participants are to have an equally favorable sending position in
it---then there is a relatively low limit to the number of persons who
can be contained or enclosed in the region. If the bounded region is to
be filled with many effectively closed sending circles of three or four
persons each, with no participant from one circle penetrating the area
enclosed by the participants in another circle, then a relatively large
number of persons can be enclosed by the region. There are geometrical
as well as empirical grounds for this statement. Two illustrations may
be given:

In Dixon the community hall dance floor is about twenty feet wide and
thirty feed long. In dances such as the ``old-fashioned waltz,'' the
sending circles consist of couples, and more than thirty couples can
easily be enclosed in the hall. When square dances such as ``Lancers''
or ``Quadrilles'' are danced, however, the hall has a capacity of only
three ``sets,'' each of these sets constituting a sending circle of
eight participants.

In Dixon it is customary to hold large birthday parties for persons of
all ages. On these occasions it is not uncommon for a family to fill
their small cottage with twenty guests. A variation of the game of
``spin the bottle'' is popular at these times. The game requires one
closed circle, however, and seems in fact to be a ceremonial exercise in
this kind of communication arrangement. Because of the size of the
rooms, occasions arise when all the guests at a party cannot be fitted
into one circle, although they can easily be fitted into a number of
smaller circles.

It is to be noted, finally, that platform-audience organization enables
more persons to be incorporated in a region of given size for purposes
of communication than can be incorporated in this region by sending
circle organization, regardless of how small the circles are and how
closely they are packed.


% CHAPTER XV: SAFE SUPPLIES
\chapter[CHAPTER XV: SAFE SUPPLIES]{Chapter XV: Safe Supplies}
\label{ch:Chapter XV: Safe Supplies}
\chaptermark{CHAPTER XV: SAFE SUPPLIES}

\newthought{When an individual enters}
\marginnote{\href{https://doi.org/10.32376/3f8575cb.e0092c28}{doi}}the perceptual range of others, a kind of
responsibility is placed upon him. Normally he must assume that his
behavior will be observed and that it will be interpreted as an
expression of the attitude he has toward those who observe him. In the
realm of undirected communication, this implies that he will be expected
to behave in a decorous manner, giving appropriate consideration to the
presence of others. The requirements of decorous behavior, in our
society and in others, will not be considered here. In the realm of
directed communication---for example, conversation---the individual must
assume that both his messages and his behavior as a recipient will be
expected to contribute to the maintenance of the working acceptance.

\enlargethispage{\baselineskip}

Once individuals have extended accredited participant status to one
another and have plunged into conversation, then it is necessary to
sustain a continuous flow of messages until an inoffensive occasion
presents itself for terminating the interplay. It appears that some
persons can be so distantly related to one another that very little
pretext may be needed to break off conversation and relapse into
silence, and that some persons can be so intimately related to one
another that on many occasions they can assume that no offense will be
given when conversation lapses. It also seems that a wide range of
social distance and of situations exists between these two extremes
where a fairly good excuse is needed before conversation can safely
lapse.

In those situations where lapse of communication is of itself
inappropriate communication, participants must make sure that someone
among them is conveying a message and that it is an acceptable or
appropriate message. Since the stream of messages must be constantly
fed, participants sometimes tend to use up all the appropriate messages
that are available to them. The problem then arises: what can be used as
a safe supply, that is, what can be used as a reliable source of
acceptable messages? At certain times, especially during lengthy
informal interplay, this problem introduces a need for a high order of
ritual management.

1. A famous kind of safe supply is found in what is often called ``small
talk,'' that is, issues that can appropriately be raised between persons
of widely different status without this fact prejudicing the social
distance between them, and to which almost everyone can be expected to
have the same attitude.\footnote{Malinowski uses the phrase ``phatic
  communion'' to refer to the exchange of gossip and small talk; see
  Supplement One to Ogden and Richards, \emph{op. cit.}, especially
  pp.~314--315. Strangers who are close to each other physically but not
  engaged in communication may often fail automatically into momentary
  accredited interaction if an unexpected event occurs that both
  patently observe and that provide momentary guarantee that their
  attitudes to the event will be similar, while at the same time
  providing some grounds for feeling that the basis of communication
  will not lead to further entangling involvements but will be easily
  terminated.} In our society, animals, children,\footnote{Animals and
  children that can be gotten to behave for a moment in a human-like
  fashion are especially useful as a safe supply.} accidents, and the
weather usually form the object of small talk. In Dixon, the catch---or
lack thereof---which the two local fishing boats made that day was
frequently a subject for comment. During the spring, lambs and foals
were also safe topics, since it was assumed that no one could be
oblivious to their charm. If anyone on the island had had an accident,
or taken sick, or died, or gotten married, these facts were constantly
employed by others in small talk. A sickness lasting a few weeks was
especially useful, for persons could ask one another several times a day
how the unfortunate one was progressing and comment sympathetically. The
weather was very frequently mentioned in Dixon and among those actually
engaged in crofting was often mentioned in relation to its effects upon
the crops.\footnote{This corresponds to what is sometimes called ``shop
  talk.''} Comments about the weather are often thought to be rather
empty things. On the island this seemed not to be the case. To farmers,
of course, weather is an important contingency, but more than this
seemed to be involved. If the weather was bad, as it usually was,
comments always played this down and conveyed the fact that the
individual was not being beaten by it. The worst days would call forth
such comments as:

\begin{quote}
``No such a good day.'' ``Aye, it's terrible weather.'' ``No very good
for the taties.'' ``No, it's not that.''
\end{quote}

Every time interchanges occurred, the participants seemed to reaffirm
their loyalty to conditions on the island and to the persons who were
staying on it.

Another widely employed source of small talk in Dixon was provided by
recent purchases of material artifacts. Everyone on the island, whether
gentry or crofter, was obliged to face many of the same conditions of
domestic discomfort and to attempt to meet them by means of the objects
available at the local shops or by mail-order. Both men and women took
an interest in these matters, and if conversation lagged, participants
could always fall back on a discussion of the merits of the latest
household tool, or gadget, or comfort that had been purchased.

Two facts of interest may be cited concerning small talk. First, some
groups seem to place special attention on skills regarding small talk
and to feel that an important symbol of membership is the capacity to
sustain a conversation of small talk whenever necessary. Members of such
groups may even undergo conscious training in this kind of behavior.
Secondly, it seems to be in the character of small talk that it is
quickly exhausted; small talk allows for comments, not discussions.
Hence when persons are to be engaged in conversation for a considerable
length of time, other safe supplies must be employed.

2. During informal interplay, participants frequently resort to a topic
of conversation that is sometimes called \emph{gossip.} This involves
reference to persons who are not present (and, sometimes, to temporarily
inactive aspects of present persons) and to past conduct on their part
which can be taken as illustrative of approved or disapproved
attributes.\footnote{Gossip is usually analyzed as an informal means of
  social control exerted by the sanction of adverse or favorable public
  opinion. This gives to gossip a social function with respect to
  community standards. This wider function of gossip is irrelevant here.
  We are concerned with gossip's social function in terms of maintenance
  of interplay.} The conduct gossiped about must be sufficiently
clearcut and spectacular to ensure that all listeners will place the
same interpretation on it. In order to maintain a working acceptance,
topics upon which persons may place opposing values must be avoided.

On the island two forms of gossip seemed popular. In one case, a speaker
aired his feelings, which had been hurt or injured by what he considered
to have been an improper action on the part of the absent person who was
the object of the gossip. Recipients were asked in this way to confirm
for the speaker the fact that he had been unjustly injured and, perhaps,
to thereby confirm the principles of justice that the injury had put
into question. In the other case, the gossiper did not refer to acts
which had offended him in particular but to conduct on the part of the
object of gossip which the speaker approved or disapproved even though
he had not directly gained or suffered by it. In these cases, the
speaker took a kind of editorial attitude---the community's point of
view---toward the conduct about which he was gossiping. It is
interesting to note that the islanders had a high awareness of community
standards and so, in commenting upon a noteworthy action of an absent
person, a speaker could merely provide a flat objective statement of the
act, with a marked lack of emphasis either linguistically or
expressively, and be correct in his assumption that this would be enough
to call forth from his recipients the expected response. The most
extreme infractions of the community's standards, as, for instance, when
an open fight occurred at a community social, would be gossiped about in
a stilled atmosphere, the speaker providing only a toneless, brief
statement of the occurrence. Outsiders, of course, would misread these
conversations, feeling that an act of no importance was being considered
or that the islanders were extraordinarily fair in their references to
social delicts.

As a safe supply, gossip is limited by the fact that the self accorded
to each participant is usually defined partly in terms of minimal
loyalties to particular persons not present. Breach of these loyalties
by gossip conveyed or tolerated may disrupt the tenor of the
interaction. An islander who is married engages in very little serious
gossip about his spouse, nor do children of whatever age gossip about
their parents. Such acts of disloyalty would be a source of
embarrassment to those who observed them. Similarly, a commoner exerts
certain controls on the amount of gossip he will indulge in about absent
commoners in conversation with the gentry and outsiders. On the whole,
only commoners who are generally disrespected and regarded as more or
less beyond the pale are gossiped about in such a context.

On the island, a very happy supply of gossip is found in what are
sometimes called ``post mortems.'' After a social, members of a
household would discuss over breakfast and lunch the previous evening's
events, assured that all participants in the conversation had had the
same experience and would be able to participate actively. Reference
would be made to what persons wore, to how they behaved, to the fact
that the local baritone could sing better but was trying out a new song,
to the fact that the boys from Northend didn't know all the words to the
song they had sung, to the fact that a local woman had gray hair showing
at the roots and that if you were going to use dye you should look after
it well, etc.

3. Another safe supply employed on the island consisted of statements
made by the speaker concerning the state of his health. This was
especially employed by older people and by women. There was an
understanding that self-references of this kind did not constitute
bragging or a request for too much attention. Recipients could be
expected to be ready with an indulgent reply. It seemed that the more
``serous'' the disability suffered by a person, the wider the range of
persons with whom he could employ his disability as a safe topic of
conversation.

4. An important variety of safe supply relies on the use of an unserious
definition of the situation. An inoffensive choice of message during
interplay may have to fulfill so many requirements that it may be
advisable for the sender to abstain from serious communication and
instead convey a message in an obvious spirit of levity. Messages
conveyed in an unserious tone may be inoffensive and yet contain
statements that would ordinarily be offensive.\footnote{A message
  conveyed in an unserious manner cannot be taken directly as a
  reflection of the valuations of the sender. Indirect judgments must be
  made on the basis of an understanding as to the kinds of persons who
  would make a point about making a joke about a particular given
  matter.} The point here is that there are many occasions when it is
easier to find a message that would be offensive if conveyed seriously
than it is to find a message that is inoffensive when conveyed
seriously.\footnote{Unserious messages may themselves be offensive if
  they refer to matters too sacred to joke about or to matters which
  ought to have been considered acceptable enough for ordinary, serious
  communication.} Levity is useful, furthermore, because it permits and
even enjoins the use of unlimited exaggeration. This kind of
clarification increases the likelihood that persons of widely different
statuses will be sensitive to the message and take the same attitude
(although in jest) to it.

Levity, as a safe supply, usually entails a kind of unserious ritual
profanation of the sender or of the persons to whom he addresses his
message. It is sometimes referred to as kidding, razzing, raillery,
joking, banter, joshing, or leg-pulling. It seems to be especially
important where persons who have always been in one specific
relationship to each other find themselves in an interplay in which
another kind of relationship prevails.\footnote{In social
  anthropological literature, the term ``joking relationship'' has come
  to signify a special privilege of familiarity and disrespect between
  two persons. The relationship serves to prevent the expression of
  hostility, even though important grounds for hostility exist. Harmony
  must be maintained because the persons are not in a position to
  express their feelings by means of conflict or avoidance. They are not
  in a position to do so because each is intimately and dependently
  related to the same third person, or to third persons who are
  themselves intimately related to one another. (For a statement and
  bibliography see A. R. Radcliffe-Brown, ``A Further Note on Joking
  Relationships,'' \emph{Africa}, XIX, 133--140.) The analysis of joking
  employed in this study follows the social anthropological one but with
  a shift in emphasis from the need to maintain a relationship to the
  need to maintain working acceptance during interplay. The position is
  taken here that the familiarity and disrespect found in joking
  relationships so obviously do not apply to the actors that these forms
  of treatment can be used as a signal for proclaiming a state of
  unseriousness. Serious communication would eventually lead to open
  hostility, and so joking is seriously necessary in order to keep
  peace.}

On the island, joking as a safe supply was especially employed between
crofters and non-crofters. Thus the doctor would complain that everyone
insisted upon joking with him when he attended socials and that no other
kind of behavior on his part seemed wanted by others. Joking seemed to
be especially prevalent and especially easy between older women of the
commoner class and young males of some outside status, possibly because
a member of one of those groups was in very little competition with a
member of the other group, and they could hence afford to be on
sufficiently easy terms with one another to allow for joking.\footnote{For
  a study of the role of non-competition in the formation of convivial
  interplay, see Edward Gross, ``Informal Relations and the Social
  Organization of Work in an Industrial Office,'' (Unpublished Ph. D.
  dissertation, Department of Sociology, University of Chicago, 1949).}

5. A safe supply is found in courtesies, especially those involving
small offerings and assistances. Thus, whenever it is possible for one
person to be defined as host or hostess, it is possible for that person
to devote many messages to solicitous enquires {[}\emph{sic}{]} after the comfort of the
guests and to offerings of food and the like. As has been suggested,
codified manners provide an island of safety to swim to when in doubt or
when you want to retreat.

\vspace{.2in}
\begin{centering}

\Large{* * * * *}

\end{centering}
\vspace{.17in}

\noindent Safe supplies have been defined as stores of messages that persons can
fall back upon when they are in a position of having to maintain
interplay and yet not having anything to say. It is worth noting briefly
that islanders employ two social strategies that are akin to the use of
safe supplies, being, perhaps, functional alternatives for safe
supplies, and yet somewhat different from them.

First, there were certain acts of a task-oriented kind, such as eating,
smoking, or knitting, which islanders, under certain circumstances,
allowed to be interspersed between messages, so that the same number of
messages could be stretched out over a longer period of time without
arousing a feeling that unwanted silences occurred. The womenfolk
especially employed this technique in the case of knitting, and three or
four women knitting together could by that means maintain themselves in
a kind of slowed or dormant interplay, where it was understood that
those present were accredited participants but where spates of knitting
and silence were permissible between messages. It was considered
improper for men to knit (although in some cases this would have
provided them with a better income than they could earn on the croft),
and they often employed pike-smoking as a substitute. The length of time
taken to cut tobacco, fill, light, and relight a pimple, and the length
of time taken on each draw provided welcome pauses between messages.
Both sexes often used the fire in open fireplaces as a resting device.
The constantly changing shape of the flame apparently expertises a kind
of sought-after hypnotism, allowing a person to pause after receiving a
message and stare into the fire before answering.

Secondly, a kind of interplay can be maintained by means of organized
recreation or games. In general, these systems of interaction allow for
the maintenance of accredited participation and a single focus of
attention, although the messages involved may not be of the linguistic
kind. In the case of games such as whist or billiards, rotation of role
of sender, length of messages, number of messages per participant and
per interplay, and the general character of messages are all determined
and accepted beforehand in terms of the general rules of the game. Each
shot or play, within the limited language and logic of the game, is a
kind of statement that must be attended to and answered in some way by
the other players. On the island, the playing of organized games was
extremely common and was to be expected whenever more than eight or nine
persons gathered together for convivial interaction. Without rather
mechanical means of this kind to organize messages, large parties, or
parties with islanders and non-islanders, could be expected to flag and
grind to an uneasy halt. Games as a source of messages is a source that
never gives out.\footnote{Group singing and cooperative participation in
  work tasks were also widely used as a means of assuring proper ritual
  relations between those present to each other in a given place.
  However, these processes do not typically have a distinctive
  \emph{inter}actional statement-and-reply character and have not been
  considered in this report.}

% CHAPTER XVI: ON KINDS OF EXCLUSION FROM PARTICIPATION
\chapter[CHAPTER XVI: ON KINDS OF EXCLUSION FROM PARTICIPATION]{Chapter XVI: On Kinds of Exclusion from Participation}
\label{ch:Chapter XVI: On Kinds of Exclusion from Participation}
\chaptermark{CHAPTER XVI: ON KINDS OF EXCLUSION FROM PARTICIPATION}

\newthought{In Chapter VIII the}
\marginnote{\href{https://doi.org/10.32376/3f8575cb.6427827e}{doi}}phrase ``in range'' was used to describe the
position of anyone who was within the zone in which reception of a given
impulse was possible. In many cases, all those who are in range of a
particular communication are also its accredited recipients.\footnote{Some
  cases where this is not true have been touched on in the discussion of
  indelicate communication chapter vi.} This is true, for example, when
two persons stop to talk to each other on an otherwise deserted road or
in an otherwise empty room, or when all the persons in a hall are being
addressed by a speaker. When all the persons who are in reception range
of an interplay are also accredited participants in it, we shall speak
of physical closure.

When four or more persons are together in the same bounded region, they
may separate off into more than one cluster or grouping, with each
cluster maintaining a separate and distinct interplay. If the size of
the region is great enough relative to the number of persons in it, it
is possible for voices to be modulated downward and for the space of the
region to be apportioned so that each interplay in the region is
physically closed.\footnote{In small interplays physical closure can
  almost be guaranteed by whispering. In Dixon, as in many other places,
  however, whispering is considered ill-mannered and does not frequently
  occur. It constitutes a disturbance for persons in other interplays;
  it signifies that something is being specifically concealed from them.
  (This is also true of the use of codes, the spelling-out of messages,
  and the use of dialect; see the discussion in chapter viii.)} This
guarantees that no interplay will either be overheard by unaccredited
recipients or be a disturbance for other interplays in the region. The
same effect is sometimes approximated when the sound intensity of voices
is modulated upwards so that the reception of a particular interplay is
jammed for all persons not in the interplay. This kind of communication
arrangement is found in crowded pubs and bars, and on streets where the
noise level is high.

Sometimes, however, physical closure is not possible, and an interplay
proceeds on the understanding that persons are in range who are not
accredited as participants. Seating arrangements in cafeterias often
produce circumstances of this kind. In any case, persons who
involuntarily find themselves in range of an interplay convey (by
appropriate undirected cues) that they are paying no attention to the
message which they are in a position to overhear. As previously
suggested, the accredited participants sometimes return the courtesy by
censoring their own messages for words that might provide too much
temptation for the outsider or that might cause him offense should he
happen to fail to keep his attention withdrawn. Communication
arrangements of this kind constitute what might be called ``effective
closure.''\footnote{Participation in an interplay from which one has
  been effectively excluded is apparently in some sense a safe thing to
  handle. An extreme illustration of this is provided by Morris
  Schwartz, ``Social Interaction of a Disturbed Ward of a Hospital,''
  (Unpublished Ph. D. dissertation, Department of Sociology, University
  of Chicago, 1951), p.~94, in reference to the conduct of a
  schizophrenic patient: ``\ldots{} the patient reveals that she is able
  to focus on others when she is not involved herself and when she feels
  unobserved in the process. In situations in which this occurs and she
  discovers she is being observed, she quickly turns her attention
  inward.''} Hotel lounges in Bergand very frequently provide the scene
for this kind of arrangement. The desire to sit close to the fireplace
(this may almost be considered a tropism in Britain) makes it necessary
for participants in different interplays to locate themselves close to
one another. Conversation is restricted to innocuous general topics, or
to domestic ones carried on by brief, affectless allusions that have
little meaning except to the accredited participants.

Effective closure is an arrangement by which accredited participants of
an interplay can act as if they were not being overheard. In formally
organized social occasions, effective closure is sometimes facilitated
by use of symbolic boundaries around areas within a region. The roping
off of a section of a hall sometimes has this effect. For example, the
music for dances held in the Dixon community hall is played on the stage
of the hall by accordionists and pianists recruited from the dancers.
Once on the stage, the performers talk among themselves with a mood and
``ethos'' peculiar to them, as if their absolute difference in function
and appreciable difference in physical elevation had produced a physical
barrier to ordinary communication with the dancers.\footnote{For
  further illustrations of this kind of behavior among musicians, see
  Howard S. Becker, ``The Professional Dance Musician in Chicago''
  (Unpublished Master's thesis, Department of Sociology, University of
  Chicago, 1949).}

Another example is to be found in the primary-grade schoolroom in Dixon.
Here groups of pupils of several different stages in schooling must be
taught in the same room. While a section in one grouping of seats is
being taught something on the board, other sections, in other groupings
of seats, act as if they are not in a position to overhear the
instructions and questions occurring a short distance away from them.
Effective closure is thus maintained, although negative sanctions on the
part of the teacher are sometimes required to keep a pupil busy with his
own work while instruction is being given to someone else close to him.
Sometimes the difference in ethos or climate between different but
adjacent sections becomes great. Subjects such as drawing require a
certain amount of movement on the part of pupils in order that they may
exchange limited equipment among themselves and compare efforts, and
discipline during these times is relatively lax. So effective can
closure become, however, that half the room can be involved in the
relatively relaxed yet humming atmosphere of the drawing period, while
the other half of the room can be the scene for lessons which require
rather continuous attention to the instructions of the teacher.
Interestingly enough, the blackboard (which is about six feet long and
four feet high, reversible, and mounted on casters) is frequently used
as a symbolic barrier. Pupils at one stage in schooling will be set to
do sums on one side of the board, and a different group of pupils will
be set to copying script written by the teacher on the other side of the
board.

It should be noted that effective closure is apparently very difficult
to arrange and maintain when the accredited participants enclose among
them, ecologically speaking, a person who is not an accredited
participants.\footnote{An exception may be cited. The interplay
  conducted in gesture language by the deaf and dumb provides some
  interesting communication characteristics, and the conventions under
  which it is conducted differ, apparently, in some ways from spoken
  interplay. For one thing, clearance interchanges seem to be more
  difficult to manage, and signs such as tugging at another's arm or
  clothing seem to be more commonly employed. Furthermore, in regions
  where there is a high noise level, intimate and easy spoken
  communication among normal speakers seems to be out of place, and
  speakers tend to restrict their talk to messages that are strictly
  required for the action at hand; deaf-mutes, on the other hand, may
  conduct intimate extended interplay under these circumstances. Also,
  since their interplays produce no disturbing sounds, and can be
  understood by few, deaf-mutes seem to feel free to conduct extended
  intimate conversation in public conveyances such as street-cars, even
  though the participants may be seated relatively far apart from each
  other. Presumably such communication neither interferes with spoken
  communication that might be going on at the time, nor does it force
  non-participants to listen to messages which they do not wish to hear
  and for whose reception they have not been accredited.} (In the case
of two-person interplay, this area would tend to be reduced to the line
of communication between the two accredited participants; an
unaccredited participant who intersects this line, blocking the path of
vision between the two accredited participants, is almost certain to
cause some embarrassment and to feel some.)

We have described two ways in which a person may find himself excluded
from an interplay; he may be physically outside its range, or he may be
effectively outside its range. A third possibility exists. He may be
treated\footnote{In the realm of undirected communication, an
  interesting closure problem arises because of windows. In Dixon, as in
  many other communities in Western society, one is supposed tactfully
  not to make use of any opportunity to look into a room by looking into
  its windows. One is supposed to act as if a physical barrier to sight
  \emph{completely}, not merely partially, surrounded the room. On
  occasions where a person does look into a cottage window he usually
  warns the inhabitants by means of a knock that he is doing so.
  Apparently one source of hostility to foreign seamen is that they do
  not obey this communication rule. They are said to wander up to a
  cottage and gaze into it through the window, doing nothing and saying
  nothing for minutes at a time, apparently unconcerned with the privacy
  rights of the inhabitants. Islanders consider this to be uncivilized
  behavior.} as a non-person, that is, as someone for whom no
consideration need be taken. A vivid illustration of this kind of
treatment is given by Orwell in his discussion of how patients in a
French charity hospital were treated and, reciprocally, how they
behaved:

\begin{quote}
On the other hand if you had some disease with which the students wanted
to familiarize themselves you got plenty of attention of a kind. I
myself, with an exceptionally fine specimen of a bronchial rattle,
sometimes had as many as a dozen students queuing up to listen to my
chest. It was a very queer feeling---queer, I mean, because of their
intense interest in learning their job, together with a seeming lack of
any perception that the patients were human beings. It is strange to
relate, but sometimes as some young student stepped forward to take his
turn at manipulating you, he would be actually tremulous with
excitement, like a boy who has at last got his hands on some expensive
piece of machinery. And then ear after ear---ears of young men, of
girls, of Negroes---pressed against your back, relays of fingers
solemnly but clumsily tapping, and not from any one of them did you get
a word of conversation or a look direct in your face. As a non-paying
patient, in the uniform nightshirt, you were primarily a
\emph{specimen}, a thing I did not resent but could never quite get used
to. \ldots{} About a dozen beds away from me was Numéro 57---I think
that was his number---a cirrhosis of the liver case. Everyone in the
ward knew him by sight because he was sometimes the subject of a medical
lecture. On two afternoons a week the tall, grave doctor would lecture
in the ward to a party of students, and on more than one occasion old
Numéro 57 was wheeled in on a sort of trolley into the middle of the
ward, where the doctor would roll back his nightshirt, dilate with his
fingers a huge flabby protuberance on the man's belly---the diseased
liver, I suppose---and explain solemnly that this was a disease
attributable to alcoholism, commoner in the wine-drinking countries. As
usual he neither spoke to his patient nor gave him a smile, a nod or any
kind of recognition. While he talked, very grave and upright, he would
hold the wasted body beneath his two hands, sometimes giving it a gentle
roll to and fro, in just the attitude of a woman handling a rolling-pin.
Not that Numéro 57 minded this kind of thing. Obviously he was an old
hospital inmate, a regular exhibit at lectures, his liver long since
marked down for a bottle in some pathological museum. Utterly
uninterested in what was said about him, he would lie with his colorless
eyes gazing at nothing, while the doctor showed him off like a piece of
antique china.\footnote{George Orwell, ``How the Poor Die,'' in
  \emph{Shooting an Elephant}, p.~22, p.~24.}
\end{quote}

\noindent We are familiar with treatment of a person as virtually absent in many
situations. Domestic servants and waitresses, in certain circumstances,
are treated as not present and act, ritually speaking, as if they were
not present.\footnote{Bertram Doyle, \emph{The Etiquette of Race
  Relations in the South} (Chicago: University of Chicago Press, 1937),
  p.~19 and p.~39, describes how slaves who were waiting on their
  master's table would be expected to participate in the table
  conversation if bidden to, thus giving sudden recognition to the fact
  that they were expected to follow the conversation, their status being
  too low to make illicit overhearing a sociological possibility. As
  non-persons they could also walk into a white church service to give
  their master a message, without being defined as an interruption. This
  is seen today in the rule that a ``good'' personal maid does not rap
  at a door before entering. Another illustration of non-person
  treatment of Negroes is given by Mrs.~Trollope, \emph{Domestic Mappers
  of the Americans} (2 vols.; London: Whittaker, Trencher, \& Co.,
  1832), II, 56--57: ``l had, indeed, frequent opportunities of
  observing this habitual indifference to the presence of their slaves.
  They talk of them, or their condition, of their faculties, of their
  conduct, exactly as if they were incapable of hearing. I once saw a
  young lady, who, when seated at table between a male and a female, was
  induced by her modesty to intrude on the chair of her female neighbor
  to avoid the indelicacy of touching the elbow of \emph{a man}. I once
  saw this very young lady lacing her stays with the most perfect
  composure before a negro footman. A Virginian gentleman told me that
  ever since he had married, he had been accustomed to have a negro girl
  sleep in the same chamber with himself and his wife. I asked for what
  purpose this nocturnal attendance was necessary? `Good Heaven!' was
  the reply, `if I wanted a glass of water during the night, what would
  become of me.'\,''} The young and, increasingly, the very old, may be
discussed ``to their faces'' in the tone we would ordinarily use for a
person only if he were not present. Mental patients are often given
similar non-person treatment.\footnote{An illustration of non-person
  treatment is given by Schwartz in his study} Finally, there is an
increasing number of technical personnel who are given this status (and
take the non-person alignment) at formally organized interplays. Here we
refer to stenographers, cameramen, reporters, plainclothes guards, and
technicians of all kinds.

In Dixon, treatment as a non-person occurred in several different kinds
of situations. Some examples may be given.

\enlargethispage{\baselineskip}

1. There was a rule that the doors of the community hall were to be left
open during times when functions were being held in the hall and that
anyone who wandered in at these times had a right to stay if he
conducted himself ``properly.'' Often, on nights when billiards were
being held, foreign fishermen whose boat happened to be anchored in the
harbor would walk down to the hall and stay for a while in the billiard
room, watching the players. On these occasions, the islanders present in
the billiard room would continue with their game and conversation as if
the intruders who were present were not present at all. The
foreign-speaking visitors would not be nodded to, or spoken to, or even
closely looked at. An attempt would be made by the islanders to act as
if no constraint or influence had been caused by the presence of the
visitors.\textsuperscript{10} In fact, of course,
players became a little self conscious and demonstrated that they were
concerned about intruders by cursing them when they were sighted coming
towards the hall or leaving the hall. Such cases seem to suggest that
there are two types of non-person treatment, a simple kind that occurs
when a person present is excluded from consideration in an automatic,
unthinking way because of his low ceremonial status, and a more complex
kind that occurs when a person is excluded from consideration as a means
by which others present can consciously and concertedly convey their
dislike of him. The more complex kind of non-person treatment is
sometimes called ``the silent treatment'' and in some situations
constitutes an extremely brutal sanction.

2.\marginnote{of the communication
  conduct of the mentally ill, \emph{op. cit.}, p.~174: ``The extent to
  which patients in Class (1) {[}socially most withdrawn{]} become
  `non-existent' and `do not count' in the eyes of other patients is
  revealed in the following. Mrs.~Stillman had, according to her
  statement, `something very confidential and important' to reveal to
  the investigator. She looked around the ward for a place in which she
  could talk to him alone. It appeared that the living room was empty,
  and she invited him to talk there. Upon entering it, she discovered
  Miss Adams sitting and twirling a thread. Mrs.~Stillman stopped and
  said, `Oh, Ann's in here,' and then carried on ( with a shrug of the
  shoulder as if to say `she really doesn't matter') to reveal the
  confidential matter to the investigator.''} Household\marginnote{\textsuperscript{10}\setcounter{footnote}{10} A further illustration may be quoted from George
  Orwell, \emph{Down and Out in Paris and London} (London: Decker am
  Warburg, 1949), pp.~180--181: ``Once the lodging-house was invaded by
  a slumming-party. Paddy and I had been out, and, coming back in the
  afternoon, we heard sounds of music downstairs. We went down to find
  three gentlepeople, sleekly dressed, holding a religious service in
  our kitchen. They were a grave and reverend seignior in a frock coat,
  a lady sitting at a portable harmonium, and a chinless youth toying
  with a crucifix. It appeared that they had marched in and started to
  hold the service, without any kind or invitation whatever. It was a
  pleasure to see how the lodgers met this intrusion. They did not offer
  the smallest rudeness to the slummers; they just ignored them. By
  common consent everyone in the kitchen---a hundred men,
  perhaps---behaved as though the slummers had not existed. There they
  stood patiently singing and exhorting, and no more notice was taken of
  them than if they had been earwigs. The gentleman in the frock coat
  preached a sermon, but not a word of it was audible; it was drowned in
  the usual din of songs, oaths and the clattering of pans. Men sat at
  their meals and card games three feet away from the harmonium,
  peaceably ignoring it. Presently the slummers gave it up and cleared
  out, not insulted in any way, but merely disregarded. No doubt they
  consoled themselves by thinking how brave they had been, `freely
  venturing into the lowest dens,' etc. etc.''} maids, in Dixon, were recruited from the upper reaches of
the crofter class to serve in the homes of the gentry and in the hotel.
These maids, typically unmarried girls between the ages to fifteen and
twenty-five, were usually related in more than one capacity to those
whom they served. At ceremonial occasions such as weddings, at community
socials, at church, at auction sales, in the shops, servers interacted
on a relatively convivial and equalitarian basis with those whom they
served. In this sense there were ``personal relations'' between employer
and employee. Thus, when a maid waited on a table in the home of a
member of the gentry or in the hotel, those who were waited on would
occasionally attempt to bring the maid into the table conversation as an
accredited though temporary participant. Occasionally, too, instead of
bringing the maid into the conversation, those at table would introduce
a momentary lull into their conversation, taking it up after the maid
had left the room, or would tactfully limit linguistic messages to the
kind that would give the involuntary eavesdropper neither offense nor
the feeling that hushed secrets were being kept from her. And of course
maids tended to cooperate in maintaining this effective closure by not
paying apparent attention to what was being said at table and by not
tarrying too long too close to the table.

However it was also very common for gentry and hotel guests to treat
those who waited on them as if they were non-persons. In accepting food
or allowing plates to be taken away, those being waited upon would often
utter a very brief thank-you or extend a small smile to the maid, but no
interruption in the table conversation would be produced.\textsuperscript{11} Non-interruption was
facilitated by the presence of table bells and table buzzers, these
allowing persons at the table to summon a maid without having to
withdraw even momentarily as sender or recipient in the mealtime
conversation. Treatment of the maids as non-persons was apparently
facilitated by obliging them to wear black dresses, pennies, dark shoes,
and hair nets, this costume apparently making it easier to view the maid
in a highly segmental capacity. More important than these factors,
perhaps, was the practice of those at table to say things in the
presence of maids that were obviously offensive to the groups with which
the maids were identified, or to say things of an intimate nature that
would ordinarily be kept from the ears of an outsider. For, example, one
afternoon at lunch the new doctor said, while a maid was present:

\begin{quote}
I wish I knew some psychology, but I don't know if psychology would
apply to a preliterate people. They have nothing whatever in their
minds. I don't know, they may be queer because of the food and air.
\end{quote}

\noindent The point here is not that untactful things are said ``in front'' of
maids, but that these offenses may symbolize for the server and for the
served that the server\marginnote{\textsuperscript{11}\setcounter{footnote}{11} Even
  this minimal consideration may become subject to question in cases of
  non-person treatment. For example, at the 1952 political conventions
  in Chicago, guest speakers were wildly cheered at the moment when they
  came up to the podium. When in a position to respond to the audience,
  it was necessary for a technician to slide in past the speakers in
  order to adjust the microphone through which they were to speak. The
  question arose as to whether speakers, in the center of world-wide
  attention, were to withdraw momentarily from their reception of the
  ovation to acknowledge the technician whose body was brushing past
  theirs and to thank him for adjusting the microphone to their
  requirements. Some speakers attempted to treat the technician at least
  for a second or two as a person; other speakers tried to solve the
  problem by treating the technician at all times as a non-person. No
  solution seemed completely to fit the situation.} is not someone whose feelings, as a person who is
present, need be taken into consideration. The maids in Dixon,
incidentally, did not seem to be so thoroughly trained to their calling
as to accept this role.\footnote{It may be noted that cab drivers in our
  society have a similar problem. Two ``fares`` in the back seat may
  treat the driver as a non-person and engage in quite intimate
  conversations and activities. The driver is sometimes left with a
  feeling that he is somehow not being treated properly.} They tell
exemplary tales of times when they have interrupted a dinner
conversation and ``told a guest what for,'' shifting their role in this
way from non-person to person. As one maid said:

\begin{quote}
They say things in front of me as if I'm not there and I don't know
whether they mean me to hear or not. Last year the breakfasts were only
egg and bread and butter and porridge and once a week bacon and I told
them {[}the hotel owners{]} what they said about it and now they have
three and sometimes four course breakfasts. But some things they say I
don't tell anyone, not even Alice {[}her co-worker and closest
friend{]}.
\end{quote}

3. When the doctor visited the cottage of a sick crofter, treatment of
him varied quite widely. Sometimes he would be treated with great
ceremony, sometimes by means of a joking relationship. These kinds of
treatment will be considered later. On occasion, however, the difficulty
of putting the doctor in a relationship that would permit interaction to
continue seemed to be too great, and those in the cottage (except for
the sick person) would merely ignore the presence of the doctor.
Sometimes, especially if the visit came when a meal was being eaten, and
when the fare and the equipment was there for the doctor to see,
crofters would be unable to maintain the strategy of ignoring him while
proceeding with their own interaction, and would fumble with their food
or stop short in eating it, poised in readiness for the doctor's
leavetaking.

A similar means of handling a person with whom interaction would be
difficult to manage was practiced by workers in the mill, quarry, and
loading dock. Sometimes when the boss, Mr.~Allen, came on his periodic
tours of inspection, and caught them during a moment's break for a brief
chat, they would act as if he was not in fact there and would continue,
albeit self-consciously, with their talk.

4. During community socials it seemed that children were disciplined and
corrected only if they threatened to disrupt radically the adult
activity in progress. (This leniency was in line with the general
permissiveness which seemed to be shown toward children in Bergand.)
During a period when the audience was involved in listening to choral
singing, the children between the ages of about four and seven would
scamper down the aisles between the rows of seated adults, playing tag.
At a moment when an auctioneer was selling objects to adults present,
using the stage for his stand, children sometimes ``tested the limits''
by crawling across the front of the stage. During a dance, children
would cut through the dancers in pursuit of a balloon or of a friend. In
these instances, adults attempted as long as possible to overlook the
presence of children who were not paying attention to the action in
progress, and while the children no doubt were partly motivated in their
actions by a desire to attract adult attention, the children on the
whole seemed to express the feeling that it was perfectly proper to be
in the midst of organized social interaction and yet not pay attention
to it or be treated as persons who ought to pay attention. On occasions
such as the Christmas party, however, young children were not allowed to
play out the role of non-person and were coaxed into participating in
children's games\\\noindent as an official part of the festivity.

% CHAPTER XVII: DUAL PARTICIPATION
\chapter[CHAPTER XVII: DUAL PARTICIPATION]{Chapter XVII: Dual Participation}
\label{ch:Chapter XVII: Dual Participation}
\chaptermark{CHAPTER XVII: DUAL PARTICIPATION}

\newthought{In the previous chapter,}
\marginnote{\href{https://doi.org/10.32376/3f8575cb.a7731302}{doi}}consideration was given to the ways in which
persons may be excluded from an interplay. We now consider ways in which
persons who are accredited participants may withdraw from an interplay.

During an interplay it is not uncommon for a participant to move away
from the spatial region enclosed by his co-participants and leave the
interplay, temporarily or permanently. This kind of departure is a
well-designed sign vehicle for conveying a negative valuation of the
participants who remain in the interplay. Departure may thus create a
sign situation. A participant who wants to leave an interplay therefore
tends to wait for a moment that is opportune---a natural break, as it
were---so that the expressive implications of his departure will be
minimized. He also tends to offer excuses to the remaining participants,
so that a natural interpretation can be placed upon his departure. If he
leaves momentarily to fix the lights, close the door, or do any or the
other minor acts which help to maintain the region in order, he usually
shows by his proximity to the disturbance or by his official role (e.g.,
as host) in these matters, that his momentary departure is not a
personal reflection upon the interplay.

Whether a participant departs courteously or openly and flagrantly
stalks out of the interplay, the remaining participants are aware of the
departure and can openly modify their communication in accordance with
this fact. They may, for example, compensate for the offense caused by
the departure by making suitably abusive comments about the person who
has departed. We may therefore think of departure---whether executed
tactfully or not---as conforming to the feed-back model of
communication.

There are, however, ways in which a participant can leave an interplay
so that the remaining participants may neither recognize this fact
openly nor compensate for it effectively. Here we have the case where a
participant leaves the interplay but not his ecological position in it.
It is a case of withdrawal, not departure.\footnote{Withdrawal is, in a
  sense, a form of insufficient involvement, but it is not treated here
  from that point of view. The question of proper degree of involvement,
  a crucial problem in its own right, will be considered later.} The
disaffected participant acts as if he were attending to the accredited
messages, while at the same time his actual thoughts and attention are
elsewhere.

An illustration of how a participant may remain in his ecological
position and retain his status as an accredited recipient in an
interplay while at the same time withdrawing into imaginary places and
imaginary interplays is found in what Bateson and Mead call
``away.''\footnote{See Bateson and Mead, \emph{op cit.}, pp.~68--69. It
  is to be noted that while persons can be away with respect to a
  conversational interplay, they can also be away with respect to more
  loosely defined interaction systems, such as social occasions. During
  the community dances, for example , most couples, when they talked at
  all, allowed their talk to be structured by the ethos of the occasion,
  using a set pattern of high-spirited small talk concerning the
  evening. Often, however, a couple could be seen who were going through
  the motions of the dance but were engaged in talk of a serious kind
  that removed them, psychologically, from the rest or the dancers. So,
  too, the musicians, whose contribution set the tone for the moment,
  would often withdraw into a distant reverie all of their own.
  Similarly, a person washing dishes as her part of a cooperative work
  venture would sometimes start to hum in a very quiet way and soon
  become oblivious to all around her. Pupils in the primary grades
  seemed especially prone to leave the classroom in this fashion and
  suddenly begin to leaf through a reader or twist and untwist the strap
  of a schoolbag in an abstracted manner.} The participant keeps his
face more or less in a position to convey attention signs to the
speaker, but his thoughts and eyes turn inward or come to focus on some
object in the room. Persons who behave in this way are sometimes said to
be day-dreaming, wool-gathering, or to have gone into a brown study.
This kind of withdrawal may be rather apparent to the remaining
participants, but the obviousness of the withdrawal is apparently
compensated for by the fact that no other participant need join the
offender in his disaffection.

In Dixon, the practice of going ``away'' seemed common and was now and
then a threat to informal social life. During meal-time conversation, it
would be common for someone to withdraw from the interplay and start
playing with the cat in an abstracted way, or roll crumbs of bread on
the table in a fugue-like manner, or become lost in the latest picture
magazine. Almost always these acts of withdrawal seemed to be resented a
little by the remaining participants, but, as was typical with
communication offenses in Dixon, only young persons were sanctioned in
an explicit way for this misbehavior.

A participant may retain his status as an accredited participant and yet
at the same time engage in another, typically less inclusive, interplay.
This less inclusive interplay he typically carries on by means of signs
such as facial gestures and eye-to-eye signals, which can only be
received from within a narrow zone, and by means of a lowered voice,
which has a short range. By relying on vehicles of this kind, care is
taken to offer minimal jamming and disruption of the message that is
accredited at the time by the more inclusive interplay. By modulation
downward of sign impulses, lip service is given to the inclusive
accredited interplay, allowing everyone to maintain the fiction that the
privilege of participation has not been treated lightly. Prior and
official right is thus given to the inclusive interplay to dominate the
situation, as it were. In other words, we may have an accredited or
dominant interplay and a subordinate interplay occurring within it.
Typically, a subordinate interplay is initiated after the dominant one
has begun, and typically the subordinate interplay is terminated before
the dominant one has ended.

The formation of a subordinate interplay is commonly a source of
tension, perhaps because partial withdrawal of this kind provides such a
ready way of expressing some kind of disrespect for the dominant
interplay or for the person who is at the time the accredited sender in
the dominant interplay. Subordinate interplays vary, it seems, in an
important way according to the degree to which excluded participants of
the dominant interplay resent or accept the smaller interplay from which
they are excluded.

There are many kinds of subordinate interplay that cause little or no
offense to excluded persons who are accredited participants of the
dominant interplay. Frequently factors in the situation will make it
obvious that the partial withdrawal of those in the subordinate
interplay is clearly not an expression of disregard for the dominant
interplay. For example, during a formally organized social occasion, it
is sometimes necessary for the chairman or other officials to enter
briefly into a huddle with one or two other persons in order to
straighten out administrative details that may have become tangled. In
such cases no attempt needs to be made to conceal the fact that a
subordinate interplay is in progress; respect is shown to the dominant
interplay by making the subordinate one as brief, as quiet, and as
affectless as possible. Similarly, during such occasions as committee
meetings, it is not uncommon for adjacent participants who are somewhat
removed from the speaker to lean over towards each other and carry on a
brief muted conversation; this sort of withdrawal causes little offense,
especially if it can be felt by others that the messages conveyed in the
subordinate interplay involve a ``take'' to the dominant message, and a
take that could be given an official hearing without thereby disrupting
the working acceptance.

Those who maintain an inoffensive subordinate interplay must attempt to
minimize the interference which they cause, but they need not attempt to
conceal the fact that they are engaged in a subordinate interplay. There
are many cases, however, where toleration of subordinate interplays is
not very high. The situation may, for instance, offer no happy pretext
which excluded participants can employ as evidence of the fact that no
disrespect is being shown. The rule that attention must be paid to the
accredited sender may be strictly drawn. The content of the subordinate
interplay may appear to be---were it suddenly given an official
hearing---quite inconsistent with the maintenance of a working
acceptance. In these and other circumstances, subordinate interplays may
be declared illegal, as it were, and have to go underground. Thus, just
as subordinate interplays vary in the degree to which they are
inoffensive, they also vary in the degree to which those who maintain
them attempt to conceal that this is the case and attempt to communicate
with one another in a surreptitious, furtive, and underhanded
way.\textsuperscript{3}

\newpage Subordinate\marginnote{\textsuperscript{3}\setcounter{footnote}{3} For completeness, a minor communication arrangement must
  be mentioned. Sometimes a recipient will convey a furtive statement
  and make a careful attempt to ensure that many of those present will
  overhear what he has said and that he has said it furtively. The
  obligation of the accredited sender to overlook all subordinate
  interplay is thus more or less consciously exploited and played with.
  We sometimes employ the term ``stage whisper'' to refer to this
  communication aggression. Of course, the accredited speaker can turn
  the tables and force the person who is playing at whispering to send
  his message in an official way.} interplays that are carried out in a quite furtive way
provide an interesting subject matter for study. Sometimes it is
possible for a small number of persons to carry on this kind of conduct
because they happen to be outside the visual line of the speaker or of
those who are more or less responsible for seeing that order is
maintained.\footnote{A crude example of this is to be found in the
  primary schoolrooms in Dixon, where pupils will hold a book up between
  their faces and the teacher in an attempt to conceal from the teacher
  the fact that ``talking'' is going on. Sometimes a pupil will grimace
  at his teacher, when he cannot be seen by the teacher, apparently
  content with establishing a collusive relationship with himself.
  Adults in Dixon seemed to have learned that collusion should occur
  with someone, not merely with oneself.} Sometimes participants of
subordinate interplays can feign the sort of expression they would have
if they were indulging in an inoffensive subordinate interplay and at
the same time convey surreptitious messages which are quite inconsistent
with the working acceptance of the dominant interplay. Sometimes this
improper communication behavior is carried on by means of ``cant,'' a
system of signals which mean one thing to the initiate and another to
outsiders.\footnote{The ``shill`` or confederate operates in this way.
  Collusion during divorce trials, where the plaintiff and defendant
  convey a permissible discord to the judge in order to settle an
  impermissible one is another case in point.} Usually, however, the
offenders mange to conceal their offensive behavior by reducing the
whole subordinate interplay to a quick glance or a ``significant''
expression of the eyes. A wink is perhaps the standard gesture for
stabilizing this relationship. In any case, those who participate in the
furtive interchange enter into collusion with each other and express a
common, and usually negative, attitude toward the dominant interplay or
toward certain participants in it.\footnote{An interesting limiting case
  is found in what might be called ``double-talk.'' By means of this
  communication arrangement, persons engage an innocuous conversation
  but phrase their messages in such a way as to convey information about
  topics which they have no right to discuss together. Double-talk
  typically occurs in communication between a superordinate and a
  subordinate upon matters which are officially outside of the
  competence or jurisdiction of the subordinate but which are actually
  dependent upon him. It is a device by which the subordinate can lead
  the superordinate without putting into jeopardy the status difference
  between them. Armies and jails apparently abound in double-talk. It is
  also found in communications pertaining to questions of law.
  Double-talk permits two persons to make an illicit agreement with each
  other without putting one participant in the vulnerable position of
  admitting this fact to the other. Police}

An illustration of how subtle the cues which establish a furtive
interplay can be may be found in the auctions in Dixon:

\begin{quote}
Household furnishings have a relatively high second-hand value in Dixon
because the freight charges from Britain to the island are very high.
The auction sales that are held about once every two months are
therefore important occasions. A person who bids at these auctions runs
the risk of showing his neighbors how much money he has. A bidder also
runs the risk of openly competing with someone who is a relative,
neighbor, or friend. There is a tendency (which may be found in auctions
anywhere) for the bidder to signal to the auctioneer by means of
unobtrusive signs, so that in many cases it is impossible for anyone but
the auctioneer to tell who has raised the bid. Even the auctioneer
frequently makes mistakes, and persons are sold things that they did not
think they had placed a bid upon. Signals such as taking one's left hand
halfway out of one's pocket are used to convey bids. In general,
however, the bidder relies upon catching the eye of the auctioneer and
giving him an extremely noncommittal look. It is understandable that
there are widely current jokes in Dixon concerning the danger of so much
as looking in the direction of the auctioneer during an auction.
\end{quote}

\noindent In Dixon, during informal conversation, it was very common tor a furtive
interplay to occur as a means by which two or more persons could express
an impermissible attitude toward another person who was
present.\textsuperscript{7} Sometimes the collusive evaluation was a
positive or favorable one. Thus, when children between the ages of six
and about twelve were drawn into adult conversation and behaved in a
charming way,\marginnote{bribery, for example, is
  usually regulated through an etiquette which allows each person to act
  as if no bribe had been made or none had been uttered. The point of
  interest here is that all the persons in the dominant interplay are
  also in the furtive one. In double-talk there is no third person. The
  roles taken by persons in the furtive interplay are a slight upon the
  roles taken by the \emph{same} persons in the dominant interplay.} the\marginnote{\textsuperscript{7}\setcounter{footnote}{7} In mediated communication arrangements, the temptation
  to enter into collusive interplays is great, partly because it can be
  so easily managed. When person A is in the presence of person B and
  interrupts their interplay to talk over the telephone to person C, or
  to read a letter from person C, then some collusive action of A and B
  against C almost invariably occurs. Thus, when the maid answered the
  hotel telephone and told the person calling that Mrs.~Tate was a
  distance away and could not conveniently come to the phone, there
  would be a collusive smile between the maid and the hotel guests
  sitting near the phone.} adults would frequently convey to each other a very
warm approval of the young performer. Usually, however, collusive
interplays directed against a person present seemed to be a way of
punishing the person for having behaved in a foolish manner or a way of
correcting for the injury he had done to the sentiments possessed by the
other participants concerning how they ought to be treated or how a
person ought to behave. Thus, when the hotel managers were more strict
than the help thought was warranted, the help used sometimes to stick
their tongues out at their employers so that all but the target of the
aggression could see.\footnote{Children in the Dixon schools employed
  the same device against their teachers when the teacher's back was
  turned, but in some of these cases it appeared as if the pupil was
  mainly concerned with expressing to himself a spirit of defiance. Here
  again, collusion seemed to be with oneself.} Similarly, in the
kitchen, when someone got too excited, or too greedy, or too vain, the
others present would glance at each other with just a faint amount of
derision sparking in their eyes. So also, during billiards, if one
player got too much caught up in the game, either taking too much
pleasure in a good shot or showing too much anger at missing a shot, the
others present would often enter into a collusive relationship against
him.

On the island, the presence of a member of the gentry was always an
opportunity for islanders to enter into collusive communication. Thus,
when Mr.~Allen would come to the pier to check up on the rate of work
and to talk to the foreman, a worker located behind Mr.~Allen's back
would sometimes make profanizing gestures. On one occasion, a worker
took up an empty bag of lime and whirled it about his head, testing the
limits to which derogatory action could be carried on behind the back of
the boss without the boss seeing it. Interestingly enough, when one
person made an effort to tease a second person by making claims that
were literally false, the teaser would sometimes enter into collusion
with the remaining persons, in part, apparently, as a means of
guaranteeing that at least someone would know that it was a joke all
along. Here the teaser seemed to employ furtive interplay as a safety
measure, to ensure that later he could establish that he was joking, not
lying. This kind of collusion was frequently established by making an
exaggerated mouth gesture from a position in the room where all but the
person teased could observe it. This of course also guaranteed that no
one would give the joke away.

Some further illustrations follow.

\begin{quote}
The hotel managers, the Tates, and a few guests are standing in the hall
leading to the scullery. The cook faces them and participates eagerly
and politely in their conversation. The scullery boy, who is behind the
cook and concealed from the others, gooses the cook, who must keep a
straight face.
\end{quote}

\begin{quote}
Mr.~Tate is feeding the cat while he and the others in the kitchen are
eating dinner. Mrs.~Tate watches him and expresses a clear look of
affection which she seems to have been practicing up. One of the maids,
who thinks it is improper for a cat to be fed at the table and for
Mrs.~Tate to show affected affection, openly grimaces at both of them,
knowing that for a moment they will not be able to see her but that the
others at the table will.
\end{quote}

\begin{quote}
Mrs., Tate is talking to a friend about the possibility of buying his
cottage. A maid comes in whose boy friend is also interested in buying
it. Mrs.~Tate conveys by her eyes that the person is supposed to act as
if something else had been under discussion. He does.
\end{quote}

\begin{quote}
A customer in Allen's shop asks the clerk for a three volt flashlight
bulb. The clerk says that they only have 2.3 but suggests it be tried.
It immediately burns out. Customer then asks manager for a three volt
bulb. The manager says they only have 2.3, and it wouldn't do to try it.
The clerk casts the customer a knowing smile. Customer and clerk say
nothing.
\end{quote}

\begin{quote}
At a crofter's house party a visiting piano tuner from Capital City
tries to monopolize the evening and suggests that there should be a
round of story telling with each person telling one. Two guests shoot
each other a collusive, ``Holy Christ!'' look.
\end{quote}

\begin{quote}
A player at billiards makes a bad shot and gets over-involved; he
swears. Others present cast each other snickering looks.
\end{quote}

\begin{quote}
At progressive whist, a new player mistakenly shuffles cards at the end
of a hand. Two of the remaining three players cast him a friendly smile,
suggesting that a trick has been played on the game but that they will
neither tell nor take it seriously.
\end{quote}

\begin{quote}
A quarry team of seven is building a garage; four of them are digging
the pit. The job of one is to scoop out water. Instead of getting into
the pit he leans over slowly and tries to lift the water out. The man in
the pit looks at another outside the pit as if to say, ``Do you see what
this fellow asks to be done for him?''
\end{quote}

% Part Five

\newpage
\thispagestyle{empty}
\begin{fullwidth}

\begin{center}
\vspace*{3in}

{\fontsize{35}{24}\selectfont{Part Five}\par}

\vspace{1in}

{\fontsize{35}{24}\selectfont\textit{Conduct During Interplay}\par}

\end{center}

\end{fullwidth}

% CHAPTER XVIII: INTRODUCTION: EUPHORIC AND DYSPHORIC INTERPLAY
\chapter[CHAPTER XVIII: INTRODUCTION: EUPHORIC AND DYSPHORIC INTERPLAY]{Chapter XVIII: Introduction: Euphoric and Dysphoric Interplay}
\label{ch:Chapter XVIII: Introduction: Euphoric and Dysphoric Interplay}
\chaptermark{CHAPTER XVIII: INTRODUCTION: EUPHORIC AND DYSPHORIC INTERPLAY}

\newthought{When persons are in}
\marginnote{\href{https://doi.org/10.32376/3f8575cb.db30da51}{doi}}each other's presence, it is possible that no one
will be made to feel ill at ease, out of countenance, nonplussed,
self-conscious, embarrassed, or out of place because of the sheer
presence of the others or because of the actions of the others. No one
will have the feeling that there is a false note in the situation. When
these conditions are present, we may say that the interaction is
euphoric. To the degree that those present have been made to feel ill at
ease, we may say that the interaction is dysphoric.\footnote{The terms
  \emph{euphoria} and \emph{dysphoria} have been employed by students of
  preliterate societies to refer to social systems that are functioning
  well or functioning badly.} In this study we are concerned with
euphoric and dysphoric interaction only in cases where those present to
each other are \emph{also} involved in accredited directed communication
with one another, i.e., in interplay. (It is to be clearly understood
that many interesting false notes arise among persons who are engaged
only in undirected communication with each other.)

In Dixon, the specific requirements for euphoric interplay seem to be
very subtle and complex. So delicate a balance seems to be required of
factors potentially opposed to each other that it is a wonder any
interplay at all is completely euphoric.\footnote{In contrast,
  observation suggests that euphoric interaction is quite common in
  situations where persons present to one another are not engaged in
  interplay nor feel obliged to be. In Dixon it seemed easy for persons
  to fulfill unselfconsciously expectations regarding proper clothing,
  proper modulation of voice and gestures, and other requirements of
  public seemliness and decorum.}

\enlargethispage{\baselineskip}

When persons engage in interplay (as in any other activity) there is a
tendency for them to become unselfconsciously, spontaneously, and
unthinkingly immersed or involved in the proceedings. During any
particular interplay, norms seem to prevail which indicate the degree to
which participants ought to immerse themselves or forget themselves in
the interaction.\footnote{It has become common to consider interpersonal
  communication as that which occurs when two persons each take the
  probable response of the other into consideration. This view seems to
  be implied in G. H. Mead and to have been carefully elaborated into a
  model of feints and strategies and infinite tactical maneuvers by von
  Neumann. It overlooks the crucial fact that a sender} It would seem that in Dixon the most
general requirement of euphoric interaction is that no participant act
in such a way as to disturb or disrupt a proper degree of involvement on
the part of the other participants. This generalization does not answer
the question of what makes for euphoric and dysphoric interaction, but
only moves the question one step back, for we must go on to ask what
sorts of behavior on the part of one participant throw the other
participants off balance and make it difficult for them to involve
themselves spontaneously in the interplay in the way required of them.

Until\marginnote{is committed to
  the expressive component of his communication, this tending to be, in
  a sense, more of an impulsive response to the situation than a
  calculated and tactical adjustment to it. By expressing himself
  spontaneously, the sender becomes intimately a part of the situation,
  instead of merely a rational manipulator of it. In a manner of
  speaking, the character of the sender becomes lodged in and infused
  into his communicative acts, giving these acts a weight and a reality
  in their own right. To the extent that actors can control their
  behavior in accordance with tactical utility, communication can
  conveniently be seen as a type of abstract rational game---a game that
  can be played at a distance, in any convenient context, at any
  convenient time, and by means of any convenient set of symbols for
  denoting individual moves. To the extent that actors cannot prevent
  themselves from conveying their feelings on a matter (or do not
  attempt to do so), interpersonal communication can conveniently be
  seen as part of a unique concrete situation, each message inseparably
  part of the context in which it occurs. It would seem that the
  unthinking impulsive aspect of interaction is not a residual category
  that can be appended as a qualification to a rational model of
  communication; the spontaneous unthinking aspect of interaction is a
  crucial element of interaction.} now in this study, interplay has been considered from a rather
mechanical point of view. It has been suggested that orderly interplay
seems, in Dixon, to have certain functional characteristics: warning
must be given as to when the interplay is to start, when it is to end,
and who is to be officially included in it; during the interplay, a
supply of messages must be assured, interruption must be controlled and
regulated, and a transition from one sender to another must be effected;
a center of focus must be maintained. When these arrangements did not
prevail, dysphoria tended to occur. However, these requirements seemed
to be necessary but not sufficient grounds for euphoric interaction.
Interplay was often conducted in a perfectly orderly way and was
nevertheless dysphoric.

It is sometimes felt that euphoric interplay is interaction in which
participants are made to feel happy or pleased, and that dysphoric
interplay is interaction in which participants are made to feel
deprived. This is by no means always the case. Apparently deprivations
can be conveyed to participants in a way which leaves them saddened but
does not disrupt the euphoria of the interaction; indulgences can be
conveyed in a way which leaves participants happy but embarrassed.
Gaiety and lightheartedness can prevail in an awkward situation, and
anger and hostility can prevail in euphoric conversation.

In the chapters that follow, no assumption is made that a complete, or
satisfactory, or systematic analysis of euphoria and dysphoria in
interplay has been given. The problem will be approached from different
points of view, some of which overlap and some of which have very little
relation to each other. As many different approaches will be attempted
as the data seem to call for.

% CHAPTER XIX: INVOLVEMENT
\chapter[CHAPTER XIX: INVOLVEMENT]{Chapter XIX: Involvement}
\label{ch:Chapter XIX: Involvement}
\chaptermark{CHAPTER XIX: INVOLVEMENT}

\newthought{It has been suggested}
\marginnote{\href{https://doi.org/10.32376/3f8575cb.f205f188}{doi}}that euphoric interplay occurs when no participant
is dislodged, as it were, from a proper degree of unselfconscious
immersion in the interplay. This, of course, assumes that the
participants are involved in the first place, that is, that they have to
a degree cut themselves off from all things external to the
interplay---that they have mobilized themselves for the interplay and
have been carried away by it.

In stating that participants of any euphoric interplay must become
caught up in the interplay, it must be clearly added that the required
level of involvement varies from interplay to interplay. Thus, when a
housewife in Dixon is going about her daily domestic tasks in the
presence of her immediate family, it is possible for her to flit in and
out of euphoric interplay while practically all her attention and
interest is patently accorded to the pots on the stove or the bannocks
in the oven. Were she talking to the gentry or the minister, however,
such casual involvement would be considered an affront, and a more
focussed orientation to the conversation would be required.

Further, it is apparent that level of proper involvement varies from
role to role within a particular interplay. For example, at political
meetings in Dixon, it is permissible for women to knit and men to smoke
while a speech is being given, but it is not permissible for the speaker
to distract himself in these ways.\footnote{It may be noted that the
  place of knitting as a permissible limitation of involvement has
  recently undergone rapid change in urban centers of Western society
  (perhaps because of the war) and is in many situations a matter of
  doubt. For example, some American college professors permit female
  students to knit in class, some do not. Smoking seems similarly a
  matter of doubt in some situations. In Dixon knitting and smoking seem
  to be permissible at a very wide range of social occasions.}
Furthermore, the level of involvement, like the level of tension and
``excitement,'' also varies from one point in the interplay to another,
perhaps starting at a rather low pitch, building up to a crescendo, and
then gradually falling away in preparation for the termination of the
interaction. Thus there is a sense in which every interplay is
characterized by an involvement contour. In spite of these variations,
however, involvement levels for a given interplay come to be
standardized in the sense that anyone who maintains a degree of
involvement that departs from the expected is felt to have committed an
offense and is likely to disrupt the involvement of others.

\newpage In Dixon it seemed that persons who had had much experience with each
other and who knew what to expect of each other could tolerate
appreciable deviations from the involvement norm without becoming
improperly involved themselves. Improper involvement that was
predictable tended to be less disruptive than unanticipated
improprieties. On the other hand, interaction with strangers, however
brief and well-structured, tended to be dysphoric.

The factor of involvement suggests an interesting contrast between large
formally organized recreation, such as a social, and small convivial
interplay among a few persons. In the first type, boredom is not rare;
in fact it is sometimes so general in the audience that it is necessary
to say that the social occasion is at fault and not the participants. On
the other hand, an uninvolved recipient can easily be overlooked amidst
the many other participants and a feigned expression of interest is not,
in Dixon, at least, considered an offense at these large occasions. In
the second type---small informal interplay---boredom when present is
more visible, more of an offense, and less permissibly concealed by
feigning interest. On the other hand, the interest of the recipient is
continuously revived by the opportunity he has of himself taking the
floor. In Dixon, as apparently in other subcultures of our society, few
persons can consistently forego the opportunity that small interplay
presents to engineer a favorable image of themselves and to uphold their
own convictions; in exercising these opportunities, their flagging
interest and involvement is revived.

The requirement that persons be impulsively involved in interplay in
which they participate is borne out by a very significant rule, namely
that interplay must not be staged or worked out beforehand. In Dixon,
when about to tell a joke, or an anecdote, or a piece of news, the
speaker would first inquire if the story was known, and, if he had
already told the story to some of those present, he would preface the
story by excusing himself to them. Similarly, singers who regularly
appeared at concerts would attempt to have at least one new song for the
occasion, showing that their behavior was not a mechanical repetition of
previous activity. The Program Committees of socials were obliged to
search for new games for the same reason. Special occasions and special
food---events which could not easily be duplicated---also served as an
expression of the uniqueness of the situation and would lend euphoria to
it. The game of ``500'' was widely praised, and felt to be superior to
whist, because in ``500'' unique problems were likely to occur.

On the island, there seemed to be two somewhat different ways in which
insufficient involvement was expressed. These will be considered
separately.

1. One kind of insufficient involvement occurred when participants
expressed too little concern for the topic of conversation and what was
being said about it. Lack of concern seemed itself to be conveyed in two
ways. First, the unconcerned participant might act with coolness,
indifference, or pointed interest in events unrelated to the interplay.
Secondly, the unconcerned participant could insist on taking in a joking
way what others in the interplay had meant to be taken seriously, or
insist on taking seriously what others had meant to be taken
unseriously. In either case, the disaffected participant would give the
impression that the issues of the interplay were not the sorts of things
that could embroil him or even touch him. By holding himself apart from
the communication in this way, he was able to convey the self-image of
someone who did not think it worth while to convey his self-image at
that particular time. Unconcern often conveyed an invidious judgment of
those participants who apparently did consider the interplay important
enough to warrant involvement.

On the island, children were explicitly taught that they must ``take``
or ``show'' an interest in any interaction of which they were accredited
participants. Perhaps youths became social adults at the point where it
was no longer deemed fitting explicitly to enjoin them to show interest
when involved in interplay, although their obligation would be
indirectly impressed upon them.

In any case, islanders very generally practiced the courtesy of evincing
involvement in interactional proceedings, whether or not they were
actually involved. The lengths to which this kind of tact can be carried
ought perhaps to be suggested.

\begin{quote}
During birthday parties, for which up to about fifteen people gather in
a crofter's cottage (more could hardly be gotten in), it is customary
for organized parlor games to be played. In the main, these games
consist of putting persons in embarrassing situations. At one party, at
which there were fourteen persons, the mother of the man who was
celebrating his birthday hit upon the ``pig game.`` In this game, one
person at a time is brought into the room where those ``in the know''
are assembled and is told he is going to see a pig. A cloth is taken
away from what is a mirror and the person sees himself. Since the
sponsor of the game could not be discouraged, the game was played,
although all those present knew its secret and thought it not a funny
game. The assembled group went through the process of playing the game
on four or five persons. During the twenty minutes that this required,
each successive ``butt'' of the joke put on an act of surprise at seeing
himself in the mirror, and each time the audience put on an act of
finding this funny. Everyone present tactfully adhered to the
involvement form for games of that type, although no one present was
caught up in the game spontaneously.
\end{quote}
\newpage
\begin{quote}
During the annual Christmas social games are staged for the very young
as well as for adults. In one game, called ``Oxford and Cambridge,'' two
lines of players race in relay against each other. A young man of
twenty-five in one line found himself running against a six-year-old
girl. He pretended to be straining as hard as he could but actually
managed only to keep pace with his young competitor. His attempt to be
considerate of the girl was conveyed to everyone but at the cost of
showing that he was not really involved in the race.
\end{quote}

\begin{quote}
In Dixon, progressive whist is played as the first part of many socials.
Tables and benches are placed around the ball in a continuous circle,
and up to forty sets of four persons play the game. After each hand the
winning men go in one direction and the winning women go in the other,
thereby making it possible for many of those present to play with each
other. Apparently by playing with many partners, social ties are
reaffirmed. At the end of each game individual scores are recorded on
individual cards, and at the end of twenty-four games each player adds
up his total score. Prizes are given for men's highest score, women' s
highest score, and lowest score. During the last few games, interest
reaches a relatively high pitch, for at this time players with high or
low scores see the possibility of a prize realized or destroyed. At the
end and climax of the round of games, when prizes are awarded, some
tension is released by spontaneous clapping, and by cheering for the
winners. It is very widely known in the community that two elderly women
cheat in recording their scores, ensuring either a very high or a very
low score. They are known to be only average players, and yet one of
them almost always wins a prize at every whist social. Presumably they
are interested in acquiring the prizes or in the moment of acclaim that
comes with winning one. In any case they ``spoil'' the game for the
others. Some players feel it is useless to get involved in playing well,
knowing that simple cheating will obtain a higher score; other players
explicitly state that they have mixed feelings about the possibility of
winning, because if they win (especially two evenings in a row) others
might think it has been by cheating. In any event, many participants are
thrown off a little by the realization that two players are acting out
ordinary involvement in the play and yet are involved in quite another
way. And yet when either of the two known cheaters wins, everyone makes
an effort to show enthusiasm and greets the award of a prize to her by
proper clapping.
\end{quote}

2. Lack of concern in the proceedings has been suggested as one type of
insufficient involvement; another type is to be found when a participant
shows too much concern with his own relation to the proceedings. The
participant may be amply involved in the interplay but i sufficiently
forgetful of his presence in it. Two varieties of undue self-concern may
be suggested.

First, the actor may give the impression of being too much concerned
with the fact that it is he who is sending or receiving the message. He
may give the impression that participation is grounds for such anxiety
that he withdraws from spontaneous communication with others, in a kind
of startle response, blotting out all concerns with worry about himself.
We call this self-consciousness.\footnote{Self-consciousness, of course,
  is also found in situations where only undirected communication
  prevails. While walking, persons in Dixon are supposed to remain
  relatively unaware of their actions and more or less forget about the
  presence of their bodies. Under some circumstances, as when persons of
  high status unexpectedly appear, an individual may become
  selfconscious; his face may get red and he may feel that his walk has
  become patently stiff and unnatural.} We detect it in others by a
characteristic look in their eyes and by characteristic fumbling
behavior on their part. When we detect it in others, the lack of ease
which it implies is likely to be transferred to us in the form of
embarrassment.\footnote{As a qualification it should perhaps be added
  that there are some interplays wherein a participant of extremely
  subordinate status is expected to show some selfconsciousness and
  causes offense if he is too much at ease.}

It may be noted that when a crofter and a member of the gentry engaged
in interaction, the crofter, especially, was likely to become
self-conscious. A meeting on neutral ground---as at an auction---was not
so likely to be dysphoric, but a meeting on either's home territory---as
in the house of either---almost always resulted in selfconsciousness.

There were times in crofting circles when unselfconscious involvement
was difficult for persons to achieve. When an individual suddenly found
himself in a position where much could apparently be lost or gained by
the nature of his behavior, or where he was the center of many persons'
attention, as when someone unused to performing performed at a community
social, then the individual found it difficult to remain
unselfconscious. Women of almost any age found it difficult to sustain
an explicit compliment with equanimity and would sometimes turn away on
these occasions, cast their heads down modestly, or rush at their
``tormentor'' with arms flailing, in a joking effort to disrupt the
interchange.

It is also interesting to note that persons under the age of about
sixteen (and the younger they were, the more this was true) found it
difficult to interact with anyone outside of the immediate family
without becoming acutely selfconscious. Frequently these persons would
feel impelled literally to hide their faces so that their embarrassment
could not be seen. However, the more likely a person was to act in this
way, the more likely it was, on the whole, for him to be defined as a
not-yet-person whose embarrassment was not an important enough thing to
embarrass the interaction in which it occurred.

There is a second variety of undue self-concern during interplay: the
actor may give the impression of being too much concerned with the
effects his message is having upon the recipients or the effects his
reception of the message is having upon the sender. Other participants
come to this conclusion because they feel the actor is attempting to
employ expressive behavior in a calculated way, and presumably no one
would do this unless they were more interested than is proper in
determining the response of others. We call this affectation. In
Cooley's words, the individual ``\ldots{} seems to be unduly preoccupied
with what other people think\ldots`` of him; affectation, he says,
``\ldots{} exists when the passion to influence others seems to\newpage\noindent
overbalance the established character and give it an obvious twist or
pose.``\footnote{Charles H. Cooley, \emph{Human Nature and the Social
  Order} (New York: Scribner's, 1922), p.~196.}

\begin{quote}
Thus there are persons who in the simplest conversation do not seem to
forget themselves, and enter frankly and disinterestedly into the
subject, but are felt to be always preoccupied with the thought of the
impression they are making, imagining praise or depreciation, and
usually posing a little to avoid the one or gain the other.\footnote{\emph{Ibid.},
  p.~215.}
\end{quote}

\noindent It may be noted that Berganders are restrained in regard to expressive
behavior and find it difficult to believe that the volatile
expressiveness of some outsiders is, for the outsiders, a natural and
spontaneous thing. Instead, Berganders tend to feel that a show of
expression must be something introduced for a calculated purpose and
that the actor is therefore insincere and something of a poseur. This
may partly explain why many Berganders feel that most outsiders are
either false or foolish, or both.

\vspace{.2in}
\begin{centering}

\Large{* * * * *}

\end{centering}
\vspace{.17in}

\noindent The islanders have access to many strategies for ensuring sufficient
involvement of participants. Some are suggested below.

One widely employed technique seemed to be to make use of tension
developed outside of the actual interplay and to offer a measured
resolution of that tension. Since almost everyone usually felt himself
to be a little hungry and concerned about this fact, the serving of
food, especially ``nice'' food, was always a way of obtaining the
requisite distraction. The use of card-games and other games of chance
seemed to operate in a similar way, introducing a state to tension and a
resolution of it. The very common practice of members of a family trying
out on each other every competitive puzzle or game that was found in
magazines and newspapers may also be mentioned in this context. At
community socials, games such as ``Oxford and Cambridge,'' musical
chairs, ``Beetle,'' guess-the-weight competitions, raffles, etc., seamed
to play a similar role. In terms of the dynamics of a social occasion,
these devices seem to be a kind of ``safe supply.`` So, too, the
alteration of persons' relationship to each other, whether sexual or
social, effected by co-participation, served as a source of involvement.
In all these cases, involvement in the interaction seemed to be a
carry-over or transfer from involvement in events occurring in the midst
of the interactants. These events seemed to serve to distract persons
away from feelings of either selfconsciousness or unconcern. Alcoholic
beverages, brewed in many households on the island and commonly served
at convivial social occasions, seemed to play the same role, but perhaps
in a more direct way.

A final factor in involvement must be mentioned. It has been suggested
that individuals may be viewed as sacred objects: they can be offended
or pleased by the expressive significance of events which occur in their
presence. Every event that occurs in the presence of a person can have
or be given the capacity to confirm or discredit the image he has of
himself and the image others have of him.

In Dixon, if the events during an interplay convey to a participant a
judgment of him that has not been taken for granted or built up in the
interplay, he tends to lose his poise and become embarrassed. This
crucial fact will be considered at length in a later chapter. In order
to prevent interactional dysphoria, participants attempt to guide their
conduct in such a way as not to express an inappropriate judgment of
themselves or others. Paradoxically, however, if they succeed in being
completely tactful, often the interplay will become stale and flat, and
the participants will find less and less cause for involvement in it.

If rules of tact are followed, often boredom sets in. If rules of tact
are broken, often embarrassment sets in. Apparently a fundamental source
of involvement consists of the slight infraction of tactful rules;
either the infraction is committed in an unserious way or care is taken
to bend the rule but not break it. This source of involvement will be
illustrated in Chapter XXII.

% CHAPTER XX: FAULTY PERSONS
\chapter[CHAPTER XX: FAULTY PERSONS]{Chapter XX: Faulty Persons}
\label{ch:Chapter XX: Faulty Persons}
\chaptermark{CHAPTER XX: FAULTY PERSONS}

\newthought{It has been suggested}
\marginnote{\href{https://doi.org/10.32376/3f8575cb.549348f7}{doi}}that when all the participants in an interplay are
sufficiently ``caught up'' or spontaneously involved in the proceedings,
the interaction may be characterized as euphoric. To the degree that
participants fail to become sufficiently involved, because of too little
concern with the proceedings or too much self-concern with them, the
interaction may be said to be dysphoric. The perception by one
participant that another is insufficiently involved or that (as will be
considered later) he is too much involved, may itself serve to make the
perceiver feel ill at ease and must be considered along with the the
other factors that can make a participant lose his spontaneous
involvement in an interplay. So too, the perception by yet another
participant that someone has perceived an offender may throw this second
perceiver out of tune with the interaction. In these cases embarrassment
seems to be a contagious and regenerative thing, feeding on itself,
spreading from one participant to another, and from him to still others,
in ever widening circles of discomfiture.

There are interplays which seem to be destined from the beginning to be
dysphoric, so that persons who usually find themselves at ease in the
presence of others feel out of countenance. Thus, when five or six men
went with a lorry to their old schoolhouse to borrow a piano for a
community concert, the presence of little desks and all the rest of the
schoolroom paraphernalia to which they had been tied thirty years ago
seemed to reinvade their earlier selves to such a degree that their
present ones could not be maintained with equanimity. So, too, when the
time came for male lambs to ``lose it,`` as the islanders say, the
presence of heaps of testicles and the necessity of holding squirming
heavy lambs while castration occurred would make it difficult for
workers to suppress a sexual definition of the situation, and the usual
quiet work self would be disrupted by a much more bawdy one. Similarly,
when a crew of men unloaded lumber which they knew had been ordered by
the undertaker and was destined for coffins, it became difficult for
them to suppress thoughts of their relation to eternity and to
concentrate on merely being workers. At all of these times the
disruption of one's ordinary workaday self would be resolved by defining
the whole situation in a joking way. At such times, participants felt
that no one was responsible for the disturbance; the situation itself
was felt to be responsible.

There are also occasions in which a particular person found himself
playing a role that was difficult to carry off with equanimity,
regardless of the poise he might have. Thus when persons are invited to
weddings, either both members of a courting couple are invited or
neither, since a wedding is an occasion for young guests to advertise
their ``intentions,'' a ceremony of couples walking from the church to
the community hall making this explicit. There is an overriding rule,
however, obliging siblings of the couple being married to sit at the
ceremonial head table. Thus a brother of the bride or groom who is
himself betrothed must arrange for someone to take his girl to the
wedding. The person chosen finds himself in a contradictory position
which he usually resolves by taking the situation unseriously.

There also seem to be certain kinds of interplay which a given person
cannot handle with equanimity, although other participants do not find
the same difficult and he himself is at ease in other interactions. And
in any person's daily round of interplays, there are likely to be one or
two individuals in whose presence he alone cannot be at
ease---individuals who appear to him to be affected, or presumptuous, or
insolent, or obsequious, to a degree that cannot quite be tolerated.

Our concern here, however, is that in any community there seem to be
some individuals who bring offense and dysphoria to almost every
interplay in which they participate, causing others to feel ill at ease
whether or not the offenders themselves are embarrassed. As suggested in
Chapter II, these offenders may be called faulty persons.

It is to be noted that persons who bring difficulty to many of the
interactions in which they participate tend to find themselves shunned,
or, if not shunned, treated in a very special way. This treatment need
not be the result of organized community reaction but may be the
unwitting consequence of the independent and often unthinking action of
members of the community. It is to be noted further that the
unsatisfactory handling of one's role in interplay is not a measure of
the way in which one handles undirected communication or tasks unrelated
to interaction. In Dixon, many quite faulty persons carried on the roles
of parent, husband, community member, and croft worker in what was
widely felt to be a very adequate way.

Perhaps the most obvious kind of faulty person is he who lacks adequate
command over the linguistic skills necessary for carrying on linguistic
communication in the given community. Ordinarily we assume that three
types of persons may lack this qualification: there are young children,
who do not yet have the capacity to carry on a conversation; there are
foreigners and outsiders who cannot manage a specific set of linguistic
symbols, although they can, of course, manage other sets; there are
defectives who do not have the intellectual capacity for communication
or who have defective communication equipment, as in the case of the
deaf, the dumb, and the blind. Defectives and foreigners, especially,
qualify as faulty persons because, unlike children, the immediate
response they call forth is an expectation that they will be able to
interact in an adequate manner. The few households in Dixon where idiots
lived seemed to be under a kind of interaction cloud; few persons not
members of these households seemed to enter without first steeling
themselves for the awkwardness that was to be anticipated.

Among commoners in Dixon, informal conversation is carried on solely in
the community's variations of the Bergand dialect. The dialect is hardly
intelligible to outsiders, and to them a Bergand version of standard
English is used. Almost all commoners in Dixon feel constrained in
situations where it is necessary for them to use standard English,
tending to lapse back into the dialect as soon as relaxation is
possible. Crofters can carry on official meetings in standard English,
or make speeches in it, but for informal conversation they find it
inappropriate and often impossible. Britons who come to the island for a
visit or who come to stay for a time for reasons of business cannot help
but disturb informal interaction in which they participate. They cannot
help but jar and distract their island listeners a little when they
talk; they cannot help ``missing,'' or not catching in time, many of the
truncated dialect ejaculations and introjections which form an important
part of informal discourse. Statements have to be translated and
repeated for them. While there are often additional reasons why
outsiders cannot be absorbed into euphoric intimate interplay, lack of
familiarity with the dialect is frequently sufficient to make these
persons faulty.

Another kind of faulty person is to be understood in terms of the fact
that individuals apparently make certain broad assumptions as to
standards of physiognomic normality that all persons ought to satisfy.
If the appearance of an individual departs too much from expected body
form (especially in directions that are valued negatively) then other
persons may be continuously distracted and diverted by the image that is
presented to them. It becomes difficult for recipients to disinvolve
themselves from the individual's offensive undirected communication, and
they therefore find it difficult to involve themselves spontaneously in
his directed communication. Few persons are sufficiently misshapen in an
overall way to become faulty persons for this reason. There is, however,
a significant number of persons who have minor physical defects
associated directly with their chief instruments of communication---the
eyes, the lips, the voice, and the face. Tics, bare-lips, cleft palates
are examples; ``bad breath`` is another. Such defects are, as it were,
always before the eyes, ears, and nose of the recipient, causing him as
much distraction as would a far greater defect less crucially
located.\footnote{Physical peculiarities are the usual but not the only
  types of disturbances. For example, Western people who have little
  experience with Hindu society often find difficulty in talking to
  Sikhs who wear turbans. Attention tends to waver from the race or the
  speaker to his head-dress.} The recipient is put in the difficult
position of having to direct his attention away from the sender's
defects in order to avoid offending the sender and to lessen the
possibility of involvement in the wrong stream of signs, while at the
same time he must direct his attention to just those areas in order to
show the sender that he is attending to him.

On the island there are some persons whose faces are not, by Western
standards, pleasant to look at. There is a tendency for these persons to
keep silent and to keep out of the view of the sender in an interplay,
except among members of their immediate family. They are thought of as
``shy'' and seem to content themselves with less social interaction than
do others. In a sense they have sacrificed themselves (for whatever
reason) to the euphoria of interaction, voluntarily withdrawing from
positions in which they might afford disturbance to the interplay.

Persons with defects of this kind did not, of course, always retreat. A
person seemed less likely to do so when he could feel that his defect
was a superimposed characteristic and not one that he was, in a sense,
morally responsible for. ``It was two years before I wasn't worried
about kids looking at my disfigurement, and now I don't mind at all,''
said one islander who had sustained an eye injury that left his face
disfigured and caused him to tear continuously.

It may be mentioned that acceptable individuals sometimes became faulty
persons for a brief period of time. A temporary disorder in
communication equipment would render a person unable, for a while, to
participate as smoothly as usual during interplay. Laryngitis,
extraction of teeth preparatory to obtaining false ones, intoxication,
nasal disturbances causing one to wheeze, a stiff neck---all these were
common reasons for temporarily transforming the individual into a faulty
person.

In the community there were a few other commoners whose behavior caused
them to be at fault in many of the interplays in which they
participated. A few of the islanders seemed to have been demoralized,
interactionally speaking, by their rise in social status, and could not
help bragging continuously about their achievements and their contacts
in the non-crofter world. These persons tended to be thought of as
insensitive and inflexible in their demands, and wherever they appeared
others present would have to make a slight effort to keep from making
apparent to their fellow-sufferers that they thought the braggart was
behaving improperly. However slight these offenses, the patience of
recipients was taxed whenever they were in the presence of these few
braggarts.

Two other faulty persons among commoners might be mentioned. There was
one boy of nineteen who was so sensitive about the moment-to-moment view
that others took of him, and so cowed and desirous to please, that even
in the circle of his own family he was self-conscious, conveying this
undue preoccupation with self to all with whom he had
contact.\footnote{For a psychoanalytical view of this kind of conduct,
  see Paul Schilder, ``The Social Neurosis,`` \emph{Psychoanalytic Review},
  XXV, 1--19.} And there was a man, Bill White, previously alluded to,
who played communication tricks; he would joke and kid like other
islanders, but he did this during serious occasions, and cut his jokes
so fine, acting so well the part he was toying with, and carrying on the
joke for so long, that persons came to distrust him.\footnote{For
  example, during a violin performance at a house party, he would turn
  to his neighbor and pretend to be whispering to him in barely
  permissible subordinate interplay, all the while conveying by his
  manner that he was merely making fun of the social arrangement which
  allowed listeners to enter into such interplay. At a whist social he
  would seriously say to a fellow crofter, ``You have to take an
  interest, you know,'' and later the crofter would learn that Bill had
  won the booby prize. Similarly, at a game of ``500'' among three
  guests in his home, he would say, with a barely perceptible twinkle,
  ``I'll surely win the next 500,'' conveying almost an admission that
  his interest was insufficiently aroused to make this even a
  possibility. On another occasion, while talking to the newly arrived
  doctor and his wife, Bill said, ``I'll be glad when these Bolsheviks
  are out and we get Britons back in power,'' knowing that everyone but
  the newly-arrived pair knew he was a local committee man for the
  Labour Party.} With him, one never knew where one stood, for there was
no easy way to discover whether he was at any given moment serious or
not. He protected himself from being considered simply mendacious by
maintaining a gleeful, aggressive air in his communication, ready when
forced into a corner to admit that he had only been joking.

\vspace{.2in}
\begin{centering}

\Large{* * * * *}

\end{centering}
\vspace{.17in}

\noindent In Dixon, when the gentry appeared in the presence of commoners,
interactional tension occurred. This dysphoria tended to be minimal when
the gentry acted in their traditional capacity, appearing on the stage
during a community concert or in specially reserved seats; it tended to
be quite acute when prolonged informal interplay was necessary between
gentry and commoner. From the point of view of the interaction that
commoners carried on, the gentry were all faulty persons. This fact
seems significant enough in the life or the community and significant
enough for an understanding of interaction there to warrant further
elaboration and analysis. In attempting this analysis, it will be
possible to extend a little the treatment in Chapter VII of sign
situations.

In the sociological literature, it is assumed that a person who has
roles or other attributes which qualify him for radically incompatible
kinds of treatment causes sociological difficulties.\footnote{A basic
  statement of this problem is given by E. C. Hughes, ``Dilemmas and
  Contradictions of Status,'' \emph{Amer. J. Sociol.}, L, 353--359. A
  clear example of the problem introduced by someone of indeterminate
  status is given by Doyle, \emph{op. cit.}, in his discussion of the
  relationship established by whites to free Negroes. See especially
  chap.~vii, ``Etiquette and the Free Negro.''} In many cases where an
individual possesses attributes which qualify him for radically
different treatments, primacy is accorded to one role in one situation
and to another role in another situation.\footnote{See Talcott Parsons,
  \emph{The Social System}, p.~302.} Thus, for any given situation,
there will always be a role defined as officially relevant and other
roles defined as irrelevant. Action, then, need not break down for want
of a pattern to follow.

\enlargethispage{\baselineskip}

In Dixon, there are some situations in which a genuine dilemma occurs as
to which of two patterns of respect, a deferential one or an
equalitarian one, the gentry are to be accorded by commoners. Explicit
discussions occur among commoners as to whether or not it is proper and
desirable to ``sir`` the laird or just to call him Mr.~Alexander, and
whether a special section of the seats at the annual concerts ought to
be reserved for the gentry or whether they should be required like
everyone else to take what is available at the time of their arrival.
These discussions were heated and had to do with expressive acts which
crofters ordinarily kept from consciousness, or at least kept silent
about. Today no consensus exists in the community as to how these
matters ought to be handled. Different commoners handle the question in
different ways. But in many cases, the decision taken by a particular
commoner is taken selfconsciously; he knows that other commoners act in
other ways and that the whole matter is problematical. When more than
one commoner is present at commoner-gentry interplay, then tension in
this matter seems especially high, for each commoner tends to feel that
the salutation and other gestures of respect he performs toward the
gentry will be examined by the other commoners for signs of undue
insolence or undue deference. The few crofters who still touch their
caps to the laird feel particularly selfconscious in this context,
finding themselves caught between what is for them ``natural`` respect
and the implied claims of fellow-crofters that the laird is no longer to
be treated as someone superior.

\enlargethispage{\baselineskip}

Another example may be cited. The retired doctor on the island---a
person of the gentry class---has an adult son who has not succeeded in
reconfirming the professional status of his father and grandfather and
has taken to operating a small farm. There are a handful of commoners
who accept this man as one of them, as he apparently wants to be
accepted. They reciprocally first-name him and participate with him in
informal convivial social occasions. For other commoners, however, he is
neither fish nor fowl. They find it difficult to decide how to treat
him, and when they do decide they cannot carry out the treatment in a
spontaneous and unthinking way. He is for them a faulty person.
Apparently he deeply feels the anomaly of his position. When in the
presence of commoners, he feels constrained to talk and act more loudly
than others, putting himself in what he apparently knows to be a
foolish, unworthy position. Apparently he feels that the only way he can
establish himself as an acceptable ordinary person is to show others, in
a continuous and relentless way, that he thinks someone like himself is
just as hopeless and impossible as (he feels) they think he is.

When there is a dilemma of status, embarrassment often results. However,
it frequently seems that the dysphoria which occurs in these situations
is not so much due to the fact that persons will decide in favor of one
line of treatment or in favor of another, but that thought and
consideration has to be given to such matters. If a definition of the
situation is not automatic and unthinking, then, from the point of view
of interaction, it does not much matter how things come to be defined,
for dysphoria is likely to arise no matter what line of treatment is
finally fixed upon. This leads us to appreciate that dysphoric
interaction can be caused by status difficulties much less blatant than
the one that occurs during a genuine dilemma of status.

During interplay in Dixon, it is customary for certain social attributes
of the participants to be declared officially irrelevant and for others
to be defined as the ones which ought to determine treatment during the
interaction. However there were always special circumstances which
forced upon the attention of participants a role usually successfully
suppressed. Thus, in a friendly interplay, the knowledge that two of the
participants were married and had been for years was allowed to enter
the interaction at appropriate times but could also be conveniently kept
from consciousness at other times. However, when a young couple was
about to get married, or had very recently been married, their new
relationship was something that often could not be suppressed from
attention by persons with whom they interacted. The new social fact
tended to disrupt the usual inattention to such matters, causing the
participants to become selfconsciously involved in the interplay.
Frequently this tension seemed to be released by jokes and ``kidding.``
Engaged persons or newly-married ones while in the presence of others
frequently treated each other in a stiff and distant fashion, apparently
in an attempt to counteract the effect they had upon the interaction.
Thus, individuals in the process of undergoing a basic change in status
tended to become, for a while, faulty persons, for their changing status
could not be kept from mind in situations where it ought to have been
irrelevant. The laird, who is in the process of selling his land and
losing his traditional status, is partly for this reason a faulty person
for the commoners.

It has been suggested that officially irrelevant roles may be handled in
such a way as not to disturb the euphoria of interaction. For example,
at meals in Dixon, everyone present (except infants) is given a helping
of about the same size, and differences in age, sex, and kinship are
momentarily set aside. However it is also expected that a participant's
officially irrelevant attributes will qualify in a minor way the
treatment accorded to him in his officially relevant capacities. Thus,
among commoners at dinner, an adult guest of either sex will be served
first, and girls and young women are expected to eat a little less than
others present. Similarly, the clerks in the Dixon shops are expected to
treat all customers equally; each customer has a right to be treated
with a modicum of civility, to be given an equal share of rationed
goods, and to be waited on in turn. Yet it is also expected that there
will be something in the tone of the treatment to distinguish islander
from foreigner, crofter from gentry, kinfolk from someone unrelated. The
point here is that in some situations no one seemed sure as to just how
much qualification of this minor kind was to prevail and where in the
interaction it was to be expressed. There would be no dilemma as to the
rights and obligations to be officially acknowledged, yet there might be
uncertainty over the covert recognitions to be given to the officially
irrelevant statuses.

Furthermore, it seemed that when officially irrelevant statuses
qualified a person for treatment that was radically different from the
kind accorded him in his officially relevant role, difficulties arose.
Spontaneous involvement in the interplay in terms of the officially
relevant roles tended to become swamped by nervousness over potential
responses that were officially irrelevant and which ought to have been
suppressed. The fiction that the participant was just another
participant became difficult to accept unthinkingly, even though
lip-service could be paid to it without difficulty. For example, when
the doctor's wife came to shop, she was treated in general like any
other customer, but it was a little difficult and embarrassing to do so.
On such occasions, the tendency of the person causing difficulty to
``lean over backwards'' to fit into officially defined patterns, or too
much enter into the spirit of things, did not succeed in preventing
dysphoria from occurring.

We have considered the fact that interaction between crofter and gentry
on the island tended, wherever and whenever it occurred, to by
dysphoric. Persons on one side did not quite know where they stood with
persons on the other side, nor where they ought to stand. Every
situation became a sign-situation, with persons on either side anxiously
examining every event, feeling that a judgment of their officially
irrelevant attributes was being conveyed, or that others might jump to
the conclusion that such a valuation was being conveyed. An unthinking
involvement in the actual events at hand was difficult to maintain.

It may be added that two general strategies seemed to be practiced as a
means of avoiding dysphoric interaction. First, there was avoidance.
Members of the gentry attempted, for example, to send the maid to do
their shopping or to telephone their orders for delivery. They also
attempted to attend as few community socials as possible. Secondly,
joking relationships were maintained, allowing participants to take an
unserious view of the confusion and dysphoria resulting from interaction
between persons who could not be at ease with each other.

% CHAPTER XXI: INVOLVEMENT POISE
\chapter[CHAPTER XXI: INVOLVEMENT POISE]{Chapter XXI: Involvement Poise}
\label{ch:Chapter XXI: Involvement Poise}
\chaptermark{CHAPTER XXI: INVOLVEMENT POISE}

\newthought{During interplay in Dixon,}
\marginnote{\href{https://doi.org/10.32376/3f8575cb.9990e006}{doi}}participants tended to set aside such of
their attributes and qualifications as were considered irrelevant and to
interact chiefly on the basis of rights and obligations felt to be
relevant. Persons expressed in this way that they were not so bound and
tied by their social roles that they could not set some of them aside
for a time and act in terms of others. A neighbor or relation who came
to help with the harvest would expect to be accorded a few ceremonial
gestures at the beginning and end of the day's efforts, but during the
work he would take his place alongside members of the immediate family
and any paid help, and tactfully restrict himself to the role of a
worker. At a community social, the oldest and youngest persons present
might dance together, a woman seventy and a boy of ten, and while they
and the others would joke a little about this, during most of the dance
the couple would simply in the capacity of dancers.

We have considered the fact that during euphoric interplay participants
express immediate involvement---and immersion, as it were---in the
proceedings of the interplay. It must be carefully stated, however, that
while participants regularly expressed unthinking involvement in the
proceedings, it was felt that there ought to be a definite limit to this
involvement, and participants made quite sure to express or feign that
this limit existed.\footnote{An historical treatment of changes in the
  etiquette of self-control is given in Norbert Elias, \emph{Uber den
  Prozess der Zivilisation} (Basel: Haus zum Falken, 1939), especially
  Vol. I, chap.~ii, ``Uber die 'Zivilisation' als eine spezifische
  Veränderung des menschlichen Verhaltens.''}

Whatever the occasion, it seemed that the individual felt strongly
obliged to show that he was not fully constrained by the events at hand;
that he had a self available for interaction that could not be
overwhelmed, a self that was not bound by any uncontrollable impulse to
act, a self that was free to answer to the interaction not merely for
the moment but wherever it might lead. Instead of conveying merely an
involvement in the proceedings, the participant conveyed a delicate
balance between involvement and self-control. He expressed the fact that
regardless of what happened during the interplay, or what commitments he
had outside the interplay, he could exercise self-control.

Participants, then, expressed the fact that they could temporarily
dissociate themselves from those of their statuses which were defined as
irrelevant for the interplay. They also, as suggested, expressed the
fact that they were not completely constrained by the events at hand
which occurred in the interplay although they were to a degree
spontaneously involved in the interplay. Participants gave the
appearance that they had mobilized their selves for the interaction at
hand, rigidly bound by only one obligation---the obligation to sustain
continued communication with the others present. Failure to exercise
this control and this readiness for interaction meant that the
participant could not be trusted to act so as not to disrupt the
involvement of others in the interplay; the appearance of someone acting
with insufficient self-control itself caused others to become ill at
ease.

In this chapter, some of the factors involved in poise---the handling of
oneself during interplay---will be considered. While poise is a tenuous
thing to study objectively, and a difficult thing to report upon, it is
a factor that can hardly be avoided in a general study of
interplay.\footnote{It would seem that the only sizable literature on
  poise is to be found in books on etiquette and manners. On the whole,
  this material has been scorned by social scientists, presumably
  because the significant observations on the moral norms of interplay
  contained therein are indiscriminately mixed both with personal
  exhortations as to how individuals ought to behave and with optimistic
  claims as to how leaders of circles now extinct (or becoming so)
  actually conduct themselves. In scorning these works we have also, of
  course, scorned to study many fundamental aspects of social
  interaction. Unfortunately, some students have similarly by-passed
  Simmel's treatment of ``sociability'' because of the courtly bias in
  some of the standards he describes.}

\hypertarget{ego-control}{%
\section{Ego Control}\label{ego-control}}

Co-participants during interplay are in a vulnerable position with
respect to one another. They are obliged to make themselves accessible
to one another and to treat each other with forbearance. They must
therefore run the risk that one among them may take unfair advantage of
the communication opportunities that have been entrusted to him.
Linguistically or expressively, he may abuse his position by conveying a
message that accords an improper valuation to himself or to others
present.

1. One of the most explicitly recognized roles of interplay on the
island is that each participant control and restrain his own demands for
approval and esteem.\footnote{Societies of course differ in rules
  regarding modesty, but certainly modesty during communication is
  stressed in many non-Western cultures. For example, see Hsien Chin Hu,
  ``The Chinese Concept of Face,`` \emph{American Anthropologist}, n.s.
  XLVI (45--64), 49. ``The over-estimation of one's ability, the
  exaggeration of one's capacity, designed to elevate one above one's
  fellows is frowned upon by society. A person given to boasting will
  not have the sympathy of his group when he fails; rather will he incur
  ridicule. A person with such poor judgment of his powers is termed
  `light and floating' (\emph{chi'ing-fou}) in character; a person
  serious in his} At the
linguistic level, it was felt that persons ought not to ``blow their own
horn, `` to brag, or in general to convey a message whose purport
redounded in their favor. At the expressive level, it was felt that
persons should not attempt to become the center of attention too
frequently or hold this position too long once it was obtained, or in
general attempt to manipulate the physical situation in order that it
might express something favorable about them.

In general, when persons were involved in conversation, they made an
effort to keep the topic of conversation away from anything having to do
with their own praiseworthy accomplishments. If this topic could not be
avoided, then there was a tendency for the individual to\marginnote{endeavors but careful in reckoning his abilities and
  circumspect in his dealings with others is called `sinking and steady'
  (\emph{ch'en-chuo}) or reliably heavy (\emph{wen-chug}). The former
  type of personality cannot be trusted, but latter is a good citizen
  and a trustworthy friend. Now is not easy to gauge one's capacity
  exactly at every point nor is it possible to foresee the outcome of
  every venture, so it is wise to underestimate one's value. In this way
  one will always have the satisfaction of hearing.one's friends deny
  this inferiority\ldots{} A person `without self-training' is one who
  shows no consideration for others or is given to boasting.''} minimize and
detract from his accomplishment, or to treat it in a joking manner as a
means of signifying that it was not bo be associated with the self
involved in participation. The more recent and the more praiseworthy the
accomplishment, the more the individual seemed to feel obliged to show
that he had not taken it too ``seriously.''

Perhaps the clearest evidence of crofter circumspection in
self-references is to be found in their use of the term ``I.'' If a
sentence could be phrased in such a way as to omit the term, it was
omitted. For example, in giving advice, an islander did not say, ``I
think you can do it this way,'' or ``This is the way I do it,'' but
rather, ``Some folk do it this way,'' or ``Let's try it this way,'' or
``Maybe it'll work this way.'' Strangers from off the island who
unselfconsciously followed the habit of beginning many statements with
the phrase ``I think that \ldots{} ,'' or ``I feel that \ldots,'' or
``In my opinion \ldots,'' were felt to be improperly concerned with self
and caused the islanders some tension during interaction.

\enlargethispage{\baselineskip}

It is helpful to look at this general rule of restraint in terms of some
of the offenses that are committed against it. There were a few persons
in the community, drawn from among those who'd had much contact with the
outside world and were rising in class status, who seemed to have become
demoralized in regard to ego discipline.\footnote{See the discussion of
  faulty persons in chapter xx.} These faulty persons would employ
strategies that were transparent to others in order to bolster the
valuation they felt others were making of them. They would tall stories
that presumed to be of interest in their own right but which in fact
merely provided the speaker with an opportunity of telling of events
which redounded in some way in his favor, or they would introduce a
topic of conversation that would inevitably lead another participant to
mention matters in which they had excelled. Or they would loudly claim
complete incapacity for the act they were about to perform, pretending
to establish a definition of self that would not be embarrassed by the
failure that was about to follow, and then perform the act successfully.
Or they would ask the opinion of someone present on such matters as the
weight of one of their new lambs or the condition of their Italian rye
grass, leaving this person questioned with no way out but that of a
compliment. Or they would make the kind of flat denial of personal
qualifications which forced others into denial of the denial, i.e.~they
would ``fish'' for compliments. Or they would make light of their
accomplishments in an insufficiently convincing fashion. And they would
attempt to monopolize the conversation. These persons had a reputation
throughout the island for this kind of behavior and they were felt to be
burdensome in conversation. Up to the age of about thirty they were
explicitly criticized, albeit in a joking way, for being braggarts.

2. In Dixon, those of the fully adult generation who had not had more
than average contact with the outside world showed strict circumspection
in dealing with their selves. It was felt that during interplay each
participant ought to be able to hold at a distance his involvement in an
event that had occurred previous to the interplay or was scheduled to
occur immediately after the interplay. It was felt that while he was a
participant these extraneous matters, however crucial for him, were to
be left unmentioned or referred to lightly. Thus, while islanders seemed
to have a deep, genuine concern for the welfare of their children, a
parent whose two children had the flu would contain his anxiety and
suggest to those with whom he happened to be conversing that it was true
the situation was a little awkward. Persons returning from the very real
hazards of a day's fishing in a bad sea, or from the hardships of a day
in the peat banks, tended to underplay in a marked way the dangers, the
hardships, and the rewards and losses of their activity. In making
self-references in the presence of non-islanders, it was common for an
islander to belittle himself, modestly referring to himself as merely a
crofter.

3. During interplay, an islander was expected to dissociate himself
modestly from any event which occurred during the interplay that gave
evidence to others of his desirable qualities. Thus, at socials, persons
winning a prize would laughingly discount their deed by such a phrase
as, ``The de'il's kind to his een.`` In making a good shot at billiards,
it was required that the player give a convincing expressive
demonstration that he did not take his luck or skill too seriously. The
youngest players, especially when first learning, often did not have
themselves in control in this way, and would convey an expression of
pure pride at making a good shot. This was thought to show weakness of
character, and caused some embarrassment. Players of the middle age
group---twenty to thirty---were aware that they ought not to take too
much open pleasure in their good shots and would only allow their true
improper feelings to escape for a moment before casting off the flow of
improper signs with a sarcastic remark, an openly posed sneer, or a
boisterous challenge to the next player. The older players, whether
beginners or experts, in the main had themselves beautifully under
control, and could make a good shot, express delight in the
accomplishment, laugh heartily and aggressively in reference to the
effect of this shot on the opposing team, and never give the impression
that they are judging themselves by the shot. After such a shot they
could say, ``Class will tell, don't you know,'' and perfectly convince
everyone present that they were not taking their excellence seriously.
During the athletic contests held at the annual gala day, the same kind
of self-control was to be found. Only in the case of two competitors,
who seemed to have seen themselves as athletes, did a too-earnest
attitude appear and a too-serious enjoyment of winning.

4. Participants tended also to try to dissociate themselves from any
role of special honor they may have been accorded in the interplay or
social occasion. At the concerts, performers would take their places in
the audience like anyone else, even paying the entrance price. At the
turn before their own, they would unobtrusively leave their seats and
make their way backstage via the kitchen. After giving their performance
they would retire to their seats in the audience by the same unobtrusive
route. And usually they would not come back to their seats with special
expressions of elation but would give the impression of being in the
same quiet mood as the audience. Those organizing or M.C.'ing socials or
concerts also tended to handle their special role in a way implying that
they took no personal credit for it. Those without too much experience
would often attempt too strongly to apologize for their special status,
on the ground that they were unworthy vessels, and cause some
embarrassment and dysphoria by little speeches of self-depreciation. But
in the main those who led the socials were able to talk to the audience
and move though the hall bent on organization tasks without giving the
feeling that they had taken their honor too seriously, or had become
distracted and confused because of it, or were trying to put too much of
themselves into it.

It seemed in Dixon that actors who possessed attributes which others
were required to suppress from consciousness during interplay were often
more alive to the disturbing effect of their peculiarities than were the
other persons who had to contend with them. Persons were always
mentioning their shortcomings and attempting to dissociate themselves
from these attributes so that in some magical way the person present to
the others would not be the disturbing one. If a commoner found that he
had to chair a political meeting because a minister or a member the
gentry could not be found to do it, he would introduce his introduction
with an apology, attempting to convince the audience that he, at least,
was not taking his honorific role seriously and that the person before
them actually was not the kind to presume to such a station. A man who
felt that others felt he was henpecked would jokingly admit that his
wife made all the decisions. A thirty-five year old woman, somewhat ugly
and with little chance of getting a husband, would joke at her younger
sister's wedding saying that if she knew there were going to be all
those presents she would get married herself. In guessing the weight of
a parcel of groceries or the number of beans in a jar---typical
competitions for raising money at socials---almost everyone making an
effort would loudly and forcibly claim that they were no good at such
things and were bound to be way off. The same remonstrance occurred when
someone took a billiard shot for the first time. And very frequently
when conversation sprang up between adults and an old person, the old
person would hastily make a depreciating remark about himself, saying
``Wasn't that pretty good for an old man,'' or ``That's as good as an
old man can do.`` In all of these instances, the apologetic actor
apparently felt that those present would be spared the effort of making
forbearance allowances for him if they could be shown that he himself
did not judge himself by the standards by which he was inadequate, or
that he did not take himself seriously.

Sometimes this kind of interactional footwork succeeded and euphoria was
maintained and even strengthened. Usually, however, the person
apologizing for himself would sound insincere or too apologetic; in any
case, the ruse often failed and increased the discomfort of others
present.

5. The sense in which individuals were required to hold themselves off
from any overinvolvement and to hold themselves ready for interaction is
illustrated by events which are a matter of life and death.

At times when life is threatened, it becomes extremely difficult for
persons to maintain themselves poised for interaction; they often forget
themselves as communicators and become solely concerned with
survival.\footnote{A clear example of this is found in so-called
  ``gallows humor.'' In situations of extreme deprivation, it is thought
  praiseworthy to joke about the situation and demonstrate that one
  still has a self free for the interaction. See the article on this
  subject by A. J. Obrdlik, \emph{Amer. J. Sociol.}, XLVII, 715--716.}
Behavior under these circumstances---whether the person in danger
forgets himself or does not---becomes, apparently, a memorable thing,
and accounts of behavior under stress are often repeated. Thus, through
two world wars the island's men had been recruited as seamen and many of
them experienced sinkings. Tales are told of the composure that some
seamen showed under these threats, behaving as participants in
interaction and not merely as men with their own lives to save, and also
tales of persons who completely forgot themselves. During the last war,
the island was strafed a little, and tales are still told of how persons
reacted.

\begin{quote}
A woman in her thirties who had worked for a time as a clerk in one of
the Dixon shops said: ``Well, we heard this shooting so we all went to
climb into the shelter. Old Jimmy Scott {[}the then manager the shop{]}
was behind me and got excited and said, 'Haste du lass,' and gave me a
push. I fell on my hands and knees and tore them and laughed. I don't
think I ever laughed louder.'' (The teller went on to suggest that it
was not so much that Jimmy lost social control of himself but that he
attempted a ludicrously thin veil of concern for others.)
\end{quote}

\begin{quote}
During the research Dr.~Wren, in testing out his new sailboat with two
commoners, a youth of fifteen and a young man of twenty-six, capsized in
a rough wind. All three managed to survive although only the doctor
could swim. Apparently for a moment it was each man for himself until
each had managed to straddle the upturned hull. For the youngest
survivor the image of the three of them forgetting each other was
memorable, and he repeated the tale many times as a subject for humor.
\end{quote}

6. There were times when a particular task required momentary placing of
one's body in a position where the give and take of communication could
not be easily maintained. At such times persons either tended to avoid
entering into interaction or attempted to initiate interplay and by
jokes and comments show that the self that could not properly
participate was not their real self but one so unrelated to them that
light jokes could be made about it. Thus, in carrying a sheep from one
enclosure to another, or in lifting a hundredweight sack of feed from
the pier dolly to a truck, or in straining a crowbar to free a piece of
rock in the quarry, men would often seek out the eyes of others and
initiate a momentary smile or openly feigned gesture of strain.
Instances such as the one mentioned below were common:

\begin{quote}
A crew of men are unloading the steamboat on a Friday night. A wall of
crates gets built up and a young member of the crew finds himself
leaning up against them to hold them up. The special hook used to grasp
the boxes cannot be found, so the crane cannot relieve the man of his
burden. In order to keep the crates from falling he must use all of his
body and not turn or twist an inch. The rest of the crew burst out
laughing at the sight of someone completely constrained in this way. He
blushes and laughs.
\end{quote}

\hypertarget{emotional-control}{%
\section{Emotional Control}\label{emotional-control}}

During interplay in Dixon, individuals tended to hold themselves back
from becoming completely involved in and committed to any particular
response they were making to the situation. This restraint characterized
both linguistic and expressive communications. The mere appearance of
anyone unreservedly throwing himself into an activity or linguistic
message tended of itself to make those who witnessed it ill at ease.
(Only children were permitted the luxury of complete expression.) It was
also felt that such behavior made unfair claims on all the participants,
for if a working acceptance was to be maintained after someone had
indulged himself in a free response of this kind, then the line
established by the uninhibited response would have to be followed by the
other participants. They would have to do all the accommodating, for in
fully committing himself the offender ceases to be able to apply tact
and make allowances for events which might yet occur. Visitors to the
island frequently caused tension in this way, for example, by too
heartily enjoying a dance at the social, or by running to get somewhere,
or by becoming so involved in a political discussion as to fall into
using profanities in the presence of women.

1. When engaged in a task in the presence of others, islanders tended to
inhibit any angry ``takes'' to unexpected task frustrations. A crofter,
finding one of his lambs tipped over in a wet ditch and weakened by a
night of cold, would just shake his head. A person on a picnic
accidentally breaking the glass around the cork in a thermos bottle
would not swear. Very irksome lengthy tasks would be undertaken, such as
taking out a few leaky planks from the side of a rowboat and replacing
them with sound ones, or fitting a cabinet into a kitchen, and no
outburst would ever occur. When a machine in the woolen mill broke down,
it was only the manager, a non-islander, who would go into a tantrum.

In the presence of task frustrations, islanders commonly attempt to
define the situation as one that ought to be approached quietly and
slowly. In acting in this stoical way, they leave themselves free to
continue with social interaction, safe from any impulsive entanglement
that would force them to withdraw from interplay.

2. A special case of overcommitment is found in what are sometimes
called emotional outbursts. It was understood that persons have a
breaking point beyond which they lose control of themselves and become
totally involved in an affective response to the situation. Fits of
anger or laughter, crying spells, and temper tantrums are cases in
point. In many such cases, the individual's action would become all
``take'' and no reply, and the take would be such that frequently all
that others could do was allow the offender to become the center of
attention or studiously avoid looking at him. Participants tended to
feel that they had on their hands an object of attention but not a
full-fledged fellow-participant. Whether the offender had given himself
up to laughter, tears, or rage, he was felt to have put himself in a
position where it was impossible for him to respond to the ongoing
events in the interplay. In Dixon these kinds of outbursts were expected
of children more than of adults, and adults who were faulty in this
regard tended to be not merely persons who lost control of themselves in
these ways but persons for whom special handling was required because it
was thought they might be capable of this kind of behavior.

As previously suggested, islanders tended to suppress signs of strong
emotions at such times as arrivals and departures. An illustration is
given below:

\begin{quote}
A well-liked young islander, John Neil, is leaving the island for a
prolonged voyage as a ship's engineer. On the eve of his departure he
spends his time, as he ordinarily would have, playing billiards at the
hall. During the game no allusion is made to his approaching departure,
and it is not thought peculiar that he should spend his last night at
home in in this way. As the time for the play to end approaches, William
Crosely {[}\emph{sic}{]}, in his fifties, a natural leader in the
community and a warm friend of John's, makes ready to leave. \\
Croseley: ``Well, lads, it's time I was off. Good night, John.'' \\
John, who apparently fails to get the slight glint of humoring
Croseley's voice, says in feigned light disapproval: ``Are you no going
to say cheerio, Will; I won't be back for eight months.'' \\
Croseley, smiling broadly as a sign that he has caught John out and has
gotten a rise out of him: ``So you won't.'' Croseley crosses over and
shakes hands with John, lightly wishes him good luck, and leaves. When
the game finally breaks up everyone says good-bye to John, no one
bothering to shake hands. Throughout no emotion has been shown. \\
Eight months later John returns. He has been to Singapore. After coming
to Capital City he gets a free ride to Dixon on one of the local fishing
boats which had been in to Capital City for repairs. A few clusters of
persons await his arrival. In one cluster is his betrothed and her girl
friend, in another a few of his male friends. As the boat comes
alongside most of the persons on the pier wave a little to John. As the
boat is made fast he steps off, nods to his betrothed, shakes hands with
his close boy friends, and immediately launches into a discussion of the
repairs that have been made on the local boat and how its engine is
standing up. A few persons come up and shake his hand but each time
there is no insistence that John make more than one or two statements on
the subject of himself or his voyage. He is allowed to fall back
immediately into the discussion that is being maintained concerning the
local boat.
\end{quote}

During crises such as deaths, crying, too, is suppressed, although
sometimes not altogether successfully. For example, Alice Simon,
twenty-four, admits that she cried at the four deaths that have occurred
in her immediate family, although these are the only occasions when she
admits to having lost control in this way. Interestingly enough, during
romantic movies shown in the darkened community hall, many women feel it
all right to weep. Presumably at these times there is no interaction
that can be embarrassed by their actions.

Protective strategies are often employed to save participants from the
embarrassment caused by a display of uncontrollable emotion. In Dixon,
when persons had become emotionally involved in the proceedings of an
interplay to the degree where they felt they were no longer in control
of the situation, and where it was not feasible for them to withdraw in
an orderly manner from the interaction to protect themselves and it,
they tended to cast their eyes downward and turn their faces away. In
this way an attempt could be made with the voice to suggest that
everything was in control and that the current message was being
responded to, while in fact expression in the eyes and face suggested
that the individual was still bound in response to the earlier
disturbing message. Of course, other persons in the interplay often
assisted the individual in the exercise of his barely permissible act of
concealment by tactfully not directing questions to him until they felt
his voice could handle it without showing emotion.

\hypertarget{on-control-of-taking}{%
\section{On Control of Taking}\label{on-control-of-taking}}

In Dixon one of the most dramatic and consistent ways in which persons
were required to show that they were in control of themselves was in the
acceptance of indulgences. When one person accepted anything that was
gratifying while in the presence of others, and especially when the
means of indulgence could be considered limited in the sense that others
present might be correspondingly deprived, then a preliminary refusal of
the indulgence or a request that it be lessened was almost always given
the participants in the interplay. On the rare occasions when this
preliminary refusal was not forthcoming, or when it was too obviously in
contradiction with the expressive behavior of the individual, or when
acceptance and eagerness were not made a joke of, then dysphoria
followed.

Interplay during meals was perhaps the time when self-control regarding
desires was most consistently expressed. When being served food, the
recipient conveyed the fact that the serving was ample by the use of
such stock phrases as, ``That's any amount.'' Whens second helping was
offered, as it invariably was, the recipient would either refuse and
wait for at least a second round of requests, or positively refuse, or
qualify an acceptance by very standard phrases such as ``just a peerie
corn, please,'' or first ask if all present had had enough. On many
occasions the hostess, after a meal, would ask if anyone wanted any
biscuits with their tea, obtain a ``no'' from everyone, then bring
biscuits out, which were then eaten by everyone. At tea-time during
socials, when persons went around the ball with wide trays of biscuits,
buns, and sandwiches, it was felt proper to refill one's plate as
frequently as desired but was felt improper to have more than three
pieces of food an one's plate at a time or eagerly to seek service
before the person with the tray had come within a few feet of one. It
should be added that it was necessary to do more than merely follow the
verbal forms of preliminary refusal; if a discrepancy was obvious
between the linguistic component of the trial refusal and the eater's
expressive behavior, then he was felt to be in some way a faulty
participant.

\begin{quote}
Mealtime in the hotel kitchen. Mr.~Tate feels he has gotten more than
his share of apple tart and more than he desires. He says to the hotel
maid, ``Here, Alice, take some.`` He cuts off a third of his tart,
preparatory to passing it. Alice remonstrates, ``No, maybe Jean {[}the
other maid{]} wants some.'' In saying this, however, her eyes are fixed
on the tart and her tone is abstracted and unconvincing. Jean refuses
any more tart, and everyone at table feels a little embarrassed at the
sight of uninhibited desire.
\end{quote}

\noindent Similarly, when a person was chosen as next in turn to play billiards,
and was aware he had a right to his turn by the system of rotation, he
would almost always offer a mild disclaimer.

In general, the please or request intonation which preceded any verbal
request seemed to serve not merely the purpose of conveying the fact
that the other was not being ordered or presumed upon, but also that the
person making the request was not completely bound by the indulgence he
was requesting.

During many economic transactions on the island, an effort was made to
demonstrate that an affection for money, though understood to be strong,
was not overwhelming. In the hotel, the maids would share their tips
with the kitchen staff and would do this with a gesture indicating that
a tip we not something to conceal from other workers out of greed. So,
too, the managers of the hotel always seemed to find it a little
difficult to take payment from the hotel guests; of their own accord
they would reduce to an even sum the bills of younger guests and would
not charge for extra meals that guests were sometimes forced to take
because of a delay in transportation service. Similarly, when someone
not a neighbor, or friend, or relative was hired for a day's work, there
would be no bickering over payment, and the hirer would always try to
add something extra to the payment. Again, when islanders sold dairy
products to outsiders, or took in their laundry, a round sum was usually
charged for the service, the server tending to make some voluntary
sacrifices (whenever necessary) in order to do so. So, too, the bus
driver would go a little out of his way for a passenger and feel that it
was, in a sense, beneath him to charge for the small extra cost of this
service to him. And when islanders came down to the pier to buy fish
from the two local boats, the skippers would feel awkward about having
to fix a price and would set some low round figure. So, too, when one
crofter gave another a haircut (there are no barbers on the island and
the islanders scrupulously adhere to the maritime tradition of neat
haircuts), the temporary barber might accept a package of cigarettes but
no money. And, similarly, if someone obliged a neighbor and slaughtered
a sheep (technically illegal), the actor would be given a meal, or a
package of cigarettes, or a drink, not money.

Control was also exerted in accepting small ceremonial indulgences from
others. When a bag of sweets would be offered, only one piece would be
taken at a time, and never more than three or four pieces altogether. At
parties and weddings, when the host would take his bottle and shot glass
and go from one guest to another giving each a drink, the men would
drink the whole shot glass the first round but on successive rounds
later in the evening only a part of the glass would be drunk.

It should also be added that Dixonites made an effort, in undertaking
any pleasurable activity, to show that they were not too eager to do so.
Thus, persons would usually come slowly to the table for a meal. When
seated to play ``500'' or another game, they would not rush into the
game with passion but allow a few minutes for general talk as a kind of
warm-up. If a man came too early for billiards, or attempted to hasten
the beginning of a game, he was lightly chided for being over-concerned.
In drinking any alcoholic beverage, men invariably preceded each gulp
with a slight pause during which the drinkers would look each other in
the eye and say ``cheers;'' to take a drink without this ceremonial
recognition of the others present would express, among other things,
overeagerness to drink. A man approaching a girl at a community dance in
quest of a dance would tend either to walk slowly or to run with openly
feigned eagerness.

\hypertarget{on-control-of-keeping}{%
\section{On Control of Keeping}\label{on-control-of-keeping}}

Those who possessed supplies of indulgences tended to show (and exert)
control over selfish enjoyment of them.

When neighbors dropped in during the day or evening, as often occurred,
the offer of a cup of tea was the minimum required gesture of
friendliness. No household crisis could excuse the hosts from this
offering. Failure to make the offer wouldn't only be considered a
discourtesy but would also show that the household was itself operating
under too much economic constraint. Similarly, few meals are prepared
but that extra fish or potatoes are included, so that second helpings
can be pressed on each participant and so that no one will have to
decide whether or not to take the last piece. (A woman who counts the
potatoes she boils for dinner, allowing a fixed and limited number each
participant, is considered mean and is gossiped about.) If a container
of bought food, such as beets, pilchards, or corned beef, were wholly
consumed, then the hostess would almost always offer to open another.
Interestingly enough, when men are alone together on a job of work,
lambing or casting peats, for example, one among them will take on the
role of housewife for the duration of a meal and will make sure that
extra cups of tea are pressed on everyone. One or two men will have
thought to bring milk for the tea and as a matter of course will pass it
around to everyone.

As in the rest of Britain, biscuits and candies---which islanders of all
ages loved dearly---were strictly rationed. Each person thus had a
supply of indulgences to do with as he pleased; he could consume them
himself, or give them to others as expressions of friendship and
respect, or use them as a means of ingratiation. Rarely is an adult seen
openly consuming self-purchased sweets but that the consumer offers the
perceiver a share. Persons who wanted to consume sweets or cigarettes in
the presence of many persons, e.g., at an auction, frequently employed
the strategy of limiting the offer to those closest, as a kind of
adaptive compromise. And while islanders would furtively pop a candy
into their mouths when they felt they would be unobserved, it seemed
that most islanders used the greater part of their sweet ration for
ceremonial purposes, as a means of communicating involvement in others
and control over private passions. So, too, in the fields around the
community hall during a social, men cache bottles of liquor which they
are forbidden by law and custom from bringing into the hall, and
throughout the night each owner of a bottle brings knots of men out with
him to have a drink. In a place where liquor is costly, difficult to
obtain, and dearly loved, the passing around of one's bottle is not only
a way of conferring high esteem upon the recipient but is also a genuine
act of self-control, showing a manly capacity to hold off one's thirst
and recognize the social amenities. Cigarettes, which are extremely
costly on the island,\footnote{At over fifty cents a package, annual
  expenditure by some crofters on cigarettes is greater than the annual
  rent they pay for their cottage and land.} are similarly used as part
of the island's sign equipment---part of its ceremonial language. At
parties, the host will pass around a box-full. A dinner guest will show
his respect for his hosts by elaborately offering cigarettes to everyone
present at the end of the meal. On the most routine work occasions, a
person taking out a cigarette will make at least a gesture of offering
one to his nearest fellow-worker. And each time this ceremonial language
was used, the speaker indicated to those around him, partly by the
patterned equanimity with which the offer was made, that his poise could
not be threatened by the passage of a valuable from himself to another.

In the last two sections it was suggested that persons exercise
self-control in accepting things from others and that persons exercise
self-control in the attachment which they express to things they already
have. It is apparent that if each person in a two-person interplay is to
demonstrate both of these kinds of self-control, and if neither
participant is to sacrifice or fail to obtain what he dearly desires and
feels properly his due, then a kind of tacit cooperation will be
required between the participants. Each will have to act in such a way
as to make it possible for the other to show generosity without losing
too much by it.

\newpage For example, when a person pays a visit to a friend, it is expected that
he will volunteer to leave before he really wants to and before he
thinks his hosts really want him to leave. It is expected that his hosts
will remonstrate and coax him to stay.\footnote{The social mechanism
  whereby both parties to an exchange feign willingness to accept
  deprivation seems to be quite generally found in societies. See, for
  example, Raymond Firth, \emph{We, The Tikopia} (London: Allen and
  Unwin, 1936), in describing courtesy patterns, p.~310: ``Night comes
  on. The man, out of politeness, 'to make his face-good,' makes a show
  of going, but is pressed to stay and sleep with the family. He does
  so.''} When a person sells something small to a friend, it is expected
that the seller will place a lower price on the article than the buyer
is willing to countenance, and that the buyer will place a higher price
than the seller thinks is fair. There regularly follows a process of
reverse bargaining, with the parties reaching about the same selling
point as they would have under normal bargaining procedures. Both
individuals show that they have not been petty and yet lose little by
showing it. When one commoner hires another by the day for his special
skill as painter, m.son, or cabinet maker, then after lunch---which the
guest-worker eats with the family he is working for---the worker makes
the first move to get back to work, and the host makes a counteractive
move to prolong the mealtime conversation with a second or third cup of
tea. Some additional everyday illustrations may be given:

\begin{quote}
Three men are helping William Croseley dig his garden. Lunch time
approaches. \\
Croseley: ``Well, that should do it for now, let 's get some lunch.''
(He puts aside his spade and starts wiping his rubber boots on the
grass.) \\
The workers continue for a moment longer, showing no haste to finish.
Croseley: ``Come on now.'' The workers put aside their spades and start
wiping their feet on the grass.\\
Croseley: ``Surely that will be enough. (He has waited to say this for a
moment, but not long enough for the men to have cleaned their boots.) \\
The workers keep wiping their feet for a few moments after they feel
they have them clean enough.\\
(Everyone feels that everyone else has behaved properly; no dysphoria is
felt.)
\end{quote}

\begin{quote}
There are four men in the billiard room and all are engaged in playing a
game. One of the men is Andy Dawson, the caretaker of the hall, who,
properly speaking, ought to be taking care of the room, not playing in
it. Ted Allen, a steady player, comes in.\\
Dawson: ``Here, you go ahead, I've played enough.'' (Makes gestures of
quitting.)\\
Allen: ``No, no, finish the game, Andy.''\\
Dawson: ``Here, boy.''\\
Allen takes up the cue.
\end{quote}

\begin{quote}
A young man is taking his guest home on a wet night by motorcycle. It is
agreed that the guest will walk from the turn of the road, a mile away
from the host's house and half a mile away from the guest's house.\\
Guest, a couple of hundred yards from the turn in the road: ``This will
do nicely. You go on home now.''\\
Host: ``Don't be daft, boy, it's terrible wet.'' He drives on until the
bend is reached.\\
Guest: ``Let me down, boy.''\\
Host: ``I'll just make the turn up here a bit.'' He drives on for
another quarter mile before dropping his guest.
\end{quote}

It has been suggested that when two persons compete over some matter
each may ``lean over backwards'' in an effort to show that he is not
overly involved in the issue. In the case of indulgences, information as
to which of the two is the less involved in the indulgence can be
reserved for transmission in the second round of offers and refusals,
the first round being devoted to showing that neither person is too much
concerned with the indulgence. The difficulty in this ``after you,
Alphonse`` interchange is that participants sometimes are unsure as to
how many circuits of offer and denial must be made before valid
information about the other is forthcoming. Each participant comes to
feel that he ought to take into account the fact that the other is
merely being polite and so waits for an extra round of offers or
denials. There is a degradation of the meaning of refusals or offers,
and the communication circuit ceases to be useful for the passage of
information. To use a term from communication engineering, a kind of
``hunting'' occurs. Thus, when one woman on the island wanted to find
out if a guest really did want some more food, she found it expedient to
break into the circle of offers and denials, repeated offers and
repeated refusals, by grabbing the guest, changing the mood of the
interchange, looking deep into his eyes and saying in a serious tone,
``You're not just being polite, are you?''

% CHAPTER XXII: ON PROJECTED SELVES
\chapter[CHAPTER XXII: ON PROJECTED SELVES]{Chapter XXII: On Projected Selves}
\label{ch:Chapter XXII: On Projected Selves}
\chaptermark{CHAPTER XXII: ON PROJECTED SELVES}

\newthought{Throughout this study it}
\marginnote{\href{https://doi.org/10.32376/3f8575cb.1d9cdcca}{doi}}has been suggested that when islanders
participate together in an interplay, countless events become available
for aptly expressing the attitudes of the participants, especially the
attitudes they have towards themselves and towards fellow-participants.
With every word and gesture, a participant can convey his conception of
himself and his conception of the others present, and every one of his
words and gestures may be taken by others as an expression of these
conceptions. The individual may, of course, attempt to conceal this
expression or actually may not (even unconsciously) make use of
opportunities for it, but in any case the others will assume that his
behavior expresses his valuation of himself and them. It will therefore
be advisable for the individual to take account of the possible
interpretations that might be placed upon his behavior, regardless of
which, if any, interpretation he thinks is correct.

When persons come together for purposes of interplay, each brings
expectations as to the rights and obligations he will enjoy, and, by
implication, a conception of himself which he expects the interplay will
sustain. He also brings a familiarity with the treatment that ought to
be accorded certain categories of persons and sufficient familiarity
with symbols of status to hurriedly place those he meets into such
social categories. And if the participants happen to know, or know of,
one another, then, as Bales suggests, each participant may become, for
the others, someone whose ``\ldots{} past actions and identity are
remembered, including what he `has done' prior to his entrance into the
group and what he 'is' outside the present in-group, and are attributed
to him in the present as a part of his total significance.''\footnote{Robert
  F. Bales, \emph{Interaction Process Analysis}, p.~71.} In other words,
each participant brings to the interplay a preliminary state of social
information.

At the moment of coming together, each participant---by his initial
conduct and appearance---is felt by others to ``project'' a self into
the situation. Given the state of social information and given the
availability of countless events for conveying expression, it seems
inevitable that even inaction on the part of an individual will be taken
by others as a positive t on his part to say something. The participant
may be non-committal and indefinite; he may be passive, and he may act
unwittingly. None the less, others will feel that he has projected into
the situation an assumption as to how he ought to be treated and hence,
by implication, a conception of himself. If this project did not
occur---if this initial social identification did not take place---then
the participants could not begin to act in an orderly way to one
another. As Simmel suggests, ``The first condition of having to deal
with somebody at all is to know with \emph{whom} one has to
deal.''\footnote{Simmel, \emph{op. cit.}, p.~307.}

In the ordinary course of events, it would seem that the selves
projected into an interplay provide a significant part of the initial
definition of the situation, for it is by these selves that each
participant knows what to expect from others and what is expected from
him. These projected selves provide the guide lines for action,
determining important aspects of the working acceptance that is sooner
or later achieved. Each person's projected self gives the other
something to go by. Whether participants accept the projected self of
another, or whether they tactfully attempt to bring it into line with
their conception of him, they are likely to use it as a starting point
and as a basis of orientation in their treatment of him. If the
interaction is not to be dysphoric, then, apparently, the self that an
individual presents to or projects into the situation must be
sufficiently familiar and acceptable to the others not to disturb their
unthinking involvement in the interplay.

The selves that are initially projected into the situation, and the
expectations associated with them, become, then, a basic premise of what
is to follow in the interplay. The activity that does follow is, in a
sense, merely an elaboration and controlled modification of the
initially accepted status quo. It would seem, then, that interplay is an
inherently conservative thing, and that all participants have a vested
interest in maintaining the validity of the initial understanding, for
if communications are intimately based upon an initial definition of the
situation, then any contradiction of this definition is likely to leave
the participants up in the air, lodged in roles and in conversation no
longer supported by a definition of the situation. If the interplay is
not to be brought to a confused and embarrassed halt, then the guiding
assumptions provided by the initially projected selves must not be
fundamentally altered or discredited, regardless of how the participants
actually feel about the assumptions they have temporarily accepted. If
the minute social system formed by persons during interplay is to be
maintained, the definition of the situation must not be destroyed.

\newpage The presence of potential disruptions to the working acceptance, and the
constant necessity of avoiding or side-stepping these difficulties, or,
if they occur, of correcting or compensating for them, seem to be
crucial conditions under which participants must operate. (While it is
true that in many interplays a particular participant will formally or
informally take on the responsibility of seeing that peace and order are
maintained, still it can be said that all the others present are sworn
in as deputies.) These crucial conditions seem to provide a very useful
perspective from which to classify and analyze interplay behavior,
leading us to bring together into one type, behaviors which bear the
same relation to the contingencies of maintaining a given definition of
the situation. While a treatment of interplay behavior based an this
point of reference is implicit in some of the previous parts of this
study, an explicit effort along these lines will be made in this
chapter.

\hypertarget{serious-disruptions}{%
\section{Serious Disruptions}\label{serious-disruptions}}

It was suggested that when an individual enters interplay he does so in
a particular capacity; whether he is aware of it or not, others feel he
has presented himself in a certain guise or light, making certain
demands, willing to satisfy certain others, and in general anticipating
that a valuation of a given kind will be placed upon him.

Events may occur during an interplay which provide information about a
particular participant that is patently incompatible with the
information that has been accepted or assumed concerning him. A self he
has openly accepted (before himself and others) as having, he proves not
to have; his projected self is discredited. And since his initially
projected self served to guide the interplay---and was meant to go on
doing so---the interplay itself becomes disordered. Two types of
discreditings resulting in dysphoria will be considered: ``gaffes'' and
``pretensions.''

1. A gaffe may be defined as any event which precipitously and
involuntarily discredits a projected self that has been acceptably
integrated into a definition of the situation. A gaffe may be produced
by the very participant whose projected self the gaffe embarrasses, or
by another participant, or by an agency other than the participants.

In Dixon, anxiety over the possibility of committing a gaffe in
interplay is often present. People in Dixon have fantasies of terrible
gaffes occurring, these fantasies presumably serving to reinforce rules
regarding proper conduct. Thus, in a favorite concert play, a
``pesceet'' {[}``stuck up''{]} outsider is portrayed as examining a
crofter's cottage for cleanliness and remarking that there is
superfluous soot on the ceiling; she places improper syllabic stress on
the word ``superfluous,`` showing that she does not have the education
that use of a long word implies (and that the audience watching the play
by implication does). The outsider's projection of a superior self is
thereby punctured.

Occasions when a gaffe almost occurred are nervously talked about for a
brief time after the occasion.

\begin{quote}
For a day after its occurrence, Dr.~Wren tells of having gone into the
hotel kitchen and upon seeing a new girl there, almost taking her for
Mabel Crown, Mrs.~Tate's niece, who was scheduled to come to help out
for a few days. The young woman in the kitchen was actually someone from
another island, hired for a few weeks as a replacement. Dr.~Wren had
been acting toward the Tates in the manner of someone who would
obviously know which person ought to be associated with the name of
Mabel Crown, and had he called the wrong girl Mabel, it would have been
difficult to sustain this manner. Ten minutes after almost misnaming the
girl, Dr.~Wren came into the den where he and other guests were eating.
and said in a tone of mixed relief, wonderment, and humor: ``I almost
made a terrible faux pas; I thought the girl in the kitchen was Mabel
Crown.''
\end{quote}

\noindent Occasions when a gaffe has occurred seem to become cautionary tales and
are retold for years as a source of humor and as a means of ensuring
involvement of participants---and perhaps as a means of playing out a
realization of anxieties. Some examples may be given.

\begin{quote}
The harbormaster, Jimmy Andrews, is recounting experiences he used to
have in his drinking days when the county inspector would arrive
unannounced to check up on Andrew's devotion to duty:\\
``I mind the time there were a good taw o three boats at the pier and I
was sittin at home in me underwear and old pants. And the inspector he
comes up in a taxi and comes to the door and asks for Jimmy Andrews. So
I say, `He's at the pier, I expect.' And the driver shouts out, `Why
there's the man himself.' I tell you I almost got the can that time.''
\end{quote}

\begin{quote}
On the island, as in Bergand in general, there is a tendency for the
task of any one person to be defined as something any other person who
happens to be near ought to lend a hand with. Also, one's body is
defined as something that may be crowded next to another's in a lorry or
in the cabin of a small boat. Congruent with this pattern, it is
customary for the person serving food to help the person being served to
a degree not sanctioned in the British middle classes. In the hotel
dining room, however, the hotel staff attempts to maintain a
middle-class definition of the situation, serving food not ordinarily
eaten by crofters and stressing individual portions: individual butter
balls are served instead of a single slab of butter, individual jam
tarts are served instead of a single pie cut into segments; milk and
sugar are served along with the tea, giving each guest an opportunity to
express individual taste and self-determination, whereas crofters
ordinarily put milk in all cups before serving tea. The scullery boy
tells of the time he was pressed into service as a waiter and put sugar
into the tea of one of the guests and mixed it himself.
\end{quote}

\begin{quote}
One Sunday afternoon in the hotel kitchen Mrs.~Tate is reminiscing about
previous ministers:\\
``We had this minister who was oh so fiery. He used to preach with his
arms waiving around in the air trying to save the folk. And he used to
read his sermon from sheets. One time I was sitting in the front and I
saw him wave with one hand and turn over the page with the other. I kint
then that he was just puttin it on.''
\end{quote}

Some examples are given below of gaffes that occurred during the
research.

\begin{quote}
During her first few months on the island, the new doctor's wife,
Mrs.~Wren, was asked to join the Women's Rural Institute and to grace
the organization's semi-annual flower show, awarding prizes for the
winners in the several competitions. Being in favor of lower-middle
class pursuits for the commoners, she consented. In accordance with the
established pattern for these matters, a member of the organization who
had a good command of standard English read off the name of each winner,
and the character of her prize, and then passed the prize to the current
president of the organization. The president would then pass the prize
to the guest of honor---in this case Mrs.~Wren---and she would pass it
to the winner, who by then would have come up to the front of the hall
in order to receive it. As each winner came up to the front of the hall,
Mrs.~Wren, following what was expected of her, would smile to the winner
in a manner suggesting that she knew the winner by name, and would
congratulate her. Since each winner would have to first rise from her
chair, and then walk up to the front of the hall, before receiving her
prize, it was possible for Mrs.~Wren to spot the person to whom she was
going to have to smile graciously before the person had come close, and
in this way an illusion could be given that the winner was actually
known to Mrs.~Wren, and that the greeting was a spontaneous consequence
of interaction with the winner. One prize, however, was won by the
president of the organization, with whom Mrs.~Wren, up to that moment,
had been carrying on what appeared to be very friendly and informal
intercourse. Not knowing the name of the president, Mrs.~Wren got her
smile ready and looked into the audience to find the person she was to
direct it upon. The president tried to save the situation by tugging at
Mrs.~Wren's arm, but before she could do this everyone was given a
glimpse into the fact that the friendliness and familiarity that
Mrs.~Wren had been showing to the president and to each successive
winner was to some extent merely a show. A painful moment of
embarrassment followed.
\end{quote}

\begin{quote}
Mr.~and Mrs.~Tate are away for the evening and the staff is in the hotel
kitchen. An elderly male guest knocks at the kitchen door.\\
Guest: ``Can I have a cup of tea, newly infused and hot, and a piece of
ginger cake.''\\
Jean: ``Yes.'' (She projects a customary tone of accommodative
obedience.)\\
The guest then closes the door. The staff has been courteous but feels
that the old man is over-demanding as well as foolish. They burst out
laughing at him when they see that the door is closed, and someone
mimics the guest. The guest pops his head back into the kitchen; he has
a look of having heard and having understood. He says, ``I see you are
all happy tonight.'' The staff becomes completely flustered. The guest's
tea is delivered in pained silence. By the next day the staff can retell
the incident as a joke.
\end{quote}

\begin{quote}
During billiards one evening twelve persons appear, this being several
more than usual and necessarily lowering the total number of games
played by each person during the evening. Two men, Tom Clark and Kenneth
Burns, both keen lovers of the game, have played two games each and
neither has played for three quarters of an hour. Three players who have
only played one game apiece and who are persons other players like to
play with are about to begin a game. By rules of fairness, either Clark
or Burns ought to be the fourth to complete the match. As customary,
Clark and Burns each claims that he does not particularly want to play
and that the other should go ahead. Three circuits of offers and
counteroffers are made by the two men, so that at last the others
present are almost convinced that Clark and Burns really don't want to
play. Burns finally decides that it will now be safe to accept and picks
up a cue in readiness for play. Most of the others present see this
action and it is assumed that Burns has now become the fourth for the
game and that Clark was not interested in playing or was too polite to
play. However Clark apparently does not see Burns' silent act that
defined the situation and, picking up a cue, he takes on the air of
someone entering the spirit of a game, of someone ``talking it up,'' and
he says jokingly, ``Well, Jimmy {[}the player he expects to be partnered
with{]}, let's show them.'' Immediately Clark sees that the situation
has been defined with him as a member of the audience, not as a player;
he loses countenance and smothers his act as quickly as possible,
stepping back from the table and out of the view of most of those
present.
\end{quote}

\begin{quote}
In the hotel kitchen during staff lunch, talk turns to the fact that
writers and motion picture people always come to Bergand in quest of the
most newsworthy lore, i.e., romantic backward peasant customs, and that
a false picture of the islands has consequently been created. {[}The
complaint seems quite justified.{]} Mr.~Tate, especially, shows great
antagonism to these practices, to the point where the maids and the
scullery boy feel he is carrying things too far. Finally Mr.~Tate says,
``How many folk have running water and electricity even though they have
to make their own water.'' This seems to discredit the standards of
propriety that have been assumed in the interplay, albeit the
discrediting was patently accidental, and one of the maids and the
scullery boy drop their eyes and bend their heads downward in an effort
to stop from bursting out laughing.
\end{quote}

\begin{quote}
During socials the practice is sometimes followed of announcing
prizewinners and performers by formal naming, e.g., ``The second prize
has been won by Mr.~John Smith'' (or Mrs.~or Miss Smith). This custom is
especially followed during occasions run by the Women's Rural Institute
and during prize-giving at flower and produce shows, for it is at these
times that members of the community most selfconsciously practice
middle-class roles. Along with formal naming, little speeches of
acceptance are given in standard English, and everyone, of course, is
dressed ``well.'' Perhaps the chief difference between the kind of
middle-class show put on at these times and similar shows that occur in
British urban centers is that except for a few outsiders who
occasionally attend, all persons present will have previously interacted
with one another on the basis of first-naming, work clothes, the Bergand
dialect, and crofter tasks, and will do so again when the social
occasion is terminated. At one social the master of ceremonies was (as
was often the case) Tom Clark, a clerk in the Allens' shop, a young man
of crofting origins who is already widely accepted as a community leader
and is a central figure in the rising middle class. When the time came
to announce the winners of the flower show he left a knot of friends,
mounted the stage, and successfully called out the first two winners,
who were women, by their formal names. Their dress, his dress, and the
manner of all of them properly sustained the air of middle-class
respectability that these competitions always project into the
situation. The third winner was Jimmy-Andrew Simon, a commoner employed
as a baker in the Allen Bakery. Tom Clark and Jimmy-Andrew Simon are
neighbors, work in the same building, and are great friends. Clark, like
almost all the commoners on the island, calls Simon by his double name.
Simon, who was in the knot of friends that Clark had left when he went
up to the stage, had worn a formal dark blue suit and was ready to
appear on the stage with middle-class dignity. When the time came for
his formal name to be announced, Clark could not think of it; he knew
who had won the third prize and where the winner was standing but he
could not think of the winner as other than Jimmy-Andrew. It was
impossible for him to say ``Mr.~Simon.'' After a confused pause, Clark
finally announced in a stutter, ``Jimmy-Andrew Simon.`` A few minutes
later, when Clark returned to his knot of friends, his face was still
red from embarrassment, and he said, ``I was never so embarrassed,
Jimmy-Andrew, I just could not think of thy name.''
\end{quote}

\begin{quote}
Mrs.~Tate has been testing the staff on a mathematical puzzle printed in
the newspaper, introducing a kind of competition in which she is almost
certain to excel and in which the cook is almost certain to fail. He
does not succeed in solving the problem and Mrs.~Tate says, ``You're not
very good in mathumatics {[}\emph{sic}{]}, are you?'' She does not
notice that she has mispronounced ``mathematics'' and that this
mispronunciation belies her assumed familiarity with the discipline. The
two maids look at each other collusively behind Mrs.~Tate's back,
furtively conveying a mocking smile to each other.
\end{quote}

\begin{quote}
The minister of the established Church in Dixon is a man of humble birth
from the mainland of Britain. University training has not covered his
``common'' accent. As is the pattern in Britain, he is given a kind of
ceremonial rank of equality by the gentry; he is invited to their larger
and more official gatherings. However, for the gentry he is a faulty
person; the person they must treat him as is too far removed from the
person they really think he is. He is given to drink and Mrs.~Wren has
whispered jokingly to her friends that he smells a little for want of a
bath. When not in his presence, the gentry use a nickname for referring
to him, taking the first syllable of his last name. Sometimes of a
Sunday he would come to the hotel for dinner, which he would take with
the Wrens. At these times, the gentry would begin by treating him
politely but often end the meal by baiting him and almost treating him
in an unserious way. Attempts on his part to sanction them for not
attending church and thereby maintain some kind of hold over them would
not meet with polite apologies but with clear counter-rebukes,
expressing the fact that it was not his place to tell them anything. On
one occasion, conversation turned to a humorous matter on which the four
persons at table (the minister, the Wrens, and the writer) could equally
join. Things became merrier and merrier, with everyone accepting the
self projected by each of the others. Suddenly the minister got carried
away by a joke---carried away a little more than is defined as proper at
a middle-class table---and leaned over and lightly slapped Mrs.~Wren's
back, a slap of goodfellowship. As the blow of familiarity fell, he and
the others present realized that the minister's earthier past had
presented itself, to the embarrassment of his present self. He withdrew
his hand limply, attempting, and failing, to maintain a note of
spontaneous involvement, then settling back into customary discomfort
for the remainder of the meal.
\end{quote}

\noindent A common strategy by which individuals dealt with gaffes was suddenly to
define the whole situation as unserious and burst into mirth. This
seemed to be a way of suddenly introducing new projected selves into the
situation, so that it would be possible to treat the discredited ones as
a joking matter and still have something to build interaction
upon.\footnote{Bergson, in a well-known contribution to the theory of
  laughter---\emph{Laughter} (London: Macmillan, 1911)---suggests that
  we laugh when a person behaves as if he were a mechanical object.
  Freud, in another well-known contribution to an understanding of
  laughter---``Wit and its Relation to the Unconscious,'' reprinted in
  \emph{The Basic Writings of Sigmund Freud} (New York: Random
  House-Modem Library, 1938)---suggests that we laugh when the
  occurrence of an event obviates the necessity of suppressing our
  inclination to have the event occur. From the point of view of this
  study, both theories seem to be saying the same thing. In the first
  case, we have an individual who presents himself as someone who is a
  person and then discredits this projected self by behaving like an
  object. In the second case we have an individual who presents himself
  as a person of a given moral status and then inadvertently.} Frequently this line of adjustment would be
initiated by the person who had made the gaffe, especially if he had
made it against himself. Only certain gaffes, of course, could be
handled in this way. Some examples of the use, successful or
unsuccessful, of this strategy may be given.

\begin{quote}
It is evening in the hotel kitchen and the managers, the Tates, are
away. The maids are polishing the guests' shoes and the cook is sweeping
the kitchen. The maids have been at the hotel all winter but the cook
just started his summer's employments month ago. The maids have been
friends since childhood and are on swearing terms with each other, but
taboos regarding such matters have not yet (as they will come to be)
broken down with the male members of the kitchen staff. Alice drops some
polish, gets angry, forgets herself, and says ``fuck'' out loud. The
relation of intimacy signified by premising to-use this word has not yet
been established and socially speaking there is no place for the word to
fall. There is a hushed moment in the kitchen, and then Alice bursts out
laughing. Jean, the other maid, blushes deeply, looks at Alice, and then
looks down.
\end{quote}
\newpage
\begin{quote}
Two\marginnote{shows that
  this is not the case. No doubt part of the laughter that is found in
  such situations arises from the fact that maintenance of a role
  requires a degree of nervous tension and that sudden breaking of the
  role acts as a release for this tension---hence the characteristically
  ``nervous'' quality of some laughter. However it would seem that
  sometimes laughter in these situations represents an effort to
  assimilate the self that has been discredited to an unserious self,
  one whose discrediting is of little moment. The two roles of laughter
  are frequently separated in situations where the self that has been
  discredited is too important to be assimilated to an unserious one. In
  such cases, nervous laughter on the part of participants may be
  rigorously repressed and only conveyed by means or collusive looks or
  by waiting for the offender to leave first. The position could be
  taken that nervous or spontaneous laughter was a means of saying that
  the whole situation, and not merely the self of the offender, ought to
  be defined as unserious} men are on a Sunday visit to the home of a third. Their host is
returning from across a loch where he has gone to see how his lambs are
progressing. He pulls his boat partway up on the shore and looks for a
rock to lay on top of the painter. His guests, thirty feet away, watch
him looking for such a rock. He finds a large one, weighing about
seventy pounds. It is expected that some strain will be expressed as the
man, who is of slight build, leans down to pick up the rock. A self
under strain is projected for him by the pair who watch him. Instead he
lifts the rock up with no apparent strain whatsoever.Both watchers
simultaneously and involuntarily look at each other and laugh. While
they knew that the man lifting the rock was reputed to be one of the
strongest men of the island, they had still projected normal
expectations as to how he would appear.
\end{quote}

\begin{quote}
On Wednesday night at eight there is to be a community concert at which
John Landor, the local orator, is to give one of his famous
extemporaneous speeches. As he is wont to say, he merely gets up on the
platform and says whatever comes into his mind. {[}He has, incidentally,
great stage presence and can handle an audience in a very professional
way.{]} So well known are these speeches that the name he uses on the
stage is a name often given him off the stage. On the morning after the
concert Alice Simon, Landor's niece, tells the following story to a few
friends gathered in the hotel kitchen: ``Last night I was walkin up the
road past Lakeview {[}her house{]} about six o'clock and there was
Johnny walkin ahead of me, not seein me, givin his speech into the
night. Bairns, I thought I'd die.''
\end{quote}

\begin{quote}
In the temporary sleeping quarters in the barn behind the hotel the
scullery boy is napping. It is late on his afternoon off. Mrs.~Tate has
to ask him something and wakes him up. Apparently he has been dreaming,
for he wakes up startled, expecting to find a world quite different from
the one around him. Mrs.~Tate expects to see someone whose face
expresses the fact that he is in an employee relationship to her,
someone ready to engage in the interaction he will find himself in as he
awakens. Instead she momentarily sees, by the look in his eyes, a person
who has been startled out of a more dignified role. She bursts out
laughing and immediately afterward recounts the incident to those in the
kitchen.
\end{quote}

\begin{quote}
At a community concert, Tom Clark is reading the names of raffle
winners, and Ted Allen, in his customary effort to remain out of the
limelight in these matters, is hidden from the audience behind the stage
curtain in the role of curtain-puller. Clark receives a ballet from the
young girl drawing ballots from a barrel and attempts to read the name
on it. The name is badly written, and he fails. In an unthinking effort
to keep the show going, Ted Allen comes from his hiding place and tries
to read it for him. He suddenly realizes that his effort to show that he
is not helping to run the social has been exposed. He turns to the
audience, blushes, and gives the audience a broad smile of admission.
\end{quote}

2. A gaffe has been defined as a sudden involuntary event which patently
discredits a projected self that has already been accepted by others and
built into the interplay. A pretension may be defined as the more or
less voluntary projection of a self which from the very beginning is
unacceptable to others and which continues, for the period during which
the individual is a participant, to inject a false note into the
situation. The pretentious projection is unassimilable in the interplay
because there is too much variance between the role the actor assumes
and what is already known about the actor or what he comes unwittingly
to reveal about himself. As Cooley suggests, ``If we divine a
discrepancy between a man's words and his character, the whole
impression of him becomes broken and painful \ldots''\footnote{Cooley,
  \emph{op. cit.}, p.~350.} Other participants may exercise forbearance,
so that the offender may never realize he has behaved in an impossible
way. Sometimes the offended persons cannot tolerate the discrepancy and
refuse to allow the offender to proceed, leaving him in a position of
blustering.\footnote{A clear example of blustering is given in W. Lloyd
  Warner and J. O. Low, \emph{The Social System of the Modern Factory}
  (New Haven: Yale University Press, 1947), p.~145, with an analysis on
  p.~154. ``Fred Jackson, one of the firm of Jones and Jackson, on the
  other hand, over-participated in the strike with disastrous results to
  himself and interruptions to the negotiations in progress between the
  manufacturers and striking employees. Here is the story as Nixon,
  president of the union, told it to an interviewer (later verified from
  interviews with management): 'One of the manufacturers, Fred Jackson,
  a ``snappy'' young fellow, came into a meeting and slapped a piece of
  paper down in front of me with a list of things Jones and Jackson
  proposed as an independent settlement. Jackson said, ``I'm going to
  make you eat that, Nixon.'' And I said, ``Well, I don't happen to like
  paper, Mr.~Jackson.'' Jackson got very red.and pulled a fifty dollar
  bill out of his pocket and slammed it down on the desk and said, ``You
  cover that, Nixon, and we'll go downstairs in the mayor's office and
  whoever comes out first wins.'' I said, ``Don't be so childish,
  Mr.~Jackson.`` I only had about forty cents in my pocket at the time.
  The story got to New York and Jackson was called down the next day and
  fired.' Jackson damaged the cause of management when he tried to fight
  the head of the union. Everyone said he blustered, and everyone said
  he acted badly when he challenged union leadership. Jackson was under
  the control of higher management and occupied an inferior managerial
  position where he had little freedom to assume command and take
  leadership. Yet he had learned} Examples of
both kinds of situations follow.

\begin{quote}
The laird's house, ``Alexander Hall,'' a historic landmark in Dixon, is
built near the shore of the inlet, and the laird has a stone pier from
which the annual boat races are run and off which the laird moors his
rowboat. A local thirty-foot fishing boat which an old crofter, Henry
Johnson, and his two sons operate during the summer months is usually
moored between this pier and the main Dixon pier, some three hundred
yards away. The Johnsons decided to moor their boat closer inshore this
summer, hence closer than usual to the laird's pier. A few nights after
they moored their boat where they wanted to, Henry Johnson, somewhat in
his cups at a social, told the following story---a story told and retold
many times since then.\\
``We put the boat there and the other night I'm walking up to the shop
and the laird stops me and says, `Henry, you've got your boat in the
place that you know has always traditionally been the mooring place for
Alexander Hall. Would you move it, please.' {[}Mr.~Alexander still has a
little of the manner of a laird even though he now has little land left,
and little power over the land he possesses. Traditionally it would
never have been necessary for an Alexander to raise the question about
mooring rights.{]} So I says to him, `Do you own the rights to that
piece of water; do you have the legal right to make me move my boat?'
Wit dat he got sore and red as a beetroot and says, 'Well, if we can't
discuss it sensibly there's nothing more to be said,' and he stalks off.
He's not back in India ordering niggers around; he can't get away wit
that sort of ting now.''\\
This story is partly confirmed by members the gentry, who say that
Mr.~Alexander had a scene with old Johnson and that Johnson had refused
to remove his boat.
\end{quote}

\begin{quote}
At a concert in Southend a young man from that community, well known
throughout the island, gives a rendition of the song ``Quicksilver.'' He
affects a cowboy manner, wearing no tie and strumming a guitar. He
attempts to carry off an informal manner and an American accent. After
his song he waves his hand and says, ``Cheerio, folks. I'll be back.''
The audience feels that in addition to a song, the performer was trying\marginnote{from William Pierce when he worked for
  him how his kind of man should act, and he knew that an owner and
  manager should assume control. It seems a reasonable hypothesis that
  the conflict between his beliefs about how a man should act (how
  Mr.~Pierce would do it) and what he was remitted to do by his status
  greatly contributed to causing his unfortunate act, an act which
  materially aided the union. He tried to take command in a situation
  where it was impossible, and he could only `bluster.'\,''}
to stage a manner as well. It is felt that whatever he was trying, he
has not carried it off. Many members of the audience feel embarrassed.
Some feel that the performer in question has always put on too many
airs. The applause is relatively light. Later in the concert when he
comes back to lead the audience in a sing-song, they resist and only
fitfully enter the singing. Next morning at the post mortem in the hotel
kitchen Jean Andrews says, ``When he said, I'll be back,' my face just
turned red.''
\end{quote}

\begin{quote}
The minister of the established Church is having trouble getting enough
people to attend church. Many factors are apparently involved. There has
been a long history of collaboration between established clergy and the
gentry in Bergand, and both groups were originally recruited, a few
hundred years ago, from the mainland of Britain and are in some sense
still outsiders. Further, the local gentry have not sufficiently
supported the established Church in Dixon, tending to attend only on
ceremonial occasions such as Christmas, and the church has failed in its
traditional role of being a place of meeting and a place of integration
of commoner and gentry. (The community hall and the Allens' shop are the
real centers of the community.) On some occasions, as few as six persons
attend a Sunday service. The minister is then required to deliver a
sermon that takes its tone and form from an institutionalized mode of
discourse appropriately designed for an audience larger than forty. The
minister finds himself projecting a self that would be appropriate for
an orator to present before a sizable audience. The few persons actually
present sense the discrepancy and feel embarrassed.
\end{quote}

\begin{quote}
A wedding is being held. The groom is a man in his thirties who lost his
first wife through sickness. The bride is a young woman who is the
unmarried mother of a sixteen-year-old boy. Both persons are highly
esteemed in the community. At the ceremony the bride is dressed in full
wedding apparel, of the kind that would do credit to a middle-class
wedding anywhere in Britain. She wears a white dress and veil,
traditional symbols of virginity. For some persons at the wedding this
is a presumption and causes them some embarrassment.
\end{quote}

\begin{quote}
Alice Simon's boy friend, John Neil, is away on an eight-month voyage.
During his absence, they confirm, by correspondence, their intention of
getting married {[}which they have since done{]}. Alice is an attractive
girl, and during John's absence two difficult situations arise.\\
The previous year a man who had come to the island to watch birds, and
who had stayed his two weeks at the hotel, had escorted Alice to the
community hall during evenings, had spent some time in the kitchen when
the Tates were away, and had shown other innocent interest in Alice's
company. The day he arrived for his second annual visit was the day
Alice was scheduled to receive a long distance call from John, and the
staff had been oriented all day toward the coming call and its
implications. Immediately upon his arrival at the hotel, the
bird-watcher came into the kitchen and asked the cook where Alice was
(she was out at the moment). His tone signified an eager expectation
that his relation with Alice would be the same, or more intimate, than
the year before. The staff felt embarrassed. As the cook said the next
day, ``Dus du kin, I was embarrassed, for I do like that boy. I just did
not know what to say.''
\end{quote}

\begin{quote}
The other incident was perhaps more serious. For years a local commoner
had been enamored of Alice and desirous of marrying her, a fact that was
rather widely known. At a time when Alice's relation to John had been
confirmed, the disappointed suitor misread the signals and on the
occasion of a Christmas visit presented Alice with a wrist watch as a
Christmas present. On the island, an investment of that kind is a
ratified symbol of engagement. Alice was forced to refuse the gift,
although she could do nothing about the giver's having publicly
committed himself to the expectation of engagement to her. The
disappointed suitor could do nothing but take himself unseriously for
the remainder of the evening, playing the fool over a matter that was
felt to be too serious for anybody to attempt to resolve in this way. As
the participants later agreed, it was a painful evening.
\end{quote}

\begin{quote}
An engineering company has sent a man to direct test drilling on the
island for chromate ore. He is a very hard worker, his wife is very much
liked, and he has no ``side.'' However, from the islander's point of
view he is a very faulty person. Time after time during informal
interplay he will immediately charge in with a recital of how the
drilling is going. He projects an assumption that those present are
aware of the state of drilling reached the previous day, of the
vocabulary of drilling, and of the contingencies of the job. Talk that
would be meaningful and perhaps interesting to his crew, were they
already engaged in shoptalk, he employs as a first message with persons
who know nothing about the job. He gives a constant impression of
presumptuous self-concern. The islanders handle the situation by
tactfully attempting to act as if they are interested, answering his
statements with terminal echos such as ``Yea, yea.'' They felt that he
felt they ought to be interested in the development of the island's
resources, but they were shocked at his undue preoccupation with his own
task.
\end{quote}

\begin{quote}
John Adamson, a man of about forty-two, is a regular billiards player.
He is not a member of the rising middle class to the degree that almost
all the other players are, nor does he associate informally with the
other players at other times to the degree that the others do. When
non-players learn that Adamson is a regular player they sometime ask,
``What's he doing there?'' Whatever his position, the regular players
feel that he shows too much eagerness in play, and while other players
are also guilty of this offense he, perhaps more than others, attempts
to maintain a show of not being overly-involved. The impression his
expressive behavior gives is inconsistent with his linguistic behavior,
and the discrepancy causes some tension among the others and brings some
dysphoria to the interplay. Thus, when the question arises as to who'll
play the next game, he follows the polite rules and disclaims any desire
to play, but there is a feeling that he is patently insincere and that
he too willingly allows himself to be pushed into playing next. When he
hits a ball he makes the customary claim that it is a poor shot, but he
keeps on watching the ball until it has come to a dead stop, instead of
expressing unconcern by turning away from the table if the shot is
fairly certain not to score. He loudly disclaims the possibility of
making a shot for which he is known to have sufficient skill, thus
seeming to build up a situation in which credit will come to him. The
gentleman's agreement rule in billiards not to directly sink the
opponent's ball is broken by him before the tension and definition of
the situation has reached a point where this aggressive act is
thoroughly acceptable; or he makes too much of not exercising an
opportunity to sink an opponent's ball, giving the impression that he
has refrained from sinking it merely to demonstrate that he is playing
in the proper spirit. In general, it is felt that close behind the self
he projects of someone who is taking the game in the right spirit is a
self that is too eager.
\end{quote}

\noindent It is to be understood, of course, that the same act on the part of a
particular sender may be quite acceptable to one set of recipients and
yet another set of recipients may feel that the actor has been
pretentious. Thus, during community concerts, a local spinster gives a
solo singing performance that islanders take seriously and think highly
of. By city standards, however, the woman's voice is so bad and her
manner of delivery is so ``old-fashioned'' that visitors to the island
either mistake the performance for a conscious satire or find it
difficult not to laugh. For the outsiders, the woman's full-throated
dramatic rendition is a pretension to, and a presumption of, talent
which outsiders feel she does not possess.

Another example may be cited. While islanders seem to be no more
superstitious than many members of the working classes in British
cities, still there are occasions when adults will discuss with full
seriousness the arguments for and against the existence of ``second
sight,'' that is, the capacity to know in advance that an event will
take place (especially dire events), or to know at a distance that a
given event has taken place. Sometimes the exploits of islanders known
for their capacity in this regard will be cited as positive evidence.
There is among the current adolescents of the community, especially
among some of the boys, a wholly rationalistic orientation to such
matters as ``second sight.'' These persons assume that no full-fledged
adult could hold superstitious beliefs, and in talking to anyone they
seem to talk on the assumption that the person they are speaking to is
not a superstitious person. When the question of supernatural powers is
seriously discussed in the presence of these young people, they often
find it hard to ``keep a straight face'' and behave politely. They give
each other sly, furtive looks conveying their attitude on these matters;
sometimes they cannot trust themselves to do this and carefully cast
their eyes down. For them, an individual who talks in a serious fashion
about supernatural powers is not a person at all, and the failure of the
superstitious speaker\newpage\noindent to realize that he is not behaving as a
full-fledged person is, for the unbelievers, a laughable rigidity and a
self-delusion.

\vspace{.2in}
\begin{centering}

\Large{* * * * *}

\end{centering}
\vspace{.17in}

\noindent During interplay among commoners in Dixon, unacceptable projections seem
to be limited by two factors. First, all residents of the island
possessed a great deal of information about one another, so that it was
quite impractical for individuals to make verbal claims for themselves
which their life apart from the interplay did not support. It would seem
that the more difficult it is for an individual to ``get away'' with a
falsehood about himself, the less likely he is to attempt to do so, and
the less frequently these falsehoods are attempted, the less
opportunity, presumably, there is for gaffes or pretensions to occur.
Secondly, there is a strong tendency for all commoners to define
themselves first and foremost as Bergand crofters and to be ready at any
time to show loyalty to this grouping. Allegiance is shown in the main
by not putting on airs---by not being ``pesceet,'' as the islanders call
it. Since crofters are recognized to be a low and humble group, those
who avow this status have no place to fall.

Perhaps the most frequent kind of unacceptable projection was one
produced by outsiders during interplay with islanders---one that the
producer usually remained unaware of having produced. Islanders as a
whole possessed much information about the physical layout of the
island\footnote{This seems to be rapidly declining today. Fifty years
  ago, when the population was more scattered throughout the island,
  proper names were current for many small landmarks, knolls, hills,
  crags, and inlets. Today even the generic terms for some of these
  identifiable formations are passing out of use.} and about the
administrative routine by which it was operated. And there is current in
both sexes a wide familiarity with croft tools and croft techniques.
Furthermore, as islanders themselves claim, nearly every man has a wide
range of specialized skills, such as carpentry, garage mechanics, and
seamanship. This information and training has come for the islanders to
be an expected attribute of man as such. An individual---especially a
male adult---who enters interplay is automatically assumed to enter with
a self qualified in these ways. Thus, outsiders who ask questions about
the island, or show lack of familiarity with its routine of activity, or
make an effort to perform an island task, or touch a boat of any kind,
inevitably discredit the self that has been implicitly imputed to them
by the islanders. When outsiders display these shortcomings and at the
same time express an air of urban assurance and superiority to
islanders, they become especially laughable to the islanders, although
of course they are rarely laughed at out loud.

\hypertarget{unserious-disruptions}{%
\section{Unserious Disruptions}\label{unserious-disruptions}}

It has been suggested that gaffes are sometimes handled by defining the
situation in an unserious way, so that the self that is discredited can
in someway be dissociated from the person whose self it was. In Dixon
much use seemed to be made of this possibility as a source of fun.
Instead of resolving an embarrassing situation by introducing an
unserious definition of the situation, disruptive events are purposely
engineered so that an embarrassing situation will arise, but care is
taken to ensure that an unserious view of it will be taken. Three
varieties of this behavior may be mentioned.

1. In Dixon many households have a member who is recognized for his
ability to ``take off,'' as they say, on others. This refers to the
practice of mimicking an individual or ``taking'' his role in
circumstances where his response can be taken as characteristic of him
and especially of his failings. Mimics in Dixon seem to be very skilled
and frequently succeed in copying the physical posture, the facial
expressions, and the accent and intonation of another, as well as the
linguistic content of his response at a characteristic moment. The
amount of laughter that a mimic evokes from his audience appears to vary
according to the accuracy of his gestural copy and the number of
behavior levels that he is able to bring into the gestural portrait.
Certain mimics become famous in a neighborhood circle for their
treatment of a given individual, and at small gatherings they will be
coaxed to perform their specialties. Mimics and their audience clearly
recognize that a mere linguistic repetition of a person's statement will
not evoke laughter. Obviously, the self projected in this way into the
interaction is neither one's own nor that of the person being mimicked
and is necessarily unsustainable.

2. Another favorite source of humor on the island is what is called
``leg-pulling.'' The typical pattern is for an individual who is to be
the butt or goat of the joke to be given information which others
present know to be false or unsound. The butt is then led into
projecting attitudes, responses, and actions which would be acceptable
and creditable were the information true. Sooner or later during the
interplay the butt learns that the information has been false and that,
consequentially, the self he has projected into the interplay is
necessarily untenable and ludicrous. It is a crucial feature of the game
that the butt does catch on or that a truthful disclosure is finally
made to him. As previously suggested, the person responsible for
building up the false impression in the first place usually makes sure
that someone else is present who can be let in on the joke, thus
ensuring that the butt will have to define the situation, after he sees
through the game, as only a game. The spirit of the game requires the
butt to do a big take of confusion and shock upon learning that he has
been ``had,`` and persons who almost get seriously angered are the best
and favorite subjects. The moment of chagrin when the butt ``catches
on'' is the high point of the game. Presumably spontaneous involvement
of those in the know derives from the fact that were the unsustainable
projection done in serious life, great embarrassment and chagrin would
result.

Leg-pulling is of sociological interest not merely because it
illustrates the effects of projecting a self that is patently
inconsistent with reality but for two additional reasons. First, men of
full adult status in the community, who had a wife, children, and no
peculiarities, were considered too dignified to have their leg pulled,
except on April First, even though many would have liked to make fun of
them in this way. Persons below the age of about twelve are considered
too easy to dupe, and apparently have too little to lose by the loss of
their dignity to make the game worthwhile; they are not fair game.
Persons entering adult status, and, especially, persons who are old
enough to have achieved adult status but have for some reason failed to
do so, are favorite butts for the game. Secondly, in order to make a
prospective butt fall into the trap of belief, it was sometimes
necessary for players of the joke to exercise very impressive skill in
the control of what are usually thought of as the purely involuntary
expressive components of behavior. So skilled are some islanders in
doing this that one feels they do not feign expressive behavior during
serious occasions because there would be a strong negative sanction for
being caught doing it, not because of incapacity to do so. Apparently
the fear of being caught out acts as an involuntary disturbance in the
art of feigning, and when things have been arranged so that the sanction
against false communication does not apply unexpected ability at
feigning is shown.

One variety of leg-pull on the island is what has sometimes been called
``sending persons on a fool's errand.`` Thus, the shops being closed
Wednesdays, a favorite pastime is to ask someone in the house to run
down to the shop to pick something up. If the person asked does not
immediately ``catch on,`` he usually either projects a self that is
willing and happy to be accommodative or projects a self that has other
immediate objections and can't oblige; in either case, sociological
disaster is inevitable, since he conveys to those present a self that
has accepted and adjusted to the right and obligation of doing a favor
or a chore, and then finds that there was no basis for the projection.
Every household seems to have a store of tales, often retold, of classic
leg-pulls. A few examples follow:

\begin{quote}
One afternoon in the hotel kitchen talk turns to famous leg-pulls.
Mrs.~Tate says: ``I remember once we decided to get one on Mary {[}a
former maid at the hotel{]}. So once when two men {[}hotel guests{]} had
geen on the o'erland early {[}early-morning overland transportation{]}
we set the clocks back and made up the bed clothes to look like they
were still there and then roused her and told her if they didn't hurry
they'd sure be late, and she went and knocked and got no answer so she
went in, saw the figure and come out and knocked agin---went and shook
the figure. My we laughed.''
\end{quote}

\begin{quote}
Later in the conversation Mrs.~Tate says, ``Once we had a girl from
Torin {[}another island in the Bergand group{]} to help and my she was
slow. Finally, I could stand it no longer and I asked her to go to James
{[}Mr.~Tate{]} in the garage and get some elbow grease. He said his was
all dirty and sent her to the shop.''
\end{quote}

\begin{quote}
Alice comments: ``Like the time they sent Willi {[}one of the
community's brasher young men{]} to the shop for short circuits and John
{[}a mild-mannered clerk{]} asked Alex {[}the manager{]} if they had
any.''
\end{quote}

It maybe noted that the practice of leg-pulling also appeared in an
organized form during games held at large house parties. As previously
suggested, the form of the game required that a butt (or sequence of
butts) be chosen, and that the remainder of those present be in on the
joke. Ordinarily a half dozen or so young persons would be kept outside
a room while the joke was being explained to persons in the room. As
each butt learned the secret by having the joke played on him, he would
be added to the audience, and another butt would take his place from
those chosen to wait outside. In general, the game consisted of
involving the butt in what felt like one line of action while in fact it
was another. At the crisis or peak of the game, the butt discovers that
he has been projecting a self fitted to one set of facts and that in
reality another set holds. The more chagrin and embarrassment he shows,
the greater becomes the spontaneous involvement of the audience.
Interestingly enough, persons upon whom such games were played often saw
through the game, or did not feel it funny, and yet would affect a show
of doing a big take when the proper time came.

3. When persons gather for interplay in Dixon, it is assumed that each
participant is deserving of certain kinds of approval and protection by
the other participants and that acts which aptly express disrespect for
him will be inhibited or avoided Thus, at the crudest level, one
participant does not shove or push another unless there is a clearly
honorable reason for doing so. When the situation has been defined
unseriously, participants have an opportunity to engage in horseplay;
they have an opportunity to commit just those acts of disrespect against
each other that would ordinarily be cause for great offense. Presumably
the strong feeling such acts would create in serious interplay is a
source of spontaneous involvement during unserious interplay.\footnote{See
  chapter xix for a discussion of strategies ensuring sufficient
  involvement.} In any case, we have a practice of what might be called
ritual profanation. The image of himself that a participant projects as
someone deserving of fundamental respect is purposely and playfully
discredited. At billiards, when the time is right for it, a player's cue
is pushed from behind by someone he cannot see so that the stroke by
which his skill was to have been expressed is made to look ludicrously
clumsy, in much the same sense that a boy who is tripped by another is
made to discredit the expectation that he can carry himself like a
person. Or, at billiards, the player is purposely ``put in balk'' by the
player before him, so that his expectations of having his playing self
treated with consideration are sharply disappointed. At a community
dance, an eight-year-old would venture closer and closer to an elderly
drunk man until he finally tweaked the man's hair and ran away in
excitement. In the kitchen of the hotel, the employees on occasion tease
each other in all manner of ways. For example, the cook would be kidded
about not having a girl, the scullery boy about having one. The maids
would kick the cook; jump on him from behind; tweak his legs; put
buttons, salt, and cookies in his tea; smear him with lipstick and bath
salts; flick their fingers at his ears; put soapy hands down his neck;
twirl a wet boiler lid at him; throw his cap away; turn his back pockets
out; pull him out of bed; and put their hands into his front pockets.
They would also tell him that his cooking was bad. He, in his turn,
would chase the maids, slap them across the neck with a fresh piece of
meat, look through their purses and pull out cosmetics, grab them and
soundly kiss them.

Interestingly enough, islanders sometimes acted towards themselves
during interplay in such a way as unseriously to discredit their own
claims to respect. At certain times a participant would act in such a
way as to make himself rather than anyone else look foolish; he would
play the buffoon. In missing an easy shot, a billiard player would
loudly curse himself, until other players started to laugh. In being
teased, a person would do an almost serious take, showing violent loss
of composure. And when it was known that an individual had ``had a few
drams,'' he would wildly act the fool for the amusement others. It
seemed, in effect, that persons would at times sacrifice their own
dignity in an unconscious desire to keep amusement and interest in the
interplay from lagging.

Ritual profanation, like leg-pulling, seemed to find an organized form
in party games. In one game, for example, a person in the center of a
large circle of persons spins a pan and once the pan has started
spinning calls out a number which corresponds to some one of the
players. The player must rush from the outside of the circle and grab
the pan before it stops spinning or pay a forfeit. To get to the pan in
time, the person whose number is called has to drop his dignity and
scramble as fast as he can. The more he disobeys the rules of acting in
an orderly fashion, the more laughter is created in the audience. In
another game a young unmarried man and woman are blindfolded and each is
given a spoon and a bowl of jello and instructed to feed the other. In
consequence, the faces of both and the breasts of the girl are ritually
profaned to a surprising degree. The butts' assumptions of cleanliness
and modesty are wonderfully discredited. This game is extremely
successful, and if the players go through their part with some
seriousness, the audience can become extremely involved in a joking way.
In another game, called ``Beetle,`` a high score and a chance for a
prize is achieved by shouting out ``beetle'' as soon as a certain
sequence of numbers has been reached by the roll of a pair of dice,
there being two teams of two for each pair of dice. To win the game one
must seize the dice-cup as soon as one's opponent has laid it down and
shout ``beetle'' as soon as the sequence is attained. To win the game
persons forget themselves and blindly grasp the cup as soon as possible.
A climax is reached when the first person to achieve the proper sequence
shouts ``beetle'' in a completely uncontrolled, unseemly way. Thus the
players discredit in a joking context the assumption that they are in
control of their passions

During community dances a pattern of ritual profanation was also
employed as a means of ensuring involvement. In Lancers and Quadrilles
the ``swing your partner`` figure always managed push some of the female
dancers past the limit of seemly involvement, into a scene where they
and others would take an unserious view of their loss of equilibrium and
self-control. In another dance, the last figure is danced to an
ever-increasing tempo until all the dancers lose their balance and
self-direction.

% CHAPTER XXIII: THE MANAGEMENT OF PROJECTED SELVES
\chapter[CHAPTER XXIII: THE MANAGEMENT OF PROJECTED SELVES]{Chapter XXIII: The Management of Projected Selves}
\label{ch:Chapter XXIII: The Management of Projected Selves}
\chaptermark{CHAPTER XXIII: THE MANAGEMENT OF PROJECTED SELVES}

\newthought{During interplay, events may}
\marginnote{\href{https://doi.org/10.32376/3f8575cb.ba0f8e35}{doi}}occur which make it difficult for
participants to accept in an unthinking way the self projected by
someone among them, or to continue to accept a projected self which they
had initially accepted in this way. On such occasions dysphoria is
likely to occur. The conscious realization that a projected self has not
been or is no longer spontaneously accepted---whether this realization
comes to the person whose projected self is not accepted or to the
others---is likely to heighten the dysphoria.

During interplay in Dixon, individuals exercise tact or social
strategies in order to maintain interactional euphoria. Some of these
strategies are preventive, serving to avoid threats to the interplay;
some are corrective, serving to compensate for dangers that have not
been successfully avoided. These strategies may be employed by the
individual causing the disturbance (if, in fact, it is felt that some
one person in particular is at fault) or by individuals for whom such a
disturbance is caused. When these strategies are successfully employed,
social harmony in the interactional order is maintained or restored. Of
course, a person who acts in such a way as to contribute to the euphoria
in an interplay may act from many different motives and intentions. Some
typical strategies are reviewed here, illustrations being provided for a
few of them.

\hypertarget{discretion}{%
\section{Discretion}\label{discretion}}

A gaffe has been defined as any act which precipitously discredits a
projected self that has been accepted in an interplay. Of the many kinds
of gaffes, two varieties seemed to stand out clearly.

First, there were what are sometimes called ``boners,'' where the person
responsible for the gaffe is also the person whose projected self is
discredited. Boners themselves vary in the degree to which the self that
is discredited is a self of which those who had initially accepted it
were suspicious and doubtful. Islanders employed two preventive
strategies regarding boners: they made an effort to express modesty
wherever possible, and they took care not to appear in situations where
a boner was likely, as during an appearance on the community hall stage,
unless they felt sufficiently poised to do so. And as described in the
previous chapter, they employed the corrective strategy of defining in
an unserious way a situation in which they had made a boner.

Secondly, there were what are sometimes called ``bricks,'' where the
person responsible for the gaffe is not the person whose projected self
is discredited. Strategies for preventing the occurrence of bricks
seemed to be an aspect of tact which islanders were conscious of as tact
and about which they had explicit expectations. Islanders felt that
adults ought to have their wits sufficiently about them not to create
what they called ``faux pas.''

During interplay in Dixon, every participant seemed to project a self
into interaction which the mention of some facts of his past life would
embarrass. These facts were usually not major ones, as may be found in
urban situations where social practices such as ``passing'' are
possible, nor were they sufficiently important so that the mere knowing
of them by the participants would discredit the person to whom they
applied. But they were sufficiently important so that if they were
raised at an inopportune time, they would cause the individual (and
sometimes others) to lose his poise and feel ill at ease, bringing to
the interplay some constraint and dissonance. A very general form of
tact practiced on the island was the avoidance of mention of anything
which would bring to any participant's attention facts about himself
which he found at the moment embarrassing. Islanders have an intimate
knowledge of each other's ``sore points'' and are thus in a position to
avoid them. The betrothal of a man and woman would not be mentioned
before the rejected suitor. Questions of paternity would not be raised
in the presence of a bastard. Strong views on politics or religion would
not be voiced until the politics and religion of all participants had
been tactfully established.

Discretion was also exercised in avoiding interplay with persons under
circumstances which might make it difficult for them to sustain the self
they would be likely to project.\footnote{See the reference to
  ``avoidance relationships'' in chapter xx.} Thus, an old crofter who
had no land and who made no attempt to keep his cottage clean was
provided with hot meals by the neighbors around him, the meals being
brought to his cottage and handed to him at the door. A woman who had
lived across the road from him for more than ten years suggested that
she always got ``one of the bairns'' to deliver the meal because she
felt the old man would be embarrassed if she came and saw how the inside
of his house looked.\footnote{0n the occasion when a section of the
  community was being canvassed for contributions to the postman's
  retirement gift, the same woman argued that the canvasser ought to
  collect some money from the old man because while he might not be able
  to afford it as well as some of the others (he was on relief) he would
  feel hurt if he thought be had been omitted.}

In the case of strangers from off the island, whose past life could not
be thoroughly known, care had to be taken to stay off topics that while
not known to be embarrassing could be embarrassing. Thus, the islanders
were sufficiently tactful towards strangers not to inquire into matters
such as religion but to stay off the topic and wait for information to
be volunteered, thereby illustrating Simmel's dictum that discretion
``\ldots{} consists by no means only in the respect for the secret of
the other, for his specific will to conceal this or that from us, but in
staying away from the knowledge of all that the other does not expressly
reveal to us.``\footnote{Simmel, \emph{op. cit.}, pp.~320--321. A
  functional implication of this kind of tact is, of course, that the
  strangers voluntarily provide information to others of the kind they
  will require in handling them.} Persons on the island exercised the
kind of tact that is calculated to make it easier for others not to be
tactless. Newcomers to the island are warned before it is too late as to
what not to talk about before whom, this warning coming either from the
sensitive person himself or from an interested third party. For example,
Mr.~Alexander's wife (not on the island) was a Catholic. Catholicism
could be discussed in his presence but not in the same way it was
discussed in his absence. The gentry took care to warn their visiting
friends of this fact so as to ``avoid embarrassment.''

\hypertarget{hedging}{%
\section{Hedging}\label{hedging}}

During interplay in Dixon, many individuals frequently employ the
preventive strategy of never committing themselves, fully and
irrevocably, upon an issue---whether the issue directly reflects upon
the self-image of the participants or does so only indirectly.
Participants tended to take care to involve themselves and commit
themselves judiciously, not allowing themselves to become clearly
identified in the eyes of the others with a self-image which
unanticipated communications might contradict.\footnote{During interplay
  it was frequently assumed that all those present were in agreement
  with each other on fundamental impersonal issues and that, by
  implication, every participant was the sort of person who would hold
  the accepted view on a given issue. The occurrence of open
  disagreement on particular issues led participants to feel that they
  had unjustifiably taken a sympathetic view of each other. The sudden
  occurrence of disagreement obliged talkers to ``back down'' in the
  interests of maintaining a show of harmony, requiring a speaker to
  fumble with the tone of assurance and authority that had been in his
  voice. Expressed disagreement, then, was a threat to the selves that
  had been projected into the situation. An important exception occurs
  in interplays that are specifically designed to provide an opportunity
  for argumentation of a disinterested, dispassionate kind. In such a
  context, disagreement over impersonal issues need not disturb the
  working acceptance, providing the participants take their disparate
  stands in accordance with the rules for cool-headedness and
  disinterest.} No matter how sure they are of the propriety of their
acts, they hold themselves back a little; they hedge a little; they
attempt to maintain a margin of safety. Thus, if an unanticipated
communication occurs which is inconsistent with the positions that have
been taken, it is still possible for participants to act as if the
positions in question were not fully or unreservedly taken. Communicated
valuations may thus come to grief without bringing a similar fate to the
person who conveyed them; salvage is possible.

One of the most interesting variants of the strategy of hedging is found
in what might be thought of as exploratory communication, or the process
of feeling a person out. The sender takes a position on a particular
matter in an ambiguous or mild way. If the recipient responds with no
encouragement, the sender is in a position to claim that the valuation
was not important to him, or that it was not meant in the way the
recipient took it. If the recipient responds with encouragement, then
the sender is in a position safely to add a little more weight and
clarity to his initial valuation. In this way the sender can go through
a sequence of steps, committing himself a little more with each step but
always remaining within a safe distance of what is an acceptable
position. Thus, with respect to any single continuum of expressed
valuation, no matter which one of the two participants is first to call
a halt, the other is left with a manageable or defensible position.
Exploratory communication occurs frequently in courting situations and
in the ``placing'' interplay by which two newly acquainted persons learn
about each other's statuses without either person conveying (at any one
stage in the process) a standard of judgment that is embarrassingly
damaging to the other or embarrassingly revelatory of himself.

A common form of exploratory communication occurred in Dixon with
respect to requests for favors and the giving of orders. If a favor was
asked of someone, and the person asked felt obliged to refuse, then the
asker was put in the position of having been presumptuous, assuming more
friendliness or good will than actually existed. In order to avoid the
appearance of such discrepancies, persons asked for a favor usually
acceded or, if not, provided a very understandable excuse for refusing.
Persons who wanted to ask a favor knew this to be the case and did not
want to put others in the position of feeling forced to accede in order
to maintain euphoria.\footnote{See references to the ``After you,
  Alphonse,'' interchange, chapter xxi.} Hence favors were often asked
in a roundabout way, so that they could be refused before the asker had
committed himself to a self of someone asking a favor. Thus if someone
was to be asked to go to the shop to make a purchase, the asker would
first inquire of the other if he were going down the road. The answer,
``No, not just now,`` would end the interchange without either person
having placed himself in an embarrassing situation. If the answer was,
``Yes, I am,'' then the request to bring something back from the shop
could be made with relative safety. So, too, when one person was acting
as guest-worker, his host would usually say, ``You can do this next if
you if you want,'' instead of commanding him. Similarly, if the managers
of the hotel wished to visit relatives for a night they would ask a
member of the staff who was entitled to have that time off if he was
doing anything that night.

In Dixon, and perhaps throughout Britain, many persons are interested in
obtaining a little more than their legal share of rationed foods and
materials, and many supply sources have something extra to distribute. A
code has apparently developed whereby the customer can convey the fact
that more than the ration would be desired without presenting a self
that might have to be refused and, by implication, found illegal, or
greedy, or immoral. The server says, ``How much would you like?'', a
phrase which can mean, ``How much of your ration do you want?'' or ``If
there were no rations to think of, how much would you want? `` Customers
frequently reply in a light-hearted way, ``As much as I can get,'' thus
making a joke of the situation and at the same time conveying something
that can mean, ``As much as I can legally get on my ration book'' as
well as what it usually means. They frequently add a sly wink to the
nearest customer as guarantee that they are not being ``serious.'' The
server is then in a position to offer the customer an extra amount, or
to say, ``We only have such and such,'' or ``You can have half a pound
on each book'' (this being the legal amount). In either case the
customer is in a position to act as if he has not projected demands of
an inappropriate kind. Further, if the customer feels righteous and
says, ``Only what I am allowed,`` the server is in a position to act as
if he, too, is following the legal code.

\hypertarget{politeness}{%
\section{Politeness}\label{politeness}}

In an early part of this study, it was suggested that individuals may be
viewed as sacred objects; they can be offended or pleased by events
which have an expressive significance even though not an instrumental
one, and signs of approval and disapproval can be found or sought in
every event that occurs in their presence.

In entering interplay in Dixon, each participant seemed to estimate
roughly the degree to which he would be approved and the basis of the
approval and lower his defenses, as it were, to a corresponding degree.
Thus, if one participant conveyed sharp approval or disapproval of
another participant, the judged participant was likely to become ill at
ease and become more self-consciously concerned with himself than is
thought proper. While islanders recognize moral and expediential {[}\emph{sic}{]} reasons
for not being too explicit in their approval and disapproval of others
present, they also seemed to be motivated by a desire to prevent the
embarrassment---the interactional dysphoria---that might be a
consequence of such expressed judgments. Thus, an image of what might
and might not give offense to another is a principal guide for one's
conduct in interplay.

\hypertarget{unseriousness}{%
\section{Unseriousness}\label{unseriousness}}

Perhaps the most frequently employed social strategy in Dixon, both as a
preventive and a corrective measure, was the introduction of an
unserious definition of the situation. If an individual found that he
had been implicitly or explicitly assigned a role that he was not sure
he could properly carry through, he would joke about his incapacity so
that if he did fail those present would feel that the self that had been
discredited was not the individual's basic or real self. If he committed
a gaffe, or if another participant committed one that was not too
serious, he could introduce an unserious definition of the situation in
an effort to restore harmony. If a sender conveyed something that
suddenly appeared to give offense or that might possibly give offense,
he could act as if it were only meant in jest.\footnote{Doyle, \emph{op.
  cit.}, p.~79, gives an example: ``A slave, in case of a breach of
  etiquette or duty, could laugh, as a sign that no offense had been
  intended.''} If a communication was received that was unacceptable to
him, he could avoid open disagreement by replying with unserious
agreement or unserious disagreement. In all of these cases, the person
who used the strategy could only be accused of a breach in taste, that
is, he could only be accused of being unserious at the wrong time or
about the wrong thing. The ability to employ this strategy so as not to
allow a potentially discreditable self to be given temporary credit was
part of what seemed to be implied in the phrase ``to have a sense of
humor.''

When, for example, a person asked for a second helping of food, he often
did so in a tone that approximated baby-speech, presumably showing that
the self that was asking for food was not the actor's real self and
could be thought greedy without disturbing the situation. In the same
way, when men got a little drunk and exercised extra prerogatives in
their behavior, they would make sure to slur their voices, even though
this may not have been inevitable, showing that the improperly conducted
self was not a real one.

\hypertarget{sangfroid}{%
\section{Sangfroid}\label{sangfroid}}

It has been suggested that improper involvement is a contagious thing,
that when one participant feels ill at ease a conscious realization by
others that this is the case is likely to disturb their involvement
also. The capacity to conceal signs of interactional discomfort is
sometimes called ``sangfroid. `` Concealment of this kind breaks the
vicious-circle effect of embarrassment and constitutes a kind of tact.
Two examples may be given.

\begin{quote}
During a community concert in Dixon about twelve six to eight-year-old
children are singing in chorus; their teacher is accompanying them on
the piano, which is on the stage. The electric lights, which have just
replaced gas-pressure lamps, fail for a moment. The audience becomes
momentarily disoriented and there is an immediate, though quite light,
murmur throughout the audience. The pianist plays a little louder and
the school children go on singing, exactly as if nothing had happened.
In a moment the audience is again silent and the lights go on.
\end{quote}

\begin{quote}
During a political meeting at which a county candidate was speaking,
Dr.~and Mrs.~Wren arrived late. They entered a situation in which all
members of the audience were formally defined as equal, a definition
borne out by the fact that the island's business family, the Allens, sat
on the same kind of benches and with no better point of vantage than the
assembled audience of crofters enjoyed, and that after the speech
crofters spoke up and asked questions with much the same confidence as
shown by the gentry in asking questions. Close to the speaker there were
two chairs, and as the Wrens entered a crofter sitting near the end of a
bench near the chairs got up and pointed to the chairs, inviting the
Wrens to take a place of preference. This action forced everyone to
remember that preferential participation rights were once accorded to
the gentry and that agreement no longer existed as to proper conduct in
these matters. Dr.~Wren laughed lightly and quickly answered the
invitation by saying that he was so big no one would be able to see
through him so he should best sit on the bench. The audience relaxed.
After the meeting, he admitted to his wife and the writer that the offer
had been a damn-fool thing and that privileges ``like that would never
do.''
\end{quote}

\noindent Presumably this social strategy differs from some other kinds of tact in
that a mere desire to exert it is not in itself sufficient; trained
capacity is required.\footnote{Romantic literature and etiquette books
  regularly attribute this capacity to the best classes. To the writer's
  knowledge, no one has actually studied the minute-to-minute behavior
  of a social elite to discover whether in fact the members do practice
  this (and other) tactful strategies more than do members of other
  classes.} Islanders seemed to differ widely in their capacity to
remain cool under social fire.

\hypertarget{feigned-indifference}{%
\section{Feigned Indifference}\label{feigned-indifference}}

Individuals base their projected selves upon certain positively valued
attributes. It has been suggested that contingencies may occur which
demonstrate that a particular individual has radically more or radically
less of a given attribute than his activity up to then implied. The
practice of feigning indifference is a preventive strategy for
overcoming this danger. It was much used in Dixon. By feigning
indifference to an attribute, the individual could project and establish
a self-image in which the attribute played no part. Once this image was
overtly accepted by others, then failure (or too much success) with
respect to the particular attribute ceased to be an uncontrolled source
of embarrassment; the working acceptance based on the projected self was
not disrupted because the projected self had been originally defined in
such a way as to exclude the attribute that was later brought into
question.\footnote{From the point of view of the others, the individual
  who feigns indifference acts like a person who has what he wants and
  doesn't want what he hasn't got. (When this maneuver begins to
  convince the very person who performs it, we have, presumably, a
  variant of a major structural element in social life, namely, pride.)
  The practice of feigning indifference serves as a means of self
  defense for the person who employs it, but it serves many different
  important functions for the interplay in which it is performed. For
  example, it seems that informal interplay could not smoothly perform
  certain of its social functions unless unwanted participants could be
  relied upon to withdraw voluntarily from the interplay (or voluntarily ab-}

\hypertarget{non-observance}{%
\section{Non-Observance}\label{non-observance}}

In Dixon, another strategy used to cope with embarrassing situations is
for all concerned to act as if the disruptive, discrediting event had
not in fact occurred. Individuals acted as if they had not seen or heard
the discrediting event. It is interesting to note that a conflict
sometimes arose, as in the case of stomach growls, burps, or the
dropping of a piece of food, as to which strategy to employ: whether to
act as if the event had not occurred or to recognize that it had
occurred and that it was to be made a joke of. Frequently the tension
and dysphoria on such occasions was created not by the offending act
itself\marginnote{stain from entering) on the basis of slight hints conveyed to them
  by other participants. Slight hints do not disrupt the most intimately
  defined working acceptance; all the participants can overtly maintain
  a spirit of friendliness and affability. And yet within such a
  context, a participant who is willing to take a hint can be led
  voluntarily to accept extreme deprivations. From the point of view of
  the interplay, a participant's sensitivity to hints means that he is
  tractable and manageable; from the point of view of the participant
  himself, it means an opportunity to protect himself by feigning
  indifference.} but rather by the state of suspension the participants found
themselves in, waiting for the offender to define which strategy he, and
hence they, would have to take.

Just as non-observance served as a means of maintaining a self another
had already projected, so a kind of non-observance occurred in response
to a person who acted with patent pretensions. If the projected self was
too serious for others to laughingly discredit, as by employing the
phrase ``come off it,`` then to save the situation participants were
often required to act as if they in fact did not sense a discrepancy
between the self projected and the self they knew or felt to exist. For
example, when a visitor to the island or a small child showed undue
enthusiasm, projecting a self that put too much stock in what was for
the adult islander a small matter, the offender would be answered with a
show of feigned enthusiastic interest. Similarly, when a husband told
anecdotes to his friends, projecting an image of someone making a fresh
and spontaneous contribution to the interplay, his wife and others
present who had already heard the same person tell the same story with
the same show of spontaneous involvement, would tactfully act as if it
were all new to them and do an appropriate ``take'' when the climax of
the tale was reached.\footnote{Other illustrations are given in chapter
  xix.}

Brief reference may be made to three other strategies employed in Dixon.
When a participant inadvertently acted in such a way as to disrupt or
discredit his projected self, he sometimes attempted to ease the
situation by providing a rationalization for his act. A rationalization
may be defined as a causal explanation offered by an offender in order
to account for his offense in a way which provides an unapparent but
acceptable reason for it. An alternative sometimes employed was for the
offender to become aggressively self-righteous and attempt to establish
the fact that his gaffe really represented a proper way of behaving and
that the others were themselves acting improperly if they felt that the
self they had accepted for the participant was consistent with the self
implied in the gaffe. Finally, the offender sometimes employed the
alternative of becoming over-apologetic. He would commit himself fully
to an act of exorcism and apology, attempting to persuade the others
present that he was at one with them in their attitude to the infraction
and that one part of him, at least, was not the sort of person who would
tolerate the offense in question.

% INTERPRETATIONS AND CONCLUSIONS
\chapter[INTERPRETATIONS AND CONCLUSIONS]{Interpretations and Conclusions}
\label{ch:Interpretations and Conclusions}
\chaptermark{INTERPRETATIONS AND CONCLUSIONS}

\hypertarget{the-interaction-order}{%
\section{The Interaction Order}\label{the-interaction-order}}

\newthought{In the study of}
\marginnote{\href{https://doi.org/10.32376/3f8575cb.9912a480}{doi}}social life, it is common to take the concept of social
order as central and to analyze concrete behavior in terms of the way it
conforms to and departs from this model. It is in this sociological
perspective that communication has been studied here.

Underlying each kind of social order we find a relevant set of social
norms. These norms are ultimate social values, differing from other
kinds of ultimate values in that they do not function as goals and
objectives that are striven for but function, rather, as a guide for
action and conduct, often establishing a kind of outer and inner limit
to the range of activity that is permissible and desirable in the
pursuance of a goal. Norms do not provide means and ends but criteria
for making choices among them.

Norms are expressed in terms of rules regarding conduct and action.
Norms, and the rules in which they are embodied, have a moral character;
persons consider norms and rules to be desirable in their own right, to
be binding in an obligatory way, and to be in some sense external to
those who are guided by them. This does not mean, of course, that norms
only function when they have been intracepted as ultimate values. There
are many occasions when it is expedient for an unbelieving actor to
acted in such a way as to give the appearance that he has acted in
response to moral norms, or at least not to contradict openly the
possibility that he has acted in this way.

Failure to obey the rules is sanctioned negatively The sanction may be
specific, formally established in advance, and administered by
officially authorized bodies. The sanction may be informal and
administered by diffuse, indirect social disapproval. Sanctions
reinforce self-regulation, and together these forces lead persons to
behave in a way that is regular and can be anticipated and in a way that
is considered legitimate and socially proper.

\newpage The occasions when persons in Dixon come together and engage in spoken
communication may differ, one from another, in basic ways:
comings-together may differ in number and identity of participants; in
the kind of ties that bind those who participate; in the motive, the
intent, and the function of the coming-together; in the social place and
context in which the coming-together occurs; and in many other ways.

On occasions in which islanders engage in face-to-face spoken
communication, the conduct and action of participants are guided and
integrated together under the influence of many different social norms.
Action is guided and integrated by the rights and obligations pertaining
to kinfolk, property-holders, contractees, citizens, friends, guests,
and the like, and by standards, such as efficiency, economy, and respect
for tradition. In one situation the social orderliness that prevails
will be largely determined by one set of norms; in another situation a
different set of norms will provide the principal guides for action.

In this study I have attempted to abstract from diverse comings-together
in Dixon the orderliness that is common to all of them, the orderliness
that obtains by virtue of the fact that those present are engaged in
spoken communication. All instances of engagement-in-speech are seen as
members of a single class of events, each of which exhibits the same
kind of social order, giving rise to the same kind of social
organization in response to the same kind of normative structure and the
same kind of social control. Regardless of the specific roles and
capacities which an individual employs when he engages in interaction,
he must in addition take the role of communicator and participant;
regardless of the particular content of the spoken communication, order
must prevail in the flow of messages by which the content is conveyed.

It is possible to consider any particular social order in a crudely
functional way and say that it serves to ensure that a particular set of
human needs or objectives will be fulfilled in an orderly, habitual, and
cooperative manner. From this point of view, a preliminary distinction
may be made among three elements: a particular set of needs or
objectives; a set of practices, conventions, and arrangements through
which these ends are fulfilled; and the particular set of norms which
supports and bolsters these arrangements. The system of practices and
arrangements considered in this study brings order not to economic life
or political life but to communication.

In work situations where constant communication is requires among
participants for the governance of work-flow, and where there is some
barrier---social or physical---to ordinary communication, a special
communication system commonly arises. An illustration may be taken from
Dixon dock work. During the loading and unloading of boats at the Dixon
pier, the man operating the winch is often cut off, aurally or visually,
or both, from the man in the best position to guide the cargo as it is
lowered or raised in the batch or in the hold. To fulfill the winch
operator's need for constant information as to the position of the cargo
he is moving, a language of hand-signals is available for signaling
above the noise and beyond the physical barriers of the operation. By
means of this system of communication, any man in the work team can
initiate an extended sequence of directional commands to the winch
operator or retransmit commands to him from someone in the hold whom the
man on the winch cannot see or hear. In order to reinforce this
communication system in which any member of a crew can, at any moment,
take over command of the crew, it is useful for all members of the crew,
regardless of rank an the job or rank with respect to wider social
statuses, to respect one another as persons whose independent judgment
will be sound and as persons from whom it is possible to take commands.
Work needs are fulfilled by a communication system, and the
communication system is in turn buttressed by moral beliefs.

Another example may be cited. There are times of crisis for members of
the Dixon community when it is imperative that any adult in the
community be able to contact quickly any other member of the community.
Accident or sudden sickness, rearrangements for an oncoming social, news
of a job opening, last minute cancellations of cooperative fishing or
crofting ventures---these are examples of such crises. The need to adapt
to these extraordinary situations is fulfilled on the island by means of
an emergency communication system.

In Dixon there were at the time of the study fourteen telephones, some
located in public buildings such as the hotel and post-office, and
others located in private houses, especially the houses of those
persons, for example car-hirers or county officials, who needed a
telephone for occupational reasons. Most of the hundred-odd households
and regular places of work in the community were connected with each of
the other houses or places of work by a known communication channel
involving two telephones and two or more households or offices. In cases
of emergency, it was understood that any adult who wished to speak to
any other adult could walk to the nearest phone, contact the phone
nearest the ultimate recipient, and have the individual who answered the
phone relay the message to the ultimate recipient or call him to the
phone. This communication system fulfilled the needs of crisis
situations so that, from the point of view of emergencies, the community
was saturated with telephones.

It will be apparent that the persistence of such a communication system
depends on the presence and maintenance of good will. Those who do have
phones must be willing to extend favors to those who do not, and those
who do not have phones must not abuse the courtesy of those who do. This
good will is due partly to the presence of kinship ties that
interconnect almost all the families, partly to the age-mate solidarity
generated by a shared ``school-hood,`` and partly to the fact that
islanders know enough about one another to appreciate a crisis from the
point of view of the persons involved in it. And reinforcing these norms
of mutual aid is the sensitivity of most islanders to the widespread
disapproval that would be accorded them if they refused communication
courtesies or unduly exploited the willingness of others to extend them.

Again, then, we see that particular needs are adapted to by means of a
communication system and that this system in turn is stabilized and
buttressed by means of social norms which underlie it.

When we take as our unit of study not a particular work situation with
its particular communication requirements, nor crisis situations, but
the daily social life of an entire community, then the connection
between needs, communication system, and moral norms becomes less easy
to be sure about but perhaps more interesting and significant.

In Dixon, as presumably in any community, there is a need for
information to be able to flow through an almost infinite number of
channels and networks, for lines of communication to be formed, altered,
and re-formed in a fluid and constantly changing pattern. This flow of
information is a condition of any social process---cooperation,
conflict, accommodation, and even avoidance. In the criss-crossing of
social adjustments in an isolated community, and in the multiple
entanglements of its relatively self-sufficient division of labor, it is
important that \emph{any} two individuals---at least any two social
adults---be able to form a link in a communication chain should the need
for it arise. And it is important that any occasion of spoken
communication terminate in such a way that all participants feel that
should a need arise any one of them will be in a position to enter again
into spoken communication with any other of the participants.

The communication needs of everyday life in Dixon, in the multitude of
situations where no special communication problem is found, are
satisfied by a communication system of rules, practices, and
arrangements giving rise to a unit of communication activity that is
here called interplay. As described in Part Four of the study, we find
that rules are observed as to who may enter into conversation with whom,
upon what topics and with what pretext, and for what length of time. A
set of significant gestures is employed as a means of initiating and
terminating a spate of communication and as a means for those who are to
participate to accredit each other as legitimate participants. A single
focus of thought and of visual attention tends to be maintained, and the
concerted visual attention of the participants tends to be transferred
smoothly from one participant who is speaking to another who wishes to
speak next, the transfer being effected by expressive cues or
``clearance signs.'' By appropriate gestures, recipients convey to the
sender the fact that they are according him their attention.
Interruptions and lulls are regulated so as not to disrupt the flow of
messages. Messages that are not part of the officially accredited flow
are modulated so as to interfere only in a limited way with the
accredited messages. Nearby persons who are not accredited participants
visibly desist in some way from exploiting their communication position
and modulate their own communication, if any, so as not to provide
difficult interference. A ``working acceptance'' is maintained, through
which participants who may be in real disagreement with one another give
temporary lip-service to actions and judgments that bring them into
agreement. There is a tendency for complex judgments to be made
concerning each participant's social attributes, and for these judgments
to determine the relative average length and the relative frequency of
each participant's messages. Finally, each spate of communication during
which a given set of participants is accredited and a single moving
focus of attention is maintained tends to be arranged into a sequence of
discrete, relatively self-sufficient interchanges, and each of these
interchanges or communication spurts contains one or more rounds of
statement and reply.

The system of rules and conventions which guides the flow of messages
during spoken communication is a normative system. Not only can it be
anticipated that islanders will adhere to these communication
conventions but also that they are, in some sense, morally obliged to do
so. When one of these conventions is broken, it is not the state or the
community that is offended, but only the other participants, and in most
cases they are obliged to sanction the offender in an inexplicit
roundabout way. Thus, conventions for guiding spoken communication on
the island constitute the kind of normative system which is sometimes
called etiquette.

The communication etiquette which brings order to the flow of messages
during spoken interaction constitutes a practical communication system
for the varied interactions of everyday life on the island. Underlying
this etiquette there is a set of social norms which apparently gives
communication conventions stability, strength, and flexibility. For
summary purposes, these norms can be roughly placed into two broad
groupings. There are norms obliging persons to inhibit their immediate
response to a situation and to convey a calculated one; and there are
norms which oblige the individual to act in just the opposite way, to
express himself spontaneously, candidly, and without consideration of
the likely response of others to him. These two sets of norms were found
to be operative wherever, whenever, and with whomsoever spoken
communication occurred on the island.

Of the norm which lead islanders to inhibit their immediate response
during interaction, three central ones may be cited. First, the
individual is obliged to suppress his ``real'' feelings about those to
whom he is talking and act in such a way as to show constant regard for
their positively-defined attributes and at the same time constantly
avoid a show of concern for their negatively-valued attributes. In other
words, considerateness and respect must be constantly shown for others
present. Secondly, the participant is obliged to hold himself
sufficiently off from all kinds of ties, constraints, and involvements
for the duration of the interaction so that he will be free, at least to
a degree, to sustain the role of communicator, to follow the course of
the interaction wherever it may lead. Typically the participant will
withdraw himself from involvements which occurred prior to the
interaction and from those which are scheduled to occur after the
interaction has terminated, lest these external involvements strain,
impoverish, or trivialize the self that he makes available for the
interaction. He attempts to refrain from uncontrolled emotional
responses to a passing object of attention, so as not to jeopardize the
continued poise and readiness he exerts as an interactant. And he
attempts to exercise restraint in his demands for attention, praise, and
other indulgences, showing that his capacity as communicator has not
been overpowered by other orientations. Thirdly, the individual is
obliged to conduct himself so that the impression he initially gives of
himself, and which others use in building up a framework of response to
him, will not be discredited later in the interaction by gaffes, boners,
disclosures, and the like, nor seem to others to be pretentious. These
three inhibitory norms, considerateness, self-control, and projective
circumspection, seemingly modify and guide the way in which an islander
performs every one of his acts while he is a participant in spoken
interaction.

Opposing these inhibitory norms, there is a set of norms obliging
islanders to become immediately and thinkingly involved in any
interaction in which they have been accredited as participants. A
participant must not seem to be indifferent to the interaction or
disdainful of it. He is expected to become sufficiently involved in the
proceedings at hand to be unselfconscious about his role in the
interaction, and it is expected that he will desist from worrying about
the impression he is making so that he can give his main attention to
the subject matter of the communication. He is obliged to be
sufficiently honest and candid in a linguistic way, and sincere and
unaffected in an expressive way, to give his co-participants confidence
in the validity of the information they are receiving from him. And he
is expected to give at least some expression to his real feelings,
regardless of the price he may have to pay for so doing.

At all times and in all places islanders tended to manifest a
fundamental action-tendency relative to communication. Whenever an
individual could be associated with an act or event, there was a
tendency for others to take this happening as an expression of the
characteristics of the individual (whether or not they were justified in
doing so), especially as an expression of the conceptions he had of
himself and of others. Further, there was a tendency for individuals to
show deep concern for the judgments and evaluations made of them,
whether these judgments were conveyed linguistically or expressively,
and whether they carried any immediate instrumental consequences or not.
Finally, islanders in all their actions tended to take into
consideration the ``meaning'' or interpretation the others would be
likely to place upon these actions (whether in fact the others did or
did not do so) and guide their actions accordingly. In brief, islanders
found that they must act under what might be called conditions of great
expressive responsibility.

It is a crucial characteristic of face-to-face communication that a host
of acts and events inevitably becomes available for aptly expressing the
conceptions participants have of one another. In order to exert
expressive responsibility, islanders must exert thorough and continued
care of their behavior while in the immediate presence of others. It is
in terms of this action-tendency and the unique communication conditions
of face-to-face communication that we can understand how communication
norms are related and articulated to the set of conventions which guides
the flow of messages. The rules that messages ought not to be
interrupted, or that a participant ought not to withdraw from an
interaction before the others are prepared for this, or that a speaker
ought to be given attention, etc., function to ensure that orderly
communication will prevail, but the manner in which these general rules
are to be applied to a particular case and often the motive for applying
the rules seem to rest on the fact that interruptions, leave-takings,
inattentions, and the like are aptly designed---apart from their role in
the ordering of messages---as signs for expressing the judgments that
participants make of one another. Islanders tended to decide how to
conduct themselves in the presence of others by considering the
interpretation that others would be likely to place upon this action,
but in guiding action on this basis, islanders found themselves acting
so that messages could flow in an orderly fashion.

It has been suggested that the Dixon community has certain general
communication needs, and that these needs are satisfied through a set of
conventions and practices giving rise to what has been called interplay.
Reinforcing this system of communication we find norms pertaining to
communication: norms requiring individuals to inhibit their immediate
response to the situation and at the same time to involve themselves
spontaneously in the interplay, and norms requiring the individual to
act with expressive responsibility. There is perhaps a kind of
functional relationship linking the needs, the communication system, and
the norms. We may start with a norm and see how it facilitates the
maintenance of the communication system, and how the system in turn
facilitates the fulfillment of the needs; or we may start at the other
end of the chain, with the needs, and see how they would tend to give
rise to the communication system, and how in turn the communication
system tends to facilitate the development of interaction norms. For
example, if information is to be conveyed from one individual to another
(this being a general need or requirement for community life), then it
is useful to have a set of communication conventions which lead a
prospective recipient to enter communication when called into it, to
give uninterrupted and uninteresting attention to the message until it
is terminated, and to signal back to the sender that the message has
been correctly received. In turn, the recipient follows these
conventions because he feels that the sender will interpret any failure
to do so as an expression of disrespect, for the interruption of
another's message or a failure to accord him visual attention is, aside
from its role in a communication system, a vehicle aptly designed for
conveying an expression of disrespect. (Thus one can see that respect
for the other has its function as a guide for face-to-face
communication, and hence need not be exerted, and relatively speaking is
not exerted, for those who are not present.) Or, starting from the other
end of the chain, we can say, for example, that an individual feels
obliged to show himself in control of his desires and his involvement,
and that, given a tendency for every act to be examined for expressive
significance, he will be required to show that he is content with the
attention quota accorded him and that he is net unduly embroiled in
events that have occurred before the interplay or not too much at the
mercy of events that are to occur after its termination. By acting in
such a way as to express continuously the fact that he is in control of
himself, the individual places himself in a position to initiate
interplay and to continue as an effective participant in it. And by
being in a position to continue as a participant wherever the
communication may lead, the individual ensures that he will not
constitute, on any occasion, a block to the free flow of information in
the community. Similarly, the minimum of respect that all islanders show
to one another means that all islanders will feel obliged, under proper
circumstances, to treat any other islander as a co-participant in
communication, and the relative degree of respect that any particular
islander has for any other particular islander has the effect of
determining the allocation of attention quota, hence providing a guide
for the flow of messages. And, too, while the obligation islanders feel
to pay ritual homage to each other by expressive gestures of respect
requires that each make himself available to the other for
communication, the obligation to exert self-control and not convey an
unfavorable judgment of the other requires that no individual take undue
advantage of this availability of the other.

\hypertarget{interaction-euphoria-and-dysphoria}{%
\section{Interaction Euphoria and
Dysphoria}\label{interaction-euphoria-and-dysphoria}}

On some occasions when islanders engage in spoken communication, all
participants tend to feel at ease, unselfconscious, and unembarrassed.
Interactional euphoria prevails. No one senses a false note. The way in
which one participant is involved in the interaction does not disrupt a
proper degree of spontaneous involvement on the part of others. At such
times a balance seems to be achieved between action which is guided by
the inhibitory norms and action which is guided by the expressive ones.
On such occasions respect for others and self-control seem to be so
deeply intracepted or so well feigned that the individual can act in a
relatively spontaneous way and yet not cause offense to others.

On other occasions when islanders engage in spoken communication, one or
more participants may feel out of countenance, flustered, out of place,
or offended. Some interactional dysphoria prevails. At such times an
imbalance is found in the opposing norms of communication.

In order to avoid dysphoric situations, or counteract those which have
not been successfully avoided, participants regularly employ strategies,
that is, rational adaptations to the normative requirements of
interaction. For example, ruses are employed by an individual in order
to secure a degree of approval from others that would cause offense to
them if openly sought after. So, too, strategies such as discretion,
hedging, unseriousness, feigned indifference, non-observance, etc., are
employed in order to guard against committing offenses oneself and in
order to make it easier for others not to commit an offense.

The occurrence of interaction tensions and disharmonies is extremely
common on the island. In response to this fact, one general strategy
seems always and everywhere to be employed. The requirement that
participants exert self-control and respect for others is allowed to
establish a working acceptance through which official linguistic
lip-service is given to the fiction that all present are behaving
properly, that all are in agreement on matters of significance, and that
all respect one another. And underneath this surface of agreement, the
vast expressive equipment that becomes available when persons are in
each other's presence is used to convey a contrary view, a view that
would disrupt the working acceptance were it conveyed openly and
officially. This undercurrent of communication, now taking the form of
furtive coalitions, now taking the form of innuendo, hints, and oblique
thrusts, provides the forbearant actor with a safe channel of free
expression. Apparently this two way-pull on communication assures that
persons with different views and even personal dislikes of one another
will yet be able to tolerate once another long enough for information to
flow back and forth between them.

\hypertarget{the-special-characteristics-of-the-interaction-order}{%
\section{The Special Characteristics of the\\\noindent Interaction
Order}\label{the-special-characteristics-of-the-interaction-order}}

The social order that obtains when persons are engaged in spoken
communication---by virtue of being so engaged---possesses some
characteristics that are perhaps less pronounced in other types of
social order.

\enlargethispage{\baselineskip}

First, interactional improprieties are typically sanctioned in an
indirect and inexplicit way. When an individual commits an interaction
offense, he still remains someone for whom respect must be shown;
interactionally speaking, to sanction him openly is only to make matters
worse. The punishment itself would be a crime. Thus we find many
interactions in which one or more participants are required to exercise
forbearance and to tolerate a sense that things are not going right or
that improprieties have been committed. Strategies and unofficial
communications must be employed as a means of responding to offenses in
an inoffensive way. Only the young who are not yet social persons can be
openly sanctioned for an interaction offense. The young have no social
face to lose, hence they can be openly criticized without producing the
embarrassing scene of someone losing face.

Secondly, there are many requirements of behavior which the actor in a
certain sense is not made morally responsible for. Thus a person who
disrupts euphoria by bragging may be indirectly sanctioned, but a person
who disrupts it because he has a tic or is cross-eyed is usually merely
avoided if possible, there being little desire on the part of the
avoiders for the avoidance to be taken as a sanction. We desire persons
to be unselfconscious and not to become flustered easily, but a certain
kind of guiltlessness attaches to those who offend in this way.

Finally, there is the paradoxical fact, less true, perhaps, of other
kinds of social order, that interactants are required to behave in a way
that is at once confirmative to an obligatory pattern and at the same
time spontaneous and unthinking. Here a voluntaristic scheme of analysis
in which unthinking response is a residual category seems somewhat
unsuitable. We must attempt to account for the uniformity of interactive
and expressive behavior, and its obligatory nature, by suggesting that
at some stage in the interactant's life proper affective conduct was
formally or informally impressed upon him; and we must attempt to
account for its spontaneous nature, and the requirement that it be
spontaneous, by suggesting that at the moment of interaction the
participant is so well and so deeply trained in the expressive patterns
of his group that he can conduct himself properly without thought.

\hypertarget{suggestions-for-research}{%
\section{Suggestions for Research}\label{suggestions-for-research}}

Experience in the field suggests that rules regarding communication
conduct are so automatically taken for granted, both by those who are
studied and by those who do the study, that it is convenient to depend
on extraordinary events to open our eyes to what ordinarily occurs. In
situations where ordinary spoken communication cannot prevail,
extraordinary arrangements with high visibility to the student are
required. On this assumption, then, some lines of further research in
spoken communication can be suggested.

First, classroom behavior seems a useful area for study because in a
classroom children can be observed who have not yet learned to keep
themselves in control or respect others and yet are sensitive to the
fact that they ought to conduct themselves in a mannerly way. In
addition, a classroom provides an excellent opportunity for an observer
to sit amidst interaction and take notes. Secondly, there are natural
field situations in which spoken communication regularly occurs and
regularly presents an interaction problem. In these situations a
fundamental requirement of interplay---a requirement that may be met
with ease and success in other situations---is not fulfilled or not
easily fulfilled; participants must give special attention to it and in
so doing often make it easier for the student to observe the
significance of it. Thus the requirement that participants act in a
spontaneous way is difficult to study in intimate family interaction,
because uncalculated involvement is apparently easy to maintain; in
staged interaction, as in that which occurs during television shows,
spontaneous involvement must be convincingly feigned under difficult
circumstances, providing a fruitful context in which to study the role
of spontaneity. So, too, the role of tact and emotional control, by
which participants conceal or overlook facts which might disrupt
euphoria, might be profitably studied in situations such as court
hearings, where persons are either not allowed to act tactfully or feel
obliged not to do so, for here the consequences of failure to exert tact
would be readily perceptible. Thirdly, it would be fruitful to study
types of interaction which were similar to spoken interaction in some
sense, but which provide very restricted examples of interplay. Examples
are found in the conversation-like interaction that occurs in the moves
and counter-moves of card and board games, fencing, wrestling, and the
like. These interplay-like activities provide simplified model-like
versions of spoken communication, the rules and conventions of the
activity highly restricting the type of messages and the type of conduct
that is allowed. Another fruitful context for study is to be found in
work situations which require a constant exchange of communication for
the guidance of work and yet which for some reason make it difficult to
employ spoken communication. Examples are found in the stock market, in
cargo-loading depots, on railroads, etc. In these contexts, too, a
simplified kind of interplay occurs, study of which might throw light on
more complicated speech systems. In all these contexts it would be
relatively easy to study the relationships among communication needs,
communication systems, and communication norms.

% BIBLIOGRAPHY
\chapter[BIBLIOGRAPHY]{Bibliography}
\label{ch:Bibliography}
\chaptermark{BIBLIOGRAPHY}

\begin{hangparas}{.25in}{1} 

Allport,\marginnote{\href{https://doi.org/10.32376/3f8575cb.573e204e}{doi}} Gordon, and Vernon, Philip. \emph{Studies in Expressive
Movements}. New York: Macmillan, 1933.

Bales, Robert F. \emph{Interaction Process Analysis}. Cambridge, Mass.:
Addison-Wesley Press, 1950.

---------and others. ``Channels of Communication in Small Group
Interaction,'' \emph{American Sociological Review}, XVI (1951),
461--468.

---------. ``The Equilibrium Problem in Small Groups.`` \emph{Working
Papers in the Theory of Action}, by Talcott Parsons, Robert F. Bales,
and Edward A. Shils. Glencoe, Ill.: The Free Press, 1953.

Barnard, Chester I. \emph{The Functions of the Executive}. Cambridge,
Mass.: Harvard University Press, 1947.

Bass, B. M. ``An Analysis of Leaderless Group Discussion.``
\emph{Journal of Applied Psychology}, XXXIII (1949), 527--533.

Bateson, Gregory. \emph{Naven}. Cambridge: Cambridge University Press,
1936.

---------, and Mead, Margaret. \emph{Balinese Character}. New York: New
York Academy of Sciences, 1942.

Becker, Howard S. ``The Professional Dance Musician in Chicago.''
Unpublished Master's thesis, Department of Sociology, University of
Chicago, 1949.

Beagler, Edmund. ``On the Resistance Situation: the Patient is Silent.''
\emph{Psychoanalytical Review}, XXV (1938), 170--186.

Blonder, Charles. \emph{Introduction à la Psychologie collective}.
Paris: Armand Colin, 1927.

Blumer, Herbert. ``Social Attitudes and Nonsymbolic Interaction.''
\emph{Journal of Educational Sociology}, IX (1936), 515--523.

Bond, D. D. \emph{The Love and Fear of Flying}. New York: International
Universities Press, 1952.

Bussard, J. H. S. ``Family Modes of Expression. `` \emph{American
Sociological Review}, X (1945), 226--237.

\emph{The Canons of Good Breeding: or the Handbook of the Man of
Fashion}. Philadelphia: Lee and Blanchard, 1839.

Chapel, Eliot D. "Measuring Human Relations.`` \emph{Genetic Psychology
Monographs}, XXII (1940), 3--147.

---------, and Coon, Carleton S. \emph{Principles of Anthropology}. New
York: Henry Holt, 1942.

---------, and Lindemann, E. ``Clinical Implications of
Interaction-Rates in Psychiatric Interviews.`` \emph{Human
Organization}, I (1942), 1--11.

Chesterfield, Earl of. \emph{Letters of Lord Chesterfield to His Son}.
Everyman's edition. New York: Dutton, 1929.

Cooley, Charles H. \emph{Human Nature and the Social Order}. New York:
Scriber's, 1922.

Critchley, Macdonald. \emph{The Language of Gesture}. London: Edward
Arnold, 1939.

Dalbiez, Roland. \emph{Psychoanalytical Method and the Doctrine of
Freud}. Translated by T. F. Lindsay. New York: Longmans, Green, 1941.

Darwin, Charles. \emph{The Expression of the Emotions in Man and
Animals}. London: John Murray, 1872.

Davidson, Levette J. ``Some Current Folk Gestures and Sign Languages,''
\emph{American Speech}, XXV (1950), 3--9.

Doyle, Bertram, \emph{The Etiquette of Race Relations in the South}.
Chicago: University of Chicago Press, 1937.

Durkheim, Emile. \emph{The Elementary Forms of the Religious Life}.
Translated by J. W. Swain. New York: Macmillan, 1926.

---------. ``Determination du Fait Moral. `` \emph{Sociologie et
Philosophie}. Paris: Presses Universities de France, 1951.

Efron, David. \emph{Gesture and Environment}. New York: King's Crown
Press, 1941.

Elias, Norbert. \emph{Uber den Prozess der Zivilisation}. Vol. I. Basel:
Haus Zum Falken, 1939.

Elliott, H. S. \emph{The Process of Group Thinking}. New York:
Association Press, 1928.

Festinger, Leon, and Thibaut, John. ``Interpersonal Communication in
Small Groups.'' \emph{Theory and Experiment in Social Communication}, by
Leon Festinger and others. Ann Arbor: Edwards Brothers, 1952.

Firth, Raymond. \emph{We, the Tikopia}. London: Allen and Unwon, 1936.

Fromm-Reichmann, Frieda. ``Notes on the Development of a Treatmens of
Schizophrenics by Psychoanalytical Psychotherapy.'' \emph{Psychiatry},
XI (1948), 263--273.

Gross, Edward. ``Informal Relations and the Social Organization of Work
in an Industrial Office.'' Unpublished Ph.D. dissertation, Department of
Sociology, University of Chicago, 1949.

Henderson, L. J. ``Physician and Patient as a Social System.'' \emph{New
England Journal of Medicine}, CCXII (1935), 1--15.

Homans, George C. \emph{The Human Group}. New York: Harcourt Brace,
1950.

Horsfall, A. B., and Arensberg, C. A. ``Teamwork and Productivity in a
Shoe Factory.'' \emph{Human Organization}, VIII (1949), 13--25.

Hu, Hsien Chin. ``The Chinese Concept of `Face.'\,`` \emph{American
Anthropologist}, n.s. XLVI (1944), 45--64.

Hughes, Everett C. ``Study of a Secular Institution: The Chicago Real
Estate Board.'' Unpublished Ph.D. dissertation, Department of Sociology,
University of Chicago, 1928.

---------. ``Dilemmas and Contradictions of Status.'' \emph{American
Journal of Sociology}, L (1945), 353--359.

Ichhesier, Gustav. ``Misunderstandings in Human Relations.'' Supplement
to \emph{The American Journal of Sociology}, LV (September, 1949).
Chicago: University of Chicago Press, 1949.

James, John. ``A Preliminary Study of the Size Determinant in Small
Group Interaction.`` \emph{American Sociological Review}, XVI (1951),
474--477.

Labarre, Weston. ``The Cultural Basis of Emotions and Gestures.''
\emph{Journal of Personality}, XVI (1947), 49--68.

Lasswell, Harold. \emph{Language and Politics}. New York: Stewart, 1949.

Malinowski, Bronislaw, Supplement I to \emph{The Meaning of Meaning}, by
C. K. Ogden and I. A. Richards. New York: Harcourt Brace, 1946.

Mead, Margaret. \emph{Soviet Attitudes Toward Authority}. New York:
McGraw-Hill, 1951.

Meerloo, J. A. M. \emph{Conversation and Communication}. New York:
International Universities Press, l952.

Mencken, H. M. \emph{The American Language, Supplement II}. New York:
Knopf, 1948.

Menninger, Karl. ``Purposive Accidents as an Expression of
Self-Destructive Tendencies.'' \emph{International Journal of
Psychoanalysis}, XVII (1936), 6--16.

Miller, Delbert C., and Form, William H. \emph{Industrial Sociology}.
New York: Harper, 1951.

Miller, George A. \emph{Language and Communication}. New York:
McGraw-Hill, 1951.

Morris, Charles. \emph{Signs, Language, and Behavior}. New York:
Prentice-Hall, 1946.

Newman, Stanley S. ``Personal Symbolism in Language Patterns.''
\emph{Psychiatry}, II (1939), 177--184.

---------. ``Behavior Patterns in Linguistic Structures.''
\emph{Language, Culture. and Personality}. Edited by Leslie Spier, A.
Hallowell, and Stanley S. Newman. Menasha, Wis.: Sapir Memorial
Publication Fund, 1941.

Obrdlik, A. J. ``Gallows Humor.'' \emph{American Journal of Sociology,}
XLVII (1942), 715--716.

Ogden, C. K., and Richards, I. A. \emph{The Meaning of Meaning}. New
York: Harcourt Brace, 1946.

Orwell, George. \emph{Down and Out in Paris and London}. London: Secker
and Warburg, 1949.

---------. \emph{Shooting an Elephant}. New York: Harcourt Brace, 1950.

Park, Robert Ezra. \emph{Race and Culture}. Glencoe, Ill.: The Free
Press, 1950.

Parsons, Talcott. \emph{The Structure of Social Action}. New York:
McGraw-Hill, 1937.

---------. \emph{The Social System}. Glencoe, Ill.: The Free Press,
1951.

Pear, T. H. \emph{The Psychology of Conversation}. London: Nelson, 1939.

Radcliffe-Brown, A. R. ``Taboo.'' (The Frazer Lecture, Cambridge, 1939.)
\emph{Structure and Function in Primitive Society}. London: Cohen and
West, 1952.

---------. ``A Further Note on Joking Relationships.'' \emph{Africa},
XIX (1949), 133--140.

Riezler, Kurt. ``Play and Seriousness.`` \emph{Journal of Philosophy},
XXXVIII (1936), 505--517.

Roethlisberger, Fritz J. ``The Foreman: Master and Victim of
Double-talk.'' \emph{Human Factors in Management}. Edited by S. D.
Hoslett. New York: Harper, 1946.

Roy, Donald. ``Quota Restriction and Goldbricking in a Machine Shop.``
\emph{American Journal of Sociology}, LVII (1952), 427--442.

Ruesch, Jurgen, and Bateson, Gregory. \emph{Communication}. New York:
Norton, 1951.

Ryle, Gilbert. \emph{The Concept of Mind}. London: Hutchinson's
University Library, 1949.

Sanford, F. H. ``Speech and Personality: A Comparative Case Study.''
\emph{Character and Personality}, X (1942), 169--198.

Sapir, Edward. \emph{Selected Writings of Edward Sapir}. Edited by David
G. Mandelbaum. Berkeley: University of California Press, 1951.

Schilder, Paul. ``The Social Neurosis.'' \emph{Psychoanalytic Review},
XXV (1938), 1--19.

Schwartz, Morris. ``Social Interaction of a Disturbed Ward of a
Hospital.'' Unpublished Ph.D. dissertation, Departmens of Sociology,
University of Chicago, 1951.

Simmel, Georg. \emph{The Sociology of Georg Simmel}. Translated and
edited by Kurt H. Wolff. Glencoe, Ill.: The Free Press, 1950.

Steinzer, B. ``The Development and Evaluation of a Measure of Social
Interaction.'' \emph{Human Relations}, II (1949), 103--121 and 319--347.

Stephen, Frederick F., and Mishler, Elliot Y. ``The Distribution of
Participation in Small Groups: An Exponential Approximation.''
\emph{American Sociological Review}, XVII (1952), 598--606.

Strodtbeck, F. L. ``Husband and Wife Interaction.'' \emph{American
Sociological Review}, XVI (1951), 468--473.

Sylvester Emmy. ``Analysis of Psychogenic Anorexia and Vomiting in a
Four-year-old Child.'' \emph{The Psychoanalytic Study of the Child}.
Vol. I. New York: International Universities Press, 1945.

Trollope, Frances M. \emph{Domestic Manners of the Americans}. 2 vols.
London: Whittaker, Treacher and Co., 1832.

Warner, W. Lloyd, and Low, J. O. \emph{The Social System of the Modern
Factory}. New Haven: Yale University Press, 1947.

Whiffman, Thomas . \emph{The North-West Amazons}. London: Constable,
1915.

Whorf, Benjamin Lee. ``Four Articles on Metalinguistics.`` (Reprinted
from \emph{Technology Review} and \emph{Language, Culture, and
Personality}.) Washington, D. C.: Foreign Service Institute, Department
of State, 1950.

Whyte, William F. \emph{Human Relations in the Restaurant Industry}. New
York: McGraw-Hill, 1948.

---------. ``Small Groups, Large Organizations.'' \emph{Social
Psychology at the Crossroads}. Edited by John Rohrer and Muzafer Sherif.
New York: Harper, 1951.

---------, and Gardner, Burleigh B. ``Facing the Foreman's Problems.''
\emph{Human Organization}, IV (1945), 1--17.



\end{hangparas}



\end{document}
