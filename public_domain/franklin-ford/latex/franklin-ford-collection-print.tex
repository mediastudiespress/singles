\documentclass[twoside,symmetric,nobib,justified]{tufte-book}

% \hypersetup{colorlinks=true,allcolors=[RGB]{97,15,11}}

% For print version
    %1. change document class \documentclass[twoside,symmetric,nobib,justified]
    %2. remove DOIs and comment out \hypersetup above      
    %3. Uncomment out the \geometry below
    %4. Go through the front matter; may need to use mix of, and movcing around, \begin{fullwidth} and \newgeometry \restoregeometry
    %5. Go through page by page, fixing footnotes, chapter titles, widows, etc.
    %6. If there are "Parts," make sure that they appear on the right hand side
    %7. If TOC is manual, check each start page
    %8. Export and look over, and shift back to the Create print PDF workflow

 \geometry{
  % showframe,
   paperwidth=8.5in,
   paperheight=11in,
   left=1.0in,
   right=0.6in,
   top=.62in,
    bottom=1in,
    includemp,
   includehead,
  % The text width and height are calculated automatically.
 }



% Book metadata
\title[Franklin Ford Collection]{Franklin Ford Collection}
\date{}
\author[edited by Dominique Trudel \& Juliette de Maeyer]{edited by Dominique Trudel \& Juliette de Maeyer}
\publisher{a mediastudies.press public domain edition}

% for references, with hanging indent 
\usepackage{hanging}

% restart footnotes each chapter
\let\oldchapter\chapter
\def\chapter{%
  \setcounter{footnote}{0}%
  \oldchapter
}

%my addition for toc
\newcounter{oldtocdepth}

\newcommand{\hidefromtoc}{%
  \setcounter{oldtocdepth}{\value{tocdepth}}%
  \addtocontents{toc}{\protect\setcounter{tocdepth}{-10}}%
}

\newcommand{\unhidefromtoc}{%
  \addtocontents{toc}{\protect\setcounter{tocdepth}{\value{oldtocdepth}}}%
}
\usepackage{hyperref}
\usepackage{bookmark}

% Just some sample text
\usepackage{lipsum}

% For nicely typeset tabular material
\usepackage{booktabs}

% Two other table packages
\usepackage{tabu}

\usepackage{longtable}

% For table spacing
\usepackage{verbatimbox}

% For graphics / images
\usepackage{graphicx}
\setkeys{Gin}{width=\linewidth,totalheight=\textheight,keepaspectratio}
\graphicspath{{graphics/}}

% The fancyvrb package lets us customize the formatting of verbatim
% environments.  We use a slightly smaller font.
\usepackage{fancyvrb}
\fvset{fontsize=\normalsize}


% Prints argument within hanging parentheses (i.e., parentheses that take
% up no horizontal space).  Useful in tabular environments.
\newcommand{\hangp}[1]{\makebox[0pt][r]{(}#1\makebox[0pt][l]{)}}

% Prints an asterisk that takes up no horizontal space.
% Useful in tabular environments.
\newcommand{\hangstar}{\makebox[0pt][l]{*}}

% Prints a trailing space in a smart way.
\usepackage{xspace}


\newcommand*{\justlastragged}{%
\leftskip=0pt plus 1fil
\rightskip=-\leftskip
\parfillskip=\leftskip
\parindent=0pt}

\newcommand{\TL}{Tufte-\LaTeX\xspace}

% Prints the month name (e.g., January) and the year (e.g., 2008)
\newcommand{\monthyear}{%
  \ifcase\month\or January\or February\or March\or April\or May\or June\or
  July\or August\or September\or October\or November\or
  December\fi\space\number\year
}


% Prints an epigraph and speaker in sans serif, all-caps type.
\newcommand{\openepigraph}[2]{%
  %\sffamily\fontsize{14}{16}\selectfont
  \sffamily\large
  \begin{doublespace}
  \noindent\allcaps{#1}\\% epigraph
  \noindent\allcaps{#2}% author
  \end{doublespace}
}

\usepackage{enumitem}
\setlist[enumerate]{itemsep=0mm}

% Inserts a blank page
\newcommand{\blankpage}{\newpage\hbox{}\thispagestyle{empty}\newpage}

\usepackage{units}

% Typesets the font size, leading, and measure in the form of 10/12x26 pc.
\newcommand{\measure}[3]{#1/#2$\times$\unit[#3]{pc}}

% Macros for typesetting the documentation
\newcommand{\hlred}[1]{\textcolor{Maroon}{#1}}% prints in red
\newcommand{\hangleft}[1]{\makebox[0pt][r]{#1}}
\newcommand{\hairsp}{\hspace{1pt}}% hair space
\newcommand{\hquad}{\hskip0.5em\relax}% half quad space
\newcommand{\TODO}{\textcolor{red}{\bf TODO!}\xspace}
\newcommand{\ie}{\textit{i.\hairsp{}e.}\xspace}
\newcommand{\eg}{\textit{e.\hairsp{}g.}\xspace}
\newcommand{\na}{\quad--}% used in tables for N/A cells
\providecommand{\XeLaTeX}{X\lower.5ex\hbox{\kern-0.15em\reflectbox{E}}\kern-0.1em\LaTeX}
\newcommand{\tXeLaTeX}{\XeLaTeX\index{XeLaTeX@\protect\XeLaTeX}}
% \index{\texttt{\textbackslash xyz}@\hangleft{\texttt{\textbackslash}}\texttt{xyz}}
\newcommand{\tuftebs}{\symbol{'134}}% a backslash in tt type in OT1/T1
\newcommand{\doccmdnoindex}[2][]{\texttt{\tuftebs#2}}% command name -- adds backslash automatically (and doesn't add cmd to the index)
\newcommand{\doccmddef}[2][]{%
  \hlred{\texttt{\tuftebs#2}}\label{cmd:#2}%
  \ifthenelse{\isempty{#1}}%
    {% add the command to the index
      \index{#2 command@\protect\hangleft{\texttt{\tuftebs}}\texttt{#2}}% command name
    }%
    {% add the command and package to the index
      \index{#2 command@\protect\hangleft{\texttt{\tuftebs}}\texttt{#2} (\texttt{#1} package)}% command name
      \index{#1 package@\texttt{#1} package}\index{packages!#1@\texttt{#1}}% package name
    }%
}% command name -- adds backslash automatically
\newcommand{\doccmd}[2][]{%
  \texttt{\tuftebs#2}%
  \ifthenelse{\isempty{#1}}%
    {% add the command to the index
      \index{#2 command@\protect\hangleft{\texttt{\tuftebs}}\texttt{#2}}% command name
    }%
    {% add the command and package to the index
      \index{#2 command@\protect\hangleft{\texttt{\tuftebs}}\texttt{#2} (\texttt{#1} package)}% command name
      \index{#1 package@\texttt{#1} package}\index{packages!#1@\texttt{#1}}% package name
    }%
}% command name -- adds backslash automatically
\newcommand{\docopt}[1]{\ensuremath{\langle}\textrm{\textit{#1}}\ensuremath{\rangle}}% optional command argument
\newcommand{\docarg}[1]{\textrm{\textit{#1}}}% (required) command argument
\newenvironment{docspec}{\begin{quotation}\ttfamily\parskip0pt\parindent0pt\ignorespaces}{\end{quotation}}% command specification environment
\newcommand{\docenv}[1]{\texttt{#1}\index{#1 environment@\texttt{#1} environment}\index{environments!#1@\texttt{#1}}}% environment name
\newcommand{\docenvdef}[1]{\hlred{\texttt{#1}}\label{env:#1}\index{#1 environment@\texttt{#1} environment}\index{environments!#1@\texttt{#1}}}% environment name
\newcommand{\docpkg}[1]{\texttt{#1}\index{#1 package@\texttt{#1} package}\index{packages!#1@\texttt{#1}}}% package name
\newcommand{\doccls}[1]{\texttt{#1}}% document class name
\newcommand{\docclsopt}[1]{\texttt{#1}\index{#1 class option@\texttt{#1} class option}\index{class options!#1@\texttt{#1}}}% document class option name
\newcommand{\docclsoptdef}[1]{\hlred{\texttt{#1}}\label{clsopt:#1}\index{#1 class option@\texttt{#1} class option}\index{class options!#1@\texttt{#1}}}% document class option name defined
\newcommand{\docmsg}[2]{\bigskip\begin{fullwidth}\noindent\ttfamily#1\end{fullwidth}\medskip\par\noindent#2}
\newcommand{\docfilehook}[2]{\texttt{#1}\index{file hooks!#2}\index{#1@\texttt{#1}}}
\newcommand{\doccounter}[1]{\texttt{#1}\index{#1 counter@\texttt{#1} counter}}

% Generates the index
\usepackage{makeidx}
\makeindex

\titlespacing*{name=\chapter,numberless}{0pt}{75pt}{10pt}

\titleclass{\part}{top} % make part like a chapter
\titleformat{\part}
[display]
{\centering\textit\normalfont\Huge}
{\vspace{3pt}\vspace{3pt} \thepart}
{15pt}
{\vspace{1pc}\Huge\itshape\MakeUppercase}
%
\titlespacing*{\part}{0pt}{175pt}{20pt}
%
  


\titleformat{\section}%
  {\normalfont\LARGE\scshape}% format applied to label+text
  {\llap{\colorbox{orange}{\parbox{1.5cm}{\hfill\color{white}\thesection}}}}% label
  {1em}% horizontal separation between label and title body
  {}% before the title body
  []% after the title body

\titleformat{\subsection}%
  {\normalfont\large\scshape}% format applied to label+text
  {\llap{\colorbox{orange}{\parbox{1.5cm}{\hfill\color{white}\thesection}}}}% label
  {1em}% horizontal separation between label and title body
  {}% before the title body
  []% after the title body

%%%% Kevin Goody's code for title page and contents from https://groups.google.com/forum/#!topic/tufte-latex/ujdzrktC1BQ
\makeatletter
\renewcommand{\maketitlepage}{%
\begingroup%
\setlength{\parindent}{10pt}

{\fontsize{24}{24}\selectfont\textit{\@author}\par}

\vspace{2in}{\fontsize{24}{52}{\allcaps{\@title}}\par}

\vspace{0.5in}{\fontsize{22}{24}{\textit{\@date}}\par}

\vfill{\fontsize{14}{14}\selectfont\smallcaps{\@publisher}\par}

\thispagestyle{empty}
\endgroup
}
\makeatother


\titlecontents{chapter}[.5in]
{\addvspace{.5pc}\Large\itshape}{}{}{~~\hfill\thecontentspage}[\ \ \ \ ]

\titlecontents{part}[0em]
{\addvspace{1.5pc}\LARGE\itshape}{}{}{}[\ \
 \ \ ]



\begin{document}

% Front matter
\frontmatter\pagenumbering{roman}\setcounter{page}{1}

% ONE full title page
\begin{fullwidth}


\maketitle

\end{fullwidth}

% TWO copyright page
\newpage

\begin{fullwidth}

~\vfill
\thispagestyle{empty}
\setlength{\parindent}{0pt}
\setlength{\parskip}{\baselineskip}
\emph{Franklin Ford Collection} contains previously unpublished works and works in the public domain.

\par Published by \smallcaps{mediastudies.press} in the \smallcaps{Public Domain} series

\par Original formatting, spelling, and citation styles retained throughout, with occasional {[}\emph{sic}{]} to indicate an uncorrected error.

\href{http://mediastudies.press}{mediastudies.press} | 414 W. Broad St., Bethlehem, PA 18018, USA

\par New materials are licensed under a Creative Commons Attribution-Noncommercial 4.0 (\href{https://creativecommons.org/licenses/by-nc/4.0/legalcode}{\smallcaps{CC BY-NC 4.0}})

\par \smallcaps{Cover design}: Mark McGillivray | Copy-editing \& proofing: Emily Alexander

\par \smallcaps{Credit for LaTeX template}: \href{https://www.overleaf.com/latex/templates/book-design-inspired-by-edward-tufte/gcfbtdjfqdjh}{Book design inspired by Edward Tufte}, by \href{https://ctan.org/pkg/tufte-latex}{The Tufte-LaTeX Developers}

\par \smallcaps{ISBN} 978-1-951399-22-1 (print) | \smallcaps{ISBN} 978-1-951399-19-1 (pdf)

\par \smallcaps{ISBN} 978-1-951399-21-4 (epub) | \smallcaps{ISBN} 978-1-951399-20-7 (html)

\par \smallcaps{DOI} \href{https://doi.org/10.32376/3f8575cb.80aee30a}{10.32376/3f8575cb.80aee30a}

\par \smallcaps{Library of Congress Control Number} 2023939244

\par\textit{Edition 1 published in July 2023}

\end{fullwidth}

% BLANK PAGE

\newpage
\thispagestyle{plain} % empty
\mbox{}

% BLANK PAGE

\newpage
\thispagestyle{plain} % empty
\mbox{}

% TOC

\newpage
\thispagestyle{empty}

\begin{fullwidth}



\vspace*{1.1in}

{\noindent\fontsize{32}{24}\selectfont\textit{Contents}\par}

\vspace{.75in}

\setlength{\parindent}{25pt}
{\fontsize{14}{12}\selectfont{``He Has Ideas about Everything'':\par}}

\vspace{.1in}

\setlength{\parindent}{25pt}
{\fontsize{14}{12}\selectfont{An Introduction to the \emph{Franklin Ford Collection}\hfill vii\par}}

\vspace{.25in}

\setlength{\parindent}{25pt}
{\fontsize{14}{12}\selectfont{Acknowledgments\hfill xlvii\par}}

\vspace{.25in}

\setlength{\parindent}{25pt}
{\fontsize{14}{12}\selectfont{The Larger Life: A Poem Dedicated to Franklin Ford \hfill xlix\par}}

\vspace{.5in}

{\noindent\fontsize{20}{24}\selectfont\textit{I. Reforming the News}\par}

\vspace{.25in}

{\fontsize{14}{12}\selectfont{Draft of Action}\hfill 5\par}

\vspace{.25in}

{\fontsize{14}{12}\selectfont{A Newspaper Laboratory}\hfill 61\par}

\vspace{.25in}

{\fontsize{14}{12}\selectfont{Banding Together the Leading Newspapers}\hfill 65\par}

\vspace{.25in}

{\fontsize{14}{12}\selectfont{The Press of New York--Its Future}\hfill 67\par}

\vspace{.25in}

{\fontsize{14}{12}\selectfont{Organization of Intelligence Requires an Organism}\hfill 71\par}

\vspace{.25in}

{\fontsize{14}{12}\selectfont{In Search of Absolute News, Sensation, and Unity}\hfill 73\par}

\vspace{.25in}

{\fontsize{14}{12}\selectfont{The News System: A Scientific Basis for Organizing the News}\hfill 105\par}

\end{fullwidth}

\newpage
\thispagestyle{empty}

\begin{fullwidth}

\vspace*{1.1in}

{\noindent\fontsize{20}{24}\selectfont\textit{II. Interconnected Flows: Money, Information, and\\\noindent Transportation}\par}

\vspace{.25in}

{\fontsize{14}{12}\selectfont{Better Credit Reporting}\hfill 115\par}

\vspace{.25in}

{\fontsize{14}{12}\selectfont{Traffic Associations}\hfill 127\par}

\vspace{.25in}

{\fontsize{14}{12}\selectfont{The Country Check}\hfill 131\par}

\vspace{.25in}

{\fontsize{14}{12}\selectfont{The Express Companies and the Bank}\hfill 147\par}

\vspace{.25in}

{\fontsize{14}{12}\selectfont{The Mercantile Agencies and Credit Reporting}\hfill 155\par}

\vspace{.25in}

{\fontsize{14}{12}\selectfont{Co-operative Credit Reporting}\hfill 163\par}

\vspace{.5in}

{\noindent\fontsize{20}{24}\selectfont\textit{III. News is Government}\par}

\vspace{.25in}

{\fontsize{14}{12}\selectfont{City News Office Needed}\hfill 167\par}

\vspace{.25in}

{\fontsize{14}{12}\selectfont{Municipal Reform: A Scientific Question}\hfill 171\par}

\vspace{.25in}

{\fontsize{14}{12}\selectfont{Government is the Organization of Intelligence or News}\hfill 199\par}

\vspace{.25in}

{\fontsize{14}{12}\selectfont{The Simple Idea of Government}\hfill 207\par}

\vspace{.25in}

{\fontsize{14}{12}\selectfont{A New and Revolutionary Government}\hfill 209\par}

\vspace{.25in}

{\fontsize{14}{12}\selectfont{News is the Master Element of Social Control}\hfill 215\par}



\end{fullwidth}

% ``HE HAS IDEAS ABOUT EVERYTHING'': AN INTRODUCTION TO THE FRANKLIN FORD COLLECTION
\chapter[``He Has Ideas about Everything'': An Introduction to the \emph{Franklin Ford Collection}]{``He Has Ideas about Everything'':\\\noindent An Introduction to the \emph{Franklin\\\noindent Ford Collection}}
\label{ch:``He Has Ideas about Everything'': An Introduction to the Franklin Ford Collection}
\chaptermark{``HE HAS IDEAS ABOUT EVERYTHING'': AN INTRODUCTION TO THE FRANKLIN FORD COLLECTION}

\vspace{0.2in}

\begin{LARGE}
    

\smallcaps{Dominique Trudel \& Juliette De Maeyer}\marginnote{\emph{This collection would not have been possible without the hard work
of Amandine Hamon and Simona Feng, to whom we offer our warmest thanks.}}

\end{LARGE}


\vspace{0.5in}



\newthought{On April 13, 1886},  a lively debate took place before the members of the
Nineteenth Century Club in New York. During a conference discussing the
press, one participant asserted the surprising opinion that the
newspapers were not as good as those of fifty years before.\footnote{``Talking
  about Newspapers,'' \emph{Democrat and Chronicle}, April 16, 1886;
  ``The Week in Society,'' \emph{New York Tribune}, April 18, 1886.} At
the dawn of the Progressive Era, such beliefs were not shared by the
majority, and were certainly not common among journalists. For the first
time in history, an extensive coverage of fresh international news was
possible, thanks to the cables of the Associated Press and the like.
Reporting was becoming a self-conscious and esteemed occupation in
American cities, and reporters were generally greeted with kudos, as
readers enjoyed the exotic adventures of the many star journalists and
``girl stunt reporters'' of the era.

The surprising comment came from the mouth of Franklin Ford
(1849--1918), the editor of the \emph{Bradstreet's Journal of Trade,
Finance, and Public Economy}. A seasoned newsman, Ford was then
embarking on a long reflection on journalism, media, and communication.
Over the next three decades, he gave conferences, published essays, and
discussed his ideas with many high-profile correspondents, including
Supreme Court Justice Oliver Wendell Holmes Jr., Columbia University
librarian James H. Canfield, and legal scholar Thomas M. Cooley. He also
launched (or participated in) many publications and schemes aimed at
changing the news, politics, education, finance, and society at
large---some of these based on what he called the ``movement of
intelligence'' or the ``triangle of intelligence.''

Today Franklin Ford is mostly known for his involvement in the
\emph{Thought News} project at the University of Michigan, alongside
John Dewey, Robert Ezra Park, George Herbert Mead, Charles Horton
Cooley, and Fred Newton Scott, as well as one of Ford's brothers,
Corydon. Between 1888 and 1892, the group planned to launch a
revolutionary ``philosophical newspaper,'' called \emph{Thought News---A
Journal of Inquiry and a Record of Fact}. According to a circular
printed in the \emph{University of Michigan Daily}:

\begin{quote}
\emph{Thought News} has but one thing to report and that is a mere
announce-\\\noindent ment---the announcement that the social organism is here. . . .
If the social organism is a fact, and not a poetic dream, it must be
studied like a steam engine, in its principle and in its practical
activity. . . . So the chasm between education and life, between theory
and practice, is bridged over once and forever.\footnote{``The Thought
  News,'' \emph{University of Michigan Daily}, April 8, 1892.}
\end{quote}

\noindent The ambitious project eventually failed, and \emph{Thought News} was
never published. In the aftermath, Ford wrote a fifty-eight-page
manifesto, \emph{Draft of Action}, in which he outlined the grand plan
behind \emph{Thought News}. Ford wrote passionately about ``systematic
inquiry'' (or ``full social inquiry'') and the nitty-gritty of a
national organization that would publish dozens of publications and
connect businesses, universities, and public authorities.

The \emph{Thought News} episode attracted the attention of many
scholars, who mostly approached the work of Ford in connection with the
theorization of media and communication developed within classical
American pragmatism and the Chicago School tradition. If this angle
turns out to be relevant and legitimate, it still seems insufficient.
The events surrounding the failure of \emph{Thought News} and the role
played by its participants are not well-known. The handful of primary
sources readily available only attest to a very small portion of Ford's
evolving ideas and many activities. The focus on the relationship
between Ford and Dewey, and on the influence of Ford over Dewey, not
only obscures other topics and key actors, but also contributes to an
unfairly negative picture of Ford, who was later called a ``scoundrel''
by Dewey.\footnote{Zena Beth McGlashan, ``The Professor and the Prophet:
  John Dewey and Franklin Ford,'' \emph{Journalism History} 6\emph{,}
  no. 4 (1979): 107--23, on 109.}

Other contemporaries had a different opinion of Ford's work and
personality. In 1896, a Detroit newspaper piece presented Ford as the
``conundrum of the day'' and draws this mysterious portrait:

\begin{quote}
He has ideas about everything, you know---about the bank, the newspaper
and the schools. He's a curious fellow, and very interesting. Let me
tell you. He has a lot of words that he always uses. Here are some of
them: Protection, publicity, unity, verbalism, post office, telephone,
newspaper and ``into relation.'' He always wants to bring everything
``into relation.''\footnote{``Who is He?'' \emph{Detroit Evening News},
  December 24, 1896.}
\end{quote}

\noindent To date, there is no archival collection dedicated to Ford. The exact
extent of Ford's work remains unknown, as his papers were destroyed when
a fire wrecked his Columbia University office in October
1914.\footnote{``Blaze Ends Fire Peril at Columbia,'' \emph{New York
  Tribune}, October 11, 1914.} The remaining records are scarce, and
scattered across dozens of archival collections, some of them yet to be
digitized. In the last couple of years, through interlibrary loans,
microfilms, and devoted research assistants and archivists, our
collection of documents slowly took shape, and at the time of writing,
amounts to fifty-five documents authored by Ford, as well as 111
documents about Ford's endeavors.

The \emph{Franklin Ford Collection} offers a curated tour of Ford's
writing, and aims at putting them ``into relation'' for the very first
time. The nineteen texts of the collection were selected for many (and
sometimes diverging) reasons. While some are typical of Ford, others are
uncharacteristic. Some stand out for the density of the theoretical
arguments, while others attest to Ford's little-known professional
trajectory. Overall, the documents expose the three core themes
developed by Ford throughout the years: 1) the specific problems of the
press and the many remedies he envisioned; 2) the interconnected flows
of money, transportation, and communication central to modern industrial
societies; and 3) the political and social theory that lay behind Ford's
projects and which became more explicit in his later years.

These documents were carefully transcribed and are mostly presented ``as
is,'' in order to allow the expression of Ford's unique style and
lexicon. We only corrected obvious spelling errors and harmonized the
punctuation (quotation marks, dashes, hyphens, etc.) and the overall
presentation (line spacing, numbering, headings, etc.) in order to
improve the readability of the documents. We also added dozens of
explanatory notes that help to contextualize the content of the
documents, identify Ford's sources, and bring things ``into relation,''
in line with Ford's own intellectual habit. As the Portable Document
Format (PDF) of all original files are included in this collection,
making these minor adjustments seemed the appropriate editorial
approach. We encourage readers to examine the PDFs, which present many
interesting details, including letterheads, illustrations, and colors
that situate them more precisely in their historical moment.

In order to better understand this material, this introductory chapter
will first turn to the existing historiography. Who has written about
Ford? And to say what? This survey will allow us to put into perspective
the various issues raised by (or related to) Ford and his work. We then
offer a detailed biographical sketch, with specific attention to Ford's
social, intellectual and historical context, and proceed to give readers
an initial overview of the themes they will find in the collection. This
introduction concludes with insights that emerged from close reading of
our archive and the content of this collection, including Ford's
eclectic intellectual lineages and role in linking proto-pragmatism with
what has now become the canonical way in which media and communication
scholars understand pragmatism and the Chicago School of social thought.

\hypertarget{historiographical-knots}{%
\section{Historiographical Knots}\label{historiographical-knots}}

The \emph{Thought News} episode is undoubtedly the gravitational center
around which the existing literature on Ford revolves. What could be
more understandable, when big parts of what is known about Ford come
from Dewey scholars, interested in this peculiar episode in Dewey's
youth?

\newpage Morton White's \emph{Origin of Dewey's Instrumentalism}, published in
1942, is possibly the first scholarly account of the \emph{Thought News}
episode. White writes that a letter in which Dewey describes his
encounter with Ford ``presages Dewey's break with idealism'' and
announced an early version of the pragmatist theory of truth. Hence,
according to White, ``the work of Ford is fascinating not only in its
own right but all the more so because of its impact on
Dewey.''\footnote{Morton G. White, \emph{The Origin of Dewey's
  Instrumentalism} (New York: Columbia University Press, 1943), 101--2.}

Willinda Savage, likewise treating \emph{Thought News} as a key moment
in Dewey's early intellectual development and interest in communication,
describes the controversy caused by the project in great detail. She
also reproduced a letter sent to her by Dewey, reflecting back on
\emph{Thought News} sixty years later: ``No issue was made; it was an
over-enthusiastic project. . . . {[}T{]}he \emph{idea} was advanced for
those days, but it was too advanced for the maturity of those who had
the idea in mind.''\footnote{John Dewey's letter quoted in Willinda
  Savage, ``John Dewey and Thought News at the University of Michigan,''
  \emph{Michigan Alumnus Quarterly Review} 56 (1950): 204--9, on 209.}
In line with White and Savage's argument, Dewey scholars from the 1950s
and 1960s commonly consider \emph{Thought News} as a pivotal point in
Dewey's philosophical trajectory.\footnote{See John Blewett, ``Democracy
  as Religion: Unity in Human Relations,'' in \emph{John Dewey: His
  Thought and Influence Influence}, ed. John Blewett, 33--58 (New York:
  Fordham University Press, 1960); George Dykhuizen, ``John Dewey at the
  University of Michigan,'' \emph{Journal of the History of Ideas} 23,
  no. 4 (1962): 513­--64; and Neil Coughlan, \emph{Young John Dewey}
  (Chicago: University of Chicago Press, 1975).}

Among these, sociologist Lewis Feuer interestingly emphasizes the
political dimension of \emph{Thought News}. He describes the group that
coalesced around Dewey as ``leftist,'' and Ford as their ``prophet.'' As
a result, Feuer underlines the religious and revolutionary underpinnings
of \emph{Thought News,} described as a ``socially messianic newspaper''
and ``an instrument for realizing socialism.''\footnote{Lewis Feuer,
  ``John Dewey and the Back to the People Movement in American
  Thought,'' \emph{Journal of the History of Ideas} 20, no. 4 (1959):
  545--68, on 549--50.} Pointing to Ford's dismissal of the ruling
class, Feuer's analysis remains unique in the historiography. It has
been praised for pointing to a crucial episode in Dewey's trajectory, or
dismissed for its ``rather simplistic description of Dewey as a
`socialistic mystic.'\,''\textsuperscript{10} Earl James Weaver's 1963
unpublished PhD dissertation offers a detailed analysis of the Ford
brothers' intellectual influences, among them Auguste Comte, Henry
George, and Lester Frank Ward,\marginnote{\textsuperscript{10}\setcounter{footnote}{10} Daniel J. Czitrom, \emph{Media and
  the American Mind: From Morse to McLuhan} (Chapel Hill: University of
  North Carolina Press, 1982), 214.} as well as a cross-reading of the 1890s
writings of brothers Corydon and Franklin Ford, John Dewey, and William
James. Pointing to a broad range of converging themes and arguments,
Weaver argues that ``the influence of the Fords on Dewey was evident in
almost everything he did or wrote from about 1890 on.''\footnote{Earl
  James Weaver, ``John Dewey: A Spokesman for Progressive Liberalism''
  (PhD diss., Brown University, 1963), 66.} That includes Dewey's
pedagogical shift, beginning in 1889, from core philosophy courses (on
Plato, Hegel, etc.) to courses on methods, ethics, and political
philosophy, as well as his final departure from the University of
Michigan in 1894, which is said to have partly resulted from the
embarrassment and tensions prompted by \emph{Thought News}.


Later book-length studies continued to dig deeper into the intellectual
history around Dewey.\footnote{See Robert Westbrook, \emph{John Dewey
  and American Democracy} (Ithaca, NY: Cornell University Press, 1991);
  Andrew Feffer, \emph{Chicago Pragmatists and American Progressivism}
  (Ithaca, NY: Cornell University Press, 1993); Steven C. Rockefeller,
  \emph{John Dewey: Religious Faith and Democratic Humanism} (New York:
  Columbia University Press, 1991); and Jay Martin, \emph{The Education
  of John Dewey} (New York: Columbia University Press, 2002).} Among
these early 1990s works, Steven C. Rockefeller's analysis stands out as
one of the most detailed accounts of Dewey's embroilment with Ford's
ideas. Rockefeller argues that Dewey was trying to harmonize his
philosophic method---then still close to Hegel's idealism---with a
scientific approach, and that Ford provided a practical solution.
Rockefeller carefully situates most of Dewey's intellectual production
of the late 1880s and early 1890s in direct connection with this aim and
with the \emph{Thought News} project. Dewey's work on logic, ethics,
poetry, and even his classes would bear the imprimatur of Ford's style
and ideas, and lead Dewey to write ``prophetically in grand historical
terms.''\footnote{Rockefeller, \emph{John Dewey}, 177.} Then at the apex
of his neo-Hegelian period, Dewey argued that the old divisions of
science and spiritual values, knowledge and practice, were overcome by
history. ``The secret of this movement,'' Dewey wrote, is ``a single,
comprehensive, and organizing unity.''\footnote{John Dewey, quoted in
  Rockefeller, \emph{John Dewey}, 178.} Rockefeller makes a similar
argument regarding the work of Fred Newton Scott, as his 1892 lecture on
``Christianity and the Newspaper'' pleaded for the newspaper reform
envisioned by Ford. Scott argued that such a reformed newspaper could be
``the voice of the real, the living Christ.''\footnote{Fred Newton
  Scott, quoted in Rockefeller, \emph{John Dewey}, 189.}

It is also in Dewey's shadow that Ford entered the field of
communica-\\\noindent tion---notably in efforts to refine and sometimes redefine the
field's intellectual history and genealogy. Ford's admission into the
history of communication research was orchestrated by none other than
James W. Carey, in his 1970s attempt to rediscover the ``Chicago
School'' tradition.\footnote{Carey's attempt to revive the Chicago
  School tradition was part of a larger movement cutting across American
  social sciences and humanities. Among notable works of the era that
  contributed to the rediscovery and revival of the Chicago School, see
  Jean B. Quandt, \emph{From the Small Town to the Great Community} (New
  Brunswick: Rutgers University Press, 1970); Fred H. Matthews,
  \emph{Quest for an American Sociology: Robert E. Park and the Chicago
  School} (Montreal: McGill-Queen's University Press, 1977); and
  Winifred Raushenbush, \emph{Robert E. Park: Biography of a
  Sociologist} (Durham: Duke University Press, 1979).} Carey was then
invested in a campaign of disciplinary reformulation, which profoundly
shook communication research. Imported by Carey from more prestigious
neighboring fields, the Chicago School storyline made its way into
disciplinary memory, and so did Chicago-inspired research.\textsuperscript{17}

In 1976, an unpublished paper authored by Carey and his PhD student
Norman Sims tracked down the \emph{Thought News} episode and stitched
together a biographical profile of Ford based on primary and secondary
sources.\textsuperscript{18} This essay is
one of the first to cite Ford\textquotesingle s \emph{Draft of Action}
extensively, alongside some of Ford's other little-known opuscules. It
also attributes to Ford some publications under his full Christian name,
William F. Ford, which Ford used until the late 1870s.

In\marginnote{\textsuperscript{17} Karin
  Wahl-Jorgensen, ``The Chicago School of Sociology and Mass
  Communication Research,'' in \emph{The International Encyclopedia of
  Media Studies: Media History and the Foundations of Media Studies},
  ed. John Nerone, 554--77 (London: Blackwell, 2013); Jefferson Pooley,
  \emph{James W. Carey and Communication Research: Reputation at the
  University's Margins} (New York: Peter Lang, 2016).} Carey\marginnote{\textsuperscript{18}\setcounter{footnote}{18} James W. Carey and Norman Sims, ``The Telegraph and
  the News Report'' (paper, Annual Meeting of the Association for
  Education in Journalism, College Park, MD, August 1976). According to
  their bibliography, Carey and Sims based their analysis on six
  documents authored by Ford, spanning from 1874 to 1903.} and Sims's account, Ford epitomizes a ``scientific'' approach
to journalism that aims to rationalize and centralize fact-gathering at
the national level, thanks to the telegraph and telephone. They opposed
Ford's view to a literary model, emphasizing ``the integrity of
feelings, personal observations and opinions, and an essentially local
and individualistic organization of society.''\footnote{Carey and Sims,
  ``The Telegraph,'' 4.} According to Carey and Sims, Ford exerted a key
influence on John Dewey, Robert Ezra Park, and the ``Chicago School'' in
the 1920s and 1930s, to the extent that the Chicago School became a
synthesis between the ``scientific'' and the ``literary'' views.
``Ford's writings introduced a temper of thought that had a vast and
direct influence on the only group of American scholars to take the
newspaper seriously, the Chicago School of Social Thought.''\footnote{Carey
  and Sims, ``The Telegraph,'' 5.} Such direct influence is manifest,
Carey and Sims added, in Dewey's \emph{The Public and its Problems}
(1927), which ``restated Ford's main ideas'' with the addition of ``a
concern for the integrity of communities and neighborhoods that was more
characteristic of the literary perspective.''\footnote{Carey and Sims,
  ``The Telegraph,'' 29--30.} Ford's touch is also to be found in Park's
classical article ``The Natural History of the Newspaper,'' in which
Carey and Sims saw echoes of Ford's earlier reference to the ``natural
history'' of governments, among other things.\footnote{Carey and Sims,
  ``The Telegraph,'' 32.}

This early rendering of Ford and the Chicago School is quite different
from Carey's later analysis.\footnote{Pooley, \emph{James W. Carey}.} In
a series of subsequent essays, Carey abandoned his earlier
characterization of the Chicago School as a synthesis between a
scientific and a literary model, and clearly associated the Chicago
School with the literary tradition alone.\footnote{James W. Carey,
  \emph{Communication as Culture} (Boston: Unwin Hyman, 1989); James W.
  Carey, ``The Chicago School and Mass Communication Research,'' in
  \emph{American Communication Research: The Remembered History}, ed.
  Everette E. Dennis and Ellen Ann Wartella, 21--38 (Mahwah: Lawrence
  Erlbaum, 1996).} The scientific approach would then be epitomized by
Walter Lippmann, with Dewey and the Chicago School positioned as his
rivals.

Although the exact contours of the Chicago School and Ford's role in it
took on different guises as Carey told an evolving story, \emph{Thought
News} remained a stable cornerstone. Carey considered that \emph{Thought
News} was nothing less than the founding event of American communication
research:

\begin{quote}
American research and scholarship on communication began as a cumulative
tradition in the late 1880s when five people came together in Ann Arbor,
Michigan. Two were young faculty---John Dewey and George Herbert
Mead---and two were students at the time---Robert Park and Charles
Cooley. The final element of the pentad was an itinerant American
journalist by the name of Franklin Ford, who shared with Dewey---indeed,
cultivated in him---the belief that ``a proper daily newspaper would be
the only possible social science.''\footnote{James W. Carey, ``Culture,
  Geography, and Communications: The Work of Harold Innis in an American
  Context,'' in \emph{Culture, Communication and Dependency}, ed.
  William H. Melody, Liora Salter, and Paul Heyer, 73--91 (Norwood:
  Ablex Publishing, 1981), on 74. The quote was later republished in
  Carey, ``Communication,'' 110.}
\end{quote}

\noindent In a slightly different narrative, Carey also recast the same characters
in a story of the origins of American sociology and modern journalism:

\begin{quote}
Systematic American Sociology and modern journalism were, to a certain
extent, twin born. When John Dewey, George Herbert Mead, Charles Cooley,
and Robert Park were joined in Ann Arbor in the 1880s by a curious
itinerant journalist, Franklin Ford, modern sociology and the desire for
a scientific, objective journalism began an implicit and reflexive
development.\footnote{James W. Carey, ``Review: \emph{The Discovery of
  Objectivity},'' \emph{American Journal of Sociology} 87, no. 5 (1982):
  1182--88, on 1182.}
\end{quote}

Following up on Carey and Sims, Zena Beth McGlashan, then a graduate
student at the University of Iowa (where Carey was named professor in
1976), contributed two pieces about \emph{Thought News}.\footnote{Zena
  Beth McGlashan, ``John Dewey and News,'' \emph{Journal of
  Communication Inquiry} 2, no. 1 (1976): 3­--14; McGlashan, ``The
  Professor.''} Her analysis pointed to the crucial influence of the
episode on both John Dewey and Robert Park, as well as to its
premonitory dimension, writing that ``Ford was anticipating polling
service, corporate and governmental information officers, a complex
specialized press---all elements which contribute to what is called
today `the communication explosion.'\,''\footnote{McGlashan, ``The
  Professor,'' 111.} Published shortly after, Daniel Czitrom's
\emph{Media and the American Mind}, which took up key aspects of Carey's
narrative,\footnote{Jefferson Pooley, ``Daniel Czitrom, James W. Carey,
  and the Chicago School,'' \emph{Critical Studies in Media
  Communication} 24, no. 5 (2007): 469--72.} focused on Ford's influence
over the Ann Arbor group and offered a detailed overview of Ford's
\emph{Draft of Action}. It also emphasized Ford's eccentricity,
describing him as a ``quixotic man,''\footnote{Czitrom, \emph{Media},
  104.} in line with prior renderings but departing significantly from
Carey and Sims, who soberly described Ford as an ``economic
journalist.''\footnote{Carey and Sims, ``The Telegraph,'' 5.}

John Durham Peters also contributed a substantial analysis of the
\emph{Thought News} episode, which situated the initiative as an early
manifestation of the\newpage\noindent progressive fascination with expertise.\footnote{John
  Durham Peters, ``Reconstructing Mass Communication Theory'' (PhD
  diss., Stanford University, 1986); John Durham Peters, ``Satan and
  Savior: Mass Communication in Progressive Thought,'' \emph{Critical
  Studies in Mass Communication} 6, no. 3 (1989): 247--63.} Peters
positioned \emph{Thought News} in the intellectual lineage of Herbert
Spencer and Auguste Comte, though without their elitism, and in parallel
with the work of French sociologist Gabriel Tarde, who also imagined a
revolutionary newspaper in the early 1890s. In Peters's analysis, the
key feature of \emph{Thought News} was to provide society with an
accurate image of itself, realizing what Spencer dubbed the ``coherent
heterogeneity of society'' and Comte the ``positivist age.'' Central to
the project was ``the wish to socialize the means of intellectual
production to make each citizen, as it were, social
scientists.''\footnote{Peters, ``Satan,'' 252.} More recently, Peters
described \emph{Thought News} as a ``daily updated
encyclopedia,''\footnote{Kenneth Cmiel and John Durham Peters,
  \emph{Promiscuous Knowledge} (Chicago: University of Chicago Press,
  2020), 47.} an interesting line of analysis which nevertheless
neglects the odd periodicity of the publication, which was to ``appear
as often as the material at hand warrants'' and had been mocked at the
time for that reason.\footnote{\emph{Detroit Tribune,} April 10, 1892.}
Peters underlined Ford's strange personality, describing him as ``a sort
of crackpot journalist-philosopher,'' but he also insisted on situating
Ford in the intellectual context of the era.\footnote{Peters, ``Satan,''
  253.}

Starting in the mid-1990s, mentions of Ford and \emph{Thought News}
became more frequent, as the episode seemed to become part of the
field's remembered past. Ford and \emph{Thought News} found their way
into widely read books, such as Dan Schiller's \emph{Theorizing
Communication} and Wilbur Schramm's posthumously published
memoirs.\footnote{Dan Schiller, \emph{Theorizing Communication} (New
  York: Oxford University Press, 1996); Wilbur Schramm, \emph{The
  Beginnings of Communication Studies in America} (Thousand Oaks: Sage,
  1997).} Schiller carefully positions Ford's concept of
``intelligence'' in the intellectual and political context of the
\emph{fin de siècle} era and suggests interesting parallels with Edward
Bellamy's utopian novel \emph{Looking Backward,} published in 1888.
Schramm describes \emph{Thought News} as a newspaper reporting changes
in public opinion that anticipated by thirty years the development of
polling and survey research. In Schramm's slightly revised narrative of
the history of communication research, Ford strategically plays the role
of a ``forefather'' of the field's ``four founding fathers'' (Harold
Lasswell, Paul Lazarsfeld, Kurt Lewin, and Carl Hovland).

\enlargethispage{\baselineskip}

Among notable recent works, Andrej Pinter further develops Peters's 1989
argument regarding the connection with the ideas of Gabriel Tarde, and
also emphasizes the parallel with the work of Albert Shäffle.\footnote{Andrej
  Pinter, ``Thought News a Quest for Democratic Communication
  Technology,'' \emph{Javnost---The Public} 10, no. 2 (2003): 93--104.}
While both favored an organicist theory of society in which the press
had to play a regulating role, Tarde would have exerted a direct
influence on the project, while the contribution of Schäffle's ideas
remained uncertain.\footnote{Considering that Ford and Dewey met in
  1888, that Tarde's \emph{Les lois de l'imitation} was only published
  in 1890 (and translated in English in 1903), and also that both Dewey
  and Robert Park were later critical of Tarde, Lana Rakow concludes
  that direct influence of Tarde on the project is unlikely. See Lana F.
  Rakow, \emph{John Dewey: A Critical Introduction to Media and
  Communication Theory} (New York: Peter Lang, 2003): 77­--78. Ford would
  later refer to Tarde\textquotesingle s work, writing that he was
  acting ``consciously on Tarde\textquotesingle s vision.'' See David H.
  Burton, \emph{Progressive Masks: Letters of Oliver Wendell Holmes Jr.,
  and Franklin Ford} (Newark\emph{:} University of Delaware Press,
  1982), 45.} Jeremiah Dyehouse, for his part, draws attention to the
role played by Fred Newton Scott, a professor of English at the
University of Michigan, in the \emph{Thought News} adventure, and
insists on the preoccupation with ``good writing'' (as a contribution to
the social organism) that was shared by Scott and Dewey.\textsuperscript{40}

Despite the relatively abundant secondary literature, Ford remains
paradoxically shrouded in mystery. The dominant focus on the
\emph{Thought News} episode and, within it, on the relationship between
Ford and Dewey, paints Ford into an uncomfortable corner, playing the
role of a secondary character that only mattered through his alleged
influence on Dewey. But there is more\marginnote{\textsuperscript{40}\setcounter{footnote}{40} Jeremiah
  Dyehouse, ``Theory in the Archives; Fred Newton Scott and John Dewey
  on Writing the Social Organism,'' \emph{College English} 76, no. 4
  (2004): 248--68.} to Ford than that---which our
research and the texts included in this collection aim to show.

In doing so, the collection reaffirms the need for and pertinence of an
archival-based approach to the intellectual history of media and
communication research.\footnote{Jefferson Pooley, ``The New History of
  Mass Communication Research,'' in \emph{The History of Media and
  Communication Research: Contested Memories}, ed. David Park and
  Jefferson Pooley, 43--69 (New York: Peter Lang, 2008).} While
historical narratives about the history of the field are often motivated
by complex disciplinary and epistemological politics, primary sources
have the important function to help keep in touch with the \emph{terra
firma} of the historical record. If we do not fully share Franklin
Ford's somewhat naive conception of ``primary facts'' and enthusiasm for
their centralization, we want to contribute to the ongoing
professionalization and sophistication of the field's historiography by
providing easy access to (and a first assessment of) historical
documents whose full interpretation remains an open-ended process.

Among the many interesting works about Ford, one stands out and serves
as an inspiration for this project. Historian David H. Burton carefully
edited and published the correspondence of Franklin Ford and Supre-\\\noindent me
Court Justice Oliver Wendell Holmes, Jr.\footnote{Burton,
  \emph{Progressive Masks.} See also David H. Burton, ``The Curious
  Correspondence of Justice Oliver Wendell Holmes and Franklin Ford,''
  \emph{The New England Quarterly} 53, no. 2 (1980): 196--211.} The two
regularly exchanged letters from 1907 to 1918, and Ford often discussed
his projects and elaborated his theories in letters longer than ten
pages, single-spaced. The letters not only highlighted Ford's
little-known activities after \emph{Thought News}---pointing to an array
of other correspondents, intellectual influences, and unknown
publications---but also helped to support the thesis that Ford belonged
to the pragmatist constellation, of which Holmes was one the brightest
stars.\footnote{Louis Menand, \emph{The Metaphysical Club} (New York:
  Farrar, Straus and Giroux, 2001); Robert Danisch, \emph{Pragmatism,
  Democracy, and the Necessity of Rhetoric} (Columbia: University of
  South Carolina Press, 2007).}

In the balance of this introduction, we first sketch a biographical
portrait that steers away from the ``scoundrel'' or ``crackpot
journalist-philosopher'' portrayals. Our material shows that Ford was
undoubtedly an original character, to say the least, but also that he
was taken seriously by his contemporaries and that his views were
considered and appreciated by many scholars, journalists, politicians,
and businessmen. By adding elements to what is already known of Ford's
life trajectory, we contextualize his arrival at Ann Arbor in 1888 in
the entourage of John Dewey and in light of his social and professional
networks and lifelong entanglement with media and communication
problems. By situating Ford in his historical, intellectual, social, and
political context, we show how the cast of this story is not limited to
the Ford--Dewey duet: It is a complex assemblage which many people and
ideas forged ``into relation.'' We have tried to give life to this
collective portrait, which is inevitably incomplete. We then offer an
overview of the material of this collection, and show how the texts
authored by Ford that we have included revolve around three
interconnected problems: news and journalism, information flows more
broadly (including finance and transportation), and political theory.
Finally, we further develop insights that emerged from our close reading
of the Ford archive, in order to add some nuance to the existing takes
on the place of Ford in media and communication scholarship. We
specifically seek to assess Ford's place within pragmatism and thereby
revisit the \emph{Thought News} episode as an example of pragmatic
inquiry.

\hypertarget{franklin-ford-into-relation-a-biographical-sketch}{%
\section{Franklin Ford ``Into Relation'': A\\\noindent Biographical
Sketch}\label{franklin-ford-into-relation-a-biographical-sketch}}

Franklin Ford was born in 1849 in Dundee, Michigan, to Valorus D. Ford,
a millwright and pattern maker, and Eliza Bell, who was born in the
north of Ireland. He was the eldest of four children: Franklin, Robert,
Sheridan, and Corydon. The four brothers would each have a career in
journalism while Corydon---who for a time was closely associated with
Franklin's journalistic projects---was also a physician. Corydon later
practiced among copper miners in northern Michigan and was deeply
involved in Ruskin utopian socialism as the editor of the most prominent
Ruskin paper, the \emph{Coming Nation}.\footnote{Corydon Ford became the
  editor of the \emph{Coming Nation} in 1901 and was also associated
  with \emph{Appeal to Reason}, another prominent socialist paper.} Very
little is known about Ford's childhood, formal education, or early
career. While still in Dundee, he developed a close relationship with
his grandfather Bell, a machinist, inventor, and owner of Dundee's water
works, and possibly began working for Detroit newspapers as a stringer.
He then worked for the \emph{Baltimore Gazette}, the \emph{Philadelphia
Record}, and the \emph{New York Sun} before taking up the editorship of
\emph{Bradstreet's Journal of Trade, Finance, and Public Economy} in
1880. Ford was thirty-one. His editorship lasted seven years, helped to
establish his reputation as a journalist, and undoubtedly marked a
turning point in his career.

\begin{figure*}
   \includegraphics[width=\linewidth]{graphics/image-one.png}
   \label{fig:fig1}
   \centering{\emph{Excerpt from} The Evening News \emph{(Detroit), December 24, 1896}}
\end{figure*}

\begin{figure*}
   \includegraphics[width=\linewidth]{graphics/image-two.png}
   \label{fig:fig2}
  \centering{\emph{The Bradstreet's building, in Moses King's} Handbook of New York
City \emph{(Boston: M. King, 1893), 816.}}
\end{figure*}

In his late twenties and early thirties---even before he started his
tenure as editor of \emph{Bradstreet\textquotesingle s}---Ford was
involved in public speaking and was an active participant in political
and intellectual life. In the mid-1870s, Ford was involved in a
reformist group, the Workingmen's Industrial Political Association,
where he served on the association's ``Committee for Political
Organization'' and presided over meetings of the association at the
Masonic Hall.\footnote{``The Workingmen's Assembly,'' \emph{New York
  Daily Herald}, January 30, 1874; ``The Industrial Political
  Association,'' \emph{New York Times}, March 1, 1874.}

Ford's subsequent interventions in public affairs turned to municipal
finance, which became one of his long-lasting topics of interest. In
1879, his work on municipal finance was presented before the
Philadelphia Social Science Organization and the New York Municipal
Society, as well as at the meetings of the American Social Science
Association in Saratoga Springs, New York. The thirty-page paper he read
at the New York Municipal Society on April 7, 1879, was deemed
interesting enough by the society to be printed and placed in
circulation. In 1881, Ford presented a second paper, titled ``Some
Points in Municipal Finance,'' at the meeting of the American Social
Science Association. A brief report of the meeting published in the
\emph{New York Times} opens and leads with Ford's contribution,
presenting a detailed summary of his paper---whereas Alexander Graham
Bell's address about his work with the deaf only earns a couple of lines
in the middle of the article.\footnote{``Social Science Themes,''
  \emph{New York Times}, September 10, 1881.}

\newpage From 1880 to 1887, Ford served as the \emph{Bradstreet's} editor. Credit
reporting agencies such as Bradstreet were the first large-scale
national information service providers in the US.\footnote{Josh Lauer,
  \emph{Creditworthy: A History of Consumer Surveillance and Financial
  Identity in America} (New York: Columbia University Press, 2017).} The
firm relied on the telegraph and the typewriter---mostly considered a
curiosity at the time---to process data through their ``system.'' The
Bradstreet agency made pioneering use of media technologies, such as
carbon paper, and of advances in the lithographic process. Reporting
could be supported by a massive archive, which counted about four
million reports by the mid-1880s, all accessible within two minutes.

During Ford's editorship, financial publications were booming and
economic journalism was transitioning into a more specialized
activity.\footnote{Wayne Parsons, \emph{The Power of the Financial
  Press} (New Brunswick, NJ: Rutgers University Press, 1989).} As a
credit agency, Bradstreet had its own model: ``Reporting'' was mostly
conducted through the branches of the agency, which counted about twelve
hundred full-time employees and sixty-five thousand collaborators. In
this sense, the making of the journal was truly a distributed and
collective activity. It was the product of a ``system,'' to use the
agency's favorite buzzword. Ford was deeply interested in the new
technologies of the era and in their application to journalism. He
dubbed \emph{Bradstreet's} his ``newspaper laboratory''\footnote{Franklin
  Ford to James B. Angell, April 13, 1887 (``A Newspaper Laboratory'').}
and sought to experiment with new ideas pointing in the direction of a
more systematic and scientific approach. Ford courted several experts
and scientists to write for \emph{Bradstreet's}, which was also eager to
open its page to statisticians.\footnote{Among others, Ford solicited
  university professors Woodrow Wilson, James Burrill Angell, and Thomas
  M. Cooley, and economist Edward Atkinson. Ford himself was sometimes
  presented as a ``statistician.'' See ``A High Authority on Wheat,''
  \emph{The Mail} (Stockton, CA), November 20, 1880.} Contrary to other
trade journals of the era, \emph{Bradstreet's} did not publish prices,
but aimed to seek ``after the influences which make prices---the primary
facts existing in relation to trade and finance.''\footnote{Clark W.
  Bryan, \emph{Credit: Its Meaning and Moment} (New York: Bradstreet
  Press, 1883), 24.} Reporting on the crop of corn, cotton, tobacco, and
other products was considered crucial since the crop constituted the
``primary fact'' behind the prices. In the same vein,
\emph{Bradstreet's} ``legal decision column'' aimed to deliver ``primary
facts,'' as it provided an overview of recent legal decisions that could
affect price-making and businesses.

During his tenure at \emph{Bradstreet's}, Ford's expert opinion was
regularly solicited on a variety of topics. In 1882, he testified before
a New York State Senate committee on speculation at the New York Produce
Exchange. He also was a guest lecturer on municipal affairs at the
University of Michigan in 1883. The practical bearing of his study of
municipal affairs was, then, to draw a ``scientific line between the
city and the state.''\footnote{Franklin Ford to Thomas McIntyre Cooley,
  November 26, 1883.} Ford's 1883 leaflet\footnote{Franklin Ford,
  \emph{Mayor Edson's Charter and the Democratic Principe} (New York:
  Bradstreet Press, 1883).} analyzed the question and recommended giving
greater executive power to the mayor and power of taxation to the city
council. In 1886, he was invited by William Russel Grace, then mayor of
New York City, to join a committee slated to present recommendations to
the New York State Constitutional Convention to broaden municipal
autonomy.

\begin{figure*}
   \includegraphics[width=\linewidth]{graphics/image-three.png}
   \label{fig:fig3}
  \centering{\emph{Ad for} Bradstreet's Journal\emph{, published in the} Daily
Chronicle \emph{(Knoxville, Tennessee), March 5, 1885.}}
\end{figure*}


\newpage By the mid-1880s, Ford's focus had turned to matters related to the
press and journalism. At the time, he planned to convince the president
of Bradstreet, Charles F. Clark, to reorganize its publishing operations
according to his views. The scheme included supplying ``leading country
papers with the city fact'' and the launch of three ``class'' papers:
\emph{Food}, \emph{Metal}, and \emph{Textiles}.\footnote{Franklin Ford
  to Edward Atkinson, October 13, 1886.} For unknown reasons, these
projects did not materialize at Bradstreet and, by early 1887, Ford was
trying to implement a similar scheme on his own, touring the ``chief
intelligence centres'' of the country. In a letter to Herbert Baxter
Adams, then professor of history at Johns Hopkins, he shared his plan to
visit Montreal, Toronto, Buffalo, Rochester, Pittsburgh, Cleveland,
Detroit, Chicago, St. Paul, Louisville, Cincinnati, Nashville, Memphis,
Galveston, New Orleans, Atlanta, and Richmond.\footnote{Franklin Ford to
  Herbert Baxter Adams, January 28, 1887.} Three months later, writing
to Edward Atkinson from New Orleans, Ford recounted his visit to Chicago
by way of St. Paul, Omaha, Cheyenne, Denver, Leadville, Kansas City, St.
Louis, Memphis, and Nashville, and planned to continue on to Galveston
and then New York by way of Birmingham, Atlanta, Savannah, and
Charleston.\footnote{Franklin Ford to Edward Atkinson, April 13, 1887
  (``Banding Together the Leading Newspapers'').} Ford claimed to have
succeeded in convincing local newspapers to form a syndicate connected
to his office-to-be in New York.

A couple of months later, in the spring of 1887, Ford successfully
teamed up with three associates to launch Ford's Special News, which
aimed at furnishing newspapers with reports on topics ``not covered by
the ordinary newspaper syndicates.''\footnote{\emph{Indianapolis
  Journal}, September 11, 1887.} He also planned to create an
investigation department to report on corporations---a business ``not
wholly unlike \emph{Bradstreet's}.''\footnote{Franklin Ford to Edward
  Atkinson, October 11, 1887.} The quartet behind Ford's Special News
had an impressive background. Lindley Vinton was the heir of a rich
Indianapolis family who studied at Amherst College, the University of
Berlin, and Columbia Law School. He would later have a successful career
as a lawyer in New York. Walter Hines Page was an established
journalist, founder of the \emph{State Chronicle} in Raleigh, North
Carolina, who had been associated with the \emph{New York} \emph{World}
and \emph{Evening Post}. He would later serve as the editor of the
\emph{Atlantic Monthly} (1896--1899) and as US ambassador to Britain
under Woodrow Wilson. The last member of the group was Frank West
Rollins, a graduate from MIT and Harvard Law School who would later be
elected governor of New Hampshire. Still, Ford's Special News folded a
couple of weeks into its operation, as Ford ``suffered a serious mental
attack'' and accused his partners of stealing his ideas.\footnote{``The
  Classes,'' \emph{Amherst Graduates' Quarterly} 33, no. 3 (1944): 258.}
In a letter, Ford recounted a ``fight'' that prevented the plan with
Page, Rollins, and Vinton from going through.\footnote{Franklin Ford to
  Edward Atkinson, October 11, 1887.}

It seems that Ford briefly tried to operate the trust (re-baptized
Ford's News) on his own, but he was soon to depart on a second tour of
the ``centers''---this time conceived as universities. In the spring of
1888, over four months, Ford visited philosophers and political
scientists at Columbia, Harvard, Yale, Johns Hopkins, and finally, the
University of Michigan, where the \emph{Thought News} project took shape
over the next four years.

Ford claimed to have met Dewey in the spring of 1888, but the details
surrounding this first meeting are not known. Dewey resigned from the
University of Michigan in the same period (March 1888) to accept a
position at the University of Minnesota. He spent most of the 1888--1889
academic year in Minnesota, before returning to Ann Arbor the next
spring. Dewey's absence from Ann Arbor in 1888--1889 suggests that Ford
had other partners that first year. Since the episode was mostly
chronicled through the lens of the Ford­--Dewey duet, we want to
emphasize the role played by other people and the specific context
surrounding the \emph{Thought News} project. The ``Ann Arbor group'' is
much wider than originally described by James Carey, and its porous
boundaries cover the connected worlds of the institutional life of the
University of Michigan, local journalism and publishing, and the broader
Ann Arbor community.

In Ann Arbor, Ford found himself in familiar territory. The University
of Michigan is located only twenty-five miles north of his hometown of
Dundee. Ford's uncle, Corydon La Ford, was a renowned professor of
anatomy at the university, where he taught until his death in 1894.
Ford's younger brother, also named Corydon, attended medical school in
Ann Arbor and was involved in many campus controversies. After
completing his degree in Minnesota, Corydon Ford was soon back at
Michigan to help with \emph{Thought News}. Dewey's correspondence during
the \emph{Thought News} project often mentions both Ford brothers, and
it is obvious that Corydon's radicalism played a part in the project.

Besides his close family, Ford arrived at Michigan with an address book
full of friends and acquaintances. Thomas McIntyre Cooley, the long-time
dean of the university's law school and father of sociologist Charles
Horton Cooley, was an old friend of Ford's. Cooley had previously
written legal columns for \emph{Bradstreet's} and had invited Ford, in
1883, to give two lectures at the university. Ford was also acquainted
with James Burrill Angell, the university's president and a former
journalist, with whom he discussed his project in great detail as early
as February 1887, at least one year before he first met Dewey. Ford was
also well-connected to local journalists, editors, and public figures.
Dewey's correspondence alluded to Ford's prominent Michigan friends,
such as Judge Edgar O. Durfee and Colonel William Ludlow.\footnote{John
  Dewey to Alice Chipman Dewey, June 19, 1891.} During the \emph{Thought
News} years, the group worked hard on making connections with a variety
of people. Participants solicited by Dewey include journalist and
scholar Joseph Villiers Denney, United States Commissioner of Education
William Torrey Harris, and Edward C. Hegeler, publisher of \emph{The
Monist}.\footnote{The copy of \emph{Draft of Action} kept at Brown
  University Library bears the inscription ``Presented to WTH {[}William
  Torrey Harris{]} by Prof. John Dewey.'' See also John Dewey to E. C.
  Hegeler, November 20, 1890 and January 12, 1891; and John Dewey to J.
  Villiers Denney, February 8, 1892.} Corydon Ford also invoked
discussions with John V. Sheehan, an Ann Arbor publisher and bookseller,
and faculty members Alfred Henry Lloyd and Henry Carter
Adams.\footnote{Corydon Ford, \emph{The Child of Democracy} (Ann Arbor:
  J. V. Sheehan, 1894).}

Among the familiar figures associated with \emph{Thought News}, Charles
Horton Cooley's role is the most mysterious. Cooley settled in Ann Arbor
in January 1889 and took several classes with Dewey, but soon
interrupted his studies to work in Washington. At the time, Cooley, an
avid reader of Herbert Spencer, was mainly interested in the railway and
its social significance. By 1892, Cooley was back at Ann Arbor as an
instructor in political economy. He then offered an answer to Spencer's
famous assertion that society has no ``sensorium,'' writing in his
journal that he had found the social sensorium in the
newspaper.\footnote{Jean B. Quandt, \emph{From the Small Town to the
  Great Community} (New Brunswick, NJ: Rutgers University Press, 1970).}
This idea also figured in Ford's 1892 \emph{Draft of Action}, which
refers to ``Herbert Spencer's hunt for the sensorium.'' Cooley's 1894
dissertation, ``The Theory of Transportation,'' can be summarized as an
attempt ``to see transportation and communication as an organic
whole.''\footnote{Edward C. Jandy, \emph{Charles Horton Cooley: His Life
  and His Social Theory} (New York: Dryden Press, 1942), 55.} It
provided one of the first overarching approaches to communication as the
sources of selves and society.\footnote{Peter Simonson, ``Varieties of
  Pragmatism and Communication: Visions and Revisions From Peirce to
  Peters,'' in \emph{American Pragmatism and Communication Research},
  ed. David K. Perry, 1--26 (Mahwah: Lawrence Erlbaum, 2001).} The
dissertation also attests to the rich intellectual cross-fertilization
within the group. How to centralize and organize transportation, how to
organize science through communication, and how to better disseminate
information in society (``publicity is not attained until facts are not
only connected but communicated'')\footnote{Charles H. Cooley, \emph{The
  Theory of Transportation} (Baltimore: American Economic Association,
  1894), 142.} are among the many Fordian themes touched by Cooley (or
Cooleyan themes also important to Ford).


After studying at the University of Berlin for three years, George
Herbert Mead arrived in Ann Arbor in the fall of 1891 as an instructor
of philosophy and was soon fully invested in \emph{Thought News}. In
February 1892, Mead sent twenty-five copies of a circular presenting
\emph{Thought News} to his brother-in-law, journalist Henry Northrup
Castle, and urged people to subscribe to the new journal. In typical
Fordian fashion, Mead described to Castle the underlying assumptions
behind \emph{Thought News}, writing that ``the conditions are free
enough now so that the organic intelligence of America can express
itself articulately as it has already dynamically in the locomotive and
the telegraph.''\footnote{Gary A. Cook, \emph{George Herbert Mead: The
  Making of a Social Pragmatist} (Urbana and Chicago: University of
  Illinois Press, 1993), 30.} Detailing how Ford had ``wrestled wholly
minded with the fact of organized intelligence---the meaning of
Hegel---and the fact has succeeded in registering itself upon him,''
Mead was glowing about \emph{Thought News}. ``The thing,'' he wrote,
``is only the greatest that the world has ever seen. It is the sudden
conscious recognition in an integral unit of society that he and all
exist only as the expression of the universal self.''\footnote{Coughlan,
  \emph{Young John Dewey,} 145.}

In 1892, on his way to South Dakota, Robert Park stopped in Detroit,
where he heard that Dewey was launching a new type of newspaper. He
decided to go to Ann Arbor, where Dewey introduced him to Ford.
According to Park's biographer, this stopover ``changed the course of
Park's life.''\footnote{Winifred Raushenbush, \emph{Robert E. Park:
  Biography of a Sociologist} (Durham, NC: Duke University Press, 1979),
  18.} Park was charmed by Ford, who assigned him the responsibility to
cover ``the relation of art to life'' for \emph{Thought
News}.\footnote{Raushenbush, \emph{Robert E. Park}, 21.} Park's letters
to his wife-to-be Clara Cahill were very enthusiastic. He supported the
project even after it was dropped by Dewey, sending Ford's \emph{Draft
of Action} and related articles to his wealthy stepfather, Michigan
Supreme Court Judge Edward Cahill. Many of the concepts Park later
developed can be understood in Ford's lineage, including his plea for a
``natural history'' of the newspaper, an approach indebted to Ford's
ideas about the natural history of governments.\footnote{Carey and Sims,
  ``The Telegraph.''}

\newpage Among the many people involved in the project whose role remained
little-known, the case of Junius E. Beal is among the most interesting.
Owner and editor of the \emph{Ann Arbor} \emph{Courier}, Beal was
supposed to print \emph{Thought News} on the \emph{Courier}'s
press.\footnote{``Thought News,'' \emph{Ann Arbor Courier}, April 20,
  1892.} In April 1892, when the project was taunted by the local press,
the \emph{Courier} republished a long article defending \emph{Thought
News} and explaining its aims.\footnote{For an overview of the
  controversy and its treatment by the press, see Savage, ``John
  Dewey''; and Westbrook, ``John Dewey,'' 55--57.} Later, Beal's paper
published the only piece detailing the reasons for abandoning the
publication, stating that \emph{Thought News} ``ha{[}d{]} evidently
perished from inanition'' and that ``it would have proven a heavy tax on
the brains and purses of its backers.''\footnote{``Local Brevities,''
  \emph{Ann Arbor Courier}, May 13, 1892.} Such explanation is
consistent with that given later by Park, but differs from Dewey's
reminiscence about a project that was ``too advanced'' for its
time.\footnote{John Dewey to Willinda Savage, May 30, 1949.} Park also
mentioned that a first issue had been prepared but not distributed. ``We
got out the copy for the first issue of the `Thought News,' but it was
never published. It was set up and then pied. My share in paying for it
was \$15.''\footnote{Raushenbush, \emph{Robert E. Park}, 20.}

An influential member of the Michigan State Editors Association, Beal
was an early advocate of journalism education.\footnote{Junius E. Beal,
  ``College Graduates vs. Practical Printers as Editors,''
  \emph{Proceedings of the Michigan Press Association at the
  Twenty-First Annual Meeting} (Pontiac: Bill Poster, 1888), 33--36.} He
had deep ties with the University of Michigan, where early initiatives
in journalism education had taken place.\footnote{In the spring of 1891,
  Fred Newton Scott began teaching ``Rapid Writing,'' a course that
  included elements of newswriting, which stands among the myriad of
  short-lived, experimental journalism courses that emerged in US
  universities in the 1880s and 1890s.} In early 1892, Beal organized
the twenty-fourth annual meeting of the association at the University of
Michigan in Ann Arbor. Over three days, the connections between
journalism and the university were explored at great length, and it
seemed clear to all that these were set to further develop. Fred Newton
Scott presented a paper on ``How to Read a Newspaper,'' followed by
local journalists, who addressed topics such as ``What Journalism Offers
to the University Graduate'' and ``The College Bred Newspaper Man.''
President Angell himself hosted a banquet for the members of the
association. Proclaiming that ``the editor is the most vital of men,''
his address drew from his own experience as editor of the
\emph{Providence Gazette}. ``If I have had any success in my life,''
Angell proclaimed, ``it is due to my experience while editing, for
several years, a daily newspaper. It makes a man a hard
worker.''\footnote{Theo E. Quinby, ``Our Last Day in the Athens of
  Michigan,'' \emph{Twenty-Fourth Annual Meeting and a Royal Outing in
  Southern Climes: Michigan State Press Association} (Howard City: B.J.
  Lowry, 1892), 8.}

\begin{figure*}
   \includegraphics[width=\linewidth]{graphics/image-four.png}
   \label{fig:fig4}
  \centering{\emph{Ad for Junius E. Beal's Printing House in} The Commencement Annual
of\\\noindent the University of Michigan\emph{, June 30, 1892.}}
\end{figure*}


Only a couple of months before the planned \emph{Thought News} launch
date, the meeting showed the deep interconnections between the worlds of
journalism and the university. Not only was the university poised to
train professional journalists, but journalism, in turn, was positioned
as a means to prepare for remarkable academic careers. In this respect,
\emph{Thought News} does not seem ``too advanced,'' but rather nurtured
by a highly favorable institutional and intellectual context---one which
should be the object of greater scholarly attention.

\emph{Thought News} was a thrilling political and intellectual project,
one also infused with profound personal ties. Dewey's correspondence
alluded to Ford as a close friend: Dewey's account of Ford's grand ideas
to his wife, Alice Chipman Dewey, is interspersed with mundane anecdotes
of both men taking walks by the river, swimming, or taking care of
Dewey's cats.\footnote{John Dewey to Alice Chipman Dewey, June 10--14,
  1891.}\newpage\noindent Despite this evident warmth, it seems that the relationship
ended rather bitterly.\footnote{In addition to Dewey's harsh comments
  mentioned earlier, Corydon Ford portrayed Dewey as an indecisive ally.
  ``Clogged of the dead institution, he could not move; his salary meant
  that he was to keep quiet as to the overturning concepts.'' See Ford,
  \emph{Child of Democracy}, 175.}

After the failure of \emph{Thought News}, Ford relocated to Detroit. As
he seemed to do in a quasi-compulsive way, he soon opened a new office.
Ford's News Office succeeded Ford's Special News and Ford's News, and
was advertised in the local press with claims that ``under telephonic
communication the scientific handling of news becomes
possible.''\footnote{``Experts Employed,'' \emph{Detroit Free Press},
  December 18, 1896.} He was also linked to a credit agency (The Credit
Office) and continued as a public speaker, giving talks with titles such
as ``The Organization of Journalism'' or ``The Political Meaning of the
Telephone.''\footnote{``Coming Entertainments,'' \emph{Detroit Free
  Press}, April 30, 1895.} In 1893, he published a piece titled ``The
Press of New York---Its Future'' (included in this collection), which
was simultaneously about the future of the press and the future of New
York. Arguing that the telephone marked the culmination of progress
initiated by the railroads and the telegraph, Ford envisioned a greater
supply of news and a boost in profits for the newspapers. Observing that
New York was located at the center of these communication networks, Ford
wrote that ``New York is the future Rome'' of a world that was
``Romeless.''\footnote{Franklin Ford, ``The Press of New York---Its
  Future,'' in \emph{Progress and Prospects of New York, the First City
  of the World, 1492--1893}, 46--47 (New York: Commercial Travelers
  Club, 1893), on 46.}

\begin{figure}
   \includegraphics[width=\linewidth]{graphics/image-five.jpeg}
   \label{fig:fig5}
  \centering{\emph{Ad for Ford's News Office in} Detroit Free Press\emph{, December
18, 1896}}
\end{figure}

While in Detroit, Ford continued to associate with controversial
projects. In 1896, he was one of the editors of \emph{The Optimist}, a
short-lived monthly publication linked to the Central Labor Union of
Detroit, alongside his brother Sheridan, Detroit newspapermen Thad
Stevens Varnum and Frank Cobb, and Thomas W. Lacey. ``Very neat
typographically'' and ``more a literary gem than a labor organ,''
\emph{The Optimist} aroused a strong reaction in literary
circles.\footnote{``Strike at St. James' Church,'' \emph{Chicago
  Chronicle}, June 6, 1896.} \emph{The Clack} described the publication
as ``the most revolting gutter filth under the name of literature. . . .
It is not even decent indecency. . . . Here one gets the unadulterated,
unperfumed stench of the sewer, without a suggestion of an artist's
excuse.''\footnote{``Clacks,'' \emph{The Clack}, July 1896, 123.}

Ford's public life also became entangled with that of Mathilde Coffin,
whom he had met in 1892. Born in 1861 and a graduate of Boston
University, Coffin had a successful career in the field of education.
After serving as principal of schools in Pennsylvania and in Michigan,
earning a national reputation for her expertise on educational subjects,
she was named assistant superintendent of the Detroit public schools in
1893. Thanks to this position, she was ``one of the most highly paid
women executives in the country.''\footnote{Wilma W. Henrickson, ``Too
  Bright for the Schools---Assistant Superintendent Mathilde Coffin,''
  \emph{Detroit in Perspective: A Journal of Regional History} 7, no. 2
  (1983): 26--41, on 28.}

In the mid-1890s, Coffin made the front page of the Detroit newspapers.
At the time, she was leading a reform movement through the city's school
system. In ideas that started to emerge in the mid-1890s and that were
to become prominent during the Progressive Era (notably in Dewey's work
on education), she argued that schools were disconnected from life.
``Let us take a look at the schools,'' Coffin wrote, ``the children are
there to connect with life, and yet, how far from real life is much of
the work in the schoolroom.''\footnote{``An Educational Revival in
  Detroit,'' \emph{The Intelligence}, November 15, 1896.} She
particularly critiqued the ``slavish'' use of textbooks that presented
abstract problems, detached from real-life situations, and she made
several attempts to create real-life pedagogical material. Coffin's
office collected ``teachers' problems,'' which sprang ``directly from
the daily newspaper''--- they were written by teachers, set in type,
printed, and distributed among the schools. Among other reforms, she
introduced ``special teachers'' to teach music, drawing, or physical
education, and believing that ``the people should get in closer touch
with the schools,''\footnote{``A Clever Actor!'' \emph{Detroit Free
  Press}, April 10, 1897.} organized educational leagues bringing
together mothers and teachers.

Coffin's progressive educational ideas raised strong reactions, and she
was accused of introducing ``fads'' into schools, stirring agitation
within the school system, exceeding her role as assistant
superintendent, and even maneuvering to secure the position of
superintendent.\footnote{``It's Out,'' \emph{Detroit Free Press,}
  December 17, 1896.} The press castigated Coffin for her independence
and her initiative (she was ``a genius with exalted ideas'' whose
ability had been ``misdirected'' in devising ``schemes of her
own'')\footnote{``The Relieving of Miss Coffin,'' \emph{Detroit Free
  Press}, December 20, 1896.} and, at the same time, accused her of
falling under Ford's influence. During the hearing that led to her
suspension, Coffin was accused of being ``spoiled'' by Ford's
ideas---which Coffin vehemently denied.\footnote{``Is Suspended,''
  \emph{Evening News} (Detroit), December 20, 1896} The press also
reported that Ford was believed to ``have a plan to get a monopoly on
the school board news.''\footnote{``The Conundrum of the Day,''
  \emph{Evening News} (Detroit), December 24, 1896.}

Ford and Coffin's 1897 marriage in New York was a surprise, as ``not
even the friends of the couples were let into the secret, and very few
even surmised that the strong friendship which existed between them
would ripen into love.''\footnote{``Miss Coffin Married,'' \emph{Detroit
  Free Press}, May 8, 1897.} Their marriage effectively ended the
dispute over Coffin's role in the public school administration, as a
teacher who married was automatically discharged (even though Coffin
persisted and brought the issue to court in 1897). The newlyweds moved
to New York, Ford's beloved ``new Rome,'' where Coffin had a long and
successful career as an educator, school expert, public speaker, and
later in her life, lecturer in psychoanalysis.\footnote{``Mrs. Ford,
  Ex-Aide of Estimate Board,'' \emph{New York Times}, July 4, 1941.} She
was a force to be reckoned with---and more than likely no mere puppet in
Ford's grand plan.\footnote{On Mathilde Coffin's career, see also
  Dominique Trudel and Juliette De Maeyer, ``The Many-Sided Franklin
  Ford and the History of a Post-Discipline,'' \emph{Communication
  Theory} 32, no. 4 (2022): 439--49.}

\begin{figure}
   \includegraphics[width=\linewidth]{graphics/image-six.jpeg}
   \label{fig:fig6}
  \centering{\emph{Wedding announcement in the} Detroit Evening News\emph{, May 7,
1897.}}
\end{figure}


After Ford and Mathilde Coffin resettled in New York, in the spring of
1897, Ford took the position of ``chief reporter'' of \emph{Textile
America}, a new trade journal. The venture had a capital investment of
\$50,000 and an experienced team that included publisher Thomas W.
Lacey, who was previously involved in \emph{The Optimist} in
Detroit.\footnote{Thomas W. Lacey was a close associate of Sheridan
  Ford, with whom he worked as an organizer in the labor movement.}
Ford's exact role in the new publication is unclear, but seems quite
important. He had lobbied for such a publication at least since his days
at \emph{Bradstreet's}, and later described its functioning in the
\emph{Draft of Action}. The new publication was presented in
recognizably Fordian prose as the ``organ of the textile division of
commerce.'' In addition to topics such as the price of Egyptian cotton
and new dying products, Ford revisited familiar arguments about the
self-government (or self-regulation) of the railroad industry and the
banking sector.\footnote{Franklin Ford, ``Traffic Associations,''
  \emph{Wichita Daily Eagle}, May 4, 1917 (reprinted from \emph{Textile
  America}); ``A Unified Banking System,'' \emph{Textile America},
  August 28, 1897.}

As part of his work for \emph{Textile America}, Ford wrote a
five-article series on ``better credit reporting,'' describing how he
helped Ryerson Ritchie and Robert J. Lyle to conceive the idea of the
Credit Clearing House, which was incorporated in New York in 1896 and
soon established in fifteen American cities. Pointing to how the free
exchange of information among merchants contributed to establishing
credit ratings, Ford theorized a relationship between freedom and
communication. ``The measure of freedom for the action,'' he
wrote, ``was of course just in proportion to the degree of communication
attained.''\footnote{Franklin Ford, ``Better Credit Reporting: A
  Revolution in Progress Through the Clearing-House Principle,''
  \emph{Textile America}, August 7, 1897, 34.} While Ford's theory
remains little-known to this day, it is strikingly similar to Walter
Lippmann's famous 1920 claim that ``liberty is the name we give to
measures by which we protect and increase the veracity of the
information upon which we act.''\footnote{Walter Lippmann, \emph{Liberty
  and the News} (Bethlehem: mediastudies.press, 2020), 21.}

\enlargethispage{\baselineskip}

Ford's name disappeared from \emph{Textile America} starting in late
1897. In the following years, he was associated with various credit
agencies, including the Credit Clearing House, the Credit Office, and
the National Credit Office.\footnote{Franklin Ford to John H. Finley,
  December 17, 1913.} It is likely that Ford was making a living from
his work in the world of credit, which allowed him to continue other
activities in parallel. With the help of Columbia's Head Librarian,
James H. Canfield, Ford resumed the organization of the ``University
Centre'' in early 1907, from an office set up for him at Columbia.
Inquiring into ``the working relation between the news centre and the
university,''\footnote{Franklin Ford to James H. Canfield, December 17,
  1904.} he also operated numerous news offices under a variety of
names, including Fords, Ford's Central News, The News Office, General
News Office, and City News Office. As usual, he spent his time writing
letters to newspapers, giving public lectures, and participating in
local political life. Ford's work in this last stage of his life
synthesized his previous interests: news, finance, transportation,
education, science, and politics. It all coalesced in a rather radical
political project that Ford summarized in a striking formula: ``News is
government.''

\begin{figure*}
   \includegraphics[width=\linewidth]{graphics/image-seven.png}
   \label{fig:fig7}
  \centering\newline{\emph{Portrait of Franklin Ford published in} Notable New Yorkers of
1896--1899 \emph{(New York: M. King, 1899), 599.}}
\end{figure*}

Ford's ideas had some resonance, but never to the full extent of their
ambitions. Everyone seemed to only see the relevance of a subset of his
project, refracted through their own interests. In 1901, for example,
New York City Comptroller Bird S. Coler praised Ford's plan for a
municipal news bureau,\footnote{``Mr. Coler's Plan to Increase City's
  Credit,'' \emph{New York Times}, December 16, 1901.} while
\emph{Electrical World and Engineer} applauded Ford's proposed use of
telephone networks:

\begin{quote}
Here is an ingenious scheme, difficult possibly, but with underlying
elements of practicability and value. But what strikes us specially is
Mr. Ford\textquotesingle s insistence on the point that with the
telephone ``or instantaneous communication,'' there should no longer be
life for lies or a lingering chance for rumor in regard to any moot
point in business, commercial or social life.\footnote{\emph{Electrical
  World and Engineer} 38, no. 1 (July 6, 1901): 2.}
\end{quote}

\noindent Others marveled at Ford's sheer charisma and his ``innumerable electric
phrases,'' not saying much about his actual ideas. In 1906, a letter of
a certain Francis D. Bailey to the \emph{New York Tribune} attributed
the phrase ``pocket nerve'' to Ford, describing him as ``that very
curious and cyclopean seer'' and adding that ``doubtless other terms
from his mind will pass into circulation, perhaps become fixed in the
language without attribution."\footnote{``From a Seer's Mind,''
  \emph{New York Tribune}, July 20, 1910.}

Ford's Columbia office was destroyed in a fire on October 10, 1914. His
papers, representing the work of twenty years, are believed to have been
ruined in the blaze.\footnote{``Blaze Ends Fire Peril at Columbia.''}
Ford died in 1918. Obituaries remembered him as ``once the editor of
\emph{Bradstreet's} and a widely known newspaperman.''\footnote{``Franklin
  Ford Dead,'' \emph{New York Times}, July 1, 1918.}

\hypertarget{a-tour-of-the-franklin-ford-collection}{%
\section{A Tour of the Franklin Ford
Collection}\label{a-tour-of-the-franklin-ford-collection}}

Now that readers are more accustomed with Ford, we would like to briefly
discuss the materials included in the present collection and to point to
some new leads they open. The organization of the materials reflects
Ford's evolving concerns, from a narrower focus on the problem of news
and its reform to the broader question of communication flows (money,
information, transportation), and finally, to their political
implications. Our overview of these texts is another opportunity to
contextualize Ford's ideas and career, as it points to several new
details and opens paths of inquiry for future research.

\hypertarget{reforming-the-news}{%
\subsection{\texorpdfstring{\emph{Reforming the
News}}{Reforming the News}}\label{reforming-the-news}}

Published or written between 1887 and 1907, these six documents attest
to Ford's long-standing project to reform the news, at both the
theoretical and practical levels. Following his work as
\emph{Bradstreet's} editor, Ford considered that ``a far-reaching
newspaper advance had become possible---this, through perceiving that we
now have the resultant of the locomotive and telegraph---the elimination
of distance.''\footnote{Ford, ``A Newspaper Laboratory.''} Under these
new conditions, which allow full access to the facts through a
technologically organized inquiry, the daily newspaper was first simply
considered ``a vehicle for selling the results of inquiry.''\footnote{Ford,
  ``A Newspaper Laboratory.''}

Alongside Ford's most well-known text, \emph{Draft of Action}, which was
deemed ``printed, not published,'' and to be ``held in confidence,'' two
letters written on the very same day, April 12, 1887, refer to Ford's
post-\emph{Bradstreet's} project to form a news trust (see ``A Newspaper
Laboratory'' and ``Banding Together the Leading Newspapers''). As Ford
describes in an 1893 essay (``The Press of New York---Its Future'')
published in a souvenir book of the Commercial Travelers Club of New
York City, the aim of this project was the ``ultimate associated
press.''\footnote{Ford, ``The Press of New York---Its Future,'' 47.}
These documents also present Ford's views on the papers of the era and
refer to various media tycoons, including James Gordon Bennett
(\emph{New York Herald}) and Horace Greeley (\emph{New York Tribune}).

A short memorandum written by John Dewey (``Organization of Intelligence
Requires an Organism'') describes Ford's ideas about journalism shortly
after they first met. Sent to Henry Carter Adams --- one of Dewey's
colleague and closest friend in Ann Arbor---the two-page piece is an all
but overlooked little gem.\footnote{Buried in the papers of Adams at the
  University of Michigan, most Dewey scholars overlooked the document.
  The memorandum is mentioned in Brian A. Williams, \emph{Thought and
  Action: John Dewey at the University of Michigan} (Ann Arbor: Bentley
  Historical Library, 1998), 30; Martin, \emph{The Education of}
  \emph{John Dewey}, 126.} The expressions that Dewey borrowed from Ford
here were intended to clarify Ford's ideas and make them more
topical---for example, Dewey wrote that the organization envisioned
would be ``automatic,'' a word used sparingly by Ford. This memo
anticipated Dewey's later work on communication. Published forty years
later, Dewey's \emph{The Public and Its Problems: An Essay into
Political Inquiry}, which lamented that the public remains disorganized
and suggested that the circulation of facts should be facilitated in
order to reorganize a genuinely democratic public, can be traced back to
this short memorandum, and the same applies to Dewey's concept of
inquiry.

The section also features a chapter (``In Search of Absolute News,
Sensation, and Unity'') from Corydon Ford's 1897 book, \emph{The Organic
State}, whose authorship is attributed to Franklin Ford. The piece
offers a critical analysis of the newspaper's coverage of the 1896
Republican and Democratic conventions and a detailed discussion of the
nature and functions of news. Ford describes his plan in the spirit of
G. W. F. Hegel and inspired by the work of the Fish Commissioner (the
``Fisheries''), whose reports involved ordinary citizens providing their
own facts, like in Ford's News Office.\footnote{Ford contributed to the
  work of the Michigan ``Fisheries'' in 1891 by providing an estimate of
  Dundee's carp population, which amounted to fifty. See \emph{Tenth
  Biennial Report of the State Board of Fish Commissioners} (Lansing:
  Robert Smith, 1893), 210.}

The section's last document is a 1907 letter (``The News System: A
Scientific Basis for Organizing the News'') to Clinton W. Sweet, the
founder and editor of the \emph{Record and Guide} and the
\emph{Architectural Record}. Ford presents his ``invention,'' the ``News
Centre,'' in great detail, and urges Sweet to invest his money ``before
others could hope to occupy the central position in the News System.''
Proposing to use his own ``General News Office'' as the basis of the
``News System,'' Ford outlines all the steps he envisioned: convincing
investors and subscribers, making connections with universities and with
the City of New York, building partnerships with businesses such as the
F. W. Dodge Company and the New York Telephone Company, and publishing
books and a series of general and trade papers.\footnote{Franklin Ford
  to Clinton W. Sweet, January 30, 1907 (``The News System: A Scientific
  Basis for Organizing the News'').} Overall, the business was ``on the
lines of the Associated Press,'' but much more ambitious. While all the
news traffic would go through the News Centre, ``news'' itself seems to
be considered on a larger scale. Information about municipal contractors
and their day-to-day activities, for instance, is ``news,'' and so is
the value of each building in New York City. Such a detailed account of
human activities is key to Ford's ``organic'' conception of news, which
aims at the registration and optimal circulation of each and every fact.
Ford considered his company a ``public institution'' and, unlike
traditional newspapers, it would not only earn its revenues from sales
and advertising, but also from corporations to whom specific sets of
facts are useful. According to Ford, ``the support of a given class
journal must come to be in direct proportion to the place of such firm
or corporation in the related industry.''\footnote{Franklin Ford to
  Clinton W. Sweet, January 30, 1907 (``The News System: A Scientific
  Basis for Organizing the News'').}

\hypertarget{interconnected-flows-money-information-and-transportation}{%
\subsection{\texorpdfstring{\emph{Interconnected Flows: Money,
Information, and
Transportation}}{Interconnected Flows: Money, Information, and Transportation}}\label{interconnected-flows-money-information-and-transportation}}

The documents in the second section, published between 1897 and 1902,
encapsulate Ford's view that the fields of transportation, credit,
banking, and many others are interconnected---all characterized by a
need for free-flowing information. It constitutes a good sample of
Ford's public-facing work at the turn of the century, and gathers
articles published in \emph{Textile America} (and sometimes reprinted
elsewhere), circulars issued by ``FORDS''---one of Ford's many
publishing ventures---and a letter to the editor of the \emph{New York
Times}. All these texts share similar themes: Ford criticizes existing
mercantile agencies, praises advances in communication technologies as
enabling ``new conditions'' in the circulation of facts, and advocates
for self-regulating information systems organized around what he calls
``clearing centers.''

Ford builds on his extensive experience at \emph{Bradstreet's} to look
at the history and current state of credit reporting, and to outline his
vision for improving the industry. In his contributions to \emph{Textile
America} and elsewhere, Ford criticizes the model of existing credit
rating agencies, including Bradstreet itself (see ``Better Credit
Reporting'' and ``The Mercantile Agencies and Credit Reporting''). These
agencies, according to Ford, have erected an ``unwieldy
machine''\footnote{Ford, ``Better Credit Reporting: A Revolution in
  Progress Through the Clearing-House Principle,'' 34.} which
artificially hinders the direct interchange of information between
merchants. Ford compares the ratings compiled by credit agencies to
``gossip about trading concerns,'' a mere ``literary procedure'' that
distorts the actual experience of merchants and creditors.\footnote{Ford,
  ``Better Credit Reporting: A Revolution in Progress,'' 34.}

In the article from the \emph{Better Credit Reporting} (1897) series
that we chose to publish here, Ford outlines the practical ways in which
a system built on the principle of the credit clearing house would favor
the ``direct interchange of experiences.''\footnote{Franklin Ford,
  ``Better Credit Reporting: Practical Operation of the Credit
  Clearing-House in Detail,'' \emph{Textile America}, August 21, 1897,
  9.} Through a system of daily reports (the article includes a blank
``reporting sheet'' and explains at length what a typical report looks
like) put into circulation by the clearing center, merchants who
participate in the ``trading circle'' could trade their ``single
experience for the experience of all.''\footnote{Ford, ``Better Credit
  Reporting: Practical Operation,'' 10.} We see another affirmation of
Ford's conception of truth and facticity, as he professes that the
``facts'' put in circulation (which are ``contained in the merchants'
ledgers, being the actual experiences of merchants with credit
seekers'') will necessarily be true: ``Reports so made up are matters of
fact; they are true. . . . The reports carry their own guaranty. In each
case the merchant is the reporter, and he cannot afford to do otherwise
than report truthfully.''\footnote{Ford, ``Better Credit Reporting:
  Practical Operation,'' 10.} But not all ``facts'' are automatically
true; some still needed to be checked, an issue that Ford equates to
that of the division of labor: Merchants can report their ``facts''
themselves (obviating the need for professional ``credit reporters''),
but these facts nevertheless require occasional verification, and that
is when the expert accountants or auditors are necessary.\footnote{Franklin
  Ford, ``Better Credit Reporting: The Development of News as a Thing of
  Trade,'' \emph{Textile America}, September 18, 1897.}

The key to Ford's system is the clearing house, a central entity through
which information flows rapidly. This concerns the credit system, but
also the banking system (see ``The Express Companies and the Banks'')
and transportation (see ``Traffic Association''). Ford establishes a
direct link between these new developments in communication
infrastructures---what he calls the ``new conditions,'' or the
conditions of ``full communication.''\footnote{Franklin Ford, \emph{The
  Country Check} (New York: FORDS, 1899), 18.} He praises the
``completion'' of the post office, the advent of express mail and money
order companies (such as American Express, founded in 1850), and of
course, the telegraph and the telephone. In ``Cooperative Credit
Reporting,'' a letter to the editor published in the \emph{New York
Times} in 1902, Ford argues that, with the ``talking wire, the plan
looks to the universal extension of the clearing house principle.''
Ford's texts are also concerned with how these new mediums reconfigured
time and space. In ``The Express Companies and the Banks'' (a 1899
circular published by FORDS under the banner of \emph{Bank News
Bulletin}), for example, he describes how the ``double movement''
necessary to convert personal checks into money (the check is first sent
to a center, and comes back over the same paths with a bank draft to
demand payment) is transformed in a ``single movement'' by express
companies---a disruption notably embodied by Traveler's Cheques,
launched by American Express in 1891.

Ford discusses at length the adequacy of this movement to the geography
of a vast territory such as that of the United States, in which the
circulation (of information, of money) must function between the
periphery and the centers, and between the different centers as well.
This is what Ford calls the ``unity of the banking system,'' in which
``the check is flying everywhere'' (see ``The Country Check''). But this
unity has consequences beyond the exchange of money as, according to
Ford, it is also ultimately about information and the ordering of facts:
``The discovery that the entire banking connection in America is a
single system, compels the adoption of a single language as means to
classifying and ordering the facts.''\footnote{Ford, \emph{The Country
  Check}, 10.} The model that he deems fit for credit reporting, for
instance, is also to be applied in every sector of life, as credit
reporting was just ``a phase of social registration.''\footnote{Franklin
  Ford, ``The Mercantile Agencies and Credit Reporting,'' \emph{Textile
  America,} June 3, 1899, 6.} Accordingly, each type of fact would be
recorded and put in circulation by the appropriate center: ``The facts
as to land ownership are registered with the Title Guarantee and Trust
Company; births and deaths are registered at the Health Office,
marriages at another center, while the bank transactions at ninety-five
clearing houses throughout the country are registered each week in New
York as the main center in the banking system and are thence distributed
to all sub-centers.''\footnote{Ford, ``The Mercantile Agencies,'' 6.}
Newspapers have a role to play, too, in the informational ecosystem, as
they were to be the ``main centre'' for all facts; trade papers, for
example, would allow bankers or merchants to ``be in constant touch with
the outlook in all divisions of commerce.''\footnote{Ford, ``Mercantile
  Agencies,'' 7.}

The texts in this section already allude to the political consequences
of this informational system. Under the ``new conditions'' described by
Ford, governments would become unnecessary, as the ``the whole system
appears as a self-regulating body controlled by its clearing
centers.''\footnote{Ford, \emph{The Country Check,} 3.} Ford continued
to explore these themes in subsequent years. Self-regulation through
information systems is the crux of the collection's final section.

\hypertarget{news-is-government}{%
\subsection{\texorpdfstring{\emph{News is
Government}}{News is Government}}\label{news-is-government}}

This last series of Ford's writings features works revolving around
politics and political theory. It articulates complex historical,
empirical, and philosophical arguments. Ford had a long-standing
interest in politics and reforms. He gravitated around the labor
movement for many years and wrote extensively on municipal governance.
Late in life, he turned to political theory as a means to articulate
what could otherwise be considered separate efforts. Political theory
means putting things ``into relation.''

``News is government'' is among the best of Ford's many home-cooked
formulas.\footnote{``News is Government,'' \emph{Wall Street Journal},
  February 7, 1907.} It refers to the central place of news as a
governing force in society, and to the triumph of ``facts'' and
``science'' over editorial opinion, which dates back to the days of the
penny press.\footnote{Michael Schudson, \emph{Discovering the News} (New
  York: Basic Books, 1981), 14.} The power of the press, in Ford's view,
has nothing to do with political journalism and very little to do with
politicians. Instead, it is about substituting the ``method of science''
and ``expert inquiry'' in place of the ballot.\footnote{Franklin Ford,
  ``The Public Necessity of Organizing Sovereignty through Credit
  Authority,'' \emph{Legal News Bulletin}, April 20, 1910.} Ford's
argument is anchored in a narrative that distinguishes between
successive modes of decision-making, from ``the primitive method of
fight,'' to the ``majority or count of noses,'' and finally, to the
extension of science brought by an ``organized news system.''\footnote{Ford,
  ``Public Necessity of Organizing.''}

Such an argument was not wholly original. In the mid-1880s, English
newspaper editor William Thomas (``WT'') Stead also suggested that
journalism was poised to succeed the House of Commons, itself successor
to government by kings. The time was considered ripe for ``government by
journalism.''\footnote{William T. Stead, ``Government by Journalism,''
  \emph{The Contemporary Review} 49 (May 1886): 653--74.} Proposing a
similar teleology, Stead shared with Ford a taste for grandiloquent
organicist metaphors, writing that ``the press is at once the eye and
the ear and the tongue of the people. It is the visible speech if not
the voice of the democracy.''\footnote{Stead, ``Government,'' 656.}

The parallels between Ford and Stead are many and worth considering.
Although there is no evidence of a direct connection between the two, it
is clear that Ford was well aware of Stead's ideas and journalistic
work. In the 1880s, Stead became a celebrity in America, especially
among his fellow journalists.\footnote{John Nerone and Kevin Barnhurst,
  ``Stead in America,'' in \emph{W. T. Stead: Newspaper Revolutionary},
  ed. Laurel Brake et al., 98--114. (London: British Library, 2012).}
Stead's ``new journalism''---an expression also used by Ford in his
\emph{Draft of Action}---emphasized the importance of journalistic
investigation in the service of social reforms, as well as the use of
maps, charts, and diagrams. Stead's 1894 best-selling study of Chicago's
underground economy included detailed folding maps locating brothels and
saloons, and initiated a broad civic reformist movement in the
city.\footnote{William T. Stead, \emph{If Christ Came to Chicago}
  (Chicago: Laird \& Lee, 1894).} The book echoed Ford's 1874 study of
Newark, which also contained charts and a folding map (but focused on
more traditional economic sectors).\footnote{William F. Ford, \emph{The
  Industrial Interests of Newark, N.J.} (New York: Van Arsdale, 1874).}
Another parallel is suggested by Ford's many attempts to operate an
information bureau committed to answering special inquiries, which
echoed Stead's paranormal communication bureau for private communication
with the other world.\footnote{Nerone and Barnhurst, ``Stead,'' 108.}

Ford's ``organization of the State under absolute communication'' reads
as a radical version of Stead's government by journalism.\footnote{Franklin
  Ford to John F. Dillon, November 4, 1904.} While Stead insisted on the
key role played by journalism in the political game and its capacity to
mobilize public opinion, Ford envisioned that journalism would soon
succeed existing political institutions. In fact, Ford's approach gives
very little place to influential professional journalists like Stead,
and to their power over politicians or public opinion. His focus,
instead, is on the infrastructures which enable collective inquiry and
the circulation of facts. Ford came to insist less on the centralized
control of communication infrastructures, a key tenet of the
\emph{Thought News} plan, and much more on the role played by
citizen-journalists, who were supposed to contribute their own facts
from the most marginal points of a networked infrastructure with many
different centers. In a letter published in June 1901 by the
\emph{Brooklyn Daily Eagle} (``City News Office Needed''), Ford
describes how every member of the community can contribute to the
circulation of information: ``In its outworking the reporting system
will connect with all sources of expert knowledge, and with every
individual in the community as any one may at times possess a fact of
value to his neighbor, to his class or to the people as a
whole.''\footnote{Franklin Ford, ``City News Office Needed,''
  \emph{Brooklyn Daily Eagle}, June 22, 1901.} The new form of
government he envisioned was decentralized by design. As Ford
rhetorically asked Columbia president Nicholas Murray Butler in a 1909
letter (see ``A New and Revolutionary Government''): ``Do you not
perceive that the Industrial State, long held in a language of metaphor,
is at last presenting itself in America on the plane of fact, and that
its regulating centres are forming independently of the inherited or
Military State?''\footnote{Ford, ``A New and Revolutionary Government.''}

Among these many centers that interested Ford, municipal governments
were of special interest. Most of Ford's writings in this section of the
collection deal with an attempt to reorganize municipal news on
scientific principles. Ford's City News Office (yet another office of
his) published a whole volume on the question in 1903 (``Municipal
Reform: A Scientific Question''). In a 1905 leaflet (see ``Government is
the Organization of Intelligence or News''), he also describes a
``General News Office'' (established in 1904) that was to be ``the main
centre for the local News System.''\footnote{Franklin Ford, ``Government
  is the Organization of Intelligence or News System,'' \emph{General
  News Office}, 1905.} While this office was to collaborate with the
existing administrative and political apparatus, Ford argues that, in
the long term, such an office and the communication infrastructure that
supports it were to replace existing political institutions from the
bottom-up---a plea for self-regulation in all aspects of society (see
``The Simple Idea of Government''). Ford was not afraid to apply this
framework to controversial events. Following the Ludlow Massacre in
April 1914, in which striking Colorado miners were killed by John D.
Rockefeller's private security services, Ford wrote:

\begin{quote}
The old State-centre or the Military Power, the strong-willed captains
of industry, the Creditlord, i.e. the typical banker of the day, are all
resisting by every means in their power the full, all-round functioning
of the new machinery; they are vainly trying to ward off its social
outcome. Why, the coal miners of Pennsylvania or Colorado are able to
combine together because they are in communication with each other by
electric wire---without communication, no common interest that can hold
together. And so, Mr. Rockefeller and his friends are simply fighting
the telephone; no wonder they have a hard time of it. The truth is that
they are thinking in terms of a past age, i.e. in language that is
already lying dead in the public mind.\footnote{Franklin Ford to Edgar
  L. Marston, April 29, 1914.}
\end{quote}

The collection ends with a 1912 letter to Oliver Wendell Holmes Jr.
(``News is the Master Element of Social Control''), in which Ford takes
stock of his eclectic intellectual influences and offers a synthesis of
his political ideas. Far from representing a conclusion to Ford's
lifelong vision, this long letter delineates all the work that remains
to be done. After admitting that he is only ``applying'' the work of
others and that his associates must be credited for the success of the
Credit Office, Ford describes the magnitude of the task at hand in
humble terms: ``I think it fair to say that I have done as much strictly
scientific work in my field as Darwin did in his, but I have to do twice
as much.''\footnote{Franklin Ford to Oliver Wendell Holmes Jr., January
  19, 1912 (``News is the Master Element of Social Control'').}

\hypertarget{what-to-make-of-ford}{%
\section{What to Make of Ford?}\label{what-to-make-of-ford}}

To conclude this introduction, we would like to offer some parting
thoughts. We circle back to the introduction's first paragraphs, and to
the ways in which Ford's life and work have found a place in the
existing literature, with the aim of adding our own piece to the puzzle.

As we have highlighted above, Ford's incorporation in media and
communication scholarship is often associated with the narrative around
pragmatism and the ``Chicago School'' initiated by James Carey and
others working in his wake. These lineages raise many interesting
questions. Should ``pragmatism'' and the ``Chicago School'' be
conflated, an implication of Carey's approach? Are they ``projective
devices,'' allowing anyone to imagine their own private version of the
Chicago School?\footnote{Lyn H. Lofland, ``Understanding Urban Life: The
  Chicago Legacy,'' \emph{Journal of Contemporary Ethnography} 11, no. 4
  (1983): 491--511.} Are they even part of the field (or ``discipline,''
to add a question within a question), since their methodological
approaches and central concerns often seem distant from what counts as
media and communication research?\footnote{This question has been
  explored in Wahl-Jorgensen, ``The Chicago School of Sociology and Mass
  Communication Research.''}

Chris Russill argues that the ``problem'' of pragmatism is twofold.
First, communication scholars largely failed to understand pragmatism as
an intellectual tradition.\footnote{Chris Russill, ``Dewey-Lippmann
  Redux,'' \emph{Empedocles: European Journal for the Philosophy of
  Communication} 7, no. 2 (2016): 129--42. Although Russill is mostly
  right, notable attempts at reconstructing the pragmatist tradition
  include works by John Durham Peters, Peter Simonson, Lana Rakow, and
  others. One of the most important efforts made by communication
  scholars to approach pragmatism historically is David K. Perry,
  \emph{American Pragmatism and Communication Research} (Mahwah:
  Lawrence Erlbaum, 2001).} While many exceptions are to be found,
communication scholars often seem ``caught'' by Carey's plot. They begin
with Dewey (or Cooley, or Park) and unravel from this origin point.
Consequently, what is upstream---that is, the intellectual traditions
which lead to Dewey (or Cooley, or Park)---remains little explored, as
is the broader intellectual context which nurtured pragmatism. The
second problem, which is a direct implication of the difficulty to
understand pragmatism historically, relates more specifically to the
concept of inquiry. While Dewey wrote incessantly about inquiry,
communication researchers often fail to engage with this notion,
preferring to ignore Dewey's admiration for scientific method and
emphasize instead his critique of objectivism and his enthusiasm for
``conversation.''\footnote{Chris Russill, ``Through a Public Darkly:
  Reconstructing Pragmatist Perspectives in Communication Theory,''
  \emph{Communication Theory} 18, no. 4 (2008): 478--504.}

The case of Franklin Ford seems the ideal locus to explore these two
problems. We turn first to the question of classical pragmatism's
intellectual roots, by exploring the connections between Ford and the
writings of Ralph Waldo Emerson and Lester Frank Ward. We then propose
to revisit the \emph{Thought News} episode as a form of pragmatic
inquiry.

\hypertarget{a-piece-in-the-pragmatist-family-tree}{%
\subsection{\texorpdfstring{\emph{A Piece in the Pragmatist Family
Tree}}{A Piece in the Pragmatist Family Tree}}\label{a-piece-in-the-pragmatist-family-tree}}

Ford's grand plan to launch a new ``movement of intelligence'' has a
rich---and somewhat messy---intellectual lineage. In his correspondence
and many of his other writings, Ford explicitly mentioned the many ideas
and authors that he found inspiring. The \emph{Franklin Ford Collection}
includes, notably, an unpublished 1912 letter sent to Oliver Wendell
Holmes, in which Ford offers a long synthesis of his ideas and
intellectual influences as he reflected on the work that ``forced
{[}him{]} to the library of Columbia University,'' which comprised
``both the introduction of a new science and its execution in the
market-place.'' The new ``universal governing organs'' he
discovered---the ``news system'' and ``banking system''---are put into
relation with Ford's vast intellectual pantheon. He invoked the enduring
influence of French anarchist Pierre-Joseph Proudhon and British legal
historian F. W. Maitland, and at least a dozen names as diverse as
Ernest Renan, Condorcet, Voltaire, Edward A. Ross, and Thorstein Veblen.
Ford's professed admiration for Proudhon---when his youthful radicalism
had long since passed---raises many questions and speaks to a strong and
unresolved political tension that cut across his different projects. On
the one hand, Ford had remained close to the world of finance and
banking. Constantly favoring monopolies, Ford's work often hinged on
central figures of classical liberalism. In addition to Condorcet, he
cites Montesquieu, Adam Smith, and John Stuart Mill. On the other hand,
his reformist agenda was irrigated by a wide array of radical
anarcho-socialist ideas and thinkers. His project to democratize
universities, for example, seems resonant with the \emph{universités
populaires} established by Jean Jaurès, and his early involvement
with the Workingmen's Industrial Political Association and in
avant-garde publications such as \emph{The Optimist} seems at odds with
the mainstream liberal doxa.

With no (or very little) formal education and with the habit of putting
things ``into relation,'' Ford may have been able to make creative and
unexpected syntheses. Emerson's radical individualism and Ward's Comtian
collectivism, discussed below, are not easy to fit together. But Ford
saw himself as a ``practical'' man rather than as an intellectual, and
despite his long theoretical digressions, he claimed that he was merely
``applying'' ideas in their practical bearings. Without trying to
resolve all the tensions within Ford's work, a closer look at some of
its intellectual underpinnings seems a necessary complement to existing
work.

While Emerson cast a long shadow over American pragmatism, his writings
are not often discussed by communication scholars. Among the notable
exceptions are John Durham Peters's \emph{Speaking into the Air} and
Peter Simonson's subtle analysis of ``overlooked forms of mass
communication'' in the lineage of Walt Whitman, Ralph Waldo Emerson,
William James, Kenneth Burke, James W. Carey, Cornel West, and John
Durham Peters, and his detailed portrayal of Charles Horton Cooley as
``the last of the nineteenth-century Emersonians.''\footnote{Peter
  Simonson, \emph{Refiguring Mass Communication: A History} (Urbana and
  Chicago: University of Illinois Press, 2010), 93.} This epithet may
very well apply to Franklin Ford too, who explicitly claimed the
Emersonian lineage in his early work.\footnote{See Ford's 1892
  \emph{Draft of Action}, which refers to the ``Economics of Emerson''
  and the ``Psychology of Emerson,'' as well as to ``Pathos of Faith
  without Sight in Thomas Carlyle.''} As Ford put it, ``the movement
begins where Carlyle and Emerson left off.''\footnote{Ford, \emph{Draft
  of Action}, 28.} Our archival material shows some connection between
Emerson's views on knowledge, technologies, and communication, and those
of Ford. Most important to pragmatism is Emerson's refusal of
foundational knowledge in favor of an open-ended quest for experimental
relations with nature. Considered one through the other, their works
gave a prominent place to media and communication in pragmatist inquiry
and reveal an often-neglected point of contact between pragmatism,
media, and communication.

\enlargethispage{\baselineskip}

Ford's writings refer to Emerson's correspondence with Thomas Carlyle,
which mainly concerns Emerson's role as Carlyle's literary agent for
America.\footnote{Ford, ``A Newspaper Laboratory''; Ford, \emph{Draft of
  Action}.} Central to their exchanges is a complaint about what Carlyle
called the ``anomaly of a disorganic literary class, the heart of all
other anomalies.''\footnote{Quoted by Ford in a letter to James Burrill
  Angell, April 13, 1887.} Such claim refers to the anarchic
mid-nineteen century context of generalized piracy and non-existent
international copyright agreements. Without addressing these specific
concerns, Ford's plan for a publishing monopoly and for ``organized
intelligence'' is primarily a grand scheme aimed at transforming the
publishing business in accordance with core organic principles. Like
Emerson and Carlyle, he considered such an operation to be the primary
step towards a ``movement'' that would affect society as a whole.
Emerson had a concept for that: the ``oversoul,'' ``a grand unifier of
society which pulsed into all men like a circulating blood.''\footnote{Quoted
  in John Durham Peters, ``Reconstructing Mass Communication Theory''
  (PhD diss., Stanford University, 1986), 43.} Among the many
implications of the oversoul, immortality comes first. Ford, like
Emerson, believed in the immortality of his soul, as he revealed to
Dewey.\footnote{``Ford believes in personal immortality. He says he
  thinks consciousness must persist---it is so damned persistent.'' John
  Dewey to Alice Chipman Dewey, June 6, 1891.}

Ford shared with Emerson, too, a distrust of inherited knowledge and
exaggerated deference to the past. Writing half a century after Emerson,
Ford could have made his own the first paragraphs of \emph{Nature} in
which Emerson complained that ``{[}o{]}ur age is retrospective, it
builds the sepulchers of the fathers. It writes biographies, histories,
and criticism,'' and asked, ``Why should not we also enjoy an original
relation to the universe?''\footnote{Ralph Waldo Emerson, \emph{Nature}
  (Boston: James Monroe, 1836), 1.} Ford's praise for the locomotive,
telegraph, and telephone were an invitation to develop such an original
relation to the universe, through the specific media technologies of the
time.

In a letter sent to James Burrill Angell in 1887, Ford described a first
model for organizing intelligence, describing a complex set of circles,
``semicircles,'' and multiple ``concentric rings.'' The model included
most of the elements that were later transmuted into Ford's ``triangle
of intelligence,'' which he explained in great detail in \emph{Draft of
Action}. This mention of circles in the context of a discussion on
Emerson directly evokes Emerson's well-known essay \emph{Circles}
(1841). Seeing circles everywhere, Emerson touched on a number of themes
in the essay, including the problem of knowledge and its relation to
technologies, which can be understood retroactively in a Fordian and
proto-pragmatist fashion:

\newpage\begin{quote}
There are no fixtures in nature. The universe is fluid and volatile.
Permanence is but a word of degrees. Our globe seen by god is a
transparent law, not a mass of facts. The law dissolves the facts and
holds it fluid.\ldots New arts destroy the old. See the investment of
capital in aqueducts made useless by hydraulics; fortifications, by
gunpowder; roads and canals, by railways; sails, by steam; steam by
electricity.\footnote{Ralph Waldo Emerson, \emph{Essays} (Boston and New
  York: Houghton Mifflin, 1883), 302.}
\end{quote}

\noindent John Durham Peters argues that, for Emerson, ``communication never
involves contact with another,'' and that ``the impossibility of
dialogue gives us reasons to celebrate the universe as a constant
transmission to those who have ears to hear.''\footnote{Peters,
  \emph{Speaking}, 155.} In other words, communication is not related to
face-to-face encounters and dialogue, but to forms of mediated
encounters with nature, given one has the right ``ears.'' For Emerson,
the right ears were akin to the new media of his times, such as
photography and phonography, as well as the daguerreotype portraits that
fascinated him. He praised the daguerreotype for providing a model for
his own writing, as he ``seeks to find modes of writing that reproduce,
for the reader, the immediacy of America's eventful present, \emph{in
writing}.''\footnote{Tobias Weber, ``On the Verge of To-day: Emerson and
  the Emergence of the American Age'' (PhD diss., University of Zurich,
  2011), 30.}

Ford was animated by similar concerns, and his work provided a
provisional answer to this line of inquiry. When he wrote that ``the
social fact is the sensational thing,''\footnote{Ford, \emph{Draft of
  Action}, 9.} he meant precisely that there is an intimate connection
between a more abstract depiction of nature, at the level of the social
fact, and its lived experience, at the sensory level. For him, the
``right ears'' were the newspapers and the advances brought by the
telegraph and the locomotive, which changed news gathering and
dissemination. For both men, the registration and organization of
intelligence through appropriate media technologies were the primary
operations of communication, defined as an epistemological problem and
as a form of inquiry. But while Emerson focused on communication at the
level of the individual, Ford was concerned with collective knowledge
and forms of action.

Peters's reference to the Emersonian ``impossible dialogue'' and escape
route into the half-solipsism of personal relation to nature suggests,
in parallel, a tragic dimension of Ford's story. There is an enormous
gap between Ford's plan for interconnection and his own social
situation, between his grand plan as seen in the performative space of
his letterheads---which attest to the existence of many
``offices''---and the reality. While numerous accounts testify to his
social eccentricity, his long letters sent to many important people
(often ten or more single-spaced pages) seem to have generated very
little dialogue. Most of his collective projects ended quickly or
bitterly, only to make way for a slightly different iteration of his
solitary meditations. Ford exemplifies what Emerson famously called
``the condition of infinite remoteness'' and embodied the Emersonian
``prescription for courageous self-reliance by means of non-conformity
and inconsistency,''\footnote{Cornel West, \emph{The American Evasion of
  Philosophy: A Genealogy of Pragmatism} (Madison: University of
  Wisconsin Press, 1989), 10.} at least in the eyes of some of his
contemporaries.

Another important source of Ford's ideas is the work of American
sociologist Lester Frank Ward (1841--1913), who is only mentioned in
passing in existing works about Ford.\footnote{Carey and Sims, ``The
  Telegraph''; Weaver, ``John Dewey.''} A
paleontologist-turned-sociologist, Ward is a somewhat forgotten figure.
His 1883 book \emph{Dynamic Sociology} is nevertheless one of the early
classics of American sociology. A response to Herbert Spencer's
\emph{Social Statics,} the book clashed with the fatalism of Spencer and
his most outspoken American disciple, Yale professor William Graham
Sumner. In place of the Spencerian \emph{laissez-faire}, Ward considered
that sociology should aim to transform society, and argued that applied
knowledge was key to social progress. Ward's concept of ``telesis''
emphasized the possibility of planned progress against Spencer's
conservative sociology. Ward's critique of Spencerian social Darwinism
was central to the early days of American pragmatism and was discussed
during a famous meeting of the Metaphysical Club in 1884.\footnote{Menand,
  \emph{Metaphysical Club}, 152.} The critique had a profound influence
on Dewey (who attended the 1884 meeting) and on the Chicago School of
sociology.\footnote{Daniel Tanner, ``From Lester Frank Ward to John
  Dewey: The Three Universal Curriculums,'' \emph{The Educational Forum}
  83, no. 2 (2019): 124--39.}

Both Ward and Ford displayed a profound enthusiasm for the dissemination
of knowledge and for planned social reform. If Ford constantly used
Spencerian organic metaphors, and even pushed them to new summits, it is
because he thought that society \emph{was} an organism in a greater
sense. It had a ``sensorium,'' a unity of consciousness, a ``center''
from which the work of its organs could be coordinated. Echoing Dewey's
argument that ``society was an organism in a deeper sense than Spencer
had perceived,''\footnote{Peters, ``Reconstructing,'' 74. Dewey's
  comment was made during a lecture heard by Charles Horton Cooley, who
  also criticized Spencer.} Ford proposed to pursue ``Herbert Spencer's
Hunt for the Sensorium.''\footnote{Ford, \emph{Draft of Action}, 39.}
Ward, for his part, had previously run backward from the organicist
metaphor, arguing that living organisms were nothing more than parts
united by ``communication'': ``All parts of the organism are
\emph{integrated} by means of channels or tracks of protoplasm in the
form of nerves, along with constant communication.''\footnote{Lester
  Frank Ward, \emph{Dynamic Sociology} (Boston: D. Appleton, 1883), 371.}
In doing so, Ward provided scientific and biological grounding for
Ford's hunt for the sensorium.

One of Ford's most important concepts, ``intelligence,'' is nowadays
mostly associated with Dewey's philosophy. It is also a key concept of
Ward's sociology, which casts intelligence as a compound of
``intellect'' and ``knowledge.'' As knowledge is ``registered
experience,'' it is by no means individual, but profoundly social. In
other words, at least half of intelligence is external to the human mind
and thereby ``social.'' Ford used the concept in a similar fashion.
Writing about the ``movement of intelligence,'' with the aim to organize
``intelligence centers'' and ``intelligence trusts,'' Ford's project
echoed Ward's sociology and his understanding of intelligence as a
collective endeavor.

Another obvious conceptual convergence is found in Ward's political
philosophy. He envisioned a new form of government called
''sociocracy.'' Expanding on Auguste Comte's conception of sociocracy as
government by sociologists, Ward favored a greater role for science in
the government of society. Standing between individualist democracy and
socialism, sociocracy would involve scientific, social, and economic
planning in the general interest. He illustrated sociocracy with the
``postal telegraph question,'' arguing that a price of ten cents
(instead of twenty-five cents) would satisfy everyone except
stockholders, and that a fair price (that is, a price to maintain and
develop the infrastructure, to provide a decent return on investment, to
promote the use of the greatest number, etc.) should be set only after
disinterested investigation.\footnote{Lester Frank Ward, \emph{The
  Psychic Factors of Civilization} (Boston: Ginn \& Company,
  1893)\emph{,} 326.} Ford's very similar views and political theory
should be considered in the lineage of Ward's (and Comte's) sociocracy.
Proclaiming that ``science, exact inquiry, is the source of
government,'' Ford gave the example of the milk trade.\footnote{Franklin
  Ford, \emph{The Simple Idea of Government} (New York: The News Office,
  1910).} In place of a regulating agency performing inspections, Ford
argued that the ``identity of interest between producers, distributors,
and consumers'' should be identified, and that, as a consequence, the
industry should govern itself.\footnote{Ford, \emph{The Simple Idea of
  Government}.}

As it is often the case with Ford, archival evidence is a bit scarce.
Ward's papers contain two letters from Ford, who straightforwardly
explained that he was acquainted with Ward's work.\footnote{Franklin
  Ford to Lester Frank Ward, December 29, 1904.} Ford enclosed a leaflet
from the General News Office, ``Government is the Organization of
Intelligence or News''---included in this collection---and boasted that
``it represents twenty years of continuous work. In order to write it I
was compelled to change the inherited point of view in social
observation, which was fully as difficult as for Copernicus and his
followers to change the viewpoint in observing the solar system.'' Ford
and Ward would later meet at the fourth annual meeting of the American
Sociological Society, held at Columbia in 1909, when Ward presented a
paper on ``Sociology and the State.'' A year later, Ford sent excerpts
of his work to Ward and proposed to visit Providence so the two could
converse.\footnote{Franklin Ford to Lester Frank Ward, February 2, 1911.}
Although we found no further documentary traces of this meeting, it may
have happened.

\hypertarget{thought-news-as-pragmatic-inquiry}{%
\subsection{\texorpdfstring{Thought News \emph{as Pragmatic
Inquiry}}{Thought News as Pragmatic Inquiry}}\label{thought-news-as-pragmatic-inquiry}}

The question of Ford's place in the pragmatist family tree could (and
should) be approached from other perspectives and is yet to be resolved.
Surprisingly, Ford's lengthy correspondence with Holmes has not yet been
considered from this angle. Another path not taken is the Jamesian lead.
Although James's work is not frequently mentioned by Ford, Earl James
Weaver suggests interesting parallels between James's \emph{Psychology}
(1890) and the writings of Corydon and Franklin Ford.\footnote{``James's
  most important book, \emph{Psychology}, published in 1890, hints of a
  functionalism, an instrumentalism, which is not unlike that contained
  in the writings of the Ford brothers. His picture of a moving, open,
  changing world, his emphasis on the concrete and the singular as
  opposed to the abstract and the general, his presentation of mind as
  activity, of an idea as an experimental instrument, coincided neatly
  in tone and in detail with the Fords' outlook.'' Weaver, \emph{John
  Dewey}, 76.} Dewey's account of the emergence of his own stream of
pragmatism, which does not mention Ford, nevertheless situates its birth
in Ann Arbor, between 1891 and 1893---the peak of his friendship with
Ford.\footnote{Writing to William James in March 1903, Dewey explained
  that ``we have all been at work at it for about twelve years. Lloyd
  and Mead were both at it in Ann Arbor ten years ago.'' Quoted in
  Dykhuizen, ``John Dewey,'' 536.} In order to recast Ford's role in the
early days of pragmatism, we now offer a few notes on the notion of
inquiry in Ford's work, by specifically revisiting the \emph{Thought
News} episode. It is also a way of circling back to the beginning and
offering our own reading of this foundational story.

In 1889, Dewey wrote to his friend Henry Carter Adams, professor of
political economy at Michigan, that Ford's idea was not simply to tell
the truth, but ``to find out what truth is; the inquiry business in a
systematic, centralized fashion.''\textsuperscript{174}\ Inquiry is arguably Dewey's most important and
complex notion. It is the epistemological cornerstone of his own stream
of pragmatism, with\marginnote{\textsuperscript{174}\setcounter{footnote}{174} John Dewey to Henry Carter
  Adams, April 4, 1889.} numerous implications for the philosopher's views on
communication, including his thesis on the ``public'' and its role in
democracies. As such, it is somehow difficult to pin down. In fact,
scholars still debate Dewey's conception of inquiry.\footnote{Russill,
  ``Through a Public Darkly.''} Dewey wrote enthusiastically about ``the
highest and most difficult kind of inquiry'' that ``must take possession
of the physical machinery of transmission and circulation and breathe
life into it.''\footnote{John Dewey, \emph{The Public and Its Problems}
  (Denver: Alan Swallow, 1927), 194.} Such inquiry is, by nature,
collective, as it both presupposes and articulates a community. It aims
at transforming life by putting into relation discrete elements in order
to ``convert the elements of the original situation into a unified
whole.''\footnote{John Dewey, \emph{Logic: The Theory of Inquiry} (New
  York: Henry Holt, 1938), 105.} It coalesces practice and theory
through experimentation; it looks for modest moments of truth while
being wary of claims to foundational knowledge.

Communication scholars often tend to rely on Carey's rather simplified
definition of inquiry as ``conversation and discussion but a more
systematic version of it.''\footnote{Carey, \emph{Communication}, 82.}
Chris Russill's work aptly points to other key dimensions of pragmatic
inquiry. First, there's its interest in science as a model for inquiry.
While pragmatism is often cast as a critique of ``scientism'' and
``positivism'' by communication scholars, Russill points to the
deforming mediation of Carey and Richard Rorty, who ``stripped
pragmatism of its interest in science as an exemplar of
inquiry.''\footnote{Chris Russill, ``Dewey/Lippmann Redux,''
  \emph{Empedocles: European Journal for the Philosophy of
  Communication} 7, no. 2 (2016): 129--42, on 130.} Second, inquiry is a
form of action (or response) to a problem as a problem (or as Russill
puts it, ``a type of action that responds to a problem \emph{as}
problematic.'')\footnote{Russill, ``Through a Public Darkly,'' 485.}
Relating to problems as problems is the first difficulty. It suggests to
refrain from immediate response in order to grasp the complexity of
situations that are usually experienced as distant. These must be
apprehended from a variety of standpoints in order to be experienced as
problems, and in light of different possible solutions. As Russill
explained, ``this requires not only facts but ideas or hypotheses for
expanding beyond the habitual and expected reactions we might offer. We
should learn to be affected by events as problems, formulated so as to
make possible a range of solutions.''\footnote{Russill, ``Through a
  Public Darkly,'' 495.}

Deweyan inquiry does not exist beyond its own logic. It is a means to
experience a problem \emph{as} a problem---that is, to cope with the
complex problem of knowledge in modern industrial societies. According
to Dewey, one of the main features of this problem is that ``the paths
of communication between common sense and science are as yet largely
one-way lanes. Science takes its departure from common sense, but the
return road into common sense is devious and blocked by existing social
conditions.''\footnote{Dewey, \emph{Logic}, 77.}

The \emph{Thought News} project fits such a conception of pragmatic
inquiry at several levels. While the many advertisements for
\emph{Thought News} emphasized the theory-to-practice perspective, the
full title of the journal, \emph{Thought News: A Journal of Inquiry and
a Record of Fact}, refers both to inquiry and to the dual need for facts
(``news'') and ideas or hypotheses (``thought''). \emph{Thought New}s
can thus be read as an early embodiment (or at least a specific model)
of inquiry, and the \emph{Draft of Action'}s plea for ``full social
inquiry''---that is, inquiry that connects specific concerns to ``the
whole''---can be read as anticipating Dewey's later writings on inquiry.
``Inquiry'' appears on thirty-eight pages of Ford's fifty-eight-page
document and is one of its pivotal notions. Scholars sometimes
understand Dewey's concept of inquiry as ``problem-solving
activities,''\footnote{Russill, ``Dewey/Lippmann Redux,'' 134.} with his
``unified theory of inquiry'' aimed at offering ``a single way of
thinking about how we resolve problematic situations in science, ethics,
politics, and law.''\footnote{Cheryl Misak, \emph{The American
  Pragmatists} (Oxford: Oxford University Press, 2013), 145.} In Ford's
version, we could say that the emphasis is on problem-solving
\emph{infrastructures}. Ford's plan is to design an information system
through which the specific concerns of individuals, ``classes'' (in the
sense of specific economic sectors), and ``the whole'' would be tied, on
different scales, and construed as problems having possible solutions.
To use Dewey's vocabulary, it is a ``machinery of
communication''\footnote{See Lana Rakow, \emph{John Dewey: A Critical
  Introduction to Media and Communication Theory} (New York: Peter Lang,
  2019).} recording facts, putting them in relation, and turning them
into ``problems.'' Ford writes that facts must be ``interpreted and
delivered in their application to life.''\footnote{Ford, \emph{Draft of
  Action,} 42.}

Science was central to such inquiry, which was defined by Ford as ``a
union of science and literature.''\footnote{Ford, \emph{Draft of
  Action}, 29.} This project remained partly modeled on
\emph{Bradstreet's} scientific reporting, which featured plenty of
quantitative data, graphs, and expert opinions. Although Carey's first
take on \emph{Thought News} rightfully approached the episode (and ``the
Chicago School'' as a whole) through the lens of this tension between
science and literature, his later work clearly adopted the Rortyan
perspective criticized by Russill.\footnote{On this shift in Carey's
  characterization of \emph{Thought News} and the crucial influence of
  Rorty, see Pooley, \emph{James W. Carey}.}

\begin{figure}
   \includegraphics[width=\linewidth]{graphics/image-eight.png}
   \label{fig:fig8}
  \centering{\emph{Ad for} Thought News \emph{published} \emph{in} Philosophical
Review\emph{, May 1892.}}
\end{figure}


The collective dimension of the project also speaks to the notion of
inquiry, one that is never a purely individual endeavor. Advertisements
for \emph{Thought News} specifically appealed to isolated researchers in
search of a community.\textsuperscript{189} Ford's \emph{Draft of Action} also makes clear that
fact-gathering was to\marginnote{\textsuperscript{189}\setcounter{footnote}{189} Published in the \emph{Inlander} of
  April 1892, an ad for \emph{Thought News} reads, ``If you are studying
  by yourself, If you are interested in the application of ideas to
  life, If you are interested in the application of theory to practice,
  You will be interested in Thought News.'' Quoted in Savage, ``John
  Dewey,'' 206.} involve everyone. Ordinary citizens were all to be
crop reporters---''the citizen king is the crop reporter,''\footnote{Ford,
  \emph{Draft of Action}, 12.} as Ford wrote in one of his typical
formulas---collaborating to define common problems that do not first
appear as such. The communal dimension of the project is also obvious in
the loose group of people behind \emph{Thought News}, as well as in its
ambition to connect journalists and scholars, the university and
society.\footnote{See Trudel and De Maeyer, ``The Many-Sided Franklin
  Ford.''} The \emph{Thought News} episode is also a telling tale of the
enduring and inescapable tension between inquiry as the experience of
problems as problems, and the typically more direct emotional response
to problems. The line between public problems and private matters is not
easy (if not impossible) to draw. Dewey's letters evoked a very deep
relation with Ford, oscillating between abstract considerations about
\emph{Thought News} and the delicate question of personal relations and
commitments. In this sense, the \emph{Thought News} episode illustrates
the pragmatic limit to pragmatist inquiry. At least, this is one
conclusion that can be drawn from this story \emph{as a failed
experiment}.

\enlargethispage{\baselineskip}

A different perspective is to consider the idea that \emph{Thought News}
did not end in 1892. Ford obviously continued several of its lines of
inquiry throughout his life, and so did the other participants of the
group. While archival evidence is scarce, many contextual elements
suggest that some personal connections survived the episode. Like Ford,
Robert Park settled in Detroit in 1892, where he worked as a reporter
for the \emph{Detroit Tribune}, alongside close associates of Ford such
as Thad Varnum. In early 1893, Park was consulting with Dewey and George
Herbert Mead about the foundation of a University Club in
Detroit.\footnote{Matthews, \emph{Quest for an American Sociology}, 29.}
Echoing the goals of \emph{Thought News}, the organization was ``to
bring people who are outside of the University in closer connection with
it and through them bring the University in closer connection with
life.''\footnote{Matthews, \emph{Quest for an American Sociology}, 29.}
While Park acknowledged on several occasions the enduring influence of
Ford's ideas on his work, their personal connections may very well have
continued.\footnote{Ford mysteriously evoked ``one of my fellow students
  who is in Chicago'' (who is possibly Park) in a letter to Holmes
  (Burton, \emph{Progressive Masks}, 114). On Ford's enduring influence
  on Park, who discussed Ford's work in his classes as late as 1921, see
  Rolf Lindner, \emph{The Reportage of Urban Culture: Robert Park and
  the Chicago School} (Cambridge: Cambridge University Press, 1996), 34.}
In 1897, Ford was associated with Delos F. Wilcox, who took classes with
Dewey in Ann Arbor in the early 1890s and shared with Ford a profound
interest in municipal government. Wilcox, to whom Ford lent manuscripts,
referred mysteriously to the ``joint work of yourself, Corydon L. Ford,
Prof. John Dewey and Mr. Thomas Lacey'' as ``a more important body of
philosophical writings than has yet been published in
America.''\footnote{Delos F. Wilcox to Franklin Ford, December 5, 1897.}
That Ford and Dewey never crossed paths at Columbia University during
the twelve years they both had an office on campus also seems
unlikely.\footnote{Dewey came to Columbia in 1905 and stayed until his
  retirement in 1930, and Ford had an office set up for him at the
  library from 1907 until his death in 1918.}

It is now time to let readers dive into the texts of the Ford
collection. This introduction is far from having exhausted all the
avenues opened by the strange character that is Franklin Ford. We have
tried to make the story of his life as collective as possible, but some
of Ford's acolytes deserve further inquiry, including his wife Mathilde
Coffin, who survived Ford by twenty-three years,\footnote{``Mrs. Ford,
  Ex-aide of Estimate Board,'' \emph{New York Times}, July 4, 1941.} and
his brothers Sheridan and Corydon, who both had tumultuous lives and
careers. We have also tried to describe Ford's intellectual pantheon,
which would benefit from further exploration.

Perceptive readers will also have noticed that many of Ford's themes
anticipate contemporary issues: the fascination with ``new''
technologies, the intertwining of the informational and financial
worlds, the workings of media ecosystems and their eminently political
character---to name just a few. We have tried to steer clear of turning
Ford into a visionary or a prophet. This does not mean that there is
nothing to say about Ford's relevance for today: We have explored these
avenues elsewhere\footnote{Our work includes a methodological experiment
  that turned Ford into a series of ``bots'' on social media. See
  @FranklinFordBot (website), last updated January 30, 2020,
  \href{http://www.franklinford.org}{www.franklinford.org}. See
  also: Trudel and De Maeyer, ``The Many-Sided Franklin Ford''; Juliette
  De Maeyer and Dominique Trudel, ``@franklinfordbot: Remediating
  Franklin Ford,'' \emph{Digital Journalism} 6, no. 9 (2018): 1270--87.}
and hope that future works will continue to offer a stimulating back and
forth between the turn of the last century and contemporary concerns.

Finally, even though we believe that the present collection offers a
coherent deep dive into Ford's writings, we also contend that it is
incomplete: Some documents of the Ford archive could not be included nor
discussed here, and we also know, given the prolific nature of Ford's
output, that other writings have yet to be found and added to the
collection. The inquiry into Franklin Ford's life and work waits to be
put into many more relations.

% ACKNOWLEDGMENTS
\chapter[Acknowledgments]{Acknowledgments}
\label{ch:Acknowledgments}
\chaptermark{ACKNOWLEDGMENTS}

\newthought{We would like} to thank the many people and organization who helped us
with this project. Amandine Hamon, Cyrus Khalatbari, Simona Feng, and
Hugo Marchand served as research assistants and contributed
significantly to this project. The support of Jefferson Pooley and Dave
Park and of their team from mediastudies.press was invaluable. Jeff
believed in this project from the beginning and his insights were
numerous. Emily Alexander ably led the linguistic revision of Ford's
texts and the book's introduction. Our research also benefited from the
feedback or many colleagues and anonymous reviewers that we thank for
their insightful and generous comments. We are grateful for the
financial support provided by the Social Sciences and Humanities
Research Council of Canada (grant no. 430-2018-00809) and MITACS, and
for the help we received from many dedicated archivists from the
AT\&T Archives and History Center, Ball State University Libraries,
Bentley Historical Library (University of Michigan), Brown University
Library, Columbia University Library, Harvard Law School Library, New
York Public Library, the Massachusetts Historical Society, Oviatt
Library (California State University), University of Chicago Library,
University of Washington Libraries, and Yale University Library.

\vspace{.35in}

\begin{Large}

\noindent\smallcaps{Dominique Trudel \& Juliette De Maeyer}

\end{Large}

% THE LARGER LIFE: A POEM DEDICATED TO FRANKLIN FORD
\chapter[The Larger Life: A Poem Dedicated to Franklin Ford]{The Larger Life: A Poem Dedicated to Franklin Ford}
\label{ch:The Larger Life: A Poem Dedicated to Franklin Ford}
\chaptermark{THE LARGER LIFE: A POEM DEDICATED TO FRANKLIN FORD}

\vspace{.2in}

\begin{LARGE}\marginnote{Excerpt from \emph{The Larger Life}, 66–67. New York: G. E. Croscup, 1904.}
    
\smallcaps{Sheridan Ford}

\end{LARGE}


\vspace{0.5in}

\begin{Large}

\noindent There was more quality in the news

\noindent\hspace{.15in} Some fifty years ago

\noindent Than, with all their prattle of `progress,'

\noindent\hspace{.15in} The current journal shows.

\vspace{.1in}

\noindent The modern newspaper has caught to be

\noindent\hspace{.15in} A kind of Pedler's pack,


\noindent With less grip of Life's moving unities

\noindent\hspace{.15in} Than rules the pedler's clack.

\vspace{.1in}

\noindent The clean sense of convincing relations

\noindent\hspace{.15in} Is wholly lost to view

\noindent In the hodge-podge of undigested slop

\noindent\hspace{.15in} Served in the daily stew.

\vspace{.1in}

\noindent The thought of integrity in news

\noindent\hspace{.15in} (The truth entirely freed)

\noindent Is one with the notion of government---

\noindent\hspace{.15in} The social daily need,

\vspace{.1in}

\noindent Communication parallels Commerce,

\noindent\hspace{.15in} And Commerce, or the State

\noindent Never reaches full organization

\noindent\hspace{.15in} Till the facts are `straight.'

\end{Large}




% MAINMATTER
\mainmatter\pagenumbering{arabic}\setcounter{page}{1}

\thispagestyle{plain} % empty
\mbox{}


\begin{fullwidth}

\part{Reforming the News}


\end{fullwidth}

% DRAFT OF ACTION
\chapter[Draft of Action]{Draft of Action}
\label{ch:Draft of Action}
\chaptermark{DRAFT OF ACTION}

\vspace{.2in}

\begin{LARGE}

\smallcaps{Franklin Ford}\marginnote{Written in 1892 in Ann Arbor, Michigan. Printed but not published.}

\end{LARGE}

\vspace{0.5in}

\hypertarget{i-discovery-of-the-principle}{%
\section{I. DISCOVERY OF THE
PRINCIPLE}\label{i-discovery-of-the-principle}}

The time has come when it will pay to act on the reality underlying the
existing newspaper. The barriers down,~i. e., all hindrances to the free
movement of intelligence removed, through the completion of the machine
for gathering and distributing news, (this machine, consisting of the
printing press, the locomotive, the~telegraph~and their belongings) the
newspaper presents itself to us as a unified thing---the business of
dealing in intelligence. In this way the journalist, hereafter the
typical man of letters, comes to have a definite position in life
independent of all other vocations, professions, or trades. He has a
commodity of his own---the truth. This discovery marks the appearance of
a new commodity in commerce. A given thing functions, gains its proper
status, as commodity in commerce when it may be dealt in profitably at
its highest reality.~From this point of view the need has been to study
publicity as commodity, with a view to organizing~intelligence for the
contemplated advance of the newspaper; it becomes clear that to move the
daily newspaper forward is to strike at the base of the whole publishing
structure. The sum of the publishing business at a given date is the
amount of intelligence brought to the~centre~and distributed. Thus
regarded, the daily newspaper is about to become the leading publishing
interest, if it has not already reached the point. In contrast therewith
the so-called book business is to be secondary or accessory. At least
the daily newspaper holds the key to future development.~~

Let me particularize further. Rightly understood, advances in the
newspaper until now have been so many improvements in the physical
machinery which the newspaper uses.~Thus~the \emph{London Times} gained
its advance through being first to carry into practical operation the
cylinder printing press. The Walters had the lead in adapting steam to
printing newspapers. This was in 1814. In those days and through the
decades following, the struggle was to secure and apply increased
mechanical power. As opposed to this, the newspaper has now at its
service a perfect working machine. Under the machinery idea, as already
indicated, are included also the locomotive and the telegraph. The
advance movement rendered possible by this discovery must~in the nature
of things~be the sum of all previous advances, being nothing less than a
new ordering of intelligence. There comes in a change in the power of
thought---a forward movement in consciousness. The need has been to set
about organizing intelligence by the new light; this to compel a prime
movement in literature with the daily newspaper as its~centre~of
action.~

Clear seeing in this matter on my part dates from 1883, when I was
editing the newspaper, \emph{Bradstreet's} (New York). To carry out the
idea thus conceived, I first sought to convince the practice men---the
directors of existing newspapers---that the time had come to act on the
reality, the underlying principle, of their business; to give way to the
free movement of intelligence; that the movement was about to compel
this. I tried to show them that it would pay best to gather and
distribute the facts, the whole truth regarding all phases of life,
without deferring to class interest; from the very~fact that nothing
stood in the way of centralizing intelligence, class interest could be
ignored, and for the first time. The proposition was to organize inquiry
and so unify the newspaper, thus getting rid of the editorial page. The
aim was to convince them that the time had come when it would pay to act
with the eye single, and that it would not pay very long to act
otherwise. All sorts of obstacles intervened. But each hindrance went to
furnish new leverage. Quite naturally, the men to whom I talked refused
to believe that their business could be furthered through taking its
inherent principle as the sole guide to action. So-hard is it to believe
that the principle of journalism is that of intelligence itself. They
were making money and was not that enough? It was even said that the
people do not want the truth, and it appeared useless to urge in reply
that a given newspaper is sold to the people only by virtue of whatever
of truth it contains. My reception~on the whole~was about such as that
of Sir Henry Bessemer\footnote{{[}Henry Bessemer (1813--1898) was a
  British inventor and industrialist whose steel-making process would
  become the most important technique for making steel in the nineteenth
  century. Ford also refers to Bessemer in ``The Press of New York---Its
  Future.''{]}} must have been, had he gone to and~for~seeking to
communicate to the~iron men the chemistry of steel making. I found
myself in the atmosphere of opinion. The crush of fact was not welcome.
In the search for help and co-workers among active newspaper men, I
visited in 1887 all the news~centres~of the country east of the~Rocky
mountains.\footnote{{[}Ford also narrated this trip in a letter sent to
  Edward Atkinson on April 13, 1887. See ``Banding Together the Leading
  Newspapers.''{]}} At the last the conclusion was forced that direct
aid could not then be had from the newspaper managers, nor from the rank
and file of journalists~as co-workers.~

The friction with these men added to previous hindrances brought me to
understand something of the wealth of suggestion bound up in the
thought. I had gotten hold of nothing less than a new sense for news. It
became plain that a good share of the stuff printed from day to day in
the papers was no longer news. On the first appearance it was news
because people were surprised~at seeing such things in print. But now
the stuff had lost the element of surprise, and was therefore no longer
news. The papers were filled with unrelated matter which was lacking in
general interest; the generic thing, the life element in news, was
absent. Merely individual things had come to be widely mistaken for
news. The newspaper was off the track, was caught in its own machinery.
The physical advance---better printing facilities, cheap~paper~and the
like---had outrun the spiritual movement. The only way out of the
confusion, the only way to new life and meaning, was through organizing
intelligence. To produce the new goods, help must be got from the
primary men. The changes in our ways of thinking consequent upon the
appearance of intelligence as commodity had to be spelled out.~

\enlargethispage{\baselineskip}

I now set out to find men professedly attached to the principle of
intelligence and seeking to follow its dictation. To this end I turned
to the universities. The need was to find a university which could
become a nursing ground for the new ideas already flowing from the
mother thought that had come to me. Failing to compel the practice men
to a belief in the basic principle of their own business, the
alternative was to bring the academic men to believe in the practice, to
show them that their business was to carry the organic principle into
action, and so complete the revolution in the publishing business.
Losing no time, I began extending the acquaintance already made among
university people. In 1888 I spent four months getting at the habits of
thought of the university men who have to do with teaching philosophy
and politics, visiting Harvard, Yale, Cornell, Johns
Hopkins,~Pennsylvania~and Michigan universities, and besides taking a
further look at the men in Columbia College, New York. The~necessity was
to find a man and men who would do more than give a passive~assent to
the principle. In this I succeeded at the University of Michigan. I got
to John Dewey, who has the chair of philosophy in that institution.
Having the sense of~politics,~he was able to comprehend the scope of the
principle and its practical bearing on the publishing business. That
recognition and support should have come to me here instead of in the
east is traceable, I think, to the freed conditions existing at the
University of Michigan, and to the further immediate fact that President
Angell in making additions to the faculty has an evident liking for men
rather than pedants.~

The situation is suggestive. It is indeed fitting that the movement
proposing an advance in the publishing business should have its
fountain-head in the heart of the country. In more ways than one, it is
a movement of the country upon the town. Ann Arbor is about~equidistant
from the Atlantic coast and the Mississippi River. The St. Louis and
Kansas City papers arrive here in the same mails with those of Boston.
The 3,000 young men and women now gathered at Ann Arbor from all parts
of the country make the place a great recruiting station. The importance
of this will be understood when it is considered that another~class~of
newspaper workers is in great part required. The new ordering of
intelligence presupposes moral daring and proceeds through an
integrating movement of men.

\hypertarget{ii-the-intelligence-triangle}{%
\section{II.~THE INTELLIGENCE
TRIANGLE}\label{ii-the-intelligence-triangle}}

The movement proposes three incorporations: (1) The News Association,
(2) The Class News Company, and (3) Fords. The News Association conducts
a general publishing business at New York~putting out daily and weekly
newspapers,~leaflets~and books. The Class News Company gathers and sells
the facts relating to whole classes or social groupings. It puts out
class papers from the \emph{Chemical News} to the weekly paper
\emph{Fruit} or the daily paper \emph{Grain}. Fords is the bureau of
information, standing for the individual application of the fact. The
news movement is thus conceived as~


    


\begin{figure}
 \setlength{\parindent}{145pt}\includegraphics[width=\linewidth]{graphics/image-nine.png}
   \label{fig:fig3}
\end{figure}


The movement proceeds from the physical fact, which presents itself in
three aspects, each of these yielding a special profit. Let me speak by
illustration. There is in hand a report on the sheep industry of Texas.
It has, first, to be read in its bearings on the whole people. The
extent of the destruction by foot-rot may have been so great as to cause
wide-spread suffering. Sympathy goes out to the stricken region in
contributions of money from other parts of the country. And more. A
demand may arise for the appropriation of money at Washington. The need
is to have the fact reported in the interest of the whole. Here is the
general news side of the fact, the peculiar field of the News
Association. Beyond, there is the direct bearing of the report on the
price of wool and the state of the sheep industry both at home and
abroad. This side of the fact will have special and technical treatment
in, to illustrate, the newspaper~\emph{Wool}, thus disclosing the
province of the Class News Company. Two profits have been taken from the
one basic fact. The third profit reveals itself in this way. Succeeding
the publication of the social and class renderings, various individual
applications at once arise therefrom. The man who has placed \$50,000 on
farm~mortgages~in certain of the counties of Texas affected by the sheep
rot wants to take account of the fact---he wants to know what there is
in it for him, how far his interests are hurt. He applies at the office
of the News Association or of the Class News Company and is referred to
the office of Fords---the bureau of information. Following upon
negotiations a special inquiry may be made for him at an agreed price.
Again, a report of the state of the textile industry in Germany is no
sooner received and published than a New England manufacturer wants help
in~determining its bearing on the demand for his make of goods. Three
profits are thus indicated, standing for the three reporting directions.
But the one organization of intelligence stands for the whole
movement---this with respect to ownership or control. The country is
reported primarily through the News Association, the parent concern
which owns the library, or fact accumulation. The special or individual
renderings are made through the supplemental organizations.~

The News Association stands for the full social inquiry and
through~it~enthusiasm is let in. Men having the right zeal, for inquiry
could not be enlisted for the work of the class journals or for the
bureau of information alone. For these men the universal (full social
inquiry)~has to~be let in. The movement~has~to~come full circle. Unless
the man-of-letters artist can organize for reporting the whole of life,
he is crippled in action. The artistic impulse is its own law---the law
of the whole. Inquiry can organize only in obedience to this law. The
principle accepted, the man of letters has a business of his own.~

Through acting on the universal we get the detailed results indicated.
Here is the unity in diversity. To centralize~on the basis of~the
thought presented is to go furthest in compelling the one life to yield
up its infinite meanings. This triple news movement is but the practical
outcome of the organization of inquiry. It is compelled by the advance
of letters to the fact. The Class News~Company and Fords have close
business relations with the News Association. A contract is conceived by
and between the News Association and the two secondary concerns. It is
proposed further that the News Association shall own a controlling
interest in the Class News Company and Fords. It is the Intelligence
Trust.~

\hypertarget{iii-the-news-association}{%
\section{III. THE NEWS ASSOCIATION}\label{iii-the-news-association}}

I am setting out a prime movement in the publishing business with the
daily newspaper as the center of action. The centralized inquiry cannot
have less than the daily newspaper as its organ. The News Association
conducts a general publishing business at New York. Three daily papers
are conceived as follows:~

\begin{itemize}
\item
  \emph{The Newsbook}~(the political newspaper)~
\item
  \emph{The Town}~(the lesser daily)~
\item
  \emph{The Daily Want}~(the special advertising medium)~
\end{itemize}

The line dividing\emph{~Newsbook}~and~\emph{Town}~is the leading
principle of news classification. Like conversation the news business
classifies according to relationship. Journalism, the registration of
life through newspaper, leaflet and book, is but conversation writ
large. The politician or the merchant who reads the~\emph{Newsbook},
wants the latest reports of the growing wheat crop and something of the
price of bread as well. The shop-girl who prefers the\emph{~Town}~wants
of the two reports only that concerning the price of bread. But both
wheat-crop and price-of-bread report must come from the one institution.
The two reports are but phases of the one basic fact. This news
classification has already appeared in rough outline. To illustrate, the
\emph{New York Times} stands, if you please, for the political
newspaper, and the \emph{Morning Journal}, also of New York, for the
lesser daily. The principle is not realized sufficiently by either of
them to make a clear guide to action. Nor can the principle be brought
to full consciousness short of conceiving the organization of inquiry to
the full and the centralized effect. The~\emph{Town~}is an~all
day~affair appearing first at noon unless important news developments
should compel an earlier edition. The afternoon paper, as such, is done
away with. The\emph{~Town~}appears at intervals up to six o'clock and
later should the news supply compel.~

The~\emph{Daily Want~}answers to the clearing-house principle of the
great city. It gathers to itself the ``want'' advertising. It is the
city's annunciator. Advertising is of two sorts. In the one case people
are seeking the advertisement, in the other the advertisement is seeking
the people.~Thus~the card of a man who wants to buy a dog is eagerly
sought after by all having dogs to sell; it is to them news. The other
advertisement is that of the dry-goods merchant which is brought to the
eyes of the people through the proximity device, that is, by~being
placed near attractive ``reading matter.'' The~\emph{Daily Want}~draws
to itself the advertisements that are news---that is, distinctly so,
such as the theatre advertisements, railway time-tables, etc. Besides it
contains the court announcements, real-estate transfers, a list of
near-by public meetings or events, etc., etc. Conditions have ripened
for introducing the~\emph{Daily Want~}in some half-dozen of our largest
cities. The organization of New York City therefor will furnish the
model for Philadelphia, Chicago, Boston, St.~Louis~and Baltimore. It is
a class paper but so nearly general in its scope that it belongs with
the publications of the News Association instead of with those of the
Class News Company. Already in Berlin and other~European cities, daily
papers containing only advertising are in full movement.~Indeed~this has
been true for years. The publication of a city's~``want''~advertising is
an ordinary agency service and it must, perforce, be done at prices to
correspond. It is not possible to maintain for this service ``monopoly''
rates---such as Bennett\footnote{{[}James Gordon Bennett (1795--1872)
  published the first penny paper, the \emph{New York Herald}, in 1835.
  Ford also refers to Bennett in ``The Press of New York---Its
  Future.''{]}} was getting in New York for the \emph{Herald}'s
real-estate advertising prior to the advent of Pulitzer. During the
first part of the last decade, the \emph{Herald} received well toward
\$250,000 annually from the real-estate advertising alone. In this and
other lines as well, it was charging more than the traffic would bear.
Herein lies one secret of what Pulitzer did. Taking advantage of the
\emph{Herald}'s exorbitant rates for the ``wants,'' he was able by much
shouting to create a new medium therefor. But the New York papers are
still quarreling in the market-place over the ``wants,''~with no one of
them perceiving that this branch of advertising must find its way into a
special~medium.~The price of the\emph{~Daily Want~}is the lowest coin of
the realm, with us~one cent. At times the~\emph{Want~}will contain
reports of special interest to its patrons, in this respect functioning
as a trade paper. Thus, it would report the organization of the
boarding-and-lodging house interest of New York with more regard for
particulars than would the~\emph{Newsbook~}or the~\emph{Town}. It will
not be a difficult thing to organize New York City for issuing
the~\emph{Want}. It can be done in advance. The leading classes of
advertisers can be seen and convinced beforehand. In relation to this I
have made certain tests. I found the New York real-estate men and other
leading advertisers sympathetic in relation to the project. At the time
in looking over the ground I obtained the opinion of William R.
Grace,\footnote{{[}William Russell Grace (1832--1904) is the founder of
  W. R. Grace and Co., a large industrial company active in natural
  resource and shipping. Grace served as New York City Mayor from 1881
  to 1882 and from 1885 to 1886.{]}} the well-known merchant. Mr. Grace
was unqualified in asserting that the~\emph{Daily Want~}is the key to
the immediate newspaper situation in New York. The success of
the~\emph{Want}~will be furthered (1) by its low price; (2) by its
classification of advertisements; (3) by lower advertising rates
consistently maintained. In the New York World the\emph{~Daily Want~}is
already marked out. The need is to change the name, to cut the price to
one cent and carry classification to the full. This done, the new medium
must obtain.~

Already this advertising sheet is differentiating itself from the
general newspaper. The separation is plainly~revealed in the
\emph{Philadelphia Record} which is aiming to classify and distinguish
the city's announcements, thus indicating the rise of the \emph{Daily
Want}. The paper will contain only such ``reading'' matter as directly
relates to its central purpose. The economies involved in the idea of
the\emph{~Want~}are great. It goes to set the whole field of advertising
in order. Advertising rates for ``wants,'' while in no direction
extortionate, will yet be such as to yield a large commercial profit. My
notion is that the~\emph{Daily Want}~should pay into the treasury of the
News Association a yearly net profit of say \$50,000---may be more.
The~\emph{Daily Want}~is the wedge with which to separate~the existing
newspaper structure. In important respects the existing newspaper is the
country store in which treacle and calico mean classification. As I take
it, the elder Bennett's working concept was that the daily newspaper
should contain something for everybody. We pass from this to the idea
that the~particular newspaper~contains everything for somebody. In the
\emph{Newsbook} the politician or the citizen finds from day to day all
the facts answering to his relationships. Those having in a way narrower
interests find the facts contained in the \emph{Town} sufficient for
their needs. The routine life of the city clears itself through
the~\emph{Daily Want}.~

Chicago,~Philadelphia~and Boston are, I think, ready for
the~\emph{Want} newspaper. Perhaps the other cities named above are also
ready for it. The smaller cities will come to the idea later. It is
probable that any city of 300,000 people would support the daily
announcement sheet. It must be borne in mind that the circulation of
the~\emph{Want}~has its value in the fact that it goes to the people who
are seeking the ``want'' advertisement and that sales of papers beyond
this do not add materially to its value. Bennett charged exorbitant
rates for real-estate advertising on the plea of the great circulation
of the \emph{Herald}; whereas the sale of papers beyond the buyers or
sellers of real-estate had little or no value, that is, for this class
of advertisers. In putting out the~\emph{Daily Want}~the aim will be to
fix upon just advertising rates---such as can be~continued indefinitely
and will yet yield a profit.~

These three journals---the\emph{~Newsbook}, the~\emph{Town},
the~\emph{Daily Want}---are together the organs of the State in the
social region; through these~all incoming~facts are rendered in the
light of the general interest. In place of writing about sociology, so
called, we proceed to publish the sociological newspaper; that is,
recognizing the social organism as attained in fact, we set about
reporting the state thereof. The principle of socialism is division of
labor. This gained in the mental region, through the organization of
intelligence, socialism is here. We distinguish the grand division of
labor. The principle~reached,~the social system is discovered. In place
of discussing ``socialism,'' we put out, in the rightful sense of the
word, the socialistic newspaper---the organ of the whole. Apprehending
this the people will lose whatever of remaining interest they may have
in editorials on the ``social problem'' and the like. The social crisis
is passed in the moment of its discovery. We recover the true meaning of
the word sensational now obscured by the falsely sensational. We
undertake to be sensational to the last degree. It is, of course, only
possible to compete with the present ``sensational'' newspapers by being
more sensational than they. Getting back to the true meaning of things,
it is seen that the craving for sensation on the part of the public is
the demand for intelligence itself. It is the business of the newspaper
to meet this demand. The social fact is the sensational thing. News is
the new thing. In truth the only sensation is a new idea. We are thus at
the gateway of the highest sensations---those relating to the integrity
of the organized social body and in turn to the welfare of the
individual. The state seen in its simplest aspect, the division of
labor, becomes an actuality, a moving mechanism of pulleys and bands and
bearings. Having the sense of~direction~we may now point out a social
hot-box or a slipping-belt with the clearness and facility of the expert
machinist. It is a question of form. The social movement is viewed with
equal regard for the individual and the common good.~Thus~the highest
sensations are at command. The \emph{Newsbook} and the \emph{Town} have
no editorial page or ``composition'' department. The money-force now
wasted in that direction will be spent in systematic inquiry. The editor
of the paper is simply its chief reporter. To his eye a murder in the
fourth ward, a big cotton crop in India, an injury to Patti's throat,
the result of a presidential election, a new sonnet, a new principle in
politics~are~all news. Its news columns make up the whole paper. The
advertising is itself regarded as news; it is the conveyance of the
private or individual intelligence. Already the newspaper of the day is
changing in this direction. In the end a large part of what is now
``reading matter'' will go over into the advertising columns, which in
consequence will become more interesting. Merely personal
intelligence---the departure of a tailor for Europe or the like---should
be paid for as advertising under its appropriate heading. A few
years~ago~the \emph{Philadelphia Record} contained a long advertisement
giving the sermon of a local preacher paid for line upon line. It is
believed that the advertising which appeals to the whole people will
have continuous development along with other changes in the general
newspaper. Mr. Dana,\footnote{{[}Charles A. Dana (1819--1897) was an
  American journalist and politician who served as the editor of the
  \emph{New York} \emph{Sun} for thirty years (1868--1897), during which
  the paper occupied a prominent place in American public life. Among
  his various contributions to the advancement of journalism, Dana's
  \emph{Sun} is known for being one of the first papers to hire college
  graduates in the 1880s and for first coining the most enduring
  definition of news as man-bites-dog. See Janet E. Steele, \emph{The
  Sun Shines for All: Journalism and Ideology in the Life of Charles A.
  Dana} (Syracuse: Syracuse University Press, 1993).{]}} of the
\emph{New York Sun}, at one time put out the suggestion that the day was
coming when we should have newspapers without advertising. As against
this the \emph{Sun} is now striving in the marketplace for the ``wants''
along with the \emph{World} and the \emph{Herald}. The reality Dana was
seeking lies in finding directing principles for bringing order out of
the present confusion in the advertising business. Large advertisers
like Wanamaker, and certain special agents, have been leading the way in
a measure. Gradually the general newspaper will come to place writers of
skill at the service of advertisers.~Thus~the writer of~``puffs''
and~``notices'' will become an advertisement writer, and this while yet
in the employ of the newspaper. The great retail houses will not be so
much~alone in furnishing attractive advertising for the eye of the
public. The introduction of the~\emph{Daily Want}~will go to set in
order the whole advertising field.~~

\hypertarget{iv-the-class-news-company}{%
\section{IV.~THE CLASS NEWS
COMPANY}\label{iv-the-class-news-company}}

Consulting our triangle again, it appears that various groupings of fact
appeal directly only to certain~classes of the people. From this we have
class papers and the Class News Company which publishes these papers.
Gradually the Class News Company will come to take charge of the whole
field of class knowledges, special scientific journals coming under the
same view. In other~words~the economies involved will compel a unifying
movement covering the whole circle of class news. A new fact in
chemistry has possibly some immediate social outcome, some new decency
or luxury of life---this for the~\emph{Newsbook}; it then has its class
value for the professional chemist. What is true for chemistry will hold
good in other lines and so make the circle. From one point of view, the
Class News Company is in the business of publishing ``trade'' papers.
These trade or class papers constitute the base of the organic
publishing business. Through them the centralizing movement I am
outlining gets its fulcrum, its position or place, in the physical
commerce. There are now fully 1,000 trade papers in the United States
alone---the growth of twenty-five years. Of these the agricultural
sheets in one form or another number 300. The revenues from advertising
in the various trade papers would sum up several million dollars. I do
not think six million would be a high estimate. In this we get to know
somewhat the amount of the tax collected from the physical commerce by
the trade paper men. In part these papers speculate on a great want
without supplying it.~Again~it is true that some of them fairly meet the
needs of their groupings. To illustrate, the \emph{Hub} newspaper is
perhaps adequate to its end as the organ of the carriage trade. It thus
appears that certain minor branches of trade are best served. The waste
of force in the trade paper business through lack of a centralized
movement is, of course, very great. It is not proposed that the Class
News Company shall at once set about buying trade papers in order that
all maybe brought under one ownership. Premature action of the sort will
be carefully avoided. The errors made in centralizing certain industries
will not be repeated on the spiritual side. What I mean here is that in
more than one centralizing movement on the physical side, unnecessary
increase of capital came in through buying out factories or plants which
might better have been left to~their own devices. The men who have had
to do with these movements see things in clearer light now. Talking with
one of them, he agreed that in some instances it would have been better
if two or three large concerns had trusted to the central principle as
the key to empire, leaving the outside concerns to break themselves
against the principle. Rightly understood, the principle looks to such
an increase of competition that the weaker concerns cannot compete; it
is the big grist and small toll. The action of the News Association is
so deliberate, its body of directing principles having been worked out
and articulated, that it can easily avoid the errors made on the
physical side. The action of the dependent corporation, the Class News
Company, partakes of this deliberateness. The wide-extended reporting
organization compelled by the News Association is in its results at the
service of the Class News Company.~

The class journals first to be put out would relate to such minor
branches of the physical commerce as are not already covered. These
groupings may be dealt with by virtue of the wide-spread organization
that is to come in. Each newspaper is to meet a plain need.~Thus~the
paper~\emph{Fruit}~might perhaps be the first publication to be issued
by the Class News Company.~\emph{Fruit}~would be a weekly paper. Let me
say here, however, that in publishing the weekly newspaper~\emph{Fruit},
leaflets containing reports of injuries from the weather would at times
be given to subscribers in advance of the regular publication day. The
telegraphic report of a nipping frost must go out at once. In the main,
the publications projected in this region will relate to the primary
groupings of the physical commerce---food,~metals~and textiles. The
journal \emph{Fruit} illustrates the advance here.~A number of~the
papers will be small dailies of the leaflet sort, as the daily
paper~\emph{Grain}. The motor-idea is the pursuit of the price-making
intelligence. The merchant, the manufacturer is eager to get at the
price-making influences in his own line of business and is prepared to
pay therefor. These he cannot well get, save as coming to him through
the organized intelligence. The movement will take close account of all
commodities whose prices are directly and easily affected by the~news of
the day, whether this be a bad turn in the weather, the exhaustion of a
mine, or rumors of foreign war. As already indicated, this trade paper
development forms the base of the enterprise.~

The very notion of the price-making intelligence as constituting a
distinct branch of news-getting is of recent origin. It first took clear
shape with me in setting about developing \emph{Bradstreet's}
(newspaper). In consequence of the experiments made there, I wrote in
\emph{Bradstreet's} August 26, 1884, as follows:

\begin{quote}
A line is ultimately to be drawn between governmental and newspaper
publicity. Wherever the sovereignty~has to~be impressed in order to get
information the work of the government is a necessity. But when the
required facts are reached through a potent sympathy and far-reaching
activity the collecting agent is the metropolitan newspaper. . . . It is
believed that the commercial side of New York journalism has not kept
pace with the gradual change of conditions. Twenty-five years~ago~the
Produce Exchange reporter had little more than tables of prices to
present.~To day~through the ramifications of the telegraph, the
influences that make prices are very largely at the command of the
journalist.~
\end{quote}

All the developments since the above was written have gone to make good
the position taken. The new journalism is to build from the soil---its
organic base in the physical commerce. The way has been prepared
somewhat by government work, but crop reporting through government
agency is and must continue to be perfunctory. The present slowness in
giving out returns will not answer. The reporting must be done day by
day. But as indicated, this is not all, sympathy~has to~be enlisted. The
reporting machinery here is primarily the social organism itself. The
citizen king is the crop reporter. He registers his county fact, say in
the black belt of Mississippi, receiving in return from the central
inquiry office the cotton belt fact, getting also the crop prospects in
Egypt and India along with the state of the spinning industry. This news
reaches him through the daily paper\emph{~Cotton}~which is published for
one point, say, at Columbus, Miss. The integrity in this initial act of
registration is secured from the fact that the master interest of the
reporter (in, if you please,~Lowndes County, Miss.) lies in directly
furthering the accuracy of the report from the whole cotton belt.
Besides, he wants correct reports from Egypt, India, and from
Manchester, but these he cannot have unless trustworthy reports from
America can be given in exchange for them. Wanting the whole~truth~he
will contribute his fact thereto. The central office sells truth for pay
and may be trusted to protect its brand. It is the play of sympathy---of
interest. Straight goods all 'round or no trade. It is the bartering of
intelligence---the great transaction of life. The freedom required
cannot be gained through the delegated authority called government. Here
is the grand division of labor. As it is reached, the organism is
detected and we gain the line between governmental and individual
activity. The man of letters---as journalist, as diurnal
man---functions, and one of the things falling to him is the whole
business of crop reporting. The machinery is provided within the social
organism.~

\enlargethispage{\baselineskip}

I need not dwell on the argument. The principle was established through
my experiments in \emph{Bradstreet's}. It only needs to be stated. Now
when the crop report of a given state of the union can be taken by
telephone in a few hours, the cumbersome methods of a government office
will not do. Various trade organizations have tried for the
mastery---notably the National Cotton Exchange ---but without avail. Not
long ago the Chicago Board of Trade pointed out the need of better crop
reporting, but nothing was or can be done. The individual trader, the
trade exchanges, the government have tried it and all have failed. The
one way out is through a ground movement in journalism. Only in this way
can the organization become automatic. There must come in the daily
newspaper~\emph{Grain}~and other like publications. One of these might
be the paper~\emph{Meat}. Here is the journal that will be produced
simultaneously at various points---at all the great markets for a given
staple. Take the paper \emph{Grain}; it should ultimately be printed
each morning, say at New York, Chicago, St. Paul, Omaha, Davenport,
Kansas City, St. Louis, San Francisco, Portland, Liverpool, Paris,
Vienna, Odessa, and Calcutta, proceeding at all these points from the
branch offices of the~News Association. The journal~\emph{Cotton}~would
be published at the leading market towns in the American cotton belt and
at the great cotton industry~centres~here and in Europe.~

It is not possible to determine before entering upon full action just
how far classification will be pushed. We organize in all directions on
the reality. I can only point out that it is the leaflet concept in the
publishing business on which we are acting.~In all likelihood~the
principle of everything-for-somebody must obtain in the region of
political or social news beyond anything we can now anticipate. The
\emph{Newsbook} itself may have detachable leaflets. We are dealing in
intelligence and it must be delivered to customers with strict regard to
their convenience.~

The price-making fact is the~centre~of interest. The surprising thing is
the tremendous news development in the region of the physical commerce
consequent upon the new point of view---the whole truth. As at first
conceived the movement took the shape of three weekly papers to be
called \emph{Food}, \emph{Metals}, and \emph{Textiles}. At the time I
received a memorandum from Edward Atkinson concerning the first named.
He gave most striking illustration of the volume of news that must at
once result from full inquiry into the food question. Mr.
Atkinson's\footnote{{[}Edward Atkinson (1827--1905)
  was~a~cotton~manufacturer,~economist, political
  activist,~and~inventor.~When~his~cotton~mills~began~to fail in the
  mid-1870s,~he~entered~the~railroad~industry and
  later~worked~as~President~of the Boston
  Manufacturers~Mutual~Insurance~Company.
  Known~for~his~abolitionist~involvement~with~the~Free Soil Party~and
  the~Boston Vigilance Committee, he also founded
  the~Anti-Imperialist~League, which~opposed~the American annexation of
  the Philippines in the late 1890s. Based on an in-depth study of
  cooking (energy consumption cost, food chemistry, nutritional intake,
  etc.), he designed the ``Aladdin Cooker," a device which prefigures
  the modern crockpot. He~was~elected~a~Fellow~of the~American Academy
  of Arts and Sciences~in~1879. The correspondence of Ford and Atkinson
  lasted from 1885 to 1889, and this book featured some of this
  material, which covered a crucial period during which Ford quit
  \emph{Bradstreet's} (to which Atkinson contributed articles) and tried
  to initiate practical attempts at reforming the press.{]}~} paper is
not in convenient form for insertion here else I would give it.
Re-arrangement would make it mine not his. The service of these primary
groupings waits upon the full organization of inquiry---the centralized
intelligence. Although food, metals, and textiles are now heavily taxed
by numerous papers, the range of fact is so great that no one of these
three great divisions can afford to raise up the necessary machinery.
The end can be reached only through the economy of a single
organization. All news directly affecting prices in this region will be
transmitted by telegraph---this whether it be a big wheat crop in India,
a decrease in the cost of aluminum, a small peanut yield in Virginia, or
a failure of the raisin crop in California.~

It is believed that the closer this class news is brought home to the
need and convenience of merchants and producers through the projected
daily and weekly papers (such as \emph{Grain}, \emph{Meat},
\emph{Cotton}, and the like) the greater will be the attraction to
advertisers. The gathering of advertising to a given trade paper must
increase just in proportion to the completeness with which the news
belonging thereto is delivered. The evolution here is with the organized
intelligence.~

The general newspaper as it stands to-day, particularly at New York, is
undertaking the impossible task of neglecting or foregoing the facts of
the physical commerce. The attention is so much concentrated on the
by-play of life that the reflection of the life actual has fallen behind
relatively. As things are, a divorce has come about between the
newspaper and the actualities of life.~A number of~years ago the
\emph{New York Times} published about the same date each fall extended
interviews with leading merchants concerning trade prospects. It has
since taken to printing ``puffs'' for the merchants for pay---these
taking the place of the former interviews. So that instead of advancing
to the organization of inquiry, the newspaper, as illustrated in the
case of the \emph{Times}, yields its~position~and becomes more
bush-whacker than ever. The men of the physical commerce~are so far in
possession or in control of the newspaper that the edge of inquiry is
turned. In great part the advertisers are editing the paper---that is,
so far as it is edited at all. Not having discovered~as yet~that he has
a business of his own, through possessing the commodity intelligence,
the newspaper publisher is as yet under the heel of class interest. He
can free himself only~through acting on the unity of inquiry.~

The point to be conveyed here is so important---so far-reaching in its
import---that it is worth illustrating. Meeting one day in New York the
chief advertising runner of the \emph{World,} he stopped to tell me that
he was about correcting what he thought to be an act of injustice on the
part of his paper. I think it was in 1885. He told me that for months
the \emph{World} had been publishing all sorts of stories concerning
poisoning from eating canned goods, and this without caring greatly as
to their truth or falsity. Continuing,~he said, ``It is all wrong, I am
going to have it corrected.'' I asked, ``How will you do that; by having
the \emph{World} look into the facts and take back all mis-statements
made?''~``Oh no,'' replied~he, ``There is a better way than that. I am
getting up a two-page article on the canned-goods trade; I have it
nearly ready; the~Thurbers,\footnote{{[}Horace K. Thurber (1829--1899)
  and Francis B. Thurber (1842--1907) were American businessmen and the
  heads of H. K. \& F. B. Thurber \& Co., a food manufacturing company
  specialized in canned goods.{]}} Austin-Nichols\footnote{{[}Austin
  Nichols \& Co. was a wholesale grocery business operating from New
  York City.{]}} and other leading canned-goods people being
represented.'' The charge was to be fifty cents or more a line; I think
he said a dollar. In few days out came the two-page ``correction'' in
the \emph{World}. Asking F. B. Thurber about the matter a few days
after, I think he told me that the sum paid by his firm on account of
the enterprise was \$300. The illustration is not extreme. It serves to
bring out the present confusion in the daily newspaper between ``printed
matter'' and advertising---between the public and the private
intelligence.~

Such action as that of the \emph{World} may be called making money~both
ways; untruth is first sold after which a heavy charge is made for
correction. Thus, as I have said, the edge of inquiry is turned. Two
falsities or at least two~half truths~take the place of the whole truth.
There results a maximum of printed matter and a minimum of fact. The
moment the point of view of the whole truth is taken, the volume of news
rises in all directions. The confusion here can only be temporary. The
transition, and the only one possible, is to begin systematically to
sell the truth about merchantable things---the price-making news---and
so make the ground connection for the great advance of letters. We do
this through the Class News Company---one side of the intelligence
triangle. To do it is to seize upon the new endowment fund. The enormous
revenues accruing to the trade papers should contribute directly
and~definitely to~the furtherance of inquiry. The endowment of letters,
as it has been called, is an~ever increasing~sum. To-day it is the gold
which the captain of industry stands ready to pay for the price-making
facts on which the prosperity of his business is turning. I have been
told that the annual net result in money to the proprietor of the
\emph{Iron Age} is \$100,000. This profit of the \emph{Iron Age},
the~million~of money which Bennett has been able to take from the
\emph{Herald} in a single year, and the enormous revenues of the
mercantile agencies stand for the three sides of our triangle. It is
proposed that the three streams of money be brought within the grasp of
a single institution. At least half of the "agency" revenues should be
carried forward to the fact---to the bureau of information.~

Let me again illustrate the unified movement in contemplation. It has
been pointed out that a single fact of the physical commerce may have
three aspects; that is, it may be sold three times, yielding three
profits. Let us suppose that a report reaches the News Association of a
short wheat crop in a particular region of the northwest. It is first
sold in its detailed form to the grain men through the daily
paper\emph{~Grain}; then as regards its political or social bearing it
is sold to the citizens through the~\emph{Newsbook~}and the~\emph{Town};
and finally, for the third time it is disposed of on its bourse side at
the office of Fords to some customer who is anxious about its effect on
certain railway shares he has long owned.~

The class papers indicated must largely take the place of the present
system of reports by United States consuls in relation to foreign trade.
It is acknowledged that the consular reports lack that degree of
efficiency which can alone give them importance and value. Detecting how
three profits are bound up in the one fact we may send an expert to
Europe in the service of the class journal~\emph{Textiles}. He will
inquire, if you please, into the state of the dress goods trade. The
fact reported will have a social value for the~\emph{Newsbook}~and in
turn a value for the individual through Fords.~The point I would make is
that the expert cannot be sent save through acting on the unity of
inquiry---through perceiving the three profits in the one fact. The
scientific touch~has to~come in here else no inquiry. This done, a great
deal of reporting now carried on by consuls or other officials goes over
to the newspaper.~

\enlargethispage{\baselineskip}

Through these class journals, reports are brought into the service of
the~\emph{Newsbook},~thus clearing up the facts of the great business of
politics. The tariff question will then be reported.~Again~we are able
to see~how, through becoming the receiver and transmitter of the
price-making fact, the journalist gains the central position of
life.~Industry is at last organized because the intelligence peculiar
thereto has become organic to the whole.~~

\hypertarget{v-fords}{%
\section{V. FORDS~}\label{v-fords}}

Fords is the Lloyds of information; it is the universal truth-shop. It is proposed to put this advertisement before the New York public:
\begin{figure}
   \includegraphics[width=\linewidth]{graphics/image-ten.png}
   \label{fig:fig10}
\end{figure}

The idea of Fords came to me early in 1883. I received one day,
addressed to the editor of \emph{Bradstreet's} a letter from a man of
business asking if he could get a report on the agricultural conditions
and situation~in a given~belt of country in a western state. He had been
asked to have to do with building a railroad through the region
indicated and wanted the lay of the land. This inquiry could not be made
as there was no fit machinery for the purpose at command. There was no
man within easy reach who could be sent out to explore the region at a
price which the inquirer would be warranted in paying. Other like
inquiries coming to me, I was prompted to organize for such work---to
bring in an association of experts. Getting no sympathy for my ideas
from the executive of the Bradstreet Company, I saw more clearly than
had previously been possible that the Bradstreet organization and that
of the mercantile agencies in general could~on the whole~only compass
the gathering and sale of rumor. This went to show that an advance was
pending, that the bureau of information must come in; and, further, that
the movement was bound up with the newspaper advance I was plotting.~

Confirmations came to me in various ways. Other leading inquiries
reached the office which, as before, could not be handled. I received
offers of special salaries to aid leading trading concerns on the
statistical side of their business. It was learned from the managers of
certain state bureaus of statistics of increasing demands for
information from politicians, traders and manufacturers which could only
be met by a bureau specially organized on the commercial principle. It
had come about that in the organization of new firms, greater care was
taken to include, where possible, a man specially fitted to look after
the inquiry side. ~

Strong proof lay, as it seemed to me, in the state of the~``mercantile
agency'' business itself. As is well known, the two leading agencies are
the Bradstreet Company and R. G. Dun and Co. Working alongside of the
former, I was able to study the business closely. The two concerns were,
and are to-day, not much apart as regards extent of business done. The
gross annual income of each was found to be not far from \$2,000,000---a
total of \$4,000,000. Large profits resulted. The Bradstreet people were
then carrying a cash surplus of several hundred thousand dollars.
Regarding the present condition of the Bradstreet Company's affairs, I
have the following statement from a trustworthy source:

\begin{quote}
They have a capital stock of \$350,000 with an accumulated surplus in
actual good assets, equivalent to cash, of \$1,250,000. In
other~words~the concern is worth, I believe, \$1,600,000 in tangible
property and largely in cash and cash assets. This does not allow
anything for the good will and established business of the concern which
according to its splendid earning power is certainly worth a good deal.
. . . Their furniture, fixtures,~etc., in the various offices are put in
at very low valuation, which they would readily bring if sold out by the
constable. . . .~They carry large~cash balances with banks and trust
companies in New York, and . . . they have always from a half to
three-quarters of a million loaned out to trust companies, etc., bearing
interest. . . . It is~one of the best paying~businesses in the country.
. . . Their stock will readily bring 700 to 800~and it is owned by a
small crowd and very hard to get.~
\end{quote}

I found that certain business concerns had got into the habit of
alternating their agency subscriptions, giving to \emph{Bradstreet's}
one year, and to \emph{Dun's} the following. The Bradstreet people, at
least, talked about two concerns being necessary in order to keep up
what they called competition. Gradually it came home to me that the talk
of competition went to conceal the double tax that was collecting. Two
prices were charged for the one service, for the one fact. The merchants
of the country were forced to sustain the two concerns where one would
have done as well or better. This was before the elimination of distance
and the mother thought bound up with it, that publicity is a commodity,
had been detected. This discovery of course speedily cleared up the
situation. It became clear that the bureau of information could be
founded only as a phase of a general advance of the publishing business.
It was seen to be an incident of the organization of inquiry. The
truth-shop must follow upon, or, if one please, accompany the advance of
letters to the fact. Fords could not be erected until the man of letters
had a business of his own. He has now become independent through
possessing the commodity publicity. Peruvian bark did not serve as
commodity until it~was found profitable to extract its inner principle
to be sold in the form of quinine.~So~with crude rubber; Goodyear had
first to wrest from it its chemical secret.~~

To~meet the quickened demand for fact,~various attempts have been made
in~Wall Street~since 1885 to establish the bureau of information.
Failure attends them because of inability to act on the commodity
concept. The business of~reporting is mixed up with that of dealing in
the things on which reports are wanted---conditions inevitably
corrupting. Fords is the universal fact shop. Its business will be
confined strictly to the purchase and sale of intelligence.~

Fords is the shop in which the facts of life are handled and sold on the
lines of individual needs. The business is parallel with and dependent
upon the development of inquiry made by the News Association. The
organization of Fords is closely~related to that of the class papers
outlined, and in turn to the~\emph{Newsbook}~and the\emph{~Town}.
Illustrations multiply, but they cannot be put down here. The
intellectual movement is one. This cannot be too often asserted as
therein is bound up the economies of the enterprise and its great
commercial outcome.~

The mercantile agencies gather and sell rumor concerning credits. Rumor
is usually sufficient here. The standing of a cigar dealer may be
learned from the gossip of his neighbors and the statements of dealers
who sell him goods. To gather such gossip and statements is the work of
the agencies. Fords will not at the outset enter this field; that is, it
will not set out to report the credit of individuals or of trading
firms. But Fords will at once report the credit of corporations and this
in all respects. What will be the ultimate effect of the development in
hand on the field so left to the agencies, need not here be said. With
this provisional exception Fords will sell information of all sorts and
descriptions, from spelling a word or verifying a quotation to the facts
showing a change in the underlying conditions of some great industry, or
the extent to which a given railroad has been borrowing its dividends in
the guise of loans for additions to rolling stock. Its range of action
is co-extensive with the interest of the individual in life. Men with
perplexing ideas will come to Fords to get them appraised, or resolved.
Even the disturbed mind will apply at its counters for the relieving
fact. The great~specialists in mental science will be found at its
offices.~

The signal act of Fords will be to set about reporting in a
comprehensive~way the new registration of industrial shares on the stock
exchanges. This need~in itself goes~far to warrant the enterprise. The
idea on which we are working enabled me to predict in 1885 the great
incorporations of the following years which are still going on.
Distance~gone,~production organizes---centralizes.~Along~with this we
have the development of~``trusts,'' followed by tremendous corporations,
the end being to bring this and that great~industry under a single
direction. Already there is hardly room on the bourse for both the new
industrial shares and the old railway stock-and-bond structure. The
industrial stocks go to antagonize the railway obligations. Yet the
movement is certain to increase in every way. Correspondingly, the need
of information from an independent source becomes imperious; there must
come in a like movement in intelligence. This need is met by Fords,
which undertakes to report the condition of all trading and
manufacturing corporations in the United States and throughout the
world.~

The action is many-sided and here can only be touched upon. Just what
will be done as regards the issuance of regular printed reports cannot
be indicated in advance.~

To single out one thing that Fords is~to do, let me say that it will
organize the law so far as it remains to be organized, that is, on its
information side. To illustrate, a man of business comes to New York
from London having an idea that he would execute in commerce, through a
corporation. What are the corporation and tax laws in New York and the
neighboring states; in short, to what conditions must he conform? As
things~stand~he would have to pay a considerable fee to some lawyer of
known trustworthiness. With Fords organized he would get the desired
information for a comparatively nominal fee. Fords may have its
own~``tickers''~in Wall and Lombard streets. Facts having to be
bulletined instantly may go out from Fords instead of from the News
Association. Time is required for the printed page.~Thus~the ticker of
Fords is the extended, multiplied bulletin board of the News
Association. Fords becomes an institution having large revenues with no
corresponding outgo.~

An annual subscription to Fords of not less than \$100 will be asked.
Great care will be taken in making the preliminary canvass for these
subscriptions in certain large cities; this especially in New
York,~London~and Chicago. Take the twenty-five leading business men of
Chicago---they need to have the scheme explained in its entirety. This
done, in the leading cities a chain of references will be raised up.
There are 8,000 bankers in the United States, all of whom are possible
subscribers to Fords. A few years ago the possible home subscription
list of the Bradstreet Company was held to be~20,000---that number of
trading concerns in the United States stood in need of the Bradstreet or
like reports. The number is nearer 25,000 now. The Bradstreet Company
had in 1886 about 12,000 actual subscribers.~

The reporting of credits is carried on through three instrumentalities:
(1) the Bradstreet,~Dun~and other agencies; (2) the agencies under a
proprietorship having to do with but one branch of trade, such as the
Furniture Exchange and the like; (3) strictly co-operative reporting
done by traders themselves through such organizations as the Merchants'
and Manufacturers' Exchange of Detroit. The last two merge into each
other more or less. It is safe to say that the total annual revenues of
the three groupings is \$6,000,000. From close knowledge it is conceived
that half of this is over-taxation, the service being worth only about
half what is paid for it. Going to the merchants of the country, it is
proposed to ask that \$3,000,000 of the sum be paid into Fords, for the
reporting of trade conditions, the standing of corporations, and for
placing all related information directly at the service of the
merchant.~

It should be noted that the ``agencies'' now make a show of reporting
corporations; they do not get to the realities. My experience on the
inside proved that they cannot pass from the minutiae of gossip and
rumor in which they are involved. The argument from public need has
passed \emph{Bradstreet's} and \emph{Dun's}; instead, the urgent thing
is to move half their revenues forward to the Fact. There are three
collections of the one thing.~One lawyer in this or that locality
reports for both Bradstreet and Dun, for the one-branch-of-trade
concerns and for a co-operative movement.~

Sympathy for Fords will be found at London where investments are widely
distributed and the need of intelligence great. A New York merchant,~Mr.
H. K. Thurber, remarked to me that the idea of Fords is a distinctively
commercial thing; that it will be welcomed by the bankers of Frankfort
and other points on~the continent. For reporting financial questions of
the complex sort special insight is~required---the whole social movement
has to be taken into account. Conditions have ripened for the
announcement of Fords.~

\hypertarget{vi-the-news-districts}{%
\section{VI.~THE NEWS DISTRICTS~}\label{vi-the-news-districts}}

The great extent of the United States, the bigness of the country, has
compelled the elimination of distance. But this was only to prepare the
way for the organization of its intelligence and the correlation
therewith of the intelligence of the~whole world. The new ideas in hand,
the next great step is to divide the country into news districts and
begin the application of these ideas to the life of the people. The way
the country is set off, politically, into states and counties goes to
facilitate this. It would almost seem that the present necessity was
anticipated in the principle of political division. There is the
grouping by states with the county as the unit of organization. The
usual news district is a single state. Through these convenient
groupings the facts of life are brought to aggregates for study and
comparison. The facts under this or that head in Michigan are compared
with the like return from Oregon. In this way the interest is always at
the full.~

Democracy in America is not organized till we have consciously brought
its intelligence to a center, and have related it to the past, that the
resulting light may be had for the morrow's guidance. The means of
communication are in place but these could not be brought to the highest
use until the realities flowing out from the locomotive and the
telegraph, their spiritual meaning, should be wrought out and made the
basis of a centralizing movement. The principle of action for which the
News Association stands is the new locomotive. By the new light we~are
able to~see that the registration of life through the present newspaper
is quantitative only. We have now to erect the~qualitative center. In
furtherance of this the United States is divided into districts, each
district~being in charge~of~a representative of the News Association. In
each locality this agent stands for the general interest. Drawing his
pay from the central office, he is freed from the control of local
prejudice. The local fact is everywhere dealt with in the light of the
whole, thus compelling the highest sensations. ~

The new principle or method is applied to each news district. Nothing is
full commodity until brought to its full use; that is, until delivered
where it is needed and as it is needed. The merchant has not taken
possession of the publishing business until this is done for
intelligence. News has been treated as if no principle were required to
get and distribute it. The discovery of the intelligence triangle is the
key to treating intelligence as full, instead of quasi, commodity.
Through the triple incorporation the machinery is set up for refining~to
its uttermost the crude material of the news business---the bare event
occurring here or there. The Standard Oil Company made an advance in the
oil business by compelling from the petroleum all its secrets. By
penetrating to the white oil at the center,~naptha~was secured at one
end of the process and paraffine at the other. The economy here lay in
attending to the commodity principle. The intelligence triangle applied
to the soil, to the physical fact, there results an analogous refining
of news, and corresponding economies.~

The self-registering machinery is already getting into place in each
district but it is only nominal. It~has to~be quickened, to be made
actual. This done the Republic of Letters is a fact resting on the
earth. The physical machinery is there but it is not utilized. Money is
not wanted to create a vast new machinery, but only to attach the new
principle to the machinery already in place~so as to~bring out its full
effectiveness.~

The districting of the country, therefore, involves placing a mind at
the headquarters of each news field, a mind responsible for that
district. The principle of co-ordination, of integration, lies at the
base. There is at present no certainty as to the pursuit of any fact to
its ultimate meaning. Were it desired to institute inquiry as to some
continuing fact, say at San Francisco, a fact having roots in the past
and great future meaning, it would be found that there~is no agency for
prosecuting the inquiry and reporting its~results to the
whole~country.~Inquiry is not carried forward from day to day; no fact,
indeed, is carried either backwards or forwards. In short, the country
is not reported. The districting of the country is the solution. From
this vantage ground the News Association~is able to~handle intelligence.
It is perceived that we have only to act upon the principle in order to
get not simply more news in quantity, but new news in quality. The
higher quality comes in through the coordination involved in constantly
working to and from a responsible center. The self-registering machinery
of society, the desire of the individual to give up his local fact in
order to receive it back with accrued interest is utilized to the full.
The use of this machinery must have been but partial and tentative until
there should come in an institution standing for the whole.~

The district of Michigan will be organized first. The work of
organization entered~upon,~a single county will be dealt with as a
complete whole. On this whole, using it as a medium, the new order of
ideas will be realized. Monroe county, Michigan, will be used as this
medium. It is the southeastern county of the state.~



\hypertarget{vii-the-first-hand-distribution}{%
\section{VII.
THE~FIRST~HAND~DISTRIBUTION}\label{vii-the-first-hand-distribution}}

Horace Greeley\footnote{{[}Horace Greeley (1811--1872) was a journalist,
  editor, and political figure. Best known for establishing
  the~\emph{New York Tribune}~in 1841 (the leading newspaper of the
  penny press era), he also advocated for the use of the telegraph by
  the press. Ford also refers to Greeley in the ``The New York
  Press---Its Future.''{]}} is credited with the prediction that the
time is coming when all general newspaper matter will proceed from a
single institution. Greeley had the vision, but did not see, or, at
least, so far as I have been able to learn, did not announce that this
single distributing concern must itself be the central publishing
business. There cannot be two centers, one for gathering and
distributing news, the other for printing it. Economy demands but one
center. We have in hand, through the News Association, the fulfillment
of Greeley's prediction. To gain the end, the highest responsibility~has
to~be assured. This responsibility is neither more nor less than the
integrity of the merchant---the continuous attention to the commodity.
On this basis, the News Association proposes a close organization of the
first-hand newspapers of the country.~

The present Associated Press grew out of the locomotive and the
telegraph. When the New York newspapers came to employ these improved
facilities, they found themselves reporting simultaneously the same
facts and paying extravagantly for each report. The demand of economy
led them to divide the cost of gathering and distributing the news among
the several sheets using it. In other words, the function of the
Associated Press was to apportion the cost of the transmission. But this
function has disappeared. The great reduction in the expense of
transmission has rendered the telegraph toll to the newspaper what the
postage stamp is to the individual correspondent---a bagatelle. As there
is no longer any bar to communication, a new principle of association is
demanded. The~new principle is division of cost of inquiry. This advance
brings new freedom and new responsibility. The press takes on a new
function, that of separating the true from the false. Hence a re-forming
and the forward movement.~

The point to be made clear is that this change could not come about
until telegraph tolls were so low that all matter for the first-hand
paper, excepting of course the local news which is gathered on the spot,
can be sent out by telegraph. The editorial page simply marks the
absence of the entire fact. The whole not being at hand, opinion does
substitute duty. Understand, there is not news and ``editorial matter,''
there is only news. Already the stream of fact is so great that the
"editorial page'' is in neglect. It is in eclipse owing to the
tremendous merchandising of fact, or the semblance of fact, which has
come in. One more step and the editorial page is thrown out. Were a half
dozen of our leading daily papers to be made up tomorrow without the
editorial, or ``composition,'' department; it would hardly be missed by
the public. And it is safe to say that in the papers now making the most
of ``editorializing,'' the left-out page would be missed the least. But
the editorial as such can only be gotten rid of by substitution---only
when the flow of news from the center is all-embracing in its character
will the absurdity of trying to lend dignity to the paper through essay
writing become apparent. Under present conditions the editorial page is
a sort of ``church''~maintained for the spurious man of letters,~i. e.,
for the~writers, as against the inquiry men, the reporters: The
``editorial'' is sheltered behind certain notions as~touts~ethical
value, but these must give way upon recognition of the commercial value
of truth. Already things have got so far, in some towns, that one and
the same ownership puts out two or three newspapers with editorial pages
of opposite views. Certain of the ``syndicates'' go a step further,
offering to furnish all shades of opinion for all sorts of papers. In
all this we detect the spirit of the sutler who looks not to the quality
of his wares.~We are in a period of the greatest confusion in
news-dealing. The reforming waits upon the incoming of the merchant, who
will handle news as he handles iron or silk. The only composition
department will then be the type-setting rooms. Instead of the many
``editorial'' rooms, we shall have the one central inquiry office. One
library will answer the needs of all.~

The work of the ``syndicates'' has helped in the forward movement. They
have undertaken a task of distribution, but, unable to get first-class
matter, they have sent out inferior goods of well-known writers along
with the falsely sensational. But the result of their efforts, since
they went to stimulate the habit of receiving from a common center and
so to prepare the way for the reality, has been to intensify the demand
for the fact. Certain of the first-hand papers have taken little~of the
mass of matter offered. While recognizing the moving principle as sound,
they have preferred to wait for the real thing. The~syndicates have
failed to satisfy the needs of these papers because they have had no
means of controlling the commodity the newspapers were anxious to buy.
The syndicate, as such, is a distributive machine without anything to
distribute. What is needed is an organization of the in-come as well as
of the out-go. The two must be made parts of the one centralizing
movement. As it is, the syndicates have simply taken advantage of the
mechanical side of the centralizing movement---increased facility in
distributing matter. But the centralizing principle must itself move out
as respects news gathering: inquiry must be organized. Only in this way
can a continuous or genuine distribution from a center be~kept up. The
centralizing movement is seen to be inevitable; but it had to wait until
the new thought, resulting from the very possibility of the movement
itself, could be worked out.

\begin{figure}
   \includegraphics[width=\linewidth]{graphics/image-eleven.png}
   \label{fig:fig11}
\end{figure}

The country is surfeited with opinion and correspondingly eager for the
fact. Public criticism of the newspaper has become more open and
general.~As a consequence, the editors of certain journals at the
outlying centers are finding fault with the service of the Associated
Press. Here and there are evidences of tentative organization. In many
cities the leading papers have consolidated or are planning so to do.
Again, at~one or two points~single papers far in the lead of their
competitors, reveal the fact that but one newspaper establishment is
needed for the service of the community. At the first-hand centers the
necessity of an advance has already entered the minds of clear-headed
managers.~Since there is no occasion for two deliveries of the one fact,
certain of these papers are marked for survival, certain others are
marked for destruction or absorption.~

There are about seventy present or prospective first-hand
cities---cities, that~is,~which are first-hand in news.~The following
is~a provisional list~of such~centers in the United States~and
Canada:~~~

\enlargethispage{\baselineskip}

To the cities enumerated in this list there must be a common delivery of
news. One result of this impartial distribution will be a leveling up of
all first-hand papers. The newspapers in the smaller cities will have an
equally prompt and complete news-service with those of the larger. Under
present circumstances the smaller daily gets much of its information at
second-hand from the columns of its larger and wealthier neighbor.~

Besides these first-hand papers there are two other general classes of
less important journals: (1) The regional papers, representing two or
more counties, and (2) the strictly local papers. It is proposed that
the News Association do not sell directly to these papers, but supply
them indirectly through the first-hand men, who, in the capacity of
jobbers, will~retail~the news to meet the demands of small dealers. In
other words, the News Association undertakes to create only the arterial
system; the lesser circulation will be taken care of by itself. The
retail market of the News Association is the metropolitan district of
New York and vicinity. It is not proposed that the News Association
shall in any case own a controlling interest in outside papers,~i. e.,
beyond New York, whether at Chicago or elsewhere. The rule will be that
it is not to have any ownership whatever in such papers. To maintain
such ownership and~especially~to undertake to control the policy of any
of them would be to violate the home-rule principle. The local
newspapers, which will always be under the direction of the local
interest, or prejudice, simply receive general news from the central
office. The connection is maintained only so long as the service is
satisfactory. The News Association has no hold on them other than the
trustworthy dealer has on his customers. Only so far as life is
registered through the News Association can its empire be maintained.
The news of local origin in the office of each first-hand paper will be
at the service of the News Association much as now. It will, however, as
indicated elsewhere (No. 6), have translation through the district
representative who stands for the whole. The local news is thus worked
over at all points in the light of the general interest. The news of
the~metropolis will be so organized as very largely to preclude the
necessity of keeping special men at New York by the outside papers as
now. Special inquiries at the metropolis prompted by local needs,
{[}\emph{sic}{]} will be given attention by men in the service of the
News Association. The district offices of the Association can be drawn
upon in like manner. It is believed that in time Washington will be
found to be the only remote point at which the first-hand newspaper will
need a special representative.~

Since the News Association serves but one concern at each first-hand
center, its operation will ultimately involve a leading question of
social organization, namely, the nature of monopoly. As I have said, the
only hold of the central office on the outside papers is the perfection
of its service; their only hold on the center lies in the fact that they
pay their bills promptly, while giving to the public all the news. The
principle of the big grist and small toll must obtain in all directions.
But the following case is likely to arise:~Some one, wanting to ``start
a paper'' to further a given set of opinions, may ask the courts to
compel the News Association to give service. The latter will make answer
that the demand is not in the public interest; that instead the end is
to bolster up merely individual opinion; that genuine opinion of every
sort, that is, the~particular opinion~which measured by the whole is a
fact, is already delivered to the people through interviews and in all
manner of ways; or, if such delivery is not complete, action will lie to
compel it. A suit of this kind would compel attention to the reality of
social organization. The appeal would be to the organic principle. In
the late bucket-shop decisions of the Illinois courts, it was held that
the quotations of the Chicago Board of Trade if sold to one must be sold
to all; but against this point, were it urged, the News Association
would undertake to show that its news was already sold to all through
its local representative. It is of course intended that the news service
shall become so full and free, so rid of bias, that the case imagined
can hardly arise. But this hypothetical case illustrates the power which
the centralizing movement will possess of compelling new concepts in
jurisprudence.~

The present position of the daily newspaper is made clearer by a glance
at the history of our popular magazines. The contents of the first
\emph{Harper's} (June, 1850)~was~made up, with the exception of the
monthly review and other minor things, of reprints from English books
and periodicals.~\emph{Harper's} was simply a monthly scrap-book. In
time, the magazine began to pay for original contributions, then to
write up~particular features~of American life, and finally to organize a
staff. The secret of the advance is that it began to prosecute original
inquiry and this in a more organized way. This is equally true of the
\emph{Century} and some other like publications. Their present great
circulation and revenues have been brought about by the application of
the principle of inquiry. The point is that the centralized inquiry and
distribution now outlined, making the daily newspaper its organ, brings
the quality of its matter to the level of the present magazine and
beyond. Even now the character of the matter of the monthly magazine is
beginning to be affected unfavorably by the advances of the daily
newspaper. As the newspaper comes to have at command agencies of inquiry
turned full upon daily life, the magazine must recede, occupying itself
more and more with the past.~

\hypertarget{viii-the-new-publishing-business}{%
\section{VIII.~THE NEW PUBLISHING
BUSINESS~}\label{viii-the-new-publishing-business}}

The social body is still under the direction of pre-locomotive ideas.
This has nowhere been better expressed than by Whitman\footnote{{[}Walt
  Whitman (1819--1892) was an American poet, journalist, and editor.{]}}
in these words:

\begin{quote}
For feudalism, caste, the~ecclesiastical~traditions, though palpably
retreating from political institutions,~still hold essentially, by their
spirit, even in this country, entire possession of the more important
fields, indeed the very subsoil of education~and of social standards and
literature.
\end{quote}

\noindent The clear departure had to wait on the American~idea---the third fact of
the century. The first fact of the century was the locomotive, the
second the electric wire. The third is the spiritual outcome of these
new physical agencies, or the resulting conception of life. Through the
elimination of distance and its social rendering, we have a flood of new
ideas, making the new American literature for which the world~has been
waiting. The new thinking, compelling the new publishing business,
corresponds to the present state of commerce. The commerce in physical
things carried to the full finds its own realization in the truth-shop,
thus disclosing the unified commerce. Social advances find their first
expression on the physical side. Realized in consciousness, we have new
ideas of political organization and in consequence a new literature.~

In the unity of~commerce~the supposed barriers between the spiritual and
the material are overthrown. We apprehend continuous movement in life as
the new psychical fact. The process is continuous and can only be stated
in terms of absolute movement. The locomotive of commerce, itself the
freed truth of the universe, in turn frees mind. Distance has been
eliminated through the march of mind. The mind of man translates into
the secret of the steam. The unity of life is disclosed to us as an
external, every day {[}\emph{sic}{]} fact; it is a moving whole. Here is
overcome the discrepancy in life, the seeming divergence between subject
and object, lamented by Mazzini as ``the perennial anarchy between
thought and action.''\footnote{{[}Giuseppe Mazzini (1805--1872) was a
  journalist and political figure who advocated for the unification of
  Italy. Mazzini's take on the relationship between thought and action
  seems to have taken place in this political context. The quote is from
  Mazzini's \emph{Italy in 1848}.{]}

  \hfill\break
} The literatures based on the separation of the God principle from life
are seen to be but dead matter. There can only remain to us the
literature of action. The pseudo mental sciences, the rubbish of an
apart ethics and the great mass of economic speculation disappear as did
the Ptolemaic astronomy before the discoveries of Copernicus.~~

We are at the center of a new birth in letters---the advance of inquiry
to the daily fact, to the social whole in movement. In place of the
merely individual literature now in its decadence, we secure new
readings from the book of life. The new literature is the report of
America---of what she has done. The movement begins where Carlyle and
Emerson left off. Writing to Carlyle in 1844 (the year before the
President's message was first flashed from Washington to Baltimore)
Emerson said: ``My faith in the Writers, as an organic class, increases
daily, and in the possibility to a faithful man of arriving at
statements for which he shall not feel responsible, but which shall be
parallel with nature.''\footnote{{[}Ralph Waldo Emerson to Thomas
  Carlyle, February 29, 1844, in \emph{The Correspondence of Thomas
  Carlyle and Ralph Waldo Emerson, 1834--1872, vol. 2} (Boston and New
  York: Houghton Mifflin, 1884), \emph{} 58. For a detailed discussion
  of Emerson's thought and its influence on Ford, see the introductory
  chapter of this book.{]}} This means the registration of life. Other
men with equal clearness of vision have dwelt upon the theme. Speaking
from his place in the Senate in 1846,~John C. Calhoun~said: ``Magic
wires are stretching themselves in all directions over the earth, and
when their mystic meshes shall have been united, and perfected, our
globe itself will become endowed with sensitiveness, so that whatever
touches on any one point will be instantly felt on every
other.''\footnote{{[}John C. Calhoun (1782--1850) was an American
  statesman who served as Vice President of the United States from 1825
  to 1832 under John Quincy Adams and Andrew Jackson.{]}} The prediction
has been made good; so much so, indeed, that the newspaper, in Calhoun's
own phrasing, becomes~the organ of the whole. All these hopes and
prophecies are brought to their fulfilment in the News Association.~

\enlargethispage{\baselineskip}

The new reading of life takes definite form in certain volumes. Twelve
of these, relating to the fundamental region of inquiry, are under way.
The method followed is not that of writing per se, but that of
registration---being identical with the principle of crop reporting. The
work of each is for all and all for each. The ``books'' are on news
lines as already laid down. The advance of inquiry to the moving social
fact not only creates new news, but it reveals new tools of science.
Discovering the social system, its working lines become methods of
interpretation and criticism. The key here is that certain beliefs long
held in the mind in a half-formed way now become veritable tools of
exploration. Certain conceptions held in an apart way, and discussed in
terms of themselves, are transformed, when intelligence appears as
commodity, into practical instruments for handling and perfecting the
commodity into ways of investigating and reporting life. To illustrate:
When inquiry is directed upon the movement of life, the need is felt of
a psychology which can be used as a workman uses a kit of tools. We get
the hand-book of psychology as we now have the hand-book of the steam
engine. Psychology becomes a key to reading men's relationships to one
another in society. Much the same is to be said of logic and ethics.~

Through this reduction of mental science to practical form (as supplying
tools for reading life), vast acres of literature, now under the
domination of mere opinion, are invaded, and annexed to the domain
of~scientific inquiry. ``The~superfluous energy of mankind,'' says
Bagehot, ``has worked into big systems~what should have been left as
little suggestions.''\footnote{{[}Walter Bagehot (1826--1877) was a
  British journalist who served seventeen years as the editor
  of~\emph{The Economist} and was considered ``a kind of patron saint of
  business journalism.'' See Wayne Parsons, \emph{The Power of the
  Financial Press} (New Brunswick: Rutgers University Press, 1989), 41.
  Bagehot's quote is from~\emph{Physics and Politics~}(London: Henry
  King \& Co., 1872).{]}} There is no more reason for confusion in
political science than there is for two multiplication tables---it is a
question of the advance of inquiry, of a further invasion of art. The
unity of inquiry being attained as fact, there is a union of science and
literature---this to an extent not possible to anticipate.~The advance
to the solar system, through publishing of its law by Kepler and Newton,
swept away the mass of opinion in that region. The advance to the social
system, through discovery of its law, sweeps away the bulky literature
of opinion in the regions of social life. The many dialects of opinion
cannot compete with the one language of action. The scientific and the
commercial advance are one. Literature turned out by partial and
defective machinery is displaced, as surely as any quasi commodity
recedes upon the appearance of more free working and better organized
methods of production.~

Beyond the substitution of hand-books of practice for apart theorizing
and argumentation, we get certain generic and positive readings of life.
Given intelligence as commodity, the central principle of life is
detected. With truth and commerce at one, the organizing and controlling
principle of society is revealed. The present status of society needs to
be reported in the light of this discovery---in its general features, as
well as in specific and daily details. A single volume focuses this.
This first volume is called the \emph{Day of Judgment}. The settlement
day in the world's affairs comes in with the organized movement of the
whole intelligence or fact. It is a statement of the existing structure
of society. Again, the light reflects backward. Seeing, at their
fullness, science and commerce as one, we look back upon history and
behold their converging lines---the approach of the ``material ''~and
the `` spiritual '' to a~centre~of action. Hence the story of the
Quickening Spirit and again that of the Conquerors of Distance. The
three next on the list are the tool-books already referred to; the six
following are more specific applications of the new tools to important
regions of life. The volumes grow up purely in response to demand---the
undertaking is commercial---but enough is already done to make the way
clear to the end. The twelve books~are as follows:~

\begin{figure}
    \setlength{\parindent}{142pt}\includegraphics[width=\linewidth]{graphics/image-twelve.png}
   \label{fig:fig12}
\end{figure}

The scope of these volumes may be thus indicated:~

\vspace{.15in}

1. \textsc{The Day of Judgment}. The notion of a day of judgment is as old as
recorded thought. With the appearance of intelligence as commodity, this
notion at last rationalizes and we detect the settling day in the
world's affairs. The organized publicity compels continuous accounting
in life---and this both ways, as to the individual and the whole.
Through the free play of fact, equal judgment is~assured. The individual
comes to judge and to be judged in the full light. Full freedom gained,
the partial fact reports to the whole for correction, and the whole, in
turn, makes up the new accounting. Adverse interests make~their showing
on the full fact. This is the judgment in life. The book indicating the
lines of this accounting provides the Organon of Democracy.~

The divisions of the \emph{Day of Judgment} book are as follows:~

\vspace{.1in}

\begin{hangparas}{.25in}{1} 

(a) The Representative Slave. The first division tells the story of a
mind seeking to realize its own movement through a desire to inquire and
report concerning the common fact. Its activity is checked by
class~interest which found its profit in keeping the common fact covered
up. The Representative Slave records the steps by which the inquiry man
became conscious of his slavery---of his hindered mental movement. It
records also the varied contact by which he discovered the conditions
making for his freedom and for that of his kind. The story recites an
individual experience in seeking to report, taking the municipality as
the special subject of inquiry, the social movement with equal reference
to the individual and the common good. Checked in this, he found the
real hindrance concealed in the fact that existing social organization
is based on a supposed antagonism between the individual and the whole.
This being the dominant idea, the newspaper could not ignore class
interest and so allow him to deliver the truth to the people. The
discovery lay in perceiving that, as regards underlying conditions, the
social movement had got to the point of harmony; that the newspaper
therefore was~lagging behind, was off the track. Here was the evolution
of commerce up to the point of breaking down, through the locomotive and
telegraph, the barriers to full inquiry and reporting, making it
possible to deal in truth as commodity. The development of interest to
the point of selling truth is the~power that frees the slave.~



(b) The Organic Letters. Inquiry~freed,~life registers itself at a
center. The results (news, intelligence) are distributed according to
their respective relationships or demands.~Thus~the business of letters
is organized; the man of letters functions. Said Thomas Carlyle:
``Men~of Letters~will not always wander like unrecognized, unregulated
Ishmaelites among us! Whatsoever thing has virtual unnoticed power will
cast off its wrappages, bandages and step forth one day with palpably
articulated, universally visible power.''\footnote{{[}Thomas Carlyle,
  \emph{Heroes and Hero Worship} (New York: John B. Alden, 1883), 119.
  Ford also discussed the work of Carlyle in his correspondence with
  Oliver Wendell Holmes, Jr. See \emph{News is the Master Element of
  Social Control}.{]}} This division records the fulfillment of
Carlyle's prophecy in the emergence of the man of letters as the diurnal
man or reporter, gaining a definite position through getting his own
assured~commodity.~Thus~the man of letters is for the first time
introduced to the world. Account is taken of the conflict through which
he has come to his independence, to his rightful estate, not forgetting
the trail of blood; it is the struggle towards organization. We
distinguish the art of arts, that of conveying intelligence, using the
word as tool. A flood of light results as regards the relative position
of the various arts. The definitions, principles and method of the
Organic Letters are thus introduced.~

(c) The Grand Division~of Labor and the Social~Organism.~Political
writers~of the last century have had much to say of the spiritual and
temporal powers and their relations one to the other. The organization
of intelligence marks the grand division of labor here. The spiritual
power is determined. The cleavage is between the man who acts by inquiry
and reporting and the man who does some external thing. As a result of
the freeing of intelligence and the flowing of the related facts of life
to a center, activities corresponding to the facts organize also. The
social system, the~differentiations~and interactions of the body
politic, is at last attained as fact. The equation making the organism
results from the free play of the individual interest, which is
constantly recast on the lines of functional, common, or related
interests. The social organism is disclosed; we perceive the full
circulation of intelligence through~the social body. Determining the
main functions and directions of the~organism, we have tools for placing
the daily and hourly event. The struggle of the century for a science of
politics is thus realized. The effect on existing economic concepts is
traced out. The discovery that commerce evolves its own control gives
the principle which, followed out in detail, reconstructs to its
entirety what now passes for economics and social science. ~

(d) The Bourse. This fourth division~inquires~into the constitution of
the bourse; into the notions underlying and sustaining the present
stock~exchange. Following upon the locomotive, the mark of the century
on the side of finance is the security factory. The century cannot
end~without bringing to book the excess security making. In England and
the United~States,~the only check to the marketing of railway securities
has been the market limit. A great stock and bond structure has
resulted, whose integrity is~subject to the social interaction. How are
the social forces moving? In getting at the reality here the idea of
function becomes a tool of inquiry. The social organism~attained,~the
blood drawn from the whole by each member is measured by the need of
each. It may be that the persistence of private taxation in the guise of
public function will be detected. Account is taken of the new industrial
incorporations and their effect on the old stock and bond structure. The
story is of the conflict between the old and the new production. In
general, account is taken of the great new fact that the socialism of
intelligence marks the last of the feudal concept.~~

(e) The Visible Church. This fifth and final division of the \emph{Day
of Judgment} book reflects on the one hand the results of an exhaustive
inquiry into the state of endowments. The auditing principle is to be
let in. It may appear that a close connection exists between the
magnitude of endowments and certain false concepts that have gone to
sustain private taxation through the bourse. Tracing out the connection,
the aim will be to learn how far the present organization is, in its
reality, a state church, and in turn how the full freedom that is to
come in will go to rekindle religious zeal. With the truth man
substituting the counter for the contribution box, there results a
marked alteration in this region. On the other hand, the division deals
with the fulfillment of the promise: ``Ye shall know the truth and the
truth will make you free.''\footnote{{[}Jn 8:32.{]}} The fact is
disclosed that the truth had first to be freed, and that this is man's
part. It is further seen that through all the turmoil the principle of
salvation has lain concealed in the activity, in the life-process of
society. The church spiritual is at last organized; and through the
distribution of truth the individual is raised into full membership. The
church in the locality sense becomes the rendezvous in life.~

\end{hangparas}

\vspace{0.05in}

2. \textsc{The Quickening Spirit}. Having attained the unity of inquiry, that is,
the development of science up to dealing with the whole, the State, it
becomes possible to write the history of the struggle through which man
has come to consciousness. This volume reports the growth of man's
thought towards its objective, full action---until, freed in commerce,
it goes clear over. The separation between thought and
action,~theory~and practice, is overcome. The spirit quickens. It is the
growth in consciousness of the spiritual power. The principle makes
possible a reconstruction of intellectual history, giving a unity to
what is now treated dispersedly, in histories of philosophy, of
literature, of art, etc., the uniting thought being the movement of
intelligence to action. The volume pays especial attention to the
spiritual history of the nineteenth century, setting forth as its secret
the last desperate struggle of thought to attain completeness and thus
reach action. The story is given of the first enthusiastic fore-glance
as distance began to disappear; of the isolation of thought in Germany
and of its violent outbreak in the streets of Paris at the close of the
last century; of the stock men took of the world they were then living
in; of the attempts to enter the promised land by some other than the
straight and narrow way; and of the dashed hopes as men finally ran up
against the~\emph{cul-de-sac}. Then comes the final suggestion of the
physical machinery which was to break down the last barrier so that the
highest thinking might go out into life. It is thus that thought takes
possession of its own, getting full citizenship in the kingdom of ends.~

\vspace{0.05in}

3. \textsc{The Conquerors of Distance}. Commerce when it completes itself in
merchandizing truth is revealed as the integrating process of life---as
the force which conquers separation and brings men together in action.
This third volume tells the story of commerce through delineating
typical cases of its struggle to realize unity in life. It thus develops
the principles of the unified economics. It is the paean of commerce.
The history of the movable type, of the printing press, of the
locomotive, the telegraph and the telephone~is~told in the struggles of
the men who brought them to birth. The~twice told~tale finds new and
powerful interest in its fulfillment. ~

\vspace{0.05in}

4. \textsc{The Working Psychology}.\footnote{{[}This publication plan has been
  partially completed by Ford's youngest brother the following year. See
  Corydon Ford, \emph{The Synthesis of Mind: The Method of a Working
  Psychology} (Ann Arbor: J. V. Sheehan, 1893).{]}} The problem has been
to grasp the varied psychic~manifestations~as a whole, in~order that
they might have related meaning. This is realized by setting up the
mental characteristics as phases of a single movement in lieu of
regarding them as apart and of themselves entities. In this, mind is
unified and so given common rendering for all men. This common rendering
of mind raised~up,~psychology becomes an instrument for the
interpretation of life. It makes definite whole regions now indefinite
and restores waste places in sciences which heretofore have found
difficulty in dealing exactly with the interaction of mind. Workers in
politics, the daily walk of business, in medicine, law, and the school,
are to find the missing tool. In one of his latest writings, Professor
William James, of Harvard University, says: ``We live surrounded by an
enormous body of persons who are most definitely interested in the
control of states of mind, and incessantly craving for a sort of
psychological~science which will teach them how to act.''\footnote{{[}William
  James, ``A Plea for Psychology as a `Natural Science,'\,'' \emph{The
  Philosophical Review} 1, no. 2 (March 1892): 148.{]}} The Working
Psychology undertakes to meet this demand.

5. \textsc{The Tools of Inquiry}. The generic ideas of life, which have been
worked~out and raised up by the philosophers as notions, appear to be,
as things stand, the especial property of philosophy and philosophers.
With inquiry freed, these notions become tools for the direct handling
of social fact. The need is for a critical account of the categories of
thought and this without~entering into~an abstruse discussion as to
their origin. This fifth volume arranges the ideas of~the philosopher as
tools for inquiring into the state of the body politic---that is,~with
reference to reporting the special event in the light of the whole or
general~interest. In turn, the light of the present advance of
intelligence will be reflected~backward by way of re-interpreting the
notions. All the factors of knowledge have now to be viewed with a
heightened consciousness as to the unity of subject and object.~

\vspace{0.05in}

6. \textsc{The Ethic of Action}. This volume states the ideas resulting from the
fuller perception now~gained of the interaction of life---the relation
of the individual to his fellows. Ethical distinctions at present are
largely derived from the pre-locomotive age---from the age when man was
isolated in his conduct through lack of full communication. With the Age
of Commerce, men get together in their action; and we realize as fact
the forces making for integrity (for responsibility and freedom) in
life. Getting to the ethic of action, we transcend the ethic of precept.
We are content to see the false ethic disappear because, perceiving the
free play of intelligence, we recognize that each situation in life
carries its own ethic---and this with relentless logic.~

\vspace{0.05in}

7. \textsc{The Word as Tool}. This volume marks a departure from existing helps
in the use of words. With the reporting of life organized, the
intellectual life identifies with that of action. Literature is the
movement of ideas. The art of reporting is distinguished. The word is
the tool of the reporter. The beautiful tool is the best tool and the
reverse. That is, in literature beauty and use (conveyance) are at one.
Literary criticism, so-called, does not differ from any other kind of
criticism; it is to judge whether a given movement has attained its
end---whether the idea of the writer has reached the point of effecting
action or has remained abortive. Account is taken of the two sides of
action---subjective and objective. It will be asked whether a given
writing is merely individual opinion or whether it parallels life---the
test~being its compelling power. The point, in fine, goes to transfer
the literature of criticism from a separated (or apart) sphere to the
practical---the life of action.~

\vspace{0.05in}

8. \textsc{The Child of Democracy}.\footnote{{[}This publication plan has been
  partially completed by Ford's youngest brother two years later. See
  Corydon Ford, \emph{The Child of Democracy: Being the Adventures of
  the Embryo State} (Ann Arbor: J. V. Sheehan, 1894).{]}} The
development here identifies the school, from the back-woods district to
the university inclusive, as the Embryonic State. It proceeds from the
struggles of those who would teach when conventional methods only
provide for ``keeping school.'' The book reveals the existing chasm
between school and life as the source of the current dissatisfaction
with the general conditions, methods, and results of instruction. In
course it sets out the detail of disorder traceable to this primary
schism or broken unity of the life movement. In contrast with this, the
working relationship between school and state is defined, the unwitting
disregard of which has its practical outcome in the withering up of the
child through his lifeless environment. In its wider reach, the book
takes account of the family on the one side as the movement towards the
school, while on the other it has to do with
the~qualifications~requisite for citizenship---the continued movement of
the individual toward full functioning in the State. In fine, a basic
principle is discovered which frees child,~teacher~and State on the side
of education. This new freedom comes in because, through the development
of commerce, contact has been so freed that the child may be directly
connected with the moving intelligence instead of getting it second-hand
through books or the mind of his teacher. The school is the State in
process of formation---in embryo.~

\vspace{0.05in}

9. \textsc{Ford's Common School}. It is proposed in this volume to deliver the
new principles of education to the teacher as working tools, the aim
being their detailed application. The movement looks to connecting the
schools with life, and this up to the point of admitting the electric
wire, the writing machine and the printer's case.~

\vspace{0.05in}

10. \textsc{University Endowment and Organization}. This volume undertakes to set
out the integrated university. In point of internal arrangement this is
found to turn on the unity of inquiry. Existing university organization
is but the reflection of prevailing mental lesions, that is,
abstractions carried to the point of dividing the one life of action.
When the necessary divisions of inquiry are made clear, a model for all
time may be erected. The university connects itself with life; taken out
of their isolation, teachers and students are no longer content with
false values. The report also deals with the external side---the base of
money supply. Is the university of the future to rest on the endowment
principle, on taxation, or on both? Now, when we~are able to~read the
movement of the bourse in a more definite way, this whole matter
of~university foundations takes on a transcendent interest. When the
university is brought into full relations with life, it may turn out
that taxes for its needs, for the higher education, will no longer
appear to the commonalty like giving money for some far away,
disconnected thing.~

\vspace{0.05in}

11. \textsc{The Truth and the Law}. Under this heading we report the growth of
accuracy in society between citizens, as realized in municipal law. By
this is meant the gradual insistence upon responsibility among men for
statements made to each other, throughout the whole range of action.
There is the rise of the law of forgery and of libel. New jurisprudence
is involved. It deals with the relations of the individual with the
newspaper, with the movement of intelligence. Must not this relationship
pass into the region of extraordinary remedies in order that correction
may be prompt and punishment~more swift? The security for individual
privacy is indicated. But privacy is at an end whenever on the one hand,
through unusual service to the state, public applause is invited, or
whenever, through infraction of the social equation, public penalty is
incurred. Does not the movement of publicity on the commercial lines
work out its own law? This whole aspect of life can now be reported in
the light of ascertained principles. It goes back to the statute of
frauds and forward into the newer day.~

\vspace{0.05in}

12. \textsc{Symptoms of Health}. With the beginnings of the elimination of
distance, owing to cheapened postage, and the like, the quacks in
medicine began advertising in the newspapers, especially in the country
prints, certain books for sale through the mails. With extreme
dishonesty these books turn the ordinary facts of physiology into
pathology. The newspapers of to-day abound in such advertising. The
remedy for all this can only proceed from a commercial advance. The
quack in the expectation of selling, say, 50,000 copies by specious
advertising, can in a month write a book in which everything is tortured
into pathology.~On the other hand, the physiologist is not led to write
and advertise his book, wrought out on the truth lines at every point,
until through the universality of the mails he is in the way of selling,
say, from 500,000 to 1,000,000 copies. Such a book, under the title
indicated, it is proposed to put out. Various approaches of late have
been made to the book, but they lack incisiveness and point. They are
not artistic,~i. e., they do~not get to the truth. The method here is
one with perceiving that large profits would result from just the right
thing. This identifies the artist and the merchant. Again, it suggests
inquiry into the whole ``patent medicine'' business, while at the same
time it looks to discovering the true state of the practice of medicine.
The book is what its title indicates---the symptoms of health.~

\newpage\begin{center}

\vspace{.15in}
    
\large{MINOR VOLUMES.}

\vspace{.05in}

\small{(FOUR TITLES COPYRIGHTED)}

\end{center}


1. Monroe County.~Among minor volumes in prospect, leading
  interest~attaches to the report on Monroe County. The summing up here
  cannot fail to make a considerable volume, since the new order of
  ideas will be realized on this one locality.~Entering into~the life of
  the region, its movements will be studied in the light of the organic
  principle---the future will be read into the present. Such a
  cross-section of life must give results of the greatest interest.
  There is as much news in Switzerland as in Russia, the difference
  being simply a question of detail. The whole~enters into~a given
  locality or part, as well as the part into the life of the whole.
  Thus, the reporting of Monroe County goes to complete all the volumes
  herein set forth.~

The following list of categories, roughly classified, will serve to
indicate the nature and scope of the reporting to be done in Monroe
County.~

\vspace{.05in}

A. Physical Basis:~

1. Soil. 2. Drainage. 3. Climate. 4. Sanitary conditions. 5. Flora and
fauna. 6. Rivers and harbors.~

\vspace{.05in}

B. State of Commerce:~

1. Agriculture. 2. Fruit and vineyards. 3. Market gardening. 4. Cattle.
5. Dairying. 6. Fishing---lake~and rivers. 7. State of labor and the
labor market. 8. Manufacturing. 9. Roads. 10. Railroads. 11.
Banks,~savings~and investments. 12. Business habits in the villages and
among the farmers.~

\vspace{.05in}

C. Social Environment:~

1. Movement of population---immigration and emigration. 2. Language. 3.
Habits of life---amusements, morals, etc.~

\vspace{.05in}

D. Institutional Development:~

1. Government organization of county; work of the local courts, etc. 2.
State of family; marriage, births, divorce, etc. 3. Embryonic State:
schools, libraries, etc. 4. The visible church. 5. Status and movements
of political parties. 6. The existing state of intelligence---means of
communication; registration machinery, condition of newspapers, etc.~

\vspace{.05in}

E. Pathological Status:~

1. Pauperism, both physical and spiritual. 2. Insane and defective. 3.
Crime. 4. Strange psychical developments. 5. State of the practice of
medicine.~

\vspace{.05in}

It should be noted that items of information are not to be
collected\emph{~per se}. Monroe County is reported not statistically but
as representative. It is viewed at every point as reflecting the life of
the whole. The harbor at Monroe, for example, cannot be reported without
touching upon the federal appropriations for such lake harbors---thus
raising up the practical, or working, relation of the general government
to the locality.~So~the fishing industry of the county cannot be treated
without asking what the fish commission of the state is doing. The
bourse interest is prominent in the county,~owing to the fact that~two
lines of the Vanderbilt system cut across it. This will compel
examination of railroad accounting at New York. To report the schools
and churches of one county is to judge those of the country at large. To
study insanity in but a single county is to report the influences
working everywhere to disintegrate the mental movement in the
individual.~Thus~the resulting volume states, by type, the existing
organization of life in America.~

\vspace{.05in}

2. Political Parties in the United States. This report depicts the
migration of parties from one to the other side of the line of battle in
the progress to freedom through organization.~

\vspace{.05in}

3. The Notion of Copyright. With the incoming of intelligence as
commodity, the notion of copyright changes. Only the form in which ideas
are presented can be protected through legal devices. The content of
this or that new outgiving is always at the disposal of the journalist,
and through him the life-bearing will be distributed. The need is to get
at the whole meaning for this region of the great new fact that has come
in.~

\vspace{.05in}

4. Property in Trade Marks. One of the astonishing things of the day is
the great money value of certain~trade marks. If the advertisement is
made general, the popular~brand of varnish or of baking powder becomes a
veritable gold mine.~Through the protection of the State, prices may be
kept so high that enormous private taxation results. It is conceived
that the only adequate remedy here must be had through an enlarged
publicity. The report to be made will take account at once of the legal
side and the trade facts.~

\vspace{.05in}

Other minor volumes must result, but it is not necessary to anticipate
further. Small books having from fifty to one hundred pages, bound in
limp cloth, will be put out. The method is but another way of setting
news dealing over against~the existing paper-and-ink business. ~

\hypertarget{the-question-of-authorship}{%
\subsection{\texorpdfstring{\emph{THE QUESTION OF
AUTHORSHIP}}{THE QUESTION OF AUTHORSHIP}}\label{the-question-of-authorship}}

Regarding authorship, the place of designer or architect is now
distinct. The notion of authorship changes with the incoming of full
registration, and the decadence of merely individual literature. Who
first conceives a given bit of reporting may indeed have no
mention~therewith.~Personal credit in the matter of authorship is a
question of exigency and of circumstance. At times, reports will find
their way into book form which from the very nature~of the work cannot
be credited to individuals, any more than can a crop~report, or the
like. Again, other reports would have little or no meaning did not the
responsibility therefor depend upon an individual name. These volumes
comprehend vast detail, but it is all at command through the primary
divisions of labor already~established.~The force of the unifying
principle is such that co-operation is at~its highest. Men are competing
together for the common end. The organic principle of life and mind in
place of being an academic fad~becomes a unifying force in relation to
which the facts of politics, of society, fall into place.~~

\hypertarget{the-newspaper-interest}{%
\subsection{\texorpdfstring{\emph{THE NEWSPAPER
INTEREST}}{THE NEWSPAPER INTEREST}}\label{the-newspaper-interest}}

The work of organizing inquiry has already proceeded so far that
publication may begin. We make connection with the Distributive
University---the daily newspaper. Fifty years~ago~new literature
embodying the outcome of an extension of inquiry was sold to the public
in ``parts.''~But with inquiry organized as a whole, the results tread
so closely on the heels of life that books, as such, are out of date
before they leave the binder. The old bottles will not hold the new
wine. What is needed is an instrument of daily communication. This is to
be found only in the daily newspaper, the common carrier of ideas. The
movement of inquiry in its highest and deepest sense squares itself with
the work of the newspaper, simply because the business of the newspaper
has come to be inquiry.~

The movement out to the newspapers of the country consists in marketing
the many fragments into which at the outset the above schedule of work
divides. The range is co-extensive with life itself. In point of length,
the matter for the newspapers when ready for shipment, will vary from
a~stickfull~to a column, or perhaps at times to a full page.~

The unity of inquiry pressed upon~life,~news rises in all directions.
The new point of view acts as a revelation. Without attempting to
characterize, here are a few things of immediate newspaper interest:~

\vspace{.1in}

    


\noindent Herbert Spencer's Hunt for the Sensorium~\\
Local Extension of Telegraph Lines~\\
The State and the Professions\\
University Development in Ann Arbor~\\
The Rise of Press Clubs~\\
The Mercantile Agency Business~\\
The Newspapers Called Religious~\\
Changes in Proverbs~\\
The Pathos in the Progress of Inquiry~\\
Beginnings of the Newspaper~\\
Labor Papers and their Difficulties~\\
Christ's Idea and its Fulfillment~\\
The American Tract Society~\\
The New Slang~\\
Changes in Libel Law~\\
The~Scunner~Against Commerce~\\
The Endowment of Letters~\\
Law of the Newspaper with Respect to Privacy~\\
Competition and Law~\\
The Trust in Congress and in the Law Writers~\\
The Schoolbook Incubus~\\
Pathos of Faith without Sight in Thomas Carlyle~\\
The Trade Paper Business~\\
Bulwer's \emph{Last of the Barons}~\\
Prime Movers in Literature~\\
The Two Enthusiasms, or the Study and the Market-place~\\
The Economics of Emerson~\\
The Psychology of Emerson~\\
From James Morrison until Now; the Dry Goods Truth Shop of\\
\hspace*{.25in}1825 at
London~\\
The \emph{New York Times} and the Rise of Tweed~\\
The Trader and the Merchant ~\\
The Notion of Conservatism Changing~\\
The Pope's Outgivings since the Locomotive~\\
The Civil Service Question and the Journalist~\\
Art in Industry; How the Printer's Roller is Made~\\
Position of the Painter and Sculptor~\\
Josiah Strong and the Nation's Crisis~\\
John Morley in his ``Voltaire'' and the Place of the Man of Letters~\\
The Rogers Typograph and Class Interest~\\
The Two Armies:~the~Spirit and the Flesh~\\
The Individual in Walt Whitman's Writings~


\hypertarget{unity-of-the-publishing-business}{%
\subsection{\texorpdfstring{\emph{UNITY OF THE PUBLISHING
BUSINESS}}{UNITY OF THE PUBLISHING BUSINESS}}\label{unity-of-the-publishing-business}}

At the doors of the News Association, the distinction between journalism
and literature breaks down. There are no~books---there are only
newspapers. There are no newspapers---there are only books. The
prediction of the Frenchman, Lamartine, that the ultimate book was to be
the morning newspaper, comes true. Literature becomes the recorded
movement of ideas---of life. That is, the publishing business gets its
unity through detection of its proper commodity---news. The
commodity~discovered,~the business organizes. There are just as many
modes of publishing as there are demands for the intelligence at
hand---no more, no less. Under this conception each publication,
newspaper, leaflet, or book, is the size of the news. Nothing is put out
beyond leaflet -size, save as compelled by the volume of intelligence.
No padded books or papers which impress by their volume and so go to
make high prices, will be issued. Given intelligence as commodity, the
transfer is made from the~``book''~business to the goods business. High
prices are no longer necessary to support literary men and
``ideas''---as on the endowment principle. Instead, the News Association
is a dealer in news. News of enduring interest is at once re-issued as
leaflet or book.~Thus~a piece~of~news printed in the \emph{Newsbook} on
a given morning may appear as a leaflet on the afternoon of the same
day. Here and there the principle has already taken effect, but only
incidentally; with the News Association it is the rule of action. Here
is the ``literary revolution.'' This does not consist in reprinting old
books at new prices, but in selling new intelligence in such volume as
to compel a re-forming of publishing methods.~

One remarkable thing is the preparedness of existing literature for the
advance. This is manifest in various ways. As to matters of controversy,
recent writers have done much to bring into clear relief both the truth
and the error contained therein. A striking illustration here is the
reduction John Morley's writings have~effected~in the pre-revolutionary
literature of France. Further, as to matters of fact: the publications
of the last twenty years---in series and handbooks---have reduced the
past to convenient shape for use.~However~it may be for the technical
scholar, the past, for the ordinary reader, is now brought up for
reference and distribution. As for the present, the accumulation of fact
has come to be far beyond the use that is made of it, or that can be
made of it, save through an advance on the part of the newspaper.~

Take, for an example, recent census reports. For the people, they have
little or no direct value; they can be utilized only by the journalist.
This is true generally of the results of government inquiry, whether
state or national. The lessons to be found therein~must reach the public
through the diurnal man. Again, specialists in science have been
lamenting the obstacles to the distribution of scientific intelligence.
In 1884 John Eaton,~one time~United States commissioner of education,
addressed the American Association for the Advancement of Science on
this question of distribution. Among other things, he said:
``The~dissemination of truth is as scientific as its discovery. . .
.~Toward the~gathering up, for man's daily use, of all the lessons of
nature, the progress of the race is tending. . . .~The era of this
diffusion of knowledge has~already commenced. Men not engaged in
scientific pursuits are gradually coming to feel the necessity of
gathering, grouping, and generalizing the data which give them a clear
measure of health, comfort, pleasure, as well as the profit and loss
involved. . . . But~the correlation of all these and their~actual
results have not yet been reached. Nevertheless, money sees the profit
of this wisdom and is more willing to pay for it.''\footnote{{[}John
  Eaton, ``Scientific Methods and Scientific Knowledge in Common
  Affairs,'' \emph{Science} 4, no. 84 (September 1884): 247.{]}} The
view put forth by Mr. Eaton has also had voice at later meetings of the
American Association. What is perhaps the last word on this point was
long ago spoken by Agassiz:~``Scientific~truth must cease to be the
property of the few; it must be woven with the common life of the
world.''\footnote{{[}Jean Louis Rodolphe Agassiz (1807--1873) was a
  Swiss-born naturalist and polymath who had a brilliant career at
  Harvard University where he played a key role in the development of
  modern scientific education. For a detailed discussion of Agassiz's
  intellectual trajectory and its complex relationship with pragmatism,
  see Louis Menand, \emph{The Metaphysical Club} (New York: Farrar,
  Straus and Giroux, 2001).{]}} At the last meeting of the British
Association, a demand was made for a central institution which should
adequately check the results of scientific inquiry. But these facts must
be more than simply ``checked''; they must be interpreted and delivered
in their application to life. True distribution of them can take place
only through a commercial advance.~

\hypertarget{the-american-idea}{%
\subsection{\texorpdfstring{\emph{THE AMERICAN
IDEA}}{THE AMERICAN IDEA}}\label{the-american-idea}}

The new publishing business transcends the judgment of European critics
regarding America. A type of these may be found in a foot-note to the
essay on the Rationality of History, written by Mr. D. G. Ritchie, and
printed in a book of essays published by English university men about
ten years ago. Succeeding his attempt to give the various nations of the
earth their spiritual rating or position, Mr. Ritchie said:

\begin{quote}
It may be objected that no account is taken of one of the greatest
nations of the world---the United States of America. But to this we can
answer that it is~as yet~too new; in spite of its immense achievements
in the material elements of civilization, it has contributed little as
yet, except a few eccentric religions and some startling experiments in
literature, to the spiritual existence of mankind. It is performing a
gigantic political and social task; but the task is not nearly
completed. Its population is constantly increasing by immigration, and
its best culture is still an echo of the ``old~world.'' Yet, even apart
from the doctrine that ``westward the course of empire takes its way,''
the American can certainly feel that to him belongs the future. Whether
the Slavonic races of Eastern Europe have an equally great future before
them is more doubtful. In any case America and Russia are not old enough
to belong to philosophic history. All study of their development is too
much that of contemporaries.\footnote{{[}David George Ritchie, ``The
  Rationality of History,'' in \emph{Essays in Philosophical Criticism},
  ed. Andrew Seth and Richard Burton Haldane (London: Spottiswoode,
  1883), 148.{]}}~
\end{quote}

Another type comes to us from Germany. It was in 1878 at Cologne
that~Professor Du~Bois-Reymond, of the University of Berlin, gave his
well-known lecture on Civilization and Science. At the last he brought
up with the dangers which, as he thought, were threatening modern
civilization. Surveying the movement of the time he saw in it ``the
decay of intellectual production.'\,' Looking further he saw a growing
aversion to going ``down into the deep well of truth.''~``Even the
universal participation in the over-estimated benefits of~political life
diminishes the respect for ideas.'' He found that ``art and literature
prostitute themselves to the gross and variable taste of the multitude,
swayed hither and thither by the daily newspaper.'' To him, ``Idealism
is succumbing in the struggle with Realism, and the kingdom of material
interests~is coming.'' And all this our scientist identifies as the
process of~``Americanization.'' ``The dreaded overgrowth and permeation
by realism of European ``civilization'' is held to proceed from America,
``where~no historic memories and literary traditions were available for
stopping the tendency of the popular life, too exclusively directed
toward the useful arts and the acquisition of wealth.''\footnote{{[}Emil
  du Bois-Reymond, ``Civilization and Science, Part II,'' \emph{Popular
  Science Monthly} 13 (August 1878): 394--5.{]}} Finally, ``America has
become the principal home of utilitarianism. While at times the very
first conditions of human society are there in dispute, it is in America
especially that those existences come into being whose wealth, luxury,
and external polish, contrasting as they do with their ignorance,
narrowness, and innate coarseness, give one the idea of a
neo-barbarism.''\footnote{{[}Emil du Bois-Reymond, ``Civilization and
  Science, Part III,'' \emph{Popular Science Monthly} 13 (September
  1878): 530.{]}}~

Casting about for a remedy, Professor Du~Bois-Reymond finds it in a
reversion to the past, to the ideas of a time that was without natural
science. He declares that ``as humanism rescued man from the
prison-house of scholastic theology, so let it enter the lists once more
to battle against the new enemy of~harmonious culture.'' It is from this
influence~``that we can most confidently hope for victory in the
struggle with the neo-barbarism which, though~as yet~its hold upon us is
loose, is, from day to day, tightening its iron grasp. It is Hellenism
that must ward off from~our intellectual frontier the onset of
``Americanism.''\footnote{{[}du Bois-Reymond, ``Civilization and
  Science, Part III,'' 530.{]}} And in so many words Prussian gymnasium
education is set over against ``the progress of Americanization.''~The
great scientist distrusts science; he believes in it in his own field
but not beyond. He does not see the practical outcome of the century's
advance in physical science; that the last result of science is the
commercializing of truth. He finds the immediate source of modern
science in the arrival of monotheism, which inspired man ``with the
ardent longing for absolute knowledge,''\footnote{{[}Emil du
  Bois-Reymond, ``Civilization and Science, Part I,'' \emph{Popular
  Science Monthly} 13 (July 1878): 275.{]}} yet he is unable to trust
the future to the principle of unity. One is reminded of Goethe who
thought politics and religion a troubled medium for art.~

The English critic Matthew Arnold left to us this reminder:~``And~so I
say that, in America, he who craves for the\emph{~interesting}~in
civilization, he who requires from what surrounds him satisfaction for
his sense of beauty, his sense for elevation, will feel the sky over his
head to be of brass and iron. The human problem, then, is~as yet~solved
in the United States most imperfectly; a great, void exists in the
civilization over there; a want of what is elevated and beautiful, of
what is interesting.''\footnote{{[}Matthew Arnold, \emph{Civilization in
  the United States: First and Last Impressions of America} (Boston: De
  Wolfe, Fiske \& Co, 1988), 181.{]}}~

The one thing to set opposite these and other like criticisms, which
have not, been altogether beyond reason, is the new order of ideas
flowing out from the very conditions that prompted the fault-finding.
The message that is to be sent across the water to Oxford and Cambridge,
to the men of England, to the professors in German universities, to the
French Academy, tells of the foundations that have here been laid; of
the incoming of the spiritual control; of the organization of democracy.
The revolution has come full circle. We~have ``the fusion of the States
into the only reliable identity, the moral and artistic
one.''\footnote{{[}Walt Whitman, \emph{Democratic Vistas and Other
  Papers} (London: Walter Scott, 1888), 10.{]}} Here is the American
Idea. Matthew Arnold and his kind were unable to read the period of
American activity now closing as its paint-grinding stage---its artistic
preparation. The message accepts as fact that ``the United States are
destined either to surmount the gorgeous history of feudalism, or~else
prove the most tremendous failure of time.'' The message undertakes to
make good the position of Whitman:

\begin{quote}
Viewed, to-day, from a point of view sufficiently over-arching, the
problem of humanity all over the civilized world is social and
religious, and is to be finally met and treated by literature. . . .
Above all previous lands, a great original literature is surely to
become the justification and reliance~(in some respects the sole
reliance) of American democracy. . . . For I say at the core of
democracy, finally, is the~religious element. All the religions, old and
new, are there. Nor may the scheme step forth, clothed in resplendent
beauty and command, till these, bearing the best, the latest fruit, the
spiritual, shall fully appear.''\footnote{{[}Whitman, \emph{Democratic
  Vistas and Other Papers,} 6.{]}}
\end{quote}

The western nation comes to consciousness. Civilization changes front to
meet the new conditions.~

The advance of letters which America compels can come about only through
an integration of men.~In spite of~his great faith and insight, Whitman
was unable to see this with any clearness. His point of view was~on the
whole~individualistic. He looked to the rise of ``two or three really
original American poets,~perhaps artists or lecturers'' who should
``give more compaction and more moral identity . . .~ to these
States.''\footnote{{[}Whitman, \emph{Democratic Vistas and Other
  Papers}, 9.{]}} He could not see that the prophet of old is merged in
organization. Once merged in the social body, the great man---the mighty
poet, the national expresser of Whitman---is more distinctive, more
individual than ever; he has~entered into~action. Each man, great and
small, at last functions---is an organ. It is the final arrival of the
individual. This from Carlyle: ``I call this anomaly of
a~disorganic~literary~class the heart of all other anomalies, at once
product and parent; some good arrangement for that would be as
the\emph{~punctum~saliens}~of a new vitality and just arrangement for
all.''\footnote{{[}Carlyle, \emph{Heroes}, 121.{]}} The man of highest
insight related, functioned, the whole is in~each individual. The
individual is the political unit in its relations. Personality is the
man, the unit, distinguished from his environment---the man in his
privacy. The historic controversy regarding the place of the individual
in the State is closed. We now have the final word in reply to Sir Henry
Maine, to whom democracy was~``nothing but a numerical aggregate, a
conglomeration of units.''\footnote{{[}The quote is not from Henry Maine
  but from John Dewey's \emph{Ethics of Democracy}, which \emph{} is
  intended as a reply to Maine's argument. See John Dewey, \emph{Ethics
  of Democracy} (Ann Arbor: Andrews \& Company, 1888), 4.{]}}~ Here,
also, is the last of the checks-and-balances notion in politics---the
end of~Calhounism---and this without moving to the other extreme,
namely, that of a state pinned together by bayonets. The social system
provides its own balance. The principle of the grand division of labor
meets the last behest of Carlyle: ``How in conjunction with inevitable
Democracy indispensable Sovereignty is to exist.''\footnote{{[}Thomas
  Carlyle, \emph{Past and Present} (London: Chapman \& Hall, 1843),
  215.{]}}~

Whitman was unable to grasp that the dreams indulged in by philosopher
and savant, of a time when the dross, the merely personal element,
should be driven from literature, were about to come true. In 1848
Ernest Renan, struggling to pierce the future, wrote of ``a state of
things in which the privilege of~writing will no longer be a right
apart, but one in which masses of individuals will only think of
bringing into circulation this or that order of ideas without appending
to them the label of their personality.''\footnote{{[}Ernest Renan,
  \emph{The Future of Science} (Boston: Roberts Brothers, 1891), 212.
  Ford also discusses Renan's book in \emph{News is the Master Element
  of Social Control}.{]}} Nor could Whitman attain to this without fully
transcending the individualistic position. ``Imaginative literature''
was still, to him, a thing apart. He put journalism to one side as a
``specialty,'' in this not being in advance of Matthew Arnold\footnote{{[}Matthew
  Arnold (1822--1888) was a British poet and cultural critic who wrote
  extensively about journalism and first coined the concept of ``new
  journalism.''{]}} to whom ``literature'' was one organ and
``journalism'' another. The daily literature was away from~both of them.
Whitman was unable to read the forces compelling a new ordering of fact
from the base, through a gathering of men acting on a common principle.
Without this, however many-sided the individual, there is only the
dreary round of opinion. An integration of men is defined by what it
does--- through it, grasping and directing the movement of fact, all
sides of life are reflected.~

The new publishing business transcends the past; it undertakes to
decipher American life, to detect its inner meaning; to wrest from the
confusion of the hour the principle of order that is silently working
toward just ends, and thus to find in Law a refuge from the curse of
endless statute making. ``The riddle of the painful earth''\footnote{{[}William
  Dean Howells, \emph{The Rise of Silas Lapham} (Boston: Ticknor, 1885),
  284. The novel is often associated with the rise of American realism
  and pragmatism. See James Livingston, \emph{Pragmatism and the
  Political Economy of Cultural Revolution, 1850--1940} (Chapel Hill:
  University of North Carolina Press, 1997).{]}} shall vex us less.~

\hypertarget{ix-business-position-and-movement}{%
\section{IX. BUSINESS POSITION AND
MOVEMENT}\label{ix-business-position-and-movement}}

Three steps are distinguished in a commercial advance: (1) the discovery
and working out of a new principle, (2) the finding of men to execute
it, and (3) the external action---this last involving the relations with
money capital.~

The News Association is in possession of a new idea. This is attested by
more than one prophecy. The most notable is that of Mazzini. Writing in
1849, he said: ``Perhaps in religion, as in politics, the age of the
symbol is passing away, and a solemn manifestation may be approaching of
the Idea~as~yet~hidden in the symbol. Perhaps the discovery of a new
relation---that of the individual to humanity---may lay the foundation
of a~new religious bond.''

The~foreglance~of the great Italian~finds~confirmation in the principle
of the grand division of labor, through which the individual is made at
one with his fellows---with humanity. It is the development of interest.
The possession of this idea has for its practical outcome a new method
in journalism---in letters. Turned upon any corner in life, the result
is a revelation. When it is considered how largely existing literature,
the reigning concepts in jurisprudence, the prevailing
social~arrangements, are built upon the supposed antagonism of truth and
commerce, it will be seen how great is the overturning compelled by the
fact that the two have come to be in harmony. It is the transforming
power of an idea. It is the method of science. ``When the right thing
comes to~hand~we shall know it by this token: it will solve many
riddles.''\footnote{{[}A similar quote is attributed to Ralph Waldo
  Emerson: ``If the right theory should ever be discovered, we shall
  know it by this token, that it will solve many
  riddles.''\href{applewebdata://D02306DF-3E46-4684-BD1A-1A323FFB2CB2\#_msocom_1}{{]}}~}~

This new method gives such a lead to the News Association as to amount
to a practical monopoly. Each step in the action can only serve to make
the lead more certain. The secret can only be communicated in the doing;
once done, however, the publishing business is centralized and the new
organization secure in its position. All the forces of the time are
making to this end. The News Association is at one with the tendencies
of the day and hour, so that in place of fighting an established order
of things, as some might think, it strikes in just when the old order,
undermined at every point, is about to give place to the new. The
contention of the News Association with present newspaper methods is but
that of~Sir Henry Bessemer, the new steel-making principle in hand, with
the iron business. The rising and dawning state of intelligence is
detected in advance. In order to provide for registration at a center it
has been necessary to subject the movement, to get it in control;
without this, registration could not obtain in the primary region.
Failing to register there, the confusion must continue. This preparation
is the work of the News Association, constituting its vantage ground.~

The way has been prepared for a co-operative movement that is
all-embracing. The partnership in all science, in all art, of which
Burke wrote, becomes an every-day fact. The inquiry men of the world,
never so numerous and strong as now, are but waiting to be organized on
a common center, so that the life-bearing may be drawn from the work of
each for the benefit of all.

\begin{quote}
The concurrence of many can never be really effective, until it finds an
individual~organ to gather it up, and concentrate it to a definite
result. Sometimes the individual comes first, fixes his mind on a
determinate purpose, and then gathers to himself the various partial
forces which are necessary to achieve it. More often in the case of
great social movements there is a spontaneous convergence of
many~particular tendencies, till, finally, the individual appears who
gives them a common center, and binds them into one whole. But~in all
cases the effective co-operation, the real social force, is not present
till it has concentrated and individualized itself.\footnote{{[}Edward
  L. Caird, \emph{The Social Philosophy and Religion of Comte} (Glasgow:
  James Maclehose \& Sons, 1885),
  36.\href{applewebdata://D02306DF-3E46-4684-BD1A-1A323FFB2CB2\#_msocom_1}{{]}}}
\end{quote}

\noindent The News Association functions as this center of activity; through it
the movement of intelligence concentrates and individualizes itself.~

The question of men is solved. Workers in whom the practical and
organizing impulse is strong---the primary men---have already come to
the support of the principle in such numbers as to act as a guarantee on
the individual side, the present position having been brought about
through their labors. Beyond, the several gradations of men are coming
into view. There can be no difficulty here, since the changes in
conditions making for the advance serve at the same time to disclose the
new order of men. The course of things in school and college during the
last ten years has gone to turn the new supply of young men in this
direction. But more, the very conditions of all life in recent years
have so far stimulated the spirit of inquiry that an order of men is at
hand who can function only by going into the intelligence business after
the manner and on the lines proposed. Evidence as to the truth of this
multiplies on every hand. The point goes to make good the claim of
monopoly, for there is need of but one gathering of men.
Inquiry~freed,~the movement of men is toward the one center.~

In secure possession of an idea, the movement gets revenue from the
start. Experience, indeed, shows that distinct advances in the
publishing business do not absorb money capital to any great extent. The
history of successful publishing ventures confirms this on all sides. It
is the failures, the misconceptions, that absorb the money without
return. A movement rightly conceived justifies itself at every step. At
the very beginning there is an accumulation of goods for sale. This
is~more true~in the present case than in any~previous advance. Greeley
made an advance in the case of the \emph{New York Tribune}. Beginning
with a stock of salable goods, his movement was almost of necessity
commercial. His business idea, though he may never have phrased it to
himself, was that social organization had about come to the point of
rejecting negro slavery. When the principle passed out into the minds of
men generally, and slavery~actually began~to be rejected---in other
words, when the civil war began---the stock of news-capital was so large
that no one paper could carry it. In~fact~it taxed the powers of all the
newspapers in this country.~Thus~the principle for which~Greeley stood,
as soon as it became effective in shaping men's actions, gave rise to an
almost unlimited amount of immediately profitable news. We may apply
this illustration in the following way. The social organism has now
reached the point where it must speedily reject all remaining
slaveries---hindrances to full social activity. Those who clearly
perceive this fact have, therefore, an unlimited stock of news at
command. They can sell to the present first-hand newspaper the
announcement of the impending rejection. When this announcement begins
to take effect, to influence the actions of men, the new organization
can report and publish the progress of the war. Moreover, having already
mastered the lines of social organization, there results a monopoly of
intelligence, not obtainable without the possession of an all-embracing
principle.~

To make the most of the change in conditions, organization is entered
upon a little in advance, that is, just before improved facilities come
to the full. Here is the timeliness of the action. The growing
completeness of facilities is perhaps best illustrated by an extract
from Postmaster General Wanamaker's last report:

\begin{quote}
One cent letter postage, three cent telephone messages, and ten cent
telegraph messages are all near possibilities under an enlightened and
compact postal system, using the newest telegraphic inventions. The
advantage of tying the rural~postoffice~by a telephone wire requiring no
operator to the railroad station must be obvious. It is not chimerical
to expect a~three cent~telephone rate; the possibilities of cheapening
these new facilities are very great. All account-keeping could be
abolished by use of stamps~or ``nickel-in-the-slot'' attachments.
Collection boxes everywhere in the cities and many places in the country
towns would receive telephone and telegraph messages, written on stamped
cards like postal cards.\footnote{{[}John Wanamaker~(1838--1922) is best
  known for being a pioneer of the American department store and for the
  invention of the price tag. After opening a men's clothing store in
  1861, Wanamaker went on converting an abandoned Pennsylvania Railroad
  depot into a store called Wanamaker's,~in 1874, inspired by central
  markets in Les Halles, Paris, and the Royal Exchange, in London.~He
  served as the United States Postmaster General from 1889 to
  1893.\href{applewebdata://D02306DF-3E46-4684-BD1A-1A323FFB2CB2\#_msocom_1}{{]}}~}
\end{quote}

\noindent Here are the conditions toward which we are moving. Just ahead is the
local distribution of the telegraph wire. In consequence, the cry is for
the entrance of the integrity of commerce into the publishing business.~

~After all is said, it is through the economies brought in that the News
Association gets its field. The saving to be~effected~by throwing out
the editorial page and so getting order in place of dire confusion will
in itself be large. The resulting economies to the first-hand papers
will be so great as to yield substantial revenues to the central office,
to the News Association. Marked economy will result from reducing the
number of daily and weekly papers, and from cutting into the business of
the present so-called book houses. In New York and Chicago to-day there
are half a dozen or more deliveries of the one fact; that is, six or
seven papers are engaged in selling the same news. The multiplicity of
papers here came about at a time when the difficulty of getting at the
fact---the whole truth---put a premium on opinion. With distance gone,
the access to the fact is complete. In this light the superfluous daily
papers in the leading cities~are seen to be but survivals from the age
of~opinion---they are mediaeval. Their displacement only waits upon the
centralized action. Much the same is to be said of things on the book
side. Taken as a whole, the existing publishing business is the
surviving piracy. The waste of time to men of business in trying to keep
at one with the fact is so great that a remedy~has to~be provided. On
the money side it is safe to say that one-half of the great sum paid by
the people to sustain the present paper and book business is just so
much waste. It is the last great division of commerce to submit to the
modern economies. The present newspaper accumulates error: the new one
will record history. Truth is organizable: untruth not. The waste in the
business of publishing trade papers---the peculiar field of the Class
News Company---is fully equal to that~in the general publishing
business. The waste on the side of the bureau of information is shown in
another place (No. 5). Of the sum drawn from the merchants of the
country each year, amounting to \$6,000,000, one-half is paid under
protest---this in response to a half-formed theory that two or three
concerns must be sustained in order to keep up so-called competition.
The advance here, as in the general publishing field and in the sale of
class news, waits upon the incoming of scientific method.~

Something like an equalization~has to~be brought in, as between the
newspapers of the metropolis and the outside centers. On a rough
estimate, the more profitable papers of New York take in one hundred
cents and pay out thirty. The News Association will undertake to reverse
this, paying out seventy cents, say, and retaining thirty, the
difference going to the public. Through compelling this~economy~the
Intelligence Trust gets its function. The existing newspaper does not
know how to pay back its revenues in the form of intelligence; it does
not know how to do legitimate merchandising. James Gordon Bennett, of
the \emph{New York Herald}, gives \$100,000 to starving Ireland and
sends explorers into Africa, while the New York treasury tangle waits
upon the man who can reduce it to order.~

The mark of the action centering in the News Association is its
deliberateness and certainty. This is possible only through the
possession of a sure business principle. For the first time in the
history of the publishing business, money can be expended in advance of
publication. Heretofore this has been possible only on the physical side
in the purchase of machinery and the like. Here was the absence of
method from the business of dealing in intelligence. To make clear the
meaning, it is necessary to think of architects and builders without the
spirit-level, the plumb-line and the square. Having found his center of
gravity, the journalist now becomes a builder. With the new tools in
hand, it becomes possible to expend money systematically in organizing
the news of New York City on original lines. The use of money in this
wise, to the amount of say \$25,000, will give to those doing it a clear
vantage ground. Yet no amount of money would avail in this direction
without the new principle possessed by the News Association. Here is the
key to the business movement. In the absence of a unifying principle,
the newspaper is compelled to wait upon the catastrophe, upon the event.
In this way the pathology of life is given undue prominence. Detecting
the central principle in life, the physiology of the State is grasped.
With sure step, the newspaper now enters the field of prediction. Thus,
with a small expenditure of money in advance, we~are able to~raise up
great areas of~\emph{new~}news.~

The underlying principle unites conservatism and radicalism in the one
business of inquiry. Impartial inquiry has its way, for the nearer the
truth the greater the profit.~

The fundamental titles in literature rest in the copyrights of the News
Association. With one or two~exceptions, these volumes are by no means
class literature; they relate equally to the school and to life.~

\enlargethispage{\baselineskip}

With respect to political ideas, the movement is the last of utopianism.
The practical everywhere obtains.~

Present reporting methods are transcended through taking more definite
account of the play of moral forces. The play of interest in life is
reported to the full. This has not heretofore been regarded as
possible.~

America is the great news field of the world. This fact is only now
coming into general recognition. The first business will be to organize
the news of this country. Its food news is the leading fact in the
world's commerce.~

One of the things that the action will cleanup is the limit to the
endowment of scientific research. Commerce will now pay scientific men,
as never before, for interpretations. The interest of the news merchant
in scientific inquiry is enlarged on all sides.~

The unity of intelligence is laid bare in the fact that men do not want
two crop reports; there is but the one report.~

To bring science into the publishing business is to take a great step in
the direction of making all business scientific, since each branch of
commerce will be subject to scientific reporting. A great struggle has
been going on for the organization of the business of the country, to
cut off waste, to reach the highest economy. Business can only organize,
become scientific, through the distribution of intelligence. The public
need here is prodigious; the profit must correspond.~

The News Association looks only to re-forming the publishing business.
It is not in the business of ``reforming'' society.~

The principle is so worked out into its resulting facts, that to
institute and carry on the new publishing business is a simple matter of
administration, like bridging a river or building a railroad.~~

In relation to districting the country it is found that each district is
a whole within itself. There results a seeming paradox. Thus, it will
pay to organize a single news-district, the state of Ohio~say,~because
of the fact that the whole country is to be dealt with in like
manner---the news of one district is to be exchanged for that of the
country and the world, and the reverse. And yet it will pay to organize
each district in and for itself---this because of the demand within the
state for village and county news. The district of Ohio may be used to
illustrate. Making its headquarters at Columbus, the News Association
will go over the state county by county, raising up the self-registering
machinery already coming into place. The motor here comes from the
necessity of organizing the whole country. The result is a large amount
of desirable and immediately salable news. This news is sold in the
first instance to the local journal, and especially to the first-hand
newspapers of Columbus, Cincinnati, Cleveland and Toledo. Pittsburg also
comes in for a share because of its interest in southeastern Ohio. It
will pay to press the triangle on each district. News of local value
must be gathered up in order to get from it the news of general
interest, for sale to the entire outside list of first-hand papers. At
the same time the class news in each region is singled out for
collection, while the needs of Fords are looked after in all ways. The
intelligence market is the most expansive known; to glut it is
impossible.~

The movement on the side of the idea, as such, has been from the whole
to the part---from the principle to the resulting facts. Reversing this,
the external action is from the part to the whole. The work of realizing
the new thinking on~Monroe County~well along, the business of organizing
the news of New York City on original lines will be entered upon.
Meanwhile, the division of the country into news-districts, and their
close organization will go forward. There are two streams of news, the
city to the country and the reverse---Urbs and Orbis. The country
organized in good part, the metropolitan news-district mastered, and
European and other foreign connections in progress, we may issue the
daily \emph{Newsbook} at New York, and so, in turn, bring up the
organization of the entire country.~

\hypertarget{x-the-incorporation}{%
\section{X. THE INCORPORATION}\label{x-the-incorporation}}

A new principle in the field of profit-getting~has to~be reduced to
action, and in such a way as to provide at once for stability and
freedom of movement. The principle asserts the unity of truth-seeking
and the money motive in the publishing business; that the interests of
the study and the market-place have come together. The ``editorial''
room and the business office of the newspaper are thus brought to their
own unity. The controlling principle of the newspaper is disclosed in
its own commodity---intelligence.~~

It is found that the action involves seven primary divisions of labor.
In this light it is proposed that the direction of the enterprise be
lodged in a board of seven trustees, to be made up of the persons
standing for the primary divisions of labor. Here is the reality of the
incorporation, the legal forms providing therefor being a question of
detail.~

The seven divisions may be thus indicated:~

\begin{enumerate}
\item
  The chairmanship, or the administrative head.~~
\item
  The department of politics. Its chief supervises the reporting of
  social organization, as such, in all directions.~
\item
  The work under the third division is closely related to that of the
  second. Its chief has an eye to reporting the Embryonic State---the
  business of teaching and the schools. The practice of medicine, and in
  general the psychical side of all social doing, comes under this
  head.~
\item
  The chief of this division has charge of the whole field of physical
  inquiry. He may pick out a chemist for Fords or for the Class News
  Company, or he may have directly to do with the meteorology side of
  crop reporting. This fourth division is closely related to the
  reporting of all price-making influences.~
\item
  The legal department of the Association is prominent. It looks to
  reporting the work of the courts and has an eye to the legal
  complications which the business will involve. To buy and sell the
  truth is to attack convention. A lawyer's care is needed here so that
  the Association may always be on sure ground.~
\item
  The treasury embraces the printing or selling department. The
  treasurer of the News Association will have general charge of the
  finances of both the Class News Company and Fords. He organizes the
  action on its counting-room side.~
\item
  This division comprehends the managing directorship on the goods or
  buying side; especially~with regard to~the daily reporting of life
  through the~\emph{Newsbook}~and the~\emph{Town}, and the first-hand
  distribution. The chief of this division looks to the artistic side
  more particularly, to the detail of the great business of expression.
  He is the chief buyer of the Association.~
\end{enumerate}

It has not been found possible to add to or take from these seven
divisions. One or more heads of divisions may act as officials of either
the Class News Company or Fords.~~

\hypertarget{xi-the-advisory-board}{%
\section{XI.~THE ADVISORY BOARD}\label{xi-the-advisory-board}}

It is proposed that an advisory board composed of merchants and
professional experts be brought together. It will be made up of men
having an intimate and comprehensive knowledge of~particular
groupings~of commerce, men whose occupations are parallel with the fact
and who at the same time are in sympathy with the enterprise in all its
bearings. A board thus composed would make up in the several departments
represented a kind of register of practical experience. One of its
members should be a representative banker, another a merchant having
knowledge of the realities of foreign trade, and so on.~~

\hypertarget{xii-the-integration-of-men}{%
\section{XII.~THE INTEGRATION OF
MEN}\label{xii-the-integration-of-men}}

To build from the fact in the publishing business is to provide for the
same integrity in the handling of its goods as now obtains on the
physical side. The methods to be brought in rank with engineering. The
degree of integrity corresponds to that insistence upon accuracy, to
that responsibility between men, which drives in disgrace from the
Produce Exchange of New York or the Board of Trade of Chicago, the
member who brands~mouldy~flour. The morale of the publishing business
with respect to the integrity of its wares is to-day so far below this
that to act on the straight-goods principle is to bring in a new order
of men. The movement enlists, therefore, the responsibility that bridges
Niagara, invents a perfecting press, or predicts an eclipse. It is these
men---the truth men---who have brought to the perfect working point the
printing press, the~locomotive~and the telegraph. It is this order of
men that is to make the advance. The energy wrapped up in the basic
movement having been freed, it may now be enlisted for the spiritual
uplifting. The forward movement has indeed waited upon the coming of
those who could construct the new machine, for the new principle with
its ramifications is no less a machine than the thing of bands and iron
which, multiplied, prints, folds and counts unlimited copies per hour of
a great newspaper. In this case the new machine is the further building
of intelligence. The one is written in iron, the other in our changed
ideas of institutions, of associated life, consequent upon the
disclosure of publicity as commodity. In either case it is an
integration, a co-ordination of parts. It is simply the order of men
referred to taking new direction and~re-forming for the advance. These
men can only think full circle, that is, relative to action.~

The present nominal organization of the newspaper has made great
progress of late toward consciousness of its own confusion. The men
themselves are widely seeking for new direction. The
definite~outleading~waits upon the new drawings. The rise of press clubs
and the like are indicative of growth in morale.~

The present situation, the strength of the position, has come about
through the joint work of men at the University of Michigan. This not
alone through those who are in official relation with the institution.
Instead, the breeding of ideas has proceeded from mixing with men on the
outside. Bringing the new principle here, it was found indeed that
approaches toward it had already been~made from the official side. But,
in course, others beside the writer had come up to the University with
kindred ideas, those men having also no official connection. Here is
illustrated the growing tendency on the part of men with idea-germs to
find their way to a university center for friction and light.~

\hypertarget{xiii-the-university-phase}{%
\section{XIII. THE UNIVERSITY
PHASE}\label{xiii-the-university-phase}}

A university is the unit of intelligence in the organized social body.
In the locality sense, it is the organized movement of life~in a
given~region through its center; the center is the university.~

The question of university organization and the newspaper problem, in
essential character and in their wider relations, are one. In each case
the solution lay in removing the hindrances to free inquiry. University
organization has got to be worked out; it cannot be thought out as a
thing apart. It is the organization of life itself. The remaining chasm
between the university and life---its isolation from the
people---coincides with the limit to inquiry. The old-time limit turned
upon the difficulty of access, distance standing in the way. Under such
conditions university habits partook of the cloister. The great
resistances cleared away, inquiry is freed for the journalist and the
college man alike. Responding to this change, the university~connects
itself with the daily movement. The alternative is the multiplying of
professors and tutors for the mere iteration of past knowledge.~

As now constituted, the university does not face any immediate demand,
is not connected~on the~whole~with any direct market, by which the
character of its product, and so its method may be exactly determined;
that is, the university does not have to answer to the movement of
intelligence. It is seen that the university lacks direction and so
definiteness. The newspaper, on the other hand, cannot obtain certain
high grades of goods, cannot obtain assured quality and kind in news,
and so do all-round and legitimate merchandising. The identity of need
here is that the university, lacking its organ of distribution, lacking
communication with its market, cannot get its daily and hourly direction
and so its function; and that the newspaper, which should be concentric
with the university as source of certain generics and particulars in
life, is cut off from this important center of supply. The newspaper is
shut~out from the laboratory of the scientist and savant. On the other
hand, the university is shut in, is apart from life. The action here
lies in~co-ordinating~these two phases of the one movement of life. On
the one side is the News Association as universal news dealers; on the
other, the university as a distinctive phase of inquiry.~

This development already has its beginnings within the university in the
Seminary.~A number of~students by co-operating under the leadership of a
single instructor become an organized instrument of investigation.
Mental forces that would otherwise be dissipated are brought to a focus,
and act with telling effect. The word seminary, used in this sense,
comes to us from Germany. In America the principle is better known among
teachers and students as the ``university,'' in distinction from
``college'' development. In German universities during the last fifty
years the seminary principle has been applied to Biblical criticism, to
philology, to the history of jurisprudence, to the past development of
economic notions, and the like. In America the spirit of inquiry in
university life has likewise been gathering force, but here the
movement~has followed the lines of ``political science.'' Evidence of
this is seen at Harvard,~Yale~and other eastern universities and in the
schools of Political Science.~

In Germany, under the direct plea of State interest, sharp limits are
set to political inquiry whether in or out of the university. In America
we have nothing of this, yet hindrances to free inquiry exist which, so
far as they go, are not less definite. Held back by merely conventional
ways of looking at life, or by the immediate pressure of class interest,
political inquiry, in our universities, has halted at the point of
greatest need; it has dwelt upon the past in place of looking into the
present and the future. Or, at most, it has touched the present in an
academic way only, that is, as if the present were dead.~

The university has long been free to study physiology in the individual
man. A further change in conditions now frees it for investigating the
physiology of the State. As with the conditions affecting the newspaper
so with the university, the limits to inquiry have been fixed
politically; class interest has barred the way. The mental tools, the
treasures from the past, may now be turned directly upon life, and this
with sole reference to~co-ordinating~the facts of life. Until these
tools are turned upon social life, the full connection for the
department of politics is not secured. Intimate union for the whole
university waits upon the political connection; there the pressure of
class interest is keenest. With distance gone, the material of the chair
of politics in the university is no longer a mass of abstractions. The
material is around everywhere, each turn of affairs presenting an object
lesson. The generals are brought to bear on the particulars, and these,
in turn, re-shape and further organize the generic ideas. The change in
direction made, near-by things will be studied first, getting access
through railroad,~telegraph~and telephone. Already in the region of Ann
Arbor, as one instance, the telegraph has been given local distribution,
the farmers taking the wire into their houses. The farmer's daughter
learns to use the Morse instrument. One step more and the wire is taken
into the schools and university, thus making the direct connection with
life. The chair of politics now has its clinic; the psychologist may
observe directly the mental habits of the people pursuing their daily
vocations. Developed to the full on the inquiry side, the university
becomes~a center of action. The university is at one with science; with
commerce which is everywhere becoming scientific. The university is then
a ganglion in the nervous system of the state.~

On its internal side the university organizes---relates with life. It is
in this way that the natural divisions of labor will be found, inquiry
subdividing as light is got in action. The university builds and expands
from its own unity---the unity of inquiry. The people now come up to the
university since it is meeting specific demands; they are in daily
communication with it. The News Association handles fact and may apply
to the university laboratories for special analyses or investigations.
The university indeed furnishes the laboratory of a given region.
Certain lines of fact can only be got there. Experts on related lines
must be within easy call. The chemist before certifying a given analysis
may have to confer with the histologist and pathologist of the
university.~

The university gets its hearing through commerce. Without the direct and
plain demand from commerce it cannot cross over to full inquiry. The
demand once made it~has to~be met, for it comes in the majesty of public
need, which crushes if not obeyed. The connection with life---the
electric wire and all that it implies---is presented by commerce.~

The allied forces proving strong enough, the university~is able to~break
through the walls of convention behind which class interest finds
shelter. To get to the people the university must go clear over and
inquire into the general interest. The university and democracy thus
identify; science and the commonalty are one. The identity~gained,~the
university organizes on its external side. The fiscal problem in
university organization finds permanent solution. There is a popular
feeling, which does not altogether lack justification, that the
university is in many respects painfully at odds with the real movement
of life. It~should now be evident how this obstacle can be overcome.
Through the organic principle, science---which is the university---and
the people may be brought together. The university connected
with~life,~its position~is commercial. That is, it is always giving a
return and, moreover, this return is obvious. The university is seen to
play its part in the interaction of life. Thus, its position
established, its necessary demands for money are met willingly as a man
pays for anything he wants: Sympathy with the university increases at
every point.~

This outlook calls up Cardinal Newman's\footnote{{[}John Henry Newman
  (1801--1890) was a British theologian who was canonized by Pope
  Francis XVI in 2019. His 1852 book \emph{The Idea of a University}
  defended the disinterested pursuit of knowledge and liberal
  education.\href{applewebdata://D02306DF-3E46-4684-BD1A-1A323FFB2CB2\#_msocom_1}{{]}}}
opinion in his book on the ``Idea of a University.'' He bore testimony
to the prime value of the free mingling of students from which~``they
gain for themselves new ideas and new views, fresh matter of thought,
and distinct principles for judging and acting, day by day.'' The effect
``may be fairly called . . . an enlargement of mind.'' After all
limitations are taken into account, students so situated, adds Newman,
``are likely to have more thought, more mind, more philosophy,~more
true~enlargement, than those earnest but ill-used persons who are forced
to load their minds with a score of subjects against an examination.''
As opposed to the extreme officialism, he preferred a university ``which
did nothing'' beyond providing a center for the coming together of
truth-seekers.~

While these points relate to university organization in general, they
have a more direct bearing on the~particular problem~of the University
of Michigan, where the considerations have been worked out. It is there
that the spirit of inquiry is yearning to mix itself with life. Already
the University has moved out so far that its future is bound up with the
organic principle.~Happily~it has not cumbered itself with a ponderous
school of philosophy, nor has the school of political science,
so-called, assumed unreal proportions. Should the organic principle
obtain, Ann Arbor becomes the seat of the representative university of
the world---the model for all time. It is the National University.~

The recent increase of students at the University of Michigan is an
indication of the way in which the university system of the United
States is organizing---centralizing. In obedience to the tendency of the
hour, a given group of states pitches upon a particular locality as its
university center.~Thus~as matters now stand Ann Arbor is the chief
center of learning for Michigan, Ontario, Ohio, Indiana, Illinois, and
perhaps Wisconsin. Students come, of course, from beyond these states in
proportion to the fame of the University. That a body of new ideas
should have been worked out at Ann Arbor is traceable, as already
indicated, to the freed conditions existing here. From the beginning the
University set its face to the future. It would be going into a
digression to detail the reasons for this attitude; it is enough to say
that, having to do without great money endowments, the University of
Michigan has had no recourse save to meet the incoming life through
outward movement. Institutions, like men, without money are thrown upon
their effective ideas. Great endowments, on the other hand, stand for
the ideas of the past. The fact is to be noted that the men and women
who come here for instruction, for inquiry, are, more than those of any
other great American university, from the homes of the people. From farm
and~village~they are here to work out the new ideas which promise to
confer upon the University its highest usefulness and distinction. It is
indeed significant that a university is the nursery of these ideas.
``Before printing,'' writes Michelet in his history of France,~``before
the supremacy of the press, under which we now live, the only channel of
publicity was the oral instruction dispensed by the
universities.''\footnote{{[}Jules Michelet, \emph{History of France,}
  vol. 2\emph{,} trans. G. H. Smith \emph{} (New York: D. Appleton,
  1882), 40.{]}} The time has now come when the confusion incident to
the great incubus of printed matter can be cleared away only by a return
to the university.~

\hypertarget{xiv-public-necessity-of-the-action}{%
\section{XIV. PUBLIC NECESSITY OF THE
ACTION}\label{xiv-public-necessity-of-the-action}}

This nation is at its mental crisis. Secession lurks in our statutes and
stalks in our courts. Our whole body of jurisprudence is built upon the
supposed antagonism of the individual and the common good. This must
persist until through the functioning, the getting there of the man of
letters the great accounting can be made. The man who sells truth
reveals the identity of the individual and the common interest. It is
the~union of the whole with the part. Perceiving all this we also see
that to avoid a fatal issue, or at least a period of dire confusion in
the life of the State, the division must be fought out in the ``still
and mental'' field; otherwise, there is a return of physical conflict.
Unless intelligence be unified here, unless a single mind can be secured
from Maine to California, the nation in the moral sense must go to
pieces. The solution of this great problem is the new Gettysburg.~

The war cry of a false socialism is heard on every hand. Through this
and that mechanical change, by some hocus-pocus in the fiscal region, or
by some other device, it is thought to heal the division in the State.
But the road to social union lies through the organization, the
socializing, of intelligence. It is the straight and narrow way; only by
following this road will the nation gain the victory over itself. Hope
lies in the very greatness of the need---it guarantees the execution.
But the situation is pressing. The centralizing tendency of the physical
commerce is seriously hindered by ignorant and hostile legislation. The
new machinery is opposed by the owners of the old. The men so hindered
in their operations are at a loss to account for it. Charged with
``selfishness,'' they have no adequate reply; are not able even to make
return in kind. Violence is opposed to violence, and only through the
incoming of the Intelligence Trust can the breach be healed.~

Great significance~is bound up in the fact that it is English-speaking
men who are to bring intelligence to a~centre~and distribute it. In this
is finally certified the power resting in the hands of England and the
United States jointly. Mr. Gladstone, writing of the English and
American peoples, said: ``They with their vast range of inhabited
territory, and their unity of tongue, are masters of the world, which
will have to do as they do.''\footnote{{[}William E. Gladstone, \emph{On
  Books and the Housing of Them} (New York: Dodd, Mead \& Co., 1890),
  8.{]}}~

\vspace{.2in}

\begin{LARGE}
    


\hfill \textsc{Franklin Ford}


\end{LARGE}

\vspace{.1in}

\textsc{Ann Arbor, Mich.}, July 1, 1892













% A NEWSPAPER LABORATORY
\chapter[A Newspaper Laboratory]{A Newspaper Laboratory}
\label{ch:A Newspaper Laboratory}
\chaptermark{A NEWSPAPER LABORATORY}

\vspace{.2in}

\begin{LARGE}
    
\smallcaps{Franklin Ford}\marginnote{Letter to James B. Angell sent from New Orleans, April 13th, 1887.}

\end{LARGE}

\vspace{0.5in}

\begin{center}
    

\Large{FROM~FRANKLIN FORD~}

\end{center}
\vspace{.1in}

\noindent Central Office, New~York~

\vspace{.1in}

\begin{center}

New Orleans, April 13,~1887~
\end{center}

~

\noindent Dr James~B. Angell\footnote{{[}James Burrill Angell (1829--1916)~was~an
  American~professor~and~diplomat. He was
  the~longest-serving~president~of the~University of Michigan,~from~1871
  to 1909, a period marked by a movement to democratize education and by
  an important development of the institution. Angell came to Michigan
  after leaving his position as professor of modern~languages~at
  Brown~University,~wartime~editor of the~\emph{Providence
  Journal}~and~President~of the~University~of Vermont. In
  1880~he~was~appointed~United States~Minister~to China.{]}}~

Ann Arbor, Mich.~

~

\noindent My dear Sir:~


You will recall my speaking to you on the campus during the last week in
February. In a half dozen sentences I tried to tell you of the work I
had undertaken. I do not think that I succeeded in conveying the point.
Let me go into details~a bit. You understand, of course, that I was
Editor of the paper called~``Bradstreet's,'' from the spring of
1880.~This is the equivalent of saying that~I developed the paper from
its crude beginnings. To the Bradstreet
Co,~the~newspaper~\emph{Bradstreet's}~was and is an advertisement. To~me
it was a newspaper laboratory---a place in which I might experiment and
conduct researches into the state of the publishing business. Confining
myself to results, let me say that about a~year ago I had fully wrought
out in detail the conclusion that a far-reaching newspaper advance had
become possible---this, through perceiving that we now have the
resultant of the locomotive and telegraph---the elimination of distance.
Distance gone, publicity becomes a commodity in the widest and fullest
meaning of the term. The truth conception becomes the commercial
conception at the counting-room of the daily newspaper. Journalism (if
the foregoing is true, is not the word obsolete? Is it any longer an
ISM?) is an organism. Inquiry is~organizible {[}\emph{sic}{]}. The facts
of life may be coordinated. Let me again repeat. Social inquiry in the
widest and fullest sense, is commercial.~

The daily newspaper becomes thus simply the vehicle for selling the
results of inquiry.~We~are to act upon the unity,~principle~and the
so-called ``editorial'' or academic page, must go. You must see that I
have been studying the physics of letters. We are at the end of the
physical age. We find the machine (printing-press, locomotive,
telegraph) has been perfected to the point of ease.~A mentality may
therefore be imposed upon it, and the newspaper becomes an articulated
thing. The publishing business (by this I mean both the book-house and
the newspaper) is undergoing a revolution: it is being resolved into the
intelligence business. The newspaper---the morning book, is becoming
primary. The book (a book-binder's term) is to be secondary. You must
see that we get here~the conception of the Distributive University.~You
may be familiar with Emerson's prediction.\footnote{{[}For a detailed
  discussion of Ford's entanglement with the work of American
  philosopher Ralph Waldo Emerson (1803--1882) and Scottish historian
  Thomas Carlyle (1795--1881), see the introductory chapter of this
  book.{]}} You will find it in the Carlyle--Emerson correspondence.~

In a letter of 1844,~Mr.~Emerson bade his English friend be of good
cheer,~for the reason~that~he (Emerson) was able to see the rise of the
Organic~Letters. Again---this~time from Carlyle himself. Take down
``Heroes and Hero Worship,'' turn to the Hero as Man of Letters, and
note this point: ``If you ask me what were the best possible
organization for the Men of Letters in modern society . . . I should beg
to say that the problem far exceeded my faculty.'' From another place:
``I think~we may conclude that men of letters will not always wander
like unrecognized,~unregulated~Ishmaelites among us.'' Again, more~to
the point:~``I call this anomaly of a~disorganic~literary class, the
heart of all other anomalies, at once product and~parent;~some good
arrangement for that would be as the \emph{punctum~saliences}~of a new
vitality and just arrangement for all.'' So much for Emerson and
Carlyle.~

Now turn if you please, to the introduction to Mills's \emph{Logic,} and
note this sentence (I may not~quote~literally): ``The key to the science
and organization of life is still an open question.''\footnote{{[}The
  original quote reads: ``The definition of the science of life and
  organization is still a matter of dispute.'' John Stuart Mill, \emph{A
  System of Logic} (New York: Harper \& Brothers, 1858), 1\emph{.}{]}}
Once more, you are doubtless familiar with the point in one
of~Coleridge's~lay sermons wherein he seeks to disclose the inner fact
of the social distress of the day; we now call it The Labor Question. As
nearly as I can recall,~Coleridge~said: ``Let us sweep away the surface
facts that are~the~common property of all, and penetrate to the~inner,
or spiritual fact. We find that the prime difficulty lies in the
overbalance of the commercial spirit with
no~adequate~counterweights.''\footnote{{[}Ford refers to the work of
  English poet and philosopher Samuel Taylor Coleridge (1772--1834), but
  the exact citation could not be found.{]}} He named three attempted
counter influences, Religion,~Philosophy~and the Aristocracy. This,
while declaring~that the desired equilibrium has not been reached.
The~Man~of~Letters is to go down into~the physical region. The need is
disinterested price making.~Carlyle's~disorganic~man must become a crop
reporter. To this end I have undertaken to~make a scientific---in the
business sense---classification of the volume of intelligence~which is
the result from the organization of inquiry. This classification
has~regard to~the two sides of the~intelligence business:~(1)~The
dynamic, or projectile side;~(2)~The bureau side. Conceive the~circle as
symbolizing the whole of Inquiry; that it is divided into two
semi-circles. The lower~semicircle may stand for the Bureau side of the
Intelligence business. This~bureau side divides artificially into three
segments, as follows: (1) The individual trader; (2) The Cooperation;
(3)~The~General Inquiry. The Bradstreet Co (and for that~matter Dun \&
Co as well) has exploited the first~segment. The remaining two segments
have yet to be organized. So much for the bureau side.~

The upper semicircle suggests concentric rings whose name is legion. I
have detected seven primary groupings,~classifications~or rings. They
are as follows:~

\begin{enumerate}
\item
  The weekly newspaper FOOD, which is to represent all that the word
  FOOD calls up to the mind, regard being paid only to the price-making
  influences;~
\item
  The weekly newspaper METALS;~
\item
  The weekly newspaper TEXTILES;~
\item
  The great morning newspaper THE NEWS BOOK;~
\item
  The lesser daily THE TOWN;~
\item
  THE WANT (advertising);~
\item
  ARCHIVES, a weekly newspaper presenting the documentary history of the
  time.~~
\end{enumerate}

In brief compass I have thus sought to give you a more definite idea of
what I am about. I~should like the advantage of a full talk with you.~It
is just possible that you will be in New York before long. A year ago,
December, I made the~first draft of a report on the state of Letters,
comprising some nine thousand words, involving both the philosophy and
the practice. On my return~to~New York~I shall recast the document,
carrying it out more in detail.~I intend you to see this. If you have
any points for us, pray write to No. 102 W 61 St,~New York.~

\vspace{.15in}

\hfill Very truly, yours,~

\vspace{0.1in}

\hfill Franklin Ford

\vspace{.15in}

\noindent\emph{Should like to hear from you. You will understand of course that the
above given points have yet to be made public.}

%%

% BANDING TOGETHER THE LEADING NEWSPAPERS
\chapter[Banding Together the Leading Newspapers]{Banding Together the Leading\\\noindent Newspapers}
\label{ch:Banding Together the Leading Newspapers}
\chaptermark{BANDING TOGETHER}

\vspace{.2in}

\begin{LARGE}
    


\smallcaps{Franklin Ford}\marginnote{Letter to Edward Atkinson sent from New Orleans.}

\end{LARGE}

\vspace{0.5in}


\begin{center}
    


Memorandum

\vspace{.1in}

\Large{FROM~FRANKLIN FORD}

\end{center}

\vspace{.1in}

\noindent CENTRAL OFFICE,~NEW YORK~~~~~~~

\vspace{.15in}

\hfill New Orleans, April 13th,~1887

~

\noindent Edward Atkinson Esq.\footnote{{[}Edward Atkinson (1827--1905)
  was~a~cotton~manufacturer,~economist, political
  activist,~and~inventor.~When~his~cotton~mills~began~to fail in the
  mid-1870s,~he~entered~the~railroad~industry and
  later~worked~as~President~of the Boston
  Manufacturers~Mutual~Insurance~Company.
  Known~for~his~abolitionist~involvement~with~the~Free Soil Party~and
  the~Boston Vigilance Committee, he also founded
  the~Anti-Imperialist~League, which~opposed~the American annexation of
  the Philippines in the late 1890s. Based on an in-depth study of
  cooking (energy consumption costs, food chemistry, nutritional intake,
  etc.), he designed the ``Aladdin Cooker," a device which prefigures
  the modern crockpot. He~was~elected~a~Fellow~of the~American Academy
  of Arts and Sciences~in~1879. The correspondence of Ford and Atkinson
  lasted from 1885 to 1889. It covered a crucial period during which
  Ford quit \emph{Bradstreet's} (to which Atkinson contributed articles)
  and tried to initiate practical attempts at reforming the press. Among
  other topics, they discussed the launch of a specialized trade paper
  to be titled \emph{Food}.{]}~}

\hspace{.15in} Boston, Mass.~

\vspace{.1in}

\noindent My dear Sir:~~

\vspace{.1in}

I have got thus far in the work of visiting the chief
intelligence~centres. I have come from Chicago by way of St. Paul,
Omaha, Cheyenne, Denver, Leadville, Kansas City, St. Louis, Memphis and
Nashville. I~go from here to Galveston. From there I shall return to New
York by way of Birmingham, Atlanta, Savannah~and Charleston. I have
succeeded in banding~together the leading newspapers to receive
intelligence from~New York. I shall begin by sending out matter for
publication which will bear~mail transportation. You know I am relying
upon you for valuable assistance. I can handle now to a greater
advantage~than~ever~before everything that you~may write, save at~times
the more elaborate matters and that too. I shall see you in Boston soon
after reaching New York. I expect to be in New York about April 25th at
the latest.~

\vspace{.15in}

\hfill Yours, very truly,~

\vspace{.05in}

\hfill Franklin Ford.~



% THE PRESS OF NEW YORK---ITS FUTURE
\chapter[The Press of New York---Its Future]{The Press of New York---Its Future}
\label{ch:The Press of New York---Its Future}
\chaptermark{THE PRESS OF NEW YORK}

\vspace{.2in}

\begin{LARGE}

\smallcaps{Franklin Ford}\marginnote{Chapter published in \emph{Progress and Prospects of New York: the First City of the World}, 1492–1893, 46–47. New York: Commercial Travelers Club of New York, 1893.}

\end{LARGE}

\vspace{0.5in}

\newthought{The future of}
 New York journalism is that of journalism itself. The
moving realities of the business must first head up at the metropolis of
America. It was there that the new machines of the century, the
locomotive and the electric wire, were first brought into full use for
newsgathering. The enterprise of the elder Bennett\footnote{{[}James
  Gordon Bennett (1795--1872) published the first penny paper, the
  \emph{New York Herald}, in 1835.{]}} was a clear step forward. Then
followed the New York Associated Press, organized to divide the cost of
news transmission when telegraphing was expensive. The cost of
telegraphing has now fallen so low that it is no longer a hindrance to
the freest action. Telegraph charges to the great newspaper are to-day
no more than the postage stamp to the individual. With the discovery of
this fact, a new departure in journalism becomes possible.~

The newspaper has now at its service an efficient machine. The
long-distance telephone marks the completion of this new machine. The
next step was to set about organizing the commodity in which the
newspaper deals---intelligence. To do this is to get and publish the
truth about all sides of human affairs. The more truth the more news,
and the more news the greater the profit. The newspapers want this, but
are coming to see that the end can only be reached by raising the
quality of their goods.~

News thus disclosed as commodity is to follow the law of all other
commodities, that is, toward improvement in quality with consequent
wider consumption. Back in the 70's the people wanted better
kerosene.~The Standard Oil~Company furnished it. The result was a
centralized industry, gradually increased consumption of goods, and
lower prices. The publishing business has got to go through the same
movement. In effecting this the distinction between news and
``editorial'' will be lost. There is only news---the new thing. The
whole publishing business is to be raised to the NEWS idea. The
so-called book business is speedily to become secondary or accessory to
the daily newspaper.~

In this New York is to take the lead. Were the movement to originate in
the heart of the country, which is not unlikely, it could only be done
in relation to New York. The historian Freeman,\footnote{{[}Edward
  Augustus Freeman (1823--1892) was an influential English historian.{]}}
in one of his lectures a few years ago, declared that the world had come
to be~Romeless, that it was without a center. ``No
longer,''~~he~wrote, ``does an undivided world look to a single Rome as
its one undoubted head. The great feature of the most modern times . . .
is the absence of any such center as the world so long gathered itself
around.'' New York is the future Rome, for, in the fullness of electric
transmission, it is to become the clearing house of the world's
intelligence. In the newer commerce, now fast gathering force, New York
is center. A remarkable thing is the fact that the world's intelligence
is to be centered and coordinated by~English-speaking men.~

The multiplication of daily newspapers at New York came about through
the premium placed on opinion. When the whole fact was inaccessible,
this~and that paper was able to sell what~some one~merely thought was
the fact. That day has passed away with the incoming of more complete
access. The daily newspaper of the future will replace editorial writing
with the skillful and full report. New York will do this first. Once
done, the need for half a dozen deliveries of the purported fact will
have disappeared.~

The change will compel a deal of preparation, but numbers of men must
already be working at it. On the one hand, the scientific center is to
be erected; on the other, the country will be reported. The one central
establishment will take account of the three sides, and, therefore,
three profits, which all news presents. For two of these sides, or
profits, the trade or class papers, and the ``mercantile agencies,''
stand.~

To effect all this, the country will be divided into districts, the
manager of each district to draw his salary from New York. The one
organization will collect all news, selling its goods through the daily
newspaper, the class paper, and the bureau of information. Concerns like
the Bradstreet Company and Dun \& Company mark the beginnings of the
last named.~

The changes pending in journalism, and, therefore, in the publishing
business as a whole, will be as profound as those following upon the
iron business in consequence of Sir Henry Bessemer's steel-making
formula.\footnote{{[}Henry Bessemer (1813--1898) was a British inventor
  and industrialist whose steel-making process would become the most
  important technique for making steel in the nineteenth century. Ford
  also refers to Bessemer in his \emph{Draft of Action}.{]}}\\\noindent The late
postmaster of New York, Henry G. Pearson, was a close student as to the
direction of the social forces. He used to say that the present
post-office is about completed; that the world must have a new one. This
new one he called the spiritual post-office, or the great organic,
centralized publishing business. `\,`Down at our place,'' Pearson would
say, ``we are arrested if we open a letter. In the post-office that is
to be, the arrest will be for failing to open them.'' He believed that
the thought of the people was to find registration.~

D. G. Croly, one time managing editor of the \emph{New York World}, but
now gathered to his fathers, insisted that ``journalism has a theory and
a practice which it is desirable to reduce to form.'' He was, of course,
right. In thus insisting, Croly thought himself ``first in
this~country.'' ``A correct theory is the first step towards
improvement, by showing what we need and what we might accomplish.'' The
theory of journalism can be nothing short of the science of politics,
making the central principle in the light of which the facts are to be
organized. The newspaper is nothing by itself, being only the existing
organization of intelligence, or lack of it. The newspaper, at any given
date, simply reflects prevailing notions. To change it means the working
out of advanced methods of reading the social life. The advance can be
gained only~through the unification of the ideas swarming from the new
conditions of life. The ordering of these ideas and their application to
reporting as indicated, is to compel a change in the newspaper which can
only be compared to the advance of the printing press of to-day over the
old Washington hand press.~

Horace Greeley\footnote{{[}Horace Greeley (1811--1872) was a journalist,
  editor, and political figure. Best known for establishing the
  \emph{New York Tribune} in 1841 (the leading newspaper of the penny
  press era), he also advocated for the use of the telegraph by the
  press. Ford also discussed the work of Greeley and the penny press in
  the \emph{Draft of Action}.{]}} once said that the time was coming
when all matter for the newspapers would proceed from a single
institution. What Greeley did not see is that this one institution must
itself be the great central publishing business, handling all news, and,
working in relation with the leading paper at each news center of the
country, constituting the ultimate associated press.~

It is to be understood that the newspaper takes to itself the central
position in life. The separation between church and life---making the
lesion in the state---which has so perplexed the minds of men, is to
disappear. In this respect we revert to the Grecian type of citizenship,
the religious and civic merging in the one life of action. A new and
prolific unity is dawning in the birth of the Organic State here
foreshadowed.~


% ORGANIZATION OF INTELLIGENCE REQUIRES AN ORGANISM
\chapter[Organization of Intelligence Requires an Organism]{Organization of Intelligence Requires an Organism}
\label{ch:Organization of Intelligence Requires an Organism}
\chaptermark{ORGANIZATION OF INTELLIGENCE REQUIRES AN ORGANISM}

\vspace{.2in}

\begin{LARGE}
    
\smallcaps{John Dewey}\marginnote{Memorandum - John Dewey to Henry Carter Adams, April 29, 1889.}

\end{LARGE}

\vspace{0.5in}


    

\hypertarget{memorandum}{%
\section{MEMORANDUM}\label{memorandum}}

\begin{enumerate}


    \item Newspapers
 should inquire into and report the actual state of things
and this scientifically and systematically i.e. journalism must be
organized.

    \item This is impossible unless there is an adequate physical basis. There
must be a machine equal to carrying it out. This is provided in
printing-press, locomotive and telegraphs. The latter, by eliminating
distance, make it possible to get outside of local interest and
ignorance and to report the whole thing i.e. to centralize the
intelligence of the country, and then to distribute it again.

    \item Inquiry cannot be organized unless it is somebody's business to
inquire - unless someone, that is, is making a living out of it. There
is an immense amount of inquiry now in the country --- economic
(illegible) --- government bureaus of statistics etc. but it arouses no
spontaneous or selfish interest and hence can't insure integrity. The
proposed organization of the newspaper will secure this.

a. It will be in the interest of every man in every business, say
cotton, to tell what he knows about cotton in exchange for what
everybody else knows.

b. The newspaper man will make use of this interest and charge for
collecting and distributing the information. This business interest in
inquiry and distribution of results will make the organization
automatic.

\item Organization of intelligence requires an organism; differentiation of
labor and corresponding centralization of differentiated parts. That is,
the state of things falls naturally into a number of subdivisions. Say
Food, Textiles, Mineral Products, Lumber, Distribution, etc. Each of
them again subdivides. There would be an organ for each subdivision
under Food---Wheat, Meat, Vegetables, and then a central journal
covering the whole field in its relations, and so on up to the top. The
beginnings of this already exist in the trade-journals.

    \item Intelligence or publicity is thus made a \emph{commodity}. The
newspaper now has something \emph{to sell}. The newspaper business is
the publicity business. This is the essence of the whole thing. The
newspaper thus gets a position---it has a function which defines it. Any
number of reforms over present newspapers grow out of this fact.

    \item That which finally touches everybody is the public
thing---politics---the state of the social organism. The newspaper, in
giving publicity to public matters (not for reform, or for any purpose
excepting that it is its business to sell facts) becomes the
representative of public interests. Thus Ford says the municipal
question is essentially a publicity question. No paper can afford now to
tell the truth about the actual conduct of the city's business. But have
a paper whose \emph{business} i.e. whose livelihood, was to sell
intelligence, and it couldn't afford to do anything else, any more than
any genuine business can afford to sell spurious goods.

\end{enumerate}


% IN SEARCH OF ABSOLUTE NEWS, SENSATION, AND UNITY
\chapter[In Search of Absolute News, Sensation, and Unity]{In Search of Absolute News, Sensation, and Unity}
\label{ch:In Search of Absolute News, Sensation, and Unity}
\chaptermark{IN SEARCH OF ABSOLUTE NEWS}

\vspace{.2in}

\begin{LARGE}

\smallcaps{Corydon Ford \& Franklin Ford}\marginnote{Chapter published in \emph{The Organic State}, 233–278. New York: Cabinet of News, 1897.}

\end{LARGE}

\vspace{0.5in}
\hypertarget{chapter-viii}{%
\section[CHAPTER VIII~]{\texorpdfstring{CHAPTER
VIII\footnote{{[}This chapter was published in \emph{The Organic
  State}, a book authored by Franklin Ford's brother, Corydon, in 1897.
  A note at the beginning of the book states that this chapter was
  originally written by Franklin Ford: ``The reporter would note on
  behalf of the Cabinet that the reduction of News, Chapter VIII, was
  originally worked out in principle by Franklin Ford during his
  editorship of \emph{Bradstreet's.}''{]}}}{CHAPTER VIII~}}\label{chapter-viii}}

\hypertarget{news-letters}{%
\subsection{\texorpdfstring{\emph{NEWS:
LETTERS}~}{NEWS: LETTERS~}}\label{news-letters}}

\emph{News answers to the demand for sensation as knowledge of order in
life}.\\\noindent The most ordinary view of this demand in the State is in the
questions, instinct with everybody:~``What is it? What is it for?'' What
a thing or a circumstance is, what it is for, involves the action of a
thing in its relations in life; and ordered news, or knowledge, about
anything is simply the statement of its phases of action in relation to
full action. Carrying this view out, the simple conclusion results that
the absolute news or sensation about anything is the absolute unity of
it, as in economic relations to the wider action which we know as the
State. The easiest and most concrete view of it, the easiest definition
of absolute news, comes thus to be the reporting of things according to
people's interests. This is the showing of how far a thing shapes itself
to what men want. In reporting a thing, giving the sensation about it,
Letters tells how far it approaches usefulness, or division of labor.~

In general, to report a thing as to the degree in which it approaches
order, we have to indicate its qualities relative to economic demand. To
report a thing as it is, it~has to~be measured by full usefulness, so
far as the latter may be known in experience. Letters is the practical
business of reporting things on this formula and cannot transcend what
is known, or actual. That is, Letters can only report what it knows. But
to be practical Letters\emph{~must}~report what it knows, must report a
thing in its completeness so far as experience has determined it. For
Letters to falsify experience in reporting life, by adding to or by
taking away, is for it to falsify itself. Letters is in so far not
Letters---having in so far vitiated its~place in life as action. We
should be able to find the reality of the report in any incident which
occurs---we need not care what.

There is a brand of mucilage containing fish glue now gaining way
into~more or less common~use so far as it comes to people's notice
through a limited distribution and publicity. There are certain facts
about this mucilage which may be given as making the main of its
absolute news. They all relate to its usefulness. The first is that,
once dry, it sticks better than the gum~arabic~mucilage.~Next, it is
lasting, less easily affected by weather; it does not soften in damp
like other brands of mucilage upon the market which have saccharine
quality and so tend to collect moisture after once being dried---it is
not a water collector, as the chemist says. In so far experience shows
this brand to be an advance; and in so~far~we have the absolute report
about it. Proceeding with our news, we inquire if the mucilage is
commercial in respect to the amount of labor involved in it. If it
cannot be put to the consumer at a price which warrants its general
consumption, then it is out of economic relation. Men cannot generally
use things if the amount of labor in them, represented by the price,
means exhaustion of time and effort so far as to disjoint life.
If~so~much labor has got to be put into the procurement and manufacture
of a mucilage, for instance, that men as a whole have to work excessive
hours to support it, or have to deny themselves considerable in food,
sleep, and clothing, then it does not have economic relation to the
whole as commerce and its limitations are accordingly set down. In this
the course of the report, the absolute news, would have to be that it is
lacking in the essential of the~unity of life---the practical statement
of this being that it is too expensive for use. On the other hand,
investigations into this phase of its unity, or usefulness, might show
that it saved at a very small cost of service a product that had
hitherto in part gone to waste. Inquiry might show that used alone or
forming the basis for~certain grades of mucilage it tended to cheapen
the whole product in the average cost, while at the same time, if we are
correct, going to help the quality. The unified report has therefore to
state that it saves labor and gives corresponding advantage in life---in
the practical language of trade, that it cheapens the cost of the
product. A fact adverse to its usefulness is its somewhat bad odor.
While this does not keep it from the writing table, it is against its
use at the point of greatest ease; many sensitive~nerves would be unable
to use it at all. In this count the report~has to~state that the product
is not in the last degree a useful one---simply that it lacks good odor.
Looking further, Letters inquires as to the unity of the distribution,
that is, whether handled at cost. Investigation here shows that, cheap
as it apparently is, it is selling at more than double its cost. The
bottle, with brush, costs half a cent; the mucilage itself costs less
than half a cent. Transportation and an organized counter should add
approximately a cent to this, making the total cost, say, two cents. In
the light of this news, people who buy the mucilage and pay five to ten
cents per bottle become aware of their status in life as affected by it,
or the division of labor it represents. They know how far they are
living under just conditions. The further development of the report is
that the State does not arrest the ``traders,'' explaining that pillage
by the so-called margin is a present condoned disorder. Touching this,
the report makes clear that if thieving be put out in one direction it
must tend to be put out in all directions. If the State organizes
against taking money in one direction without working for it, without
equal exchange of value, it must needs organize against it in other
directions. Letters~has to~show that the attendant conditions to the
unity of a bottle of mucilage at cost, two cents, is railroad fares at
one cent, or at cost, telegrams at approximately a cent, houses at fifty
cents a week, sugar at two cents a pound, etc. And this in one outlook
measurably covers the report on the bottle of mucilage.~

It must be gathered from this that News does not need~at every turn to
obtrude in its reports the principle, or tool, by which it interprets,
or gives sensation---rather, it for the most part conveys the principle,
as operative, or applied, by giving the good or bad qualities of a
thing, being its action viewed from every side. Thus, the simple
statement about the mucilage---that it is wearable, double price and of
bad odor---conveys to man its economics. At once, the sensation is
whether it fits or disagrees with their needs---the measure of unity.
The sensation about the murder case, again, is that the murderer is
found or that he is not. In one case it fits men\textquotesingle s needs
and in the other case it does not. Thus, the murder report carries on
its face the economic fact. In general, people understand murder in its
principle---that it is hurtful to society. This has been so effectively
raised up in cases that it has become common knowledge, or is common
instinct. To formalize the principle, or harm of it, would not be news.
It is only the new detail, the new murder, that is news, the public
applying the well-known principle themselves. In the case of the action
which is the margin of profit, men may not at first understand its
principle, its harm to society, until it is carried out in action. In
which case its harm is seen as readily as in the case of the murder.
Obviously, in any report, the fuller explanation is to show how a thing
acts in all its effects, which is to say, its relations; and is fuller
news.~~

The ample sensation, or unity, of the report is, evidently, its
qualification of men's wellbeing---that men~see in it on the one hand
the way to improve their conditions, or on the other hand, find that
they have attained to~well being. In other words, the realization of the
report shows the way to be free, or, it tells that freedom is attained.~

The full sensation, or unity, of the above report is that it carries
the~\emph{way~}to freedom for men. They see that transportation at cost
will go a long way to help mucilage at cost.~The most casual mind gets
the inference of moving to aid government ownership of
railroads---the\emph{~way~}to freedom.~

To gain the other aspect of the full realization in report, we may
suppose that the development in life has gone on until investigation
shows that the mucilage is the best that man thinks to produce; and that
on the side of its distribution it is passing over the counter in that
further quality of relation which is strictly cost. The fuller unity, or
sensation, of the report is, in this view, the knowledge of the full
usefulness of the mucilage, of man's perfect relation to it---that, as
affected by it, he has reached the point of greatest ease. The sensation
is, in fine, to give man the consciousness that he is free.~

To extend the illustration of the unified news, we may suppose that
people want to know the reality of~the man of leisure. The report must
measure him by the absolute relation in life---division of labor. And,
in the practical, we know how far he lessens or increases the bondage of
men if we see him in the concrete relation which is the price of any
article of commerce. We may take the man of leisure who is loafing on
the accumulated margins lifted from trade by his ancestor. Measured by
the absolute, the man is holding other men in bondage in so far as he
does not perform his proportionate share of work. The amount in which he
enslaves another man is a matter of ready figures; for, if there are two
men out of five idle then the three men who work must increase their
labor two-fifths and the average of bondage placed upon men is forty per
cent,~as over the normal claim of the State upon them. The man who need
only work four or five hours a day on a normal footing is bound to work
eight or ten under the conditions which do not equalize labor. The
practical form which this bondage takes under present organization is
the increased cost of products, necessary to support the fiction of
interest or other margin in the employment of so-called capital, the
device of the man of leisure for taxing a certain number of men who are
from two-fifths to one-half his slaves. To pay the fiction of ``profit
on capital'' by which this man lives in leisure we have the increased
price on commodities. And the man who pays a multiple advance on his
transportation or on his clothes or house or mucilage, may trace a large
portion of it as the measure of his bondage to the~disorganic~man. Like
the other, the sensation of the unified report regarding the man of
leisure is that it points the way to freedom for men. It points the way
how to get rid of the slave-driver who, in the guise of phrase, is the
man of a fortune. We, of course, get rid of him in the ultimate and
absolute by organizing the Exchange so that he is unable to tax anyone
beyond his exact division of labor, raising him into the common fortunes
of men. We know that an advance toward this is organization in any
direction subjecting the individual to the control of the Whole.~

A phase of life which under a unified report results in a reduction like
that of the man of leisure, is the man who through spoliation by margins
spends his money under conditions of non-production. To take extreme
example, we have the man of large ``income'' in some cases supporting a
stable of fifty racing and carriage horses and keeping in one or another
capacity a score of men in their care, that is, in more immediate
support of this one man's part in the extravagant play which culminates
in the race track or the fox hunt. Allowing for the developing of the
higher qualities in horseflesh, which it is claimed is the product of
these stables, they must in ninety per cent of the labor employed be
found wholly unproductive. If we consider the large element given to the
so-called sporting institution which centers in the play mentioned, we
have essentially several hundreds of practically idle men involved in
the conditions of one man's prehension by margin. This feature of life
is multiplied in all the discussions of the non-productive employment of
money. We find it in the yacht of the millionaire, taking the labor in
many instances, directly or indirectly, of a hundred men. It is said
that Mr. Vanderbilt's yacht carries fifty seamen besides his chef and
retinue of servants. And there are the princely estates rising in
various directions over the country. Mr.~Rockefeller, a little above New
York on the Hudson, has taken out of production a thousand acres,
answering in so far to the baronial estates of the older feudalism. The
newspapers report that Mr. Rockefeller is putting into the support of
his palace and grounds something like two million days of labor at the
outset, figuring it on the average price of labor, which at the present
time approximates a dollar. We have another phase in the appropriation
of transportation to peculiar private uses. One of the Gould family
recently made up a party in a special train for a run across the
continent.\footnote{{[}Jay Gould (1836--1892) was a railroad magnate who
  was known to design luxurious train cars for his private use.{]}} With
the labor drawn upon in the non-productive employment of building a
dozen private coaches and the labor of the vineyards for champagne, in
amounts corresponding to several thousand days, and the use of railroad
service and its army of men, we have, in a realization of the one
reduction, the conditions of alarming disunity in the commonwealth, a
frightful dismemberment of democracy. These conditions of non-union and
disloyalty in~the State carried to full computation make for practical
idleness, for non-production, of from one-third to one-half its members,
as sustaining no unified relation in life. The man who finds himself a
slave to ten hours of labor, when five should be his limit, sees his
bondage directly in these conditions of betrayal of the State. The claim
that capital in its forms of non-productive use is beneficial to the
laboring man has, in the unified report, to be revised to read that it
is of doubtful benefit to laboring men to have disorder sack them of the
product of their work in order that they may be held to the slavery of
double time and all the conditions of stringency which come as the
outgrowth of disunity in the social. The organic news regarding the man
of leisure and the kindred reductions is, then, that they increase
prices, starve our children, double our day's~work~and breed disloyalty
and anarchy.~

As phase of demand in the State men want to know the unity of
breath-\\\noindent ing---What it is?---What it is for? Letters has long worked at the
drawing up of this report. It was very much helped by the chemist
Priestly,~who made record of his experience regarding the phenomenon of
combustion. He found that there was a specific element in the air which
we know as oxygen, and that combustion was the process of the chemical
union of this oxygen with elements in substances like wood, that burn.
It is now known that the different forms of erosion or decay, like rust
and wood-rot, under exposure of substances to air and moisture, are
different forms of this combustion, or oxidation. Along with this
experience, or discovery, it was found that breathing is the process of
conveying air to the blood for the combustion of the body. The blood,
curtained from the air in the lungs by scarcely tangible membrane, gives
up in carbonic acid certain burned debris of the body, and takes on
oxygen for conveyance to all parts of the body in the continued life
process. The heat and health of the body it was found had its dependence
upon the combustion which resulted from oxygen supplied by breathing. In
this first cast of the unity of the report we see man's breathing
connected widely with all process of life and change. In its further
reduction, as coming up through this, we~have to~describe a phase of its
unity as having its practical effect in the organization and liberation
of man. The first effect, as so far described, of course makes for the
liberty of man in his mental enlargement and order. But there is a
reduction which directly shapes his material surroundings and in added
effect frees him. And, while the unity of knowledge in relation to man's
experience with oxygen builds for the material organization in chemistry
and other lines of~trade,~we may keep to that effect of it which
directly touches the point of view with which we started, namely,
breathing as life process of men. Through knowledge here the
architect~is able to~organize his phase of life with relation to the
number of cubic feet of air which a room must supply to a person. This
question is reduced to a simple mathematical problem, so that neither
the architect nor the State at large are longer hampered by conditions
of ignorance respecting it. Thus, in two aspects we have~gathered the
reduction of breathing into its unity, or news---into its qualification
of freedom, the highest sensation for man.~

Demand in the State could ask to have measured the unphonetic spelling
as taught in the schools. We have in the first instance of the reduction
to see that the unphonetic spelling violates the whole organization of
the school. Education has its basis of order in the logical process of
mind. The unphonetic spelling is strictly a violation of this. It may be
said that teaching the unphonetic spelling is the induction of the child
into illogical method, which the attendant machinery and more advanced
work of the school attempts to reverse. But there remains the lamentable
condition that the illogical process is ever with the child, working
indirection and discomfiture of mind. In further reduction, we have the
enormous waste to child and teacher of this lesion of the system. It is
probably an under-estimate to say that one-third of the time of child
and teacher is wasted on the illogical method of the word---not to go
beyond and into the sentence. It is hardly a contradiction of this to
bring the claim that the illogical spelling has in it the logic of
preserving derivation; this is preserved for those who have need of it
by the very simple process of the~book of reference.~

There has been a demand in the State, wellnigh universal with men from
the beginning of knowledge, to know the unity of mind and matter---of
the ``spirit and the flesh.'' The unity of such report has waited upon
experience in the region of psychology, as also of physics. The need has
been to get the relation or dependence of mind upon matter, and vice
versa. We have seen a phase of the development of this report rising
through Hegel,~Caird, and other students as men of Letters. And latterly
we have had its reduction to stable and more practical unity through
advancing the language of mind to motion. As reducing to commerce, we
may gain its final aspect of unity. This is the waste it saves. It
disposes of the church as a distinct institution in life, freeing a
misapplied energy for whatever useful function it may find. It disposes
of the church because the ``spirit''~and~``morals''~which the church
essayed to treat attains rational explanation, beyond any call upon the
pulpit. In so far as the preacher has adaptability for explaining the
unity of life he reduces to the man of Letters. Where he has not the
capacity of rational report, he is distributed into other callings.~

\vspace{0.05in}

\emph{News organizes on the divisions of labor representing the
different fields of technical knowledge~entering into~the complete
report}. A while ago dispatches appeared in some of the leading Eastern
newspapers, the \emph{New York Herald} among others, depicting a case of
partial asphyxiation of a carload of people through the accidental
inhaling of ether. The circumstances according to the dispatches were
that a surgeon had boarded a train at Syracuse, on the New York Central,
on his way to a neighboring town to perform an operation. It was said
that the can of ether which he carried was uncorked by the jolting of
the train and within a few seconds all the passengers in the car were
close upon suffocation, being only saved from speedy death by the
fortunate appearance of the brakeman, who came to the door to cry his
station. The account stated that but for this timely arrival there would
have been a carload of dead passengers. The brakeman was represented as
assisting resuscitation by opening car windows and fanning passengers. A
multitude of people who read the report credited implicitly the
statements which it contained. A newspaper editor went so far as to
comment on the carelessness of the surgeon in carrying a can of ether
liable to become uncorked. The lack of unity, or truthfulness, in the
report as it went through the news channels is traceable to the fact
that there was no division of labor in its treatment by a man having
technical knowledge of medicine and the well-known conditions of
anaesthetization. The facts in the case related are, in the first place,
that ether is usually carried about by surgeons in close soldered cans,
without corks. The further facts are that to etherize a patient a
tolerably~close-fitting cap~has to~be shut down on his face and the
fluid continuously dripped upon it. In most cases two or three strong
persons are required to hold the subject while he is coming under the
influence of the drug. The doctor considers himself fortunate if he
is~not longer~than ten or fifteen minutes in bringing his patient under.
Had the report in its movement through News been referred to the merest
novice in technical knowledge of medicine it would have been stopped.~

\newpage The present post office building in Chicago, which is generally known to
be going to pieces, and recently pronounced unsafe, is a witness to the
lack of technical division of labor in reporting. When this structure
was erecting in the middle seventies there was a great deal said by the
Chicago newspapers and by dispatches over the country about the
inadequacy of the foundations. But it took the form of indefinite
general statements and surmise, removed from qualified evidence
sustaining the charge. No sufficient and incontrovertible facts were put
out. Added to this, some newspapers contradicted the charge of
unsubstantial work on the building. These latter statements were
likewise short of adequate particulars. In the confusion, the general
public was left without any understanding of the matter.~Mr.~Mullett,
the architect, went on with the job to meet the after deplorable facts
in the case. The need of publicity was for technical knowledge of the
actual conditions of the fraud going on in the foundation ditches. At
the time~there were in the vicinity easily some score of competent
engineers and architects who could have told exactly the fault with the
first stone of the foundations and of every subsequent item in the
building. Organization~which would have united news in employment of one
or more of these men in constant watchfulness of construction to report
defect in its smallest item, would have enabled the public to fix the
responsibility for the offence. In the aftermath we of course have
knowledge of the failure of the building and the loss of its million
days' labor, but too late for efficient placing of blame. ~

So far as development has carried it, News now acts on the recognition
of the principle. It is the prevailing practice at newspaper offices to
turn on to reports men who have a special or technical knowledge fitting
the requirements. If a man is to be sent out for a report on some
happening in water circles, the managing editor takes a reporter who has
shown a readiness in these things. It often develops that a report of
this kind requires the labor of several men, each having adaptability of
knowledge for one of its several phases. Though the newspaper offices
are not always organized for efficiency on this principle they work to
it as much as they can. A ship one time arrived in the harbor of New
York with most of its crew dead or prostrate from scurvy. A leading New
York newspaper in making up its report turned in what it called its
marine reporter, as man in charge of the case. This reporter was
instructed to call in any other available men on the paper. Before the
news was supplied there had been brought into the work a
medical~attache~of the paper and a marine lawyer. In supplementary news
on the matter, physicians and marine lawyers were interviewed, their
statements being in effect a division of labor in the news. The medical
reporter on the paper and the marine lawyer substantially organized this
part of the report. The result was a measurably unified presentation of
the news of the plagued ship. The outcome found the captain essentially
convicted of man-slaughter in failure to furnish the proper ingredient
of vegetable in the ship's food.~

We see the recognition by the public of this principle of division of
labor, in its demand upon the newspaper for technical news. A man
recently wrote to one of the New York dailies asking if they wouldn't
find for the public the truth or falsity of the talk about beer being
adulterated. He wanted to be able to buy a glass of some brand of beer
with a knowledge of its purity. The correspondent asked the newspaper to
have the several brands analyzed and reported upon. He asked also that
experts in knowledge of beer-making be put to watch the whole process of
manufacture from day to day and report the breweries making good beer.
It was suggested that manufacturers who wouldn't allow close inspection
of this kind would prima facie stand before the public as adulterers of
beer. It was submitted by the correspondent that the money which the
paper paid to its editors for talking around the facts of life could be
employed in~actually reporting~a few things like this. There has been
the same demand for knowledge as to the adulteration of other foods. And
people have recently asked the newspapers to ascertain through expert
navigators and engineers the truth of the statement that ocean
steamships were running at a dangerously high rate of speed. And the
demand, of course,~more or less universally~exists to know the actual
cost at the present time of a glass of beer, a loaf of bread, a pound of
nails, a watch, a bicycle, a ride to Liverpool or San Francisco, a coat,
a bonnet---both from the standpoint of the profit margin and of the
exchange at cost. People would like to know the actual cost on these
lines of everything they buy. In short, they demand the News in all
fields.~

The U.S. Census Report, though nominally apart from the regular
organization of news, is~in reality a~phase of it. In making this report
the government may be regarded as engaged in supplementing deficiencies
of newspaper organization. The parallel is exactly as though the
railroads were deficient in fuel and cars and the government voted them
gratuitous supplies~helping out~the lack. We see marked division of
labor on technical lines in this report. A Commissioner of Education
standing for the~Educational Report, gets out his specific part of this
news.~Similarly~we have the Agricultural Report and the Fisheries. And
there is the report of the Commissioner of Labor. The state bureaus of
statistics issuing their technical reports are likewise essentially a
division of News.~

The consciousness of the newspapers as to the principle has some
indication in the way in which they blazon their more specialized
reports. They put upon the front of their paper in their bold type
announcements of these reports, emphasizing technical phases of the news
which~their enterprise has secured. That they are partly conscious of
need of organization under the principle seems also apparent in the
specific praise which they take to themselves for rising to technical
division of labor in such cases as may chance.~

\vspace{0.05in}

\emph{News has its organ of centralization, or head, in a Cabinet of
Intelligence composed of men representing technical knowledge of the
various fields of industry; this is the method of the fuller report and
provides representation of the interests of each Class in the State}. In
looking for the working centralization of Intelligence, we sound
publicity for phase of outgiving which shapes the utterance of News in
its body as a whole. It is to be found in any direction yielding some
principle which shapes news, whether occurring in newspapers, so-called
books or upon the platform. News does not get its innate character from
its mouthpiece, but from the nature of the utterance. We look at the
publicity in the late political canvass.~

The St. Louis Convention of 1896---the National Republican Party---was
marked by the inertia of old forms of social forces which persist until
the issue is clearly drawn against them as element obstructive to
advance. This convention marked the cleavage in the conscious action of
the nation between the radically obstructive elements and the forming
order. The adduction which brought it together had nothing of the
gravity of a principle striving to shape the out moving forces. It was
rather that amazing chemical process of the State in which the old base,
becoming weakened in its hold upon the economic life, is attacked by the
element of fixation. The resulting immobility is the precipitate
separating itself from the fluid of social action. The destiny of the
failing thing is a new resolution of being, but through decay in its
isolation from life. The weight of utterance delivered in the Convention
was the~moulding~into doctrine of a past nodding to its fall. When the
mallet rapped the Convention to order, the nation was fretting in
disunion; America was a disunited commerce. The center of the nation,
New York, had severed the provinces from itself by phenomenal levy of
transportation and other tolls under the margin concept. Communication
between all parts of the country had similarly been cut. Exchange,
having its center in the New York Clearing House, was bleeding the
extremities of the country into the East. There was the disunion of
reciprocal flow in exchange between the center and the provinces that is
marked by the hot head and cold feet. The reform looking to restoration
of loyalty between the sections could hardly lie in an application of
the leech of a dishonest Bourse. It could not lie in the dogmatic
retention of methods which had passed their life and become irritant.
The remedy lay in the removal in some degree of the levy upon commerce,
and the recognition of the organizing industries. The reform was in part
in so simple a thing as transportation and communication at cost through
the proposition to government ownership of railroads and telegraphs. And
in another~direction~it was so simple a thing as the proposition looking
to arbitration in industrial spasms. But this was the abdication of all
that the Convention was. It was the resignation of the king. The
Convention was organized by the constabulary of the disloyal forces that
had dismembered the nation. It was officered by the instruments of the
private interest and it attended upon the king's extremities. Advised by
the mediaeval chiefs it recast into inflexible assertion measures which
had come to be ruinous to the people. The Convention in the sum of its
action had come to accept and enact into design the growing breach in
exchange, having its effect in ``making the rich richer and the poor
poorer'' and discarding every~principle tending to resolve them into the
fraternity of equal exchange and freed commerce. The body of the
convention was brought from the ward tricksters in every locality---men
with a reputation among their neighbors for extreme~partizanship
{[}\emph{sic}{]}~or for sale of political service. The mediaeval secrecy
cloaked the proceedings. Bribes whispered in the lobbies.~

In the Chicago Convention\footnote{{[}The Democratic National
  Convention, held in Chicago in the summer or 1896, nominated William
  Jennings Bryan as the Democratic Party candidate.{]}} of the
succeeding days of the summer, we have the contrast of a body gathered
in openness and freedom to abjure the decadent concepts; they had met to
throw over the obstructive measures. The next thing to knowing positive
conditions of order is to arrive at certitude of the conditions of
disorder. It may be said that the Chicago Convention had attained to a
considerable measure of the latter. By whatever words it was put, it
nevertheless is true that the men of the Chicago Convention had
discovered the inequality and fracture of the State bred of the fiction
of gold as instrument of exchange---bred of the disorder of the attempt
to raise gold out of its function of common commodity. These interests
had got so far as to discern the unstable equilibrium of gold in this
attempt to force its function. They had rightly made out that this
fictitious use had given it fictitious value. They pointed to the
overweighting of the significance of gold as forcing its index up until
the payment of a farm mortgage meant dispossession to the farmer. It
required little~presience {[}\emph{sic}{]}~for the men of the Convention
to know this, for in one way or another they were the direct and so
conscious victims of the system of an unequal exchange, a phase of which
they could understand as exaggerated by the fiction of gold. But their
consciousness was only as the victims. They did not grasp the fault of
the evil looked at in the fuller method of its correction, they did not
get through to the absolute principle of money---the simple notion of
the bill of exchange as record of the transaction, based on the unit of
a day's labor. Their consciousness was the reaction from the irritation
of the single gold standard, the reaction from the single commodity as
measure of exchange. The proposition to inject silver into the money
situation they clearly saw as lessening the fiction of the one
commodity, as standard. As they caught it, they understood that they
were coming to increase money. But the reality was in so~far~a movement
toward the reduction of gold to common level of all commodity. The
reality of their proposition favoring~the coinage of silver was the
freeing of this metal which rested in the falsity of regarding only one
commodity, gold, as measurement of values, whereas all commodities
should have equal place. Though they did not get the principle, they
moved in its direction---the reduction of all commodity to the same
equal plane of exchange.~On the whole, the rationality of their action
was the determination to try this much of their discernment. It would
not be well, like the St. Louis Convention, to enact into phrase the
conditions of their own evil; having freer agency than that body, they
could try something that had not yet been tried. The magnificence of
their action was their proposal to try some new thing, having to them
measure of promise, to determine it by its results. The greatness of the
proceeding at Chicago, in contrast with St.~Louis, was the turning upon
vicious conditions, and the movement out into the open of a new venture
in the State. It was the rise of the people from degradation, trusting a
new freedom.\footnote{{[}Historian Charles A. Beard later presented the
  1896 conventions as turning points in American history. According to
  Beard, ``Deep underlying class feeling found its expression in the
  conventions of both parties and particularly the democrats, and forced
  upon the attention of the country in a dramatic manner a conflict
  between great wealth and the lower middle and working classes
  {[}\ldots{]} The sectional or vertical cleavage in American politics
  was definitely cut by new lines running horizontally through
  society''. Charles A. Beard, \emph{Contemporary American History} (New
  York: McMillan, 1914), 64.{]}}~

It appears not improbable that the out-movement toward the equable
exchange~has to~be the breaking of the fiction of the gold standard by
the free injection of another commodity, like silver. This fiction gone,
there could be the wholesome resolution toward unification of exchange
on the rational basis of the labor standard. The banker, torn by the
currents, would not unlikely be found ready to move in organization on
the legitimate lines. It seems certain that if silver can be injected to
diversify and complicate the present fictions of exchange, that the
banker will have no recourse left but to follow the one proposition
having promise. He can, apparently, find no other outlet, for it will be
said by the people that both gold and silver have been tried, and the
remedy is not in these.~

We have thus discerned the Chicago Convention in line with the gathering
forces, and whatever else~may be said, that the Convention marks a line
which the people have crossed. While, commensurate, we~have to~expect
that a like body of 1900, whatever name it may take, will advance the
issue to transportation at cost and the referendum in government, on
which we now see the flood of favoring utterance.~

The morning after the action at Chicago, upon silver and what the baron
names the as allied heresies of the proletariat, the \emph{New
York~Sun}, in the most trenchant of leaders sprang instantly to
the~defence~of the feudal system. Thirty years before like writers had
called upon patriotic men, in newspapers the counterpart of
the~\emph{Sun}, to come to the defense of property vested in the African
slaves. In the dread days of the upheaval~preceeding
{[}\emph{sic}{]}~and throughout the Civil War, such writers were all
that Mr. Lincoln was not. It is the brutality which stands guard by old
issues when the ideas of men are on the rack of birth. In~1860 these
mediaeval warders gave the instant cue to a part of the press of the
country, which followed them in the spleen of decadence.~So~in 1896 they
gave the instant cue of the meaning of the Chicago Convention to the
sectional press. The~\emph{Sun~}leader was the next morning copied
entire by the \emph{New York}~\emph{Herald}, with the announcement that
it was their guide for the canvass. Throughout the country the sectional
papers lined up on the issue made by the~\emph{Sun}~editorial---the
short of which was that the Democratic party had passed into socialism
beyond recall and was already inviting the abrogation of private
property.~With the exception of~the~\emph{Journal,}~the whole New York
press took this cue to the situation and denounced everything as
unpatriotic that was not mediaeval. Men turned anxiously to
the~\emph{Journal}~for some utterance at the center that should be
nearer the commonalty. With measure that was dramatic in its
lonesomeness, the \emph{Journal} accepted the full cause, taking up the
battle against the monied intrenchment. In contrast with the venom of a
shoaling decay, the~\emph{Journal}~in much had the pleasantry of sight
and kindness of sentence that marks resource of principle. The extremity
of the opposing press was detected in its lack of any humor in
discussing their cause. They had reached the tension in the hatred of
the Idea that gave them place among the enemies of men. The \emph{New
York}~\emph{Journal}, previously little known, rapidly passed the half
million mark in its circulation. A list of papers throughout the
country, more especially the provincial weeklies---less under the blight
of Lombard Street---followed the\emph{~Journal}~in its leading
utterances, throwing themselves with the struggling cause.~

Underneath the drama of this we discern a centralized weighing of
conditions in the Nation,~more or less in~unity with advance. The action
of the \emph{New York}~\emph{Journal}~may stand for this qualification
of the machine of Letters. Seen as direction of the publicity of the
country it virtually rises into the head, or centralization, of News.
Before the\emph{~Journal}~wrote its leader in support of the Chicago
Convention---the new consciousness in democracy---it weighed the
opinions of the leading writers and other publicists of the time. The
result was the outgiving of what it considered the weight of principle
after various views had been canvassed. The reality of this is that
the~\emph{Journal}~essentially gave utterance to a majority vote of a
Cabinet meeting of leading publicists in democracy. We know that
the~\emph{Journal~}consulted the views of men competent to speak of the
interests of various lines of business. The consulting of these
interests, in a conference of men in which they were widely canvassed,
furnishes the make-up of the Cabinet of News in divisions of labor
standing for the various industries, or Classes. The action affords in
effect a Cabinet composed of men having technical knowledge of the
different lines of industry in the State.~

The outgiving under the lead of the New York~\emph{Sun}~gets its reality
as schism in the Cabinet. This conflict in publicity must be seen as the
expression of disorganization in the News Class as a body and in the
Cabinet as directing head. Organization of the News Cabinet presupposes
that it is organized for singleness of action. This means action on a
majority vote of the Cabinet. It has its model in the organization of
the United States Supreme Court. The latter body is the head of the
Judiciary Class in~America and its majority vote makes the rule of
action on questions of principle for the whole judiciary~body of the
country, otherwise there is no organization and no action in the
Judiciary Class and the administration of justice falls into chaos.
Similarly, under degree of efficient organization, the majority vote of
the Cabinet of News would have to determine the paramount utterance of
the News body, going to control the whole machinery of publicity.~The
minority report of the Cabinet could only be considered in the light of
distracting utterance, not to be regarded as the action of News. The
utterance of the whole press of the country, as controlled by the head,
would be a unit with the majority report of the Cabinet. The minority
report would not be considered a matter of general news and would only
be held in pamphlet at the service of some demand should it appear. In
the outlook it is thus apparent that the wrangling of Newsmen, if
existent at all, would be as confined to themselves in the working out
of utterance, in the formal discussion by a body under rules of stable
proceeding. The Nation would not be disturbed by partial and immature
utterance.~

We thus get the unified movement of publicity through determination of
its line of utterance by the News Cabinet, as at once the~centre~of News
and the cerebrum of democracy.~

\vspace{0.05in}

\emph{The working head of the Cabinet and of News}. If the parallel with
the higher judiciary be carried further, we see the necessity of an
organizing head for the News Cabinet and for News in general,
corresponding to the Chief Justice of the National Supreme Court. As
regards this Court the Chief Justice may be said virtually to direct, or
organize, its action. He may be considered as making its rules of
proceeding, formulating its outgivings, etc., but by advice and consent
of~the majority of~the court. In essentials the members of the court can
overrule him by a majority vote. The Chief Justice may be regarded as
going forward of his own motion in all matters, making the responsible
head of the Judicial~business in America, though he acts~with regard
to~the will of the Judicial Cabinet, and, further, with regard to
criticism by the entire judicial body. It is on the same plan that the
News Cabinet organizes. It has its president who is virtually directing
head of News for both the Cabinet and the whole News body. The president
of the Cabinet is president of News, or News General. He makes the
general rules of his Class, formulates principles, indorses
{[}\emph{sic}{]} important outgivings, directs the Cabinet. But he does
this~with regard to~the will of the latter and of the News body, who may
each overrule him. Any action of his permitted by the Cabinet is in
effect the action of the Cabinet. In new and important~matters~he calls
his meeting of Cabinet or otherwise sounds it; but moves independently
in all matters, making the responsible head of News.~

It is~proposable~that there are a limited number of ex-officio members
of the Cabinet drawn from the News body by the different Cabinet
members, who may need to have advisers near them, forming altogether the
philosophic body of democracy.~

\vspace{0.05in}

\emph{The organization of the Newspaper makes for action in the State}.
As a result of not carrying to the full the principle of organized
action upon which they themselves partly work, as a result of the
partial relation of ideas through failure to attain a common head and
set all phases of life into proportion, the Newspapers put out
incomplete reports leading to confusion and conflict of the people. And
in turn the newspaper body has slow test in the action of the State as
to the truth of its utterance. The illustration of this we have seen in
the political canvass. Part of the papers speak for one action and
another part speak for conflicting action. Whereas if the papers through
a central body or head would agree upon one line of utterance the people
could get together in their action and try the result of a given idea.
The probable desirability of the extension of the mail service to
include the telegraph is in the air and half sought by the people. If
the newspapers could harmonize their utterance and say it is thought
best to try this action the people would get together and issue their
warrant, and then it would be known certainly whether it is a good
thing. Or a central newspaper Cabinet might determine that the newspaper
body should speak in favor of trying the Government ownership of
railroads so that the people could get together on this point and try
it. The~Cabinet would not be omniscient any more than are men, but it
would be efficient in the sense that a body of men represents more
experience than a single person. Knowledge is the result of action. And
it cannot be known beforehand exactly what the result of new action will
be. But there is the reasonable view when things are not right that
certain remedial action should be tried. If the Cabinet and the people
try one considerate view and find it~wrong~they will then say another
way should be right, and they will try that way. The great trouble now
is that the people are stopped in their action through the disorganized
publicity and can only through great turmoil and suffering get their
slow direction. A poll of the people now may mean little more than that
a majority are confused. With the clarified news, they could as a body
more certainly get action in some direction where they now stand still
in tedious struggle. It could have been the utterance~of the press in
1896 to let the people get together and try some way out of their
trouble. They could come to determine whether silver is the corrective
or not by the~newspapers saying ``We may try it, since a gold standard
has been tried and a large body of the people are not satisfied.'' The
Cabinet might not be ready to say that silver is the absolute remedy but
it should be prepared with a single voice to say, getting nearest to the
popular discontent, that we may try silver and then we will know whether
it is a good thing; and if it does not~work~we can try something else.
It could say this, or with singleness propose any seemingly
determinative course, throwing itself upon the test of action. But with
the principles of State once worked out in action, a Cabinet of the
Organized Intelligence may as often go right in public~polity~as a body
of United States Engineers in determining the foundation of a bridge, or
the structure of a lighthouse.~

\emph{The integrity or faithfulness of the Cabinet of Intelligence}. The
element which creates an integrity in men is responsibility in action.
The direct result of organization is the fixing of responsibility. A
man~has to~be responsible for his part. The force of it as to the action
of men is seen in some railroad accident where the aim of investigation
is to get at the responsible party, with men endeavoring to shift the
blame upon their fellows.~So~it is everywhere. In the recent
investigation into abuses in the government of the city of New York, the
effort was to locate the man responsible for it. He was not found owing
to the slack organization of the municipal government. The well-known
integrity of the United States engineers is laid in the fact that if a
man puts a channel wrong or builds a lighthouse insecure, he~has
to~stand the consequences. A United States engineer is put in charge of
a division or a piece of work and is~absolutely~answerable~for the
conduct of it. And a general probity goes along with the trueness of
mind acquired in the exactions of responsible art. There have been few
cases of financial corruption in the United States engineering service
for the last fifty years, extending over the disorder and looseness of
the Civil War. Able men of every calling as a rule possess the general
probity of mind. It is the requirement of the action of mind that it
cannot be half false and half true as a practice; it must work
to~trueness as a whole, or~fail of its corruption. The News Cabinet
would have the undivided~responsibility~for action-making utterance and
upon~it~error would return. It would be made to feel the same weight
that the train dispatcher has in bringing his trains through. And in the
case of the Statesman-philosopher, central to life in the Cabinet of
Intelligence, a more profound responsibility might be thought to accrue.
After all is said, the simplest view of the principle is that
organization makes a man responsible for his own disorder. Men commonly
recognize this as the method of compelling integrity. The everyday
illustration of this is the practice of some men of paying a doctor by
the year to keep them well, or paying a mechanic by the year to keep a
given machine in repair. This is simply so much organization of the
doctor or mechanic against results, as responsibility.~

With a Cabinet conscious of exact relations of men and measuring in its
make-up the phases of the life of the nation, utterance must be more
removed from the factional prejudice. The Civil War in America~was the
result of the divided ideas of democracy; and these ideas might be
counted as nearer whole in the years verging upon~1860 if the
responsibility for the utterance of the newspapers of the time on the
question of the mending of the broken Nation could have been put upon a
central body of men with integrate leaning. In the present crisis one
section of the press speaks for the bankers as against the farmer,
applying sophistries in its presentation; and another section of the
press speaks for the farmers against the bankers finding no logical
common ground of interest. The members of the News Cabinet could but be
driven to surmount such schism in their body, striving for the resultant
view until they got a majority utterance of the common ground of
interest for all factions. This must tend to the organic.~

\vspace{0.05in}

\emph{The Cabinet of News gets its practice in dissecting or organizing
the General News movement, revealing the organization at the Center}.
The General News is news that is not restricted in its interest to
any~particular Class, persons, or locality, but, on the contrary, is of
interest to everybody in every place. This news coming through the
regional heads to the center is determined as to its character and
make-up. The offices at the center are organized for handling General
News of every kind. Each specialization or department of General News,
as chemistry or farming,~is in charge of~a member of the Cabinet
answering to such particular side of knowledge. This man organizes his
department in its various phases for handling his specialty in news.
General farming news, for instance, coming in, is put upon the hooks of
this department, or dissection, of news. But at the same~time~it is hung
upon the hooks of all the other departments for any suggestions or
modifications they may offer. These modifications are considered and
incorporated into the report, or on the other hand excluded, by the desk
having charge of Farm news. The suggestion of this central organization
is in the present action of the newspaper offices at the large centers
which refer given news to an editor best competent to handle it, who in
turn asks other editors or reporters to work it over for suggestions.~

This movement illustrates the working organization of the News Class
through a head. As indicated, the organization of the Cabinet in this
working movement~is in charge of~a president as managing head, who is
necessarily responsible for the organization at the center and in turn
of the news movement of the entire country---much as the
Postmaster-General is responsible for the immediate organization of the
Mails center at Washington and in turn the whole postal movement of the
country.~

\vspace{0.05in}

\emph{The Regional head, or center of the territorial division, of the
News Class is provided in the dissecting newsman and his organized
office at the commercial center of each Region}. The news of Michigan,
for instance, as one of forty or more Regions, comes to a point like
Detroit, or wherever the wires of commerce center for such a division.
The part of this news which the Michigan head would see had interest
beyond Michigan is sent on, reflected, to the National center and the
part of the news having a purely Michigan interest is made up in its
bearings and sent back, reflected, to the local papers of the state. We
have in this the reflex movement of the intelligence of the locality
back upon the entire area of the Region and the reflex on to the
cerebral center, the Cabinet of Intelligence at New York, or wherever it
is located.~

\vspace{0.05in}

\emph{The lesser area, as unit of the organized intelligence subordinate
to the Region, is the news movement of the~County, or precinct, to its
center}. A News agent as this subordinate center, with his organized
office, dissects the purely County, or precinct, news from the larger
movement, which latter he passes on to the~Regional~head at a point like
Detroit. The purely County news goes no further than the County head,
being~reflected back~in the classified local paper, or County bulletin.
This comprehends the lesser reflex movement of intelligence.~

\vspace{0.05in}

\emph{The organ of intelligence which furnishes the immediate contact
with life on the side of news gathering is the reporter, or news-man,
who is intimate with the movement of things in the restricted sense~of
the village, the country township, or the city ward}. He gives
expression to the contact, or movement, of his circumscribed locality.
He stands for the peripheral nerve-ending of democracy. In news
extraordinary, as requiring exhaustive and technical reporting, this man
is helped by reporters sent down from the County head; and for some of
these, in extra instance, the County head may draw upon
the~Regional~head, as at Detroit. The peripheral man is helped out,
supplemented, by the~Regional~head, as the external nerve-ending in the
animal body is helped out by the next higher sense organs. The
peripheral man is a single element in a unit news movement of which the
higher centers are the fuller divisions of labor.~

\vspace{0.05in}

\emph{Within the general movement of news there is a dissection, or
editing, which is of interest in its larger detail to a given Class,
being the Class News}. For instance, there may be special detail news
relating to the railroads of Michigan which is sent out by a department
of the~Regional~head at Detroit as Regional railroad news and goes
into~the Transportation Class paper or railroad man's journal;~and there
is a movement of railroad news of wider interest through Detroit and the
other Regional heads to the general, or National, center and back again
to the railroad~papers or bulletins of the entire country. This Class
News~in its detail intimately relates to the interests and working of
the Class to which it belongs. This is the reflex from the external
sense-organs in every community to the special functions, or organs, of
the social body.~

\vspace{0.05in}

\emph{The Individual News is the dissection, or classification, of news
that is not of interest to the general public or to any body of
Class~workers, as a whole}. This final dissection and distribution of
the news, as of the restricted interest, is held at the News-Office for
anybody who may apply for it. This provides the Bureau of Inquiry in the
organized News. A man who wants some fact which is his own interest
purely will get it at this Inquiry Bureau, or counter, of the
News-Office, by asking for it and paying the price. It is something that
arises in personal contact, or interest, on which there is insufficient
information in the General and Class movement of News. It may be some
matter that is not of immediate history and has passed into accumulated
or compiled news. A farmer might want the exact contribution by Watt to
the steam engine, or a railroad man might want something in a popular
way about the oyster or the plow. Naturally these cannot be daily loaded
upon the wires of the General or Class. The News Office would sell the
inquirer a book or leaflet covering the question, or it might be dealt
with in ten or a hundred words, as in answer to~some one~who should ask
for the conditions under which crucibles explode. The buyer would apply
to the nearest News-Office, situated in every town; and if this office
did not know, it would call up the News-Office of its Regional head.
This, the Inquiry Bureau in life, from the standpoint of the purely
personal need, has its physiological analogy in the consideration that
any point of contact, any village, any person, however remote, may get
ordered information through nerves to and from the intelligence center.~

\vspace{0.05in}

\emph{The News is in degree now organized and moving from its external
nerve-endings upon its~Regional~heads and central Cabinet, coming back
in sensation to the general public, to the Class, and to the
Individual}. The newspaper now has its local correspondent, though in a
hapless and~irresponsible way. It has essentially its Regional, or
divisional, heads in the several newspaper editors in centers like
Detroit. Here the organization is deficient in that the several
newspaper editors or managers do not form one News-Board, under a single
managing head. The result of this disorganization in the~Regional~head
is the conflicting Regional, or state, news distributed to an area like
Michigan. But certain essential movement of General and Local news
nevertheless exists. News coming into Detroit that is reckoned of
general interest is passed on to New York or to other large center of
news movement, being dissected and transmitted by the Associated Press
agents, who stand for the general movement of news to the Cabinet center
and back to the~Regional~heads. The Regional center is deficient again
in that the Michigan head at Detroit, for instance, has no adequate
organization for dissecting the General News coming from its territory
to be sent~on to New York, being instead the inadequate and hap-hazard
notion of some person representing the above Press agency as to what is
of wider interest beyond the local demand. At New York or other center,
like Chicago, where the General~News is made up and sent to the country
we have~more or less faithful~transmission of what comes, but with no
organization for making it up into its bearings. Such news as comes to
New York over the Associated Press wires goes back to the several
newspapers of the country essentially a reflex current, though, as
indicated, more in the mechanical sense, having had little skilled
treatment. And through deficient organization at all points for the
handling of news it generally gets out to the people touched with the
prejudice of a local editor representing the factional views of a
locality. And the movement of news through the centers is frequently
colored by the investment interests. The Newsmen here, as elsewhere,
often deliberately mislead the public in prop of the money interest and
the existing slavery. There is no correcting head to enforce integrity
of the general interest against the narrow.~~

Regarding the reflex, or movement, of Class News through the Regional
and National heads and back to the several Classes through the several
Class journals, there is very little direct classification. But the
movement exists. The Class news is mixed in with the general movement
and printed in the general news columns of the papers, or it is
discarded altogether if it is seemingly not of interest to the larger
public. In the one case it is usually so much clipped of its detail that
it does not fill the wants of the Class, and in the other case it is
often lost to the Class. The Class journals, which have everywhere
sprung up, partly fill the deficiency in the movement. For instance, the
Railway Engineer's journal takes what it thinks is of interest out of
the general newspapers~and also~publishes news from its own
correspondents, who write with knowledge of the happenings of the Class.
But it should be said that the Class flow of news has~as yet~hardly come
above the horizon of the newspaper men as a part and dissection of the
movement.~

\newpage The Inquiry Bureau is found to have quasi-existence in the inquiry
columns of the papers. But it has small compass, with practically no
organization, and is compelled to turn away questions which it
characterizes as not of enough general moment to interest more persons
than the applicant. The~\emph{Ladies' Home Journal}, of Philadelphia,
has largely been built up on this limited inquiry side. Newspaper men
complain that people flood them with questions over the telephone and by
letter that are beyond the scope of their organization. This is a phase
of the strangulation of the State by the disordered news.~

\vspace{0.05in}

\emph{The outlook on the organized intelligence reduces and~gathers
together~under one movement the various phases of the publishing
business}. The outlook is first on that phase of the Triangle of
Intelligence which classifies as the General News, this, as seen, being
news of general interest---something that everybody is supposed to want.
The first form of the publication of this, answering to the general
newspaper of to-day, is~\emph{The Newsbook}. This~central dissection out
of the stream of news has equal interest for a man in California, a man
in New York or a man in Texas. In character it simply answers to what is
the general news now published simultaneously each morning and afternoon
in all the great dailies of the country. It is in the main such news as
is furnished by the Associated Press, except, as we have seen, it has
more systematic and unified handling. And since this General News~has
to~come to the center, or Cabinet, for its treatment and for its
distribution we see that~\emph{The Newsbook}, the great political daily,
is made up in its entirety at the commercial center of democracy. This
paper as fast as it is made up at the office at the center is put upon
the general telegraphic circuit and taken off and printed in its
entirety by each~Regional~center. It would thus occur that the general
publication,~\emph{The Newsbook}, at San Francisco or Galveston would be
an exact copy of~\emph{The Newsbook}~at New York or Chicago. The
publication would practically fall upon the country~as a whole in~the
same moment of time and in the same make-up. The first projection of
this distribution of the General News from the center would naturally be
put upon the wires in the form of condensations, or bulletins. The
Regional centers would take these bulletins off the larger circuit and
put~them upon the circuit of the Region. Each County head would in turn
transfer them to the local wire, or circuit, of its division,
corresponding to the news ticker circuit now become common in the
cities. It would result that the whole country would simultaneously get
the condensations of important General News ---the fuller detail coming
to them in the publications from their nearest centers.~

The next phase of the publication is the general news of each Region
strictly, corresponding to the present state news. This could issue from
the~Regional~head in company with~\emph{The Newsbook}, but as a distinct
print or classification, called~\emph{The Region}. It is general news
corresponding to circumscribed territorial interest. Everybody in the
Region wants~it.~Its condensations, or bulletins,~preceeding
{[}\emph{sic}{]}~its more detail publication, would be put upon
the~Regional~circuit to be taken off by the lesser, or County, circuit,
interspersed with the General News bulletins proper.~

There is a third form of news, still more restricted in its interest,
but general to its locality. This is the local news of the town and its
adjacent territory. As transportation becomes cheaper and multiplies its
facilities, a given territory like a county, embracing a town or city,
comes to have more marked community of interest. A man in town through
familiarity with the embracing section of suburb or country wants its
happenings; and the man in the country adjacent to the town and familiar
with it looks to be furnished its news. This classification entirely
local, embracing the county, is the publication called~\emph{The Town}.
Aside from the printed paper, the Local movement has its bulletins, put
upon the wire interspersed with the General News bulletins and
the~Regional~bulletins.~\emph{The Town}~has its further character in
treating at considerable length the details of life, for readers who
like all the story. Nor does it strictly confine itself to the local
happenings of its own borough, but copies freely, by wire or clipping,
from similar publications over the country, wherever a good story is
found.~

In the organic view advertising reduces to a form of local news. The
reality of this is easily apparent in scanning the advertising columns
of the newspapers. The Post Class publishes the arrival and departure of
its mails. The Judiciary Class announces its proceedings. The various
stores tell of their attractive things and novelties. The meaning of
this is that each line of industry puts into the papers the news about
its business. Many a housewife who takes up the morning paper turns
first to the advertising column to get the news about dress-goods or
groceries. Announcements of new things in~food and clothing, theaters,
changes in arrival and departure of trains and mails, must be seen as a
permanent feature of the news columns of all local publications. Not the
least interesting news in~\emph{The Town}~of any time must be the
announcement of the arrival of some carloads of bananas or peaches at
attractive prices. This form of news will be classified under the heads
of the various industries, Food, Clothing, Mails, Transportation, etc.~

The second phase of the Triangle of Intelligence is the Class News. As
we have seen, it gets its publication in the different Class bulletins
and Class journals, being the unified technical intelligence.~

The third phase of the Intelligence Triangle, self-evident, is the
Individual News, or Inquiry Bureau.~

It must be seen that all publication, or news, of whatever kind passes
through one of these forms. The writings of the students of economics at
the Universities or outside of them, through books, pamphlets,
periodicals, or newspaper publication must be regarded as ultimately
coming within the one organization of news. Such a man finds his place
in the News Cabinet or in some subordinate position. Writers on
technical~subjects~place in the several phases of Letters which look
after the treatment and handling of news belonging to the various lines
of industry, the Class publications. A man, for instance, writing
technically on chemistry gets his publication in the Class
paper,~\emph{Chemistry}. All other writers of whatever kind become
incorporate in the one field of Letters and have their assignments of
work according to their~particular qualifications. Matter of permanent
interest passing through any of the several publications is preserved in
pamphlet or book form.~

The delivery side of news in its active movement, we have elsewhere
seen, finds its place in the supply store, or distributive station. This
counter carries~\emph{The Newsbook},~\emph{The Town}~and the various
Class papers which have demand in the locality, like~\emph{The Weaver,
The Farm, Food, Textile, The Fireman, Mines}, etc. It will carry all
that is preserved of literature as supplying the active
demand---corresponding to the book counter of the department store at
the present time. The Inquiry Bureau is to be regarded as a phase of the
one News counter, supposed to carry all lines of the business. This
gathers at the News-Office in the department store the several phases of
the news business on the side of its active distribution.~

Literature that has become dead to the general interest will, if
it~have~any economy or interest whatever, be found on the shelves in the
libraries at the public gathering place, the library being another
counter of the news business, where books are rented.~

Much reduction in the permanent store of writings may be expected in the
development. We have already entered upon this. The writings in the
Blackwood Classics and similar series, and the reports and condensations
of Mr. Morley and others, have put into a few books many hundred
volumes. This should go on until the line is clearly drawn between the
living literature and the great dead and cumbersome mass. The novel in
the unreal sense comes to be displaced as now by the new, or novel,
truth that is stranger than fiction, and the unreal poetry comes as now
to be displaced by the poetry of action in the report of the daily
event. New advances tend to put into small compass the writings in the
region of mind, morals, ethic, etc. Writings in an apart ethic and
speculation in sociology fall away with the real thing, which we have
seen as one with the advancing organism in democracy.~

\vspace{0.05in}

\emph{The state of consciousness of News touching the principle of its
own business as sensation}. The newspapers strive for sensation but do
not uniformly act up to the~principle underlying it. Its realization
with them is little removed from the accidental. We know that sensation
is the taking of the report through to unified relation in the social.
The sensation in a divorce, for instance, is its relation to the
well-being of the State.~Can men and women have more liberty in their
home relation without disrupting life? The sensation about the ``ever''
Magdalene is the knowledge of whether the phenomenon is not the outcome
of too great stringency in divorce, or whether it is not one of the
prevalent monstrosities traceable to the disordered exchange. How
many~have to~do it to get food? The reality involved in a man going with
a woman of the street may be that the conditions inherent in the
disorganization of the State are such that he cannot go with a wife.
This principle, of the part related to the whole, put against the
details will determine their meaning.~Thus~it is seen that the full
account of a man's participation with this offence of the social is not
told in his having been found in a brothel; the fuller fact may be that
Society was found in a brothel, entirely disregarding the particular man
or woman. The sensation is to put fornication upon all men instead of
upon one---that is, to make the offence bigger. The news is manifestly
short in consistently bringing through to the final situation. The
papers are not alone deficient in their discernment of last relation,
but they are deficient in the details which carry it. People want the
straight detail as much as they want the straight principle involved;
the former embodies the latter; the two are one. In the rush
of~disorganization~the papers print such fragmentary matter as may come
to them, regardless.~

A report often mistakes irritation for sensation. The reader is
irritated by the partial thing, wanting the fuller detail; and he is
irritated in the absence of the principle. The reader is much fretted
through the exaggerating or making prominent of certain parts which in
the absence of the full matter of the report is thought to make up for
the deficiency. The reporter writes in red at the juncture where he
lacks the simple facts which go to complete the unity of the news.~

As reinforcing the principle, we ask ourselves whether ``United States
Trunk Lines, Division New York Central,'' is not more sensational
than ``New York Central.'' And we put to ourselves whether the~reporting
of a strike on~the New York Central would not have more sensation if it
were related to the social body by showing that it is the friction
incident to the evolution of the organization of a Class in Democracy.
The reality of a railroad president resisting a strike, properly
related, is that he is a laborer with others of his Class but has
viciously usurped authority and voice, fortifying himself in certain
illegal tenets to resist the advance of the life of the State. Would not
the great sensation be the turning of the tables and declaring the
railroad president on strike against the advancing order of his Class
and the State? The management would be depicted as embodying anarchistic
elements obstructive to the whole. Further, as properly relating a
railroad accident we may see that the sensation is to locate it in the
organizing head, as far as he is to blame. The one mainly responsible
for a railroad accident is the man who has so lamely organized the
system that it is possible to have a railroad accident at all. On a~well
organized~road it could not occur in the sense of a collision, an open
switch or a defective bridge. The papers themselves incline to fix the
responsibility but have divided notions about it and do not carry it
through to the reality, the agency responsible for full order. In the
end, the short-handed organization of the Class is responsible for the
railroad accident.~

It is certain that the newspapers cannot be unmindful that so far as
they are now~successful~they act on the principle of sensation---full
truth. They can but recognize that men buy the newspapers for such truth
as they contain.~

\vspace{0.05in}

\emph{The Editorial results from incompleteness in the News report}.
Sometimes the ``editorial'' is such only in name, its reality being that
it is a news report on the ``editorial'' page. But strictly speaking,
the editorial stands for some deficiency in the news column; it supplies
some portion of the fact which is absent in the latter.~So truly is this
the logic of newspaper practice~that reporters~are ridiculed or
discharged for making their reports so complete as to leave no room for
so-called editorial comment. Good reporters who bring in the full facts
have been discharged on the ground that they were trying to write
editorials.~Mr. Brisbane~of the \emph{New York}~\emph{World~}in the
first half of 1894 was writing a column in that paper, being a summary
view of the day to the hour of going to press. He had access to the
telegraphic and other reports up to the closing of the forms, and he
gave some measure of unified tone to the news in a happy presentation in
the light of his somewhat wide knowledge of affairs. It was found that
there was little left for the editorial, or essay, page. Mr. Brisbane
would put an ordinary half-column editorial into a sentence. The editors
took the alarm and the column was stopped.~

The ``editorial'' and the ``news'' report now conflict in practice and
make the lesion in publicity. A newspaper will be found saying that its
policy, meaning its editorial, has a certain tenor, no matter what the
news columns may contain. It will say that its reporters are told to get
the news regardless of the policy of the paper. This is equivalent to
saying that it does not matter whether the facts conflict with the
editorial column or not. In practice this means that the editor is
frequently compelled to restate, or interpret, the facts to make them
fit the partial interests which his ``policy,'' or view, represents. It
is the partial interpretation, or deficiency in news. The editorial is a
factor in the divided action of the people who get their notions from
the paper, since the editor writing independently, and indifferent to
the full report, is more prone to make assertions inclining to partial
interests.~

Again, the editorial is seen as a part retention of the old notion that
the ``ought''~needs~to be asserted alongside the fact. It is the pulpit
reiteration of the precept brought over to find its small preserve in
the types. The editorial is the little church within the newspaper. So
far as such attains, it is the preaching of the ``ought'' in the absence
of the fact---the authoritative pronouncement. The~``ought''~is~in
the~\emph{is}---is in the full fact. For~instance, we do not have to
tell a man he ought not to fall off a ten-story
building;~the~ought~is~in the fact of the action fully stated in
results, namely, a fall from a ten-story building is nine times out
of~ten, the equivalent of a coffin. If the fact is sufficiently stated,
the ought, or direction to action, is in it. In general, it is to be
said that news has only to state the fact in its full bearings, leaving
men to gather their own line of interest and action. Neither
pulpit-priest nor editor-priest can strictly know~the~ought for any man,
as that has personal and private bearings with which they are not
conversant. It might be supposed that a man ought to vote for government
ownership of railroads, for instance. But no editor is justified in
saying so. He can only report the fact that shows it to be the way to a
general five-cent fare. The rest may be safely credited to men's
intelligence. In view of the organic ethic, or freedom, based upon the
analysis of will as unity in a consciousness determined by contact with
life, it becomes apparent that the ought cannot be superimposed upon a
man outside of his own convictions of action, as self-determined on the
fact. All such is the mental degradation, or restriction. The editorial
lesion exhibits in general the present brutality of news which
forces~personal opinion, or comment, upon the reader in his desire for
truth. ~

\vspace{0.05in}

\emph{The organism reveals that the movement is away from the arbitrary
censorship in Letters and the other fields of industry}. At the present
time there is nothing in the absolute to compel the newspaper to print
up to the demand. The tendency is to work out of this. A man belonging
or not belonging to the regular organization of Letters who has
something written may offer it to the News-Office, just as it is now
offered to the less organic print, in some newspaper or other publisher.
If News refuses to accept it as Letters that will be but the technical
rejection of it, just as Music might reject a score or an opera
presented to them, or as Transportation might reject some supposed
invention in their business. But this need not end the matter should the
author be dissatisfied with the position~of the News-Office. It remains
his privilege to expose the writing, together with the criticism of
News, in the public market place of the locality, that is, at the
nearest News counter. The author may of course add any remarks he likes
in answer to the criticism which News has put upon his
writings.~Thus~the public of the locality may have access to the entire
proceeding. The people may themselves determine the validity of its
rejection by the Organized Letters. Anybody may read it, copy it, and
propagate it, at will, short of using the regular machine of News. If
it~have~merit adapted to the time it will appeal to men. And should the
sentiment favoring it grow in the locality so that a majority of the
people should come to want it in print, they can compel Letters to print
it, by use of the local ballot, the machinery for which is a
consideration of the~``Negative,''~farther on. By the ballot a majority
can negative, or forbid, the action of News in rejecting the manuscript,
and News has left to bring its action in accord with the demand. The
force of this is that nothing is news in the sense of the life demand,
in the sense of publication, unless the people want it. If only the
writer wants it, he has it in the copy which he holds. Supposing that
the writing comes finally to be commercially printed, it will be sold at
any point where a demand springs up and a request is made for it. In the
same way, a person whose composition has been rejected by the organized
Music would have the privilege of convincing people that the Class was
wrong in its rejection of his composition. Whereupon, they could demand
its rendering by the local orchestra. In a similar way an inventor,
painter, or other, may expose in public his work when rejected by any
Class. If he can get enough people to endorse him, the Class will be
negatived in its action and compelled to put the work into use
sufficiently to demonstrate its quality. The point is, in every phase of
the State, that we are working out of the arbitrary censorship and over
to the censorship of fact; the censorship of the actual test in life,
the test of action. At the present time the demand~is for certain unity
of utterance by News in its active machine, the newspaper, but the
people cannot get it; the private interest too much controls.~

\vspace{0.05in}

\emph{The advance of the State is pushing the newspaper toward
recognition of its own law---its own principle of being lying bare
before it}. The real Letters, as standing for the organic concept, is
not in control of the newspaper. This has not alone divided the people,
through the resulting multiplicity of view,~but it has denied them as a
body the quicker forethought of the Nation as to the development in
democracy. Observation must show that the orderly thought is mainly
outside the newspapers trying to get expression. Old and used-up
concepts become disorder when a nation is struggling to pass them. The
growth towards organization has in all the Classes been hindered by
deficient and short-handed publicity. The newspaper under the domination
of the counting-room has the attitude of keeping the new out; it has the
attitude of restricting the quality of publicity: as though medicine or
chemistry were organized to resist new formulas; as if in practice men
ignorant of chemistry should be found interfering and overturning the
formulas for refining oil or for making quinine. Ideas arising in the
newspaper offices that conflict with the notions of the counting-room
are more often summarily turned down and the authors dismissed.
Reporters and editors are specifically told that they must not in
essentials antagonize the investment notion. It is the private interest
obstructing the organization of the news. Such is the paucity of
consciousness by the newspaper as to order in the State that it is
divided in its thought regarding the underlying method of its own
business, as one with all business. On the surface, it assaults the
Standard Oil Company, the railroad combines, the sugar trust, etc., as
so-called monopolies menacing the State, when the newspaper itself is
working under a similar union in the Associated Press Company. There is
no closer monopoly than this Association. To start a paper in New York
with the benefits of the Associated Press requires the consent of every
newspaper belonging to the service in the City. It would be supposed
that thrown against the conditions of its own growth the newspaper would
in time have to recognize the principle of the organism in the State.
And there is the most startling fact pressing for recognition, that the
Associated Press, belonging in its function to all the newspapers that
take from it, is in effect the abrogation of private property. All this
is the nearness of the newspaper to an estimate of its own nakedness.~

\vspace{0.05in}

\emph{There is discoverable a high degree of integrity touching news on
the side of the mechanical distribution---on the side of the types, the
printing press, etc}. Well-nigh perfect organization has developed here,
and it is possible by multiplying type-setting machines and presses to
throw off an unlimited number of newspapers per hour, and through the
capability of the mail and its adjuncts send them out. Standing and
looking upon the marvelous precision and efficiency of the web press in
the cellars of the newspaper offices one must reflect that this feature
of the news business is doing its work.~

The question of whether the mechanical side of the newspaper is a part
of the News Class proper may be left open. A sign pointing to the
inclusion of the typesetter and the pressman in the one Class with the
reporter is the fact that all such are employed exclusively by the news
industry. Unlike the telegraph or railroad, which may be employed by
other Classes as well as by news, printing places exclusively in the
service of the latter. But, again, it is a business having its own
technique apparently distinct from the technique of reporting, or News
proper. The printers could be a distinct Class, simply employed and paid
by News, or they could be a distinct branch of News, having in either
case their own rules, or autonomy, answering to the demand of News upon
them.~

\vspace{0.05in}

\emph{It is from the mechanical or integrating mind that advance in News
is expected}. It is the mechanical order of mind that thinks with the
precision which is action in relation to the economy of life. The
mechanical method is the principle underlying efficient thought. This is
simply to see the part as in active relation or division of labor within
an organ and the action of the whole organ trained upon some~particular
office~in the environment. The mechanical mind is the philosophical
mind, the philosopher being only a mechanic who has to do with the
larger machine---the State. Throughout the phases of~life~the
mechanical, or philosophical, order of mind has been trained upon the
building of the many machines which go to make for the liberty and
action of the larger one. The locomotive, the telegraph, the typograph,
the web press, the sewing machine, the reaper, we discern as underlying
parts of an integrate democracy. The primary work~being much along and a
surplus of the mechanical mind freed for larger action, we approach the
juncture where we are likely to find invention, or advance, in methods
for the closer organization of the State. And it is from this freeing of
mind for action in the more culminating regions that we look for
invention to organization in that phase of the State which is Letters. A
good reporter has simply the mechanical notion of relationship; and the
philosopher, or chief-reporter, as of the Cabinet and the Regional news
boards, must be expected to arise from this character of thought. The
newspaper management is waiting upon the carpenter-and-joiner of fact.
We have as much difficulty in conceiving the mechanical mind failing
when freed in newspaper direction as we have of the Hoe press lacking
the joining of its wheels and the revolution of its types.~

\vspace{0.05in}

\emph{An avenue which makes for an~outmoving~in the organization of the
newspaper is the growing necessity for integrity in its business}. The
great daily at the present time has reduced the price of its issue to a
point that does not pay the cost of news-service. The half-cent for
which the morning paper now sells to news-dealers does~not well cover
the mechanical production.~The newspaper~on its revenue~side is in the
position of~unstable~equilibrium which has no base of support~in its
own~legitimate business---the sale of news. The reality of advertising
we have seen as a phase of news. The newspaper has fallen upon taxing
this intelligence to an extent which goes to offset the deficiency of
revenue in the general news movement. The store of O'Neill and Company,
New York, not the largest advertiser, pays \$100,000 a year to the
newspapers. The result is that the advertiser, the distributive trade,
is carrying excessive burdens in helping to support from one to a dozen
daily newspapers in a single town. Furthermore, the great newspaper
concerns to keep going have of late been resorting to blackmail on the
most stupendous scale. A New York paper recently took from the Standard
Oil Company \$100,000 on the general proposition that this corporation
might~some day~need to have the right word said for it. The big combines
of capital in every direction are, on the plea of mutual interest,
understood or expressed, making a divide on profits with certain of the
newspapers. Some of the big corporations own controlling stock in
newspapers which are situated to~forward their projects. When the
Standard Oil Company some years ago was pushing its pipe line into
Toledo from the Ohio gas fields it purchased outright the Toledo
Commercial, putting its own newspaper man in charge to manipulate news
in the Company's interests. Similarly, the management of the Great
Northern Railroad controlled a prominent daily in Minneapolis at the
time of the strike on its lines in 1894. All such are phases of the
subsistence of the newspaper on means outside of the legitimate sale of
news, giving force to the fact that the newspaper is without basis in
normal revenue. Should anything therefore arise in the movement of
things that would tend in any direction to disturb some portion of their
illegitimate revenue, the newspapers would of a certainty be compelled
to get together to save themselves, by organizing for cutting off the
present waste of several newspaper plants all doing the same thing in
one town. And in the long run, owing to the growing insecurity of the
newspapers, the tendency of things must be to compel them to the
economics of one management, more than now.

\enlargethispage{\baselineskip}

The conditions that drive any line of trade to the economy of
organization under one management must be the tendency of the
newspapers. The conditions~preceeding {[}\emph{sic}{]}~the organization
of the Standard Oil Company were of a score of different refineries
located at various points and warring against each other for advantage,
the public paying the bills. The saving idea of a pipe-line to the sea
came to one of the refiners and he started to put it into execution,
finding that he could make it pay its way section by section to tide
water. He had his first section down, eight miles. Alarm seized the
other oil men who saw that the trade would not support twenty pipe lines
to the Atlantic. Mr. Flagler, Mr.~Rockefeller~and others of the refiners
got together and made the compact known as the Standard Oil Company,
each of the manufacturers taking stock in the general Company in
proportion to the value of their plants. Out of this combination
the~Standard Oil Company has improved the quality of oil and reduced the
price 50 per cent. One reality of their princely revenues is that in
cheapening the cost of oil they have divided the profits with the
consumers, making large consumption.~

A single economic idea may likewise overturn the newspaper and compel
its organization. It lies, in one direction, in the driving of a wedge
between the newspaper and the current advertising business. The
present great flux of advertising has its source in the
disordered~exchange. Under the private notions of property, advertising
is constricted in its character as news. It is virtually the individual
inquiry at the News counter, made~necessary by the existing state of
trade which drives individuals to search for knowledge of where to buy
or sell. Advertising thus tentatively classifies as the private, or
personal, intelligence. It should for convenience and economy issue in a
single publication, properly called~\emph{The Want}, to contain
essentially only classified advertising. The idea is in successful
operation in Paris and Berlin, and virtually draws to itself all the
advertising of the city. Some newspaper, or good business man moving
independently, might combine in an advertising pool the retail merchants
of a city, whose interests are all in the direction of concentrated
advertising, so that they may have to pay for only one advertisement
instead of half a dozen. Such an alliance of merchants could, after a
time, hand their advertisements only to~\emph{The Want}. This would
carry the publication in its revenue until the lesser advertisers could
be brought to it from sheer interest. This daily~\emph{Want~}would be
sold for one cent, or more, either independently, or along with any
newspaper that should manipulate the move. Ordinary wants would of
likelihood be published in it for five cents, and possibly one cent.
This would be the business acumen which gets its revenue from the large
grist and small toll, at the same time crushing out opposition because
it is so near the interests of the public. No one can compete against a
thing that approaches perfection in both price and quality. Such a
newspaper, liable to get on its feet, would spread the alarm among the
big city newspapers and they would find it necessary to combine for
their existence.~

\enlargethispage{\baselineskip}

Again, any State control or regulation of the great trusts which would
reduce their revenue to a minimum of margin, would tend to cut off the
blackmail money which the newspapers are drawing from them. So that on
both sides, through advance in democracy, the trend would be to bring
the newspapers together under one management.~

Acting~together~the newspapers can, of course, fix a revenue supporting
their business aside from channels compromising the news. The saving
through concentrated action should alone effect this. And the very
gravitation of the large organization, cutting clear of the outside
private interest, would be to improve the quality of the news, so far as
it freed the mechanical, or artistic, mind.~

In this we face the Intelligence Trust. Made amenable to the needful
general laws regulating all trusts, or~centralization of industry, it
becomes the Trust of the people. It matters not what it is called,
whether Government control, or, popularly, ``Government ownership,'' of
the newspaper. With ``Government ownership,'' or control, of Carriage
and Exchange, effecting these functions at cost, the power of the king
for harm in other directions is no longer to be feared. With the day
growing, the Intelligence Industry settling toward centralization, as of
the sure end in democracy, labor may take stock of an advocate at
court---the FACT.~

\vspace{0.05in}


\emph{The state of ethic, or freedom, of the newspaper}. News is to-day
under the domination of its own~disorganic~ideas and of all the
disorderly features of the State. The~disorganic~elements, which in the
growth of things would be attending to their business in other Classes,
are obstructing the newspaper. If the bankers, railroad presidents, and
all the components of the private interest were not curtailing the
movement of the News Class and creating schism in its utterances,
publicity would doubtless be found at this juncture speaking rationally
and standing for advance in the organization of the Classes. The rank
and~file of the newspaper men chafe for freer utterance on these lines.
They ask to stand more for the equalization of the returns of labor.
They would find arbitration, for instance, if they were not governed by
the counting-room; many papers are~abreast of this now, but there is
conflict of publicity on this and kindred points. Many newspapers now
would doubtless throw overboard the private interest entirely if they
could see the one step further into the greater order beyond it. We may
speak of the \emph{New York}~\emph{Journal}, as at present occupying
advanced position here. When News has more attained its organization and
is freed from the friction of the unorganized element, the forces will
drive it up to its integrity through its own self-interest and artistic
momentum. The newspaper, ruled by the advertiser and others, cannot yet
speak for the interests of the Whole as against the private concern.
Publicity does not report its advertisers, does not persistently label
spurious wares in every field, from the exaggerations of the real estate
man to the fraudulent medical advertiser---does not report them save in
general terms. It~dare~not specifically report certain of its
advertisers, though in some respects it goes far, as in its playhouse
news. And we have seen that in all directions of the private interest it
does not unify publicity. Last, the newspaper does not move to the
surpassing sensation, the uncommon freedom, of reporting itself. There
is a mysterious balance-sheet and certain unknown transactions. The
newspaper cannot yet act up to the part of full fact-giving wherever it
touches life. It waits on fuller organization for its free action, its
ethic.~

That aspect of the American newspaper which has brought upon it the
charge of a ``rawness'' of utterance and lack of respect for privacy~has
to~be sounded in its ethic, or essential right, from the side of the
dynamic forces. If we compare it with the more reserved utterance of the
English press, we may attribute much of it to the greater lengths to
which inquiry is carried in this country. It may be thought to mark the
impetus of the news over to full fact. The movement is in the direction
of breaking down the barriers to full inquiry. There is likely to come
uppermost out of the crush the status of real respectability. It is
likely to result that nothing can lay claim to respect except the fact
itself, established after the most searching inquiry in the fullest
publicity. There is likely to result the status that nothing is
respectable that cannot stand the day. As to why the movement should
apparently have its front in America, we can but attribute it to the
reach of this country. The Organic Letters quickens in the womb of
distance.~

\vspace{0.05in}

\emph{Property in the News Class is function of publicity in terms of
its prints, bulletins, and personal inquiry counter}. It is borne by all
Classes in their several divisions of labor. The News-Office may only
tax for its direct support, its particular division of labor. It may put
a cost price upon its goods.~~


% THE NEWS SYSTEM: A SCIENTIFIC BASIS FOR ORGANIZING THE NEWS
\chapter[The News System: A Scientific Basis for Organizing the News]{The News System: A Scientific Basis\\\noindent for Organizing the News}
\label{ch:The News System: A Scientific Basis for Organizing the News}
\chaptermark{THE NEWS SYSTEM}

\vspace{.2in}

\begin{LARGE}

\smallcaps{Franklin Ford}\marginnote{Letter to Clinton W. Sweet sent from New York.}

\end{LARGE}

\vspace{0.5in}

\begin{center}

\Large{GENERAL NEWS OFFICE~}

\end{center}

\hfill January 30, 1907

\vspace{0.2in}

\noindent Mr. Clinton W. Sweet\footnote{{[}Clinton W. Sweet (1842--1917)~was one
  of the founders of Sweet-Orr \& Company, a clothing company employing
  thousands of workers.~He later launched specialized publications such
  as the \emph{Real~Estate~Record and Guide}, the \emph{Architectural
  Record}, and~\emph{Sweet's}, a catalogue of building materials based
  on the model of the dictionary or the phonebook. Sweet's publications
  attracted the attention of Ford, who wanted to join forces in order~to
  bring~into~existence~a~project he~then~called~the News Centre or the
  News System. Ford discussed Sweet's publications on many occasions,
  writing that the \emph{Real Estate Record and Guide} was a recognized
  authority in the municipal field and ``an organic centre of credit
  information.'' See \emph{Municipal Reform A Scientific Question} (New
  York: City News Office, 1903), 20; \emph{The Public Necessity of
  Organizing Sovereignty Through Credit Authority} (New York: News
  Office, 1910).{]}~}

819 Broadway, New York~~

\vspace{0.2in}

\noindent Dear Sir:~

I have invented the News Centre which follows upon the Telephone, and
have ready for final revision and publication the new literature of
Politics which results from the appearance of a positive or systematic
news system. This literature is, in the first instance, educational; it
amounts also to universal advertising material for the project in hand.
For the first time, the ADVERTISING of a business can be sold.~

The invention I have perfected is, in effect, an extremely valuable
trade secret, forming a scientific basis for organizing the news Trust.
However much we may reveal the nature and detailed application of the
principle, before others could hope to occupy the central position in
the News System, and so supplant us, they would be compelled, in great
measure, to pass through the experience which has resulted in the new
discipline. Moreover, since news organization is essentially unitarian,
before anything of moment could be done in that direction, we would be
so far along that all attempts at competition would but operate to
advance our own work.~

I have now to put down for you the procedure to be followed in launching
the news Trust.~

1. I would incorporate the General News Office as an ordinary business
corporation, under the laws of New York or some other state, making the
basis of the incorporation my accumulated literature and copyrights. The
capital stock at the outset should be small, and the shares should
remain for the time being in the hands of myself and my immediate
associates.~~

2. As the literature in hand is the gateway to general acceptance for
the project, I would next proceed to revise and print in pamphlet form
the following seven reports:~

\vspace{0.1in}

\begin{hangparas}{.25in}{1} 

(a) THE INDUSTRIAL STATE AND ITS GOVERNMENT~

(b) THE RAILWAY TRAFFIC ASSOCIATION, AN ORGAN OF\\ SELF-GOVERNMENT

(c) HISTORY OF THE POST OFFICE; A KEY TO THE TRUST QUESTION~~~

(d) THE DEVELOPMENT OF A POSITIVE OR SYSTEMATIC NEWS SYSTEM~~

(e) EVOLUTION OF BANKING AS AN ORGAN IN THE STATE~

(f) PRESENT POSITION OF NEW YORK\textquotesingle S MUNICIPAL
GOVERNMENT~~

(g) THE REPRESENTATIVE SLAVE\footnote{{[}In a letter to Oliver Wendell
  Holmes Jr. sent on May 6, 1907, Ford also refers to this report. He
  writes that ``The `Slave' story will be all color, as the general
  public must be addressed in a picture language; the first draft of it
  was done in 1890, and consists of some 18,000 words'' and that the
  final version would ``make well toward 50,000 words.''{]}}~

\end{hangparas}

\vspace{0.1in}

\enlargethispage{\baselineskip}

The~last named~report is the story of my collision with the Newspaper.~~

I would print one thousand copies of each of these reports, and have the
page forms of each stereotyped.~

3. I would then set about introducing the project to the representative
men of business in New York (possibly two hundred in number), seeking to
get from each of them individual subscriptions to the reports named
above; the subscriptions should be \$100, or perhaps \$250. The purpose
I have in mind here is to make of this individual approach the entering
wedge for large sales of the literature to the great corporate interests
of the country, which for the most part head up in New York. The aim
should be to take to those leaders of commerce written committals to the
project from the foremost class newspapers, and other important
news~centres.

\enlargethispage{\baselineskip}

4. On securing adequate support from the leading commercial interests, I
would issue an edition of the reports large enough for distribution,
under~well known~references, to all prominent newspaper managers in the
United States, to the judiciary, to the foremost lawyers, to bank
officers, to members of~Congress and the state legislatures, to leading
merchants, and to the masters of industry; the~last named~class to
include the railway men. The aim will be to invite criticism from
representative men, with a view to incorporating in the new literature
such of the results as are valuable.~

5. With the action carried to this point, we will have in full operation
the Educational Division of the General News Office. It has to {[}be{]}
kept in mind that we are, primarily, organizing a business action, and
that the literature already in possession is a by-product of this
action. The first thing is to use the literature for enlisting money
support, and at the same time for educating the news managers up to the
point of free co-operation through the General News Office as
main~centre~of the News System. Following upon this, we will own book
copyrights constituting, in and of themselves, a profitable educational
publishing business. The preliminary work on a Commercial Geography, of
world-wide usefulness, is already far along. There exists an insistent
demand for the true commercial geography, but it cannot be produced
without the application of the principle underlying the General News
Office.~

6. As an incident in the development of the Educational Division,
working relations will be established with all University~Centres, with
a view to imparting the principle or method to the schools of political
science, and so to bring about far-reaching co-operation in political
inquiry. It is proposed to~effect~the university connection through the
Social and Economic Section of the American Association for the
Advancement of Science. Three members of the sectional committee have
already considered the plan favorably.~

7. As an outgrowth of the Educational Division, a Business Pens Service
for the daily press of the country and the world will be established, on
the lines of the Associated Press. One main purpose of this division
will be to create a market for one part of the by-products of the class
newspapers, consisting of general news relating to each division of
commerce. As an ultimate outcome here, it is proposed to introduce a
weekly report on the state of trade and industry, to consist of
contributions from the class journals representing the various divisions
of commerce.~

8. As soon as the Educational Division and the Business News Service are
in the hands of competent managers, I would begin organizing the news of
New York's municipal system. Just two things require my further personal
touch, namely, the full development of the literature of Politics and
the systematic reporting of municipal news. The report on the Present
Position of New York's Municipal Government would be used for
introducing the new~centre~to this field. With this literature
distributed judiciously, I would proceed to make demands on the city's
general accounting~centre, the Department of Finance, for information,
each item at the outset to involve an important development in
statistical accounting. For example, our first request would be for a
report on the city's miscellaneous revenues, which are not now so much
as audited by the Finance Department, yet they are increasing rapidly
and now amount to a good many million dollars annually. The Comptroller
could not do other than fall in with our demands, and so be led into
direct cooperation with the News Office. The vantage ground would be
with us, since we have mastered the procedure and know the end. By this
means the News Office will pass into organic relations with the Finance
Department. We would bring up with a full report on the Evolution of the
City Government of New York, thus resulting in an epoch-making political
document.~

9.~With the working relation with the Finance Department established in
a practical way, we would proceed to issue a daily bulletin giving the
transactions of the Finance Department for the twenty-four hours,
consisting of all claims filed and the completed audits. At the
same~time~we would clear each day, by mail and telephone, all contract
or opportunity news of the municipality. About a thousand contractors
are constantly doing business with the city. The stream of contract news
should be sold through the F. W. Dodge Company. This double step will
for the first time open up real publicity for~the municipal affairs of
New York. Each item, when registered, which promises news of general
interest, would be followed up and reported in all essential details. A
monopoly of municipal news would result.~

10. Following upon the mastery of the opportunity news in the local
municipal system, together with all news of the same order in the
metropolis, we would proceed to clear the contract or construction news
for all divisions of the recognized governments, national, state, and
municipal, and for the entire country, the Dodge Company to be selling
agent for the whole field. In addition, all \emph{general} news in the
building and construction line should come to be cleared by the Dodge
Company.~

11. ~A universal or central information bureau for the metropolis can be
announced at the time of issuing the daily bulletin giving the
transactions of the Department of Finance. The conditions are ripe for a
telephonic news~centre, through which all parts of the local information
system can be called up, and each made to contribute its quota of fact
for the guidance of the business public and the people generally. With
the list of local news offices perfected, and a business relation
established with each, the General News Office will be able to advertise
all inquiry~centres~in the system through a single outgiving. A trading
relation with the New York Telephone Company is proposed, which will
permit all pay stations in Greater New York to give the News Office wire
to any one in need of information, the fee, by special arrangement with
the telephone people, to be paid to the station operator. Touching the
development of revenue for the telephonic news~centre, S. S. Pratt,
editor of the \emph{Wall Street Journal}, has been saying to me that
very many individuals and trading concerns will gladly pay a reasonable
fee for the right to call up the~centre~for miscellaneous or routine
information. A news stamp could be introduced for attaching to letters
of inquiry.~

12. Revenue creation in support of the Municipal News Division should
have special treatment. We should introduce the~pro rata~principle, as
it has obtained in the field of ordinary taxation; that is, we should
induce a bank or other corporation to make a yearly subscription
amounting to a small percentage on its volume of business, and then
proceed to extend this over the entire revenue field. It appears that
all taxation began with voluntary contributions. We are proposing a
re-formation in Government which will result in vast social benefits,
and the fact should govern our action. Already, what amounts to a new
system of taxation has come to exist in the city, consisting of regular
contributions by the varied business interests to all manners of
"municipal reform societies,'' several hundred thousand dollars being
paid to them annually. These revenues must be systematized and made
permanent by us, if we are to achieve the highest success. Through the
centralizing of industry, news revenues in general are undergoing a
radical alteration; the change is basic and we must follow the tendency.
To organize the news business, we must place its revenue system on a
permanent basis. It is a system of \emph{taxation} which permits the
publication of daily papers like the \emph{New York Times} and the
\emph{World} for one cent, the price being hardly more than the cost of
distribution.~

13.~ It has occurred to me that we might with advantage enter into some
sort of business arrangement with the New York Board of Trade and
Transportation for delivering to it certain of our municipal reports.
They could recommend action on the strength of our scientific findings.
I have known the secretary and active manager of the Board since my
Bradstreet days.~

14. The traffic in news between sub-centres~and the General News Office
will be handled on a commission basis, at least to begin with, and on a
basis, say, of fifty percent of the net receipts. For example, if the
General News Office sells a report furnished by the \emph{Real Estate
Record and Guide} one half should go to the \emph{Guide}. It may be that
later on the source of supply should receive more than one half.
Regarding contracts with sub-centres, I take it that nothing more than
memorandum agreements will be needed, as the parts of the system are
bound together by the closest inter-dependence; they must buy~and sell
of each other, yet each is entirely free in its own sphere.~

15. In relation to the larger development of the Educational Division,
it may be that an alliance should be made with an old established book
house, especially when it is considered that a great central literary
factory is to result at the main news office. With respect to this, I am
in touch with George Haven Putnam of G. P. Putnam's Sons.~

16. A firm contract is proposed between the General News Office and The
Credit Office, which will provide for a continuing payment to the former
of a fixed percentage of the latter's receipts. The large promotion of
The Credit Office is dependent on its introduction to the News System by
the General News Office. The class newspapers will be important factors
in perfecting the centralized organization of the credit news field.
Thus, all credit information concerning the building industry of the
metropolis, apart from the opportunity news, should be cleared through
The Credit Office, and in furthering this the \emph{Real Estate Record
and Guide} can be made a potent force. It is proposed to develop the
weekly sheet now issued by The Credit Office, under the same name, into
a universal newspaper organ of the banking or credit system; there is no
class journal on modern lines in this field, yet the necessity of such
an organ is imperative.~

17. ~I invite attention to the peculiar relationship to be occupied by
the General News Office in re the development of revenues for the class
news system. The class or technical papers have reached their present
position through two sources of revenue, subscriptions and the sale of
advertising space. But now, through becoming scientific~centres~for the
registration of fact on a great scale, their expenditures inevitably
rise so that the certain tendency is to outrun the old-time basis of
revenue. It means that the payments from corporations or firms for the
support of a given class journal must come to be in direct proportion to
the place of such firm or corporation in the related industry, in place
of measuring the payments merely by the amount of advertising space
used. In other words, while the editorial or reporting work is
progressing on organic lines, the revenue system has continued on a
merely individualistic basis in accordance with the ordinary notion of
competition. The basis of revenue cannot be changed without the rise of
a general news office through whose action the place and claims of each
class paper can be determined and given universal publicity. In short,
the situation is such that the main news~centre~necessarily becomes a
revenue-getting agent for the class news system. I withhold judgment as
to whether the General News Office should seek by direct method to exact
a share of such enlarged revenues.~

18. Both the General News Office and The Credit Office must be treated
by ourselves as public institutions, though operating under the
principle of Contract, which means that full details as to their
finances must be published at proper intervals.~

19. It is proposed to organize the professional accountants through
registration at The Credit Office, so that the standing of an accountant
will come to be tested by the fact of his acceptance or rejection by the
universal credit reporting~centre. The developed Credit Office will have
rank as the Bank Clearing House, both being organs in the Industrial
State.~

20. I withhold judgment, both in general and detail, as to the wisdom of
placing stock of the General News Office with the class newspapers and
other sub-centres.~

21.~We are to introduce the natural or simple method of classifying and
organizing news. If news can be organized on a scientific basis it
cannot long be handled in any other way. I may add that the principle
laid down is common or easy to men of the mental order of your Mr.
Desmond or Mr. Reinhold.

\begin{center}
\vspace{0.2in}
\noindent\rule{2in}{0.5pt} 
\vspace{0.2in}
\end{center}


I have sought to put down for you the leading
ideas controlling the action in hand. I ask you to join me in the work
of conferring upon commerce the greatest single benefit in all history.
My feeling is that~the character of the enterprise gives it a just claim
upon your large knowledge, your experience, and practical judgment, not
to speak of your direct business interest on account of your prominence
in the class news field. The facts spell Opportunity.~

The force of the principle is such that all related news interests must
enter the proposed combination or be left outside the breastworks. A
universal trading relation is required in the news business as means to
furthering the general advance in co-operation which the new conditions
permit and compel throughout the entire field of trade and industry. In
point of principle, the co-operation in the news field which I am
proposing is nothing more than has already taken place in what I may
call the commerce in physical things; for example, you buy buttons from
other concerns and sell them as parts of the finished garment.~

The path of an enterprise so new and original as this cannot of course
be marked out in advance in any great detail, but it is plain to me that
if~sufficient~freedom can be secured to enable us to revise, print, and
circulate the new literature of Politics, all else will be certain and
easy; all further steps will take place in their natural order, and the
right men will come to hand at the right time. The literary preparation
has been difficult, but I have accomplished it to the letter. The time
element is with us in all respects, and, owing to the nature of the
business, the social forces will further the action at every point. The
chief risk was moral or scientific rather than financial, but I have won
everything on that score up to this writing.~

A credit of a few thousand dollars is needed in order that the boldness
of the first steps may be up to the level of the principle. I ask you to
furnish this credit on such terms as may be agreed upon. I have no
alliances or obligations which can stand in the way of my making an
exclusive contract with you. I am satisfied that with freedom gained for
the initial handling of the new literature, all the succeeding steps
will be insured by the creation of revenues. The individual who will
have the courage to make the initial credit which is required will of
necessity reap a large money reward, since, as already indicated, the
service to commerce will be unexampled. I have already said to you that
the man who will take a new proposition at its face value is about as
rare as its creator.~

A powerful financial institution is contemplated. There is money in the
scheme for all concerned.~
\vspace{0.2in}
\begin{center}
    

Very truly yours,~
\end{center}

\vspace{0.1in}

\hspace{2in}Franklin Ford~~
\addtocontents{toc}{\newpage}
\begin{fullwidth}


% PART TWO
\part{Interconnected Flows: Money, Information, and Transportation}

\end{fullwidth}

% BETTER CREDIT REPORTING
\chapter[Better Credit Reporting]{Better Credit Reporting}
\label{ch:Better Credit Reporting}
\chaptermark{BETTER CREDIT REPORTING}

\vspace{.2in}

\begin{LARGE}

\smallcaps{Franklin Ford}\marginnote{Published in \emph{Textile America}, Vol. 1, No. 17, p. 34–35.}

\end{LARGE}

\vspace{0.5in}


\hypertarget{practical-operation-of-the-credit-clearing-house-in-detail}{%
\section[Practical Operation of the Credit Clearing House in
Detail]{\texorpdfstring{Practical Operation of the Credit Clearing\\\noindent House
in
Detail\footnote{{[}This is~the second of five~articles in a series
  titled ``Better Credit Reporting''~published in 1897~by Franklin Ford
  in \emph{Textile America}.{]}}}{Practical Operation of the Credit Clearing House in Detail}}\label{practical-operation-of-the-credit-clearing-house-in-detail}}

\hypertarget{working-relation-of-credit-departments-to-the-clearing-centersa-division-of-labordescription-of-a-typical-report}{%
\subsection{\texorpdfstring{\emph{Working Relation of Credit
Departments to the Clearing Centers---\\\noindent A Division of Labor---Description
of a Typical
Report}}{Working Relation of Credit Departments to the Clearing Centers--- A Division of Labor---Description of a Typical Report}}\label{working-relation-of-credit-departments-to-the-clearing-centersa-division-of-labordescription-of-a-typical-report}}

\newthought{The first credit report} was the word-of-mouth statement of one merchant
to another concerning his experience with a given applicant for credit.
A great advance in communication had to come about before a group of
merchants could hit upon the~labor saving~measure of employing an agent
to collect such experiences from the several members of the group, to be
distributed in turn to all the members for their common benefit. In a
marked way such action gave proof of a growing community of interest.~

It has been shown that the beginnings of credit reporting in America
date from the advance of post-office facilities in 1840. This or that
group of merchants was then able to conceive~within restricted areas, of
a prompt exchange of experiences. Such action went to disclose an
overwhelming need, and this permitted the rise of Bradstreet's and
Dun's. Their reporting systems,~as everybody knows, gradually spread
over the United States and Canada, parallel with the further growth in
communication. But, as we have seen in place of holding to the principle
of exchanging experiences while overrunning the country, the two
``mercantile agencies'' proceeded to build up an artificial system of
credit ``reports'' and~``ratings''~which to-day, as regards the
important work of determining credits,~has the effect of keeping
merchants apart in place of bringing them into close working relation.
The imperious need of commerce with respect to credit reporting is
proven by the continued extraction of enormous revenues on the part of
the two ``agencies;'' they trade on this want without supplying it.~

\hypertarget{interchange-of-experiences}{%
\subsection{\texorpdfstring{\emph{INTERCHANGE OF
EXPERIENCES}}{INTERCHANGE OF EXPERIENCES}}\label{interchange-of-experiences}}

The great fact to be noted is that under the conditions outlined no
change short of a revolution could take place. The struggle has been to
revert to the original principle of a direct interchange of experiences,
and by giving it universal application to compel the desired reformation
in credit reporting. Illustration of this is disclosed in the attempt of
particular trades, such as the Hardware Board of Trade, and the like~to
improve credit reporting, each on its own account. On the whole, these
efforts have only served to emphasize the need. A clear departure has
been waiting on the discovery that the principle of direct interchange
of experiences may be applied to the entire country without let or
hindrance; that under changed conditions the methods in use at the
outset of credit reporting may be given universal application.~

The clearing house movement in credit reporting is premised on the
completion of the post-office, meaning thereby the omnipresent business
of letter carrying. Between night mail trains and fast time, along with
free delivery at all centers of importance, the swift interchange of
experiences in credit making turns upon adequate organization only; the
conditions therefor have been reached. The post-office---seen as the
whole business of communication---is having further development in the
telegraph and the telephone, but there is no occasion to wait thereon,
as the facilities of the old post-office alone give ample warrant for
employing the clearing-house principle in credit reporting. It is fair
to say, that beyond the addition of pneumatic tubes, the letter carrying
division in the post-office has reached its ultimate; the fast mail
trains may become a little quicker, but no radical change is to be
expected. The progress noted has made possible the Credit Clearing
House.\footnote{{[}Founded in 1886, the Credit Clearing House was a
  mercantile agency specialized in textiles. Ford's role in the Credit
  Clearing House seems in some respect similar to the one he had at
  \emph{Bradstreet's} in the 1880s, as he was publishing articles
  promoting and theorizing the work of the agencies. The exact nature
  and duration of Ford's association with the company remains unclear as
  Ford was involved with many credit agencies after his return to New
  York in 1897, including his own Credit Office.{]}}

\hypertarget{growth-in-mail-facilities}{%
\subsection{\texorpdfstring{\emph{GROWTH IN MAIL
FACILITIES}}{GROWTH IN MAIL FACILITIES}}\label{growth-in-mail-facilities}}

When the system of Bradstreet and Dun was given shape, the mail
facilities were away behind present standards, so much so, indeed, that
in place of organizing to~effect~a direct exchange of the experiences of
merchants in credit making, the thing done was to send interviewers to
merchants and so build up the literary procedure which came in time to
occupy the whole field. In those days something like a week was required
for an exchange of letters between New York and Buffalo, whereas now an
exchange is~effected~between the evening of one day and the early
morning of the second day, a period of thirty-six hours. The letter
carrying division of the post-office reaches out and embraces the whole
people. At a nominal cost---the possession of a two-cent stamp---the
individual may command its services at all times. No town is so small as
not to come within the network of post routes.~Sir
Rowland~Hill,\footnote{{[}Rowland Hill (1775--1878) reformed British
  postal service during the Victorian era. He introduced prepayment (as
  opposed to payment upon reception), postal stamps, and universal
  service, among other innovations aiming at democratizing the
  service.{]}} the projector of penny postage in England, would himself
be more than astonished were he able to-day to contemplate the work of
the American post-office.~

The omnipresence of the letter carrying division of the post-office is
explained by its commercial necessity. The post-office is compelled to
go out~everywhere to the smallest hamlet in order that its patrons may
never be compelled to ask the points to which letters can be sent. Were
it ever necessary to ask WHERE letters can be sent, the post-office as a
commercial enterprise---and it is only this---would thereby be seriously
crippled. The post-office embraces all particulars belonging to its
function, and is therefore a~true universal.~

\hypertarget{a-social-body}{%
\subsection{\texorpdfstring{\emph{A SOCIAL
BODY}}{A SOCIAL BODY}}\label{a-social-body}}

From all this the fact is developed that the Credit Clearing House
premises a social body, having free communication throughout all its
parts. The action, therefore, takes no account of distance, since with
respect to the function to be executed there is no distance. By means of
the machinery of the post-office, all the facts as to the experience of
merchants in credit making may be gathered from all points, with direct
reference to delivery at all points, and this without hesitation as
regards the cost of transmission. Such action is organic and the
revelation is the social organism.~

The idea of news as a thing of trade dates from systematic
communication. Gradually the business of gathering and selling news has
been gaining its true place in commerce until now, with communication at
the full, news has recognition as a commodity, as something to be
universally dealt in on the basis of fact. Proof of this is seen in the
rise of the Credit Clearing House, whose function, in ultimate
outworking, comprises the gathering and sale of all credit-making news.
Its function, as an organ of commerce, is as definite, is as much
subject to clear definition, as the bank clearing house. Moreover, the
Credit Clearing House, as a further step in the organization of
commerce, takes equal rank with the bank clearing house; they are
co-ordinate.~

At the outset of this inquiry the rise of credit departments in all the
great mercantile houses was noted, and further, that the advance had led
to a national organization comprising the heads of such departments. It
was seen that these men had come to be chiefly solicitous as to the need
of improved credit reporting. Accustomed to the ``mercantile agencies''
the first impulse was to petition them for relief. As well might the
advance from iron to steel have been sought through petitioning the iron
men,~where the need was the incoming of a~Bessemer~and a new principle.
But in struggling toward the light, the credit men prompted the
disclosures of this inquiry, and the discovery of all that follows.~

\hypertarget{a-division-of-labor}{%
\subsection{\texorpdfstring{\emph{A DIVISION OF
LABOR}~}{A DIVISION OF LABOR~}}\label{a-division-of-labor}}

The way out is through the far-reaching and absolute division of labor
indicated by the Credit Clearing House. The credit manager of a given
house determines its credits, but to gather up and exchange the
experiences of all credit man in a given circle of trading the reporter
has to function, and this with reference to the whole movement. To the
National Association of Credit~Men,\footnote{{[}Founded~in~June~1896~with~the
  help of the Toledo~Chamber~of Commerce, the~National Association
  of~Credit~Men~started~with~one hundred~credit~managers~across~the
  United States~responding~to Toledo local~credit~manager O.
  G.~McMechen's~call to~unite. The~organization~took~part
  in~economic~and~financial~policy-making and~professional~training,
  published~monthly~bulletins, etc. It~was~later~renamed~as the National
  Association of~Credit~Management.{]}~}~the country is one,~and so the
work of reporting credits has to be dealt with under a like singleness
of vision. The organization of credit reporting runs parallel everywhere
with that of the credit man as such. The experience~of the credit
manager is the fact of the credit reporter, who finds his function in
the necessities of the grantor of credit, the constant desire of the
latter being to act in the light of all the facts.~

\enlargethispage{\baselineskip}

An old Spanish proverb has it that ``when need is highest, help
is~highest,'' and so at the very time when the credit men of the United
States are pushing their organization over the whole country, it turns
out that the clearing centers for credit facts are for the most part
already in place. It is sufficiently accurate to say that one-half the
work of the credit department in a given commercial house is to
prepare~the facts~which constitute from day to day the reports of its
action, for transmission is demanded to the nearest center of~the Credit
Clearing House. The rise of the credit department and the national
organization of credit men are but preparatory to this juncture of
forces. The various credit reporting~``agencies''~are in the way of
merging into one organism. An identity of interest becomes apparent
between the chain of credit departments and the Credit Clearing House,
and the identity is as absolute as that obtaining between the several
offices in a given banking system and its central clearing house. One
reading of the credit department is the rise of exact method, but this
method cannot come full circle unless the facts are cleared through
common centers. While on one side the growth of the credit department
rendered the clearing-house principle practical, on the other the rise
of the Credit Clearing House provides the avenue through which the
credit department men are enabled to work toward perfect organization.~

\hypertarget{identity-of-interest}{%
\subsection{\texorpdfstring{\emph{IDENTITY OF
INTEREST}}{IDENTITY OF INTEREST}}\label{identity-of-interest}}

Credit men cannot attain to systematic effort save through division of
labor. The identity of interest between them and the further progress of
the Credit Clearing House corresponds to that between the head and the
hand in the action of the human body. The centers of the cerebral system
register the action of the hand, and so in turn are enabled to supply
directing intelligence.~

The remark is heard that the merchants should do it themselves. They
might but for the~ever present~distinction between the act and the fact.
The right handling of the fact belongs to the province of the reporter.
The reporting system, in one aspect, is of course the chain of credit
departments, but to achieve rapid and effective handling, the facts are
forwarded to the clearing centers. This is done to save time. The
relation to the credit departments is constant, and the appeal is
directly to the self-interest of merchants.~

As showing the state of things, some merchants who are backward in
accepting the clearing-house~principle have been more energetic than
formerly in trying to correct the ``agency'' deficiencies through house
to house canvassing themselves, much as in other days, while in numerous
instances the special attempt is made to carry on direct correspondence
with the various references given by traveling men. But all this only
goes to reveal the overwhelming economy of the clearing-house principle
and its necessary acceptance by all.~

\hypertarget{an-advance-in-publicity}{%
\subsection{\texorpdfstring{\emph{AN~ADVANCE IN
PUBLICITY}}{AN~ADVANCE IN PUBLICITY}}\label{an-advance-in-publicity}}

The rise of the Credit Clearing House marks a great advance in exact
publicity. Regarded in its ultimate development it stands for the
credit-making aspect of all news.~

The complete interchange of merchants'~experiences in credit making is a
distinct and important step in the organization of experience as a
whole, that great idea of which the philosophers have made so much. It
is impossible, therefore, to exaggerate the importance of putting credit
reporting on the right basis. The integrity of commerce itself lies at
the heart of the question.



\hypertarget{clearing-principle-in-practice}{%
\subsection{\texorpdfstring{\emph{CLEARING PRINCIPLE IN
PRACTICE}~}{CLEARING PRINCIPLE IN PRACTICE~}}\label{clearing-principle-in-practice}}

The practical working of the Credit Clearing House is of exceeding
interest. Light will be gained by examining a typical report, but first
as to the operation of getting reports. As already learned, the ``agency
reporter'' is done away with. The desired facts are contained in
merchants' ledgers, being the actual experience of merchants with credit
seekers. To enable merchants to communicate such facts freely, it must
be done in such a way as not to disclose the private affairs of a given
house. In short, provision has to be made for entire secrecy on one side
and the fullest inter-change of experiences on the other. To compass
this, a key to the lines of trade is provided, the result being to
remove the possibility of members becoming acquainted with each other's
accounts. The facts are read in their public aspect only.~

Each clearing center in the system issues a daily inquiry sheet which
gives a list of the trading concerns on which reports are wanted. These
go to all houses that are members of the Clearing House. This list is
examined and a statement of experience is made out for all the concerns
with which a given credit department happens to be doing business,
provided always that the combined report is desired in each case from
all members of the Clearing House belonging to the particular trade.
There is no compulsion here, as any member of the clearing system may
refrain from reporting in a given case. It is provided, however, that if
a report is~desired~one must be given in exchange. The plan enables a
merchant to trade his single experience for the experiences of all
within the trading circle.~

Reports so made up are matters of fact; they are true. The Credit
Clearing House is therefore under no necessity of publishing that the
accuracy of reports cannot be guaranteed;~instead~the reports carry
their own guaranty. In each case the merchant is the reporter, and he
cannot afford to do otherwise than report truthfully. For one thing,
each member contracts with the center that all reports of experiences
shall be honest and faithful on pain of forfeiting the Clearing House
relation. Such reports tell what a trading concern is doing in place of
what it is saying of itself. Each report is a story of action. By means
of the clearing-house~principle,~credit reporting is removed at a bound
from the region of opinion.~

\hypertarget{checking-unsafe-credits}{%
\subsection{\texorpdfstring{\emph{CHECKING UNSAFE
CREDITS}}{CHECKING UNSAFE CREDITS}}\label{checking-unsafe-credits}}

Under the clearing-house principle the multiplication of unsafe credits
becomes impossible, as unusual action in any part of the organism is at
once felt in every other part. With the system fully extended fraudulent
practices in ordering goods could not get a foothold. The moment
it~were~attempted suspicion would be aroused and a record made. Defects
of character, in the sense of loose morals or incompetency, are
registered the moment action is affected. In the endeavor to make
unwarranted purchases and so contrive a profitable failure, a frequent
trick is to buy outside of the natural market; through the
clearing-house principle such action is at once detected.~

On the next page will be found a copy of the reporting blank furnished
to all members.~

These blanks facilitate the making of reports which reveal the actual
conduct of traders with their creditors; they tell how much a man owes,
how many goods he is buying and where he is buying them. Reference to
the key shows at a glance the lines of trade in which purchases are
made. Such reports quickly reveal a tendency toward bankruptcy, so that
credit reporting under the clearing-house principle is something more
than a commercial death record. The work of science here, as everywhere
else, is successful prediction.~

\begin{figure*}  
   \includegraphics[width=\linewidth]{graphics/image-thirteen.png} % the image
   \label{fig:fig13}
\end{figure*}

\hypertarget{ultimate-effect}{%
\subsection{\texorpdfstring{\emph{ULTIMATE
EFFECT}}{ULTIMATE EFFECT}}\label{ultimate-effect}}

When the force of the principle is considered, this opinion from a
leading dry goods house is not surprising: ``The universal use of the
system will revolutionize the methods of handling credits.'' The
principle makes rapidly for the highest possible morale of commerce, as
in the most direct way it brings into play the influence of the highest
character and ability on the poorest, the effect being to bring up the
whole. With the principle in full play, the morale of commerce will
closely approximate the character of its~foremost members. The old saw
that ``trade is~corrupting'' is destined to become worse than
groundless. It is indeed worth recalling that the man of business
invented the idea of integrity as test of character; it grew out of a
transaction in trade. The church has been crying for centuries that
honesty is the best policy; commerce is in the way of proving it.~~



\hypertarget{a-typical-report}{%
\subsection{\texorpdfstring{\emph{A TYPICAL
REPORT}}{A TYPICAL REPORT}}\label{a-typical-report}}

The typical report referred to was prompted by an inquiry concerning one
of the largest department stores in the central West. The inquiry
originated~at New York at the hands of a house in receipt of a first
order. The department store in question first appeared on the New York
inquiry sheet January 7, 1897. Forty houses in the State of New York at
once responded with their experience in detail. Twelve in Pennsylvania
followed, six in Maryland, five in Massachusetts, eleven in Ohio, two in
Kentucky, one in Wisconsin, one in New Jersey, five in Illinois, three
in Michigan and one in Indiana; a total of 87. At the time this
department store~was buying from probably 130 wholesale houses. The
experiences of so large a proportion of creditors revealed just what the
store was doing, and the resultant, telling as it did of straitened
circumstances, worked benefit on all sides. The facts could not have
been reached without the clearing-house principle.~

This department store was buying in New York, Philadelphia, Baltimore,
Cincinnati, Cleveland, Louisville, Milwaukee, Chicago, Trenton, Detroit
and Indianapolis. The report when made~up covered all the points named.
One report, therefore, under the clearing-house principle, is all
reports. In such a case one can imagine the absurdity of an ``agency
reporter'' trying to get at the realities through any number of
interviews.~

Three days after this department store inquiry appeared on the New York
sheet, returns were in from the local region---New York, Philadelphia,
Baltimore, Boston---and in seven days Chicago and Milwaukee had
reported. The credit departments interested are prompted to quick
replying for they are all anxious to get the combined report of
experiences; the play of interest is automatic. Each member of the
Clearing House is a reporter, and this both at the initial point and
when receiving the combined experiences, which are so tabulated that the
trained eye quickly discovers the amount owing, amounts overdue, and the
like. In this way skillful tabulation amounts to a synopsis. The grantor
of credit is the only possible expert in translating~each report; the
interested man is always the expert.~~

\hypertarget{freedom-of-action}{%
\subsection{\texorpdfstring{\emph{FREEDOM OF
ACTION}}{FREEDOM OF ACTION}}\label{freedom-of-action}}

The whole machinery of the Credit Clearing House is at the service of
each member, just as the post office is subject to the will of any
individual who is armed with a~two cent~stamp. Any member of the
Clearing House may start an inquiry at pleasure with the certainty of
receiving promptly the experiences of all other members. The clearing
center receiving an inquiry knows where a given concern is buying, and
sends to such centers accordingly for entry in the inquiry sheets.~

\hypertarget{integrity-of-the-center}{%
\subsection{\texorpdfstring{\emph{INTEGRITY OF THE
CENTER}}{INTEGRITY OF THE CENTER}}\label{integrity-of-the-center}}

The integrity of the clearing center is protected by its own
self-interest, by its own selfishness, as the organization can have no
prosperity save on the basis of absolute integrity. In illustration of
this Mr. Thomas S. Couser, manager of the Credit Clearing House at New
York,\footnote{{[}Thomas S. Couser was later associated with Ford as
  they teamed up to operate the ``Credit Office'' at 346 Broadway, New
  York, circa 1904--1906.{]}} has kindly permitted me to make a
transcript of certain correspondence. On August 6 last, a member of the
Clearing House addressed the New York office as follows:~

\begin{quote}
For certain reasons we would prefer not to have our experience on \ldots~appear on report. We enclose blank showing that we have his order for
\$------. This is the first time that we~have asked anything like this,
and it will probably be the only time, and we would appreciate it if you
could send us report as requested.~
\end{quote}

This application elicited the following reply from the Credit Clearing
House:~~

\begin{quote}
We return your ticket on . . .~and in reply to your letter would say
that, as you will note by our contract, we are conducting an interchange
of absolute experiences, and under no circumstances could we give the
absolute experiences of the combined trade without receiving first from
you your complete experience. That is the basis of equity upon which the
clearing system is founded. If we made an exception in your~case~you
would be justified in thinking and believing that we made exception in
the case of others, and the system would fall to the ground.~We regret
that we cannot accommodate you in this, but feel sure that you have
nothing whatever to fear from giving us your complete experience in this
case, as we happen to know this man personally and think there could be
nothing in your dealings which would injure you,~him or the trade.~
\end{quote}

The inquiry sheet of the Credit Clearing House has two groupings. Dry
goods and such allied interests as boots and shoes and clothing make up
one grouping, while groceries, hardware and other things as nearly
kindred as possible~constitute the second grouping.~

\hypertarget{signed-statements}{%
\subsection{\texorpdfstring{\emph{SIGNED
STATEMENTS}}{SIGNED STATEMENTS}}\label{signed-statements}}

The Credit Clearing House adds to the exchange of merchants'~experiences
wherever possible the signed statements of the trading concerns on which
inquiries are~made. In all cases application is made for such
statements, and a very large proportion quickly respond. Various causes
are contributing toward the fullest freedom in the giving of statements
by traders, one of these being, of course, the influence of the Credit
Clearing House itself. At the same time, the banks are insisting upon
signed~statements from applicants for advance credits in a greater
degree than ever before. Such statements in the hands of the Credit
Clearing House go only to those having a right to examine them. In fact,
any exchange of experiences under the clearing house principle is in the
nature of a privileged communication.~

\newpage Great interest attaches to determining the exact progress of the
clearing house principle in credit reporting;~also, to the influences
making for its complete acceptance, with a reference to the obstacles in
the way. Beyond, there remain the relation of the system to the bank
clearing house, and the wide effect of the development on the
organization of commerce as a whole.~

\vspace{0.2in}

\hfill FRANKLIN FORD

\begin{figure*}
    \centering
    \includegraphics{graphics/image-fourteen.png}
    \centering{\emph{Advertising piece for the Credit Clearing House published in the \emph{American Wool and Cotton Reporter}\\ (December 5, 1912), page 1587.}}
    \label{fig:fig14}
\end{figure*}



% TRAFFIC ASSOCIATIONS
\chapter[Traffic Associations]{Traffic Associations}
\label{ch:Traffic Associations}
\chaptermark{TRAFFIC ASSOCIATIONS}

\vspace{.2in}

\begin{LARGE}

\smallcaps{Franklin Ford}\marginnote{Published in the \emph{Wichita Daily Eagle}, May 4, 1897.}

\end{LARGE}

\vspace{0.5in}



\noindent (From ``Textile America'')

\newthought{The decision of the} supreme court in the case of the Trans-Missouri
Freight Association\footnote{{[}United States v. Trans-Missouri Freight
  Association, 166 U.S. 290 (1897) was a Supreme Court decision
  confirming that railways were common carriers and that anti-trust
  provisions applied to them. Ford disagreed, and so did his most famous
  correspondent, US Supreme Court Justice Oliver Wendell Holmes Jr. See
  Holmes to Ford, May 3, 1907 in David H. Burton, \emph{Progressive
  Masks: Letters of Oliver Wendell Holmes Jr., and Franklin Ford}
  (Newark: University of Delaware Press, 1982), 43--44.{]}} went to
disturb business confidence because of a lack of understanding as to the
nature and effect of such efforts on the part of railway managers. It is
not yet perceived that some method of bringing the complex freight and
passenger traffic under a common direction is as necessary as was the
introduction of the telegraph to the movement of trains. When it was
first proposed to run trains by telegraph a deal of opposition was
aroused, and numbers of railway men resigned because of it. The attacks
on the traffic associations are due to ignorance of their true function,
and are certain to disappear through the growing effectiveness of the
very machinery that is condemned.

The traffic association in the railway system is of the same nature and
function as the clearing house in the system of banking, each being
instrumental of self-regulation---the natural outgrowth of new
conditions. The New York Clearing House is without so-called legal
incorporation, yet it is gradually assuming greater powers as an organ
of government, and this without exciting even passing remark. The boards
of trade are centers of regulation for the several businesses connected
therewith, their purpose being to facilitate transactions and compel
integrity therein. As communication, seen as a distinct branch of
commerce, reached administrative unity in the post office, so now the
great business of transportation is struggling for its unity as the
means of passing to the highest economy and effectiveness. The common
instrument to this end is the derided traffic association.

Happily, the supreme court decision and all other forms of obstruction
will go to further self-regulation for the railway. They will aid to
clear away misunderstanding, and to teach the necessity of avoiding
conflict with wayside opinion by finding the most direct road to the
goal that must be reached. The notion is widespread that regulation of
``commerce between the States'' is only possible through the
intervention of congress; the movement of commerce without such
interference is thought to be lawless. The opposite of this, that
commerce of necessity works out its own regulation, cannot be proved
save by actual progress. A showing of fact here on the port
{[}\emph{sic}{]} of the railway managers would be of value. But any such
report would be lacking in point were it not boldly prefaced by the
assertion that the traffic associations are themselves of wholesale
government: that all the governing is not done from Washington and
Albany, however loud the clamor.

It is difficult to see how a clear solution can be gained without
defining the issue as one between ``the government at Washington'' and
the railway association as an organ of self-government. The new
legality, in fact, is to be seen in the outworking of the transportation
system itself under the guidance of its own inherent law. Congress may
give registration to this law when determined by the action, but it can
do no more. Commerce is everywhere working toward the highest quality of
service at the lowest price, involving the widest distribution. One
means to this end is an enlarged publicity. It is fair to say that the
true measure of the service of the inter-state commerce commission is
its work as an agent of publicity, especially as one effect thereof has
been to promote uniform railway accounting.

Railways have to be run by railway experts---there is no other way. The
railways in Belgium or Switzerland may be run by a bureau of ``the
government'' or for that matter ``owned'' by it, but for the sweep of
country lying between Maine and California it will have to be done by
experts---by science, which is government. Again, it will be done by the
individual, who is everywhere the instrument in man\textquotesingle s
action.

The idea of ``government ownership,'' so far as it has any claim to
attention, is simply the dream of the unified transportation system.
There is still a widespread fear of unity, which should, however, find
correction through observing the post office, whose perfect diversity is
due to the fact that it is operated as a single system. The railway
traffic associations stand for the approaching unity of the
transportation interests.

It is easy to lay down the rule that freight rates are to be equal and
uniform, but this, like other good things, has to be worked out in
practice. The problem is vastly complicated owing to the fact that the
railway tariff continues to be based on distance, and, further, that it
varies as to classes of goods. Statutes assert that rates shall be
uniform, but it so happens that the statutes fail to provide for changes
in rates. From the very nature of things, important reductions in
freight charges came about in the past through the aggressiveness of
individuals, who by offering guarantees of large business or by other
means secured for themselves advantages which were not at once given to
the whole people. The need is to reach such a stage of organisation that
successful warfare by individuals for lower rates will take effect an
{[}\emph{sic}{]} once for all shippers. To get to this the governing
intelligence has to be centered in such bodies as the traffic
associations: in other words, the railway men have to provide for their
own law-making organizations, and this has to be made up by selections
from their highest experts.

One source of confusion and error lies in the prevailing belief that the
associations have as their end the maintenance of rates at a certain
high level, and for this the traffic managers are largely to blame. The
end is to regulate and govern rates and to arbitrate the various
questions which are constantly arising in railway administration. It is
not possible for any set of managers, however compact the organization,
to resist the countless influences which are determining railway freight
and passenger charges. To accomplish it they would have to hold down the
forces of nature, and put an effective check upon the inventive genius
of man. A given traffic association is but the instrument through which
the competitive forces are moving, and these forces gain freedom and
effectiveness just in proportion to concentration of power---to unity or
administration. The locomotive is competing for its place in the round
up of commerce, and all possible traffic associations can only
facilitate the action.---Franklin Ford

% THE COUNTRY CHECK
\chapter[The Country Check]{The Country Check}
\label{ch:The Country Check}
\chaptermark{THE COUNTRY CHECK}

\vspace{.2in}

\begin{LARGE}

\smallcaps{Franklin Ford}\marginnote{Issued by Fords, June 20, 1899.}

\end{LARGE}

\vspace{0.5in}
\begin{center}
    

{\huge FORDS}

\vspace{0.2in}

{\LARGE 68 Broad Street, New York}

\vspace{0.2in}

{\large FRANKLIN FORD, Director}

\vspace{.35in}

{\LARGE THE COUNTRY CHECK}

\vspace{0.2in}

{\Large \emph{IT MEANS A SINGLE BANKING SYSTEM
AND A
UNIVERSAL~CHECK}}

\vspace{0.2in}

{\Large THE TRUE MONEY QUESTION}

\vspace{0.2in}

{\large \emph{Action of the New York Banks and its Real
Import}}

\end{center}

\vspace{0.2in}

\begin{bfseries}
    
\begin{hangparas}{0.25in}{1}
    

From Gold Coin to Bank Check---The Old Order Changes---The Banking
System a Self-governing Body under the Control of its Clearing
Centers---New York is Main Center and its Clearing House Legislates for
the Country in Money Matters---The Express Companies and the Banks---A
Single Language for the Money Question---Credit a Commodity and Money
the Instrument of Transfer---The Bill of Exchange~has~Universalized in
the Bank Check, but is Needing Regu-\\\noindent lation---The Check Book to be
Certified---A Kiting Check the New Wildcat Money--- The Clearing~House
Certificate a New Legal Tend-\\\noindent er---Need of Better Credit
Reporting---Congress Not in Touch with the New Realities---The Bank as
Regulating Organ in the System of Commerce.

\end{hangparas}

\end{bfseries}

\newpage New York has in hand a~``country'' check problem just as fifty years ago
it had to deal with~``Michigan,''~``Wisconsin''~and~``Indiana''~money.
The unregulated~check is the wildcat money~of~to-day.~~

The deposits in the banks~of the United States outran their note
circulation about~1855, and in due course~the~bank-check came~to be the
usual medium for paying~balances between traders in given localities.
After 1885 a change set in.~After~long use of checks in a local~way,
traders began sending them to remote towns, and~finally anybody who had
a bank account anywhere insisted upon sending~his check everywhere. The
movement went on without attempt at correction until recently when New
York bankers found themselves in daily receipt of some twelve million
dollars~of~``country''~checks. Checks are received at New York from
Pacific coast points for less than a dollar. In the early practice of
the Bank of England checks below ten pounds were not permitted.

\enlargethispage{\baselineskip}

Out of this has come the transference of the cost of~domestic exchange
from the remitter (where it properly belongs, as payment for service to
him) to the banks and merchants at main centers. When a country bank
receives for clearing or collection a check on itself, it makes an
average deduction of one-tenth of one per cent, showing the
charge~which~the~local bank takes for clearing, or paying, checks drawn
on itself. Gradually this expense came to be saddled on the banks at
main centers, and it is this charge which the New York banks are putting
back on their~depositors.~

Through exacting a commission for collecting checks on themselves, the
country banks have been trying to hold on to their old revenue from the
sale of domestic exchange.~On account of the~growing use of
personal~checks, the country banks no longer charge their depositors for
drafts on New York or Chicago, and so the remitter has mostly got rid of
paying exchange on his remittances. In fact, the country bank encourages
the use of personal checks for remote points as it is able to exact a
commission when asked to remit. It is altogether a question of revenue,
and has resulted in a scramble among the banks to get from receivers of
checks a change for collecting.~

It is sure paper all the way through. The country debtor remits his
personal check, and when it is returned the bank on which it is drawn
sends in payment its own draft on New York, Chicago or~other~center.~

The action of the New York Clearing House\footnote{{[}Founded in 1886,
  the Credit Clearing House was a mercantile agency specialized in
  textiles. Ford's role in the Credit Clearing House seems in some
  respect similar to the one he had at Bradstreet's in the 1880s, as he
  was publishing articles promoting and theorizing the work of the
  agencies. The exact nature and duration of Ford's association with the
  company remains unclear as Ford was involved with many credit agencies
  after his return to New York in 1897, including his own Credit Office.
  For a detailed discussion of the work of the Credit Clearing House,
  see \emph{Better Credit Reporting}.{]}} shows the real condition of
banking in America. It ordered that from April 3 the country should be
divided into two~parts, and that all checks on points north of the
two~Virginias and Kentucky, inclusive, and east of the Mississippi
River, inclusive of Missouri, should be charged one-tenth of one per
cent by the collecting banks. Checks on points further west or south
must pay one-fourth of one per cent. It was ordered at the same time
that charging should be discretionary for checks on Boston, Providence,
Albany, Troy, Jersey City, Bayonne, Hoboken, Newark, Philadelphia
and~Baltimore, the upshot of this being that items on the~points named,
and including New York, are cleared without cost to depositors.~

As finally amended the rules fix a minimum charge of ten~cents on single
checks, and this while permitting all small items deposited at one time
from either group of states indicated to be added together and treated
as one. This discriminates in favor of the large receivers of small
checks, giving~them a rate of exchange hardly more than nominal when
comparison is made with the cost of small postal or express orders. So
that under the plan of the New York Clearing House small checks, in
quantities, on the remotest points in the United States are current in
New York at a rate of exchange varying from one-tenth to one-fourth per
cent.~



\hypertarget{unity-of-the-banking-system}{%
\subsection{\texorpdfstring{\emph{UNITY OF THE BANKING
SYSTEM}~}{UNITY OF THE BANKING SYSTEM~}}\label{unity-of-the-banking-system}}

By the recent ruling a number of startling facts have official
attestation. In the first place, the entire banking connection in
America is now to be seen as a single, undivided system with the New
York Clearing House as the main governing center. New York has
legislated concerning charges for domestic exchange everywhere. This
moves away from the inherited notion that the banking business is
regulated from Washington, or from such local centers as Albany or
Harrisburg. By the terms of the order,~``national,'' ``state'' and
``private'' banks are put upon a level. They are all the same at the
Clearing House, and the whole system appears as a self-regulating body
controlled by its clearing centers.~~

That New York is main center in a system has~been plain for a long time.
This is proved by a comparison of its weekly bank clearings with those
of the whole country. The bank transactions of the nation are told over
again in the New York movement. All the checks all over the country have
to be swapped, or cleared in one way or another. Some are cleared
locally, but the wide movement of checks only gets clearance through a
center common to all. The check is a bill of exchange and follows trade
and~communication, centering at New York. The metropolis as the main
clearing center is thus the growth of the country's necessities, and her
banks must answer as effective instrument. Any shortcoming at the center
throws the trade of the country into disorder. ~

What New York does,~instantly affects every corner. Everybody remembers
the action of the New York Clearing House in 1893 when the prompt
issuing of Clearing House Certificates steadied the country. The chief
bank center and the outlying banks as a whole are held together under
such pressure of mutual interest that~the~entire bank business of the
nation stands revealed as an inter-related system, with its goal in an
organized body working under general rules from its Clearing Houses or
governing centers. It is the National Banking System in fact, answering
to Alexander Hamilton's early dream of a single bank. The banks are to
furnish ``a national currency'' through the check book. Progress in bank
regulation can only be gained through perceiving the growing unity of
the system. It is not possible by any other means to get command of the
facts.~

\newpage Authoritative action at New York is permitted now to an extent which
would not have been tolerated by the banks of the country until
recently, and things which have become commonplace can now be seen as
steps in a development. For instance, New York receives each week by
telegraph the bank clearings at more than ninety centers, extending from
Halifax and Jacksonville to San Francisco, Portland and Los Angeles.
Further, the summing up at the main center is furnished simultaneously
to all outside centers in~order that the summary may give operation to
the whole country. It is by such means that the system is~passing under
a common direction. The Clearing House is public authority in money
matters.~~

And, the rise of the American Bankers'~Association as an organ of
national regulation is in point. Most of the states now have bankers'
organizations, delegates from these are admitted under definite rules to
the central body, or American Association. It is a growth~in
representation. At the other end, the banks in~many states have come
together in local groups with some half dozen counties in each group. By
means of the telephone the bankers of a dozen countries~are enabled to
confer together, so that the experience of each is any time at the
service of all the rest.

\enlargethispage{\baselineskip}

\hypertarget{a-universal-check}{%
\subsection{\texorpdfstring{\emph{A UNIVERSAL
CHECK}}{A UNIVERSAL CHECK}}\label{a-universal-check}}

~It next appears that the bill of exchange, whose handiest form is the
bank-check, has come to universal use in the United States. This fact
has to be grasped before the attempt of the New York Clearing House to
regulate the check movement can be understood, and this advance is one
with perceiving that the check is moving freely through the centers of a
single system. Until the new fact indicated is perceived and understood
by the New York bankers, they must necessarily fall short in their
methods of dealing with the country check.~

Progress to the readiest money has everywhere tallied with the spread of
the bank-check. In the beginning the check was a mere order for the
payment of money, but it has moved steadily toward its own universal as
absolute instrument for transferring values. The bank-check has come to
its farthest use owing to the ease with which debts are paid by
set-off.~

The preferred currency is the volume of bank-checks and, to practical
men, the~``currency question''~of~the hour is how best to regulate the
use of the check. The handy check is so active, the people sending it
everywhere, that the banker is troubled in managing it.~

The kiting check marks the reappearance of the old wildcat
money\footnote{{[}``Wildcat money'' refers to currency issued by
  poorly-capitalized banks between 1836 and 1865, a period during which
  there was no national banking system.{]}} in the developed banking
system, and some way must be found to get rid of it.~The depositor does
not want to be bothered with a local check; he wants one that will go
through all the banks and that nobody will question anywhere. The people
don't want personal checks if they can get better ones. That's why they
buy post-office and express checks. The demand can only be met by a
universal check and it must be furnished by the banks. ~

\enlargethispage{\baselineskip}

To~effect~this the bank authority must be added to all checks and so
get entirely rid of the merely personal check. In short, the check-book
has to be certified. The check must be laid hold of and made a bank
instrument, and this has to be done before a universal charge can be
made for its use. Rightly understood, the fees for clearing or
collecting checks are charges for their use, but the method employed is
lacking in system. Before there was a Clearing House at New York, local
checks were~``collected''~between city banks.~To control~the check,
while at the same time charging for its use, it must be universalized by
uniting in one instrument the credit of the depositor and that of the
bank. It is only by adding the bank authority to all checks that the new
currency with wildcat money or kiting checks can be remedied. The
post-office and express banks have no wildcat money because they control
the issuing of checks by certifying them. People make their own
bank-checks now, but to get post-office and express checks they've got
to buy them. The bank must get its revenue from the man who buys the
check, as the post-office and express banks do. Wildcat money can only
be cured by not allowing it to start, and so the banker must himself
issue all checks. The only safe and absolutely self-regulating check is
the certified check.~

The plan exists to-day in the banker's circular letter of credit, which
amounts to the certification of the checkbook. The express companies
have adopted it by selling their checks in quantities, which in turn
certifies the check-book. No charge is made by the banks for clearing
express checks since payment was exacted at the point of issue. The
express check is a New York check, and that is a world check.~

It appears then that the problem of regulating the use of bank checks
and the necessity of maintaining a revenue for the banks from the
domestic exchange movement are one and the same thing. The bankers have
a complete example as to what must be done in the action of the express
companies. Express drafts are sold in quantities to all comers, and the
charge for their use is uniform throughout the world.~They treat the
world as one. The express companies~are opening~deposit accounts as men
like to deal where they can get the handiest check. To-day the
bankers~are~clearing all express drafts at any point without charge
while demanding a fee for handling bank drafts of precisely the same
character drawn by their fellow bankers. Such an anomaly cannot endure.
The express draft is no longer ``collected,'' but is paid by set-off at
the clearing centers; it is a paper certificate of value, which is
everywhere good. ~

The way out for the banks is through the introduction of an authorized
check for use at all offices in the system. All signs point to the
advent of the American Check along with world-wide acceptance for it. To
make the check universally good, it must be certified from the main
center of the system, New York.~

\hypertarget{the-need-of-revenue}{%
\subsection{\texorpdfstring{\emph{THE NEED OF
REVENUE}}{THE NEED OF REVENUE}}\label{the-need-of-revenue}}

The need of revenue is driving the banks~forward~to this action.
The~interest rate has been declining and a further decrease is
foreshadowed. It is imperative therefore, that a proper charge be made
for the use of the check, and to this end it must be made in fact, as it
is in reality, a banking instrument. The banks are doing more work and
they must get more pay. Certain of them are beginning to exact fees for
the keeping of small accounts, and this may be extended gradually to all
depositors. The interest charge is for maintaining the system, and as
the interest return declines other sources of revenue must be found. The
activity of the banking system has outrun its regulation and adequate
remedies have to be provided.~The~plan for a universal~check has already
been broached at meetings of the American Bankers' Association. ~

The banking system is moving forward to its largest utility in the
interest of depositors and people alike. The banks are clearing postal
and express drafts as low as twenty-five cents. The American post-office
order or draft would break down were it not permitted to clear at the
centers of the banking system. When the post-office order was introduced
in 1864, the aim was to keep the system entirely distinct from the
banking movement; postmasters~were~``prohibited from depositing money
order funds in any bank.'' So late as 1890, Postmaster General Wanamaker
spoke of the order as `` a means of remitting small sums without
interfering with vested banking interests.''~Yet the post-office has of
late been compelled to give the order the form and character of a
banker's draft. The daily receipt of postal orders at New York is now
about \$50,000, one half of which is paid by set-off through the
Clearing House. At all leading centers the post-office now pays its
orders through the local Clearing House, and the movement is extending.
These facts are cited to show the growing dependence on the clearing
privilege.~

Certify the check-book, and the banking system proper, would of
necessity, occupy the whole field of exchange. The exchange systems of
the post-office and the express companies were but anticipations and the
business in its entirety is to pass to the banks. With the banks
handling post-office and express drafts in all denominations, it is but
a step, however important, to the sale of small money orders on their
own account. Even now the Canadian banks have introduced a small draft
system as means of competing directly with the post-office and express
companies.~

It is apparent from all this that the move of the New~York Clearing
House in dealing with the country check question is a step in the right
direction, but other steps will follow. The end will be reached when all
depositors at all offices of the banking system are charged for the use
of the check; the domestic rate of exchange will be the price paid for a
check-book. Any attempt to prescribe rates for exchange on the main
center affects every bank that clears through it, and they must be
treated alike for the protection of all interests. The country check
involves the entire movement of domestic exchange, and it must be dealt
with~so as to touch the country equally at all points. So universal is
the check movement that one might as well talk of a country postage
stamp as of a country check. When a check~gets to~New York it may be
defective owing to a lack of bank authority or certification, but it is
not a~``country''~check, if we look at it from the side of its movement
instead of its mere starting point.~

The success of the New York move can only be partial, but clear
direction will result. The New York~Clearing House must legislate for
the banking system as~a whole and not for its home banks and a few
near-by~points. Were the move to succeed, the charges for~domestic
exchange would be loaded upon merchants at~main centers, instead of
being put back upon the remitter where they properly belong. Through the
main center's acting for itself in a narrow sense, by clearing free
for~the home banks and a few favored towns outside, New~York is set over
against all other centers in the system,~with each of them trying to
prescribe a rate of exchange on all points. The crudity of permitting
special~free points in the clearing system is like proposing
that~certain places in the exchange systems of the post-office~and
express companies be made ``free'' points. The~true remedy must unite in
its support the country~banks and the banks and merchants at main
centers.~

\hypertarget{the-boston-plan}{%
\subsection{\texorpdfstring{\emph{THE BOSTON
PLAN}}{THE BOSTON PLAN}}\label{the-boston-plan}}

Happily, the thought of a general free clearing for country checks has
not taken hold at New York. The~state of bank revenues, as I have
indicated does not permit of free clearing, even~were~it practicable on
other accounts. The plan of the Boston banks is to erect a single
clearing center for New England checks and to compel the local banker to
remit at par, which if carried out, would destroy the country banker's
revenue from the sale of exchange. In England, all country checks are
cleared through London, so that there exists there no charge for
exchange to bank depositors, which amounts to the same thing as the
Boston plan. In both cases the area is covered one night's mail. In the
United States, conflicting areas are presented, as a check drawn on a
bank in Maine may be sent to a town in Florida. The check is flying
everywhere.

\enlargethispage{\baselineskip}

The difficulty with which the bankers of the nation have to deal~has
resulted from the overcoming of distance; postage is the same to all
points. Yet the New York plan of correction allows distance to
complicate and hinder. A check drawn on a bank in Rock Island pays
one-tenth of one per cent, while a Davenport check across the river has
to pay one-fourth of one per cent. It was doubtless found necessary to
add Missouri to the one-tenth of one per cent zone, because St. Louis is
in that State. It is not yet a fact to the New York banker that the
system is driving forward to a working unity. He has divided the country
into two parts as though distance intervened to erect a wall, and at the
same time is compelled by natural law of the centering movement of his
business to answer to it as a whole, but hindered by the unnatural
division which he has made. The scheme of regulation can no more take
account of parts or distance than is now the case with the exchange
systems of the post-office and express companies.~

\hypertarget{the-need-of-a-single-language}{%
\subsection{\texorpdfstring{\emph{THE NEED OF A SINGLE
LANGUAGE}}{THE NEED OF A SINGLE LANGUAGE}}\label{the-need-of-a-single-language}}

The discovery that the entire banking connection in America is a single
system compels the adoption of a single language as means to classifying
and ordering the facts. Men are disputing over the money or banking
question owing to the absence of a scientific basis for handling bank
news. To~effect~this, it is necessary to hark back to the beginnings of
money and credit. Men had first to conceive clearly of debt and credit
before the need of instruments for transferring them.~

The function of the banks in commerce is to conduct the system of
payments. Their efficiency turns upon the degree of integrity or
certainty of payment which is attained, upon the mobility or ease of
payment, and upon the wisdom displayed in regulating or governing the
system. But for the presence of an exact standard of payment in the gold
unit, the Clearing House returns could not be added together, while~the
modern banking system is possible owing to the ease of payment by
set-off or clearing. The problem of bank regulation or government is
encountered whenever area or extension has to be provided for.~ ~

\hypertarget{credit-money-and-the-bank}{%
\subsection{\texorpdfstring{\emph{CREDIT, MONEY, AND THE
BANK}}{CREDIT, MONEY, AND THE BANK}}\label{credit-money-and-the-bank}}

In defining credit, money, and the bank or clearing center, each has to
be placed in exact relation to the other two. Credit is the thing dealt
in by the banks; it is their commodity. A loan is an advance of credit,
to be made good whenever called upon or on a day named.
The~bookkeeper's~language is always scientific here; he has no illusion
concerning the nature of a credit and does not confound it with a loan.
Credit is organized in markets. Money, whatever the form it may take, is
the instrument for transferring credit. When money is telegraphed a
credit fact is transmitted. The business of the banker is to register
and certify credits---of corporations, of trading firms, of single
individuals.~A credit is immediately available means. The English courts
decided a number of years ago that a bank deposit, subject~to check, is
``ready money.'' The bank or clearing-house is the regulating center in
the credit system. It is there that the debts and credits of the people
find adjustment.~A bank determines credit alike when certifying a check
and when marking it N. G.~

\enlargethispage{\baselineskip}

\hypertarget{secretary-gage-on-credit}{%
\subsection{\texorpdfstring{\emph{SECRETARY GAGE ON
CREDIT}}{SECRETARY GAGE ON CREDIT}}\label{secretary-gage-on-credit}}

In a late address Secretary-of-the Treasury Gage\footnote{{[}Lyman J.
  Gage (1836--1927) was an American financier who served as Secretary of
  Treasury under William McKinley and Theodore Roosevelt. He is known
  for his role in securing passage for the Gold Standard Act in 1900,
  which established gold as the only standard for redeeming paper
  money.{]}} made a statement that ``credit, with its multiform
instruments, is the real money of commerce.''~``It is created as
transfers of goods and wares take place.'' Had the necessity of
distinguishing money as the instrument of transfer occurred to Mr. Gage,
he might have separated completely the tangle of banking literature,
while introducing to the Credit System, which is the monetary or
accounting division of commerce, embracing in its circle of action the
remotest bookkeeper. The Credit System comprises all the agencies
directly engaged in registering or certifying credits and in adjusting
payments. Men of business pay their debts with their credits, since no
other means of payment exist. Every purchase or sale involves the
transfer of a debt or credit, the accounting thereof being in terms of
money---the universal language of value and bargaining.~

Money has to be defined with direct reference to its function, which has
not changed since gold was first coined at Aegina. In the progress of
commerce gold came to be, after subjection to trustworthy coinage, a
universal certificate of credit; a gold coin conveys a credit fact. The
only possible use of a gold coin, as such, is~either to open or close an
account.~

A primitive certificate of credit is seen in the~``store~order ''~given
by a farmer to his workman; it is good at one place only. In it the farm
hand possesses a credit, and the farmer has discharged a debt, while the
order functions as money just so far as it is transferable. Progress
from this to the gold coin, which is good at all stores, everywhere, was
a change in degree but not in kind, as the gold coin is still a ``store
order.''~

The owner of a gold coin, or a bank deposit, is a creditor of the entire
community; either certifies that he has rendered a service to some one
of its members. The books of the New York savings banks show an
indebtedness of society to their depositors; the savings banks aid in
keeping the score; they are one agency of the Credit System. ~

With the introduction of secure means of certification, such as gold and
silver coin, exchangeable credits began to accumulate at trading centers
in the hands of individuals, and in time men made a business of dealing
in credit or of lending money. The traffic in credit rose in importance
and dignity with the extension of trade and the growing necessity of a
distinct system for effecting commercial payments. Some men accumulate
titles to lands and houses, while others serve the community by storing
up credits. The process has continued until now, when, under more
perfect communication, the bank emerges as center of registration for
the credits of the whole community. The direction of the Credit System
is the banker's~division of labor in the State; he functions as
accounting center of the moving commerce. The credit of each individual
in the community is registered and determined at the banking center. The
bank registers credits, just as land titles are~recorded at~the Register
of Deeds office. The credits of the people are~stored at the banks. The
tendency is constant for everything to get into and through the banks.~

The arrival of a single standard in the gold unit is the central fact in
the growth of integrity or certainty of payment. The use of gold moved
parallel with the conquest of distance, becoming a necessity as
transportation grew and extended itself, until with the incoming of the
bill of exchange and the bank, the need of gold began to lessen. It was
thus that the way was prepared for the adoption of a given weight of
gold as unit of calculation for the entire system of payment, whatever
the instrument of transfer.~

\hypertarget{paying-debts-with-credits}{%
\subsection{\texorpdfstring{\emph{PAYING DEBTS WITH
CREDITS}}{PAYING DEBTS WITH CREDITS}}\label{paying-debts-with-credits}}

Mobility or ease of payment scored a great advance when the sale or
transfer of debts became a legal right. The advance had its grounding in
the systematic bookkeeping which was given to commerce by the
Romans.~The recording of debts assured, creditors conceived of selling
their claims, as against their debtors, and in time the paying~of~debts
with other debts or credits became customary; at the outset, the consent
of the debtor had to~be secured.~The principle of compensation or
set-off,~so commonplace now,~was only legalized after a struggle.~It was
a tortuous road, therefore, to the present deposit-and-check development
in banking, which stands for the largest freedom in the sale and
transfer of debts.~

\hypertarget{bank-regulation}{%
\subsection{\texorpdfstring{\emph{BANK
REGULATION}}{BANK REGULATION}}\label{bank-regulation}}

There remains the supreme question of bank regulation, which is seen in
a new light under American conditions. Modern banking is the organic
development of credit and of clearing. The difficulty of bank regulation
from Washington has increased just in proportion as the
deposit-and-check system has obtained, until with more rapid growth in
deposits bank government from Washington has become impossible. The
deposit-and-check system is so inter-woven with the acts of the
individual that its government must be part and parcel of the movement
itself; at this point commerce provides new centers of control. Since
proprietary control, as exemplified in the Bank of France, is impossible
here, the alternative is self-regulation through the Clearing Houses. ~

The possession of the clearing privilege in the banking system is power.
As one hand cannot get to the other save through the brain center, so
one bank can only compass the trading circle through the clearing~house.
The bank at Harlem touches the National City Bank through the clearing
center in~Cedar Street. By means of the clearing system, the banks of
the country are in organic relation with each other, which means that
each center is connected at once with all other centers. The accession
of power on the side of the Bank Clearing House is constant.~

\hypertarget{the-clearing-house-certificate}{%
\subsection{\texorpdfstring{\emph{THE CLEARING HOUSE
CERTIFICATE}}{THE CLEARING HOUSE CERTIFICATE}}\label{the-clearing-house-certificate}}

The issuing of Clearing House Loan Certificates at important junctures
from 1860 to 1893 is the signal illustration of self-regulation through
the clearing-house. When forbidden by the statutes to continue
discounting, the banks have kept on under authority from the Clearing
House, which speaks for the system. A new governing unity intervened. It
is extremely important that the country should be brought to understand
the real meaning of the Clearing House Certificate. In it a new legality
is presented. The action is in no way different, in point of principle,
from the day by day business of all the banks. Men take to the banks
their gold, their greenbacks or bank-notes, their own notes and those of
their customers, their drafts on traders pinned to bills of lading,
their stocks and their bonds. The gold may be underweight and so subject
to a discount, while various judgments have to be passed on the value of
the securities presented, but the end is to have the bank determine the
amount of credit to be awarded. When the banks, acting singly, found
that to continue certifying credits, or discounting the notes of their
customers, they must do~so in defiance of the statutes, resort was had
to the authority of the Clearing House. The procedure amounts to
re-discounting the notes of customers by a pledge of stocks, bonds and
commercial paper at the Clearing House or central office of the system.~

It has been proposed to ``legalize'' the Clearing House Certificate by
means of a statute from Washington, but this is not necessary since the
action carries its own justification. A true legality is something
against which it is impossible to legislate. In the face of decaying
statutes a doubt arose as to what was good money and the New York
Clearing House presented a new legal tender. The growth of the
deposit-and-check system has been such that a new universal is required.
At one time the National Treasury stood for the largest fiscal unity and
therefore had the say through the outgivings of Congress as to what is
good money, but gradually the sovereignty has passed to the Bank
Clearing House. Commerce, the transforming agent, is carrying forward
our ideas of law to new issues and new legal concepts.~~

\enlargethispage{\baselineskip}

\hypertarget{the-idea-of-legal-tender}{%
\subsection{\texorpdfstring{\emph{THE IDEA OF LEGAL
TENDER}}{THE IDEA OF LEGAL TENDER}}\label{the-idea-of-legal-tender}}

To universalize the check by attaching to it the bank authority is but
to extend the principle of the Clearing House Certificate. The legal
tender idea is of modern development. It began with the king's edicts
fixing the ratio between gold and silver during the progress to the gold
standard. In these edicts the king certified what was good money. When
the universal check is brought in, under the authority of the banking
system, a new legal tender will have occupied the field. The incoming
American Check will have back of it the organized credit of the nation;
it will be legal because customary and everywhere good; the legality
will follow upon the fact. Under old-time notions it is thought that
only the ``Government at Washington''~can speak for the whole in money
matters, but in the face of a unified banking system this idea has now
to be given up.

\enlargethispage{\baselineskip}

The gold standard will continue because, as things are, commerce can no
more get rid of the gold unit than it can dispense with the
multiplication table. In all such matters parliaments have only
registered the decrees of commerce. It is a natural thing at this time
for gold to find its way into bank reserves, and were all the statutes
relating to the matter to be repealed, the situation would not be
essentially altered. Since the artificial rules as to bank reserves
remain in the statutes to cause distrust and panic, it is important that
the presence of a new legal tender, namely, the Clearing House
Certificate, be published to all the world. A way must be found to get
rid of the uncertainty.~

\hypertarget{sound-banking-to-follow}{%
\subsection{\texorpdfstring{\emph{SOUND BANKING TO
FOLLOW}}{SOUND BANKING TO FOLLOW}}\label{sound-banking-to-follow}}

The certification of the check will act as the final unification of the
banking system. The change will be of such a character that each banking
office will become responsible to all the banks for right conduct. The
highest co-operation is presented. Sound banking will be assured on all
sides as each clearing center will see to it that every bank in its
district pays for its checks, which each sub-center will be held
responsible to the main center. The method is already at work
successfully in the exchange systems of the post-office and express
companies. The final organization of the banking system, under the
certified check, presents a free trading relation throughout all its
parts under an administrative unity. An example is presented in the
post-office, which is a completed unity. The banking system can but obey
the law of commerce, which moves everywhere toward to highest quality,
the lowest price and the widest distribution of product. There can be
but one system of credit and exchange. ~

With the certified check in universal use, the money of the country will
be furnished by all the depositors in all the banks. The money supply
will then, in fact, be equal to the volume of trade. The currency
question will be settled permanently and the issuing of money will be
the privilege of everybody who can command a bank account. Then as
always the regulation of the money system will be a ``function of
government,''~save that the centers of control will be the Bank
Clearing~Houses. But every member of~the community will have a hand in
the work of regulation through his daily acts. The individual is to
issue the money of the world, and the long discussion is ended.~

A profound change is pending in our ways of thinking concerning these
matters. Under the increasing registration of the credits of the people,
the place of so-called private capital in banking operations is
constantly lessening. The capital account of the Clearing House banks of
New York is now some ten millions less than in 1860. In each of the last
two decades the deposits in the New York banks have doubled, and they
are now promising to double again.~The ``capital'' of the banks~is
coming to be seen as the first deposits. Here and there throughout the
country the tendency is to reduce the capital account in banking in
order to escape hurtful taxation, but the real condition of such banks
does not alter as the~``capital'' remains in the shape of deposits. It
is only a change in bookkeeping.~

\hypertarget{washington-not-in-it}{%
\subsection{\texorpdfstring{\emph{WASHINGTON NOT IN
IT}}{WASHINGTON NOT IN IT}}\label{washington-not-in-it}}

The notion that the banks can only be regulated from Washington belongs
with the idea of a single centered State, which is bound up in the pages
of Blackstone. Under old-time conditions, when a week was required for
carrying the mails between Philadelphia and Pittsburg
{[}\emph{sic}{]},~``the Government'' was the one regulating organization
having a common extension over a given area, whereas now, under full
communication, a number of governing organs whose extension in each case
equals that of ``the State'' have come in. The State becomes an object
in space~and time, and the business of government is a division of labor
therein. As one among other regulating organs the Bank occupies a
central position, as it has directly to do with every division of
commerce. All men may not eat wheaten bread, but every man is compelled
in some way to possess credits and money for transferring them. It is of
the first importance, therefore, to raise up this idea of the bank as
REGULATING ORGAN IN THE SYSTEM OF COMMERCE, in order to dispel the
notion of the banker as a usurious money lender, who is supposed to act
under merely arbitrary rules. Besides, it is important to bring the true
work of the banker to recognition in order that bankers may themselves
be freed from a false dependence on the legislatures and Congress. ~

The banker is an agent whose work is to register and certify the credits
of the people. One can see this in the county banking center, which is
connected by telephone with all local points in order that the debts and
credits of the county may clear freely against each other.~

The regulation of the system will proceed under the largest publicity.
Self-government for the banking system through the Clearing Houses
cannot be divorced from right in any case; the act must be paralleled by
the fact. Never before was government so responsible.~~

The bank is a place for setting off credits against debts. Any debt is
cleared when a credit is set against it. The bank is itself a
clearing-house and the Clearing House proper only a larger bank. The
movement of the certified check through the centers of the clearing
system may now be seen in the express check. The twelve express
companies act as a unity, any one of them clearing for all the others.
When an express order is once cleared or paid, it is extinguished as in
the case of the Bank of England note, which is never reissued. At
London, all clearing-house balances are paid in checks on the Bank of
England. A universal paper instrument has to intervene in the United
States, which will be cleared or extinguished when it gets home, and
which will pay a debt at every turn or transfer. The chief clearing
centers are now marked out in the reserve cities.~

A number of ideas once thought to be visionary are coming to reality in
the certified check. The dream of mutual banking is brought to the fact,
while the~``labor value check'' is an everyday affair. The notion that
the Washington authorities, who are thought to be the sole
representatives of the general interest, should issue the money of the
country is met by the discovery that the individual is supplying the
currency with his checkbook, through the bank as regulating center.
Whatever the modifications in the use of the~check, the direction is
with the banks.~

It is the natural course of things that the deposit-and-check system
should come to a universal through the Clearing House Certificate and
its reduction in the certified check. The banker's circular letter is
now preferable to gold for transferring credit to remote points, yet at
one time gold was the only available letter of credit when among
strangers. At one time England had numerous local mints which were
brought to a unity in the national mint.~There never will be a universal
coinage. The commercial world is now coming to its unity with the paper
certificate of credit as the universal money. When men take gold to a
bank and receive for it a mere entry in a pass-book, it becomes plain
that the office of the mint in marking the coin is the same as the
bookkeeper in making the entry.~~

\hypertarget{credit-reporting}{%
\subsection{\texorpdfstring{\emph{CREDIT
REPORTING}}{CREDIT REPORTING}}\label{credit-reporting}}

The test of sound banking is the safety of all the transactions. The
bank can register and certify only such credits as are based on fact;
the bank cannot certify beyond the fact. A forward stride in sound
banking is, therefore, dependent upon a clear advance in credit
reporting. The business of credit reporting is to-day fifty years behind
the present means of communication. A crude method has been everywhere
extended, but that is all. The clearing principle has to be introduced
at all points in this important field of news. The check clearing has to
be paralleled by a clearance of fact. The banks have to clear the facts
in their possession as to transactions in commercial paper, so that the
information held by each may be distributed for the benefit of all. The
facts are centering at the banks. Already the banks at one or two
centers are doing something toward clearing their own credit news. Apart
from the banks, certain steps have been taken to provide a clearing
system for the facts as to merchandise credits, but free play for the
principle has yet to be gained. To become universal the method must be
public in all ways through inviting publicity to itself. The institution
that would clear credit news must first clear itself. A necessity exists
for subjecting all credit reporting organs to the fullest publicity, as
means to promoting the needed reform. The banking system has itself come
under a searching publicity which has now to be extended to the business
of supplying credit news. The responsibility of bankers is increased
owing to the rise of industrial securities, which in turn is demanding
better credit reporting. The lower rates of interest necessitate
increased watchfulness, and always the key is advance possession of the
fact. The situation is in the hands of the banks, and it lies in their
power to compel reform. ~~

\hypertarget{the-real-currency-question}{%
\subsection{\texorpdfstring{\emph{THE REAL CURRENCY
QUESTION}}{THE REAL CURRENCY QUESTION}}\label{the-real-currency-question}}

The advance of the banking system to its farthest utility rests with the
bankers themselves. They are confronted with the fact of their own great
success. I have shown that the need of regulation and the revenue
necessity both point to the certification of the check. But more is to
be said. There is a currency question of the most pressing character,
and its solution turns upon bringing the check to universal use by
adding to it the bank authority. On this point no help can be had from
Washington. The remedy is to facilitate the shipping of bank credit ``to
move the crops.'' The instrument is the certified check. Rightly
understood, Congress has no authority in the premises, it cannot get
into touch with the new~realities. Congress meets once a year, the
Clearing House every day. There can be no further legislation of any
moment from Washington concerning the currency question. Owing to
radical changes in conditions, the question has been withdrawn never to
be returned. Gradually one thing after another has been removed from the
gaze of parliament. At one time the English parliament tried to direct
the making of pins. The swift moving action constantly outruns the slow
work of Congress. The plans of all the currency reformers look to
emitting bills through the banks, but wherever the bank goes, there goes
the check as preferred instrument. The notion of branch banks is right
enough, but each bank is appearing as an office of the system. Every
deposit-and-check bank is now a bank of issue through the power of
certification. The~dislike of banks and the demagogy which fosters it
cannot be removed by publishing primers for farmers\textquotesingle{}
reading or anything of the sort. The prejudice can only be overcome by
carrying forward the banking system to its utmost usefulness. Such
statutes as relate to a state of things which has passed away will have
to be ignored. It may be that another crisis like that of 1893, and a
further improvising of currency by means of the certified check, will be
necessary to compel the full and systematic application of the
principle, but in any case the masters of banking will be brought face
to face with the responsibility under which they are resting.


% THE EXPRESS COMPANIES AND THE BANK
\chapter[The Express Companies and the Bank]{The Express Companies and the Bank}
\label{ch:The Express Companies and the Bank}
\chaptermark{THE EXPRESS COMPANIES}

\vspace{.2in}

\begin{LARGE}

\smallcaps{Franklin Ford}\marginnote{\emph{Bank News Bulletin}, No. 2, issued by Fords, February 8, 1899.}

\end{LARGE}

\vspace{0.5in}

\begin{center}
    

{\huge FORDS}

\vspace{0.2in}

{\LARGE 68 Broad Street, New York}

\vspace{0.2in}

{\large FRANKLIN FORD, Director}

\end{center}


\vspace{.35in}

\noindent {\bfseries\Large Bank News Bulletin}

\vspace{.01in}
\hspace{0.25in}{\bfseries\Large No. 2}

\vspace{0.2in}


\newthought{The trouble at main} centers over the~flood of country checks, the
growing use of express~``orders''~ and~the effect on the money market of
currency shipments to ``move the crops,''~are incidental to the
further~development of the deposit system, and a still wider use of the
check as the preferred instrument in transferring bank credit.~

The deposits in the banks of the United States outran their note
circulation about 1855. Since~then~the check has been steadily coming
into general acceptance, along with a corresponding neglect of the bank
bill. Between 1855 and 1885, a period of thirty years, the check came to
be the customary and usual medium for paying balances between traders~in
a given~locality. During this period, when making remittances between
centers, and especially when arranging to pay debts at the metropolis,
the rule was to buy exchange on New York, or, in exceptional cases, on
some leading provincial center. In other words, the country trader down
to about 1885 bought his New York exchange from the local bank, and of
course paid for it the customary charges. It is said that country banks
in those times were frequently able to make ordinary expenses from the
sale of domestic exchange. ~

\newpage After 1885 a far-reaching change began to come in. Having grown
accustomed to the use of checks for meeting all local demands, the
single individual or trading firm took to sending them to remote towns.
The movement gradually extended itself, until it has come to pass that
everybody who has a bank account anywhere is insisting upon sending his
check everywhere. ~

It is nearly ten years since New York bankers woke up to find themselves
in daily receipt of a huge volume of country checks. The movement has
increased until now, when some twelve million dollars or more~of checks
drawn on local banks throughout the country are~daily deposited by the
merchants of New York in their respective banks.~The causes which have
led to this are still operating, and the receipt of merely personal
checks is certain to increase until an adequate remedy is provided.~

One effect has been to transfer the cost of domestic exchange, so far as
the habit indicated is followed, from the remitters (where it properly
belongs) to the banks and merchants at the main centers. When a country
bank receives for collection a check on itself, which has been sent to
New York or~some other center by one of its depositors, an~average
deduction of one-tenth of one per cent is made; in other words, the
local banks exact a charge for the payment of checks drawn on
themselves.~The banks at main centers~have sought to make the receivers
of~country checks meet the cost of collection, but,~in spite of~all
efforts, more~than three-fourths of the expense is now borne by the
banks.~~~

It is a shipment of~paper throughout. The country debtor remits his
personal check, and when it is returned for collection the bank on which
it is drawn sends in payment its draft on New York, Chicago,
or~other~center. The movement~as a whole suggests~a further development
in the use of the paper instrument for transferring bank credit, which,
as things stand, is only partially attained. ~

The personal check is sent to New York or Chicago, only to come~back
over the same path with a demand for payment in another paper
instrument, namely, a bank draft on New York or Chicago. This double
movement is required before actual payment is held to have been made.~

Against this the express companies have introduced a plan whereby
payment at remote points is~effected~in one movement, and a single paper
instrument, which the remitter is enabled to issue from his own office
when wanted, just as he does his personal check. In short, the express
people are certifying the check-books of traders through the sale of
their money orders in bulk, the users signing as agent.~

The American Express Company, in particular, is~now selling its orders
throughout its field for the use of commission merchants and others who
have frequent need of certified checks. The American Express people have
in this way~brought bank credit and that of the individual together in a
single check; the action amounts to systematizing the use of the
cashier's check. Each express office so acting is in effect a bank of
issue. Commission merchants in Detroit and other cities keep accounts at
the American Express bank because of the convenience it affords them
through being able to get from it certified checks in quantities, to be
issued at convenience~from~their own offices, just as personal checks or
bank bills are issued. The~understanding is that the orders are sold in
quantities to merchants at one-third off from regular rates.~

The~express companies must~be regarded as banks, so far as their
work has to do with the sale and movement of exchange. Moreover, they
are beginning to receive~deposits, being compelled to do this by the
sale of orders in bulk. They are an integral part of the banking
system.~

The general problem that is demanding solution looks to finding the most
convenient and inexpensive manner of shipping bank credit. The example
of the express companies in selling their orders in quantities, to~be
signed and issued at the pleasure of the buyers, suggests, at least, the
incoming of a universal paper certificate of value. The express people
solve the collection of personal checks by meeting the demand for a
universal check.~

The notion of collecting an express order has now quite left it; they
are paid by set-off through the clearing centers of the banking system.
About \$150,000 of express orders are now passed daily through the New
York Clearing House. The average size of these orders is not far from
ten dollars, so that 15,000 pieces of express paper are handled each day
at the Clearing House. The bank clearing houses of the country pay the
universal express draft, but the banks proper~are~without such draft of
their own. The banks pay the express companies for shipping currency,
but make no charge for handling their orders at the clearing centers.~

The local or personal check when received at main centers is sent to the
point of origin as quickly as possible in order to hasten collection,
but the very contrary is true of the express draft. Until a local check
is paid, the forwarding bank is not in possession of actual value on
account thereof. The force of the express order is not hurt or lessened
by the time element; each order is in effect so much New York exchange;
the longer it remains out, the greater the number of debts that will be
paid by it. ~

The twelve express companies of the country, inclusive of the Canadian
lines, act as a unity with reference to dealing in exchange through the
sale of express orders. Any one of them stands ready to pay the orders
of all the others at the Clearing Houses of the banking system.
Telegraphic transfers of credit are made by and between the twenty-five
thousand offices which the express system of the country maintains.~

Through the sale of orders in quantities, to be used in the manner
indicated, the express people meet the demands of merchants who have
need to use certified checks widely. By supplying a universal draft at
low~cost~they lessen the movement of personal checks.
The~``travellers~check'' has been introduced, and is sold to merchants
for the use of their salesmen when moving from~town to town~remote from
headquarters.~~

There is no record of any successful attempt at raising an express
order.~

Nothing has been done by the express companies to bring either the order
or the~traveller's~check to bear upon the problem of shipping bank
credit to ``move the crops,'' yet the check-books of the crop-movers are
as much in need of certification as those of the commission
merchants.~As indicated, the express order, the country check, and the
disturbances due to currency shipments, are all phases of
one~problem.~~~

The express companies have struck out an advance in the organization of
the Credit System. Their work abounds in lessons for the masters of
banking. It is worth noting that the American Express Company is
successfully marketing their~travellers\textbf{~}check for use in all
parts of the commercial world, foreshadowing in this way universal
recognition for the American Check. ~

The express order has a signal advantage over the post-office
order~owing to the fact that~the latter is only negotiable at the town
or post-office on which it is drawn. The express order is equally good
when presented at any center, and is everywhere regarded as New York
exchange.~

When the post-office order was introduced in 1864, aim was to keep the
system entirely distinct from the banking movement; post-masters
were~``prohibited from depositing money-order funds in any bank.'' So
late as 1890 Postmaster-General Wanamaker\footnote{{[}John
  Wanamaker~(1838--1922) is best known for being a pioneer of the
  American Department store and for the invention of the price tag.
  After opening a men's clothing store in 1861, Wanamaker went on
  converting an abandoned Pennsylvania Railroad depot into a store
  called Wanamaker's,~in 1874, inspired by central markets in Les
  Halles, Paris, and the Royal Exchange, in London.~He served as the
  United States Postmaster General from 1889 to
  1893.\href{applewebdata://D02306DF-3E46-4684-BD1A-1A323FFB2CB2\#_msocom_1}{{]}}~}
spoke of the order as~``a means of remitting small sums without
interfering with vested banking interests.'' Yet the post-office people
have been compelled during recent years to give it the form and
character of a banker's draft; and to-day the postal order would be
shorn of its vitality were it~not to have free movement through the bank
clearing houses. ~

The daily receipt of postal orders at New York is \$50,000. Of these,
one-half are paid by set-off through the Clearing House, the Chase
National Bank acting as clearing agent. Of the remainder, \$15,000 are
paid by check to a few large receivers of orders, so that only \$10,000
are paid by tellers at the post-office, and the latter item is
constantly decreasing. At all leading centers the post-office pays its
orders through the local clearing house, and the movement is extending;
just recently at Toledo the local post-office became a member of the
clearing house.~The New York banks handle postal orders as low as
twenty-five cents in value.~

These facts are cited to show the growing dependence of all sellers of
exchange on the clearing privilege. In 1897 the total sales of postal
orders were \$175,000,000, the average value for the year being \$6.93.
At last accounts the total sales of express orders and~travellers~checks
were something like one-half those of the post-office. ~

It appears that in the progress of the clearing principle the
post-office and express orders for the payment of ``money'' have
themselves come to function as money, or instruments for the direct
transfer of credit, being paid by set-off at the clearing centers; in
other words, the daily claims against the express companies at the New
York Clearing House are met by them with checks on their respective
banks. A profound change is indicated. Payment by set-off has now
progressed so far that it is to be seen~as the typical~action in
banking, giving to the bank its definition.~Banks~register and certify
credits, and conduct the payments of commerce through set-off~or
clearing. Any bank, therefore, is~itself a clearing~house, through which
debts are paid with credits. The~New York Clearing House acts as a main
center for numerous lesser banks or clearing centers, and, in fact, for
the whole country. The Cedar Street institution in New York is the
National Clearing House.~

With the deposits in the New York banks doubling every ten years, it is
not possible to hold longer to a narrow view of the banking function. At
one time it was thought to~be~``a liberal expansion'' for a bank ``to
loan two-and-a-half times the amount of its capital.'' Owing to the
rapid advances in registering credits, the shareholders of a bank are
coming to be seen simple as its first depositors. To-day the individual
depositor, in place of buying a bill from the bank, is everywhere
insisting upon making his personal check serve as a bill of exchange.
The action is already so far-reaching that the union of bank credit and
that of the individual in one check is presented as one solution of the
problem. Here is the American Express Company advertising that
its~travellers~check ``is practically a certified check, payable to
one's own order,'' and that the checks ``are virtually a universal
currency.'' ~

As bearing on the solution demanded, the work of the Georgia
banks in providing what they call a ``circular check'' is in point, and
will repay investigation.~~

The right solution of the country check problem cannot fail at the same
time to correct the disturbances in the money market resulting from
shipping currency to~``move the crops.'' While the depositors in country
banks are sending their personal checks to the main centers to the
extent of millions of dollars each day, the banks are compelled to ship
currency to the interior.~

Progress is constant toward increasing ease of payment, and the
appearance of the country check in great numbers at main centers bears
every indication that an advance~in~banking organization is not far
away. The effect of improved facilities of communication has outrun bank
regulation.~~

~A demand~has~arisen for special clearing centers, through which the
country checks can be paid. It is~based on the fact that~paper of this
order does not move at par. There is no demand for special clearing
machinery in the case of the post-office and express orders, and~for the
reason that~they both move at par through the clearing centers already
in existence. ~

The right solution of these important matters is bound up with the
question of bank revenues. As things stand, the banks are paid in the
shape of interest for making advances of credit through discounts or
loans, but not, save infrequently, for credit certifications. Owing to
the larger and larger registration of credit, the rate of interest is
everywhere declining, thus forcing the revenue problem to the fore. Now,
the post-office and the express companies are receiving pay for
CERTIFYING credit; they sell credit certificates, which go everywhere at
par. At one time, as already noted, the country banks derived
considerable revenue through the sale of exchange, or credit
certifications, but this has in the main been taken from them.~~

It remains to be asked whether the banks as a class are not now
compelled to get pay for their primary work of registering and
certifying credit. The general bookkeeping of commerce is in the hands
of the banks, but for this service they are not paid.~~

The personal check is the quickest means of transferring bank credit,
and is therefore the readiest money. The possession of a check-book is
quasi evidence that one has a bank account, but the bank's
certificate~has to~be added before the evidence is conclusive. When the
bank credit and that of the individual are united in one instrument, the
result is a valid order for goods---that is to say, it is money. The
personal check by itself simply is no more than a debt, subject to
collection; it does not carry the authority of the bank.~

From this it appears that the first need in dealing with the country
check problem is to distinguish the checks clearly as debts. Bank credit
in all its forms moves freely through the clearing centers, and is paid
by set-off when reaching the point of origin. Personal checks are taken
at par in the localities where drawn. The determining influence is the
time element.~

In England the country check is cleared through supplementary~machinery
at the London Clearing House, because all the English banks, barring a
few remote exceptions, are within one night's mail of London. Each
bank~has~a clearing agent at London, and the entire area is~cared for in
a single movement. It is a matter of twenty-four hours or less, instead
of one or two hours, as in the case of local checks.~

It has been proposed to divide the area over which the American banking
system extends into districts for the clearing of country checks; but
the difficulty here is that checks drawn in one district may be sent to
all other districts, and especially to the main centers. The difficulty
is extreme, owing to the variation in the time element.~

A radical solution is compelled by the increasing need of economy in
bank management. The initiative on one side has passed to the
individuals who are in possession of bank accounts and~check-books, and
who are insisting upon making transfers, however remote the point of
payment, by direct issues from their own offices.~

The remedy lies between a chain of district clearing houses for country
checks, where checks from main centers would be sent for collection, and
a refusal of all recognition at main centers for country checks, save so
far as they are authorized by the home bank, which means the
certification of the check-book, as the express companies are already
doing. In either case, the outlook is toward a more commanding unity, a
further reach in self-regulation, for the American banking system.~

To sum up: At one time remittances to remote points were of necessity
made in gold, as under early conditions nothing else would have had
recognition among strangers. Afterwards gold and silver coins came to be
deposited in bank, and certificates issued therefor were used in making
payments over wide areas.~At a later date~bank bills, which were issued
without direct reference to deposits of gold and silver, found shipment
here and there as a means of effecting commercial payments.
Subsequently, the bank check rose and grew into the customary medium of
payment between local centers. Until now, for the most part, the
personal check and the banker's draft have been used separately. But it
now appears, as already shown, that the express companies, both in
their~``money order'' and the ``travellers~check,'' are uniting bank
credit and that of the individual in a single instrument. Moreover, they
are doing this on so large a scale that it is of importance for the
masters of banking to take strict account of the movement.~

\vspace{0.2in}

\hfill{\Large FRANKLIN FORD}

\vspace{0.1in}

\hspace{0.25in}{\large February 8, 1899}

% THE MERCANTILE AGENCIES AND CREDIT REPORTING
\chapter[The Mercantile Agencies and Credit Reporting]{The Mercantile Agencies and Credit\\\noindent Reporting}
\label{ch:The Mercantile Agencies and Credit Reporting}
\chaptermark{THE MERCANTILE AGENCIES}

\vspace{.2in}

\begin{LARGE}

\smallcaps{Franklin Ford}\marginnote{Published in \emph{Textile America}, Vol. 3, No. 11, p. 5–7, June 3, 1889.}

\end{LARGE}

\vspace{0.4in}
\begin{center}
    
\begin{spacing}{1.25}


{\LARGE Announcement of a New Scheme---The Work of ``Textile~America''---The
National Association of Credit Men and~their Attempts to Bring the
Agencies to Book---Necessity of Publicity for the Reporting
Organ---Place of the Clearing-House Principle}

\end{spacing}

\vspace{0.25in}

{\LARGE \emph{FRANKLIN FORD ON THE SITUATION}}

\vspace{0.15in}

\enlargethispage{\baselineskip}

\end{center}

\newthought{The issuing of a} prospectus for a new scheme in credit reporting by
Erastus Wiman and associates\footnote{{[}Erastus Wiman (1834--1904) was
  the Director of the Canadian branch of R. G. Dun, which became Dun,
  Wiman \& Co. in 1861. Wiman later became the manager of the New York
  office of the Dun agency, a position from which he initiated several
  projects, including the quarterly publication of the firm's book of
  reference (which was previously published on an annual basis).{]}}
brings up again the growing necessity of a clear advance in this
all-important field. The plan, as outlined, aims at bringing together
some of the lesser reporting concerns, the combination to be known as
the Mutual Mercantile Agency. The president of the new concern is Mr.
Franklin Edson. Its capital stock is fixed at \$2,000,000, one-half of
which is preferred, and subscriptions are asked to its shares.~

In 1897 TEXTILE AMERICA opened up the whole problem of credit reporting
in a series of articles which gave the defects of the Mercantile
Agencies, and showed the way to the only possible reform. It was shown
that while enormous revenues have accrued to the Agencies, little or
nothing has been done to develop and further apply the principle of the
business. The prospectus referred to confirms the statement made by
TEXTILE AMERICA two years ago that one of the leading Agencies paid for
ten years or more a dividend of 80 per cent~to its shareholders, and
that the second well-known Agency, which is a partnership, has returned
a yearly profit of some \$500,000.~

\hypertarget{the-demand-for-an-open-door}{%
\subsection{\texorpdfstring{\emph{THE DEMAND FOR AN OPEN
DOOR}}{THE DEMAND FOR AN OPEN DOOR}}\label{the-demand-for-an-open-door}}

Attention was drawn to the Mercantile Agencies in 1896 by the National
Association of Credit Men, and a demand made for better reporting.
Cour-\\\noindent age was required for this, as the business had gone on for years
without investigation. Seemingly protected behind the great wealth which
the business had yielded, the agencies were proof against criticism
unless it were to become general and systematic. In demanding a new
departure, the credit men undertook a great service to American
commerce. At the outset they were hindered by the lack of freedom in
speaking of the Agencies, intrenched as they were behind big money
reserves and tall office buildings, but the spell was broken by the
outgivings of TEXTILE AMERICA in 1897, and thereafter the most retiring
credit man was free to speak his mind. By reason of the searching
publicity so gained, the way was opened to compelling from the Agencies
the amplest statements of their own affairs, their methods of doing
business, and the extent to which revenues have~been~wrongfully diverted
into private channels. When a given thing or institution has once been
submitted to discussion, it can never be recalled, and, in consequence,
progress has been rapid toward the same freedom of inquiry for all
credit-reporting organs as that to which the post-office has long been
subjected. The goal will be reached when nothing whatever is concealed
from the credit men as to agency revenues and expenditures.~

\hypertarget{attitude-of-credit-men}{%
\subsection{\texorpdfstring{\emph{ATTITUDE OF CREDIT
MEN}}{ATTITUDE OF CREDIT MEN}}\label{attitude-of-credit-men}}

The credit men have not rested in their attempt to bring the Agencies to
book, but the results have been far from satisfactory. The sorry outcome
is summed up in resolutions which are to be submitted to the association
at the June meeting in Buffalo by the Committee on Mercantile Agency
Service. It is set~out therein that the Agencies have declined to
respond in detail to suggestions or to comply with requests, and ``have
indicated a spirit of independency and self-satisfaction scarcely
befitting institutions of public service.'' As seen by the committee,
``there is no branch of business more important than the Mercantile
Agencies, and no work which demands such accuracy and completeness or
which is susceptible of such advancement.'' And, finally, the belief is
asserted that if

\begin{quote}
inattention as to reasonable requests is continued, the time will then
be ripe for the development of a new Agency, the policy of which shall
be liberality in the payment of qualified reporters and other sources of
information; accuracy in reports, and a constant endeavor to comply with
the suggestions of practical credit men as to what is important and
desirable in Agency reports and service.
\end{quote}

So far, then, nothing has been done beyond fencing with the credit men
in their righteous attempts to get at the truth as to Agency methods and
to compel reforms. The explanation is simple, for just so soon as the
right of inquiry on the part of the National Association of Credit Men
is admitted, no limit can be set to it; the Mercantile Agency, as a
credit-reporting organ, becomes a public institution, to be seen and
dealt with as we now see and deal with the Post-office or the Bank
Clearing House. The credit men have need to recognize this, as the end
in view cannot be {[}original text could not be retrieved{]}. The
credit-reporting organizations themselves be subjected to the freest
inquiry, and when this is done the business will be at once a vast
public good and a private benefit.~

\hypertarget{the-diversion-of-agency-revenues}{%
\subsection{\texorpdfstring{\emph{THE DIVERSION OF AGENCY
REVENUES}}{THE DIVERSION OF AGENCY REVENUES}}\label{the-diversion-of-agency-revenues}}

The fact is that the Mercantile Agencies, as now conducted, are not
modern, in the sense that the Post-office and Bank Clearing House are
modern. The revenues are so far diverted into private hands that
progress and development is not possible; the money is not kept in the
business. The point is made clear when it is recalled how, at times,
when an old house fails, the remark is heard that it's no wonder, for
they ``robbed the business.'' It is only the extraordinary demand for
credit news which permits such diversions of revenue along with the
continuance of a business. The thrifty Mercantile Agencies are
paralleled on the other side of the news field by such a paper as the
\emph{New York Herald}, whose enormous revenues are diverted in like
manner in place of being kept in the business and expended for inquiry
in the public interest. The revenues of the post-office are now kept in
the business, yet at one time it was held to be the king's ``private
enterprise.'' The growing need of integrity in the transactions of
commerce is prompting the discovery that the business of
credit-reporting can no longer be held as a ``private'' affair in the
sole interest of this or that group of individuals. There is ample
evidence of this in the clearer views of the National Association of
Credit Men. In fact, to be practical from this on all the~centres~of the
News System must be seen as public functions.~

\hypertarget{one-open-organ-sufficient}{%
\subsection{\texorpdfstring{\emph{ONE OPEN ORGAN
SUFFICIENT}}{ONE OPEN ORGAN SUFFICIENT}}\label{one-open-organ-sufficient}}

Among numerous credit-reporting organs a struggle has begun for the
survival of the fittest, with the outside waking up to the fact that
only one is wanted. The surviving concern, whether it be one of the old
Agencies or a new project yet to come in, must freely open itself to
publicity. Anything less than this will not go. The credit men have made
good progress in subjecting the agencies to publicity, but the end
cannot be reached without coming full circle. There must obtain the same
freedom of criticism concerning the credit-reporting~centre~as now
exists in relation to the rates of postage and the money paid for
carrying the mails, because the nature of the business is a public
transaction, the same as that of the post-office. President Clark, of
the Bradstreet Company, has referred to the agency as ``a business of
the merchants, by the merchants, for the merchants.'' Taken as a matter
of fact, this means that the information possessed by all merchants
concerning applicants for advances of credit is reported through the
various~centres~and distributed by these~centres~where needed. The
machinery for this collection and~distribution belongs as much to the
credit departments of the country as to the collecting~centres, whether
manned by one of the old Agencies or any other set of men. Neither the
proposed Mutual Mercantile Agency nor other concern can make great
headway without recognizing this.~

\hypertarget{nature-of-credit-reporting}{%
\subsection{\texorpdfstring{\emph{NATURE OF CREDIT
REPORTING}}{NATURE OF CREDIT REPORTING}}\label{nature-of-credit-reporting}}

The first credit report was a question asked by one merchant of another
as to how Smith, of Utica, was paying. Afterwards a dozen traders in the
dry goods district of New York City employed a man to gather together
the information possessed by all of the twelve and to distribute the
combined resultant to each. Following upon this, as post-office
facilities grew, the Mercantile Agency rose and the collection of
reports was extended over the country. Numerous reporters of moderate
capacity were employed at the chief centers. Organization has been
everywhere extended, but in point of efficient character it is only
nominal. The truth is that the old Mercantile Agencies as they stand
serve as a bar to progress and reform in credit reporting for, under the
present post-office facilities, it has become possible for all merchants
to clear their credit information anonymously through properly organized
centers without the intervention of reporters in the Agency sense. In
short, the universal adoption of the clearing-house principle has become
possible, and it is key to all advances in credit reporting.~

\hypertarget{the-clearing-house-principle}{%
\subsection{\texorpdfstring{\emph{THE CLEARING-HOUSE
PRINCIPLE}}{THE CLEARING-HOUSE PRINCIPLE}}\label{the-clearing-house-principle}}

During the last fifteen years or thereabouts the work of determining
credits in wholesale and jobbing houses has gradually come to be
recognized as a division of labor calling for special talent and
distinctive methods of organization, to such an extent indeed that all
houses of any magnitude now maintain what is known as the Credit
Department. In former times the head of the office force usually carried
the responsibility as something incidental to his regular duties,
important or difficult cases being referred to the financial partner or
manager. But with credit departments everywhere coming into place, the
need has arisen for clearing the information which they gather. There
would be no difficulty in effecting this if either of the~well
known~Agencies were able to get rid of old methods and adopt modern
ways.~

Unfortunately for the business of credit reporting, the notion arose
that it is the function of the Agencies to determine credits. Scarcely
anything could be more untrue, as credits are determined by the merchant
in the very act of making sales, and to aid him in this he is on the
lookout for facts from all directions. The wide acceptance of the
Agencies as makers of credits, through absurd ``ratings'' and the like,
went to nurse their credit, in the banker's sense, at the expense of
their credibility. In the struggle to compile a ``rating'' book the
materials gathered by the Agencies were compressed to the point of
distortion, while the supplementary reports have come to be held by
merchants as scarcely more than the merest trade gossip. It has thus
turned out that so far as relates to the Agencies the true principle of
credit reporting passed into neglect, while the need of its further
development was constantly increasing. In place of gathering and
distributing the actual experiences of merchants with applicants for
credit, a sort of ``literary'' procedure intervened. All manner of
people were employed as Agency reporters to pick up gossip about the
trading concerns which were under inquiry. In the vain attempt to
generalize the results and cover deficiencies the only possible resort
was to ambiguity. The merchants themselves must be the reporters through
their own credit departments.~

\hypertarget{need-for-a-single-organization}{%
\subsection{\texorpdfstring{\emph{NEED FOR A SINGLE
ORGANIZATION}}{NEED FOR A SINGLE ORGANIZATION}}\label{need-for-a-single-organization}}

Prompted by the unexampled growth in communication, the trading
interests of the country are now seeking a common center, a single
organ, through which all facts, regarded in the light of their bearing
on credits, may have universal distribution. The rise of the National
Association of Credit Men attests this. The present needs of commerce
with respect to reporting credit news is as much beyond the present work
of the Mercantile Agencies as the telephone is beyond the locomotive as
means of transmitting~intelligence. To the everyday observation it is
supposed that the Agencies are doing the work required of them; that the
goods they offer are the best obtainable. But to the inquirer who can
get below the surface of things it is well known that credit reporting
is only in the infancy of its development. The pioneers among Agency men
did a great work in extending the movement everywhere, but at the last,
in place of building up the direct exchange of experiences on the part
of merchants parallel with the growth of communication, they have
erected an unwieldy machine which, aided by an artificial prestige,
stands as a positive hindrance to the better credit reporting made
possible by new conditions.~

\hypertarget{an-organizer-needed}{%
\subsection{\texorpdfstring{\emph{AN ORGANIZER
NEEDED}}{AN ORGANIZER NEEDED}}\label{an-organizer-needed}}

The situation has been waiting on the man or men able to perceive, and
act upon, the fact that the avenues of communication---the quickness and
certainty of movement---are so far open as to do away with the Agency
reporter, as such, and instead to permit direct reporting by merchants
through common centers, which means, as already indicated, the adoption
of the clearing-house principle. The need of the banker, which prompted
the absolute interchange of facts at the Bank Clearing House concerning
each particular check, is identical with the necessity of the merchant
who must now erect a clearing house for exchanging the immediate facts
concerning mercantile credits.~

The old Agencies have reference to the clearing-house principle when
they write and talk of ``trade reports'' which ``give the experiences of
jobbers and manufacturers with their customers,'' while Mr. Wiman and
his friends look toward ``a co-operative instrumentality for the
interchange of information obtainable in the trade, after the manner of
a great credit clearing-house.'' The need is to perceive that true
credit reporting cannot exist at all save through ``a co-operative
interchange of information;'' it is the nature of the business.~

One difficulty here lies in the verbal distinction between
``co-operation'' and ``commerce.'' It has to be learned that the more
co-operative a given branch of trade becomes the more commercial it is.
Some years~ago~an enthusiastic woman raised in New York a big sum of
money and started a so-called co-operative dry goods store. It failed
because it was not so co-operative as Macy's.~

\enlargethispage{\baselineskip}

\hypertarget{credit-reporting-as-a-phase-of-social-registration}{%
\subsection{\texorpdfstring{\emph{CREDIT REPORTING AS A PHASE OF
SOCIAL
REGISTRATION}~}{CREDIT REPORTING AS A PHASE OF SOCIAL REGISTRATION~}}\label{credit-reporting-as-a-phase-of-social-registration}}

Credit reporting when rightly understood is seen as a phase of social
registration. The facts as to land ownership are registered with the
Title Guarantee and Trust Company; births and deaths are registered at
the Health Office, marriages at another center, while the bank
transactions at ninety-five clearing houses throughout the country are
registered each week at New York as main center in the banking system
and are thence distributed to all sub-centers. The credit men, in turn,
are wanting to register their facts at the trading centers to be there
summed up and distributed for the common guidance, but the Mercantile
Agency people continue blind to the real situation and refuse to treat
on a basis of equality.~

Two years ago, when inquiring into credit reporting methods, TEXTILE
AMERICA dwelt upon the work of the Credit Clearing House as an attempt
in the right direction, but the one rule has now to be applied to it and
to all. The clearing-house principle has the future, but it cannot
obtain through any institution which is held, on the side of property
rights, as a narrow interest. The selfish interest must be there,~the
business must be ``run to make money,'' but the merchants and credit men
cannot concentrate upon any one organ without the amplest safeguard
through full publicity. The concern which will meet the demands of the
National Association of Credit Men by accomplishing the desired reform
will earn a generous money reward, but the advance cannot be gained
without first foregoing the thought of eighty per cent dividends and
yearly takings of half a million dollars. When understood and grasped as
a legitimate business, such profits are seen to be more than the traffic
will bear.~

Given a single organization, with full integrity at its centers, and
free movement, all credit news will~be registered and distributed. In
the absence of a unified system and freed conditions, no expenditure of
money for ``reports'' can~effect~the desired reform.~

\hypertarget{five-aspects-of-registration}{%
\subsection{\texorpdfstring{\emph{FIVE ASPECTS OF
REGISTRATION}}{FIVE ASPECTS OF REGISTRATION}}\label{five-aspects-of-registration}}

The registration of credit news presents five aspects, namely: (1) what
the banks say of a dealer concerning his transactions with them; (2)
what the dealer's merchandise creditors say of him; (3) what the dealer
says of himself, which is called his~``statement;'' (4) his pedigree or
trading history, and (5) the state of a dealer's particular branch of
trade. The last named cannot but enter into the problem of a given
applicant's claims at the credit department. All five, in one way or
another, come under the clearing house principle.~

The banks of Louisville and Baltimore have made progress in clearing the
facts as to commercial paper. In Louisville an applicant for discount at
any one of a dozen banks has to face the fact that his record at all of
the banks is at once accessible. There is everywhere at this time a
growing belief among bankers that an interchange of facts as to
transactions in commercial paper at bank counters must be brought about.
There is opposition of course, but so there was when the clearing house
for setting off checks against each other was first proposed.~

\hypertarget{the-clearing-of-merchandise-credit-news}{%
\subsection{\texorpdfstring{\emph{THE CLEARING OF MERCHANDISE
CREDIT~NEWS}}{THE CLEARING OF MERCHANDISE CREDIT~NEWS}}\label{the-clearing-of-merchandise-credit-news}}

The progress already gained in clearing the facts as to merchandise
credits gives promise that the principle will obtain on all sides at a
nearly day. The tendency of the old Agencies to offer facilities for
``trade reports'' and the ``interchange of information'' is full of
meaning. Just as soon as the clearing movement obtains generally, the
credit-reporting business can no longer be held under a narrow
proprietary. One might as well talk of turning the Bank Clearing House
at New York into a ``private'' corporation. The clearing centers of the
credit news system parallel the centers of the banking system. They both
belong to the Credit System, which is the monetary or accounting
division of commerce, embracing in its circle of action the remotest
bookkeeper.~

The statements of applicants for advances of credit will be registered
or cleared through the centers of the one system. Mr. F. R. Boocock,
secretary of the National Association of Credit Men, reports good
progress in getting to a uniform property statement blank. He states
that two years ago there were scarcely any two firms using the same
form, while to-day about five hundred concerns are using the blanks
prepared by the Association. The president of the Association, Mr. James
G. Cannon, has appealed to various State banking organizations
concerning the use of the blanks, and the response has been favorable.~

\hypertarget{the-newspaper-as-a-factor}{%
\subsection{\texorpdfstring{\emph{THE NEWSPAPER AS A
FACTOR}}{THE NEWSPAPER AS A FACTOR}}\label{the-newspaper-as-a-factor}}

The pedigree or trading history of a dealer has to be got from his
neighbors and from those who have done business with him. No
thorough-going reform can be brought about on this side of credit news
without bringing the daily newspaper of a given region into the
movement. The general newspaper is main~centre, and as the organization
proceeds will be brought to register and supply, in great part, the
credit news of its region.~

\enlargethispage{\baselineskip}

An application for credit may at times turn upon the state of trade. It
may be that a silk manufacturer, whose affairs are otherwise in good
shape, is bent upon a false move as measured by a pending change in
fashions. The class or trade paper must come to be a more important aid
in supplying this need, so that the banker or merchandise credit man may
be in constant touch with the outlook in all divisions of commerce.~

The necessity of a far-reaching change in the business of credit
reporting is confirmed on the side of the bank, which is coming to be
more and more the center of action for the Credit System. Bank
deposits~are rising on all sides owing largely to the immense Trust
movement and the huge volume of industrial shares. The integrity of bank
transactions is dependent on a clear advance in the business of credit
reporting.

\vspace{0.2in}

\hfill{\Large FRANKLIN FORD}

% CO-OPERATIVE CREDIT REPORTING
\chapter[Co-operative Credit Reporting]{Co-operative Credit Reporting}
\label{ch:Co-operative Credit Reporting}
\chaptermark{CO-OPERATIVE CREDIT REPORTING}

\vspace{.2in}

\begin{LARGE}

\smallcaps{Franklin Ford}\marginnote{Letter to the Editor of \emph{The New York Times}, Sep. 17, 1902.}

\end{LARGE}

\vspace{0.4in}


\noindent\emph{To the Editor of The New York Times}:

\vspace{0.1in}

Your issue of Aug. 31 gave a striking account of the work of the
National Clothiers' Association in co-operative credit reporting under
the direction of its President, Mr. Marcus M. Marks.\footnote{{[}Marcus
  M. Marks (1858--1934) was an American businessman who was involved in
  several business associations, including the National Clothing
  Association and the Merchants' Association of New York. He also served
  as President of the Borough of Manhattan from 1914 to 1917.{]}} The
success gained has the greatest possible meaning in relation to the
present commercial development, so much so that the important thing is
to bring out its general bearing and the further action to which it
points. The immediate facts are that the leading~centres~in the clothing
industry are now clearing their information concerning debtors through
the credit offices of~clothiers' organization. But one city, Cleveland,
remains outside of the movement, and when it is brought in along with
certain minor additions, the whole industry will have attained to a
self-reporting system whereby the credit facts in the ledgers of each
concern will be at the command of all, and on a basis of expenditure not
exceeding the cost of the operation. In this case the best is the
cheapest.~

By this method credit reporting is brought up to the level of the
present means of communication, which permits the direct exchange and
co-ordi-\\\noindent nation of the experience of merchants as recorded in
their ledgers. Prompt-\\\noindent ed by the talking wire, the plan looks to the
universal extension of the clearing house principle. It marks a clear
advance, amounting to a revolution, in the handling of merchandise
credits. The essential facts are registered immediately and are
distributed with equal promptness to all concerned. It is through such
means that the honest trader finds recognition and protection. The
mercantile agency system, which came into existence over half a century
ago, answered to the imperfect communication of that time.~

As President Marks of the Clothiers' Association insists, it is only by
this direct method that overtrading can be guarded against and failures
prevented. It may fairly be questioned whether the fifty-year-old credit
reporting ever resulted in the prevention of failures. Much, indeed, can
be said for the view that through~a false reliance failures~were
actually encouraged. Under old-time conditions credit information was
drawn chiefly from the home of the debtor, but through the methods
introduced by Mr. Marks and his associates the ledgers of creditors
constitute the main source. The reporting is from first hand, and facts
are substituted for gossip and rumor.~

The further point is to be noted that the movement is of necessity
co-operative, which means that the~centres~for the interchange of credit
news cannot be brought under the notion of private ownership. It is too
big for that. Besides, the merchants are in possession of the record,
and they are discovering that the only need is for~centres~through which
the facts can be exchanged, and interpreted at the hands of experts.~

Merchants generally should be brought to know the dignity and
far-reaching usefulness of the action in hand. The example of the
National Clothiers' Association is prompting the extension of
co-operative reporting to all branches of mercantile~credits.~

The action is one with the natural~development~of the commercial
organism, which, owing to the instantaneous communication of the time,
is at least to be~seen as a whole. The movement has back of it,
therefore, the growing conviction that the welfare of each organ or
division of commerce is dependent on the prosperity of all other
divisions. It is apparent that President Marks is animated by this
vision of the general welfare, perceiving, as he must, that the full
execution of the principle of interchange in the field of credit news
will~effect~a great advance in the scientific or responsible
organization of commerce. ~

As bearing on the much-discussed question of the National regulation of
commerce,~the work of the National Clothiers' Association is exceedingly
suggestive. While the public gaze continues to rest on Washington, as
the supposed single~centre~of regulation and control, new~centres~of
National supervision are appearing, the clothiers' organization being
but one among numerous others of like character and meaning. They have
their organs or Congressional Records in their respective class
journals, and altogether the development is so impressive that the
specialists in commercial regulation would do well to take the facts
into account. It is plain enough that commerce is evolving its further
control through these new~centres. Men have yet to learn that the moving
principle of commerce is both constitutive and regulative.~~

\vspace{0.15in}

\hfill{\Large FRANKLIN FORD}

\vspace{0.1in}

{\large New York, Sept. 17, 1902}

\newgeometry{left=1.05in}


\part{News is Government}

\restoregeometry


% CITY NEWS OFFICE NEEDED
\chapter[City News Office Needed]{City News Office Needed}
\label{ch:City News Office Needed!}
\chaptermark{CITY NEWS OFFICE NEEDED}

\vspace{.2in}

\begin{LARGE}

\smallcaps{Franklin Ford}\marginnote{Published in the \emph{Brooklyn Daily Eagle}, June 8, 1901.}

\end{LARGE}

\vspace{0.5in}

\begin{center}

\begin{spacing}{1.25}


    
{\LARGE\bfseries Franklin~Ford Tells Why a Clearing House\\of Facts Would Be
of Inestimable
Value}

\end{spacing}

\end{center}

\vspace{0.15in}

\noindent To the Editor of the \emph{Brooklyn Eagle}:~~

\vspace{0.05in}

The letter of Mr. Samuel~H. Bishop\footnote{{[}The letter by Samuel~H.
  Bishop published in the June 4, 1901 issue of the \emph{Brooklyn Daily
  Eagle} is itself a reply to a previous letter by Ford published in the
  same newspaper. Bishop's letter supports Ford's project for an
  ``organization of intelligence.''{]}} in your issue of June 4 supports
my contention that the reform in city government of which we hear~so
much necessarily involves the definite and systematic organization of
municipal~information or news. Mr. Bishop sees no danger "in
concentering power at the City Hall" if means are provided whereby the
people of every section may "know what they want and know how to convey
what they want to the City Hall."~

The end marked out will be reached by giving to the existing news system
a responsible center through which the facts can be~co-ordinated~and so
transformed into a governing force. Municipal betterment, therefore, is
a question of commercial organization; it is as far removed as possible
from the plaints of reformers who have no remedy to propose beyond the
defeat of one party and the substitution of another.~

The new center would report city business as a whole, and~in all its
parts:~it would become the clearing house for all municipal facts.~~

The conduct of city affairs must be confused and wasteful in the absence
of a comprehensive system of reporting and of central registration.
Until all the facts are accessible at some one point, the various
divisions of the city government can have no clear understanding of each
other and no adequate working relation.~

\newpage The true re-formation of city affairs must be based on a scientific and
full organization of municipal news, in contrast with this or that
propaganda of opinion.~

In the telephone, or~instantaneous communication, the modern press, and
the fast mail train, the newspaper has come to possess a perfect working
machine. Prior to these conditions news organization could not on the
whole be more than nominal, but it can now be made actual and
responsible on all sides. The method of journalism is to become
scientific in all particulars. The news traffic involves in its sweep
the entire organization of intelligence from the university center to
the remotest social happening.~

The news system consists of the daily paper, the trade or class journal,
and the bureau of information. In its outworking the reporting system
will connect with all sources of expert knowledge, and with
every~individual in the community as any one may at times possess a fact
of value to his neighbor,~to his class or to the people as a whole. But
the daily newspaper is center of action.~

The proposed organization of municipal news~involves the closest
co-operation between the various organs of the system. As now, the
different parts of the news system do not buy and sell of each other
save very incidentally. It is a singular thing that when a new fact is
disclosed in the news movement it carries three possible sales or
profits:~its general meaning to be distributed through the daily paper,
its class bearing to be cold {[}\emph{sic}{]} through the trade paper,
and its value to particular~individuals which they can obtain at the
bureau of~information. The desired co-operation will be reached through
division of labor and the freedom of action which the telephone
permits.~

The new center will be called the City News Office. It will act as a
common medium for the exchange of all information having a direct
bearing upon the regulation of city affairs, doing in this respect what
the Ship News Office does in its field and with equal precision. The
City News Office will bring the municipal facts of Chicago,~London and
Berlin side by side with those of New York. At the same time the
facts~of New York's government will become accessible to the world, and
the natural demand for authentic news~in this field will be met.~

Through gaining a common center all the parts of the
city\textquotesingle s news system will be~at the service of~every
citizen; without this it~is much in the condition that the telephone
system would be were there no main center connecting all together.~The
town is replete with useful bureaus of~information which will become
available through the City News Office.~

As Mr. Bishop writes from the Brooklyn Bureau of Charities, of which
he~is assistant secretary, I am led to point out the~inevitable effect
of a scientific news~center on the important matter of charity
regulation. The Charity Organization Society of Manhattan was~instituted
eighteen years ago. It was intended "to be a center of
intercommunication between the various churches and charitable agencies
in the city. To foster harmonious co-operation between them,''~and
to~``promote the general welfare of the poor by social and sanitary
reforms.''~It was to be a clearing house for the city's charity system.~

Great good has been accomplished~by~the Charity Organization Society and
its Brooklyn ally, the Bureau of Charities, but the results are small in
comparison with the benefits that would follow the~incoming of a
responsible general center for the city's news system. Such a center
would~invite and in fact compel from the charity news center so great
efficiency in clearing that the utmost of~wise government would be
assured.~

This lesson from the field of charity regulation goes to make plain that
the one road to city government reform is through the~systematic
handling of municipal science~or news. ~

\vspace{0.15in}

\hfill{\Large FRANKLIN FORD}

\vspace{0.1in}

{\large City Club, Manhattan, June 8, 1901}


% MUNICIPAL REFORM: A SCIENTIFIC QUESTION
\chapter[Municipal Reform: A Scientific Question]{Municipal Reform: A Scientific\\\noindent Question}
\label{ch:Municipal Reform: A Scientific Question}
\chaptermark{MUNICIPAL REFORM}

\vspace{.2in}

\begin{LARGE}

\smallcaps{Franklin Ford}\marginnote{Self-published book, 1903.}

\end{LARGE}

\vspace{0.5in}

\begin{center}

{\LARGE A GENERAL OUTLINE}

\vspace{0.15in}

{\Large OF}

\vspace{0.15in} {\LARGE REPORTS ON THE CITY'S DEPARTMENT OF
FINANCE IN ITS WORKING RELATION WITH THE MUNICIPAL
SYSTEM}

\vspace{0.25in}

{\Large FRANKLIN FORD}

\vspace{0.25in}

{\LARGE Publications of the City News Office}

\end{center}

\vspace{0.25in}

\noindent 1. \emph{History of the Department of Finance}---Its development during
the four years under Comptroller~Coler---Progress to an all-round legal
position as center of audit and control---The growth of a century~~

New York city's annual budget is \$100,000,000, in round numbers, and
its extraordinary expenditures forty~millions~more. The city government
centers round the Finance Department. Charged by statute with the
"control of the fiscal concerns of the corporation,'' its function is to
supervise the city accounting, and in this way regulate the conduct of
every official. The powers of the Comptroller as chief of the department
are so sweeping that his action touches every phase of the city's
business.~Facts~rule, and the Finance Department is fountain-head for
the facts that shape on every side the city's administrative policy.~

Ancient precedents define the comptroller as ``an officer who has the
inspection, examination, and controlling of the accounts of other
officers.'' He is the ``keeper of the counter-roll,~i.e., a roll
intended as a check upon the rolls or accounts of other
officers.''\footnote{{[}Alexander M. Burrill, \emph{A Law Dictionary and
  Glossary} (New York: John S. Voorhies, 1850), 279--80.{]}}~

New York's Finance Department is a law story from the start. It is a
progress toward full legal authority for the "inspection and revision''
of all city accounting by the Comptroller, and, in step with an
increasing need owing to the greater complexity, the conditions have at
last appeared for realizing the idea in practice.~

The tendency has been constant to bring the city government to unity of
organization through the Finance Department.~

\vspace{.15in}

\enlargethispage{\baselineskip}

\noindent2.\emph{ Function and organization of the Department of Finance}---It is
the department of departments---It is now the clearing-house or
governing center for all departments of the city government---It has the
power to organize the facts---Its several divisions and the duties of
each---Working relation with the other departments ---Chart of the
municipal system with the Comptroller\textquotesingle s office as main
center---Annual cost of the Finance Department~

\vspace{.15in}

\noindent 3. \emph{Bookkeeping Division}---It is the record office of the Finance
Department and the real Bureau of Municipal Statistics---Development of
statistical accounting by the bookkeeping division, in~distinction from
the financial accounts---The end in statistical accounting is to make
available at all times everything in the Comptroller's record, which
includes the records of all other departments---The Bureau of Statistics
cannot organize outside of this record---The full organization of
statistical accounting carries with it exact supervision of all
department accounts.~

Beginning in 1901 under Comptroller~Coler, and especially since January
1, 1902, under Comptroller Grout, the bookkeeping division of the
Finance Department has been moving forward on the lines of statistical
accounting. This form of accounting has always had a place in the
Finance Department, inasmuch as the issuing of annual reports or other
public documents by the Comptroller is an effort in that direction. It
is true that under old conditions such reports were hardly more than
transcripts of the financial accounts and, therefore, had little or no
meaning for any one beyond the bookkeepers themselves; but the fact
remains that all reports of this order have been attempts at statistical
accounting.~~

A difference in the cost of two bridges may be used to illustrate the
place of statistical accounting. One may cost two million dollars more
than the other. The financial accounts reveal this, but to know the
cause of the greater cost of the one a development in statistical
accounting is necessary. The required facts are contained in the
vouchers and the need is to render them available. The statistical is
the primary accounting.~~

Until now each city official has been compelled to hunt his own
information. No effective administrative machine of like order and
magnitude in these days is without a system of statistical accounting.
Division of labor at this point is everywhere prevailing and the
municipality must fall into line.~

It is important to learn just how far the principle of uniformity can be
followed in the city's~bookkeeping; or, what is the same thing, how far
disparity in bookkeeping is inherent in city administration. The
business of street cleaning differs greatly from that of the water
department.~

\vspace{.15in}

\noident 4.\emph{ Division of Awards}---The accounting office for all~claims
against the city due to street openings, the taking of land for parks,
for school houses, police stations, armories, fire houses, baths, or
other public buildings, for bridge approaches, etc.---Awards are
received from the board of assessors, from the courts, and at times from
change-of-grade damage commission-\\\noindent ers---On their receipt all
verifications are made, including searches to see that no liens of any
sort are standing against the property, after~which the vouchers are
made out and turned over to the auditing bureau for certification and
return to the awards room for final payment---The division of awards
acts as its own disbursing officer---It is in direct contact with the
owners of the properties involved---The total money outgo under the head
of awards for 1900 and 1901 was as follows:~

\begin{figure}
    \centering
    \includegraphics{graphics/image-fourteen-b.jpg}
    \label{fig:fig14a}
\end{figure}

5.\emph{ Auditing of the City\textquotesingle s Revenues}---The
Comptroller\textquotesingle s annual report for 1901 presents the
results of a careful examination into all sources of city revenue---The
work was done by the bookkeeping division---It marks a clear advance in
statistical accounting---It provides a comprehensive basis for auditing
the city's revenues, which was long a neglected feature of the Finance
Department---Nearly all departments, to a greater or less extent, are
receivers of city revenue.~

The present city charter directs that the auditing bureau of the Finance
Department~``shall keep an account of each claim'\,' both ``for and
against the corporation,'' thus pointing to a scrutiny of both revenue
and expenditure. From the beginning the auditing bureau has investigated
and certified all accounts against the city, but a like vigilance did
not obtain with reference to the city\textquotesingle s revenues.
Successive New York charters contained directions as to an audit of
revenue; the charter of 1870 expressly provided for an ``auditor of
revenue,'' and for a time an officer of the Finance Department bore that
title, but the early conditions were not such as to compel execution of
the idea--- the need was not sufficiently urgent. A radical change has
now come in, and the introduction of an absolute audit of revenue is
seen to be a necessity. The work of disclosing all sources of revenue
through the bookkeeping division was the first step. In the last
twenty-five years the revenues of the old city of New York from
miscellaneous sources, outside of taxes and water rates, have doubled,
while further increases of revenue have resulted from the consolidation
of 1898, as each of the boroughs had its specific sources of revenue.~~

\vspace{.15in}

\noindent 6. \emph{Management of the City Debt}---The forms and methods of the old
city of New York in relation to debt handling were extended to the
enlarged city without change, save the great increase in volume and
detail---Bonds were issued by New York city in 1900 to the amount of
\$85,000,000, and comprising \$38,000,000 of long time bonds,
\$46,000,000 of revenue bonds, and \$1,000,000 of assessment
bonds---Revenue bonds in anticipation of taxes are issued as money is
needed from soon after January 1 to near the first of October---The
lowest rate of interest paid by the city of New York was two and
one-half per cent in the early 90's---During 1900 the Sinking Fund
Commission purchased \$17,500,000 of the new bond issues---It is
customary with the commissioners to confine their investments to new
issues of New York city bonds, as new bonds of the city are constantly
offering---Extent to which debt was taken over from the outlying
boroughs---The methods of the Finance Department in managing the city
debt turn, of~course, on the policy of the city in the use of its
credit.~~

\enlargethispage{\baselineskip}

The use of revenue bonds is a striking feature of the city's financial
system. Under the charter the Comptroller is authorized to borrow ``from
time to time, on the credit of the corporation, in anticipation of its
revenues, and not to exceed the amount of such revenues, such sums as
may be necessary to meet expenditures under the appropriations for each
current year.'' The method is of long standing as a custom of the old
city of New York. It has had gradual extension in the form of "special
revenue bonds''~to provide emergency funds and the like, until now the
charter enumerates nine purposes for which such special revenue bonds
may be issued. It appears that under the law the city's first resort for
funds is the loan market, in advance of taxation.~

\vspace{.15in}

\noindent 7.\emph{ Care of the City\textquotesingle s Vouchers}---One great part
of a city's history is contained in its accumulation of treasury
vouchers---The vouchers of the old city of New York have been preserved
from the beginning and all are at ready command for reference---The
oldest date of the New York vouchers is 1679---City of Brooklyn vouchers
do not exist for dates prior to 1855, the previous accumulation having
been destroyed---The voucher rooms of the Comptroller's office contain
the vouchers for the current year and for the two previous
years---Manhattan vouchers as far back as 1885 are kept in the basement
of the Stewart building, while the old-time vouchers are filed on the
fourth floor of No. 296 Broadway.~~

\vspace{.15in}

\noindent 8.\emph{ Development of the Auditing Bureau in the Department of
Finance}---Its classification and the duties of each
division---Importance of an absolute audit as means to compelling right
conduct in city affairs---When the central accounting is given the last
exactness, the discipline reaches every department---The city is
affected as a whole---Present efficiency of the auditing bureau---Time
required for passing vouchers through the bureau as compared with former
years---Cost of the auditing bureau's work---Comparisons, both as to
efficiency and cost, with the auditing offices of the great
corporations~

In 1802, when the office was first created for the city of New York, the
Comptroller was directed by ordinance to add to his audit, when
necessary,~``a statement of facts.''~In 1903, as in 1802, the pursuit
and mastery of the fact is the main business of the Finance
Department.~The added functions of the Comptroller have all developed
out of the auditing power. He is now a legislator through his membership
in the Board of Estimate and Apportionment, an administrator through his
connection with the Sinking Fund Commission and other official bodies,
and the city\textquotesingle s agent in negotiating loans. All these
duties go to emphasize the primary necessity of his being constantly in
command of the facts.~

\vspace{.15in}

\noindent 9.\emph{ Law and Adjustment Division of the Auditing Bureau}---It acts
under the section of the charter which empowers the Comptroller to
``settle and adjust all claims in favor of or against the
corporation''---It is a connecting link between the Comptroller's office
and the city's legal department, as its findings enable the corporation
counsel to act~in the light---Importance of its work--- Character and
range of the claims submitted to it---Effect of the work of the law and
adjustment division on the volume of municipal litigation.~

\vspace{.15in}


\noindent 10. \emph{Investigations Division of the Auditing
Bureau}---It~inquires~into the justness of claims which, in point of
complexity, are beyond the scope of the auditor's examiners and
inspectors, and yet cannot be assigned at once to the law and adjustment
division---The investigations division conducts special inquiries for
the Comptroller, which are directly related to the needs of the auditing
bureau.~

\vspace{.15in}

\noindent 11. \emph{Examiners of Accounts of Institutions}---This division of the
auditing bureau checks the accounts of all charities and hospitals which
receive city money---A reporting and recording system has been
introduced to compel strict accounting on the part of the institutions
affected---The city appropriates nearly \$3,000,000 annually toward the
support of about 140 charities and hospitals---The money is paid over in
monthly parts, according to the amount of work done by each institution,
and all claims are~verified at the Finance Department---The stricter
accounting illustrates the more exact methods which the enlarged city
made necessary---It involves a rigid examination of lists in detail,
books, vouchers, and accounts, including all records for facts as to
population and expenditure.~

\vspace{.15in}

\noindent 12.\emph{ The Municipal Civil Service Commission a Factor in the
Comptroller\textquotesingle s Audit}---This is due to its work in
checking payrolls---With some exceptions, city payrolls are examined and
certified by the commission---The names of employees are verified there.

\vspace{.15in}

\noindent 13. \emph{Receiver of Taxes, a Bureau in the
Comptroller\textquotesingle s Office}---Cost of collection, giving
comparisons with previous years---The busy period of the tax receiver's
office is from the first Monday in October, when all real estate and
personal property taxes are payable---On January 15 such personal
property taxes for the current year as have not been paid to the tax
receiver are turned over for collection to a city marshal, whose office
is in the tax receiver's bureau---The marshal receives his appointment
from the Mayor, while his warrants for enforcing collection are issued,
in the first instance, by the tax receiver---By a gradual process the
marshal turns over unsatisfied personal property taxes to a bureau in
the corporation counsel's office---Importance of a searching inquiry
into the business of collecting arrears of personal taxes---The personal
property tax in Manhattan for 1899 was \$11,845,000 and on January 15,
1900, the tax receiver turned over to the city marshal, as the
uncollected portion of said tax, \$4,909,000---Arrears of water rates
are turned over to the receiver of taxes for collection at the end of
the first year--- In the third year they go to the bureau for the
collection of assessments and arrears.~

\vspace{.15in}

\noindent 14. \emph{Collection of Assessments and Arrears}---This bureau in the
Finance Department collects the arrears of all taxes on real estate, all
assessments for benefits on account of street openings,
paving,~sewering, and all local improvements, including the taking of
lands for parks and public purposes generally, and the arrears of water
rates---About \$25,000,000 of claims are constantly on the books of the
collector of assessments and arrears, while the annual receipts of the
office have been about \$12,500,000---Improvements in the Borough of the
Bronx have added greatly to current business.~

The cost of opening,~sewering, widening, and the paving of the city's
streets presents an~important subject of inquiry. The cost falls on the
assessment district affected and, being a burden only on a comparatively
small body of citizens, does not awaken the general interest as the
other expenditures do.~

\vspace{.15in}

\noindent 15. \emph{Bureau for the Collection of City Revenue and of
Markets}---This division of the Finance Department collects the railroad
fees due the city, including franchise percentages and car license fees,
the revenue accruing from rents and interest on bonds and mortgages,
revenue arising from the use or sale of property belonging to the city
or managed by it, and the charges for the use of stalls or stands in the
city's markets---It collected on these accounts in 1900
\$1,076,871---The market rents for the year were \$315,473, the house
and ground rents \$116,089, the car fees \$73,640, and the receipts on
account of railroad franchises \$479,454---The car licenses in Manhattan
Borough are \$50 per car per annum (with the exception of the Ninth
Avenue road, which pays \$20), and in Brooklyn an average of \$20 per
car per annum---The allowance in the budget of 1901 for the year's
expenditures of the bureau of city revenue was \$26,050.~

\vspace{.15in}

\noindent 16.\emph{ The City\textquotesingle s Disbursing System}---All city
payments are passed through the Comptroller\textquotesingle s disbursing
room---Salaries of heads of departments, salaries of judges, moneys due
on contracts, and all supply bills are paid directly from the disbursing
room---All payrolls pass through it and are paid by warrant to the order
of the city paymaster, who makes the detailed distribution to salaried
subordinates and the city's day laborers---Almost all payments, save to
daily wage receivers, are made by check---More systematic accounting has
enabled the disbursing officer to facilitate payments and so meet the
public convenience, while securing at the same time increased safety.~

\vspace{.15in}

\noindent 17. \emph{Work of the Paymaster's Office}---The methods which had
obtained in the paymaster's office of the old city of New York were
extended and applied to the enlarged city---All per diem laborers are
paid weekly and the amounts due are put into the hands of each man in
the locality where he is employed---To effect this one paymaster sends
his clerks to all parts of the city---Increased cost of the weekly
payments as compared with the old-time bi-weekly method---About 20,000
day laborers are on the city rolls, or~one-half the civil list---The
work of the paymaster\textquotesingle s office is completely
centralized, which aids in securing the utmost of safety, economy, and
facility.~

\vspace{.15in}

\noindent 18.\emph{ City Chamberlain}---The chamberlain is the city and county
treasurer---In early ordinances the title of chamberlain was used
interchangeably for that of treasurer---Prior to the creation of the
comptroller's office in 1802 in the old city of New York the
comptroller's duties fell to committees of the Common Council--- The
chamberlain is appointed by the Mayor---Extent of the chamberlain's
power under existing statutes---The chamberlain is clothed with power
through his membership in the Sinking Fund Commission and he has a vote
in disposing of the city's bank deposits---His signature is required on
all city warrants---The chamberlain's salary is fixed by law and all
fees collected by him are turned into the city treasury---Checks upon
the conduct of the chamberlain---Section 195 of the charter directs that
the chamberlain's books shall be examined by the commissioners of
accounts.~

\vspace{.15in}

\noindent 19. \emph{Municipal Development at New York and the Place of the Finance
Department therein}---Municipal reform a scientific question---To effect
any reform in government is to extend the organization of
science---Through the acquiring of full legal authority of inquiry and
audit by the Finance Department, New York has now a centralized
government---Until there is one center which can be held responsible for
the possession of knowledge concerning all phases of city business, the
government cannot be called organized in any adequate sense---This
center at New York is the Finance Department---Power is now lodged with
the Comptroller to organize the facts---Through his office all the facts
may be known and the city government brought to a working unity---Need
of an enlarged and authoritative publicity and the way to secure
it---The Department of Finance is primary in the publicity development.~

The idea of an official Bureau of Municipal Statistics, at which an
attempt was made in 1898, can only be realized through the Department of
Finance, which is the established clearing-house for the facts. The
information must be organized by the officers who are responsible for
the record; that is, by those who are acting on the facts. The work of
the official bureau, to be up to date, must tally with the daily action,
and it is only in the Comptroller's office, where the diurnal record of
the city\textquotesingle s business is kept, that such a bureau is
possible. Anything short of this belongs to the empirical.~

As indicated above, the development is already taking place~in the
bookkeeping division of the Finance Department. The key is statistical
accounting. Its full organization will make of the Finance Department in
all respects the keeper of the counter-roll. Which means the inspection,
examination, and control of the accounts of all city departments and
officers. We are now able to understand the meaning of the word control
as used in the statute. It is the control of the facts.~

All the facts will always control the action. The Finance Department is
the common center through which the knowledge of city business gains
expression. Through its command of the facts the Comptroller's office
has a new recognition in the public mind; it stands at last for the
whole municipal organization. The Finance Department at least has a new
charter. A clear field is presented for compelling integrity in all
departments of the municipal system.~

The new and unprecedented authority with which the Finance Department is
clothed amounts to the coming on earth of a new~idea. The
Comptroller\textquotesingle s office is at last seen in all its manifold
meaning as the center of municipal regulation. It has now attained to
complete freedom of action. It has power to~stop the~facts for
examination and entry on the record, and to bar all unauthorized action.
This is accomplished by the laws which compel all transactions to be
legalized through registration at the Comptroller's office. The progress
to all this has been through a natural, irresistible development.~

The power of the Finance Department to furnish wise direction for the
conduct of city affairs is just in proportion to its co-ordination of
the facts. The larger authority of the Comptroller strikes down below
the surface alterations of the hour and presents the real change in the
city government which took effect January 1, 1902. The next step is to
realize the new position in action by organizing and making available
the facts contained in the records of the Finance Department and of all
other departments. The new power will not help unless the principle is
carried into practice by mastering and applying the facts to the whole
problem of municipal regulation. To know the way, and to do the proper
thing at the proper time, the Comptroller must possess the governing
facts.~

The necessity is to so centralize information that the actual knowledge
of each city official, in all departments, will be at the service of
everybody. The entire record will become available as means to
furthering the common interest. A final departure having been made from
the personal system of the past, under which one official was supposed
to hold in memory the entire record, no stop is possible until a
perfectly articulated machine is reached, whose efficiency will be so
great as to guarantee its own integrity. All this has become a
necessity,~because of the fact that~the personal energies of the
Comptroller are now fairly absorbed in his duties as legislator in
shaping the general administrative policy of the city through his
membership in the Board of Estimate and Apportionment.~

There is no other course, no gateway to municipal reform at New York
that will be at all comprehensive, which does not involve absolute
organization of the Department of Finance. The division of statistical
accounting therein will be a public information bureau accessible to all
citizens. It will be dominated by the spirit of science, which is a
guarantee of continuing benefits. Municipal betterment cannot
be~effected~from the outside by means of popular agitation; it must be
gained through an advance in scientific organization within the
establishment.~

Statistical accounting has now to be carried into municipal organization
for the same reasons that are compelling it in the case of the great
corporate aggregations or Trusts,~i.e., because of the greater sweep and
complexity of the action. Greater New York may, in fact, be regarded as
a municipal Trust.~The~Trust movement in commerce cannot be understood
without seeing it as a progress in accounting.~

Full statistical accounting will bring the Finance Department into close
touch with each of the city departments, amounting to a direct clearing
of facts therewith. Under early conditions the theory of the Finance
Department as center of regulation and control could not be realized in
practice. There was very little of inquiry and inspection in the early
auditing; it was enough to know that vouchers were regular or according
to statute.~

The need of building up the function of the Finance Department as main
accounting center is in proportion to the growth in power of executive
officers. The real city legislature, the Board of Estimate and
Apportionment, is a group of executive officers. At the middle of the
century the aim was to deprive aldermen and other local legislators of
all executive duties.~

The new position of the Finance Department is one with modern
communication through the telephone and rapid transit, which places
every part of the city and all the facts within easy reach. Moreover,
the advance in legal position and the fact of instantaneous
communication must be read together, as the latter has prompted the
former.~~

Reform in city government must follow the lines of more systematic
accounting and scientific publicity. The only thing to be relied upon is
the continuity of science, and its infinite organization in space~and
time.~

The organization of the facts of the municipal system is one with
ordering its activities~on the basis of~the highest economy and
efficiency. Maladministration and waste cannot be discovered without
organizing municipal~intelligence as a whole. It is only by going after
all the facts that leaks and abuses can be discovered and the narrow
interest defeated.~

In the municipal field, as in the general commerce, improved
administration is always a question of scientific method. It is
important to get this truth into the public consciousness. Popular
agitation in relation to the choice of elective officers has its place,
but the prime necessity is to gain acceptance for the fact, as already
put down, that municipal reform is a scientific question.~

Self-interest, in the ordinary acceptation of the term, does not protect
the integrity of municipal administration; that is, it does not operate
in the same immediate way as in the affairs of an individual or of a
firm. The scientific interest~has to~be let in through a development in
accounting to take the place of the ordinary safeguards.~

The rise of professional accounting is a factor in the advance. The
accountant stands for the scientific interest; his work begins and ends
with the fact. It belongs to him to organize municipal information.~

Full statistical accounting in the Finance Department will be a measure
of economy, as a great deal of time which is wasted in the~vain~pursuit
of fact by individuals would be saved. Each official, under complete
organization, will apply for information to the division of statistical
accounting. All related legal knowledge, so essential to general
officers and to the auditing bureau, would become available.~

The auditing bureau stands in special need of statistical accounting.
The record of prices and of all departmental action~has to~be perfected
to enable the auditors to act in the light. All action centers in the
city\textquotesingle s pocketbook, and the auditors are its guardians.~

Through statistical accounting, the public reports of the Finance
Department will be freed from technicality. Up to this time the
customary annual reports of municipalities have been practically
meaningless to the general public. With New York's reports made clear
and systematic, all cities will have models to follow.~

The determination of standards of efficiency in municipal service is an
important matter which will be~effected~by the development in
statistical accounting.

The division of statistical accounting in the Finance Department must be
far more than a surface compilation of official figures; it must be a
living, moving force that grows and changes every hour. It must itself
become a public idea. It involves a continuous clearing of information
between the Comptroller's office and the departments~on the basis of~the
actual facts of municipal expenditures as contained in the vouchers.
Each of the city departments is a bureau of information on its own
account.

There must be constantly at command in the Finance Department exact
knowledge of each transaction---the supplies furnished and all services
to the city, along with the money paid therefor---in addition to the
customary financial accounting. The financial accounting is not enough;
it guards against~improper payments, but does not yield readily the
information which must be at the service of the Comptroller's officers,
the public press, and citizens generally, if friction is to be lessened
to easy working. The bookkeeper's point of view and that of the seeker
after statistical information are quite different in scope.~

The facts from the departments~on the basis of~the vouchers will tally
with the statistical accounting at the Comptroller's office, and in turn
with the financial accounting. Heads of departments cannot but be in
accord with the development and will gladly co-operate in carrying it
out. It will guarantee publicity for all the facts and so compel
adherence to business principles, while it will insure recognition for
official~integrity.~

The advance in statistical accounting in the Finance Department, taken
with the continuous clearing of information by all departments, will
compel a full and exact classification of city expenditures. The
movement of city business will thus be grasped in its totality.~

This discovery of the Comptroller\textquotesingle s office---its great
function and the lines of its development---identifies municipal reform
once for all. The word organization, as used in the field of government,
gets a new and appropriate meaning. In the first instance, it looks to
the collection and co-ordination of the facts, and proposes organs to
this end.~

A new Department of Finance is revealed. Through all the years the
Comptroller's office had been crippled in its power to~audit the supply
bills of departments, and for the last quarter of a century its
authority to adjust and compromise claims was in dispute, while the
power of departments to create obligations independently of the Board of
Estimate and Apportionment prevented registration of the fact by the
Comptroller. In wonderful contrast, the Finance Department has now
attained to the amplest legal authority; it has ``control of the fiscal
concerns of the corporation.''~

The present position of New York's Finance Department~is unique. There
is no other such development. It is a true advance in government. It
points to municipal reform~in fact, and is an example for the direction
of other cities. The Finance Department appears as the organ of the fact
in the City-State. The way is opened for a great forward step in
administrative organization; it is epoch-making in the history of local
government.~~

BUT IT IS ONLY IN ACTION THAT THE NEW POSITION CAN BE REALIZED. THE
CONDUCT OF NEW YORK'S MUNICIPAL BUSINESS MUST REMAIN INEFFECTIVE UNTIL
ITS INTELLIGENCE IS SYSTEMATIZED AND SUBJECT TO THE COMMAND OF
EVERYBODY.~

\vspace{.15in}

\noindent 20. \emph{City News Office}---The new outlook for the Department of
Finance implies and compels a new and authoritative center in the local
news system---This necessity has precipitated a movement for connecting
up the whole city---The advanced position of the Comptroller's office
demands direct connection with a technical and responsible news center,
since organization of municipal knowledge involves the closest
co-operation with the news system---The new center will connect with the
Department of Finance at the very point of the latter's
development---The reporters of the news center will work directly with
the officials of the Finance Department, and the reports of both will be
equally trustworthy or official---The experts of the news center will
co-operate in the organization of statistical accounting by
systematizing and interpreting the information of the various municipal
departments.~

This new center in the local news system is the City News Office. Its
work up to this time has been to lay the foundation for the action in
hand, as indicated by this General Outline of Reports. As a co-operative
movement in news organization, measured by the necessities of the time,
the City News Office development corresponds in great degree with the
rise of the New York Associated Press in 1856.~~

The municipal reform question at New York is not primarily a matter of
statistics or detailed information, nor has the notion of framing an
ideal system of accounting, whose adoption would be compelled by the
legislature, any place in the problem. Its solution turns upon the power
to detect the development of functions and to determine their working
relation. In due course the Department of Finance has attained to an
all-round legal position which makes it the absolute center in the
municipal accounting system. This, in turn, has compelled a new and
technical news center which will make intelligent demands on the
Comptroller's office. The function of the local news center, in relation
to the conduct of municipal affairs, is co-ordinate with that of the
Department of Finance.~

In the conduct of city~business~the action of the Department of Finance,
of the Mayor, or the Board~of Estimate and Apportionment, frequently
turns upon facts outside of the Comptroller's record, which belong to
commercial and general knowledge. The City News Office will organize
inquiry in order that all such knowledge may be at the service of the
municipal officers whenever required.~

Municipal organization cannot be radically improved independently of a
distinct development in the news business. Moreover, it is true
generally that the further progress of science in the organization of
commerce, and therefore in the field of government and social reform, is
waiting on its advent and control in the publishing business, which
touches everything.~

The writers on the municipal question have not seen in its true
perspective the growing function of the daily newspaper in the field of
government.~A surprising change has now fully appeared in the attitude
of the daily press toward city government matters. One could hardly
learn from the papers in the early years of the last century that such a
thing as city government existed, while today hardly anything in the
range of city business is neglected by the newspapers. The newspaper has
come to be omnipresent; it enters each day every household in the city;
it is the daily book of the people. Because of this the developing
Finance Department must have systematic connection with the news
system.~

The next step is to bring order into this branch of reporting. The City
News Office will organize a body of experts for reporting city business,
as the daily newspaper has already done in the field of sporting news.
Municipal affairs should be reported with the same certainty and
celerity as a ball game, a yacht race, or a prize fight. Customary
reporting of municipal news lacks point and completeness---much as
though the report of a baseball contest were to stop short of giving the
score. The same influences which have led up to a reform in the methods
of the Finance Department point to a corresponding advance in the
handling of municipal news by the daily press.~

The correct reporting of various important phases of the city's life
involves a technical knowledge of the municipal government ---its
administrative methods, its underlying and controlling ideas. The
reporting of crime news, if it is to be in any way systematic and
beneficial, compels a comprehensive knowledge of the criminal code. The
efficiency of the city\textquotesingle s firemen may be affected
adversely by an unscientific organization of the fire department. The
adequate reporting of a leak in a water main calls for knowledge of the
latest engineering devices for detecting water waste. The great business
of teaching the youth of New York in the city's school system is
practically unreported by the daily press. It cannot be accomplished
save by reporters who are able to weigh and determine the value of the
ideas which shape and govern the prevailing methods of instruction.~

Reports of the City News Office~will at all times~tally with the
official record, and will in each instance further the needs of the
Department of Finance.~

An important feature in the work of the auditing bureau of the Finance
Department, when dealing with the supply bills of departments, is the
determination of just prices. When the auditing bureau subscribes to a
trade or class paper, or receives a price-list, it thereby connects
through the news system with the world of commerce. The need indicated
will be met by the City News Office, as through it the auditing bureau
will be able to connect instantly with expert authority touching the
price of this or that commodity at a given date. Trustworthy news
organization is implied in the power of the Comptroller to inquire into
the correctness of prices. The proposal assumes that prices are
determinable, that they are a matter of public knowledge, and the facts
must be obtained from the news system. In fact, when an audit was first
conceived the organization of~intelligence was begun.~

\newpage As advances in communication gradually gave rise to integrity in mail
carrying, so now the journalist is prompted to organize the news of the
municipality, and of the whole city,~in the light of science. The facts
of the metropolis may now be reported~on the basis of~truth.~

The City News Office will act as a common center for the gathering and
dissemination of all~information having a direct bearing on the
regulation of city affairs, doing in this respect what the Ship News
Office at the Battery does in its field, and with equal precision. It
will bring the municipal facts of Chicago, London, and Berlin side by
side with those of New York. At the same time, the facts of New York's
government will become accessible to the world, and the natural demand
for authentic news in this field will be supplied.~

It will seek out and connect the numerous bureaus of information doing
business in the metropolis, thus providing a common trading center
through which they can be made public and the interests of each
developed. As things stand, the local news system is~in much the same
condition that the telephone system would be in were it without a main
center through which all connections are made available. The
corresponding center in the news system will be supplied by the City
News Office, which can advertise all information centers in a single
announcement.~~

\enlargethispage{\baselineskip}

The first business of the City News Office is with the municipal system,
but to complete this relation it~must of necessity~become a universal
information bureau concerning every interest within the limits of
Greater New York, as at any time the otherwise~insignificant fact may
have an important bearing on some municipal problem. Quick access to the
facts being complete, the life of the whole city becomes clearly
objectified so that the data may be~co-ordinated~and shaped into a
governing force.~

The rapid progress in trade or class journalism in recent years is an
important factor in the preparation for the City News Office.~A number
of~technical journals are recognized authorities in the municipal field.
The\emph{ Real Estate Record and Guide}, the \emph{Engineering News},
and the \emph{Electrical World} at New York may be mentioned. A
co-operative relation will be established with the technical journals,
and gradually one class journal after another will be drawn into the
trading circle. The technical journals are centers of inquiry in the
news system. To organize the reporting of a particular branch of
municipal engineering, the appropriate class journal will be called in;
that is, the reference will be to an institution instead of to an
individual engineer. Each class paper is a bureau of~information for~its
own division of commerce.~

It is important to recognize the fact of a news system. IT IS A NEW
OBJECT IN COMMERCE. Men are writing of the American railway system. The
advent of a railway system is, of course, due to the work of the
locomotive in overcoming distance. Owing to this clearance the parts are
gradually seen as a working whole---as a single system. Under
instantaneous communication, which gives universal contact, the traffic
in news is presenting itself in a like orderly and systematic way.~~

On the side of distribution, the news system has a triple
organization---in the daily paper which handles general news, the class
or trade paper which deals in class news, and the bureau of~information
which supplies individual or personal news. It is a singular thing that
when a new fact is disclosed in the news movement it carries three
possible sales or profits: (1) its general application to be distributed
through the daily paper, (2) its class application to be sold through
the trade paper, and (3) its special application to individuals which
they can obtain through the bureau of information.~

The ultimate efficiency of each of these three parts of the news
system---the daily paper, the class paper, and the bureau of
information---is waiting on a buying and selling relation with the other
two. In other words, any general advance in news organization must be
based on the recognition of this triple system. Up to this time the
different parts of the news system have not to any great extent bought
and sold of each other, whereas the present need is a constant traffic
on all sides. The publishing business is the one industry conducted on
national lines which allows nearly all its by-products to go to waste.~
Organized, the daily newspaper will draw systematically from the class
news office, at a price to be agreed~upon, the general news concerning
each division of commerce; it is a by-product of the class paper.~

The new position of the class paper is shown by the extent to which the
daily newspapers are compelled to draw upon the editors of the class
papers for articles on the chief divisions of commerce to appear in the
annual reviews of trade. Each real class paper, under the pressure of
necessity, is possessing itself of all the news concerning its division
of commerce. The very influences which have produced the great
industrial unities or Trusts are operating at the same time to bring in
a succession of class news centers, each an integral part of the news
system.~

The action of the City News Office will lead directly to the general
recognition of the class paper in its true function and dignity. This
general recognition will secure to the class news system a market for
its by-products, namely, the general news concerning each~particular
trade~or division of industry, and the stream of facts to meet the
larger demand from individuals which will follow upon the rise of the
City News Office as the central bureau of information.~

Having regard to points of origin, the news system connects with the
whole field of science, or the divisions of exact inquiry. All~are
embraced by the communicating or news-carrying system. Every expert is
at the end of a wire.~

The physical science men, especially in England, have been trying for
some years to devise plans for "distributing the results of science."
Working under the endowment system as they do, the suggestion could not
easily occur to them that the end must be gained through a commercial
advance; that is, by a forward step in the news business. Already
certain divisions of physical science have instituted some sort of news
gathering on their own account. Chemists the world over, in order to
keep abreast of the fact, subscribe to a news leaflet which is published
by the chemists of Berlin, the center of chemical science. But any
isolated effort is ineffective owing to the absence of a main news
center through which the results of scientific inquiry on all sides can
be translated into their life bearing. The City News Office at New York
will stimulate and make certain the development indicated.~

It is important to bring the man of business to see the news system as a
function in the State---in the social system. News is not merely
detailed, isolated, or unmeaning facts; the news movement is one with
the unfolding of science. The most subversive and revolutionary fact is
always the biggest news. ~

To effect any reform in government is to extend the organization of
science. The measure of all true government, or social regulation, is
always the extent to which exact inquiry is trained upon the object. All
normal action turns upon the mastery of the facts. Progress in
government is co-extensive with the organization of intelligence, or the
reign of science, which means only a clear understanding of things. At a
given juncture the iron trade of the world was compelled to answer to
the new fact which Bessemer laid down. Likewise in 1840 the English
post-office had to respond to the facts in Rowland Hill's report. At no
time could rational government exist save so far as intelligence was
organized; it began with one who was found to be more clear-sighted than
his fellows and was consulted by them; he was their direction-finder,
their governor. ~

The natural history of parliament or the legislature brings out clearly
the identity between progress in government and the successive steps in
news organization. The early legislatures came~in as convenient
machinery, the best then possible, for finding out the fact and
promulgating it as law. The facts were brought in from the various
districts and~co-ordinated~as means to right~action. The gradual
emergence of the news business as a distinct branch of commerce is
easily marked. In this progress the year 1771 is an important date, as
it was then that the journalists at London succeeded in wresting from
parliament the privilege of printing its debates.~

The general government, then, is the system of news, of intelligence, of
science. And the need of the~hour is to~lift up~its organization to the
level of the present means of communication. The first step is to make
systematic the reporting of municipal news at New York. The~fast
printing~press and the talking wire have come in, but the manner of
classifying and handling news---the method of interpretation---has not
changed essentially. The daily newspaper is still conducted in
accordance with the ways of thinking which obtained before either the
power press or the locomotive was invented.~

News reporting---the method of reading the facts of the social body---is
as much subject to change with advancing conditions as is the method in
any other field of inquiry. As astronomy passed from the Ptolemaic to
the Copernican point of view, as alchemy gave birth to chemistry, and as
the physiologist gradually came to see the human body as an organic
unity, so now the principle of classifying and interpreting the facts of
the social body is also subject to an illuminating change, equally
radical and beneficent, and whose inevitable entrance compels a clear
advance in the news business. The social system becomes an object for
science.~

With the conquest of~distance~a new fact, amounting to a new force, has
come into the world. The approaches to this have, of course, been
gradual from the first invention of sails as a propelling power in
navigation, but the play of the forces so liberated could not come full
circle prior to the instruments of instantaneous communication. The new
force appears on the completion of the social mechanism which provides
for the full circulation of news throughout the body. The State is
objectified and the all-embracing principle of intelligence, which is
both constitutive and regulative, becomes the unerring guide in carrying
forward the organization of news. The quality of all organization is
determined by the quickness and certainty of communication.~

The new social force is the first Fact {[}\emph{sic}{]} of the twentieth
century; the first fact of the nineteenth century was the locomotive;
the second, electric communication.~

The proposed organization of municipal news, therefore, involves a
change in method---the introduction of a new point of view, of a new way
of ideas. Scientific method has entered nearly~all the great branches of
manufacturing,~but~the new forces have yet to be~co-ordinated~in the
news field. The change in the news trade, corresponding to that wrought
by chemistry in refining petroleum, has been awaiting the development in
science which has called the City News Office into existence. The
petroleum industry was organized under the guidance of chemistry, so
that in the process of refining all waste came to be transformed into
useful by-products; the crude oil was compelled to yield its last
values. The new way in handling news has the same scientific precision
as the method of chemistry.~

Progress in the news business marks time with the successive steps in
the conquest of distance. News may, in fact, be called the central
product of communication. As is the sun to the solar system, so is news
to the social system. Advances in realizing the commodity news are also
one with the growth in consciousness of the sense of a social whole, and
this, whether the whole be called the State, the body politic, the
social body, or the social organism. The very beginnings of a general
social consciousness were, of course, dependent on some degree of
communication, which always spells community of interest. This movement
in consciousness has now reached the great crossing-over point because
of the full circulation of intelligence and the consequent freeing of
all social forces for co-operative action.~

The still regnant view of the State dates from about 1700, which marked
the incoming of a~definitely~constituted~parliament in England. Since
that date we have had the ``divided'' State, with ``the Government" seen
as though on one side and the so-called private commerce on the other.
In sharp contrast therewith the State is now to be seen as a unified
object in space and time. To see an object in the light of its principle
is to transform and transfigure that object.~~

In line with this thought the action of the City News Office is grounded
in a completely worked out~science, or classification, of the social
body. The very idea of news implies a social classification. Under~any
and all~conditions, every news reporter, in the selection and grouping
of facts, is governed, however unconsciously,~by some sort of general
classification, which is his way of looking at the object or his notion
of the social order. To Aristotle the normal State was divided into
citizens and slaves, while Charles~Ⅰ lost his head through insisting on
the permanence of a given classification; he could not see the new place
and function of parliament. All progress in news organization has
tallied exactly with the successive steps in classification until, with
full circulation attained, the universal or objectively valid
classification comes to view.~

It was a clear advance at the time, following upon the rise of
parliament, to see ``the Government'' as a grouping of certain public
organs and to mark off the remaining action as private enterprise, but
the fact remains that a divided State is one which is only partially
seen. The human body was divided before Harvey, as the arteries were
thought to contain "a refined sort of air or spirit.'' There is need to
recognize that the old social classification of 1700 is no longer
useful; in fact, its application as a tool of inquiry is now productive
of more error than truth.~

Through progress in division of labor, the organic State, or social
body, foreshadowed by the speculative writers, becomes a clearly defined
object in space and time; its system of organs~is complete. The
divisions of labor therein are but another reading of the classification
of news. The news system is revealed and journalism becomes the orderly
handling and sale of fact, and this whether the news be of a new comet,
a development in physical science, or a mysterious murder. The
scientific treatment of news takes the place of the old classification
just as the Copernican astronomy displaced the Ptolemaic system. The
social system is seen and reported from the universal point of view, and
the resulting literature does away with the old or merely speculative
politics much as the books of the alchemists receded at the incoming of
chemistry. THE BODY POLITIC IS REPORTED IN TERMS OF ITSELF. We have the
key to its diversity, and by the light gained all the facts of society
fall into right relation with each other. The individual appears as
organic function in the State.~

We are in the secure path of science, and a single tongue~is provided by
means of which men can confer together over social questions in place of
fruitless arguing from varying premises. Alchemy passed into chemistry
through the unified language which Lavoisier laid down. To accomplish
the same thing for the science of politics the organs of the social body
had to be defined in their working relation; they had to be seen in
fact, and not as received from opinion. But the new tongue~is the
language of action, of commerce, of the world of business. All this
gained social observation, or prediction, passes into the domain of
science; the tangled web of fact is seen as ordered movement.~The new
science appears as a new practice in obedience to the demands of the
news business. It is the development and expansion of interest under the
pressure of human wants and activities.~

Science, exact inquiry, is the source of law and government.~The
organization of the general government~is a division of labor in the
State. The legislature, no longer adequate as the fact-finding or
law-declaring body,~is transcended in the system of news or science. In
the moving intelligence an organic unity is presented which embraces all
the differences. The abiding social bond is discovered. The organic
concept becomes a working principle for the conduct of the news
business. The way is marked out whereby the daily newspaper, to use the
words of Calhoun,\footnote{{[}John C. Calhoun (1782--1850) was an
  American statesman who served as Vice President of the United States
  from 1825 to 1832 under John Quincy Adams and Andrew Jackson.{]}} will
become ``the organ of the whole."~

The second division of labor in the governing system of the State is the
banking or credit system, now fast coming to recognition. It is a true
universal, as all divisions of commerce are subject to its correcting
influence; credit touches the heart of everything. The succession of
clearing-houses are its regulating centers, from the head center at the
metropolis to the remotest village bank, which is itself
a~clearing-house where credits are set off against debts.~

The court of arbitration continues, with enlarging power, as a universal
organ in the governing order. With the legislatures receding as a source
of fact or science, the courts are very properly more disposed to
overrule them, and the tendency is certain to increase in a marked
degree. The courts are almost constantly in session and are, therefore,
in touch with the moving and guiding intelligence as compared with the
legislatures which meet but once a year or, in some cases, but once in
two years.~

Without this vision the proposed organization of municipal news at New
York would not be practicable. At the outset, the clearance has revealed
the true place of the Department of Finance ---its function as center of
registration in the municipal system. The present scheme of taxation
cannot be reported adequately without reference to the new regulating
centers which commerce has been creating, and these can only be seen
clearly as organs in the social body. Thus, the city maintains a boiler
inspection bureau whose efficiency cannot be determined unless
comparison is made with the work of the boiler insurance companies. It
may be that the city\textquotesingle s bureau should be discontinued in
order to further the universalizing of the insurance principle. Taxes
are laid for bank and insurance regulation, but the bank and fire
insurance systems are both attaining to self-government in fact through
their own regulating centers. The changes which are appearing in the
school system cannot be reported without weighing the centralizing
influences of the time. To understand the ins and outs of the so-called
municipal ownership question, the fact~has to~be grasped that all social
activities capable of collective action are coming to unity of control
independently of the old or recognized government and that it may not be
necessary or wise to attempt their inclusion in a formal way under the
unity for which the~City Hall stands. Municipal unity does not
necessitate a common treasury. The situation again suggests the
importance of recognizing the larger and all-inclusive governing unity
which is presented in the news or science system. In short, the city
government is not an isolated entity; it is an integral part of the
social body. The various social antitheses, each of which has produced a
conflicting literature of opinion, are resolved.~

With the appearance of a social body in fact the whole question of
social regulation changes front. Recognition of this body is already
widespread in the public mind. The physical separations having
disappeared, the separateness in thought cannot longer be maintained; we
are now, in fact, members of one body. There cannot be separation after
it is discovered that each member is dependent for heat and food on the
harmonious working of all divisions of labor in commerce, which is the
universal exchange of services. In the words of Henry B. Eddy, ``This
Commerce is a giant clockwork process, compared to which the old
sea-traffic is as crude as the Columbus clock to our present timepieces.
It~is an evolution that gives promise of far greater complexity, of
becoming a system of members so delicate that not one invoice shall go
astray but the loss shall be known and appreciated by the whole
organism. Compare this era with the dying age of sea-traffic, the era of
publicity,~knowledge~and logic, with the age of secrecy, mystery and
romance, when the loss of a great ship was a vague calamity that only
years could verify. It is an evolution of childhood into manhood; of
boyish dreams into manly ambitions.''\footnote{{[}The original source is
  unknown but the quote is replicated in a book by one of Franklin
  Ford's brothers. See Sheridan Ford, \emph{The Larger Life} (New York:
  George E. Croscup, 1994), 97.{]}}~

We have realized the conditions which Calhoun contemplated when,
speaking from his place in the Senate in 1846, he predicted that with
the completion of electric communication then~just coming in, the earth
would "be endowed with sensitiveness so that whatever touches it at any
one point would be instantly felt at every other.'\,' The social body
exists~actually as~a working whole; it is a self-constituted and
potentially a self-governing organization. A final departure
is~effected~from the merely metaphorical use of the phrase social
organism. We are now able to see really what other men in the past were
only able to see ideally. The social body is governed by laws which are
inherent in its own organization; the governing intelligence is one with
the body. By the application of science in the conduct of the news
business the~natural laws of the body are discovered and political
guidance furnished; the physiology of the State is brought into use for
a clear understanding of its pathology. A further and inspiring belief
in freedom and law is gained. With clear recognition of a social body,
appeals to the natural law of its development, as against the worse than
useless tinkering through the legislatures, can be successfully made. We
get clear light on the~ever present~police question at New York. The
unfortunate policy of trying to correct~the morals of society through
extreme repression by the police is displaced by an active faith in the
freer play of the social force as means to self-development. The
eloquent words of W. K. Clifford have now a deeper meaning: ``It is idle
to set bounds to the purifying and organizing work of Science. Without
mercy and without resentment she plows up weed and briar; from her foot
steps behind her grow up corn and healing flowers; and no corn is far
enough to escape her furrow.'\,'\footnote{{[}William K. Clifford, ``Body
  and Mind,'' \emph{Fortnightly Review} 16 (July-December 1874): 736.{]}}~

The City News Office brings into commerce a new idea---integrity~in news
handling. News becomes a commodity in the full meaning~of the word. It
is a thing of convenience and may be dealt in the basis of
responsibility for statements made. A directing and all-pervading
principle may be let into the business. The organization of journalism,
though everywhere extended, must have remained nominal until the arrival
of conditions which permit the incoming of integrity, which far outruns
the mere idea of veracity.~The idea of independence in journalism is to
be realized through the appearance and acceptance of a science of news
which will~make organization actual or real. The passing of the ideal
into the real is of frequent occurrence in the development of commerce.
When the traffic in Peruvian bark became the~quinine~trade, the ideal
had become real. The crudities of journalism are certain to find remedy
through scientific and artistic advance. The time has indeed arrived for
the great development in daily~journalism, but it cannot be accomplished
by any individual paper however capable; it can only be gained by
applying scientific method to the whole news system through a new
center. In a word, the movement must be organic. ~

Men in high office are recommending publicity as the one effectual means
of solving the Trust question,~but,~failing to understand publicity,
they are looking to action by the old organs of government, with the aid
of the police power, as the way to accomplish the end. The enlarged
publicity which is very properly desired can only be secured through
recognition of the new system and its development on scientific lines.
The successive~steps in the rise of modern publicity are identical with
progress in true social regulation, but they were usually gained~in
spite of~the police power as representing established government.
Commerce has been evolving its own control through the definite
organization of intelligence. Rightly understood, the Trusts are
themselves centers of commercial regulation, and their interests~are
identical with the public demand for a parallel advance~in~news
organization. The beneficent side of the Trust movement will find
explanation through the development in literature which the City News
Office has achieved. This new literature sets the~individual in working
relation with the Whole. IT REVEALS THE PRECISE STAGE OF SOCIAL
ORGANIZATION IN AMERICA.~

\newpage The City News Office is the next step in the further organization of
democracy. The end is the exact application of the principle of voting
or registration. This will be~effected~by developing the
newspaper~interview into a system of fact registration which will
connect, in the first instance, with expert sources of information
touching all matters directly relating to the needs of New York City's
government. Thus, whenever required by the public interest the new
center will gather and co-ordinate the facts relating to a given problem
in municipal engineering, and will communicate the results to city
officials and the daily press. The development is already marked out in
the widespread interviewing which is carried on by the newspapers, but
it cannot be made systematic and authoritative without adequate
principles and methodical procedure.~

The modern practice of social registration is a political conception
whose meaning is yet to be~realized. It is practiced in various ways. At
one office land titles are recorded, at another births, marriages, and
deaths are entered, while titles to credit are registered at the bank.
The citizen~has to~register prior to offering his vote at the polls. The
system of universal suffrage through the ballot is the great example of
social registration. It is worth noting that universal suffrage could
not have been conceived practically before the incoming of means of
easier communication---~the improved highway, the mails, the locomotive,
and finally the electric wire for quickly gathering the results. Through
the ballot or other means of expression, every citizen is armed with a
negative regarding questions of public policy. On certain matters the
people are experts; everybody is an expert on the effectiveness of the
street-cleaning department. Policies are instituted by the individual,
by the technical men, but the power of negation is with the people and
will always remain with them. Universal suffrage is the Negative
Registration.~

Without reflection, it might appear that the arrival of universal
suffrage was the ultimate in democratic organization, but not so as the
degree of communication which prompted the idea of one man one ballot
was but the midway point in the application of the voting principle. The
pending advance is the Positive Registration. Under it the news or
intelligence system will form the~directive organization of the State;
the principle of representation will attain to its highest development.
In this light the doubt which exists concerning the power of democracy
to carry forward its own organization is forever dispelled. The cure
lies in the advancing self-regulation of commerce. The full remedy
appears~at the point of greatest complexity or danger when, through
electric communication, the last hindrance to the organization at
distribution of science has been removed. Under a single organizing and
transforming idea, expert voting will come to be legally constituted,
and universal arbitration will be substituted for the arbitrary will.
Countless disputes will be ended by a plain tale or fact. It is the
great co-operation of commerce and its furtherance through the news
system. The goal in social organization is to bring all the members of
the body into direct working relation~with one another,~i.e., with the
whole; each according to his~particular value~or function.~

The more the principle is used the clearer it becomes that the idea of
communication is the master key for classifying and~co-ordinating~the
new environment, the need being to bring the~net-governing~centers into
better working relation with the old~organized~Political science, once
completely worked out, is so simple and obvious that one wonders over
the difficulties which men have built up in their minds during the
progress to clear seeing.~As~in the case of all the physical sciences,
the path has been from~the myth to the object.~~

The objective existence of the social body---a thing of space and
time---must soon be a commonplace fact both to the scientific and
general consciousness. Social organization is unitarian~in reality
and~always has been. The division in the State exists only in opinion,
in the uncritical mode of reporting or interpreting life in the science
or news system. Without reflection, men of business~proceed of necessity
on the assumption that there is a social body; they contract with each
other on the basis of this assumption~and their~disputes are referred to
an organ of the body---the court arbitration. The central fact in the
development of jurisprudence~is the tacit recognition of a social body.
~

The universal is the only practical point of view for
organizing~municipal news, the preparation therefor having been made.
The procedure is as certain as that of the engineers in constructing the
Rapid Transit tunnel. They see their~object as a whole,
and~the~city\textquotesingle s~intelligence is seen in the same way,
once an adequate principle has been introduced.~~

All the positive forces of the hour are trying to attain to the end in
view. We are at the culmination of a long series of influences, all
converging on the one resultant.~~

As things stand, New York is without an authoritative news center. The
present lack of authority~cannot be remedied save through a central
office which will bring all science to bear on the conduct of affairs.
The action of the city government will be paralleled by the organization
of fact, thus providing a sure check on the movement of city business.
The new authority is the infallibility of~Science.~~

In the City News~Office~the citizen will have a center to which he will
be led to report all the facts in his possession bearing on the
city\textquotesingle s welfare. With exact method at the news center,
the town is self-reporting, especially when furthered by the larger
community of interest which will result from the new methods. A great
impulse will be given to the news movement of the metropolis, causing an
endless variety in the reporting of life. All experts will report to the
center; the "letters to the editor'' will be replete with fact.~

Under the sway of~science~the news movement will be the primary
influence in city affairs. Pages have been written concerning the credit
for the downfall of Tweed, while all the while the interest has centered
in accounting for the rise of Tweed. With full responsibility in
handling municipal news, such abuses as have gathered round the name of
Tweed could not come into existence. Owing to the growth of the
Comptroller's authority and that of other executive officers, the local
news organization must be made scientific as means to the only effective
check on the taxing power. The news system never exercises power beyond
the compulsion which resides in the fact.~~

The City News Office will~effect~a combination of scientific interests
which will take the place of spasmodic movements for municipal reform.
By concentrating on the~facts,~the forces of order will all be utilized,
while as now the waste of time and energy is enormous, not to mention
the useless expenditure of money. The mere sentiment of good government
is not enough. It must have a firm basis of fact authoritatively stated,
and such a basis can only be provided in the definite organization of
municipal science or news. It is only through scientific inquiry at the
Department of Finance, and at the centers of the news system, that the
demand for non-partisanship in the conduct of city affairs can be met.~

The City News Office will act as a general clearing-house for the
numerous societies in New York which are seeking municipal betterment,
and its literature of fact will be at their service. The usefulness of
such societies as have a definite place in the local news system will be
enhanced by the incoming of a main center, toward which they have~been
more or less~consciously working. A report submitted to the City Club of
New York on April 23, 1902 formally proposed a central clearing-house of
this order, and recounted the names of forty societies whose aim is
social reform. These organizations and the commercial news centers will
work together in furtherance of the common end. In association with the
commercial principle the mere fad element will be eliminated. An
important news center exists in the Charity Organization Society of
Manhattan. It operates over Greater New York through its allies in
Brooklyn and other boroughs. It has the Borough of Manhattan mapped and
divided into news districts. It aims to provide scientific direction for
the city's charities, but it cannot reach the highest effectiveness
without comprehensive action through a main news center. To become ``a
clearing-house of registration" for all charitable activities,~ease~and
certainty of communication with the daily press and the general public
is necessary, and this compels the organization of the city's
intelligence as a whole through a central office.~

In recent years, owing to the rise of city government leagues on both
national and state lines, the idea of a general clearance or exchange of
municipal facts has come to the front. Such an authoritative exchange of
news must organize on New York as with other lines of trade, and a new
center is needed to facilitate the movement.~Without concentration on
New York a wide exchange of information is impossible. The government of
New York City presents the universal news~interest in the municipal
field. Of late the engineers of provincial cities have complained of the
excessive demands on them for~information from other cities, and they
have suggested the need of a general center at which the experience of
each center could be lodged for distribution. The City News Office at
New York will do this for the country and the world.~

Speculative writers on municipal reform have been trying to devise the
ideal charter, while others, with keener vision, have sought reform
through an enlarged publicity. The latter have a correct idea while
trying to realize it in a wrong way. They would raise up a state
publicity department at Albany for city inspection and audit, in place
of finding the correction through a development in the local municipal
organization, the full governing machinery being already in place. The
movement must have its first center within the establishment; the
Comptroller's office is primary~in the desired advance. The writers
could not grasp the question practically. They could not understand that
the desired bureau of publicity must, in the first instance, be an
integral part of the municipal system, and, not grasping this, they have
failed to see the Finance Department at New York in its true relation.
They could not see that the Comptroller\textquotesingle s office is
center of registration in New York's government, and that any general
scheme of reform must turn upon developing its efficiency.~

The degree of news organization which electric communication permits
will be a development in government corresponding to the rise of
a~definitely constituted~parliament in England at the close of the
seventeenth century. The political reformation of that time involved far
more than the substitution of one king or boss for another. It consisted
in pushing forward the intelligence system by means of a fully organized
parliament. A like change, though of vastly greater meaning, has now to
be~effected~through the news system.~

The organized news system of New York will take the place of the
old-time Common Council. The local legislature of sixty years ago was
prompted by distance. The city was divided into districts and the
members from each came together with the neighborhood facts to confer
upon what should be done. It was a long way from the City Hall to
Fifty-ninth Street in those days, but now all parts of the enlarged city
are as one through the talking wire and rapid transit.~It is a
revolution in conditions, and as a result the central governing body is
a group of executive officers constituting the Board of Estimate and
Apportionment while as we have seen, the Finance Department has arrived
at extraordinary powers of inquiry and control. At present the
Comptroller's experts do the work which once fell to the committees of
the Common Council. The news system appears as the chief organ of the
general interest, and the necessity arises for subjecting~it to exact
method. The facts must be brought to a center by some agency, since the
old local legislature is passing into decay. The work of protecting the
general interest becomes a business pursuit.~

Municipal reform is again identified, as the problem from any point of
view is the publicity question. As the city problem is the heart of the
social question, the central need in American statecraft to-day is to
organize the municipal news of New York City in such a way that the
daily press will be enabled to report the facts in place of the present
irrelevant gossip, and worse. It becomes obvious that at no time was any
sort of municipal reform possible which did not involve and compel an
extension of scientific inquiry or publicity. It will soon appear
strange that there should have been so much discussion concerning
governmental progress since it is all bound up with extending the
organization of intelligence. It is only by this means that the idea of
municipal reform can be grasped and carried out.~The very nature of a
scientific advance is reformatory.~~


The immediate factors are compelling municipal reform on the~lines
indicated. On the one hand the necessity of carrying forward the
organization of the Finance Department compels a new center in the local
news system, while on the other the Comptroller\textquotesingle s office
is driven forward by the increasing demand of~the news interest for more
light on municipal affairs.~

\vspace{0.15in}

\hfill{\large FRANKLIN FORD},\\
\hfill 280 Broadway, New York

\vspace{0.1in}

January 15, 1903

\begin{center}
\vspace{0.15in}
\noindent\rule{2in}{0.5pt} 
\vspace{0.15in}
\end{center}

\noindent Through gaining a common center all parts of the city\textquotesingle s
news system will be brought into the public service. For a moderate
fee~the City News Office will furnish reports to any citizen or
tax~payer concerning his relations with the city government. Inquiries
will be made as demand may arise. The charge for special and prolonged
investigations will be in proportion to the work involved. News will be
sold to the newspapers of New York City and to the press of the country
and the world. All comers will~be~treated alike, as the principle
demands a free interchange of news on all sides at equal charges for the
same service.~

\enlargethispage{\baselineskip}

Yearly subscriptions from individuals, firms, and corporations will be
the first source of revenue of the City News Office. At present some
hundreds of thousands of dollars are annually subscribed by New
York\textquotesingle s men of business for the support of municipal
inquiry, through various bureaus and societies, from the viewpoint of
the general interest.~These payments show that the need already exists
in the public mind for a main center, equipped with an adequate
principle, which will develop the revenues indicated, and which, above
all, will deliver the goods for which the money is paid. The City News
Office will issue daily bulletins or reports concerning the operations
of the city government for the benefit of subscribers, and it will make
systematic the reporting of the Albany legislature. When full
organization has been~attained the City News Office will publish the
Municipal Yearbook~of New York.~

The City News Office will be a public institution, organized under the
membership corporation laws of the state of New York. It~is~at once an
educational and a commercial institution. It will succeed commercially,
through its advanced methods, in a business whose revenues from the sale
of news run into the millions at New York alone, while to bring order
into the reporting of the municipal affairs of the metropolis will be
educational in the highest degree.~

\vspace{.05in}

\hspace{3in}{\large F. F.}

% GOVERNMENT IS THE ORGANIZATION OF INTELLIGENCE OR NEWS
\chapter[Government is the Organization of Intelligence or News]{Government is the Organization of\\\noindent Intelligence or News}
\label{ch:Government is the Organization of Intelligence or News}
\chaptermark{GOVERNMENT IS THE ORGANIZATION}

\vspace{.2in}

\begin{LARGE}

\smallcaps{Franklin Ford}\marginnote{Leaflet from the General News Office, 1905.}

\end{LARGE}

\vspace{0.8in}

\begin{center}
    {\LARGE GENERAL NEWS OFFICE}

    \vspace{.3in}

    {\Large FRANKLIN FORD, Director}

    \vspace{.3in}

    {\large 280 Broadway, NEW YORK}
    
\end{center}

\vspace{.5in}

\newthought{Municipal Government in} New York has reached the stage where its further
progress requires~the erection on scientific lines of a~main~centre~for
the local News System, and to meet this need the General News Office has
been established. Its first business is the responsible or scientific
organization of New York's municipal news. The telephone and rapid
transit have given instant access to all the facts, and so have made
possible the absolute registration and co-ordination of municipal
affairs. As Government is the organization of intelligence, municipal
progress in New York is directly dependent on the action in hand. The
political question of the hour is presented. The General News Office
is~a~public institution under the principle of Contract.~

\hypertarget{the-general-news-office-will-co-operate-with-the-centres-of-the-municipal-system}{%
\subsection{\emph{THE GENERAL NEWS OFFICE WILL CO-OPERATE WITH THE\\\noindent CENTRES~OF
THE MUNICIPAL
SYSTEM}}\label{the-general-news-office-will-co-operate-with-the-centres-of-the-municipal-system}}

All the facts in any~particular case~will always control the action. The
whole evolution of the city Government has been toward the registration
and delivery of the facts for the guidance of both officials and the
public. This publicity development from within the municipal system has
culminated in the present Department of Finance, which is the city's
general accounting~centre. It has the power to compel the recording of
all transactions. The function of the Finance Department is to report
the entire range and character of the city business, but its accounting
cannot come full circle without a general news~centre~which will make
intelligent demands on the Comptroller's office, and which will be to
the local news system~as a whole what~the Finance Department is to the
municipal system. A reciprocal working relation with the Department of
Finance is implied, the foundations therefor having been laid in a
prolonged study of the Department and its place in the municipality. A
like relation will obtain with all departments of the city Government,
each of which is a bureau of municipal information on its own account.~

\hypertarget{rounding-out-the-city-government}{%
\subsection{\emph{ROUNDING OUT THE CITY
GOVERNMENT}}\label{rounding-out-the-city-government}}

A scientific news~centre, or news clearing house, is needed to complete
the organic form of the city Government. Until there is one~centre~which
can be held responsible for the possession of knowledge concerning all
phases of city business, and its prompt delivery, the government of the
municipality is not organized in any adequate sense. Within the
municipal system, a general accounting~centre~is at last provided in the
Department of Finance, but this is not enough, as the whole field of
knowledge, both local and general,~has to~be laid hold of and made to
contribute to the city's needs. The General News Office will bring the
municipal facts of Chicago, London, and Berlin side by side with those
of New York. At the same~time~it will make the facts of New York's
government accessible to the world, and thus supply the growing demand
for authentic news in this field.~

\hypertarget{the-true-idea-of-municipal-reform}{%
\subsection{\emph{THE TRUE IDEA OF MUNICIPAL
REFORM}}\label{the-true-idea-of-municipal-reform}}

The governing principle of the General News Office reveals the true idea
of municipal~reform. Reform in city government must follow the lines of
systematic accounting and scientific publicity. The Department of
Finance will provide the former and the General News Office will supply
the latter. The only thing to be relied upon is the continuity of
science, and its infinite organization in space and time. The rise of
the professional accountant, who stands for the scientific interest, is
an important factor in the advance. Without his full functioning,
municipal information cannot be organized. Unbiassed inquiry is the only
possible non-partisanship.~

\newpage\begin{center}


{\LARGE THE NEWS MOVEMENT PRIMARY}

\end{center}


\noindent{\bfseries Under the sway of~science~the news movement will become the primary
influence in municipal affairs. Pages have been written concerning the
credit for the downfall of Tweed, while the real interest~centred~in
accounting for his rise. With full responsibility in handling municipal
news, such abuses as are typified in the story of Tweed could not come
into existence.}

\hypertarget{public-reports-of-great-value}{%
\subsection{\emph{PUBLIC REPORTS OF GREAT
VALUE}}\label{public-reports-of-great-value}}

The exhaustive preparation necessary for the founding of the General
News Office has resulted in a series of original reports of the greatest
value regarding the development of municipal government in New York.
These~reports show how the advance in conditions is making for better
city government, and mark out the way in which all orders of men can
co-operate with the natural forces in solving the municipal problem. The
full remedy appears~at the moment~of the greatest~complexity and danger,
just when electric communication has removed the last hindrance to the
organization of intelligence. For the first time, we have in the reports
of the General News Office a literature of the municipal question~on the
basis of~fact. These reports possess universal interest for men of
affairs and all students of politics. A municipal advance in New York is
of world-wide interest.~

\hypertarget{procedure-with-reference-to-the-department-of-finance}{%
\subsection{\emph{PROCEDURE WITH REFERENCE TO THE DEPARTMENT OF\\\noindent FINANCE}}\label{procedure-with-reference-to-the-department-of-finance}}

Reports of the General News Office will tally with the~official record.
They will further the needs of the Department of Finance and all other
city departments. The organization of the Finance Department has been
charted, and will be reported in sufficient detail to inform the public
concerning its function and working relation with the municipal system.
It is important at this juncture that the people be brought to a new and
larger recognition of the Comptroller's office. The word organization,
as used in the field of Government, now takes on a more definite
meaning. It means primarily the collection and co-ordination of the
facts, and the perfecting of organs to this end. The conduct of New
York's municipal business cannot reach the highest efficiency until its
intelligence is systematized and at the command of everybody. Proceeding
further, comprehensive reports will be made on the function and
administrative methods of all divisions of the city Government. These
reports will have direct value for the Mayor, the Comptroller, and other
city officers, as well as the business public.~

\hypertarget{a-sure-check-on-the-taxing-power}{%
\subsection{\emph{A SURE CHECK ON THE TAXING
POWER}}\label{a-sure-check-on-the-taxing-power}}

Owing to the growth in authority of the Comptroller and the other
executive officers, who now constitute the real city legislature in the
Board of Estimate and Apportionment, there is need of a new and more
effective check on the taxing power. This must be provided through
making the local news organization scientific.~

\hypertarget{a-working-relation-with-existing-civic-organizations}{%
\subsection{\emph{A WORKING RELATION WITH EXISTING CIVIC\\\noindent ORGANIZATIONS}}\label{a-working-relation-with-existing-civic-organizations}}

The General News Office will serve as a clearing house for the numerous
societies~in New York which are seeking municipal improvement, and its
literature of fact is at their service. The usefulness of such societies
as have a definite function will be enhanced by the incoming of a
main~centre, while the fad element will be eliminated through
association with the commercial principle.~

\hypertarget{citizens-will-contribute-their-facts}{%
\subsection{\emph{CITIZENS WILL CONTRIBUTE THEIR
FACTS}}\label{citizens-will-contribute-their-facts}}

The General News Office invites all citizens to register the facts in
their possession bearing on the city's welfare. With exact methods and a
main~centre, the town will~speedily become self-reporting. A larger
community of interest will result from the new methods. By giving the
news system unity we make it real, and so are enabled to introduce
division of labor~on the basis of~science. The facts may now be
classified and~set in relation, each according to its actual value. The
central office will do for the whole movement of municipal news what the
Ship News Office at the Battery does in its field, and with equal
precision. The numerous bureaus of information already existing in New
York will be~connected together, and the resources of all made
available.~

\hypertarget{the-registration-of-expert-opinion}{%
\subsection{\emph{THE REGISTRATION OF EXPERT
OPINION}}\label{the-registration-of-expert-opinion}}

The General News Office connects with all~centres~of expert inquiry in
relation to the needs of New~York's government. A~complete system of
registration will be developed. Thus, whenever the public interest
requires, the new~centre~will gather and co-ordinate the facts relating
to a given problem in municipal engineering, and will communicate the
results to city officials and the daily press. Such class or technical
newspapers as have become expert~centres~in this field will be drawn
upon as occasion may prompt. Under modern communication, the movement of
intelligence is becoming as organic as the action of the Post
Office,~which connects with all interests and all individuals.~

\enlargethispage{\baselineskip}

\hypertarget{the-idea-of-integrity}{%
\subsection{\emph{THE IDEA OF INTEGRITY}}\label{the-idea-of-integrity}}

The General News Office brings into commerce a new concept---integrity
in the handling of news. This far outruns the mere idea of veracity.
News becomes a commodity in the full meaning of the word. It is a thing
of convenience, and may be dealt in~on the basis of~responsibility for
statements made. A directing and all-pervading principle is let into the
news business.~

\hypertarget{the-organization-of-science}{%
\subsection{\emph{THE ORGANIZATION OF
SCIENCE}}\label{the-organization-of-science}}

Using the municipal need of New~York city as the first objective, the
General News Office proposes the organization of science. The physical
science men, especially in England, have been trying for years to devise
a plan for ``distributing the results of science,'' but, working under
the endowment system as they do, they have not been able to see that the
end must be gained through a commercial advance, by a forward step in
the organization of news.~

\hypertarget{a-far-reaching-advance-in-government-inspection}{%
\subsection{\emph{A FAR-REACHING ADVANCE IN GOVERNMENT
INSPECTION}}\label{a-far-reaching-advance-in-government-inspection}}

The development of a positive or systematic~news system will insure
universal and trustworthy Government inspection. All this is involved in
the scientific idea of news. For the first time in history, the
promotion of the general welfare, which is the province of Government,
becomes a normal business pursuit. Self-interest is the duct of
sympathy. Under the inspection of a responsible and omnipresent news
organization, the late ``Slocum'' horror in the East River\footnote{{[}In
  June 1904, the \emph{General Slocum} caught fire and sank in the East
  River, killing more than a thousand people.{]}} would have been
impossible; a rotten life preserver would be instantly
reported.~PUBLICITY IS THE~FORCE WHICH CORRECTS AND REGULATES THE ACTION
OF THE SOCIAL BODY.~

\hypertarget{the-further-organization-of-democracy}{%
\subsection{\emph{THE FURTHER ORGANIZATION OF
DEMOCRACY}}\label{the-further-organization-of-democracy}}

The problem of municipal government in New York is the very heart of the
democratic question. To systematize New York\textquotesingle s municipal
reporting, and after it the whole news movement of the metropolis, is to
carry forward the organization of democracy. The Science of Politics has
been worked out by the General News Office as the basis of this
enterprise. Its formulation is in the language of commerce.~Its
application is parallel with practice,~entering into~all the
relationships of men. This new science gives the key to social order. It
is the discovery and classification of the social system, and like all
science has the last simplicity.~

\hypertarget{progress-in-the-news-business-waits-upon-this-action}{%
\subsection{\emph{PROGRESS IN THE NEWS BUSINESS WAITS UPON THIS\\\noindent ACTION}}\label{progress-in-the-news-business-waits-upon-this-action}}

The reporter is everywhere, but the news traffic has lacked a governing
principle through which all parts of the news system could trade
together and so develop co-operation on all sides. The General News
Office will~communicate the needed principle to the entire news
system.~While scientific method has invaded other divisions of commerce,
its entrance and control in the news field has waited upon the
development in science which underlies the General News Office.
The~modern economies, which have transformed other lines of business,
are now to be applied to the commerce of letters. The present advance in
co-operative reporting involves a universal application of the organic
principle. This principle has been lacking in~the Associated Press.~

\hypertarget{a-central-bureau-of-information}{%
\subsection{\emph{A CENTRAL BUREAU OF
INFORMATION}}\label{a-central-bureau-of-information}}

The General News Office is ready to supply the~particular needs~of
individuals, firms or corporations for information concerning New York's
municipal affairs. Clients will be registered on payment of fees,
varying according to the range and extent of their business. Reports
will be made by telephone when so desired. In fact, the entire local
news system will be on the telephone, so that any part of it can be
called up at will. The General News Office is~a universal bureau of
information. Its service is on the engineering, legal, or professional
level.~

\hypertarget{order-and-economy-in-the-local-news-system}{%
\subsection{\emph{ORDER AND ECONOMY IN THE LOCAL NEWS
SYSTEM}}\label{order-and-economy-in-the-local-news-system}}

During recent years the growing demand in New York for comprehensive
inquiry into municipal affairs has been met~in a partial way through the
rise of all manner of societies and reform clubs. Numerous information
bureaus operating under the profit-seeking principle are also in the
local field. The combined yearly money collections of the reform
societies and the commercial bureaus are hardly less than \$500,000. The
General News Office will bring the two sides of this confused system to
a working unity by introducing a directing and organizing idea. It will
bring order and economy into the local news system. The new~revenues
which have arisen will be developed and utilized for legitimate news
gathering. The varied functions will be classified, and each will be
stimulated to increased usefulness in its peculiar field. The advent of
a science of politics will enable the~University~centre~to co-operate in
practical inquiry. A radical advance in the organization of~credit news
is involved. The municipal system abounds in credit news, all of which
will be extracted and placed at the service of the business public. The
recognition of a news system, with a free trading relation between its
parts, is a necessary prelude to the further growth of independence on
the part of the daily newspaper.~

\hypertarget{daily-bulletins-of-municipal-news}{%
\subsection{\emph{DAILY BULLETINS OF MUNICIPAL
NEWS}}\label{daily-bulletins-of-municipal-news}}

As soon as its revenues will permit, the General News~Office will issue
to members or subscribers daily bulletins of municipal news. An
authoritative and universal publicity concerning city business will
result. The point to be grasped is that the general publicity which is
demanded in this field cannot be~attained without first collecting,
classifying, and reporting the municipal facts from the technical point
of view in order to meet the needs of all interests and all classes. The
publication of a Municipal Yearbook is proposed.~

\begin{center}
    \vspace{0.2in}
    \noindent\rule{2in}{0.5pt}
    \vspace{0.2in}
\end{center}


\begin{center}

\begin{spacing}{1.25}
    

{\LARGE MODERN COMMUNICATION AND THE EVOLUTION OF THE
STATE}

\end{spacing}

\end{center}


\noindent{\bfseries I should add that the conclusions set out above with respect to the
Government of the metropolis and the necessary lines of its development
are incidental to the larger work of the General News Office in
deciphering the effect of modern communication on the organization of
the State as a whole. With the conquest of distance, a new force,
amounting to a fundamental alteration in conditions, has come into the
world. It is the business of~Science~to interpret the new conditions for
social guidance, and this the General News Office has done. It has in
hand a series of public reports which, measured by the present political
necessity, correspond to the achievement of Alexander Hamilton and his
fellow publicists at the time of the adoption of the Constitution.
In~truth, the existing crisis in the American State may be compared to
that of 1789, when the necessities of commerce led to a liberating
advance in the organization of Government. The General News Office has
carried forward the literature of politics and jurisprudence to
the~level~of the great new action which is everywhere bursting forth in
America. We are now face to face with the political outcome of the
progress of the last century in the field of physical invention. The
result is the most important advance in actual Government since the
invention of the representative principle. Startled by the great modern
development of corporate activity, the foremost jurists of Europe are
struggling to translate the new realities, but the light must proceed
from America, where the free play of the principle of association under
Contract has reached its highest development.}

\vspace{.25in}

\hfill{\Large FRANKLIN FORD}

\vspace{.15in}

\hfill\emph{\large Director}


% THE SIMPLE IDEA OF GOVERNMENT
\chapter[The Simple Idea of Government]{The Simple Idea of Government}
\label{ch:The Simple Idea of Government}
\chaptermark{THE SIMPLE IDEA OF GOVERNMENT}

\vspace{.2in}

\begin{LARGE}

\smallcaps{Franklin Ford}\marginnote{Leaflet from the News Office, June 13, 1910.}

\end{LARGE}

\vspace{0.5in}

\newthought{Government began}, the social relation came to view, on the appear­ance of
one who was surer and quicker than his fellows in the scientific power
to determine fact, to find the way or law. Men govern, and are governed,
by means of all the relations which they hold to society. The
fact-finder, the man of uplifting influence, is a governing centre for
all who are in contact with him. The strong man, in the first instance,
was always the direction-giver, the element of physical force being
secondary. When Warwick,\footnote{{[}Richard Neville, Earl of Warwick
  (1428--1471), is known as ``the Kingmaker'' for the central role he
  played in English politics during the first half of the Wars of the
  Roses (1455--1485) which opposed the House of Lancaster to the House
  of York.{]}} the king-maker, failed to detect the incoming of new and
revolutionary conditions he was himself unmade. Science, exact inquiry,
is the source of law and government. The soldier or po­liceman is but
incidental to any scheme of government; he is an attend­ant upon the
court of arbitration, his function being to compel obe­dience whenever
necessary. The bouncer in a hotel is an important of­ficial at certain
junctures, relative to the hotel, but after all he does not direct the
business.

\hypertarget{regulation-0f-the-milk-trade}{%
\subsection{\texorpdfstring{\emph{REGULATION OF THE MILK
TRADE}}{REGULATION 0F THE MILK TRADE}}\label{regulation-0f-the-milk-trade}}

The state of the milk trade in leading cities will help to make clear
the whole question of government or commercial regulation. As now, under
the accepted theory, the trade is governed from the City Hall. The milk
inspectors are appointed by the city government, and so the regulating
agency stands apart from the trade. The results are indifferent.
Inspection is perfunctory, spasmodic; it does not in­spect. The interest
of the inspectors is not that the trade shall reach perfect regulation,
but instead that the milk shall stand in con­stant need of inspection,
and of course from the City Hall. In this way a false or `government'
interest comes to exist.

\newpage The truth is that the milk trade of any city will not get right save
through self-regulation; to be well governed it must govern itself.
There are two parties in the milk trade: one standing for honest and the
other for dishonest milk. It is to the real interest of both par­ties to
be honest, but in spite of this the dishonest minority has to be
coerced. The correcting force must proceed from among themselves where
the interest is actual. Scientific testing instruments are now in the
hands of the pure-milk men. The need is to organize or unify the trade,
and so bring it to the point of self-inspection. This done, every
dairyman would sell under the one brand, while the public, coming to be
rightly informed, would not buy from parties outside the organ­ized
trade, or Trust. Through the resulting identity of interest be­tween
producers, distributers, and consumers, the one trade mark would find
ample protection; that is to say, the quality of all milk sold would be
as certain as that of the postage stamp.

All this illustrates, or goes to make clear, the forward stride in
self-government which commerce is making under the new conditions; thro.
{[}\emph{sic}{]} it the entire Trust question is laid bare. The notion
that voting at a ballot-box was the limit in self-government is a
strange survival.

\vspace{0.1in}

\hspace{2.5in}{\large Franklin Ford}

\vspace{0.1in}


\noindent Columbia University, New York,\\ June 13, 1910

% A NEW AND REVOLUTIONARY GOVERNMENT
\chapter[A New and Revolutionary Government]{A New and Revolutionary Government}
\label{ch:A New and Revolutionary Government}
\chaptermark{A NEW AND REVOLUTIONARY GOVERNMENT}

\vspace{.2in}

\begin{LARGE}

\smallcaps{Franklin Ford}\marginnote{Letter to Nicholas Murray Butler, sent from New York, February 17, 1909.}

\end{LARGE}

\vspace{0.5in}

\begin{center}

{\Large THE NEWS OFFICE}

\end{center}

\hfill February 17, 1909~

\vspace{.2in}

\noindent Dr. Nicholas Murray Butler\footnote{{[}Nicholas Murray
  Butler (1862--1947) was an American philosopher, diplomat,
  and educator. He was a professor
  of philosophy at Columbia University, where he was later appointed
  president of the institution. Butler was also a national leader of the
  Republican Party. He was involved in each Republican National
  Convention from 1888 to 1936. Well-known for his promotion
  of internationalism, he was awarded the Nobel Peace Prize in 1931. One
  of Butler's books, \emph{The American as He is}, published in
  1908, caught Franklin Ford's attention
  and motivated the present letter. Their correspondence continued as
  Ford and Butler crossed paths on Columbia's campus, where Ford had an
  office set for him in the library.{]}} 

Columbia University, New York~

\vspace{0.1in}

\noindent Dear Sir:~

A friend has drawn my attention to a review of your recent
book,~\emph{The American as He Is}, in the \emph{London Spectator} of
January 30th. In the quotation given you say:~

\begin{quote}
Great, therefore, as is the unifying and uniting influence of the
government of the United States, its policies and its activities, the
unifying and uniting forces and influences outside of the government are
more numerous and more powerful still. They are educational, social and
economic, and they are ceaselessly and tirelessly at work.~
\end{quote}

\noindent You also indicate that in America the
words~`governmental'~and~`public'~are no longer interchangeable.~

It may not have occurred to you that the forces of which you take
account have been developing a new and revolutionary government. Do you
not perceive that the Industrial State, long held in a language of
metaphor, is at last presenting itself in America on the plane of fact,
and that its regulating~centres~are forming independently of the
inherited or Military State? In short, that a fundamental alteration in
the social constitution must soon come to general recognition. Consider
the astonishing disclosure that while the talk on all sides is of
extending the control of the Washington government, the actual
development in commerce is in the opposite direction. Take the single
example of effective control over the money or credit system which has
passed from Washington never to return, all real power being now lodged
in the Bank Clearing House, allowance being made for possible appeal to
the Courts.~

The general government of the Industrial State is comprised in the News
System and the Credit System, the operating or main~centre~of the former
being the rising News Office and of the latter the Bank Clearing House.
The two~centres~are co-ordinate or on the same level though it will be
readily admitted that no two functions can be \emph{absolutely} on one
level, as, were this true, the principle of division of labor could not
obtain, and motion would be impossible. In this case the primacy is with
the news organization, \emph{per se}; that is, the power is on the side
of the governing Fact. Yet neither~centre~can act without deferring to
the other, as each is independent in its peculiar sphere. On the one
hand, the Credit System cannot act save in the light of the directing
fact which it receives from the News System, while, on the other hand,
the News System can act freely only on the authority of a credit~which
is granted by the nearest banking~centre. Moreover, the credit cannot be
made save on a showing of evidence that it will be returned on the date
named in the bond.~

Even now, the organs of commerce, or the divisions of labor in the
American State, are making new \emph{political} maps regardless of the
old state lines; that is to say, the News System, the Credit System, the
Transportation System, the Telephone and Telegraph Organization, the
Insurance Exchange, and other branches of trade and industry, have found
it necessary to district the country on organic lines for the purpose of
commercial regulation. It all means that the \emph{national} regulation
of commerce is proceeding, in whole and in part, without reference to
the Washington government.~

Evidence abounds that the prevailing idea of government continues to be
the arbitrary will, and this in spite of the much vaunted reign of law.
The customary thought is that the only possible correction of real or
imaginary commercial abuses lies in setting the police power over
against the supposedly arbitrary~rule of the corporations. Despite all
sorts of investigations hereabouts, save the pursuit of the whole truth,
there is little or no recognition that true direction or control must
turn upon the compelling~Fact. In the main, we encounter everywhere the
erroneous and hurtful idea of sovereignty which was elaborated by
Austin; that is, the King rather than the Individual is still regarded
as the one and only governing~centre~in the State, and this whether the
monarch be hereditary or elective. Here is President Roosevelt, and
after him Judge Taft, vainly trying to carry us back to the methods of
the Tudors in England as means to an improved regulation of commerce.~

But, reversing the picture, it is perceived that government through the
new industrial organs (e.g. the Railway Traffic Association) is
constantly under the necessity of being right; that is, the decrees,
say, of the Bank Clearing House to be enforceable, must tally with
reality. Was ever government so limited?~

I would have you note that the potentiality of the money or accounting
relationship in society, regarded as an instrument of government, is
infinite, and this whether as to rewards or punishments; yet the
accepted books on Jurisprudence take no account whatever of this
universal and mighty truth. As I read the facts, the real or effectual
coercive force (police power) in the region of Contract is the Credit
System, since a denial of credit is social death.~

I have discovered that one has to stare at the~Fact for long in order to
see once clearly what the orthodox Jurisprudence of the world has
forbidden us to see at all.~

A far-reaching, in fact revolutionary, social change is pending, being
no less than the appearance of an organic banking system as function of
the whole in the political body. It means the end of
the~`private'~capitalistic regime as historically understood, and all
due to the further progress in communication so far as relates to the
immediate influence. Under the new regime, private credit will no
longer~be able to get a profit through \textquotesingle money
lending\textquotesingle{} as the entire business will be done by the
bank~centres~at a uniform and comparatively trivial rate of interest. It
is impossible to convey to you, though I were to write all day, the
social amelioration that will result from this change; it will break
through from a~centre~and reach every nook and corner of life.~

The bank check has become the standard money of American commerce, money
being seen as any instrument for transferring credit and varying as to
its universality. In this light, the mint appears as a credit
institution where one commodity---gold---has an absolute market. The
Organic Bank certifies credit without having to save
\textquotesingle money\textquotesingle{} therefor; as example, note the
action of the New York Bank Clearing House in October--November, 1907.
Credits were registered and certified during the panic to the extent of
\$100,000,000 at New York alone, and this without using any saved-up
money for the purpose. Confessedly, the bankers had no~`money,' so they
proceeded to make new money in the course of the transactions under the
name of clearing house certificates. The Bank Clearing House has now
original jurisdiction in money matters.~

In America more particularly, the industrial organization is nearing the
point where the large credits needed must be made by the bank clearing
houses acting as organs of society as a whole. The system of production
is moving forward on organic lines (recognizably so I mean, as the
movement was always organic in point of principle) while the
credit-making machinery has remained individualist, and the social
necessity demands that the antithesis be done away with. The producing
system has come to be at war, in fact, with the methods of exchange.~

The new, organic, public Credit System is to displace the inherited
individualist bank organization because the former is the more highly
organized machine. The received system of private capital will give way
just as the stage-coach receded before the locomotive. The old system
will go into liquidation; it cannot be displaced suddenly since modern
civilization is integral with the existing stock-and-bond~structure. A
social debt of large proportions has accrued in favor of the capitalist
class, which has had the direction of production for, say, the last four
hundred years. A new political or governing class is appearing in an
order of men who have the scientific habit of mind; the new banker, for
example, is to be the master accountant. The unified banking system is
the apotheosis of the~Morgans~and the~Stillmans. All credits will be
made through the bank~centres~after the manner of the clearing-house
registration in the~Fall of 1907.~

The Bank Clearing House is taking the place of the Stock Exchange.~

To the general consciousness, and for that matter the world of business
as well, the economic relation in society has not changed essentially
since the invention of gunpowder, yet a revolution in this respect is
now impending. And the matter is so simple and clear that were it to~be
communicated by authority the news would spread like wildfire and become
almost at once a popular conviction, since everybody is interested in
the money question. Besides, it is not commonly understood that under
modern communication all parts of the social body are in the closest
touch with each other, either directly or through~centres, that is, a
sensation received at any one point is instantly felt at every other
point.~

It is true that the so-called money economy long ago displaced the
regime of personal service and payments in kind peculiar to the
feudalism of the Landlord, and that slavery gradually passed into the
modern wage system, but nevertheless the fact remains that, from the
central viewpoint, no radical change in the economic relation has taken
place. To obtain a large credit 'securities' have still to be peddled to
Tom, Dick and Harry.~

Schmoller,\footnote{{[}Gustav von Schmoller (1838--1917) was a German
  economist interested in social policy related to urbanization and
  industrialization.{]}} the German economist, helped me to understand
that the constant progress of civilization has been toward greater and
greater economic unities, each taking the place of a passing
individualist system. At last, the mediaeval or individualist exchange
system is to pass under the rod, and the sooner the alteration is
brought to consciousness the better as an infinite development in
commercial or social co-operation is waiting on the revolution.~

The conduct of exchanges through the intervention of money and progr-\\\noindent ess
in division of labor have moved parallel with each other, the one being
everywhere~necessary to the other. In fact, progress in freedom of
exchange and co-operation through division of labor are but two readings
of the one social movement. Again, in modern society production never
outruns facilities of exchange, while, on the other hand, a breakdown in
the exchange system halts all progress on the side of the producing
interests. Free exchange and the mobility of property are one and the
same thing.~

Under separate cover, I am sending certain printed papers which should
have interest for you. Two of them are copies of certain Bank News
Bulletins which I issued through the Credit Office just before and
during the panic of 1907. In the issue of December 4th,~I have marked a
memorandum on the Future of Social Organization. You are welcome to keep
the printed papers.~

\newpage With this, I beg to enclose a letter received last year from one of my
fellow students, Justice O. W. Holmes of the U. S. Supreme Court. Please
return the letter.

\vspace{0.2in}

\begin{center}
    
{\large Very truly yours,}

\vspace{0.1in}

\hspace{0.3in} {\large Franklin Ford}

\end{center}

% NEWS IS THE MASTER ELEMENT OF SOCIAL CONTROL
\chapter[News is the Master Element of Social Control]{News is the Master Element of Social Control}
\label{ch:News is the Master Element of Social Control}
\chaptermark{NEWS IS THE MASTER ELEMENT}

\vspace{.2in}

\begin{LARGE}

\smallcaps{Franklin Ford}\marginnote{Letter to Oliver Wendell Holmes Jr., sent from New York, January 19, 1912.}

\end{LARGE}

\vspace{0.5in}


\begin{center}

{\Large THE NEWS OFFICE}

\end{center}

\vspace{0.1in}

\hfill New York, January 19, 1912

\vspace{0.2in}

\noindent Hon. O. W. Holmes,\footnote{{[}Justice Oliver Wendell Holmes Jr.
  (1841--1935) was an American jurist who served as an Associate Justice
  of the Supreme Court of the United States (1902--1932). One of the
  most cited American legal scholars of the twentieth century, Holmes
  supported the constitutionality of state economic
  regulation and advocated broad freedom of
  speech under the First Amendment. A key figure of American pragmatism
  and progressivism, Holmes cofounded the Metaphysical Club in 1872,
  alongside philosophers William James and Charles Sanders Peirce. The
  Ford-Holmes correspondence started in September 1907 with a letter from Ford
  to Holmes and lasted until Ford's death in 1918. Their correspondence
  was later partly published but this letter was not included in the
  collection. See David H. Burton, \emph{Progressive Masks} (Newark:
  University of Delaware Press, 1982).{]}}~~\\
Washington, D. C.~~

\vspace{0.1in}

\noindent My dear Sir:~

\vspace{0.05in}

It turned out that I could not deal adequately with the questions
submitted to me in your letters of last summer from Beverly Farms
(August 17, 18, and 26) without more reflection than I at first thought
necessary, which compelled delay. After a bit, I was led to see that
what you wanted from me was not merely specific replies to this and that
query, but instead the rounded summary of my findings, so far as I
should be able to arrive at them.~

Happily, since my last communication, I have been able to conclude the
research work which forced me to the library of Columbia University, so
that on the score of history I now feel myself better equipped for
submitting to you the conclusions reached. Besides, certain further
insights touching the business aspect of my project have reacted on the
ideas involved, and, consequently, have helped to make the theoretical
exposition more satisfactory, at least to me. As previously put down for
you somewhere, my task proved to be of a dual nature, comprising both
the introduction of a new science and its execution in the market-place.
Regarding the counting-room aspect of the job, it occurs to me that a
striking letter which I received in September from a representative New
York journalist, Mr. Frank Parker Stockbridge, may prove of interest to
you. I enclose a copy, which you need not return.~

\newpage The seeming delay has gone to clarify my response to your questions and
demands; so much has happened in the interim, as, for example, the
messages of President Taft and the suit for the dissolution of the Steel
Trust.~The difference between the old State-centre~and the Industrial
System is so fundamental that I am unable to see how it can be
compromised any longer with safety; the real Trust Question must soon be
met. My understanding is that the working arrangement for a century has
been a compromise, but, as frequently happens in politics with crude
adjustments of the sort, the outcome is a wider gulf.~

The care and exactness required in trying to meet your demands is a
valuable thing for me. I get the impression from your letters that,
while we are looking in the same direction and at the same things, the
value you give to the facts, your perspective, is different from mine,
though not radically so. Were we not looking in the same direction, it
would hardly be~worth while {[}\emph{sic}{]}~to re-cast and sum up the
matters between~us.~I feel that their importance justifies the utmost
effort on my part to bring out, and if~possible~to reconcile, the points
on which we are at variance; I hope to be able, at least, to establish a
clear difference. The only workable conclusion is that any fault in the
premises attaches to me. Your letters intensify my desire to meet you
face to face, but my thought is that I must first do my best to clear up
and enforce in writing the contentions which from time to time I have
put before you.

\enlargethispage{\baselineskip}

Allow me, first, to clear the ground by bringing up, in a measure, the
record of our interchange of fact and opinion. You will recall that I
was led to send you the outcome of my inquiry into the state of society
because of a reference to you by Sir Frederick Pollock,\footnote{{[}Sir
  Frederick Pollock\textbf{ }(1845--1937) was an
  English jurist and law professor at the University of Oxford
  (1883--1903). One of the leading English legal historians of his time,
  he also served as editor of the \emph{Law Quarterly Review} and
  the \emph{Law Reports,} the volumes in which decisions of the English
  courts were published. He is best known for his \emph{History of
  English Law before the Time of Edward I}, written with F. W. Maitland,
  and his lifelong correspondence with US Supreme Court Justice Oliver
  Wendell Holmes.{]}} who, as he said, had been helped by two
men, yourself and the late F. W. Maitland.\footnote{{[}Frederic William
  Maitland (1850--1906) was an English jurist and historian.{]}} On the
word of Pollock I turned to your writings, and, pursuing them, I was
particularly impressed with the statement of your economic faith as
revealed in your dissent in the Massachusetts labor case
of Veghelahn vs. Guntner (167 Mass., 92).\footnote{{[}Veghelahn vs. Guntner,
  167 Mass. 92 (1896) is a United States labor law decision from the
  Supreme Judicial Court of Massachusetts. Seeking to raise attention
  to their cause and hoping for higher wages, a union had picketed in
  front of an employer\textquotesingle s business with the objective
  to convince current employees and job applicants to not enter
  the building. The court found that the coercion and
  intimidation found to have occurred interfered with the right of an
  employer to} In this dissenting opinion
you said:

\begin{quote}
It is plain from the slightest consideration of practical affairs, or
the most superficial reading of industrial history, that free
competition means combination, and that the organization of the world,
now going on so fast, means an ever increasing might and scope of
combination.~It seems to me futile to set our faces against this
tendency. Whether beneficial~on the whole, as I think it, or
detrimental, it is inevitable, unless the fundamental axioms of society,
and even the fundamental conditions of life, are to be changed. One of
the eternal conflicts out of which life is made up is that between the
effort of every man to get the most that he can for his services, and
that of society, disguised under the name of capital, to get his
services for the least possible return. Combination on the one side is
patent and powerful. Combination on the other is the necessary and
desirable counterpart, if the battle is to be carried on in a fair and
equal way.
\end{quote}

\noindent It is worth noting that this case was decided by the Massachusetts
Supreme Court in the fall of 1896, the majority ruling against the
custom of picketing on the part of striking workmen.~

Almost\marginnote{hire whom it pleases and ruled that the
  union was guilty of an intentional tort. However, Justice
  Holmes disagreed, equating the use of collective force by workers to
  the corporate use of force to compete.{]}} at the opening of our correspondence, you referred me to an
outgiving of yours in the \emph{American Law Review} as far back as 1872
(vol. 6, p. 725) in which, summing up certain lectures on Jurisprudence
which you had given at Harvard University, you said:~

\begin{quote}
Sovereignty is a form of power, and the will of the sovereign is law,
because he has power to compel obedience or to punish
disobedience,~\emph{and for no other reason} (ital. mine). The limits
within which his will is law . . . are those~within which he has, or is
believed to have, power to compel or punish. It was shown . . . that
this power of the sovereign was limited not only without . . . but
within . . . by organizations of persons not sharing in the sovereign
power.
\end{quote}

The above extracts helped me to see that you view the social body as the
outcome of an evolutionary process, or a development in time; that it is
subject to ceaseless change, developing indeed under our very eyes as
new conditions enter in. I caught from your writings, and from your
first letters to me, that you read history in order to~understand~the
real present, and, such is your persistent youthfulness of mind, that
you are always facing the to-morrow of things. It is, therefore, not
surprising that I at once placed the highest value on your criticism,
and I resolved to do all that in me lay to prompt your giving it to
me.~~

The finding of Maitland, by the way, was an event in my life. It came to
me in 1904. I had been watching the intellectual frontier of Europe for
long in the hope of finding a change of front among conservative men
regarding the constitution of the State. I detected the change in
Maitland. The~particular reference~is to Maitland's Introduction to his
translation of a part of Dr. Otto~Gierke's~``Political Theories of the
Middle Age.'' This~Gierke--Maitland volume was published in 1900. For
me, it went to connect the advanced legal thought of Europe with the new
political facts~which~are demanding recognition in America. I was
enabled to see that the European jurists were seeking to grasp ``the
State,'' or body politic, as an organism. I saw that they were getting
into line with the organic view which is compelled by the new and
far-reaching action that is everywhere unfolding itself in America. What
they were trying to see and express in terms of legal principles, I was
seeking to formulate in terms of fact in obedience to the necessities of
the news business. The Germans, and after them their co-workers in
England, were endeavoring scientifically to conceive the very nature of
social organization. The aim, in the words of Prof. Maitland, was to
``give . . . precision and legal operation to thoughts which are in all
modern minds.'' I had been writing and speaking of the inherited,
single-centred~or Military State and, by contrast, of the new,
many-centred~or Industrial State. The distinction appeared strange, and
therefore difficult, to some of my friends. You can imagine, then, my
satisfaction over the discovery that Professor Maitland of Cambridge
University, England, was writing of the ``unicellular State'' of earlier
days and, in sharp divergence therefrom, of the ``multicellular State''
of the present time. The venturesome newspaper man suddenly found
himself in orthodox company.~

Then you came along and gave me yet greater reason for believing that I
was on the right track.~Under date of May 3d, 1907, {[}\emph{sic}{]} you
wrote me concerning the observations I had been sending you as
follows:~``In general they are in the direction of what I have thought
for many years, although of course I have not had the specific knowledge
which enabled me to see the actualities in detail.''~In this same letter
you spoke of your reflections~``on the extent to which new organs were
replacing old in the structure of the State.'' In a communication to me
of February 8th, 1908, you said:~``Of course I agree that the~movement
of~society is organic, that the separation of government from the
other~centres~of power is philosophically empirical, and that
other~centres~are growing faster than government technically so
called.''~Also,~``In some points you {[}I{]} seem to think that I differ
from you~where~in fact I fully agree.''~In a previous letter (April 29,
1907) you told me that you had~``never been deluded by the academic
legal theory of government into supposing that legislation was more than
one exhibition of social power.''

After this, I felt warranted in concluding that my task was no more nor
less than to work out the implications of your own economic and legal
position, and this by observing and recording the new actualities of the
rapidly advancing modern environment. The need was to~\emph{measure}~the
extent to which new organs are~``replacing old in the structure of the
State.''~By the way, I write it~\emph{dis-placing}. What are these new
organs, and what of the old ones are they displacing? Above all, the
necessity is to determine the function and scope of the new~centres~of
power. Allow me to say that the only claim of~Science~to our attention
and respect is its ability to measure, to fix limits, to identify this
or that idea as revealed to us on the plane of fact.~

More than once in your always helpful letters you assume that my aim is
to set out some new and fundamental concept, and you make the demand
that I define it in precise terms. Thus, in your favor of August 18,
1911, you say: ``You {[}I{]} claim~renown . . .~for revelations of a
philosophical~significance,''~and you ask,~``What is the new
enlightening thought?''~I have been, and am now, at a loss to understand
how it is that you could have been led to give me such a rating, as I am
unable to identify it in any of my letters or writings. In my talks
to~friends~I have constantly put aside all such claims. In my
co-ordination of the facts about us, or, to use your own phrasing, the
teachings of affairs, I am of necessity moved and controlled by the
organic concept, but I found it,~\emph{qua}~concept, in the books.~You
yourself are under the influence of this same idea when you detect the
emergence of new organs in the body. I have been applying or
interpreting the organic idea in terms of fact, but I had no share in
its discovery.~That has been the slow work of centuries, as it gradually
found place in the mind as a tool of inquiry, of exact observation. I
take it that the idea of an organism is our highest, or most
comprehensive working concept.~

\newpage I will say, however, that the fact which came to you---the replacing of
old organs in the State by new ones---is admirably calculated, once it
is grasped by the general consciousness, to influence our
social~philosophy in the profoundest manner.~

My work is wholly of the practical order. The late T. M. Cooley, Chief
Justice of the Supreme Court of Michigan, with whom I had a lasting
friendship, used to say of me: ``Whatever comes of Ford's work, it is
plain that~his~is the practical mind; that is to say, nothing would
interest him profoundly save something to be done in space.''
Theory-makers are very important people, but I am not in their class. It
is true that I have done a lot of laboratory work, historical and
otherwise, but that was because, after long searching, I could find no
one else to do it. At the outset of my attempt to organize the traffic
in news I was in the habit of saying that if I could fill the chair
of~Politics~I could accomplish it. The result was that I had to take the
post myself.~

I found that, on the plane of principle, the idea of an organism was
sufficiently set out in the books. Proceeding to apply the idea, under
the half-conscious influence of the new and more organic environment,
such philosophers as the late Edward~Caird,\footnote{{[}Edward L.~Caird
  (1835--1908)~was~a~professor~of philosophy and~one of the key figures
  of the idealist movement that dominated British philosophy from 1870
  until the mid-1920s.{]}} of Oxford University, have surmised that
radical social changes are near at hand. For example,~Caird~wrote in his
book on Kant (vol. 2, p. 376):~``In modern~times . . . neither the State
nor the Family any longer represents the highest moral unity of which we
can conceive; although, as a matter of fact, no higher unity has yet
taken an organized form.''~Edward Jenks, the English law writer,
ventured upon the prediction:~``In truth we look, for the future of
Contract, not to the gentile organization of the Clan, nor to the
military organization of the State, but to some~as yet~undeveloped
institution, which shall supersede them both.''\footnote{{[}Edward
  Jenks, \emph{Law and Politics in the Middle Age} (New York: Holt,
  1908), 291--92.{]}}

This new institution is the News System, taken with its cognate organ
the Banking or Credit System.~My~job was, first, to lay the
foundation~therefor~in the new, organic literature of Politics, parts of
which I have had the honor to send you, and, second, to give motion
thereto through promoting, as a money-making proposition, the News
Trust. The first job I have done; the second is in hand. I think it fair
to say that I have done as much strictly scientific work in my field as
Darwin did in his, but I~have to~do twice as much.~

You ask what I have done in the way of ``getting and communicating
news.'' Dismissing my previous career as a reporter on daily newspapers
and as editor-in-chief for~seven~years of the weekly newspaper
\emph{Bradstreet's}, I may say that I have been efficient, though my
associates have borne the brunt of the work, in erecting the Credit
Office as a practical working~centre~for the registration of
credit-news. It is hardly more than a successful working model as yet,
though it has a firm foothold in the textile industry, but the future
belongs to it. It has been immensely serviceable to me as an instrument
for getting at and~co-ordinating~the facts touching the real state of
the Banking System.~Our method is that of the physical
science---experiment. I have learned through constant and oft-repeated
trials. The butt-end of the general news which I have gathered is the
universal interpretation of your own disclosure, namely, that new organs
are appearing in the structure of the State.~In all this, you reported
to me a portentous truth, in fact revealing the social secret of the
hour. Given its full rendering in terms of the world of business, it
ranks as sensational news of the first order, being second only to the
solution of the mystery of life itself, toward which the biologists are
struggling.~

I fear that you are far from grasping the full significance of the
function of News, seen as the master element of social control. To me,
the organic news system is the lineal~descendant of the Courts.~What, I
pray you, is the function of your honorable Court if it is not to
determine and announce the governing facts of society as they unfold
themselves before you? It is impossible to conceive in any~great
detail~the infinite social meaning of the organized or centralized news
traffic which the near future has in store for us, and this without
reference to anything my friends and I may do in the way of furthering
it.~

To organize anything is to centralize it. I gather from the biologists
that when Lady Nature sets out upon a new~job, such for instance as the
construction of an eye, she first fixes upon a~centre~and then builds
around it. In the field of news, the best example of systematic or
responsible organization that I know of is time-news. The astronomer
appears to us as the~Newsmaster-General. In this and a few other
particulars, Society has already an adequate sensorium or brain-centre,
for example, the reporting of sporting-news and ship-news. To some
extent the news of staple commodities has become organic or complete,
especially with respect to stocks of goods, the course of prices, etc.,
the weekly statement~re~the visible supply of cotton being a striking
instance especially as the facts come from the ends of the earth. An
adequate report bringing out the extent to which news is now organized
on the plane of science will close the discussion as to whether society
is an organism.~We have in hand such a report.~

A few lines concerning your use of the word~\emph{commercial}. I read in
your letter of August 17th, the reference being to the present social
juxtaposition;~``Of course~every one~sees that all sorts of powerful
organizations have arisen outside of government, and that they are real
powers. You {[}I{]} are interested in the commercial side.''~This
oblique use of the word~\emph{commercial}~has no place, properly
understood, with reference to the incidence of an organism, such as
society has now, in fact, become.~I regard the handling of time-news as
a perfect~``commercial''~operation.~On the basis of~a living, moving
organism the adjectives~``commercial''~and~``social''~are
interchangeable. The new organs of government under the principle of
Contract are appearing just because a higher efficiency is required to
meet the~fast~increasing~social complexity. You and I were born~in the
same decade, and we received,~therefore, about the same social
inheritance, and along with it certain mental lesions or false
separations.~It has been a struggle with me to get rid of some of them.
For instance, in some of my old notes there are references to the
necessity of determining the~``province of government,'' whereas in an
organism the function or scope of government is as universal as the play
of sunlight.~

It would have been more to the point had you noted that I have been
drawn to the necessity of bringing all aspects of government under a
commanding unity. I find the key to this in the movement of News. The
genius of the news business cannot be understood at all save from the
governmental viewpoint; in short, to make the news traffic systematic or
responsible in any direction is to develop government. It is impossible
to exaggerate the importance of any fact within the area of its
application or control. If you don\textquotesingle t agree, try it. As
things stand, men have no common tongue by which to confer together
touching the social relation, and so we are compelled to a new language,
or rather to a reform in language. The reform in political thinking
which the times demand is directly dependent on a reform in language.
The newspaper discussion of the day offers abundant testimony that, thus
far, men are no nearer to understanding each other concerning the
over-arching question of government than were the men of the seventeenth
century, when the way to a settlement was~a fight.~

In your communication of~August 18th,~you say:~

\begin{quote}
The general fact that you {[}I{]} emphasize, that other organizations
than the old States philosophically are to be regarded as on the same
plane as the State, as popularly understood, is familiar. When lawyers
use the word~\emph{law}~in the narrow~sense~they do it on practical
grounds. The object of their study, as I said in an address in 1897, is
the prediction of the incidence of the public force through the
instrumentality of the Courts.~Therefore~they confine the word law to
the body of prophecies as to the cases in which the axe will fall. There
is no objection to their doing so unless a practical is mistaken for~a
philosophical division, of which I should think there long had ceased to
be any danger on the part of those who count.
\end{quote}

The above is well enough as a matter of academic discussion, or as
regards yourself and your fellow judges in your private intercourse,~but
the trouble is that the general public, under the centuries-old teaching
of the lawyers, has mistaken the~``practical''~for
the~``philosophical''~classification, so much so indeed that the run of
men have been led to believe that the old State-centre~(the Legislature
and the Courts) is endowed with a law-finding and law-declaring
monopoly. The only idea of law which the people are ably to grasp and
hold is that which is~regnant. The difference between us here is
profound, going to the root of things.~

The ``practical'' classification, which you pass over so lightly, is
taught in all the common schools and universities of the country as the
sole orthodox view, one of the points being that the issuing of
``money'' belongs only to the Washington government. The notion that the
general government of the nation is entirely and wholly~centred~at
Washington controls the political thinking of President Taft, of Bryan,
and of Roosevelt. The prevailing theory of the corporation springs
directly from it. Our official teachers tell us on every hand that the
``Money Trust'' and the corporations are ``usurping the powers of
government.'' When passing the New York Bank Clearing House, in Cedar
Street, not long ago I spoke of it to a companion as a new institution
of government, when he at once hooted at me the intimation being that I
was talking nonsense. Roosevelt, as President, held out on every
occasion that commerce runs lawless save as it is ``regulated'' from
Washington. The hostile attitude of the so-called workingmen toward the
new industrial unities or Trusts, of which you complain in your letter
of August 17th, is traceable to these same official teachings, while, as
the record shows, both Congress and the Courts are under the spell of
kindred ideas, nor could any other outcome be expected as neither
legislators nor judges can act independently of the prevailing opinion;
the hands of the judges are tied. A Columbia University professor when
lecturing recently on Jurisprudence described ``law as that part of the
social order which by virtue of the social will may be supported by
physical force.''\footnote{{[}Munroe Smith, \emph{Jurisprudence} (New
  York: Columbia University Press, 1908), 14.{]}} It seems to me that it
would be nearer the truth to say that law is that part of the social
order which does not require physical force for its maintenance. To me,
both statements are of course vague and unsatisfactory. You yourself,
somewhere in your writings, use this language: ``the universal
will,~i.e. by the State, acting through its organs, the
courts,''\footnote{{[}Oliver W. Holmes Jr., \emph{The Common Law}
  (Boston: Little, Brown and Company, 1881), 207.{]}} the fair inference
from this being that the social will has no other expression. The idea
was once true enough, but it has become archaic, and this owing simply
to the further evolution of society. And here is a new and highly
respectable English textbook, published last year, (Tenth edition of
Broom's \emph{Commentaries on the Common Law}) which informs us that
``there are two kinds of law---the law of God and the law of the State.
There is no other kind of law.'' The concession to God is worthy of all
praise.~

I am putting down things as they occur to me without strict reference to
logical order; that is, with something of the freedom of conversation. I
shall draw freely from my notes, and in due course will deal with your
specific questions.~

I have found profit in studying your address in 1897 before the Boston
University School of Law (Harvard Law Review, v. 10). In it you say:~

\begin{quote}
We do not realize how large a part of our law is open to reconsideration
upon a slight change in the habit of the public mind. . . . We are
still~far from the point of view which I desire to see reached. No one
has reached it or can reach it~as yet. We are only at the beginning of a
philosophical reaction, and of a reconsideration of the worth of
doctrines which for the most part still~are~taken for granted without
any deliberate, conscious, and systematic questioning of their grounds.
. . . For the~rational study of the law the black-letter man may be the
man of the present, but the man of the future is the man of statistics
and the master of economics. . . . The present divorce between the
schools of political~economy and law seems to me an evidence of how much
progress in philosophical study still remains to be made.~
\end{quote}

\newpage It is interesting and suggestive to compare with what {[}\emph{sic}{]}
you said in 1897 certain remarks of Prof. Simon N. Patten, speaking as
President of the American Economic Association in 1908.~He first
declared, touching the work of the economists, that~

\begin{quote}
no great American problem has been solved. With every vital~question we
stand on a halfway ground, halting between the old and the new, and if
these~half truths~are all we have to offer we may harm the public more
than we aid them. Confusion and defeat stare us in the face politically,
morally, and economically, if the disappearance of old customs,
traditions, and modes of thought is not followed by the rise of new
concepts, ideals, and institutions.~We cannot afford to be mere
iconoclasts. We must lay the foundation of a new civilization and show
how economic forces will remedy evils that may soon be unbearable.~
\end{quote}

Further on in the same address Patten had this to say concerning the
relation of law to economics:~

\begin{quote}
Law . . . is the one social science that~has advanced solely by
evolution, and we have much to gain by acquiring its spirit. And law
would gain equally by an alliance with us (i.e. with the economists);
for the socializing of law is the most important and pressing need of
the American people. Legal~encrustments~of social traditions are the
worst foe of progress. Law can be made mobile only by the proper
appreciation of economic change; economics can be saved from a series of
revolutions only by the spirit of law. When these two sciences are
properly blended evolution will be constant and progress orderly.~
\end{quote}

It hardly needs to be said that so long as Jurisprudence and Political
Economy are divorced from each other we cannot have a true idea either
of social law or of economics.~

As I read the facts, the inherited Legalism is bankrupt, in the face of
the vastly increased social complexity. The necessity has arisen to get
beyond the fiction that the old State-centre~is the one and only source
of social law, but this involves nothing short of a revolution in the
mind and practice of the race; we are all monarchists by inheritance,
and the professional Socialists are the worst. If we are to advance at
all beyond the sixteenth century idea of the social relation---that of
the single-centred~State---which has continued regnant until now, we are
face to face with a change in political theory as great as was the
alteration in the science of astronomy in passing from the Ptolemaic to
the Copernican viewpoint. But the change has occurred in fact, and the
new attitude of mind is inevitable. A knot~has to~be cut which is tied
in the human brain alone.~

The need is for a new universal authority, a new sanction, taking effect
through new organs, and this without directly attacking the prestige of
the Military Power. Moreover, a new instrument of coercion is necessary,
an instrument as definite in its operation as the jail or the headman's
axe, and at the same time infinitely more effective in detail else the
new complexity cannot be successfully dealt with. All forms of human
association must contain within themselves some effective mode of
control; that is to say, any manifestation or extension of social power
must develop the capacity to reward or punish. Mere physical~force,
whether~exerted against persons or property, is an exceedingly crude
instrument of control, so much so indeed that it was only adapted to the
conditions of primitive society. And all experience shows that~physical
force~cannot~be made more coercive by adding to the penalty, even though
to the point of slow torture.~Of course~any development in social law
must find expression in new institutions in order to gain popular
recognition; that is the people apprehend the operation of law only in
its execution. The point of execution is the dramatic and visible thing;
the government is seen in the hangman. Again, the new mode of control
must be rational or in accord with the existing constitution of
society.~

So far as I am aware, our sociologists and jurists have done nothing
decisive for us by way of tracing the evolution of punitive force and
its application in society. Certain changes in modes and severity of
punishment are recorded, but to report this order of facts is an easy
task. We are told that culprits were both hanged and quartered in the
seventeenth century, but,~as yet, no one has had anything of moment to
say concerning refinements in the public force in spite of the fact that
the flexibility or adaptability of force must keep pace with growth in
social integration or the increase in complexity. It is hardly necessary
to add that inquiry in this direction could not progress without a clear
departure from the received viewpoint. Concerning this inheritance, when
reporting to Ex-Judge John F. Dillon\footnote{{[}John Forrest
  Dillon (1831--1914) was an American jurist who served as a Justice of
  the Iowa Supreme Court and a United States Circuit Judge. He is known
  for his theory of state preeminence over
  local governments, called Dillon\textquotesingle s Rule. Ford
  corresponded with Dillon with whom he discussed legal and historical
  questions.{]}} in 1905 touching the progress of my work, I wrote as
follows:~

\begin{quote}
Gradually, I came to realize something of the extent to which the public
mind is resting under the tyranny of the inherited notion that the
general government of the nation is entirely~centered~in Washington, and
that all petitions for relief must be sent there. This idea of the
old~centre~has passed into what is little better than a superstition.
The undue emphasis concerning it is easy to understand as the rise of
parliament was identical with the progressive triumph of the democratic
spirit. It is a natural sequence, therefore, that both politicians and
newspapers should contribute to what has now become a distortion of
fact. I was enabled to see that the popular consciousness is held in
thrall by an idea which, though once true in good part and useful as a
tool of inquiry or method of classification, has become false and
misleading.~
\end{quote}

\noindent Since 1905 the shadow of the old~centre~has continued to darken counsel.
All the Utopists, inclusive of President Taft, Col. Roosevelt, and Mr.
Bryan, save of course the anarchists, are more energetic than ever in
preaching the political philosophy of the sixteenth century.~``The
enduring work of the sixteenth century,''~says J. N. Figgis,~``was the
modern State. Its legal omnipotence and unity, the destruction of all
competing powers, separate or privileged, was assured, and a universal
all-embracing system of law became possible.''\footnote{{[}John N.
  Figgis, \emph{Studies of Political Thought from Gerson to Grotius,
  1414--1625} (Cambridge: Cambridge University Press, 1907), 133.{]}}
While you, in your private capacity, have not been deluded by~``the
theoretical omnipotence of the technical law-maker,''~the great majority
have been fooled by it, and, in your public function as judge, you are
compelled to work by it.~

Of course~we are able to account for the prolonged hold of the
old~centre~on the mind when we reflect over the awful importance of the
political unity, or the social bond. No~wonder the idea took ground that
the King could do no wrong; he was the fountain of justice, or, to use
Maitland's words, the~centre~of the~centre. The safety which men felt in
the idea of the single~centre~very naturally bred in them a fear of new
controlling unities.~

I~should note that there is nothing new in the perception that powers
have been coming into place outside, and independently of, the
old~centre~of control, but, until now, they have always been treated as
illegitimate. One of the most striking things on this score that I have
run across is the following from the \emph{London Spectator} of April
11, 1857:~

\begin{quote}
The present~state of affairs~on the Continent suggests the existence of
some influence which is not generally recognized, though its power must
be overruling and its operation universal. . . .~We~perceive . . . that
it may become~dangerous alike to the material condition, the political
independence, and the domestic order of states. Nor are we speaking of
any imaginary or mere~``moral''~influence; we speak of a powerful
combination more than political, more personal than a congress of
diplomats or~princes. . . . The money merchant obtains his profit
entirely from the simple act of exchange. . . . He is not a
safe~councillor~for those who have in charge~the permanent interests of
states. . . . The power of that order . . . proceeds in its action
independently of ordinary political movements, and shows itself pursuing
its course uninterrupted,~undiverted, whatever may . . . be the action
of ordinary statesmen. . . . It is a new order, a~new administration in
the world. . . . The class is alien to any~particular country, and yet
is deeply rooted in the administration of each country. . . .~This grand
council of~millionaires~has proved that it is~superior to the political
administration of the separate countries. . . . It is republican, but of
the aristocratic republic,~more close~than the Grand Council of Venice,
infinitely more arbitrary.~Like~that commercial~republic, kings bow down
to it; but the kings that now bend are the giant~emperors of our day,
not the brawling leaders of the~middle ages. The debates of this council
are not reported; its constitution is~as yet~unascertained and
undetermined. We feel its power before we can define it. It is
independent of political councils, higher than political
responsibilities, ignorant of constitutional checks. . . .~And it
extorts from~us the question whether any account has yet been taken of
the immense institution that has sprung up while emperors and common
politicians were thinking to settle the world with armies and treaties.
\end{quote}
\enlargethispage{\baselineskip}
Over half a century has passed since the above publication, but science
has remained dumb to the demand of the Spectator for an accounting.
Instead of a clarifying report the mystery has deepened. Sir
Oliver~Lodge, the noted English scientist, tried his hand at it in the
\emph{Economic Review} for October 1898, the title of his article
being~``The Functions of~Money,'' but the best he could do was to bring
up with this dogmatic statement:~``There is a fifth estate of the realm
now, more powerful than any of the other four, and the nation bows down
before it. The supreme power is the power of the purse. The latest of
the functions of money is to rule the modern world.''~

\newpage Just here I want to bring under your eye two other things from the
books. The first is from~\emph{Money and Civilization}~by Alexander
Del~Mar, published at London in 1886, and thus reads:~

\begin{quote}
Nor does the co-relation (i.e. between Money and Civilization) end here;
it also relates to the~\emph{forms of money and society}~(italics~Del
Mar's); in other words, it is structural as well as functional. With the
development of society from the rudimentary to the highly organized
condition, from the simple to the complex, from the homogeneous to the
heterogeneous form, so has money developed from slaves and cattle to
corn; from corn~to metals; from metals, which are not susceptible of
limitation, to coins, which are; and from a limited number of coins to a
limited number of symbols of any material. These co-relations~between
money and civilization hold good as well in periods of decay as of
growth: so that when we shall have acquired a sufficient degree of
intimacy with the subject, it will become as possible for us to reason
out from its monetary system the entire structure of any given society,
or State, as it was for Cuvier to trace from a single bone the form and
functions of an unknown animal.~
\end{quote}

The second is from the French economist, Proudhon, the date being 1840:~

\begin{quote}
But at this point a gigantic and complicated conspiracy is hatched
against the capitalists. The weapon of the exploiters is~met by the
exploited with the instrument of commerce; a marvelous invention,
denounced at its origin by the moralists who favored property, but
inspired without doubt by the genius of labor, by the Minerva of
the~proletaires. The principal cause of the evil lay in the accumulation
and immobility of capital of all sorts---an immobility which prevented
labor, enslaved and~subalternized~by haughty idleness, from ever
acquiring it. The necessity was felt of dividing and mobilizing wealth,
of making it pass from the hands of the possessor into those of the
worker. Labor invented~\emph{money}. Afterwards, this invention was
revived and developed by the bill of exchange and the Bank. The first
man who conceived the idea of representing a value by a shell, a
precious stone, or a certain weight of metal, was the real inventor of
the Bank. . . . By~this means, oppressed equality was enabled to laugh
at the efforts of the proprietors, and the balance of~justice was
operative for the first time in the tradesman's shop. The trap was
cunningly set, and accomplished its purpose so thoroughly that in idle
hands money became only dissolving wealth,~a false symbol, a shadow of
riches. An excellent economist and profound philosopher~was~that miser
who took as his motto,~`When a guinea is exchanged, it evaporates.'~This
explains the constant fact of history, that the noble---the unproductive
proprietors of the soil---have everywhere been dispossessed by
industrial and commercial plebeians. . . . The greatest enemy~of the
landed and industrial aristocracy to-day, the incessant promoter of
equality of fortunes, is the banker. . . . The banker is at once
the~most potent creator of wealth, and the main distributor of the
products of art and nature. And yet, by the strangest antinomy, this
same banker is the most relentless collector of profits, increase, and
usury ever inspired by the demon of property. The importance of the
services which he renders leads us to endure, though not without
complaint, the taxes which he imposes. Nevertheless, since nothing can
avoid its Providential mission, since nothing which exists can escape
the end for which it exists, the banker (the modern Croesus) must~some
day~become the restorer of equality. Whence it follows that the Bank,
to-day the suction pump of wealth, is destined to become the steward
of~the human race.\footnote{{[}Pierre-Joseph Proudhon, \emph{What is
  Property}? trans. Benjamin R. Tucker (Princeton: Benjamin R. Tucker,
  1876) 312--13.{]}}
\end{quote}

I have been led to the above references partly as a help in raising the
central question between us up to the level of its true worth and
dignity. I am not trying to adjust a mere personal difference of
opinion, and I am sure that it means far more than this to you else I
could not have won your attention. Instead, I am seeking to present to
you at once the course of social evolution, the nature of the existing
crisis in the American State, and the one road to its peaceful solution.
As I look at the situation it is charged with greater peril than was the
crisis of 1789, at the adoption of the Constitution, or the conflict of
1861.~

1 have now to remark further on the absence, so far as relates to our
public teachers, of any recognition, even to the bare possibility, of
the appearance of new organs of social control in the legitimate sense,
along with a new instrument of coercion. It is true that various writers
are anticipating a great development of the Common Law, but the
hall-mark of the~law they have in mind is always physical force---the
fine, the jail, and the gallows. For example, here is Bruce Wyman, a
Harvard professor, with a book on~``Control of the Market.'' He is
looking for~``a legal solution of the Trust problem''---it must
be~``legal''~or nothing---and this without the least suspicion, so far
as he reveals himself, that the new combinations may turn out to be, in
and of themselves, simply a further outworking of the intrinsic law of
the social organism. Wyman is serious of course, but just the same his
book is a humorous production. His industry in working over the mass of
case law is so exacting that he has no time to look out of doors for the
true facts. His naive belief in the Police Power as a remedial agent
reminds one of the ill-starred~attempt~of the English courts towards the
end of the seventeenth century to bring the control of the news market
under the Common Law. The learned Chief Justice,~Scroggs, held that~``to
print or publish any newsbooks or pamphlets of news whatsoever is
illegal; that it is a manifest intent to the breach of the peace, and
that they may be proceeded against by law far an illegal thing.~Though
the thing is not scandalous, yet it is illicit and without authority''
(R. v.~Carr, 7 St.~Tr., 1127). The Common Law was thought to be
sufficiently tough and elastic to put down the news mongers, but it
turned out that the inflexible thing was the news movement itself, and
this, if I may so write, in accordance with the law of the social body
and~in spite of~``the law of the land''~as interpreted by the courts. It
is worth noting that it was about this time, 1680, when some hungry
newspaper man at London first conceived that he could get a
living---food, clothing, shelter---by selling news. It appears, from the
history of opinion, that a given social order is always at the point
of~its highest pretension in the very moment of its decline.~~

Sir Frederick Pollock, in his Harvard lectures on \textquotesingle``The
Expansion of the Common Law,'' takes, as is natural to him, a broad view
of the matter. He says:~

\begin{quote}
Just now (1903) there is a group of questions before courts of common
law both in America and in England, arising out of the rapid modern
development of trade combinations, which go to the very foundation of
the law of personal liberty and of civil wrongs. . . . The problem is
nothing~less than to reconcile the just freedom of new kinds of
collective action with the ancient and just independence of the
individual citizen. This much is certain, that no merely technical
resources of the law will suffice for the task. In whatever jurisdiction
the decisive word is spoken, it will be founded on knowledge of the
world, and on broad considerations of policy. Natural law will have . .
. a large and probably a dominant~part in it.
\end{quote}

In Pollock's view,~``the Common Law has largely enriched and is still
enriching itself by associating the Law of Nature with its
authority,''~but I doubt if Pollock, notwithstanding his open mind, is
prepared,~as yet, to recognize and welcome the revolutionary conception
of Social Law which the Trust development implies and compels, reaching
down to and transforming the customary ideas both of sovereignty and
jurisdiction.~

I once wrote you that it is not easy to see once clearly what the
orthodox jurisprudence of the world has forbidden us to see at all. A
measure of this difficulty is afforded in Brooks Adams's book,
\emph{Centralization and the Law}. After sketching the rise of the
King's courts in England under Henry II and the further development of
the Common Law in obedience to the wants of the vigorous mercantile
class, which represented the new economic power, Mr. Adams points out
that~

\begin{quote}
the law, if we view it right, presents a series of phenomena, evolved by
the conflict of social forces; and if we would understand those
phenomena, we must begin by understanding the society which caused them.
. . . The law is the~envelop with which any~society surrounds itself for
its own protection. The rules of the law are established by the
self-interest of the dominant class, so far as it can impose its will
upon those who are weaker. These rules form a corpus which is~more or
less flexible~according to circumstances, and which yields more or less
readily to pressure. When the society, which is the content of the
envelop which we call the law, expands or contracts regularly and
slowly, the envelop, yielding gradually, tends to conform without
serious shock; conversely, when society breaks suddenly with its past
because of the instantaneous injection of some new energy which disturbs
the habits of life, the law may not automatically adapt itself to the
change, but may be rent by what we call a political revolution. In the
nineteenth century our society broke with its past by the introduction
of steam.~
\end{quote}

\noindent Coming down to the present time, he asserts that~``within seventy-five
years social conditions have changed more profoundly than they had done
before since civilization emerged from barbarism, and, apparently, we
are only at the beginning. . . . In fine,~modern life is evolving
conceptions not only different from, but often antagonistic to the
old.''~And he continues:~``Whither we are drifting we know not, but this
much seems to me clear. In a society moving with unprecedented rapidity
unintelligent conservatism is dangerous. No explosion is more terrible
than that which shatters an unyielding law.~As yet~our legal system is
unyielding. . . . The character of competition has~changed, and the law
must change to meet it, or collapse.''\footnote{{[}Brooks
  Adams,~\emph{Centralization and the Law~}(Boston: Little, Brown, and
  Company 1909), 45-50.{]}}~

And so, Adams brings up with a negative only. He appears able to see in
law only the operation of arbitrary power, and he vainly seeks to learn
where the new absolute will~is~to be lodged. He detects no new
expression of the general interest. While he~is able to~recognize a
fundamental alteration in conditions, he lacks a universal concept by
which to order his facts and thus lead up to the clear and radical
advance in the very idea of Social Law which is~implicit in all his
writings on this subject.~He uses the tool of inquiry which he himself
rejects---the notion that, of necessity, absolute power is lodged in one
part of the body politic. He cannot compass the organic view of society;
to him, it is only a mechanism. His work is valuable to me, but only
through re-action from it.~In spite of~his startling avowals, I find no
trace in Adams of any perception that the body of law which grounds
itself in the might of the~billy~and the bayonet~can not~by any
possibility expand and become more common; that instead the development
of law which the new era demands must find vent through new organs,
involving, as I shall undertake to show you, a new instrument of
coercion. Adams does not know it, but it is true that the instrument of
mere physical force has reached its limit as a mode of social control.~

This view is confirmed by the striking article of Mr. Adams in the
\emph{Atlantic Monthly} for July, 1910, entitled, ``A Problem in
Civilization.'' The problem, as he puts it, is how to control the social
energy which modern science and modern inventions have liberated, and he
starts out by defining civilization~``as tantamount to centralization.''
~``An organized social system can exist on no other foundation than
monopoly.''~``The impulsion given society by the forces set in action by
applied science has gathered volume, until now it sweeps before it our
laws and institutions.''~A lesson is drawn from the unifying of the
American colonies by means of the~Revolution, culminating in the
Constitution of 1789. Emphasis is laid on the type of mind which at that
time caused our society to cohere, and the demand is made for another
Washington who will organize or fuse the present society, whose
condition, we are told, is infinitely complex, going to the root of the
social system. A tribunal is demanded which will~be open to the
individual, and which will have jurisdiction over the prices charged by
monopolists.~``The alternatives are, to bring monopoly under
the~jurisdiction of the courts, or else for the monopolist to enroll~an
armed police~which will enforce his will.''~The need is the entrance of
an administrative mind whose genius and power will tally with the
present stage of scientific method, and through such a mind to
develop~``a central intelligence''~which will satisfactorily regulate
prices, but to Adams the only visible instrument therefor is the
inherited court of law. Yet this is not all, as the necessity has arisen
for a social authority equivalent to that exercised by the father in the
family under the old civilization, which, as our writer says, is dead.~

The entire article betrays again that Mr. Adams is wholly unable to
detect that a new social order is breaking through the shell of the old.
Notwithstanding his important discoveries, he continues to hold to the
State-idea of the sixteenth century---the body with a single
governing~centre.~Thus~he gravely tells us:~``Justice must be a
monopoly. There can be no competition in justice. That the State, and
not the citizen, shall punish wrong, is the first principle of
civilization.''~Again, the appeal is to arbitrary power; the natural
interactions of the organism are away from him. He is unable to ask the
question.~What would be~the effect on the distribution of even-handed
justice were the~News System and its cognate organ, the Credit System,
to undergo absolute centralization, which of course could only
be~effected~under the impartiality of scientific method. It so happens
that, under the Law of Society,~untruth cannot he organized to any great
extent. Mr. Adams recites for us the achievements of exact method in the
world of business. He tells us that~``speaking generally, in the United
States, whatever concerns are based upon science are well administered,
those based on law are ill administered.''~His colleague in the Boston
University Law School, Dean Bigelow, goes so far as to say that the
directing~centre~of~``the new order,'' meaning by this of course the
organ of the general interest, will have to adopt~``nothing short of the
scientific~precision .~. . of the Standard Oil Company.''~``The
winning~order .~. . must be the most perfect embodiment of skill and
power.''~It is plain, however, that neither Adams nor Bigelow has
sufficient faith in science to lead up to the suggestion that the
science of society is in store for us. Nor is either of them aware that
social science is but another name for the science of business. They
have not yet attained to the outlook of W. K. Clifford:~``It is idle to
set bounds to the purifying and organizing work of Science. Without
mercy and without resentment she plows up weed and briar; from her
footsteps behind her grow up corn and healing flowers; and no corner is
far enough to escape her furrow.''\footnote{{[}William K. Clifford,
  ``Body and Mind,''~\emph{Fortnightly Review~}16 (July--December 1874):
  736.{]}}

I have been led to dwell upon the~joint work of Adams and Bigelow
because, for one thing, it represents perhaps the most serious attempt
on the part of our legal teachers to explain the present juncture of the
social forces and to indicate how or in what direction they are
moving.~I~cannot see that they throw any light on the situation except
of course to heighten our sense of its gravity. I think their efforts go
to show that the training of the lawyer works against clear seeing in
this field, though it is the one field in which he is supposed to have
expert knowledge and power. As you may know, Bigelow published a new
book last year under the title,~\emph{A False Equation}, in which he
discloses a slavish adherence to the received idea of State
organization: the absolute State, facing the absolute individual, to use
the words of Maitland. Bigelow deplores~``the weakness of the State''~in
dealing with the monopoly question, and yet his demand is for
more~``legislation.'' To Adams, the~``community~lives very largely in
defiance or in disregard of the law,''\footnote{{[}Adams,
  \emph{Centralization}, 47.{]}}~but he does not perceive~that new
legalities or new ideas of law~are pressing for recognition. The latter
has delved into history with good results which aided me somewhat in
tracing the course of the money or credit system. He has written of the
relative changes in position of gold and silver as instruments of
exchange, though unable to recognize that the paper instrument is
displacing both metals. He is at no time able to grasp the evolution of
commerce, or what is the same thing of society, as a whole, and~so to
mark out definitely the successive rise and decay of new~centres~or
modes of social control which commerce raises up in obedience to its own
necessities,~i.e. for its own regulation. He appears not to know that
the~ever advancing~movement of commerce is but another name for the
natural expansion of human wants and activities; that it is a
self-moving organism ever incorporating in its majestic sweep new
energies as they are liberated through the discoveries and inventions of
Science. I have never met Adams but last Spring, when in Boston, I had a
talk with the editor of the \emph{Atlantic Monthly}, Ellery Sedgwick,
who said to me that Adams is constantly laying stress on the need of
greater social co-operation,~but,~I take it, he is not sufficiently
aware that the over-arching co-operation among men is through the
profit-seeking principle of the despised dollar-hunters. Such writers
have yet to learn that the feudalism of the Landlord, the
State-centre~of the sixteenth century and of Roosevelt, and the organism
of countless governing~centres~now coming to recognition are but varying
expressions of the Spirit of Commerce.~~

By the way, I want to put down that I was helped distinctly,~in the
course of~my struggle to grasp commerce as a moving whole of action, by
the German economist,~Schmoller, especially by his short account of the
Mercantile System and its Historical Significance. He led me to see more
clearly that the true history of commerce is the development of
its~centres~of control, that on anything short of this resultant we have
only a jumble of unorganized facts, and not a~\emph{story}, and that the
constant progress of civilization has been towards greater and greater
economic unities, each taking the place of a passing individualist
system.~Schmoller~sketches the transition from local trading customs to
the territorial and thence to the national economy. In his view
mercantilism was~``nothing but state making---not state making in a
narrow sense, but state making and national economy making at the same
time; state making in the modern sense, which creates out of the
political community an economic community, and so gives it a heightened
meaning. The essence of the system lies not in some doctrine of money,
or of the balance of trade.~. . but of something far greater, namely, in
the total transformation of society and its organization, as well as of
the State and its institutions, in the replacing of a local and
territorial economic policy by that of the national
State.''\footnote{{[}Gustav Schmoller, \emph{The Mercantile System and
  Its Historical Significance} (New York: Macmillan, 1896), 50--51.{]}}~It
is thus that he breaks through the shell and husk of history to its
economic core. He makes this suggestive statement:~``The great and
brilliant achievements of history, both political and economic, are wont
to be accomplished at times when economic organization has rested on the
same foundations as political power and order.''\footnote{{[}Schmoller,
  \emph{The Mercantile System}, 3.{]}}~Again, he says:~``The idea that
economic life has ever been a process mainly dependent on individualist
action---an idea based on the impression that it is concerned merely
with methods of satisfying individual needs---is mistaken with regard to
all stages of human civilization, and in some~respects~it is
more~mistaken the farther we go back.''\footnote{{[}Schmoller, \emph{The
  Mercantile System}, 3--4.{]}}

Concerning the growing place of science in commerce and the attention it
is beginning to receive from representative men, I have been struck by
an address of the English~philosopher and politician, A. J. Balfour, in
1908, entitled, ``Decadence.'' After considering from various viewpoints
the causes underlying the making and the unmaking of nations, he says:~

\begin{quote}
A social force has come into being, new in magnitude if not in kind,
which must favorably modify {[}such hindrances to progress as he had
mentioned{]}. This force is the modern alliance between pure science and
industry. That on this we must mainly rely for the improvement of the
material conditions under which societies~live . . . is obvious,
although no one would conjecture it from a historical survey of
political controversy. Its direct moral effects are less
obvious;~indeed~there are many who would altogether deny their
existence. To regard it as a force fitted to rouse and sustain the
energies of nations would seem to them absurd: for this would be to rank
it with those other forces which have most deeply stirred the emotions
of great communities,~have urged them to the greatest exertions,~have
released them most effectually from the benumbing fetters of merely
personal preoccupations---with religion, patriotism, and politics . . .~
\end{quote}

``I believe,''~adds Mr. Balfour,~

\begin{quote}
this view to be wholly misleading, confounding accident with essence,
transient accompaniments with inseparable characteristics. . . . All
great social forces are not merely capable of perversion, they are
constantly perverted. . . . In talking of the alliance between industry
and science my emphasis is at least as much on the word science as on
the word industry. . . . It is on the effects which I believe are
following, and are going in yet larger measure to follow, from the
intimate relation between scientific discovery and industrial
efficiency, that I most desire to insist. . . . I do not myself believe
that~this age is either less spiritual or more sordid than its
predecessors.~I believe, indeed, precisely the reverse. . . .~If in the
last hundred~years the whole material setting of civilized life has
altered, we owe it neither to politicians nor to political institutions.
We owe it to the combined efforts of those who have advanced science and
those who have applied it. If our outlook upon the Universe has suffered
modifications in detail so great and so~numersous~{[}\emph{sic}{]} that
they amount collectively to a revolution, it is to men of science we owe
it, not to theologians or philosophers. On these indeed new and weighty
responsibilities are being cast. They~have to~harmonize and to
co-ordinate, to prevent the new from being one-sided, to preserve the
valuable essence of what is old. But science is the great instrument of
social change, all the greater because its object is not change but
knowledge; and its silent appropriation of this dominant function, amid
the din of political and religious strife, is the most vital of all the
revolutions which have marked the development of modern civilization. .
. .~This process brings vast~sections of every industrial community into
admiring relation with the highest intellectual achievement, and the
most disinterested search for truth; that those who live by ministering
to the common wants of average humanity lean for support on those who
search among the deepest mysteries of Nature; that their dependence is
rewarded by growing success; that success gives in its turn an incentive
to individual effort in no wise to be measured by personal expectation
of gain; that the energies thus aroused may affect the whole character
of the community, spreading the beneficent contagion of hope and high
endeavor through channels scarcely known, to workers in fields the most
remote; if all this be borne in mind it may perhaps seem not unworthy of
the place I have assigned to it.~
\end{quote}

I have reproduced Mr. Balfour's address in considerable detail as it
comes from a representative English observer. It is the most striking
thing of the sort that I have seen from a high political quarter. He
states his points with some hesitation, more in fact than is shown
by~the parts I have copied, feeling evidently that his audience might
not be in full sympathy. The outgiving prompts me~to say that the
wonderful achievements he recites compel an immediate further advance of
like order and this to an extent that is not easy to grasp or
appreciate. The truth is that the resulting social complexity cannot be
controlled or directed by any means short of the power which created it;
that is, the power of science itself. I mean that at a certain stage in
the industrial development the enlarged governmental need compels a
resort to scientific methods, and this in an exclusive sense. A new
order of government is required and it must rise from the genius of the
locomotive and the electric wire. An organic climax in society has been
reached, so much so indeed that the parallel culmination on the side of
government must be an extension of the rule of science.~

The outcome to be dealt with is the modem industrial system which
presents a new problem in social control which cannot be solved by any
possible extension of the rule of physical force,~i.e. of the police
power, and the movement~must of necessity~be international, as the
Industrial State is the World-State. The degree of scientific control
which is required cannot operate through the old State-centre, since
compulsory co-operation and voluntary co-operation are not workable
through one organ.~But the advance can only obtain, or become~regnant,
through a successful struggle for control on the part of the men of
science. Their line of procedure is through the news organization of the
world, and it must be~in harmony with, or by virtue of, the
profit-seeking principle. It is a noteworthy though~little known~fact
that progress in true or~rational political control is directly as to
the extension of venality in society, which means that the scientific
ordering of the news movement must he grounded in business methods in
order that the action may become organic in the highest degree.~

\newpage I fancy that this principle---the self-seeking method of the
counting-room---would surprise and possibly perplex Mr. Balfour as the
idea flies in the face of our inherited prejudices, but the fact remains
that the ultimate government of the industrial system must itself be a
business undertaking. The method must be on all fours with the principle
of Contract or voluntary agreement on which modern commerce has been
built up. Happily, the conditions are everywhere favorable as the
extension and control of the international credit system---the key to
the advance---necessitates the systematization of the news movement,
which runs parallel at every point. The idea involves the organization
of human experience on a world-wide scale, and this through an
institution that will be prepared to sell, at a price which everybody
according to his needs can afford to pay, the experience of all man in
relation to any given problem or combination of circumstances. The
masters of physical science of late years have been plotting ways and
means, to use their own words, for distributing the results of science,
each fact to be rendered in its life bearing, but they have yet to learn
that the job must be let out to the men of business; that is, to the
money catchers.~

In 1884 John Eaton,~one time~United States Commissioner of Education,
addressed the American Association for the Advancement of Science on
this question of distribution. Among other things, he said:~

\begin{quote}
The dissemination of truth is as scientific as its discovery. . . .
Toward the gathering up, for man\textquotesingle s daily use, of all the
lessons of nature, the progress of the race is tending. . . . The era of
this diffusion of knowledge has already~commenced. Men not engaged in
scientific pursuits are gradually coming to feel the necessity of
gathering, grouping, and generalizing the data which give them a clear
measure of health, comfort,~pleasure, as well as the profit and loss
involved. . . . But the correlation of all these and~their actual
results have not yet been reached.''~
\end{quote}

\noindent At a meeting of the British Association some years ago a demand was made
for a central institution which should adequately check the results of
scientific inquiry. This growing need has further illustration in a
remark by Prof. Louis Agassiz:~``Scientific truth must cease to be the
property of~the few; it must be woven with the common life of the
world.''\footnote{{[}This paragraph, including the quotes from Eaton and
  Agassiz, is copied from Ford's \emph{Draft of Action}.{]}}~

A later and more insistent demand of the same order appears in Dr. Henry
Maudsley\textquotesingle s book, \emph{Life in Mind and Conduct},
published in 1902. Dr. Maudsley says:

\begin{quote}
There is apparent need now of a superior scientific or philosophic
society, a select council of wise men conversant with all the sciences
yet engulfed in none, an organ of scientific synthesis, to understand,
interpret, co-ordinate and blend their different knowledges---in fact,
to make them wisdom.~Until that be done, although knowledge grow, wisdom
will linger. . . .~Is there no remedy, then? There is none yet~visible.
The strange irony of the situation in England is that the highest
scientific Society is entirely occupied with the prosecution of minute
researches, doing nothing whatever to co-ordinate results, yet calls by
the name of~``Philosophical Transactions'' the huge volumes in which it
accumulates the scattered gleanings of laborers who, if they were all
congregated in one room, would hardly understand a word of each other's
language.
\end{quote}

But none of these authorities have perceived, so far as I am aware, that
the results of science to date could not be laid hold of and given
universal distribution until the science of society should be formulated
and let in at the base, functioning as the scheme of classification. Nor
could this happen until directly compelled by the rising economic need;
that is, as I have already indicated, the desired advance could not take
place until the system of industry should reach such a degree of
complexity as to necessitate the substitution of scientific direction
for the inherited haphazard methods of social control, Society does not
change its habits upon the strength of intuitions; it decides only upon
the authority of facts. It all means, in short, that the demand of Dr.
Maudsley could not be met until the constitution of society should
become, in fact, organic in the full sense of the term. The demand is
for the highest human tribunal yet conceived, far outrunning the
function and scope of Parliament, and this though the latter be taken at
its greatest pretension to sovereignty.~

The suggestion points to the control by~Science~of our whole
existence. There is as yet~no general recognition of the function of
science or impartial inquiry in the field of government, that is, so far
as this aspect of social life is dominated by the old State-centre.~It
is true of course that on the side of Contract or voluntary agreement
social activities have come under scientific direction to an immense
extent, but the idea has little or no place as a political conception.
Indeed, it is fair to say that the spontaneous social co-operation, on
which are dependent our regular supplies of food,~raiment~and shelter,
would break down but for the presence and control of exact inquiry. Step
by step the various divisions of commerce have been passing under the
direction of science but the fact has no recognition in the general
consciousness. It is true that now and then the old
governing~centres~call to their aid the training and skill of scientific
men, but this only on condition that the function or usefulness of the
old organs shall not be~looked into; what Maitland called~``the
impeccability of the State''~is still the ruling idea. In the field of
so-called political action there must be no thought of scrapping old
machinery. Yet if we were to seek the most general force which has
been~active in historical times, and is still active, the answer would
be:~\emph{The conquest of all intellectual fields by science}.~And this
factor is increasing with each social advance, in fact it is the
impelling cause of such advances. The chief obstacle standing in the way
of the universal acceptance of science as the guide of life is the
continued reverence for, or the unquestioned obedience to, the edicts of
the old State-centre; that is, we regard political and moral laws with a
feeling different from that with which we look upon physical or chemical
laws. We are accustomed to regard our law-makers, jurists, and political
organizers as leaders of men to the neglect of those who make our
scientific discoveries and inventions. But, happily, the tide of opinion
is turning and I think we are in the way of recognizing the irresistible
power of science as a matter of daily thought and experience. I have
been drawing together the evidence as to the extent to which scientific
method is even now directing social action, preparatory to making a
commanding summary of the wonderful story, and I am hoping that the
report will go far toward correcting the inherited view.~The need is to
change the objects of social adoration.~

The difference between the old and the new civilization is expressed in
the one word:~\emph{science}. If we go to the bottom of the~matter~we
find that our entire social order rests on the fact that we can and do
look into the future. In~fact~the height of any civilization turns upon
the thoroughness with which its prophets understand their calling, and
are able to predict the future. The ancient oracles predicted the
outcome of a war from the viscera of a sacrificial animal;
Moltke\footnote{{[}Helmuth von Moltke (1800--1891) was a Prussian
  military commander who authored several books about strategy and
  military history, including \emph{The Franco-German War of 1870--71},
  trans. Archibald Forbes \emph{} (London: James R. Osgood, McIlvaine \&
  Co., 1893).{]}} made his prophecy on the ground of his scientific
inquiry into the relative military conditions of France and Germany.~

I take it that the notion of the scientific treatment of history and
society; that there is a law in the succession of social states, to be
ascertained by examining into the collective phenomena of the past, took
its first clear start in the mind of Condorcet. Certain stones of the
edifice were laid by Montesquieu, but the larger and rounded expression
waited upon Condorcet. In his \emph{Progress of the Human Mind}
Condorcet in 1794 gave glad acclaim to the fact that social inquiry had
been escaping from governmental tyranny, and that knowledge had
become~``the object of an active and universal commerce,'' but an entire
century had to pass before intelligence could be organized~on the basis
of~science, as such action necessitated an advance of the social
mechanism to the stage of un-hindered or absolute communication: to the
telephone age. Pray bear in mind that only now is the telephone in the
way of becoming a social function, thirty-six years having been required
for it to pass into general use. It is worth noting that the advent of
the telephone has much the same relation to the appearance of social
science as the incoming of the telescope bore to the development of
astronomy. Galileo ranks as the true founder of descriptive astronomy,
but the telescope was his necessary instrument. As the telescope
operated to free the mind, so does the telephone in the field of social
observation. There has now occurred a mental liberation on an infinite
scale.~

I have been interested in noting the quickening effect of the incoming
of the telegraph in 1844 on the master minds of that time. The most
striking thing I have seen was the prediction of John C.
Calhoun.~Speaking from his place in the Senate on the Oregon question in
1844, he counseled against war with England on the ground of the
entrance of new forces in society; that their benign influence would be
retarded. He said:~

\begin{quote}
The two great agents of the physical world have become subject to the
will of man, and have been made subservient to his wants~and enjoyments;
I allude to steam and electricity. . . . The former has overcome
distance,~both on land and water, to an extent which former generations
had not the least conception was possible. . . .~Within the same period,
electricity,~the greatest and most diffuse of all known physical agents,
has been made the instrument for the transmission of thought, I will not
say with the rapidity of lightning, but by lightning itself. Magic wires
are stretching themselves in all directions over the earth, and when
their mystic meshes shall have been united and perfected our globe
itself will become endowed with sensitiveness, so that whatever touches
on any one point will be instantly felt on every other.~
\end{quote}

\noindent Calhoun looked to the~``dawn of a new civilization, more refined, more
elevated, more intellectual, more moral, than the present and all
preceding it.''

The flight of years has brought us to the full realization of
Calhoun\textquotesingle s prophecy, but only so far as relates to the
change in underlying conditions. We are now face to face with the
over-arching problem of social re-organization compelled by the
fundamental alteration which he foresaw. Our task is to interpret modern
communication in its influence on the organization of the State, to note
the re-action of mechanical and scientific progress on human life and
thought, involving the investigation of facts as well as the analysis of
ideas. Our ways of thinking have been changed radically in many
directions of which we have various accounts, but above and beyond all
the interesting and engaging thing is to sum up the effect of the new
environment on our forms of political thought. We are now able to get,
as though in a single night, the political outcome of the progress of
the last century in the field of physical invention. An alteration has
occurred in relations of fact, and the need is to~interpret it in the
general terms of principle or law. We are under the necessity of
bringing ultimate principles into living touch with common experience
and the~actual facts~of life. We are standing at the~centre, and are
therefore able to see the moving spectacle of action from the universal
point of view.~We are at the heart of commerce which appears as the
transforming agent, as the great civilizer of men and nations. We~have
to~record a development in government which is as new as was the
institution of parliament when it first came to recognition.~

Definite or full account~has to~be taken of the larger freedom for the
individual, under the limitations of fact or law. The change is so
profound that we are forced to provide for new circumstances with only
old experience to go by; the resort must be to scientific thought. Just
here the luminous distinction of W. K.~Clifford is in point:~``The
difference between scientific and merely technical thought is this: Both
of them make use of experience to direct human action; but while
technical thought or skill enables a man to deal with the same
circumstances that he has met with before, scientific thought enables
him to~deal with different circumstances that he has never met with
before.''\footnote{{[}William K. Clifford, ``Aims and Instruments of
  Scientific Thought,'' \emph{Popular Science Monthly} (November 1872):
  95.{]}} I will only add to Clifford's analysis that the instrument is
an observed uniformity in the course of events. By its use we are
enabled to extract information transcending our experience; it leads us
to infer things we have not seen from things we have seen.~

The American State is re-organizing right under our eyes, and it
behooves those in authority to take account of the fact. Prompted by the
new freedom, all classes of society are demanding entry into the
economic councils of the Industrial State; they want to be represented.
Political freedom is a matter of fact; progress therein turns upon the
degree of access and movement, and not upon the state of opinion merely.
An advanced sense of justice~is taking form in peoples and races. The
judge on the bench is now compelled to assimilate new matter, to take up
fresh material, as the public is properly struggling toward wider legal
rights. Society is on all sides, and new legal doctrines are evolving to
be applied in new tribunals. There is hardly any limit to the
application of the organic idea; the bars are all down. In 1821, and
later, in New York State all orders of men, stirred by the change in
conditions, demanded and received the social negative that is contained
in the right to suffrage, but the insistence now is for a far more
intimate relationship; each class is seeking its proportionate voice in
the direction of the system of industry; they all want to wear, each
according to his function, the garments of the State.~That is to say,
while~during the first third of the last century the struggle was for
recognition at the doors of the old legislatures, the present demand is
for representation through the~centres~or organs of the Industrial
State. The trade union movement, for example, is but a further assertion
of the representative principle. And, to crown all, a new political
class---the men of scientific training and habit---is moving toward the
direction of the new universal governing organs, the News System and its
cognate organ, the Credit System.~

I have sought to put down for you something of the anticipations of men
touching the convergence of literature upon the life actual. I am doing
this as means to taking you in some measure over the course of my own
education, I~am well aware that~one cannot successfully convey to
another a new validity in any field of science through a few epigrams or
other short-hand process. It is~a number of~years now since I began to
realize that victory would not be mine unless I should be able to unfold
the science of society in a simple yet comprehensive literature which
could be taught in the schools and to the whole people.~

Springing out of my experience as a working journalist, there was
revealed to me, as by a flash of light, the existence in fact of the
social body. I meet numerous people who hold it~\emph{ideally}~or who
think they do,~i.e. in a language of metaphor, but I have brought myself
to hold it~really as~a working concept or tool of inquiry with which to
measure the social relation, I believe that I am doing this now as
certainly as the astronomer holds to~\emph{his}~entity or objective.~

I have had to reach detailed results in face of the fact that for long I
could find no one who so much as believed in the bare possibility of a
science of society as a matter of the immediate future. Such hospitality
as I got usually came from the masters of physical science, who know
what science is, from engineers, and from men of projecting minds
generally,~i.e. from men of large affairs, notion here is that men who
do things in a big way make their calculations of necessity~on the basis
of~an organism, else their prophecies would not work out in practice.~It
is not surprising, therefore, that men of this order understood me in
good part when I talked to them an organic language, I have been
attracted by a saying of the engineers, who speak of the controlling
facts---the necessity of mastering them before they can proceed safely.
When one considers that to set forth a scientific or valid account of
the social~objective is to make as abrupt a departure in social
observation as Copernicus effected in astronomy, it is no wonder that I
have had a hard job in making myself understood. The most difficult
thing in the use of language is of course to fit words to
a~\emph{new}~object. But for the further and rapid integration~in the
new environment~during the course of~my inquiry or observation the task
would have been hopeless.~My feeling is that I am even with the facts,
and that the showing of evidence is ample for the demonstration; the
notion of an organism is~in the air.~

By the way, I am not inclined to favor the retention of the
word~\emph{sociology}, at least I do not intend to adopt it. It is worth
remarking that the men as a class who are
merely~\emph{professing}~sociology, and the like, in our universities
are not in position to welcome the science of society, as its appearance
publicly would go to put them out of business. Here is the professor of
sociology in Columbia University (Giddings)\footnote{{[}Franklin Henry
  Giddings (1855--1931) was a professor of sociology at Columbia
  University from 1894 until 1931. He is considered as one of the ``four
  founders'' of American sociology.{]}} who has, with untiring industry,
published some hundreds of pages to prove that birds of a feather flock
together. He calls the legend Consciousness of Kind, as the homely
phrasing would not have~effected~the desired imposition.~A professor at
the University of Vienna (Böhm-Bawerk)\footnote{{[}Eugen von Böhm-Bawerk
  (1851--1914) was an Austrian economist. He held positions at the
  University of Vienna and at the University of Innsbruck. At the time
  of Ford's writing, Böhm-Bawerk had left the university to lead a
  political career. He was Austria's Minister of Finance between 1895
  and 1904, and also became the president of the Academy of Sciences in
  1911. Ford probably refers to his \emph{Capital and Interest}.{]}} has
printed something like one thousand pages to prove that a bird in hand
is worth two in the bush, which sums up his theory of economic interest.
Had he used this simple statement it would have destroyed his book. What
is more, the professional economists in Europe and here have written
thousands of pages discussing~Böhm-Bawerk's~thesis. In the days to come
interest will be plainly defined as the charge for registering and
certifying credit, and it will be taught in the schools of the world.
The ways of university men are a wonderful thing to me. There is no body
of men in the world that approaches them in the all-round and confirmed
habit of self-laudation; they write books endlessly about each other's
books, hold frequent dinners to bestow fulsome compliments upon each
other, and the like. The only possible explanation of the matter that I
can detect is that they have nothing else to do of absorbing interest;
that is, they have no objective in the full sense of the word. The
officers of an army have an objective, and so have the officials of a
railway system. That the university men have none seems to me highly
symptomatic as to the real state of society, I take it that at the
beginnings of university organization in the~middle ages~the situation
was altogether different. Think I wrote you once that to my mind the
existing university system, regard being had to its lofty function or
pretension, is the narrowest trade union in the world. An extensive and
thorough contact at various university~centres~is my warrant for this
judgment.~~

I think the fruitless attempt of Stuart Mill to find the key to social
science helps to confirm the view that only so fast and so far as the
organism is revealed to us as a working fact, with a clear view as to
its mode of operation and lines of development in space, could the
science of society come to maturity. The statement appears commonplace
to me, and it will become such to all others. Mill, as laid down in
his~\emph{System of Logic}, sought~``some one element in the complex
existence of social man, pre-eminent over all others as the prime agent
of the social movement,''\footnote{{[}John Stuart Mill,~\emph{A System
  of Logic~}(New York: Harper \& Brothers, 1858), 606\emph{.}{]}}~and
from the accumulated evidence he concluded~``that the order of human
progression in all respects will depend on the order of progression in
the intellectual convictions of mankind.''\footnote{{[}Mill, \emph{A
  System of Logic}, 586\emph{.}{]}}~Such merely speculative views are of
no help to us now in framing the descriptive science of the body
politic, save of course as they emphasize the~transcendant
{[}\emph{sic}{]}~importance of the quest. Concrete results were not
possible at the date of~Mill's~writing, the middle of the last century;
to be sure the social body was rapidly~evoluting {[}\emph{sic}{]}~before
his eyes but he could not see the direction of the movement. The
locomotive and electric communication were there, but only in their
first stages. Yet Mill held clearly and firmly that the future had the
science of society in store. He looked forward to~``the birth of a
sociological system widely removed from the vague and conjectural
character of all former attempts, and worthy to take its place, at last,
among established sciences.'' ``When this time shall come,''~he
added,~``no important branch of human affairs will any longer be
abandoned to empiricism~and unscientific surmise: the circle of human
knowledge will be complete, and it can only thereafter receive further
enlargement by perpetual expansion from within.''\footnote{{[}Mill,
  \emph{A System of Logic}, 609\emph{.}{]}}~

At this point I~have to~submit two further references. The English
philosopher~Shadworth~Hodgson, writing in 1870, had this to say touching
the point that the rise of social science has had to wait on the
ripening of conditions:~

\begin{quote}
The further construction of a logic of politic . . . depends upon a
further analysis and classification of the phenomena of society. . . .
The de facto forces at work in the social organism must be known, before
a criterion can be discovered. . . .~The question of criterion is in
politic a question of the future, reserved for a more complete state of
knowledge. . . . In other words, the logic of the structure and
functions of society is still only in its tentative stage, because the
phenomena have not yet been sufficiently examined, or discovered in
their true relations. . . .~Contrary to the usual opinion I cannot but
think that the knowledge which we have of the structure and functions of
the individual consciousness is~more complete and accurate~than that
which has been attained of the corresponding structure and functions of
society.\footnote{{[}Shadworth H. Hodgson, \emph{Systematic: The Logic
  of Practice} (London: Robson and Sons, 1870), 93--95.{]}} ~
\end{quote}

As late as 1900, Lindley M.~Keasbey, a scientific investigator of
ability and standing, wrote in the \emph{International Monthly} as
follows, his subject being ``The Constitution of Society'':~

\begin{quote}
After centuries of speculation on the subject, Society is as much a
mystery as ever. Our knowledge of the universe notwithstanding, we live
and move and have our being~in the midst of~a social world, of whose
laws we have but an inkling and whose purposes we but dimly divine.
Science has enlightened nearly every other path but we are still groping
about for a satisfactory point of departure from which to explain the
complex of collective phenomena. It was easier for the philosophers of
the last century, because all were then agreed that Society was to be
rightly constituted by victorious analysis. . . .~But now doctrinaires
no longer hope to reconstruct society upon a fabulous state of nature;
scientists are seeking, instead, to discover the laws of social
evolution.~
\end{quote}

Mr.~Keasbey~endeavors to lead up to the science of society by tracing
the development of co-operation among men. It is all very well, but it
is hardly too much to say that the lines of social evolution cannot be
marked out in any complete or satisfactory way until the social body is
clearly and~definitely disclosed~to us as an object for science, so that
we can fairly apply the kodak method to it. In other words, we cannot
expect a faithful account of social development until the past can be
read in the light of the future, but this necessitates a revelation. To
be more concrete, it is obviously impossible to write the history of
commerce until the successive rise of its~centres~of control can be
discerned and outlined, and this, in turn, is impossible until society
appears to us as having arrived at autonomy. Again, progress in division
of labor cannot be traced until we have a determined body before us. We
can then give an account of its organs and their working relation, or
the divisions of labor in the body, when at the same time a rational and
summary account of the past with respect to the appearance of one
division of labor after another can be given. As with the physiological
body so with the social; the organs of the former~(brain, heart, lungs,
etc.) and their functions were gradually identified and defined only so
fast as the body came to be grasped as a unified whole and so brought
under a single operating or controlling principle. It was through
reading unity into the human body that the scientist made it real.~

Referring again to the attempts at forecasting the convergence of
letters upon life, or the coming together of theory and practice, of
science and politics, it is worth noting that the men~of physical
science are~fairly alone~in suggesting an institutional development, the
rule being that all references of the sort are of the individualist
order only. Thomas Carlyle, whose great faith never passed into sight,
wrote in his \emph{Hero Worship} of the ``Organization of the Literary
Guild,'' but he was unable to compass the organic viewpoint, which of
course was impossible at the time he wrote, in 1840. Yet he was able to
say:~

\begin{quote}
Complaint is often made . . . of what we call the disorganized condition
of society. . . . But perhaps if we look at this of Books and the
Writers of Books,~we shall find here . . .~the summary of all other
disorganization---a~sort of~heart, from which, to which, all other
confusion circulates in the world. . . . The writers of Newspapers,
Pamphlets, Poems, Books \emph{are} the real working effective Church of
a modern country. . . . Whoever~can speak, speaking now to a whole
nation, becomes a power, a branch of government, with inalienable weight
in law-making, in all acts of authority. . . . It seems to me that the
Sentimental by and by~will have to give place to the Practical. . .
.~Whatever thing has virtual unnoticed power . . .~will one day step
forth with palpably articulated, universally visible power. . . . And
yet alas, the~\emph{making}~of it~right--- what a business for a long
time to come! . . . And yet there can~be no doubt but it is coming;
advancing on us,~as yet~hidden in the bosom of centuries: this is a
prophecy one can risk. . . . I call this anomaly of
a~disorganic~Literary Class the heart of all other anomalies, at once
product and parent; some good arrangement for that would be as
the~\emph{punctum saliens}~of a new vitality and~just arrangement for
all. . . . The man of intellect at the top of affairs: this is the aim
of all constitutions and revolutions, if they have any aim.\footnote{{[}Thomas
  Carlyle,~\emph{Heroes and Hero Worship~}(New York: John B. Alden,
  1883), 121. Ford also quotes this passage in the \emph{Draft of
  Action.}{]}}~
\end{quote}
There is perhaps nothing in all literature more suggestive of the need
of a science of society than these burning words from Carlyle. In another place he was constrained to write:~
\enlargethispage{-\baselineskip}
\begin{quote}
That a~``Splendor of God,''~in one form or other, will have to unfold
itself from the heart of these our Industrial Ages too; or they will
never get themselves~``organized'';~but continue chaotic, distressed,
distracted, evermore, and have to perish in frantic suicidal
dissolution. . . . How, in conjunction with inevitable
Democracy,~indispensable Sovereignty is to exist:~certainly~it is the
hugest question ever heretofore propounded to mankind!\footnote{{[}Thomas
  Carlyle, \emph{Past and Present} (London: Chapman \& Hall, 1843), 215.
  Ford also quotes this passage in the \emph{Draft of Action.}{]}}~
\end{quote}

\noindent It is plain that Carlyle's~perception did not pass the individualist
viewpoint; that is, he conceived only of some~sort of universal society
of particular individuals regarded as a distinct or separate social
class---writers or~\emph{men}~of letters---perhaps a glorified ``Royal
Society.'' His assumption that book writers are men of intellect was of
course very wide of the mark. The necessity is to bring into clear
relief the function of literature or the moving intelligence in the
social body. But, as I have already indicated in various ways, to arrive
at the right solution of the question we must be able to recognize a
body, clearly determined before our sight on the plane of fact or
actuality. Concerning our word~\emph{body}, the \emph{New English
Dictionary} has this to say:~``The word has died out in German, its
place being taken by~\emph{leib}, originally ``life,''
and~\emph{korper}~from Latin, but, in English, \emph{body} remains as a
great and important word.''~Shakespeare:~``Imagination bodies forth the
forms of things unknown.''\footnote{{[}William Shakespeare, \emph{A
  Midsummer Night's Dream}, 5.1.15.{]}}~``Whether that the body public
be a horse, whereon the governor doth ride.''\footnote{{[}William
  Shakespeare, \emph{Measure for Measure}, 1.3.25.{]}}~

To repeat, and it cannot be too often repeated, the theoretical solution
of the question which Carlyle set up can only be reached through the
prior presentiment of a social body. Moreover, the answer in theory and
the practical solution are one and the same thing, thus affording an
example of the essential unity of theory and practice.~Freedom of action
for the~``man of letters,''~or his full functioning in society, differs
no whit in point of principle from~the freeing of any other individual
or of all individuals. The subject of Politics is always and everywhere
the practical relation of the Individual and the Whole. In~fact~we may
say that the entire course of social evolution is but a gradual
emancipation of the individual both in his mind and in his body, until,
under organic conditions in the full sense, all individuals whatever
their function or division of labor pass into free working relation with
the whole of action, and the man of letters with the rest. The measure
of freedom in society is freedom to act to the limit of one's true or
normal function. The individual is defined as any~centre~of social
action, inclusive of both the single individual and the
corporation-group.~~

I~have not found any forecast that the function of the man of letters
would become in the natural course of things directly integral with
pract-\\\noindent ice---at one with the action of the world. The writer is not to
furnish direction, he is to remain a critic merely, an instrument
standing apart from life; the evolutionary view is nowhere presented.
Even John Morley,~writing in his \emph{Voltaire}, is unable to foresee
that with the~ever advancing~social integration literature, science was
certain to become as closely related to life as the brain to the hand.
Echoing Voltaire\textquotesingle s views of this point, Morley said:~

\begin{quote}
Though himself perhaps the most puissant man of letters that ever lived,
he (Voltaire) rated literature as it ought to be rated below action, not
that written speech is less of a force, but because the speculation and
criticism of the literature that substantially influences the world,
make far less demand than the actual conduct of great affairs on
qualities which are not rare in detail, but are amazingly rare in
combination---on temper, foresight, solidity, daring---on strength, in a
word, strength of intelligence and strength of character.~
\end{quote}

\noindent The essential unity of literature and exact inquiry was not apparent to
Mr. Morley, nor that they move forward in the closest possible working
relation to their common goal---the illumination of society. The telling
phrase, in fact all grace of style, is but the feather on the arrow of
truth which Science designs and frames.~

The Frenchman, Renan, sounded a clearer note when he said in his
\emph{Future of Science}:~``The scientific organization of humanity is
the final word of modern science, that is, its bold but legitimate
pretension. . . . The master science will investigate the aims and
conditions of society. . . . The day will come when the government of
humanity will no~longer be given to accident and intrigue, but to the
rational discussion of what is best, and the most efficacious means of
attaining that best.''\footnote{{[}Ernest Renan,~\emph{The Future of
  Science}~(Boston: Roberts Brothers, 1891), 30. Ford also discusses
  Renan's book in the \emph{Draft of Action}. {]}}~Renan remarked upon
the advance~``which has transformed literature into journalism and
periodical writing, which has reduced every work of the intellect to a
thing of actuality that will be forgotten in a short time.~The work of
intellect ceases to be a monument in order to become a fact---a lever of
opinion---there remains only the practical outcome.''\footnote{{[}Renan,
  \emph{Future of Science}, 211.{]}}~He looked forward to a state of
things~``in which the privilege of writing will no longer be a right
apart, but in which masses of individuals will only think of bringing
into circulation this or that order of ideas without appending to them
the label of their personality.''\footnote{{[}Renan,~\emph{Future of
  Science}, 212.{]}}~The realization of this idea is foreshadowed in the
newspaper interview; no matter how crude its present working may be
regarded, the interview contains the germ of an infinite development.
But Renan had no inkling that the revolution of which he dreamed~must of
necessity~be carried out through the commercial or profit-seeking
principle; on the contrary he rather frowned upon the ways of business.~

At this point I desire to lay stress on my use of the
word~\emph{venality}~above. Only so far as knowledge can be bought and
sold without let or hindrance is it possible to socialize intelligence;
it is by this road only that knowledge becomes power in the last sense.
Progress~in venality is, therefore, the true touchstone of social
development, or the goal of rational government. Action through the
commercial principle is of necessity highly organic whether in the field
of news or elsewhere; the constant struggle of the part of the dollar
hunters is to keep step with the social development, else their goods
will not sell. Vested interests have always frowned upon any extension
of the news-market; the first daily newspaper was of course, to
respectable opinion, the ``yellowest'' of all. You will remember that in
England almost on the appearance of printing, the Crown assumed that no
subject had any business to publish his thoughts without royal license.
After centuries of misguided zeal and energy it was discovered that
thought and speech are beyond the reach of all artificial restrictions
on the part of government, I have already referred to the fruitless
attempt of the English courts near the close of Charles II's reign to
bring the control of the printing press under the Common Law. If I
recall rightly, when the Sophists rose in Greece about 450 B.C. the
complaint against them was not alone that they were spreading knowledge
among the people, but that they~\emph{sold their wares}. Protagoras, it
is said, was the first Sophist~who taught for pay. The new
class~regarded their~money making~power~as~the measure of their
skill.~The people,~by the way,~always want in the~line~of news what~they
ought to~have; it~is the part~of science and art~to~give it to
them.~News is the most fluid~of~commodities,~and the~news-market cannot
be glutted.~

A word regarding the idea of~\emph{profit}, which has been tortured into
a wrong meaning in the economic discussion of the time.~The lesson of
biology is helpful here.~When rightly understood profit in the world of
business is no more than the necessary margin over waste to preserve the
integrity of a given operation.~

Francis Bacon looked upon religion as the ultimate social bond; today it
is seen to be the moving intelligence. It is the medium of association
among men. The more perfect the power of association the more does
society tend to take a natural form, and the greater the tendency to
durability or stable progress; association and the individualizing
process move forward together. The new test of social power is the
extent of one's control over the governing intelligence; private crop
reports, for example, are of the past. The traffic in news is now at
home in the world, having reached point of free exchange. The commerce
in knowledge which Condorcet saw has so far developed that an average of
one newsbook a day is now delivered to each household in New York City.
The unhindered exchange of Credit, the new legal instrument of social
coercion under the dominion of~Science, is to follow. It all means that
the news or science system is fast becoming the First Estate of the
realm. The future head of the State will be the~Newsmaster-General.~

The cry is now heard on all sides for an enlarged publicity, but its
promoters little realize the social meaning of their idea. The ordinary
notion of publicity is of a kind of lantern merely which will reveal
wrong doing for the courts and the police to correct, whereas the
fullest measure of news organization will abolish in great part our
inherited governing system. To see an object in the light of its
principle is to transform that object. The late Edward~Caird, writing of
the organization of the State, declared~``that that which we now really
aim at, and are demanding to become, is something which we should not
recognize if we now saw it in a completed form.''~And so, the demand for
greater publicity, or a new illumination touching the Corporation
Question and kindred matters, points, however unconsciously, to a
revolutionary departure from the received forms of political thinking.~

I would have you note that all the signal advances in publicity were
brought about~in spite of~the police power. A notable example was the
prolonged struggle on the part of the newspaper men at London, during
the first half of the eighteenth century, for the right to report the
proceedings of Parliament. It is hardly strange, in the face of the
present conditions, that I have difficulty in getting men to accept this
fact. The privilege was not conceded by the House of Commons until 1771,
even now the standing rule forbids the public reporting of~debates;
the~rule was simply allowed to become a dead letter. For several
decades~prior to 1771 offenders were brought before the bar of the House
and fined heavily. The victory belonged to the money hunters, who thus
accomplished an important forward step in the field of government.~The
day-by-day reporting of the English Parliament completed the socializing
of intelligence so far as it could be accomplished through that organ of
legislation; the doings of government were brought into organic relation
with the social body.~

It is only in the anarchist writings that I~find references
to~the~organization of science, regarded as a direct instrument of
government.~Thus, Proudhon wrote:~

\begin{quote}
By means of self-instruction and the acquisition of ideas, man finally
acquired the idea of~\emph{science---}that is, of a system of knowledge
in harmony with the reality of things, and inferred from observation. He
searches for the system of organic bodies, the system of the human mind,
and the system of the universe: why should he not also search for the
system of society? But, having reached this height, he comprehends that
political truth, or the science of politics, exists~quite independently
of the will of sovereigns and his king is the demonstrated truth; that
politics is a science, not a stratagem; and that the function of the
legislator is reduced, in the last analysis, to the methodical search
for truth. Thus,~in a given~society, the authority of man over man is
inversely proportional to the stage of intellectual development which
that society has reached; and the probable duration of that authority
can be calculated from the more or less general desire for a true
government---that is, for a scientific government. And just as the right
of force and the right of artifice retreat before the steady advance of
justice, so the sovereignty of the will yields to the sovereignty of the
reason, and must at last be lost in scientific socialism. . . . The
science of government rightly belongs~to one of the sections of the
Academy of Sciences, whose permanent secretary is necessarily prime
minister; and, since every citizen may address a memoir to the Academy,
every citizen is a legislator. But, as the opinion of no one is of any
value until the truth has been demonstrated, no one can substitute his
will for reason\emph{---}nobody is king. All questions of legislation
and politics are matters of science, not of opinion. . . .~What is it to
recognize a law? It is to verify a calculation; it is to repeat an
experiment, to observe a phenomenon, to establish a fact. The law, it is
said, is~\emph{the expression of the will of the sovereign}: then, under
a monarchy, the law is the expression of the will of the king; in a
republic, the law is the expression of the will of the people. Aside
from the difference in the number of wills, the two systems are~exactly
identical: both share the same error, namely, that the law is the
expression of a will; it ought to be the expression of a
fact.\footnote{{[}Proudhon, \emph{Property}, 34.{]}}~~
\end{quote}

But~of course~at the date of this writing (1840) Proudhon could see
little or nothing of the evolution of society to the stage of scientific
organization. This fact should increase our respect for his towering
faith. The French have this subtle proverb:~Might is right till right is
ready. It is interesting and suggestive to note Proudhon's firm belief
in the real existence of the social body. In his \emph{Contradictions
}(1846), he said:~

\begin{quote}
Most philosophers, like most~philologists, see in society only a
creature of the mind, or rather, an abstract name serving to designate a
collection of men. . . .~To the true economist, society is a~living
being, endowed with an intelligence and an activity of its own, governed
by special laws discoverable by observation alone, and whose existence
is manifested, not under a material aspect, but by the close concert and
mutual interdependence of all its members. Therefore, when we give a
name to the social being, an organic and synthetic unit, our language
is~in reality not~in the least metaphorical. In the eyes of anyone who
has reflected upon the laws of labor and exchange the reality, I had
almost said the personality, of the collective man is as certain as the
reality and the personality of the individual man. The only difference
is that the latter appears to the senses as an organism whose parts are
in a state of material coherence, which is not true of society. But
intelligence, spontaneity, development, life, all that constitutes in
the highest degree the reality of being, is as essential to society as
to man: and hence it is that the government of societies is
a~\emph{science}---that is, a study of natural relations---and not
an~\emph{art}---that is, good pleasure and absolutism. Hence it is,
finally, that every society declines the moment it falls into the hands
of the ideologists.\footnote{{[}Pierre-Joseph Proudhon, \emph{System of
  Economical Contradiction,} trans. Benjamin R. Tucker (Boston: Benjamin
  R. Tucker, 1888) 114--15.{]}}~
\end{quote}

If I were to single out the writings of two men who have helped me, I
would name Proudhon and Maitland. I regard Proudhon as the greatest
economic mind of the last century. Houston Chamberlain, in his book on
Richard Wagner, speaks of Proudhon as~``one of the most acute minds of
the century, on whom by some inconceivable paradox the dreaded title of
anarchist has been bestowed, after his having demonstrated the complete
anarchy of the~\emph{present}~order of things, and recognized in our
constitutions, the legalization of chaos.'' Proudhon understood by a
revolution, adds Chamberlain,~``not the building up of a new order by
violent means, but the end of anarchy.''\footnote{{[}Houston S.
  Chamberlain, \emph{Richard Wagner}, trans. G. Ainslie Hight \emph{}
  (London: J. M. Dent, 1897), 140.{]}}~Maitland aided me in seeing more
vividly the hierarchical view of the State, its continuance as the
dominant force, and the danger to social peace that lurks in it. I find
his idea of the impeccability of the State exemplified in the news of
the Beef Trust trial at Chicago, which discloses a clear avoidance on
the part of the government to get at the whole truth of the matter. Here
is a group of men doing business, through a working unity, over the
entire country (a previously~unheard of~thing), yet the facts are
handled from the individualist viewpoint strictly, as though they were
merely looking into the affairs of a village grocer. But how can the
government pursue the truth if the inherited fiction that all power is
concentrated in the Washington organ is to~be maintained? In a little
book by H. A. L. Fisher, which you have probably seen, there is a fac
simile {[}\emph{sic}{]} of a page of Maitland's manuscript containing
this sentence:~``It seems possible that we may easily overestimate the
creative power of lawyers, and courts and
legislators.''\footnote{{[}Herbert A. L. Fisher, \emph{Frederick William
  Maitland} (Cambridge: Cambridge University Press, 1910).{]}}~Looking
through my notes I find this from Maitland:~``The set of thoughts about
Law and Sovereignty into which Englishmen were lectured by John Austin
appears to Dr.~Gierke~as a past age. For him Sovereignty is an
attribute, not of some part of the State, but of
the~\emph{Gesammtperson}, the whole organized community. For him it is
as impossible to make the State logically prior to law (\emph{recht}) as
to make law logically prior to the State, since each exists in, for and
by the other.''\footnote{{[}Otto Gierke, \emph{Political Theories of the
  Middle Age}, trans. William F. Maitland \emph{} (Cambridge: Cambridge
  University Press, 1900), \emph{} xliii.{]}}~John Morley wrote
somewhere that the man who is carrying forward a difficult work, an
impossible if you please---something never done before---must have a
little triumph now and then to keep him in heart. Such a victory came to
me through Maitland's writings. I have looked at his letters to friends,
so far as they have been published, to see if he did not divulge in
them, touching the present conflict of the social forces, something not
contained in his~more formal writings, but I do not find much. He could
not cross over, so to speak, and become the law-finder or law-speaker of
the Industrial State; to prompt that close contact with the American
environment would have been necessary. But Maitland broke in thought
with the past, liberated as he had been by his historical studies.
He~was brought to see that things are, as never before since the
invention of gunpowder, in a flux that is full of portent for the
future.~~









\end{document}
