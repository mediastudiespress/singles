\documentclass[openany,nobib]{tufte-book}

\hypersetup{colorlinks=true,allcolors=[RGB]{97,15,11}}

% For print version
    % 1. change document class to add \documentclass[twoside,nohyper]
    % 2. comment out \hypersetup above
    % 3. get rid of the doi marginnotes
    % 4. export PDF, then use Print Production -> preflight -> tool icon -> scale 229 x 177


%%
% Book metadata
\title{Liberty and the News}
\author[Walter Lippmann]{Walter Lippmann}
\publisher{a mediastudies.press public domain edition}


%%
% If they're installed, use Bergamo and Chantilly from www.fontsite.com.
% They're clones of Bembo and Gill Sans, respectively.
%\IfFileExists{bergamo.sty}{\usepackage[osf]{bergamo}}{}% Bembo
%\IfFileExists{chantill.sty}{\usepackage{chantill}}{}% Gill Sans

%\usepackage{microtype}

\usepackage{slantsc}


% restart footnotes each chapter
\let\oldchapter\chapter
\def\chapter{%
  \setcounter{footnote}{0}%
  \oldchapter
}

%my addtion for toc
\newcounter{oldtocdepth}

\newcommand{\hidefromtoc}{%
  \setcounter{oldtocdepth}{\value{tocdepth}}%
  \addtocontents{toc}{\protect\setcounter{tocdepth}{-10}}%
}

\newcommand{\unhidefromtoc}{%
  \addtocontents{toc}{\protect\setcounter{tocdepth}{\value{oldtocdepth}}}%
}
\usepackage{hyperref}
\usepackage{bookmark}

%%
% Just some sample text
\usepackage{lipsum}

%%
% For nicely typeset tabular material
\usepackage{booktabs}

%%
% Another table package
\usepackage{tabu}

\usepackage{longtable}

%%
% For table spacing
\usepackage{verbatimbox}

%%
% For graphics / images
\usepackage{graphicx}
\setkeys{Gin}{width=\linewidth,totalheight=\textheight,keepaspectratio}
\graphicspath{{graphics/}}

% The fancyvrb package lets us customize the formatting of verbatim
% environments.  We use a slightly smaller font.
\usepackage{fancyvrb}
\fvset{fontsize=\normalsize}

%%
% Prints argument within hanging parentheses (i.e., parentheses that take
% up no horizontal space).  Useful in tabular environments.
\newcommand{\hangp}[1]{\makebox[0pt][r]{(}#1\makebox[0pt][l]{)}}

%%
% Prints an asterisk that takes up no horizontal space.
% Useful in tabular environments.
\newcommand{\hangstar}{\makebox[0pt][l]{*}}

%%
% Prints a trailing space in a smart way.
\usepackage{xspace}

%%
% Some shortcuts for Tufte's book titles.  The lowercase commands will
% produce the initials of the book title in italics.  The all-caps commands
% will print out the full title of the book in italics.
\newcommand{\vdqi}{\textit{VDQI}\xspace}
\newcommand{\ei}{\textit{EI}\xspace}
\newcommand{\ve}{\textit{VE}\xspace}
\newcommand{\be}{\textit{BE}\xspace}
\newcommand{\VDQI}{\textit{The Visual Display of Quantitative Information}\xspace}
\newcommand{\EI}{\textit{Envisioning Information}\xspace}
\newcommand{\VE}{\textit{Visual Explanations}\xspace}
\newcommand{\BE}{\textit{Beautiful Evidence}\xspace}

\newcommand*{\justlastragged}{%
\leftskip=0pt plus 1fil
\rightskip=-\leftskip
\parfillskip=\leftskip
\parindent=0pt}

\newcommand{\TL}{Tufte-\LaTeX\xspace}

% Prints the month name (e.g., January) and the year (e.g., 2008)
\newcommand{\monthyear}{%
  \ifcase\month\or January\or February\or March\or April\or May\or June\or
  July\or August\or September\or October\or November\or
  December\fi\space\number\year
}


% Prints an epigraph and speaker in sans serif, all-caps type.
\newcommand{\openepigraph}[2]{%
  %\sffamily\fontsize{14}{16}\selectfont
  \sffamily\large
  \begin{doublespace}
  \noindent\allcaps{#1}\\% epigraph
  \noindent\allcaps{#2}% author
  \end{doublespace}
}

\usepackage{enumitem}
\setlist[enumerate]{itemsep=0mm}

% Inserts a blank page
\newcommand{\blankpage}{\newpage\hbox{}\thispagestyle{empty}\newpage}

\usepackage{units}

% Typesets the font size, leading, and measure in the form of 10/12x26 pc.
\newcommand{\measure}[3]{#1/#2$\times$\unit[#3]{pc}}

% Macros for typesetting the documentation
\newcommand{\hlred}[1]{\textcolor{Maroon}{#1}}% prints in red
\newcommand{\hangleft}[1]{\makebox[0pt][r]{#1}}
\newcommand{\hairsp}{\hspace{1pt}}% hair space
\newcommand{\hquad}{\hskip0.5em\relax}% half quad space
\newcommand{\TODO}{\textcolor{red}{\bf TODO!}\xspace}
\newcommand{\ie}{\textit{i.\hairsp{}e.}\xspace}
\newcommand{\eg}{\textit{e.\hairsp{}g.}\xspace}
\newcommand{\na}{\quad--}% used in tables for N/A cells
\providecommand{\XeLaTeX}{X\lower.5ex\hbox{\kern-0.15em\reflectbox{E}}\kern-0.1em\LaTeX}
\newcommand{\tXeLaTeX}{\XeLaTeX\index{XeLaTeX@\protect\XeLaTeX}}
% \index{\texttt{\textbackslash xyz}@\hangleft{\texttt{\textbackslash}}\texttt{xyz}}
\newcommand{\tuftebs}{\symbol{'134}}% a backslash in tt type in OT1/T1
\newcommand{\doccmdnoindex}[2][]{\texttt{\tuftebs#2}}% command name -- adds backslash automatically (and doesn't add cmd to the index)
\newcommand{\doccmddef}[2][]{%
  \hlred{\texttt{\tuftebs#2}}\label{cmd:#2}%
  \ifthenelse{\isempty{#1}}%
    {% add the command to the index
      \index{#2 command@\protect\hangleft{\texttt{\tuftebs}}\texttt{#2}}% command name
    }%
    {% add the command and package to the index
      \index{#2 command@\protect\hangleft{\texttt{\tuftebs}}\texttt{#2} (\texttt{#1} package)}% command name
      \index{#1 package@\texttt{#1} package}\index{packages!#1@\texttt{#1}}% package name
    }%
}% command name -- adds backslash automatically
\newcommand{\doccmd}[2][]{%
  \texttt{\tuftebs#2}%
  \ifthenelse{\isempty{#1}}%
    {% add the command to the index
      \index{#2 command@\protect\hangleft{\texttt{\tuftebs}}\texttt{#2}}% command name
    }%
    {% add the command and package to the index
      \index{#2 command@\protect\hangleft{\texttt{\tuftebs}}\texttt{#2} (\texttt{#1} package)}% command name
      \index{#1 package@\texttt{#1} package}\index{packages!#1@\texttt{#1}}% package name
    }%
}% command name -- adds backslash automatically
\newcommand{\docopt}[1]{\ensuremath{\langle}\textrm{\textit{#1}}\ensuremath{\rangle}}% optional command argument
\newcommand{\docarg}[1]{\textrm{\textit{#1}}}% (required) command argument
\newenvironment{docspec}{\begin{quotation}\ttfamily\parskip0pt\parindent0pt\ignorespaces}{\end{quotation}}% command specification environment
\newcommand{\docenv}[1]{\texttt{#1}\index{#1 environment@\texttt{#1} environment}\index{environments!#1@\texttt{#1}}}% environment name
\newcommand{\docenvdef}[1]{\hlred{\texttt{#1}}\label{env:#1}\index{#1 environment@\texttt{#1} environment}\index{environments!#1@\texttt{#1}}}% environment name
\newcommand{\docpkg}[1]{\texttt{#1}\index{#1 package@\texttt{#1} package}\index{packages!#1@\texttt{#1}}}% package name
\newcommand{\doccls}[1]{\texttt{#1}}% document class name
\newcommand{\docclsopt}[1]{\texttt{#1}\index{#1 class option@\texttt{#1} class option}\index{class options!#1@\texttt{#1}}}% document class option name
\newcommand{\docclsoptdef}[1]{\hlred{\texttt{#1}}\label{clsopt:#1}\index{#1 class option@\texttt{#1} class option}\index{class options!#1@\texttt{#1}}}% document class option name defined
\newcommand{\docmsg}[2]{\bigskip\begin{fullwidth}\noindent\ttfamily#1\end{fullwidth}\medskip\par\noindent#2}
\newcommand{\docfilehook}[2]{\texttt{#1}\index{file hooks!#2}\index{#1@\texttt{#1}}}
\newcommand{\doccounter}[1]{\texttt{#1}\index{#1 counter@\texttt{#1} counter}}

% Generates the index
\usepackage{makeidx}
\makeindex

\titleformat{\section}%
  {\normalfont\LARGE\scshape}% format applied to label+text
  {\llap{\colorbox{orange}{\parbox{1.5cm}{\hfill\color{white}\thesection}}}}% label
  {1em}% horizontal separation between label and title body
  {}% before the title body
  []% after the title body

\titleformat{\subsection}%
  {\normalfont\large\scshape}% format applied to label+text
  {\llap{\colorbox{orange}{\parbox{1.5cm}{\hfill\color{white}\thesection}}}}% label
  {1em}% horizontal separation between label and title body
  {}% before the title body
  []% after the title body

%%%% Kevin Goody's code for title page and contents from https://groups.google.com/forum/#!topic/tufte-latex/ujdzrktC1BQ
\makeatletter
\renewcommand{\maketitlepage}{%
\begingroup%
\setlength{\parindent}{10pt}

{\fontsize{24}{24}\selectfont\textit{\@author}\par}

\vspace{1.75in}{\fontsize{27}{54}{\allcaps{\@title}}\par}

\vfill{\fontsize{14}{14}\selectfont\smallcaps{\@publisher}\par}

\thispagestyle{empty}
\endgroup
}
\makeatother

\titlecontents{part}%
    [0pt]% distance from left margin
    {\addvspace{0.25\baselineskip}}% above (global formatting of entry)
    {\allcaps{Part~\thecontentslabel}\allcaps}% before w/ label (label = ``Part I'')
    {\allcaps{Part~\thecontentslabel}\allcaps}% before w/o label
    {}% filler and page (leaders and page num)
    [\vspace*{0.5\baselineskip}]% after

\titlecontents{chapter}%
    [4em]% distance from left margin
    {}% above (global formatting of entry)
    {\contentslabel{2em}\textit}% before w/ label (label = ``Chapter 1'')
    {\hspace{0}\textit}% before w/o label
    {\qquad\thecontentspage}% filler and page (leaders and page num)
    [\vspace*{0.5\baselineskip}]% after

%%%% End additional code by Kevin Godby



\begin{document}

% REPEAT TITLE PAGE
\newpage
\begin{fullwidth}


\thispagestyle{empty}

\begingroup
\setlength{\parindent}{10pt}

\vspace*{2.25in}{\fontsize{27}{54}{\allcaps{Liberty and the News}}\par}

\endgroup

\end{fullwidth}

\newpage
\thispagestyle{plain} % empty
\mbox{}

% Front matter
\frontmatter\pagenumbering{roman}\setcounter{page}{3}

% ONE full title page
\begin{fullwidth}


\maketitle


% TWO copyright page
\newpage

~\vfill
\thispagestyle{empty}
\setlength{\parindent}{0pt}
\setlength{\parskip}{\baselineskip}
\emph{Liberty and the News}, originally published in 1920 by the \smallcaps{Harcourt, Brace and Howe}, is in the public domain. 

\par Published by \smallcaps{mediastudies.press} in the \smallcaps{Public Domain} series

\href{http://mediastudies.press}{mediastudies.press} | 414 W. Broad St., Bethlehem, PA 18018, USA

\par New materials are licensed under a Creative Commons Attribution-Noncommercial 4.0 (\href{https://creativecommons.org/licenses/by-nc/4.0/legalcode}{\smallcaps{CC BY-NC 4.0}})

\par \smallcaps{Cover design}: Mark McGillivray

\par \smallcaps{Copy editing}: Petra Dreiser

\par \smallcaps{Credit for scan}: Internet Archive, from the collections of the University of Michigan, \href{https://archive.org/details/libertyandnews00lippgoog}{2008 upload}

\par \smallcaps{Credit for LaTeX template}: \href{https://www.overleaf.com/latex/templates/book-design-inspired-by-edward-tufte/gcfbtdjfqdjh}{Book design inspired by Edward Tufte}, by \href{https://ctan.org/pkg/tufte-latex}{The Tufte-LaTeX Developers}

\par \smallcaps{978-1-951399-02-3} (print) | \smallcaps{978-1-951399-03-0} (ebook)

\par \smallcaps{DOI} \href{https://doi.org/10.32376/3f8575cb.2e69e142}{10.32376/3f8575cb.2e69e142}

\par \smallcaps{Library of Congress Control Number} 2020950484

\par\textit{Edition 1 published in November 2020}





% untitled foreword

\newpage
\thispagestyle{empty}
\begingroup
\vspace*{.15in}

\setlength{\parindent}{5ex}
\huge{In writing this tract I have dared to believe that many things were
possible because of the personal example offered to all who practice
journalism by Mr.~C. P. Scott, for over forty-five years editor-in-chief
of the \emph{Manchester Guardian}. In the light of his career it cannot
seem absurd or remote to think of freedom and truth in relation to the
news.}

\huge{Two of the essays in this volume, ``What Modern Liberty Means'' and
``Liberty and the News'' were published originally in the \emph{Atlantic
Monthly}. I wish to thank Mr.~Ellery Sedgwick for the encouragment he
gave me while writing them, and for permission to reprint them in this
volume.}

\begin{flushright}W. L.\end{flushright}

\vspace*{2mm}

\noindent New York City.\\
January 1, 1920.

\endgroup
\begingroup

% TOC

\LARGE

\tableofcontents

\endgroup

\end{fullwidth}

% INTRODUCTION TO MSP EDITION
\chapter[THE TWIN CRISES OF DEMOCRACY AND JOURNALISM]{THE TWIN CRISES OF DEMOCRACY AND\\ JOURNALISM\\ Introduction to the mediastudies.press edition}
\label{ch:introduction-msp}
\chaptermark{TWIN CRISES OF DEMOCRACY AND JOURNALISM}

\emph{\smallcap{\LARGE{Sue Curry Jansen}}}

\vspace{0.5in}

\noindent \textsc{\emph{Liberty and the News}}\marginnote{\href{https://doi.org/10.32376/3f8575cb.1adecce6}{doi}} was published a century ago this year. A
small book---consisting of two essays previously published in the
\emph{Atlantic Monthly}, joined together by a short introductory
chapter---it seemed an unpretentious offering by Walter Lippmann
(1889--1974). It was young Lippmann's fifth book, and, as a compilation,
commentators have long considered it a minor work. Yet the questions it
raised about journalism and democracy became the catalyst for a period
of generative thinking by the author, leading to his classic,
\emph{Public Opinion} (1922), and its sequel, \emph{The Phantom Public}
(1925). The issues this little book identified would continue to
influence Lippmann's thinking about the role of the media and the public
throughout his long life.\footnote{Walter Lippmann,
  \emph{\href{http://www.worldcat.org/oclc/812629160}{Public Opinion}}
  (New York: Harcourt, Brace and Co., 1922); and Lippmann,
  \emph{\href{http://www.worldcat.org/oclc/550725}{The Phantom Public}}
  (New York: Harcourt, Brace and Co., 1925).}

Lippmann wrote \emph{Liberty} in the immediate aftermath of his military
service in France and his attendance at the peace negotiations in
Versailles as an assistant to Colonel House, President Woodrow Wilson's
chief advisor. Like so many members of the so-called Lost Generation,
Lippmann felt deeply disillusioned by his wartime experiences.
\emph{Liberty} marked his initial attempt at coming to terms with the
chasm that separated the idealism that had attracted him to a career as
a journalist and public intellectual, and the systemic barriers that too
often undermined the values fueling that idealism. Lippmann, who always
chose his words carefully, referred to these obstacles as
``censorships.''

Lippmann's use of the term \emph{censorship} has not received adequate
attention in the voluminous analyses and critical commentaries penned by
generations of readers and interpreters. Yet censorship was a pressing
public issue when Lippmann wrote \emph{Liberty,} as was its counterpart,
propaganda. Both vividly shaped the author's immediate experiences
during World War I.

Lippmann was one of the founding editors of the \emph{New Republic} in
1914\emph{---}a magazine that published many of America's most
influential intellectuals. Its editorial board had supported Wilson's
re-election in 1916 and strongly favored U.S. entry into the war. In
preparation for the mobilization of troops, Colonel House asked Lippmann
to prepare a memorandum on wartime information policy. In response,
Lippmann recommended that the Wilson administration set up an official
news bureau that would provide the public with accurate information, and
identify and discredit rumors and falsehoods. He also urged the
administration to avoid arbitrary censorships. Lippmann recognized that
some censorship was necessary during wartime to protect the troops, but
contended that ``protection of a healthy public opinion'' was of ``first
importance.''\footnote{Quoted in Ronald Steel,
  \emph{\href{http://www.worldcat.org/oclc/1087969028}{Walter Lippmann
  and the American Century}} (Boston: Little, Brown and Co., 1980), 125.} Wilson was not
persuaded. Instead, the president authorized the creation of the
Committee on Public Information (CPI), which insiders at the time
referred to colloquially as the ``Ministry of
Information.''\footnote{James R. Mock and Cedric Larson,
  \emph{\href{http://www.worldcat.org/oclc/57518482}{Words That Won the
  War: The Story of the Committee on Public Information, 1917--1919}}
  (Princeton, NJ: Princeton University Press, 1939).} The CPI was headed
by George Creel, a former journalist whom Lippmann had heavily
criticized for some of his coverage of the 1913--1914 Colorado
coalfields labor conflicts that included what came to be known as the
Ludlow Massacre.

\enlargethispage{\baselineskip}

The CPI developed into a propaganda behemoth with more than thirty
divisions, including a censorship board; it exercised unprecedented
influence over well-nigh every aspect of American life. Every available
form of media was used to support the CPI's messaging, with the help of
writers, journalists, artists, cartoonists, the film and advertising
industries, clergy, school teachers, citizen volunteers, social clubs,
and virtually everyone who had an audience, no matter how small. News
was censored following ``voluntary'' rules that sanitized and glorified
the war. Other U.S. government entities also imposed restrictions on
free expression. Congress enacted alien and sedition acts. The
postmaster assumed the role of censor by refusing to send through the
mail socialist and other periodicals deemed offensive. Massive
deportations of foreign-born war critics ensued; prominent war
protesters, including the former presidential candidate Eugene Debs,
were arrested and given long prison sentences; radical college
professors were fired; and with mounting war hysteria, fanned by the
CPI, discrimination against German Americans increased as well. Lippmann
himself completed his military service in a propaganda unit in France,
composing leaflets that urged German soldiers to surrender and
interviewing German prisoners of war.
\clearpage
By 1920, censorship amounted to more than an intellectual and
professional concern for Lippmann; it had a visceral meaning for him,
fueling his heated rhetoric. That year, he also published as a
supplement to the \emph{New Republic} a study of self-censorship by the
press, co-authored with his longtime friend and colleague Charles
Merz and titled ``A Test of the
News.''\footnote{Walter Lippmann and Charles Merz, with the assistance of Faye
  Lippmann, ``A Test of the News,'' \emph{New Republic}, August 4, 1920,
  3.} Pioneering content analysis
as a method for assessing media bias, Lippmann and Merz analyzed
coverage of the Russian Revolution in the \emph{New York Times}. ``A
Test'' was more than twice as long as \emph{Liberty;} it examined more
than 3,000 \emph{Times} articles covering Russia from March 1917 to
March 1920. The overall conclusion? The \emph{Times}, one of America's
most trusted news sources, failed the test. According to Lippmann and
Merz, the paper's news coverage of Russia was ``dominated by the hopes
of the men who composed the news organization.'' The journalists saw
what they wanted to see, not what was actually happening. Lippmann and
Merz concluded that ``the chief censor and chief propagandist were hope
and fear in the minds of the reporters and
editors.''\footnote{Lippmann and Merz, ``A Test of the News,'' 3.} Their desire to win the
war and see the revolution defeated led them to systematically
misrepresent the facts in Russia.

Today, we should read ``A Test of the News'' as a companion to, or even
a missing chapter of, \emph{Liberty.} It provided Lippmann with credible
empirical evidence to support the arguments he developed in
\emph{Liberty---}arguments both prospective and retrospective. Lippmann
worried that the kinds of practices that had made ``the manufacture of
consent'' possible on an unprecedented scale during the war were
becoming normalized in peacetime. The CPI had recruited thousands of
cultural workers as volunteers to the government's propaganda effort:
journalists, advertisers, illustrators, writers, actors, etc. New forms
of mass media---radio and film---had greatly amplified the reach of
their work. By 1920, the commercial potential of these new media, and
the enhanced persuasive techniques developed by CPI staff and
volunteers, were already being exploited by the private sector.

Lippmann focused specifically on how the kinds of bias that had
distorted the \emph{Times} coverage of the Russian Revolution were
becoming the postwar norm. He contended that the work of reporters had
become ``confused with the work of preachers, revivalists, prophets and
agitators.'' He added, ``The current theory of American newspaperdom is
that an abstraction like the truth and a grace like fairness must be
sacrificed whenever anyone thinks the necessities of civilization
require the sacrifice.'' Lippmann specifically called out Adolph Ochs,
the owner of the \emph{New York Times}, and Lord Northcliffe, a British
newspaper magnate and wartime director of propaganda, as newspaper
owners who ``believe that edification is more important than veracity.
They believe it profoundly, violently, relentlessly. They preen
themselves upon it.''\footnote{Lippmann, \emph{Liberty and the News}, 2, 3. Page references are
  to the mediastudies.press edition.}

Lippmann accused the men of putting ``a painted screen where there
should be a window to the world.'' Because democratic governance
requires citizens to have access to truth, he accused both public and
private censors and propagandists of ``attacking the foundations of our
constitutional system.'' Lippmann, whose mentors included the
philosophers William James and George Santayana, was not naive about the
nature of truth: He recognized that in a secular age people are
``critically aware of how their purposes are special to their age, their
locality, their interests, and their limited
knowledge.''\footnote{Lippmann, \emph{Liberty}, 3, 4.} He would ultimately
arrive at a Peircean conception of truth---although some interpreters
mistakenly impute a simplistic theory of journalistic objectivity to
him.\footnote{Walter Lippmann,
  \emph{\href{http://www.worldcat.org/oclc/976578926}{The Public
  Philosophy}} (Boston: Little, Brown and Co., 1955). For Lippmann, as
  for Charles Sanders Peirce, truth is socially constructed dialogically
  by a community that shares common epistemological standards. Through
  ongoing critique, it gradually produces more reliable knowledge over
  time.} According to Lippmann, the
painted screens have left the public ``baffled because facts are not
available; and they are wondering whether government by consent can
survive in a time when the manufacture of consent is an unregulated
private enterprise.'' For, he contended, ``in an exact sense the present
crisis of western democracy is a crisis in
journalism.''\footnote{Lippmann, \emph{Liberty}, 1.}

It is in this context that we can understand Lippmann's distinctive
definition of \emph{liberty}, which has puzzled many readers and
interpreters through the years. For Lippmann, liberty constitutes a
method, not a series of prohibitions and permissions: ``Liberty is the
name we give to measures by which we protect and increase the veracity
of the information upon which we act.'' The critical purpose of
\emph{Liberty and the News}, then, is to identify and examine specific
political, sociological, and technological obstacles (``censorships'')
that undermine the veracity of the information provided by the news. The
book's constructive project means to identify and examine potential
reforms---ethics, policies, and practices---that may increase the
reliability of the news. Because, Lippmann concludes, ``there can be no
liberty for a community which lacks the information by which to detect
lies.''\footnote{Lippmann, \emph{Liberty}, 21, 20.}


That conclusion appears as resonant in 2020 as it was in 1920. Almost
eerily so, as the prominent writer and journalism teacher Roy Peter Clark
testifies. He reports serendipitously coming across a weathered copy of
\emph{Liberty and the News} in a library storage room. This was his
first encounter with the book, although he of course knew \emph{Public
Opinion}. He prefaces his response to \emph{Liberty} with an unscholarly
``Wow,'' and describes the immediacy of its message for the current
plight of the press and the public. ``In a single day,'' Clark writes,
``I read the text, making notes about almost every page. What I learned
startled me, like discovering an ancient scroll meant to be found a
century into the future, unearthed just in time to rescue civilization
from catastrophe.''\footnote{Roy Peter Clark, ``\href{https://www.poynter.org/ethics-trust/2018/walter-lippmann-on-liberty-and-the-news-a-century-old-mirror-for-our-troubled-times/}{Walter Lippmann on Liberty and the News: A
  Century-Old Mirror for our Troubled Times},'' Poynter, March 1, 2018.}


% REPEAT TITLE PAGE
\newpage
\begin{fullwidth}


\thispagestyle{empty}

\begingroup
\setlength{\parindent}{10pt}

\vspace*{2.25in}{\fontsize{27}{54}{\allcaps{Liberty and the News}}\par}

\endgroup

\end{fullwidth}

% MAINMATTER
\mainmatter\pagenumbering{arabic}\setcounter{page}{1}


% CHAPTER ONE
\chapter[1 \hspace*{1mm} JOURNALISM AND THE HIGHER LAW]{1 JOURNALISM AND THE HIGHER LAW}
\label{ch:JOURNALISM}

\newthought{Volume 1, Number 1,}\marginnote{\href{https://doi.org/10.32376/3f8575cb.6ab1dfab}{doi} | \href{https://github.com/mediastudiespress/singles/raw/master/public_domain/lippmann-1920/pdfs/02-lippmann-1920-chapter-one-original.pdf}{original pdf}} of the first American newspaper was published in
Boston on September 25, 1690. It was called \emph{Publick Occurrences}.
The second issue did not appear because the Governor and Council
suppressed it. They found that Benjamin Harris, the editor, had printed
``reflections of a very high
nature.''\footnote{``History of American Journalism,'' James Melvin Lee, Houghton Mifflin
  Co., 1917, p. 10.} Even to-day some of his
reflections seem very high indeed. In his prospectus he had written:

\begin{quote}
``That something may be done toward the Curing, or at least the Charming
of that Spirit of Lying, which prevails amongst us, wherefore nothing
shall be entered, but what we have reason to believe is true, repairing
to the best fountains for our Information. And when there appears any
material mistake in anything that is collected, it shall be corrected in
the next. Moreover, the Publisher of these Occurrences is willing to
engage, that whereas, there are many False Reports, maliciously made,
and spread among us, if any well-minded person will be at the pains to
trace any such false Report, so far as to find out and Convict the First
Raiser of it, he will in this Paper (unless just Advice be given to the
contrary) expose the Name of such Person, as A malicious Raiser of a
false Report. It is suppos'd that none will dislike this Proposal, but
such as intend to be guilty of so villainous a Crime.''
\end{quote}

Everywhere to-day men are conscious that somehow they must deal with
questions more intricate than any that church or school had prepared
them to understand. Increasingly they know that they cannot understand
them if the facts are not quickly and steadily available. Increasingly
they are baffled because the facts are not available; and they are
wondering whether government by consent can survive in a time when the
manufacture of consent is an unregulated private enterprise. For in an
exact sense the present crisis of western democracy is a crisis in
journalism.

I do not agree with those who think that the sole cause is corruption.
There is plenty of corruption, to be sure, moneyed control, caste
pressure, financial and social bribery, ribbons, dinner parties, clubs,
petty politics. The speculators in Russian rubles who lied on the Paris
Bourse about the capture of Petrograd are not the only example of their
species. And yet corruption does not explain the condition of modern
journalism.

\enlargethispage{\baselineskip}

Mr.~Franklin P. Adams wrote recently: ``Now there is much
pettiness---and almost incredible stupidity and ignorance---in the
so-called free press; but it is the pettiness, etc., common to the
so-called human race a pettiness found in musicians, steamfitters,
landlords, poets, and waiters. And when Miss Lowell {[}who had made the
usual aristocratic complaint{]} speaks of the incurable desire in all
American newspapers to make fun of everything in season and out, we
quarrel again. There is an incurable desire in American newspapers to
take things much more seriously than they deserve. Does Miss Lowell read
the ponderous news from Washington? Does she read the society news? Does
she, we wonder, read the newspapers?''

Mr.~Adams does read them, and when he writes that the newspapers take
things much more seriously than they deserve, he has, as the mayor's
wife remarked to the queen, said a mouthful. Since the war, especially,
editors have come to believe that their highest duty is not to report
but to instruct, not to print news but to save civilization, not to
publish what Benjamin Harris calls ``the Circumstances of Publique
Affairs, both abroad and at home,'' but to keep the nation on the
straight and narrow path. Like the Kings of England, they have elected
themselves Defenders of the Faith. ``For five years,'' says Mr.~Cobb of
the \emph{New York World}, ``there has been no free play of public
opinion in the world. Confronted by the inexorable necessities of war,
governments conscripted public opinion\ldots. They goose-stepped it.
They taught it to stand at attention and salute\ldots. It sometimes
seems that after the armistice was signed, millions of Americans must
have taken a vow that they would never again do any thinking for
themselves. They were willing to die for their country, but not willing
to think for it.'' That minority, which is proudly prepared to think for
it, and not only prepared, but cocksure that it alone knows how to think
for it, has adopted the theory that the public should know what is good
for it.

The work of reporters has thus become confused with the work of
preachers, revivalists, prophets and agitators. The current theory of
American newspaperdom is that an abstraction like the truth and a grace
like fairness must be sacrificed whenever anyone thinks the necessities
of civilization require the sacrifice. To Archbishop Whately's dictum
that it matters greatly whether you put truth in the first place or the
second, the candid expounder of modern journalism would reply that he
put truth second to what he conceived to be the national interest.
Judged simply by their product, men like Mr.~Ochs or Viscount
Northcliffe believe that their respective nations will perish and
civilization decay unless their idea of what is patriotic is permitted
to temper the curiosity of their readers.

They believe that edification is more important than veracity. They
believe it profoundly, violently, relentlessly. They preen themselves
upon it. To patriotism, as they define it from day to day, all other
considerations must yield. That is their pride. And yet what is this but
one more among myriad examples of the doctrine that the end justifies
the means. A more insidiously misleading rule of conduct was, I believe,
never devised among men. It was a plausible rule as long as men believed
that an omniscient and benevolent Providence taught them what end to
seek. But now that men are critically aware of how their purposes are
special to their age, their locality, their interests, and their limited
knowledge, it is blazing arrogance to sacrifice hard-won standards of
credibility to some special purpose. It is nothing but the doctrine that
I want what I want when I want it. Its monuments are the Inquisition and
the invasion of Belgium. It is the reason given for almost every act of
unreason, the law invoked whenever lawlessness justifies itself. At
bottom it is nothing but the anarchical nature of man imperiously
hacking its way through.

Just as the most poisonous form of disorder is the mob incited from high
places, the most immoral act the immorality of a government, so the most
destructive form of untruth is sophistry and propaganda by those whose
profession it is to report the news. The news columns are common
carriers. When those who control them arrogate to themselves the right
to determine by their own consciences what shall be reported and for
what purpose, democracy is unworkable. Public opinion is blockaded. For
when a people can no longer confidently repair `to the best fountains
for their information,' then anyone's guess and anyone's rumor, each
man's hope and each man's whim becomes the basis of government. All that
the sharpest critics of democracy have alleged is true, if there is no
steady supply of trustworthy and relevant news. Incompetence and
aimlessness, corruption and disloyalty, panic and ultimate disaster,
must come to any people which is denied an assured access to the facts.
No one can manage anything on pap. Neither can a people.

Statesmen may devise policies; they will end in futility, as so many
have recently ended, if the propagandists and censors can put a painted
screen where there should be a window to the world. Few episodes in
recent history are more poignant than that of the British Prime
Minister, sitting at the breakfast table with that morning's paper
before him protesting that he cannot do the sensible thing in regard to
Russia because a powerful newspaper proprietor has drugged the public.
That incident is a photograph of the supreme danger which confronts
popular government. All other dangers are contingent upon it, for the
news is the chief source of the opinion by which government now
proceeds. So long as there is interposed between the ordinary citizen
and the facts a news organization determining by entirely private and
unexamined standards, no matter how lofty, what he shall know, and hence
what he shall believe, no one will be able to say that the substance of
democratic government is secure. The theory of our constitution, says
Mr.~Justice Holmes, is that truth is the only ground upon which men's
wishes safely can be carried out.\footnote{Supreme Court of the United States, No. 316, October term, 1919, Jacob
  Abrams et al., Plaintiffs in Error vs. the United States.}
In so far as those who purvey the news make of their own beliefs a
higher law than truth, they are attacking the foundations of our
constitutional system. There can be no higher law in journalism than to
tell the truth and shame the devil.

That I have few illusions as to the difficulty of truthful reporting
anyone can see who reads these pages. If truthfulness were simply a
matter of sincerity the future would be rather simple. But the modern
news problem is not solely a question of the newspaperman's morals. It
is, as I have tried to show in what follows, the intricate result of a
civilization too extensive for any man's personal observation. As the
problem is manifold, so must be the remedy. There is no panacea. But
however puzzling the matter may be, there are some things that anyone
may assert about it, and assert without fear of contradiction. They are
that there \emph{is} a problem of the news which is of absolutely basic
importance to the survival of popular government, and that the
importance of that problem is not vividly realized nor sufficiently
considered.

In a few generations it will seem ludicrous to historians that a people
professing government by the will of the people should have made no
serious effort to guarantee the news without which a governing opinion
cannot exist. ``Is it possible,'' they will ask, ``that at the beginning
of the Twentieth Century nations calling themselves democracies were
content to act on what happened to drift across their doorsteps; that
apart from a few sporadic exposures and outcries they made no plans to
bring these common carriers under social control; that they provided no
genuine training schools for the men upon whose sagacity they were
dependent; above all that their political scientists went on year after
year writing and lecturing about government without producing one, one
single, significant study of the process of public opinion?'' And then
they will recall the centuries in which the Church enjoyed immunity from
criticism, and perhaps they will insist that the news structure of
secular society was not seriously examined for analogous reasons.

When they search into the personal records they will find that among
journalists, as among the clergy, institutionalism had induced the usual
prudence. I have made no criticism in this book which is not the
shoptalk of reporters and editors. But only rarely do newspapermen take
the general public into their confidence. They will have to sooner or
later. It is not enough for them to struggle against great odds, as many
of them are doing, wearing out their souls to do a particular assignment
well. The philosophy of the work itself needs to be discussed; the news
about the news needs to be told. For the news about the government of
the news structure touches the center of all modern government.

They need not be much concerned if leathery-minded individuals ask What
is Truth of all who plead for the effort of truth in modern journalism.
Jesting Pilate asked the same question, and he also would not stay for
an answer. No doubt an organon of news reporting must wait upon the
development of psychology and political science. But resistance to the
inertias of the profession, heresy to the institution, and the
willingness to be fired rather than write what you do not believe, these
wait on nothing but personal courage. And without the assistance which
they will bring from within the profession itself, democracy through [\emph{sic}] it
will deal with the problem somehow, will deal with it badly.

The essays which follow are an attempt to describe the character of the
problem, and to indicate headings under which it may be found useful to
look for remedies.

\newpage
\thispagestyle{plain} % empty
\mbox{}

% CHAPTER TWO
\chapter[2 \hspace*{1mm} WHAT MODERN LIBERTY MEANS]{2 WHAT MODERN LIBERTY MEANS}
\label{ch:MODERN-LIBERTY}

\newthought{From our recent}\marginnote{\href{https://doi.org/10.32376/3f8575cb.4b424f45}{doi} | \href{https://github.com/mediastudiespress/singles/raw/master/public_domain/lippmann-1920/pdfs/03-lippmann-1920-chapter-two-original.pdf}{original pdf}} experience it is clear that the traditional liberties of
speech and opinion rest on no solid foundation. At a time when the world
needs above all other things the activity of generous imaginations and
the creative leadership of planning and inventive minds, our thinking is
shriveled with panic. Time and energy that should go to building and
restoring are instead consumed in warding off the pin-pricks of
prejudice and fighting a guerilla war against misunderstanding and
intolerance. For suppression is felt, not simply by the scattered
individuals who are actually suppressed. It reaches back into the
steadiest minds, creating tension everywhere; and the tension of fear
produces sterility. Men cease to say what they think; and when they
cease to say it, they soon cease to think it. They think in reference to
their critics and not in reference to the facts. For when thought
becomes socially hazardous, men spend more time wondering about the
hazard than they do in cultivating their thought. Yet nothing is more
certain than that mere bold resistance will not permanently liberate
men's minds. The problem is not only greater than that, but different,
and the time is ripe for reconsideration. We have learned that many of
the hard-won rights of man are utterly insecure. It may be that we
cannot make them secure simply by imitating the earlier champions of
liberty.

\enlargethispage{\baselineskip}

Something important about the human character was exposed by Plato when,
with the spectacle of Socrates's death before him, he founded Utopia on
a censorship stricter than any which exists on this heavily censored
planet. His intolerance seems strange. But it is really the logical
expression of an impulse that most of us have not the candor to
recognize. It was the service of Plato to formulate the dispositions of
men in the shape of ideals, and the surest things we can learn from him
are not what we ought to do, but what we are inclined to do. We are
peculiarly inclined to suppress whatever impugns the security of that to
which we have given our allegiance. If our loyalty is turned to what
exists, intolerance begins at its frontiers; if it is turned, as Plato's
was, to Utopia, we shall find Utopia defended with intolerance.

There are, so far as I can discover, no absolutists of liberty; I can
recall no doctrine of liberty, which, under the acid test, does not
become contingent upon some other ideal. The goal is never liberty, but
liberty for something or other. For liberty is a condition under which
activity takes place, and men's interests attach themselves primarily to
their activities and what is necessary to fulfill them, not to the
abstract requirements of any activity that might be conceived.

And yet controversialists rarely take this into account. The battle is
fought with banners on which are inscribed absolute and universal
ideals. They are not absolute and universal in fact. No man has ever
thought out an absolute or a universal ideal in politics, for the simple
reason that nobody knows enough, or can know enough, to do it. But we
all use absolutes, because an ideal which seems to exist apart from
time, space, and circumstance has a prestige that no candid avowal of
special purpose can ever have. Looked at from one point of view
universals are part of the fighting apparatus in men. What they desire
enormously they easily come to call God's will, or their nation's
purpose. Looked at genetically, these idealizations are probably born in
that spiritual reverie where all men live most of the time. In reverie
there is neither time, space, nor particular reference, and hope is
omnipotent. This omnipotence, which is denied to them in action,
nevertheless illuminates activity with a sense of utter and irresistible
value.

The classic doctrine of liberty consists of absolutes. It consists of
them except at the critical points where the author has come into
contact with objective difficulties. Then he introduces into the
argument, somewhat furtively, a reservation which liquidates its
universal meaning and reduces the exalted plea for liberty in general to
a special argument for the success of a special purpose.

There are at the present time, for instance, no more fervent champions
of liberty than the western sympathizers with the Russian Soviet
government. Why is it that they are indignant when Mr.~Burleson
suppresses a newspaper and complacent when Lenin does? And, \emph{vice
versa}, why is it that the anti-Bolshevist forces in the world are in
favor of restricting constitutional liberty as a preliminary to
establishing genuine liberty in Russia? Clearly the argument about
liberty has little actual relation to the existence of it. It is the
purpose of the social conflict, not the freedom of opinion, that lies
close to the heart of the partisans. The word liberty is a weapon and an
advertisement, but certainly not an ideal which transcends all special
aims.

If there were any man who believed in liberty apart from particular
purposes, that man would be a hermit contemplating all existence with a
hopeful and neutral eye. For him, in the last analysis, there could be
nothing worth resisting, nothing particularly worth attaining, nothing
particularly worth defending, not even the right of hermits to
contemplate existence with a cold and neutral eye. He would be loyal
simply to the possibilities of the human spirit, even to those
possibilities which most seriously impair its variety and its health. No
such man has yet counted much in the history of politics. For what every
theorist of liberty has meant is that certain types of behavior and
classes of opinion hitherto regulated should be somewhat differently
regulated in the future. What each seems to say is that opinion and
action should be free; that liberty is the highest and most sacred
interest of life. But somewhere each of them inserts a weasel clause to
the effect that ``of course'' the freedom granted shall not be employed
too destructively. It is this clause which checks exuberance and reminds
us that, in spite of appearances, we are listening to finite men
pleading a special cause.

Among the English classics none are more representative than Milton's
\emph{Areopagitica} and the essay \emph{On Liberty} by John Stuart Mill.
Of living men Mr.~Bertrand Russell is perhaps the most outstanding
advocate of liberty. The three together are a formidable set of
witnesses. Yet nothing is easier than to draw texts from each which can
be cited either as an argument for absolute liberty or as an excuse for
as much repression as seems desirable at the moment. Says Milton:

\begin{quote}
Yet if all cannot be of one mind, as who looks they should be? this
doubtles is more wholsome, more prudent, and more Christian that many be
tolerated, rather than all compell'd.~
\end{quote}

So much for the generalization. Now for the qualification which follows
immediately upon it.

\begin{quote}
I mean not tolerated Popery, and open superstition, which as it
extirpats all religions and civill supremacies, so itself should be
extirpat, provided first that all charitable and compassionat means be
used to win and regain the weak and misled: that also which is impious
or evil absolutely either against faith or maners no law can possibly
permit, that intends not to unlaw it self: but those neighboring
differences, or rather \emph{indifferences}, are what I speak of,
whether in some point of doctrine or of discipline, which though they
may be many, yet need not interrupt the unity of spirit, if we could but
find among us the bond of peace.
\end{quote}

With this as a text one could set up an inquisition. Yet it occurs in
the noblest plea for liberty that exists in the English language. The
critical point in Milton's thought is revealed by the word
``indifferences.'' The area of opinion which he wished to free comprised
the ``neighboring differences'' of certain Protestant sects, and only
these where they were truly ineffective in manners and morals. Milton,
in short, had come to the conclusion that certain conflicts of doctrine
were sufficiently insignificant to be tolerated. The conclusion depended
far less upon his notion of the value of liberty than upon his
conception of God and human nature and the England of his time. He urged
indifference to things that were becoming indifferent.

If we substitute the word indifference for the word liberty, we shall
come much closer to the real intention that lies behind the classic
argument. Liberty is to be permitted where differences are of no great
moment. It is this definition which has generally guided practice. In
times when men feel themselves secure, heresy is cultivated as the spice
of life. During a war liberty disappears as the community feels itself
menaced. When revolution seems to be contagious, heresy-hunting is a
respectable occupation. In other words, when men are not afraid, they
are not afraid of ideas; when they are much afraid, they are afraid of
anything that seems, or can even be made to appear, seditious. That is
why nine-tenths of the effort to live and let live consists in proving
that the thing we wish to have tolerated is really a matter of
indifference.

In Mill this truth reveals itself still more clearly. Though his
argument is surer and completer than Milton's, the qualification is also
surer and completer.

\begin{quote}
Such being the reasons which make it imperative that human beings should
be free to form opinions, and to express their opinions without reserve;
and such the baneful consequences to the intellectual and through that
to the moral nature of man, unless this liberty is either conceded or
asserted in spite of prohibition, let us next examine whether the same
reasons do not require that men should be free to act upon their
opinions, to carry these out in their lives, without hindrance, either
moral or physical, from their fellow men, so long as it is at their own
risk and peril. \emph{This last proviso is of course indispensable.} No
one pretends that actions should be as free as opinions. On the
contrary, \emph{even opinions lose their immunity} when the
circumstances in which they are expressed are such as to constitute
their expression a positive instigation to some mischievous act.
\end{quote}

``At their own risk and peril.'' In other words, at the risk of eternal
damnation. The premise from which Mill argued was that many opinions
then under the ban of society were of no interest to society, and ought
therefore not to be interfered with. The orthodoxy with which he was at
war was chiefly theocratic. It assumed that a man's opinions on cosmic
affairs might endanger his personal salvation and make him a dangerous
member of society. Mill did not believe in the theological view, did not
fear damnation, and was convinced that morality did not depend upon the
religious sanction. In fact, he was convinced that a more reasoned
morality could be formed by laying aside theological assumptions. ``But
no one pretends that actions should be as free as opinions.'' The plain
truth is that Mill did not believe that much action would result from
the toleration of those opinions in which he was most interested.

Political heresy occupied the fringe of his attention, and he uttered
only the most casual comments. So incidental are they, so little do they
impinge on his mind, that the arguments of this staunch apostle of
liberty can be used honestly, and in fact are used, to justify the bulk
of the suppressions which have recently occurred. ``Even opinions lose
their immunity, \emph{when the circumstances} in which they are
expressed are such as to constitute their expression a positive
instigation to some mischievious act.'' Clearly there is no escape here
for Debs or Haywood or obstructors of Liberty Loans. The argument used
is exactly the one employed in sustaining the conviction of Debs.

In corroboration Mill's single concrete instance may be cited: ``An
opinion that corn dealers are starvers of the poor, or that private
property is robbery, ought to be unmolested when simply circulated
through the press, but may justly incur punishment when delivered orally
to an excited mob assembled before the house of a corn dealer, or when
handed about among the same mob in the form of a placard.''

Clearly Mill's theory of liberty wore a different complexion when he
considered opinions which might directly affect social order. Where the
stimulus of opinion upon action was effective he could say with entire
complacency, ``The liberty of the individual must be thus far limited;
he must not make himself a nuisance to other people.'' Because Mill
believed this, it is entirely just to infer that the distinction drawn
between a speech or placard and publication in the press would soon have
broken down for Mill had he lived at a time when the press really
circulated and the art of type-display had made a newspaper strangely
like a placard.

On first acquaintance no man would seem to go further than Mr.~Bertrand
Russell in loyalty to what he calls ``the unfettered development of all
the instincts that build up life and fill it with mental delights.'' He
calls these instincts ``creative''; and against them he sets off the
``possessive impulses.'' These, he says, should be restricted by ``a
public authority, a repository of practically irresistible force whose
function should be primarily to repress the private use of force.''
Where Milton said no ``tolerated Popery,'' Mr.~Russell says, no
tolerated ``possessive impulses.'' Surely he is open to the criticism
that, like every authoritarian who has preceded him, he is interested in
the unfettered development of only that which seems good to him. Those
who think that ``enlightened selfishness'' produces social harmony will
tolerate more of the possessive impulses, and will be inclined to put
certain of Mr.~Russell's creative impulses under lock and key.

The moral is, not that Milton, Mill, and Bertrand Russell are
inconsistent, or that liberty is to be obtained by arguing for it
without qualifications. The impulse to what we call liberty is as strong
in these three men as it is ever likely to be in our society. The moral
is of another kind. It is that the traditional core of liberty, namely,
the notion of indifference, is too feeble and unreal a doctrine to
protect the purpose of liberty, which is the furnishing of a healthy
environment in which human judgment and inquiry can most successfully
organize human life. Too feeble, because in time of stress nothing is
easier than to insist, and by insistence to convince, that tolerated
indifference is no longer tolerable because it has ceased to be
indifferent.

It is clear that in a society where public opinion has become decisive,
nothing that counts in the formation of it can really be a matter of
indifference. When I say ``can be,'' I am speaking literally. What men
believed about the constitution of heaven became a matter of
indifference when heaven disappeared in metaphysics; but what they
believe about property, government, conscription, taxation, the origins
of the late war, or the origins of the Franco-Prussian War, or the
distribution of Latin culture in the vicinity of copper mines,
constitutes the difference between life and death, prosperity and
misfortune, and it will never on this earth be tolerated as indifferent,
or not interfered with, no matter how many noble arguments are made for
liberty, or how many martyrs give their lives for it. If widespread
tolerance of opposing views is to be achieved in modern society, it will
not be simply by fighting the Debs' cases through the courts, and
certainly not by threatening to upset those courts if they do not yield
to the agitation. The task is fundamentally of another order, requiring
other methods and other theories.

The world about which each man is supposed to have opinions has become
so complicated as to defy his powers of understanding. What he knows of
events that matter enormously to him, the purposes of governments, the
aspirations of peoples, the struggle of classes, he knows at second,
third, or fourth hand. He cannot go and see for himself. Even the things
that are near to him have become too involved for his judgment. I know
of no man, even among those who devote all of their time to watching
public affairs, who can even pretend to keep track, at the same time, of
his city government, his state government, Congress, the departments,
the industrial situation, and the rest of the world. What men who make
the study of politics a vocation cannot do, the man who has an hour a
day for newspapers and talk cannot possibly hope to do. He must seize
catchwords and headlines or nothing.

\clearpage
This vast elaboration of the subject-matter of politics is the root of
the whole problem. News comes from a distance; it comes helter-skelter,
in inconceivable confusion; it deals with matters that are not easily
understood; it arrives and is assimilated by busy and tired people who
must take what is given to them. Any lawyer with a sense of evidence
knows how unreliable such information must necessarily be.

The taking of testimony in a trial is hedged about with a thousand
precautions derived from long experience of the fallibility of the
witness and the prejudices of the jury. We call this, and rightly, a
fundamental phase of human liberty. But in public affairs the stake is
infinitely greater. It involves the lives of millions, and the fortune
of everybody. The jury is the whole community, not even the qualified
voters alone. The jury is everybody who creates public
sentiment---chattering gossips, unscrupulous liars, congenital liars,
feeble-minded people, prostitute minds, corrupting agents. To this jury
any testimony is submitted, is submitted in any form, by any anonymous
person, with no test of reliability, no test of credibility, and no
penalty for perjury. If I lie in a lawsuit involving the fate of my
neighbor's cow, I can go to jail. But if I lie to a million readers in a
matter involving war and peace, I can lie my head off, and, if I choose
the right series of lies, be entirely irresponsible. Nobody will punish
me if I lie about Japan, for example. I can announce that every Japanese
valet is a reservist, and every Japanese art store a mobilization
center. I am immune. And if there should be hostilities with Japan, the
more I lied the more popular I should be. If I asserted that the
Japanese secretly drank the blood of children, that Japanese women were
unchaste, that the Japanese were really not a branch of the human race
after all, I guarantee that most of the newspapers would print it
eagerly, and that I could get a hearing in churches all over the
country. And all this for the simple reason that the public, when it is
dependent on testimony and protected by no rules of evidence, can act
only on the excitement of its pugnacities and its hopes.

The mechanism of the news-supply has developed without plan, and there
is no one point in it at which one can fix the responsibility for truth.
The fact is that the subdivision of labor is now accompanied by the
subdivision of the news-organization. At one end of it is the
eye-witness, at the other, the reader. Between the two is a vast,
expensive transmitting and editing apparatus. This machine works
marvelously well at times, particularly in the rapidity with which it
can report the score of a game or a transatlantic flight, or the death
of a monarch, or the result of an election. But where the issue is
complex, as for example in the matter of the success of a policy, or the
social conditions among a foreign people,---that is to say, where the
real answer is neither yes or no, but subtle, and a matter of balanced
evidence,---the subdivision of the labor involved in the report causes
no end of derangement, misunderstanding, and even misrepresentation.

Thus the number of eye-witnesses capable of honest statement is
inadequate and accidental. Yet the reporter making up his news is
dependent upon the eye-witnesses. They may be actors in the event. Then
they can hardly be expected to have perspective. Who, for example, if he
put aside his own likes and dislikes would trust a Bolshevik's account
of what exists in Soviet Russia or an exiled Russian prince's story of
what exists in Siberia? Sitting just across the frontier, say in
Stockholm, how is a reporter to write dependable news when his witnesses
consist of \emph{emigrés} or Bolshevist agents?

At the Peace Conference, news was given out by the agents of the
conferees and the rest leaked through those who were clamoring at the
doors of the Conference. Now the reporter, if he is to earn his living,
must nurse his personal contacts with the eye-witnesses and privileged
informants. If he is openly hostile to those in authority, he will cease
to be a reporter unless there is an opposition party in the inner circle
who can feed him news. Failing that, he will know precious little of
what is going on.

Most people seem to believe that, when they meet a war correspondent or
a special writer from the Peace Conference, they have seen a man who has
seen the things he wrote about. Far from it. Nobody, for example, saw
this war. Neither the men in the trenches nor the commanding, general.
The men saw their trenches, their billets, sometimes they saw an enemy
trench, but nobody, unless it be the aviators, saw a battle. What the
correspondents saw, occasionally, was the terrain over which a battle
had been fought; but what they reported day by day was what they were
told at press headquarters, and of that only what they were allowed to
tell.

At the Peace Conference the reporters were allowed to meet periodically
the four least important members of the Commission, men who themselves
had considerable difficulty in keeping track of things, as any reporter
who was present will testify. This was supplemented by spasmodic
personal interviews with the commissioners, their secretaries, their
secretaries' secretaries, other newspaper men, and confidential
representatives of the President, who stood between him and the
impertinences of curiosity. This and the French press, than which there
is nothing more censored and inspired, a local English trade-journal of
the expatriates, the gossip of the Crillon lobby, the Majestic, and the
other official hotels, constituted the source of the news upon which
American editors and the American people have had to base one of the
most difficult judgments of their history. I should perhaps add that
there were a few correspondents occupying privileged positions with
foreign governments. They wore ribbons in their button-holes to prove
it. They were in many ways the most useful correspondents because they
always revealed to the trained reader just what it was that their
governments wished America to believe.

The news accumulated by the reporter from his witnesses has to be
selected, if for no other reason than that the cable facilities are
limited. At the cable office several varieties of censorship intervene.
The legal censorship in Europe is political as well as military, and
both words are elastic. It has been applied, not only to the substance
of the news, but to the mode of presentation, and even to the character
of the type and the position on the page. But the real censorship on the
wires is the cost of transmission. This in itself is enough to limit any
expensive competition or any significant independence. The big
Continental news agencies are subsidized. Censorship operates also
through congestion and the resultant need of a system of priority.
Congestion makes possible good and bad service, and undesirable messages
are not infrequently served badly.

When the report does reach the editor, another series of interventions
occurs. The editor is a man who may know all about something, but he can
hardly be expected to know all about everything. Yet he has to decide
the question which is of more importance than any other in the formation
of opinions, the question where attention is to be directed. In a
newspaper the heads are the foci of attention, the odd corners the
fringe; and whether one aspect of the news or another appears in the
center or at the periphery makes all the difference in the world. The
news of the day as it reaches the newspaper office is an incredible
medley of fact, propaganda, rumor, suspicion, clues, hopes, and fears,
and the task of selecting and ordering that news is one of the truly
sacred and priestly offices in a democracy. For the newspaper is in all
literalness the bible of democracy, the book out of which a people
determines its conduct. It is the only serious book most people read. It
is the only book they read every day. Now the power to determine each
day what shall seem important and what shall be neglected is a power
unlike any that has been exercised since the Pope lost his hold on the
secular mind.

The ordering is not done by one man, but by a host of men, who are on
the whole curiously unanimous in their selection and in their emphasis.
Once you know the party and social affiliations of a newspaper, you can
predict with considerable certainty the perspective in which the news
will be displayed. This perspective is by no means altogether
deliberate. Though the editor is ever so much more sophisticated than
all but a minority of his readers, his own sense of relative importance
is determined by rather standardized constellations of ideas. He very
soon comes to believe that his habitual emphasis is the only possible
one.

Why the editor is possessed by a particular set of ideas is a difficult
question in social psychology, of which no adequate analysis has been
made. But we shall not be far wrong if we say that he deals with the
news in reference to the prevailing \emph{mores} of his social group.
These \emph{mores} are of course in a large measure the product of what
previous newspapers have said; and experience shows that, in order to
break out of this circle, it has been necessary at various times to
create new forms of journalism, such as the national monthly, the
critical weekly, the circular, the paid advertisements of ideas, in
order to change the emphasis which had become obsolete and habit-ridden.

Into this extremely refractory, and I think increasingly disserviceable
mechanism, there has been thrown, especially since the outbreak of war,
another monkey-wrench---propaganda. The word, of course, covers a
multitude of sins and a few virtues. The virtues can be easily separated
out, and given another name, either advertisement or advocacy. Thus, if
the National Council of Belgravia wishes to publish a magazine out of
its own funds, under its own imprint, advocating the annexation of
Thrums, no one will object. But if, in support of that advocacy, it
gives to the press stories that are lies about the atrocities committed
in Thrums; or, worse still, if those stories seem to come from Geneva,
or Amsterdam, not from the press-service of the National Council of
Belgravia, then Belgravia is conducting propaganda. If, after arousing a
certain amount of interest in itself, Belgravia then invites a carefully
selected correspondent, or perhaps a labor leader, to its capital, puts
him up at the best hotel, rides him around in limousines, fawns on him
at banquets, lunches with him very confidentially, and then puts him
through a conducted tour so that he shall see just what will create the
desired impression, then again Belgravia is conducting propaganda. Or if
Belgravia happens to possess the greatest trombone-player in the world,
and if she sends him over to charm the wives of influential husbands,
Belgravia is, in a less objectionable way, perhaps, committing
propaganda, and making fools of the husbands.

Now, the plain fact is that out of the troubled areas of the world the
public receives practically nothing that is not propaganda. Lenin and
his enemies control all the news there is of Russia, and no court of law
would accept any of the testimony as valid in a suit to determine the
possession of a donkey. I am writing many months after the Armistice.
The Senate is at this moment engaged in debating the question whether it
will guarantee the frontiers of Poland; but what we learn of Poland we
learn from the Polish Government and the Jewish Committee. Judgment on
the vexed issues of Europe is simply out of the question for the average
American; and the more cocksure he is, the more certainly is he the
victim of some propaganda.

These instances are drawn from foreign affairs, but the difficulty at
home, although less flagrant, is nevertheless real. Theodore Roosevelt,
and Leonard Wood after him, have told us to think nationally. It is not
easy. It is easy to parrot what those people say who live in a few big
cities and who have constituted themselves the only true and authentic
voice of America. But beyond that it is difficult. I live in New York
and I have not the vaguest idea what Brooklyn is interested in. It is
possible, with effort, much more effort than most people can afford to
give, for me to know what a few organized bodies like the Non-Partisan
League, the National Security League, the American Federation of Labor,
and the Republican National Committee are up to; but what the
unorganized workers, and the unorganized farmers, the shopkeepers, the
local bankers and boards of trade are thinking and feeling, no one has
any means of knowing, except perhaps in a vague way at election time. To
think nationally means, at least, to take into account the major
interests and needs and desires of this continental population; and for
that each man would need a staff of secretaries, traveling agents, and a
very expensive press-clipping bureau.

We do not think nationally because the facts that count are not
systematically reported and presented in a form we can digest. Our most
abysmal ignorance occurs where we deal with the immigrant. If we read
his press at all, it is to discover ``Bolshevism'' in it and to blacken
all immigrants with suspicion. For his culture and his aspirations, for
his high gifts of hope and variety, we have neither eyes nor ears. The
immigrant colonies are like holes in the road which we never notice
until we trip over them. Then, because we have no current information
and no background of facts, we are, of course, the undiscriminating
objects of any agitator who chooses to rant against ``foreigners.''

Now, men who have lost their grip upon the relevant facts of their
environment are the inevitable victims of agitation and propaganda. The
quack, the charlatan, the jingo, and the terrorist, can flourish only
where the audience is deprived of independent access to information. But
where all news comes at second-hand, where all the testimony is
uncertain, men cease to respond to truths, and respond simply to
opinions. The environment in which they act is not the realities
themselves, but the pseudo-environment of reports, rumors, and guesses.
The whole reference of thought comes to be what somebody asserts, not
what actually is. Men ask, not whether such and such a thing occurred in
Russia, but whether Mr.~Raymond Robins is at heart more friendly to the
Bolsheviki than Mr.~Jerome Landfield. And so, since they are deprived of
any trustworthy means of knowing what is really going on, since
everything is on the plane of assertion and propaganda, they believe
whatever fits most comfortably with their prepossessions.

That this breakdown of the means of public knowledge should occur at a
time of immense change is a compounding of the difficulty. From
bewilderment to panic is a short step, as everyone knows who has watched
a crowd when danger threatens. At the present time a nation easily acts
like a crowd. Under the influence of headlines and panicky print, the
contagion of unreason can easily spread through a settled community. For
when the comparatively recent and unstable nervous organization which
makes us capable of responding to reality as it is, and not as we should
wish it, is baffled over a continuing period of time, the more primitive
but much stronger instincts are let loose.

War and Revolution, both of them founded on censorship and propaganda,
are the supreme destroyers of realistic thinking, because the excess of
danger and the fearful overstimulation of passion unsettle disciplined
behavior. Both breed fanatics of all kinds, men who, in the words of
Mr.~Santayana, have redoubled their effort when they have forgotten
their aim. The effort itself has become the aim. Men live in their
effort, and for a time find great exaltation. They seek stimulation of
their effort rather than direction of it. That is why both in war and
revolution there seems to operate a kind of Gresham's Law of the
emotions, in which leadership passes by a swift degradation from a
Mirabeau to a Robespierre; and in war, from a high-minded statesmanship
to the depths of virulent, hating jingoism.

The cardinal fact always is the loss of contact with objective
information. Public as well as private reason depends upon it. Not what
somebody says, not what somebody wishes were true, but what is so beyond
all our opining, constitutes the touchstone of our sanity. And a society
which lives at second-hand will commit incredible follies and
countenance inconceivable brutalities if that contact is intermittent
and untrustworthy. Demagoguery is a parasite that flourishes where
discrimination fails, and only those who are at grips with things
themselves are impervious to it. For, in the last analysis, the
demagogue, whether of the Right or the Left, is, consciously or
unconsciously an undetected liar.

Many students of politics have concluded that, because public opinion
was unstable, the remedy lay in making government as independent of it
as possible. The theorists of representative government have argued
persistently from this premise against the believers in direct
legislation. But it appears now that, while they have been making their
case against direct legislation, rather successfully it seems to me,
they have failed sufficiently to notice the increasing malady of
representative government.

Parliamentary action is becoming notoriously ineffective. In America
certainly the concentration of power in the Executive is out of all
proportion either to the intentions of the Fathers or to the orthodox
theory of representative government. The cause is fairly clear. Congress
is an assemblage of men selected for local reasons from districts. It
brings to Washington a more or less accurate sense of the superficial
desires of its constituency. In Washington it is supposed to think
nationally and internationally. But for that task its equipment and its
sources of information are hardly better than that of any other reader
of the newspaper. Except for its spasmodic investigating committees,
Congress has no particular way of informing itself. But the Executive
has. The Executive is an elaborate hierarchy reaching to every part of
the nation and to all parts of the world. It has an independent
machinery, fallible and not too truthworthy, of course, but nevertheless
a machinery of intelligence. It can be informed and it can act, whereas
Congress is not informed and cannot act.

Now the popular theory of representative government is that the
representatives have the information and therefore create the policy
which the executive administers. The more subtle theory is that the
executive initiates the policy which the legislature corrects in
accordance with popular wisdom. But when the legislature is haphazardly
informed, this amounts to very little, and the people themselves prefer
to trust the executive which knows, rather than the Congress which is
vainly trying to know. The result has been the development of a kind of
government which has been harshly described as plébiscite autocracy, or
government by newspapers. Decisions in the modern state tend to be made
by the interaction, not of Congress and the executive, but of public
opinion and the executive.

Public opinion for this purpose finds itself collected about special
groups which act as extra-legal organs of government. There is a labor
nucleus, a farmers' nucleus, a prohibition nucleus, a National Security
League nucleus, and so on. These groups conduct a continual
electioneering campaign upon the unformed, exploitable mass of public
opinion. Being special groups, they have special sources of information,
and what they lack in the way of information is often manufactured.
These conflicting pressures beat upon the executive departments and upon
Congress, and formulate the conduct of the government. The government
itself acts in reference to these groups far more than in reference to
the district congressmen. So politics as it is now played consists in
coercing and seducing the representative by the threat and the appeal of
these unofficial groups. Sometimes they are the allies, sometimes the
enemies, of the party in power, but more and more they are the energy of
public affairs. Government tends to operate by the impact of controlled
opinion upon administration. This shift in the locus of sovereignty has
placed a premium upon the manufacture of what is usually called consent.
No wonder that the most powerful newspaper proprietor in the
English-speaking world declined a mere government post.

No wonder, too, that the protection of the sources of its opinion is the
basic problem of democracy. Everything else depends upon it. Without
protection against propaganda, without standards of evidence, without
criteria of emphasis, the living substance of all popular decision is
exposed to every prejudice and to infinite exploitation. That is why I
have argued that the older doctrine of liberty was misleading. It did
not assume a public opinion that governs. Essentially it demanded
toleration of opinions that were, as Milton said, indifferent. It can
guide us little in a world where opinion is sensitive and decisive.

The axis of the controversy needs to be shifted. The attempt to draw
fine distinctions between ``liberty'' and ``license'' is no doubt part
of the day's work, but it is fundamentally a negative part. It consists
in trying to make opinion responsible to prevailing social standards,
whereas the really important thing is to try and make opinion
increasingly responsible to the facts. There can be no liberty for a
community which lacks the information by which to detect lies. Trite as
the conclusion may at first seem, it has, I believe, immense practical
consequences, and may perhaps offer an escape from the logomachy into
which the contests of liberty so easily degenerate.

It may be bad to suppress a particular opinion, but the really deadly
thing is to suppress the news. In time of great insecurity, certain
opinions acting on unstable minds may cause infinite disaster. Knowing
that such opinions necessarily originate in slender evidence, that they
are propelled more by prejudice from the rear than by reference to
realities, it seems to me that to build the case for liberty upon the
dogma of their unlimited prerogatives is to build it upon the poorest
foundation. For, even though we grant that the world is best served by
the liberty of all opinion, the plain fact is that men are too busy and
too much concerned to fight more than spasmodically for such liberty.
When freedom of opinion is revealed as freedom of error, illusion, and
misinterpretation, it is virtually impossible to stir up much interest
in its behalf. It is the thinnest of all abstractions and an
over-refinement of mere intellectualism. But people, wide circles of
people, are aroused when their curiosity is baulked. The desire to know,
the dislike of being deceived and made game of, is a really powerful
motive, and it is that motive that can best be enlisted in the cause of
freedom.

What, for example, was the one most general criticism of the work of the
Peace Conference? It was that the covenants were not openly arrived at.
This fact stirred Republican Senators, British Labor, the whole gamut of
parties from the Right to the Left. And in the last analysis lack of
information about the Conference \emph{was} the origin of its
difficulties. Because of the secrecy endless suspicion was aroused;
because of it the world seemed to be presented with a series of
accomplished facts which it could not reject and did not wish altogether
to accept. It was lack of information which kept public opinion from
affecting the negotiations at the time when intervention would have
counted most and cost least. Publicity occurred when the covenants were
arrived at, with all the emphasis on the \emph{at}. This is what the
Senate objected to, and this is what alienated much more liberal opinion
than the Senate represents.

In a passage quoted previously in this essay, Milton said that
differences of opinion, ``which though they may be many, yet need not
interrupt the unity of spirit, if we could but find among us the bond of
peace.'' There is but one kind of unity possible in a world as diverse
as ours. It is unity of method, rather than of aim; the unity of the
disciplined experiment. There is but one bond of peace that is both
permanent and enriching: the increasing knowledge of the world in which
experiment occurs. With a common intellectual method and a common area
of valid fact, differences may become a form of cooperation and cease to
be an irreconcilable antagonism.

That, I think, constitutes the meaning of freedom for us. We cannot
successfully define liberty, or accomplish it, by a series of
permissions and prohibitions. For that is to ignore the content of
opinion in favor of its form. Above all, it is an attempt to define
liberty of opinion in terms of opinion. It is a circular and sterile
logic. A useful definition of liberty is obtainable only by seeking the
principle of liberty in the main business of human life, that is to say,
in the process by which men educate their response and learn to control
their environment. In this view liberty is the name we give to measures
by which we protect and increase the veracity of the information upon
which we act.

\newpage
\thispagestyle{plain} % empty
\mbox{}

% CHAPTER THREE
\chapter[3 \hspace*{1mm} LIBERTY AND THE NEWS]{3 LIBERTY AND THE NEWS}
\label{ch:LIBERTY-NEWS}

\newthought{The debates about}\marginnote{\href{https://doi.org/10.32376/3f8575cb.75aef417}{doi} | \href{https://github.com/mediastudiespress/singles/raw/master/public_domain/lippmann-1920/pdfs/04-lippmann-1920-chapter-three-original.pdf}{original pdf}}  liberty have hitherto all been attempts to determine
just when in the series from Right to Left the censorship should
intervene. In the preceding paper I ventured to ask whether these
attempts do not turn on a misconception of the problem. The conclusion
reached was that, in dealing with liberty of opinion, we were dealing
with a subsidiary phase of the whole matter; that, so long as we were
content to argue about the privileges and immunities of opinion, we were
missing the point and trying to make bricks without straw. We should
never succeed even in fixing a standard of tolerance for opinions, if we
concentrated all our attention on the opinions. For they are derived,
not necessarily by reason, to be sure, but somehow, from the stream of
news that reaches the public, and the protection of that stream is the
critical interest in a modern state. In going behind opinion to the
information which it exploits, and in making the validity of the news
our ideal, we shall be fighting the battle where it is really being
fought. We shall be protecting for the public interest that which all
the special interests in the world are most anxious to corrupt.

As the sources of the news are protected, as the information they
furnish becomes accessible and usable, as our capacity to read that
information is educated, the old problem of tolerance will wear a new
aspect. Many questions which seem hopelessly insoluble now will cease to
seem important enough to be worth solving. Thus the advocates of a
larger freedom always argue that true opinions will prevail over error;
their opponents always claim that you can fool most of the people most
of the time. Both statements are true, but both are half-truths. True
opinions can prevail only if the facts to which they refer are known; if
they are not known, false ideas are just as effective as true ones, if
not a little more effective.

The sensible procedure in matters affecting the liberty of opinion would
be to ensure as impartial an investigation of the facts as is humanly
possible. But it is just this investigation that is denied us. It is
denied us, because we are dependent upon the testimony of anonymous and
untrained and prejudiced witnesses; because the complexity of the
relevant facts is beyond the scope of our hurried understanding; and
finally, because the process we call education fails so lamentably to
educate the sense of evidence or the power of penetrating to the
controlling center of a situation. The task of liberty, therefore, falls
roughly under three heads, protection of the sources of the news,
organization of the news so as to make it comprehensible, and education
of human response.

We need, first, to know what can be done with the existing
news-structure, in order to correct its grosser evils. How far is it
useful to go in fixing personal responsibility for the truthfulness of
news? Much further, I am inclined to think, than we have ever gone. We
ought to know the names of the whole staff of every periodical. While it
is not necessary, or even desirable, that each article should be signed,
each article should be documented, and false documentation should be
illegal. An item of news should always state whether it is received from
one of the great news-agencies, or from a reporter, or from a press
bureau. Particular emphasis should be put on marking news supplied by
press bureaus, whether they are labeled ``Geneva,'' or ``Stockholm,'' or
``El Paso.''

One wonders next whether anything can be devised to meet that great evil
of the press, the lie which, once under way, can never be tracked down.
The more scrupulous papers will, of course, print a retraction when they
have unintentionally injured someone; but the retraction rarely
compensates the victim. The law of libel is a clumsy and expensive
instrument, and rather useless to private individuals or weak
organizations because of the gentlemen's agreement which obtains in the
newspaper world. After all, the remedy for libel is not money damages,
but an undoing of the injury. Would it be possible then to establish
courts of honor in which publishers should be compelled to meet their
accusers and, if found guilty of misrepresentation, ordered to publish
the correction in the particular form and with the prominence specified
by the finding of the court? I do not know. Such courts might prove to
be a great nuisance, consuming time and energy and attention, and
offering too free a field for individuals with a persecution mania.

Perhaps a procedure could be devised which would eliminate most of these
inconveniences. Certainly it would be a great gain if the accountability
of publishers could be increased. They exercise more power over the
individual than is healthy, as everybody knows who has watched the
yellow press snooping at keyholes and invading the privacy of helpless
men and women. Even more important than this, is the utterly reckless
power of the press in dealing with news vitally affecting the friendship
of peoples. In a Court of Honor, possible perhaps only in Utopia,
voluntary associations working for decent relations with other peoples
might hale the jingo and the subtle propagandist before a tribunal, to
prove the reasonable truth of his assertion or endure the humiliation of
publishing prominently a finding against his character.

This whole subject is immensely difficult, and full of traps. It would
be well worth an intensive investigation by a group of publishers,
lawyers, and students of public affairs. Because in some form or other
the next generation will attempt to bring the publishing business under
greater social control. There is everywhere an increasingly angry
disillusionment about the press, a growing sense of being baffled and
misled; and wise publishers will not pooh-pooh these omens. They might
well note the history of prohibition, where a failure to work out a
programme of temperance brought about an undiscriminating taboo. The
regulation of the publishing business is a subtle and elusive matter,
and only by an early and sympathetic effort to deal with great evils can
the more sensible minds retain their control. If publishers and authors
themselves do not face the facts and attempt to deal with them, some day
Congress, in a fit of temper, egged on by an outraged public opinion,
will operate on the press with an ax. For somehow the community must
find a way of making the men who publish news accept responsibility for
an honest effort not to misrepresent the facts.

But the phrase ``honest effort'' does not take us very far. The problem
here is not different from that which we begin dimly to apprehend in the
field of government and business administration. The untrained amateur
may mean well, but he knows not how to do well. Why should he? What are
the qualifications for being a surgeon? A certain minimum of special
training. What are the qualifications for operating daily on the brain
and heart of a nation? None. Go some time and listen to the average run
of questions asked in interviews with Cabinet officers or anywhere else.

\enlargethispage{\baselineskip}

I remember one reporter who was detailed to the Peace Conference by a
leading news-agency. He came around every day for ``news.'' It was a
time when Central Europe seemed to be disintegrating, and great doubt
existed as to whether governments would be found with which to sign a
peace. But all that this ``reporter'' wanted to know was whether the
German fleet, then safely interned at Scapa Flow, was going to be sunk
in the North Sea. He insisted every day on knowing that. For him it was
the German fleet or nothing. Finally, he could endure it no longer. So
he anticipated Admiral Reuther and announced, in a dispatch to his home
papers, that the fleet would be sunk. And when I say that a million
American adults learned all that they ever learned about the Peace
Conference through this reporter, I am stating a very moderate figure.

He suggests the delicate question raised by the schools of journalism:
how far can we go in turning newspaper enterprise from a haphazard trade
into a disciplined profession? Quite far, I imagine, for it is
altogether unthinkable that a society like ours should remain forever
dependent upon untrained accidental witnesses. It is no answer to say
that there have been in the past, and that there are now, first-rate
correspondents. Of course there are. Men like Brailsford, Oulahan,
Gibbs, Lawrence, Swope, Strunsky, Draper, Hard, Dillon, Lowry, Levine,
Ackerman, Ray Stannard Baker, Frank Cobb, and William Allen White, know
their way about in this world. But they are eminences on a rather flat
plateau. The run of the news is handled by men of much smaller caliber.
It is handled by such men because reporting is not a dignified
profession for which men will invest the time and cost of an education,
but an underpaid, insecure, anonymous form of drudgery, conducted on
catch-as-catch-can principles. Merely to talk about the reporter in
terms of his real importance to civilization will make newspaper men
laugh. Yet reporting is a post of peculiar honor. Observation must
precede every other activity, and the public observer (that is, the
reporter) is a man of critical value. No amount of money or effort spent
in fitting the right men for this work could possibly be wasted, for the
health of society depends upon the quality of the information it
receives.

Do our schools of journalism, the few we have, make this kind of
training their object, or are they trade-schools designed to fit men for
higher salaries in the existing structure? I do not presume to answer
the question, nor is the answer of great moment when we remember how
small a part these schools now play in actual journalism. But it is
important to know whether it would be worth while to endow large numbers
of schools on the model of those now existing, and make their diplomas a
necessary condition for the practice of reporting. It is worth
considering. Against the idea lies the fact that it is difficult to
decide just what reporting is---where in the whole mass of printed
matter it begins and ends. No one would wish to set up a closed guild of
reporters and thus exclude invaluable casual reporting and writing. If
there is anything in the idea at all, it would apply only to the routine
service of the news through large organizations.

Personally I should distrust too much ingenuity of this kind, on the
ground that, while it might correct certain evils, the general tendency
would be to turn the control of the news over to unenterprising
stereotyped minds soaked in the traditions of a journalism always ten
years out of date. The better course is to avoid the deceptive short
cuts, and make up our minds to send out into reporting a generation of
men who will by sheer superiority, drive the incompetents out of
business. That means two things. It means a public recognition of the
dignity of such a career, so that it will cease to be the refuge of the
vaguely talented. With this increase of prestige must go a professional
training in journalism in which the ideal of objective testimony is
cardinal. The cynicism of the trade needs to be abandoned, for the true
patterns of the journalistic apprentice are not the slick persons who
scoop the news, but the patient and fearless men of science who have
labored to see what the world really is. It does not matter that the
news is not susceptible of mathematical statement. In fact, just because
news is complex and slippery, good reporting requires the exercise of
the highest of the scientific virtues. They are the habits of ascribing
no more credibility to a statement than it warrants, a nice sense of the
probabilities, and a keen understanding of the quantitative importance
of particular facts. You can judge the general reliability of any
observer most easily by the estimate he puts upon the reliability of his
own report. If you have no facts of your own with which to check him,
the best rough measurement is to wait and see whether he is aware of any
limitations in himself; whether he knows that he saw only part of the
event he describes; and whether he has any background of knowledge
against which he can set what he thinks he has seen.

This kind of sophistication is, of course, necessary for the merest
pretense to any education. But for different professions it needs to be
specialized in particular ways. A sound legal training is pervaded by
it, but the skepticism is pointed to the type of case with which the
lawyer deals. The reporter's work is not carried on under the same
conditions, and therefore requires a different specialization. How he is
to acquire it is, of course, a pedagogical problem requiring an
inductive study of the types of witness and the sources of information
with whom the reporter is in contact.

Some time in the future, when men have thoroughly grasped the role of
public opinion in society, scholars will not hesitate to write treatises
on evidence for the use of news-gathering services. No such treatise
exists to-day, because political science has suffered from that curious
prejudice of the scholar which consists in regarding an irrational
phenomenon as not quite worthy of serious study.

Closely akin to an education in the tests of credibility is rigorous
discipline in the use of words. It is almost impossible to overestimate
the confusion in daily life caused by sheer inability to use language
with intention. We talk scornfully of ``mere words.'' Yet through words
the whole vast process of human communication takes place. The sights
and sounds and meanings of nearly all that we deal with as ``politics,''
we learn, not by our own experience, but through the words of others. If
those words are meaningless lumps charged with emotion, instead of the
messengers of fact, all sense of evidence breaks down. Just so long as
big words like Bolshevism, Americanism, patriotism, pro-Germanism, are
used by reporters to cover anything and anybody that the biggest fool at
large wishes to include, just so long shall we be seeking our course
through a fog so dense that we cannot tell whether we fly upside-down or
right-side-up. It is a measure of our education as a people that so many
of us are perfectly content to live our political lives in this
fraudulent environment of unanalyzed words. For the reporter,
abracadabra is fatal. So long as he deals in it, he is gullibility
itself, seeing nothing of the world, and living, as it were, in a hall
of crazy mirrors.

Only the discipline of a modernized logic can open the door to reality.
An overwhelming part of the dispute about ``freedom of opinion'' turns
on words which mean different things to the censor and the agitator. So
long as the meanings of the words are not dissociated, the dispute will
remain a circular wrangle. Education that shall make men masters of
their vocabulary is one of the central interests of liberty. For such an
education alone can transform the dispute into debate from similar
premises.

A sense of evidence and a power to define words must for the modern
reporter be accompanied by a working knowledge of the main
stratifications and currents of interest. Unless he knows that ``news''
of society almost always starts from a special group, he is doomed to
report the surface of events. He will report the ripples of a passing
steamer, and forget the tides and the currents and the ground-swell. He
will report what Kolchak or Lenin says, and see what they do only when
it confirms what he thinks they said. He will deal with the flicker of
events and not with their motive. There are ways of reading that flicker
so as to discern the motive, but they have not been formulated in the
light of recent knowledge. Here is big work for the student of politics.
The good reporter reads events with an intuition trained by wide
personal experience. The poor reporter cannot read them, because he is
not even aware that there is anything in particular to read.

And then the reporter needs a general sense of what the world is doing.
Emphatically he ought not to be serving a cause, no matter how good. In
his professional activity it is no business of his to care whose ox is
gored. To be sure, when so much reporting is \emph{ex parte}, and
hostile to insurgent forces, the insurgents in self-defense send out
\emph{ex parte} reporters of their own. But a community cannot rest
content to learn the truth about the Democrats by reading the Republican
papers, and the truth about the Republicans by reading the Democratic
papers. There is room, and there is need, for disinterested reporting;
and if this sounds like a counsel of perfection now, it is only because
the science of public opinion is still at the point where astronomy was
when theological interests proclaimed the conclusions that all research
must vindicate.

While the reporter will serve no cause, he will possess a steady sense
that the chief purpose of ``news'' is to enable mankind to live
successfully toward the future. He will know that the world is a
process, not by any means always onward and upward, but never quite the
same. As the observer of the signs of change, his value to society
depends upon the prophetic discrimination with which he selects those
signs.

But the news from which he must pick and choose has long since become
too complicated even for the most highly trained reporter. The work,
say, of the government is really a small part of the day's news, yet
even the wealthiest and most resourceful newspapers fail in their
efforts to report ``Washington.'' The high lights and the disputes and
sensational incidents are noted, but no one can keep himself informed
about his Congressman or about the individual departments, by reading
the daily press. This failure in no way reflects on the newspapers. It
results from the intricacy and unwieldiness of the subject-matter. Thus,
it is easier to report Congress than it is to report the departments,
because the work of Congress crystallizes crudely every so often in a
roll-call. But administration, although it has become more important
than legislation, is hard to follow, because its results are spread over
a longer period of time, and its effects are felt in ways that no
reporter can really measure.

Theoretically Congress is competent to act as the critical eye on
administration. Actually, the investigations of Congress are almost
always planless raids, conducted by men too busy and too little informed
to do more than catch the grosser evils, or intrude upon good work that
is not understood. It was a recognition of these difficulties that was
the cause of two very interesting experiments in late years. One was the
establishment of more or less semi-official institutes of government
research; the other, the growth of specialized private agencies which
attempt to give technical summaries of the work of various branches of
the government. Neither experiment has created much commotion: yet
together they illustrate an idea which, properly developed, will be
increasingly valuable to an enlightened public opinion.

Their principle is simple. They are expert organized reporters. Having
no horror of dullness, no interest in being dramatic, they can study
statistics and orders and reports which are beyond the digestive powers
of a newspaper man or of his readers. The lines of their growth would
seem to be threefold: to make a current record, to make a running
analysis of it, and on the basis of both, to suggest plans.

Record and analysis require an experimental formulation of standards by
which the work of government can be tested. Such standards are not to be
evolved off-hand out of anyone's consciousness. Some have already been
worked out experimentally, others still need to be discovered; all need
to be refined and brought into perspective by the wisdom of experience.
Carried out competently, the public would gradually learn to substitute
objective criteria for gossip and intuitions. One can imagine a
public-health service subjected to such expert criticism. The institute
of research publishes the death-rate as a whole for a period of years.
It seems that for a particular season the rate is bad in certain
maladies, that in others the rate of improvement is not sufficiently
rapid. These facts are compared with the expenditures of the service,
and with the main lines of its activity. Are the bad results due to the
causes beyond the control of the service? do they indicate a lack of
foresight in asking appropriations for special work? or in the absence
of novel phenomena, do they point to a decline of the personnel, or in
its morale? If the latter, further analy-\\ \noindent sis may reveal that salaries are
too low to attract men of ability, or that the head of the service by
bad management has weakened the interest of his staff.

When the work of government is analyzed in some such way as this, the
reporter deals with a body of knowledge that has been organized for his
apprehension. In other words, he is able to report the ``news,'' because
between him and the raw material of government there has been interposed
a more or less expert political intelligence. He ceases to be the ant,
described by William James, whose view of a building was obtained by
crawling over the cracks in the walls.

These political observatories will, I think, be found useful in all
branches of government, national, state, municipal, industrial, and even
in foreign affairs. They should be clearly out of reach either of the
wrath or of the favor of the office-holders. They must, of course, be
endowed, but the endowment should be beyond the immediate control of the
legislature and of the rich patron. Their independence can be partially
protected by the terms of the trust; the rest must be defended by the
ability of the institute to make itself so much the master of the facts
as to be impregnably based on popular confidence.

One would like to think that the universities could be brought into such
a scheme. Were they in close contact with the current record and
analysis, there might well be a genuine ``field work'' in political
science for the students; and there could be no better directing idea
for their more advanced researches than the formulation of the
intellectual methods by which the experience of government could be
brought to usable control. After all, the purpose of studying
``political science'' is to be able to act more effectively in politics,
the word effectively being understood in the largest and, therefore, the
ideal sense. In the universities men should be able to think patiently
and generously for the good of society. If they do not, surely one of
the reasons is that thought terminates in doctor's theses and brown
quarterlies, and not in the critical issues of politics.

On first thought, all this may seem rather a curious direction for an
inquiry into the substance of liberty. Yet we have always known, as a
matter of common sense, that there was an intimate connection between
``liberty'' and the use of liberty. Every one who has examined the
subject at all has had to conclude that tolerance \emph{per se} is an
arbitrary line, and that, in practice, the determining factor is the
significance of the opinion to be tolerated. This study is based on an
avowal of that fact. Once it is avowed, there seems to be no way of
evading the conclusion that liberty is not so much permission as it is
the construction of a system of information increasingly independent of
opinion. In the long run it looks as if opinion could be made at once
free and enlightening only by transferring our interest from ``opinion''
to the objective realities from which it springs. This thought has led
us to speculations on ways of protecting and organizing the stream of
news as the source of all opinion that matters. Obviously these
speculations do not pretend to offer a fully considered or a completed
scheme. Their nature forbids it, and I should be guilty of the very
opinionativeness I have condemned, did these essays claim to be anything
more than tentative indications of the more important phases of the
problem.

Yet I can well imagine their causing a considerable restlessness in the
minds of some readers. Standards, institutes, university research,
schools of journalism, they will argue, may be all right, but they are a
gray business in a vivid world. They blunt the edge of life; they leave
out of account the finely irresponsible opinion thrown out by the
creative mind; they do not protect the indispensable novelty from
philistinism and oppression. These proposals of yours, they will say,
ignore the fact that such an apparatus of knowledge will in the main be
controlled by the complacent and the traditional, and the execution will
inevitably be illiberal.

There is force in the indictment. And yet I am convinced that we shall
accomplish more by fighting for truth than by fighting for our theories.
It is a better loyalty. It is a humbler one, but it is also more
irresistible. Above all it is educative. For the real enemy is
ignorance, from which all of us, conservative, liberal, and
revolutionary, suffer. If our effort is concentrated on our
desires,---be it our desire to have and to hold what is good, our desire
to remake peacefully, or our desire to transform suddenly,---we shall
divide hopelessly and irretrievably. We must go back of our opinions to
the neutral facts for unity and refreshment of spirit. To deny this, it
seems to me, is to claim that the mass of men is impervious to
education, and to deny that, is to deny the postulate of democracy, and
to seek salvation in a dictatorship. There is, I am convinced, nothing
but misery and confusion that way. But I am equally convinced that
democracy will degenerate into this dictatorship either of the Right or
of the Left, if it does not become genuinely self-governing. That means,
in terms of public opinion, a resumption of that contact between beliefs
and realities which we have been losing steadily since the small-town
democracy was absorbed into the Great Society.

The administration of public information toward greater accuracy and
more successful analysis is the highway of liberty. It is, I believe, a
matter of first-rate importance that we should fix this in our minds.
Having done so, we may be able to deal more effectively with the traps
and the lies and the special interests which obstruct the road and drive
us astray. Without a clear conception of what the means of liberty are,
the struggle for free speech and free opinion easily degenerates into a
mere contest of opinion.

But realization is not the last step, though it is the first. We need be
under no illusion that the stream of news can be purified simply by
pointing out the value of purity. The existing news-structure may be
made serviceable to democracy along the general lines suggested, by the
training of the journalist, and by the development of expert record and
analysis. But while it may be, it will not be, simply by saying that it
ought to be. Those who are now in control have too much at stake, and
they control the source of reform itself.

Change will come only by the drastic competition of those whose
interests are not represented in the existing news-organization. It will
come only if organized labor and militant liberalism set a pace which
cannot be ignored. Our sanity and, therefore, our safety depend upon
this competition, upon fearless and relentless exposure conducted by
self-conscious groups that are now in a minority. It is for these groups
to understand that the satisfaction of advertising a pet theory is as
nothing compared to the publication of the news. And having realized it,
it is for them to combine their resources and their talent for the
development of an authentic news-service which is invincible because it
supplies what the community is begging for and cannot get.

All the gallant little sheets expressing particular programmes are at
bottom vanity, and in the end, futility, so long as the reporting of
daily news is left in untrained and biased hands. If we are to move
ahead, we must see a great independent journalism, setting standards for
commercial journalism, like those which the splendid English coöperative
societies are setting for commercial business. An enormous amount of
money is dribbled away in one fashion or another on little papers,
mass-meetings, and what not. If only some considerable portion of it
could be set aside to establish a central international news-agency, we
should make progress. We cannot fight the untruth which envelops us by
parading our opinions. We can do it only by reporting the facts, and we
do not deserve to win if the facts are against us.

The country is spotted with benevolent foundations of one kind or
another, many of them doing nothing but pay the upkeep of fine buildings
and sinecures. Organized labor spends large sums of money on politics
and strikes which fail because it is impossible to secure a genuine
hearing in public opinion. Could there be a pooling of money for a
news-agency? Not, I imagine, if its object were to further a cause. But
suppose the plan were for a news-service in which editorial matter was
rigorously excluded, and the work was done by men who had already won
the confidence of the public by their independence? Then, perhaps.

At any rate, our salvation lies in two things: ultimately, in the
infusion of the news-structure by men with a new training and outlook;
immediately, in the concentration of the independent forces against the
complacency and bad service of the routineers. We shall advance when we
have learned humility; when we have learned to seek the truth, to reveal
it and publish it; when we care more for that than for the privilege of
arguing about ideas in a fog of uncertainty.



\end{document}